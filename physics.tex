\documentclass[leqno,b5paper]{book}
%\documentclass[leqno]{book} %the geometry package defines the paper size

% xelatex --shell-escape calculusAndAnalyticGeometry   %must run to avail gnuplot functionality. gnuplot must be installed.
\typeout{to avail gnuplot functionality must run   "xelatex --shell-escape calculusAndAnalyticGeometry"}

%===================
%testing tikz inside book
\usepackage[arrowmos,oldvoltagedirection]{circuitikz}
\usepackage{pgfplotstable}
\usepackage{pgfplots}

\usepackage{tikz-3dplot}
\pgfplotsset{compat=newest,}
\usepgfplotslibrary{units}
%\pgfplotsset{compat=1.9}
\usepgfplotslibrary{polar}
\usepackage{ifdraft}
\usepackage{multicol}                                        %for getting multicolumn itemize, enumerate etc
\usepackage{enumerate}                   %for getting better auto numbering in enumerate


%pathmorphing gives the ripples like look
\usetikzlibrary{shapes,snakes,3d,shadings,fadings,intersections,calc,decorations.markings,decorations.pathreplacing,external,shapes.misc,decorations.pathmorphing,patterns}


%\tikzexternalize[mode=list and make] %disable to generate figures
%\tikzexternaldisable  %enables figures after this command. put this is the tex file

\usepackage{scrextend}   % \begin{labeling}{}  \end{labeling}   just like Description
%for vertical spacing in tables use\Tstrut and \Bstrut  
\newcommand\Tstrut{\rule{0pt}{2.6ex}}       % Top strut
\newcommand\Bstrut{\rule[-1.2ex]{0pt}{0pt}} % Bottom strut
%\renewcommand{\arraystretch}{2} give this before tabular for vertical space 

\definecolor{lgray}{cmyk}{0,0,0,0.2}
\colorlet{llgray}{gray!10}
\definecolor{dgray}{cmyk}{0,0,0,0.7}
%========================
%\usepackage[hidelinks]{hyperref}  %used in machines book but not here
\usepackage{./tex/khalidUrduBooks}                     %my sty file
\usepackage{amsbsy} %for bold Poynting, lessgtr symbol
\usepackage{mathrsfs}   %for Poynting symbol
\usepackage{IEEEtrantools}
\usepackage{multirow}   %for multiple row cells in a table column
\usepackage[misc]{ifsym} 

\newcommand{\tikzmark}[1]{%
  \tikz[overlay,remember picture] \node (#1) {};}

 
\newcommand*{\kStrokeOne}{|}                                      %tally marks (statistics)
\newcommand*{\kStrokeTwo}{|\!|}
\newcommand*{\kStrokeThree}{|\!|\!|}
\newcommand*{\kStrokeFour}{|\!|\!|\!|}
\newcommand*{\kStrokeFive}{\cancel\kStrokeFour}


%\renewcommand{\arraystretch}{2}   %space in tables and arrays


\sisetup{
math-micro=\textup{µ},text-micro=µ,math-ohm  =\upOmega,text-degree=°,
  math-degree=\textup{°}}   %with mathpazo this is needed else must not be here. now micro is smaller
\DeclareSIUnit \var {var}    %used in electric circuits volt-ampere-reactive

\input longdiv.tex		%this is a package but doesnot load with usepackage. it writes pen and paper long division
\usepackage{polynom}		%this writes pen and paper long division for polynomials
\usepgfplotslibrary{fillbetween}
\usepackage[]{gnuplottex}
%============================
%%%%%%%%%%%%%%%%%%%%%%%%%%%%%%%%%%%%%%%%%
\usepackage{newfile}                                                                  %writing answers to the end of the book
\newwrite\tempfile
\immediate\openout\tempfile=answer.tex
\newcommand*\wf[1]{\immediate\write\tempfile{#1}}
%
%%the  \اصطلاح and \حاشیہب have been disabled  in "myUrduCommandsCalculus.tex" and instead 
%%the following is used. this retains their original definitions and writes them to a "technicalTerms.tex" text file too.
%\newwrite\tempfileTerms				 %collecting technical terms. uses newfile package
%\immediate\openout\tempfileTerms=technicalTerms.tex
%\newcommand*\wTechTerms[1]{\immediate\write\tempfileTerms{#1}}
%\newcommand*{\اصطلاح}[1]{\wTechTerms{\unexpanded{ur, #1 ,}}{\urduTechTermsfont{#1}}}
%\newcommand*{\حاشیہب}[1]{ \wTechTerms{\unexpanded{en, #1,}}{\raggedright{\footnote{\textenglish{#1}}}}}
%%%%%%%%%%%%%%%%%%%%%%%%%%%%%%%%%%%%%%%%%%%%
%============================                                                            
%=================================
%=================================

\newcommand{\krightharpoonup}[1]{\overset{\rightharpoonup}{\rule{0pt}{.9ex}\smash{#1}}}
\newcommand{\kleftharpoonup}[1]{\overset{\leftharpoonup}{\rule{0pt}{.9ex}\smash{#1}}}


\newcommand{\koverleftharp}[1]{\overharp{\leftharpoonup}{#1}{.7}}
\newcommand{\koverrightharp}[1]{\overharp{\rightharpoonup}{#1}{.7}}
\newcommand{\koverleftharpdown}[1]{\overharp{\leftharpoondown}{#1}{.9}}
\newcommand{\koverrightharpdown}[1]{\overharp{\rightharpoondown}{#1}{.9}}
\newcommand{\kunderleftharp}[1]{\overharp{\leftharpoonup}{#1}{-1}}
\newcommand{\kunderrightharp}[1]{\overharp{\rightharpoonup}{#1}{-1}}
\newcommand{\kunderleftharpdown}[1]{\overharp{\leftharpoondown}{#1}{-.8}}
\newcommand{\kunderrightharpdown}[1]{\overharp{\rightharpoondown}{#1}{-.8}}

\newlength{\argwd}  \newlength{\arght}%-Two variables
\newcommand{\overharp}[3]{%        -The command name
  \settowidth{\argwd}{#2}\settoheight{\arght}{#2}%
  %                                    -Set the variables
  \raisebox{#3\arght}{%                -Put the harp 6/10 of a line higher
    \makebox[0pt][r]{%                 -Put everything in a box ;           corrected by me for Flush Right
      \resizebox{\argwd}{.8\arght}{\!$#1$}% 
      %                                -Set harp to right length
    }%
  }%
#2}%                                   -Print the argument

                       %testing harpoons


%this file describes urdu commands for the commonly used english latex commands

%chapter, section etc
%  \newcommand*{newcommand}[arguments]{actual command}
\newcommand*{\باب}[1]{\chapter{#1}}                                      %defining commonly used commands
\newcommand*{\حصہ}[1]{\section{#1}}
\newcommand*{\جزوحصہ}[1]{\subsection{#1}}
\newcommand*{\جزوجزوحصہ}[1]{\subsubsection{#1}}

\newcommand*{\بابء}[1]{\chapter*{#1}}                                      %defining commonly used commands
\newcommand*{\حصہء}[1]{\section*{#1}}
\newcommand*{\جزوحصہء}[1]{\subsection*{#1}}
\newcommand*{\جزوجزوحصہء}[1]{\subsubsection*{#1}}


%english text in urdu mode
\newcommand*{\تحریر}[1]{\textenglish{#1}}	% english text in urduMode
%\newcommand*{\موٹا}[1]{\textbf{#1}}
%\newcommand*{\ترچھا}[1]{{\textit{#1}}}
\newcommand*{\موٹا}[1]{{\urduTechTermsfont{#1}}}
\newcommand*{\ترچھا}[1]{{\small{#1}}}

%%\newcommand*{\اصطلاح}[1]{{\color{red}{#1}}}   %colours spills to next word when there is index or footnote entry with the word
%%\newcommand{\اصطلاح}[1]{{\urdufontBig{#1}}}

\newcommand{\اصطلاح}[1]{{\urduTechTermsfont{#1}}}  %%%moved to main file
\newcommand{\اصطلاحء}[1]{{\small{#1}}} %highlighted where  chance of mixup with  regular urdu meaning


%end commands cannot be redefined and as such these two are not usable
\providecommand*{\ابتدا}[1]{\begin{#1}}
\providecommand*{\انتہا}[1]{\end{#1}}

%include and input directives for adding external files into the main document 
\newcommand*{\بشمول}[1]{\includeonly{#1}}
\newcommand*{\شامل}[1]{\include{#1}}
\newcommand*{\داخل}[1]{\input{#1}}

%to use extra latex packages
\newcommand*{\استعمال}[1]{\usepackage{#1}}

%footnotes and indexes
\newcommand*{\حاشیہب}[1]{{\raggedright{\footnote{\textenglish{#1}}}} }          %  moved to main file. footnote to the left hand side
\newcommand*{\حاشیہد}[1]{{\raggedleft{\footnote{#1}}}}
\newcommand*{\حاشیہط}[1]{\marginpar{#1}}

\newcommand*{\فرہنگ}[1]{\index{#1}}

%references and labels
\newcommand*{\شناخت}[1]{\label{#1}}
\newcommand*{\حوالہ}[1]{\textenglish{\ref{#1}}}
\newcommand*{\حوالہء}[1]{#1}			%dummy to enter the figure number directly while the figure is not ready
\newcommand*{\حوالہصفحہ}[1]{\pageref{#1}}

%counters
\newcommand*{\فاصلہ}{\vspace*{10mm}}
\newcommand*{\فاصلہء}{\quad}

%itemize, bullets and numbered items   
\newcommand*{\اشیاء}{itemize}                               %used in   \begin{itemize}
\newcommand*{\شے}[1]{\item {#1}}			%used in    \item, \description
%description
\newcommand*{\جزو}[1]{\item[#1]}                      %used in \begin{description}
%maths commands
%\newcommand*{\عددی}[1]{\: \ensuremath{#1} \:} % in-line math & inside math mode
%\newcommand*{\عددیء}[1]{\ensuremath{#1}}
\newcommand*{\عددی}[1]{\:\(#1\)\:} % in-line math & inside math mode
\newcommand*{\عددیء}[1]{\(#1\)}
\newcommand*{\سمتیہ}[1]{\ensuremath{{\bf{#1}}}}
\newcommand*{\سمتیازیرنوشت}[2]{\ensuremath{{\boldsymbol{#1}}_{\textup{#2}}}}

\newcommand*{\ضرب}{\time}					%multiplication symbol
\newcommand*{\نکطہد}{\cdot}
\newcommand*{\نقطے}{\ensuremath{\cdots}}

\newcommand*{\زیرنوشت}[3]{\: \ensuremath{{#1_{#2 \textrm {#3}}}} \:}   %english+urdu subscript \زیرنوشت{V}{CE}{غیرافزائندہ}
\newcommand*{\سیدھازیرنوشت}[2]{\: \ensuremath{{#1_{\textup{#2}}}} \:} %RC

\newcommand*{\قریب}[1]{\mbox{#1}}  %dissallows splitting along two lines 
%\newcommand{\سن}[1]{؁\,\ensuremath{#1}}
\newcommand{\سن}[1]{
#1 ؁
}
\newcommand{\زور}[1]{\aemph{#1}} %overline urdu text to emphasize

\newcommand{\kQuote}[1]{“\ignorespaces#1\ignorespaces”}
\newcommand{\kquote}[1]{‘\ignorespaces#1\ignorespaces’}
\newcommand{\قولء}[1]{’\ignorespaces#1\ignorespaces‘}
\newcommand{\قول}[1]{”\ignorespaces#1\ignorespaces“}


\renewcommand{\indexname}{فرہنگ}        %does nothing here. must be placed within begin{urdufont} environment to be the last to take effect 
%===============================

%===============================
%===============================
%needed for correct format in toc, 

%%numbering scheme
\renewcommand*{\thefigure}{\thechapter.\arabic{figure}}
\renewcommand*{\thetable}{\thechapter.\arabic{table}}
\renewcommand*{\theequation}{\thechapter.\arabic{equation}}
\renewcommand*{\thesection}{\thechapter.\arabic{section}}
\renewcommand*{\thesubsection}{\thechapter.\arabic{section}.\arabic{subsection}}
\renewcommand*{\thesubsubsection}{\thechapter.\arabic{section}.\arabic{subsection}.\arabic{subsubsection}}
%=======================================

%=======================================


%================
%my environments
%================

%environment for examples مثال
%\newcounter{examplecounter}[section]
%\renewcommand{\theexamplecounter}{\arabic{examplecounter}}
\newcounter{examplecounter}[chapter]
\renewcommand{\theexamplecounter}{\thechapter.\arabic{examplecounter}}

\newenvironment*{مثال}
{\par\noindent\ignorespaces    مثال \refstepcounter{examplecounter} \theexamplecounter :\quad}%
{\hfill\qedsymbol  \vspace{\baselineskip}\par}
%{\noindent\ignorespaces \vspace{\baselineskip} \hrule \vspace{\baselineskip}  مثال \refstepcounter{examplecounter} \theexamplecounter :}%
%{\par\noindent \hrule  \vspace{\baselineskip}}

%------
%practice problems environment مشق

%\newcounter{practicecounter}[section]                              %practice here means مشق
%\renewcommand{\thepracticecounter}{\arabic{practicecounter}}
\newcounter{practicecounter}[chapter]                              %practice here means مشق
\renewcommand{\thepracticecounter}{\thechapter.\arabic{practicecounter}}

\newenvironment*{مشق}
{\par\noindent\ignorespaces \vspace{\baselineskip} \hrule \vspace{\baselineskip} مشق \refstepcounter{practicecounter} \thepracticecounter :\quad}%
{\par\noindent \hrule  \vspace{\baselineskip}}
%---------

%end of chapter questions environment سوال

\newcounter{questioncounter}[chapter]                           %for reseting at every section. to be used only during writing stage
%\renewcommand{\thequestioncounter}{\arabic{questioncounter}}
%\newcounter{questioncounter}[chapter]						%when book is finished, use this instead of the above
\renewcommand{\thequestioncounter}{\thechapter.\arabic{questioncounter}}

\newenvironment*{سوال}				
{\noindent\ignorespaces  سوال \refstepcounter{questioncounter} \thequestioncounter :\quad}%
{\par\noindent\ignorespaces }
%--------------------

%defining a LAW   قانون

%\newcounter{lawcounter}[section]
%\renewcommand{\thelawcounter}{\arabic{lawcounter}}
\newcounter{lawcounter}[chapter]
\renewcommand{\thelawcounter}{\thechapter.\arabic{lawcounter}}

\newenvironment*{قانون}				
{\par\medskip \refstepcounter{lawcounter} \quad\nopagebreak}%
{\par\hfill\qedsymbol  \vspace{\baselineskip}\par }
%{\par\medskip }
%--------------------
%--------------------

%defining a THEOREM   مسئلہ

%\newcounter{kthcounter}[section]
%\renewcommand{\thekthcounter}{\arabic{kthcounter}}
\newcounter{kthcounter}[chapter]
\renewcommand{\thekthcounter}{\thechapter.\arabic{kthcounter}}

\newenvironment*{مسئلہ}				
{\par\noindent\ignorespaces  مسئلہ \refstepcounter{kthcounter} \thekthcounter :\quad}%
{\par\noindent }
%--------------------
%--------------------

%defining a Proof   ثبوت

%\newcounter{kprcounter}[chapter]
%\renewcommand{\thekprcounter}{\thechapter.\arabic{kprcounter}}

\newenvironment*{ثبوت}				
{\noindent\ignorespaces  ثبوت :\quad}%
{\par\hfill\qedsymbol  \vspace{\baselineskip}\par }
%{\par\noindent\qedsymbol  \vspace{\baselineskip} }
%--------------------
%--------------------
%++++++++++++++++++++++++++++++

%--------------------

%defining a COROLLARY   ضمنی نتیجہ


\newcounter{kcocounter}[chapter]
\renewcommand{\thekcocounter}{\thechapter.\arabic{kcocounter}}

\newenvironment*{ضمنی نتیجہ}				
{\par\noindent\ignorespaces  ضمنی نتیجہ \refstepcounter{kcocounter} \thekcocounter :\quad}%
{\par\noindent }
%--------------------
%--------------------

%defining a Proof   ثبوت ضمنی نتیجہ

%\newcounter{kprcocounter}[chapter]
%\renewcommand{\thekprcocounter}{\thechapter.\arabic{kprcocounter}}

\newenvironment*{ثبوت ضمنی نتیجہ}				
{\noindent\ignorespaces  ثبوت ضمنی نتیجہ :\quad}%
{\par\hfill\qedsymbol  \vspace{\baselineskip}\par }
%{\par\noindent\qedsymbol  \vspace{\baselineskip} }
%--------------------
%--------------------
%++++++++++++++++++++++++++++++++++++++++++++

%defining a Definition تعریف

%\newcounter{kdfcounter}[chapter]
%\renewcommand{\thedfrcounter}{\thechapter.\arabic{kdfcounter}}

\newenvironment*{تعریف}				
{\par\noindent\ignorespaces  تعریف :\quad}%
{\par\hfill\qedsymbol  \vspace{\baselineskip}\par }
%{\par\noindent }
%--------------------
%----------------------------
%defining a Definition تعریفات

%\newcounter{kdfcounter}[chapter]
%\renewcommand{\thedfrcounter}{\thechapter.\arabic{kdfcounter}}

\newenvironment*{تعریفات}				
{\par\noindent\ignorespaces  تعریفات :\quad}%
{\par\hfill\qedsymbol  \vspace{\baselineskip}\par }
%{\par\noindent }
%--------------------
%--------------------

%defining a Definition مفروضہ

%\newcounter{kAscounter}[chapter]
%\renewcommand{\theAsrcounter}{\thechapter.\arabic{kAscounter}}

\newenvironment*{مفروضہ}				
{\par\noindent\ignorespaces  مفروضہ \quad}%
{\par\hfill\qedsymbol  \vspace{\baselineskip}\par }
%{\par\noindent }
%--------------------
%--------------------

%defining a RULE   قاعدہ

%\newcounter{krucounter}[section]
%\renewcommand{\thekrucounter}{\arabic{krucounter}}
\newcounter{krucounter}[chapter]
\renewcommand{\thekrucounter}{\thechapter.\arabic{krucounter}}

\newenvironment*{قاعدہ}				
{\par\noindent\ignorespaces  قاعدہ \refstepcounter{krucounter} \thekrucounter :\quad}%
{\par\noindent }
%--------------------
%--------------------

%defining a Proof   ثبوت قاعدہ

%\newcounter{kprRcounter}[chapter]
%\renewcommand{\thekprRcounter}{\thechapter.\arabic{kprRcounter}}

\newenvironment*{ثبوت قاعدہ}				
{\noindent\ignorespaces  ثبوت قاعدہ :\quad}%
{\par\hfill\qedsymbol  \vspace{\baselineskip}\par }
%{\par\noindent\qedsymbol  \vspace{\baselineskip} }
%--------------------
%--------------------

%defining a TEST   پرکھ

%\newcounter{kttcounter}[section]
%\renewcommand{\thekttcounter}{\arabic{kttcounter}}
\newcounter{kttcounter}[chapter]
\renewcommand{\thekttcounter}{\thechapter.\arabic{kttcounter}}

\newenvironment*{پرکھ}				
{\par\noindent\ignorespaces  \refstepcounter{kttcounter} \quad \nopagebreak}%
{\par\hfill\qedsymbol  \vspace{\baselineskip}\par }
%--------------------

\newenvironment*{ثبوت پرکھ}				
{\noindent\ignorespaces  ثبوت پرکھ :\quad}%
{\par\hfill\qedsymbol  \vspace{\baselineskip}\par }
%{\par\noindent\qedsymbol  \vspace{\baselineskip} }

%---------

%questions environment جواب

%\newcounter{answercounter}[section]                                           %for reseting at every section. NOT NEEDED
%\renewcommand{\theanswercounter}{\arabic{answercounter}}
%\newcounter{answercounter}[chapter]						%when book is finished, use this instead of the above
%\renewcommand{\theanswercounter}{\thechapter.\arabic{answercounter}}

%\newenvironment*{جواب}				
%{\noindent\ignorespaces\wf{\thequestioncounter)\noexpand\quad}}%
%{\par\noindent\ignorespacesafterend}
%==================

\newenvironment*{سوالات}				
{\noindent\ignorespaces\wf{
\موٹا{حصہ} 
\thesection
\quad
\موٹا{صفحہ}
\thepage}
\wf{\unexpanded{\begin{description}\setlength{\parskip}{0pt} \setlength{\itemsep}{0pt plus 1pt}}}
}%
{\wf{\unexpanded{\end{description}}}\par\noindent\ignorespacesafterend}
%==================
\newenvironment*{جواب}  %the next is a better environment. it take less input		
{\noindent\ignorespaces\wf{\unexpanded{\item[}}\wf{\thequestioncounter)}\wf{\unexpanded{]}}}%
{\noindent\ignorespacesafterend}
%%==================
%%==================
%%this takes one input hence there is no need to write \wf{\unexpanded{...}} in every answer
%\newenvironment*{جواب}[1]			%put actual answer in {} examples {\(\sqrt{2}\)} 	
%{\noindent\ignorespaces\wf{\unexpanded{\item[}}\wf{\thequestioncounter)}\wf{\unexpanded{]}}\wf{\unexpanded{#1}}}%
%{\noindent\ignorespacesafterend}
%==================
%\newenvironment*{جوابء}				%i think is wrong and not used	
%{\noindent\ignorespaces\wf{\unexpanded{[}}\wf{\thequestioncounter)}\wf{\unexpanded{]}}}%
%{\par\noindent\ignorespacesafterend}
%==================

                  %turning latex into urdu
% Greek Letters for Urdu Latex usage

\newcommand*{\ایلفا}{\alpha}
\newcommand*{\بیٹا}{\beta}
\newcommand*{\گیما}{\gamma}
\newcommand*{\ڈیلٹا}{\delta}
\newcommand*{\ایپسلان}{\epsilon}
\newcommand*{\متغیرایپسلان}{\varepsilon}
\newcommand*{\زیٹا}{\zeta}
\newcommand*{\ایٹا}{\eta}
\newcommand*{\تھیٹا}{\theta}
\newcommand*{\متغیرتھیٹا}{\vartheta}
\newcommand*{\ایوٹا}{\iota}
\newcommand*{\کاپا}{\kappa}
\newcommand*{\لیمڈا}{\lambda}
\newcommand*{\میو}{\mu}
\newcommand*{\نیو}{\nu}
\newcommand*{\ژاے}{\xi}
\newcommand*{\پاے}{\pi}
\newcommand*{\متغیرپاے}{\varpi}
\newcommand*{\رھو}{\rho}
\newcommand*{\متغیررھو}{\varrho}
\newcommand*{\سگما}{\sigma}
\newcommand*{\متغیرسگما}{\varsigma}
\newcommand*{\ٹو}{\tau}
\newcommand*{\اپسیلان}{\upsilon}
\newcommand*{\فاے}{\phi}
\newcommand*{\متغیفاے}{\varphi}
\newcommand*{\چاے}{\chi}
\newcommand*{\ساے}{\psi}
\newcommand*{\اومیگا}{\omega}

\newcommand*{\بڑاگیما}{\Gamma}
\newcommand*{\بڑاڈیلٹا}{\Delta}
\newcommand*{\بڑاتھیٹا}{\Theta}
\newcommand*{\بڑالیمڈا}{\Lambda}
\newcommand*{\بڑاژاے}{\Xi}
\newcommand*{\بڑاپاے}{\Pi}
\newcommand*{\بڑاسگما}{\Sigma}
\newcommand*{\بڑاساے}{\Psi}
\newcommand*{\بڑااومیگا}{\Omega}



\newcommand*{\kvec}[1]{{\ensuremath{{\boldsymbol{#1}}}}}
\newcommand*{\kvecsub}[2]{{\ensuremath{{\boldsymbol{#1}}}_{\textup{#2}}}}

\newcommand*{\ax}{\ensuremath{{\boldsymbol{a}}_{\textup{x}}}}
\newcommand*{\ay}{\ensuremath{{\boldsymbol{a}}_{\textup{y}}}}
\newcommand*{\az}{\ensuremath{{\boldsymbol{a}}_{\textup{z}}}}
%
\newcommand*{\arho}{\ensuremath{{\boldsymbol{a}}_{\rho}}}
\newcommand*{\aphi}{\ensuremath{{\boldsymbol{a}}_{\phi}}}
%
\newcommand*{\ar}{\ensuremath{{\boldsymbol{a}}_{\textup{r}}}}
\newcommand*{\atheta}{\ensuremath{{\boldsymbol{a}}_{\theta}}}

\newcommand*{\aN}{\ensuremath{{\boldsymbol{a}}_N}}
\newcommand*{\aR}{\ensuremath{{\boldsymbol{a}}_{\textup{R}}}}
\newcommand*{\aL}{\ensuremath{{\boldsymbol{a}}_{\textup{L}}}}

\newcommand*{\au}{\ensuremath{{\boldsymbol{a}}_u}}
\newcommand*{\av}{\ensuremath{{\boldsymbol{a}}_v}}
\newcommand*{\aw}{\ensuremath{{\boldsymbol{a}}_w}}

\newcommand*{\ai}{\ensuremath{{\boldsymbol{i}}}}
\newcommand*{\aj}{\ensuremath{{\boldsymbol{j}}}}
\newcommand*{\ak}{\ensuremath{{\boldsymbol{k}}}}

\newcommand*{\uu}{\ensuremath{{\boldsymbol{u}}}}


\newcommand*{\Ex}{\ensuremath{{\boldsymbol{E}}_x}}
\newcommand*{\Ey}{\ensuremath{{\boldsymbol{E}}_y}}
\newcommand*{\Ez}{\ensuremath{{\boldsymbol{E}}_z}}
%
\newcommand*{\Erho}{\ensuremath{{\boldsymbol{E}}_{\rho}}}
\newcommand*{\Ephi}{\ensuremath{{\boldsymbol{E}}_{\phi}}}
%
\newcommand*{\Er}{\ensuremath{{\boldsymbol{E}}_r}}
\newcommand*{\Etheta}{\ensuremath{{\boldsymbol{E}}_{\theta}}}

\newcommand*{\TE}[1]{\ensuremath{\textup{TE}_{#1}}}
\newcommand*{\TM}[1]{\ensuremath{\textup{TM}_{#1}}}
\newcommand*{\TEM}{\ensuremath{\textup{TEM}}}
%===========================
\newcommand{\RightAngle}[4][5pt]{\draw[gray] ($#3!#1!#2$)--($ #3!2!($($#3!#1!#2$)!.5!($#3!#1!#4$)$) $) --($#3!#1!#4$) ;        }



\DeclareMathOperator{\sech}{sech}
\DeclareMathOperator{\csch}{csch}
\DeclareMathOperator{\cosec}{cosec}
\DeclareMathOperator{\arcsec}{arcsec}
\DeclareMathOperator{\arccot}{arcCot}
\DeclareMathOperator{\arccsc}{arcCsc}
\DeclareMathOperator{\arccosine}{arcCos}
\DeclareMathOperator{\arccosh}{arcCosh}
\DeclareMathOperator{\arcsinh}{arcsinh}
\DeclareMathOperator{\arctanh}{arctanh}
\DeclareMathOperator{\arcsech}{arcsech}
\DeclareMathOperator{\arccsch}{arcCsch}
\DeclareMathOperator{\arccoth}{arcCoth} 
\DeclareMathOperator{\erf}{erf} 
\DeclareMathOperator{\erfc}{erfc} 
%the following two Sine Integral symbols doesnot clash with SI units package
\DeclareMathOperator{\kSi}{Si} 
\DeclareMathOperator{\ksi}{si} 
\DeclareMathOperator{\kS}{S} 
%cosine integral, exponential integral, logrithmic integral
\DeclareMathOperator{\ci}{ci} 
\DeclareMathOperator{\kC}{C} 
\DeclareMathOperator{\Ei}{Ei} 
\DeclareMathOperator{\li}{li} 
%Fresnel Cosine and SineIntegrals and their Auxiliary integrals
\DeclareMathOperator{\FC}{C} 
\DeclareMathOperator{\FS}{S} 
\DeclareMathOperator{\FAC}{c}                  %complementary Fresnel integral
\DeclareMathOperator{\FAS}{s}                     %complementary Fresnel integral
\DeclareMathOperator{\gammaQ}{Q}                     %Incomplete gamma function
\DeclareMathOperator{\gammaP}{P}                     %Incomplete gamma function
%Hermite polynomials
\DeclareMathOperator{\He}{He} 
%Complex Natural Logarithm, Principal Value
\DeclareMathOperator{\Ln}{Ln} 
%Residue 
\DeclareMathOperator{\Res}{Res}
%Lagrange interpolation formula
\DeclareMathOperator{\Lagrange}{L}
%statistics UpperControlLimit and Lower Control Limit, Average Outgoing Quality
\DeclareMathOperator{\UCL}{UCL} 
\DeclareMathOperator{\LCL}{LCL} 
\DeclareMathOperator{\CL}{CL} 
\DeclareMathOperator{\OC}{OC} 
\DeclareMathOperator{\AOQ}{AOQ} 
%projection of a vector
\DeclareMathOperator{\proj}{proj} 
     %\sech, \csch, \arcsh, \arcs   hyperbolic and arc-secant etc

\pgfmathsetmacro{\x}{2}     %smallest resistor sizes
\pgfmathsetmacro{\y}{2}
\pgfmathsetmacro{\xx}{2.5}   %somewhat larger resistor leads. gives more space
\pgfmathsetmacro{\yy}{2.5}
\pgfmathsetmacro{\xxx}{3}   %still larger resistor leads. gives even more space
\pgfmathsetmacro{\yyy}{3}
\pgfmathsetmacro{\dx}{0.2}     %moving labels beyond resistor outline
\pgfmathsetmacro{\dy}{0.2}
\pgfmathsetmacro{\pin}{0.3}

\pgfmathsetmacro{\boxW}{0.5}   %width of box circuit
\pgfmathsetmacro{\boxH}{2.5}   %height of box circuit

%=============================
%complex numbers, squared voltages
\newcommand*{\bZ}{{\ensuremath{{\boldsymbol{Z}}}}}           %complex impedance
\newcommand*{\bY}{{\ensuremath{{\boldsymbol{Y}}}}}           %complex admittance
\newcommand*{\bZCC}{{\ensuremath{{\boldsymbol{Z}}^{*}}}}                                            %complex conjugate impedance
\newcommand*{\bYCC}{{\ensuremath{{\boldsymbol{Y}}^{*}}}}                                            %complex conjugate

\newcommand*{\bVrms}{{\ensuremath{\hat{V}_{\textup{rms}}}}}           %phasor voltage
\newcommand*{\bIrms}{{\ensuremath{\hat{I}_{\textup{rms}}}}}           %phasor current
\newcommand*{\Vrms}{{\ensuremath{V_{\textup{rms}}}}}       %rms voltage
\newcommand*{\Irms}{{\ensuremath{I_{\textup{rms}}}}}           %rms current
\newcommand*{\Arms}{{\ensuremath{A_{\textup{rms}}}}}           %rms amps
\newcommand*{\VrmsS}{{\ensuremath{V^2_{\textup{rms}}}}}       %rms squared
\newcommand*{\IrmsS}{{\ensuremath{I^2_{\textup{rms}}}}}           %rms squared
\newcommand*{\bVrmsCC}{{\ensuremath{\hat{V}^{*}_{\textup{rms}}}}}                  %conjugate phasor voltage
\newcommand*{\bIrmsCC}{{\ensuremath{\hat{I}^{*}_{\textup{rms}}}}}           %conjugate phasor current


\newcommand*{\kx}[1]{{\ensuremath{{\boldsymbol{#1}}}}}                  %complex quantity
\newcommand*{\bS}{{\ensuremath{{\boldsymbol{S}}}}}                         %complex power
\newcommand*{\bH}{{\ensuremath{{\boldsymbol{H}}}}}                       %network functions
\newcommand*{\bA}{{\ensuremath{{\boldsymbol{A}}}}}                        %voltage gain

\newcommand*{\pf}{{\ensuremath{{\textup{pf}}}}}
\newcommand*{\rms}{{\ensuremath{\textup{rms}}}}           %rms
\newcommand*{\BW}{{\ensuremath{{\textup{BW}}}}}   %bandwidth

\newcommand*{\Laplace}{\mathcal{L}}   %Laplace transform
\newcommand*{\Fourier}{\mathcal{F}}   %Fourier transform

\newcommand*{\kB}[1]{{\ensuremath{{\textup{#1}}}}}  %Laplace symbol general use. 
									%following were used too often so gave them specific symbols
\newcommand*{\bF}{{\ensuremath{{\textup{F}}}}}    %Fourier transform of 
\newcommand*{\bP}{{\ensuremath{{\textup{P}}}}}   %Laplace fraction
\newcommand*{\bQ}{{\ensuremath{{\textup{Q}}}}}  %Laplace fraction
\newcommand*{\bV}{{\ensuremath{{\textup{V}}}}}  %Laplace Voltage
\newcommand*{\bI}{{\ensuremath{{\textup{I}}}}}  %Laplace Current
%Matrices and Vectors
\newcommand*{\bM}[1]{{\ensuremath{{\boldsymbol{#1}}}}} 
 % resistor sizes and Laplace, Fourier, Complex, Phasor, etc  symbols
%draws left and right arrows where needed e.g.  
% \draw[->-=0.5] (0,0)--(3,0); draws arrow at the middle
\tikzset{->-/.style={decoration={markings, mark=at position #1 with {\arrow{latex}}},postaction={decorate}}}
\tikzset{-<-/.style={decoration={markings, mark=at position #1 with {\arrow{latex reversed}}},postaction={decorate}}}
\tikzset{osquare/.style={draw,solid,fill=white, rectangle, minimum size=4pt, inner sep=0pt, outer sep=0pt}}   %node[osquare,fill=black]{}

%this puts small orthogonal tick marks along a curve at selected points. a point at each side of the location has to be provided to %the \path[] section of the command as shown.
% 		\draw[smooth, domain=0:2]plot ({\x},{\x^2});
%		\foreach \t in {0.5,1,1.5}{\path[| mark=0.5] ({\t-0.1},{(\t-0.1)^2}) -- ({\t+0.1},{(\t+0.1)^2});}
\tikzset{| mark/.style={postaction=decorate,decoration={markings,
mark=at position #1 with {\draw[line cap=round,mark segment] (0,-2pt) -- (0,2pt);
}}},mark segment/.style={thick}}


%draws right angles \RightAngle{A}{B}{C}
\providecommand{\RightAngle}[4][5pt]{\draw[] ($#3!#1!#2$)--($ #3!2!($($#3!#1!#2$)!.5!($#3!#1!#4$)$) $) --($#3!#1!#4$) ;     }
%colours
\definecolor{lgray}{cmyk}{0,0,0,0.2}
\definecolor{dgray}{cmyk}{0,0,0,0.7}
%draws a cross just like ocirc, circ; usage \fill(0,2)circle(1.5py);\draw(0,2)node[kcross]{};
\tikzset{kcross/.style={cross out, draw, 
         minimum size=2*(2pt-\pgflinewidth), 
         inner sep=0pt, outer sep=0pt}}


%tikz, pgfplot TABLE
\pgfplotsset{select coords between index/.style 2 args={
    x filter/.code={
        \ifnum\coordindex<#1\def\pgfmathresult{}\fi
        \ifnum\coordindex>#2\def\pgfmathresult{}\fi
    }
}}
%
%boxed circuits
%=========================================
%\leftBox[K]{3,2}   draws a box with lower end at (3,2) and the terminals called Ka and Kb
\newcommand{\boxLeft}[2][p]{
\coordinate (a) at (#2);
\draw (a)++(-0.025,0.5) coordinate (b);
\draw (a)++(-0.04,1) coordinate (c);
\draw (a)++(-0.12,1.5) coordinate (d);
\draw (a)++(-0.2,2) coordinate (e);
\draw (a)++(-0.15,2.5) coordinate (f);
\draw (a)++(0.5,3) coordinate (g);

\draw (a)++(0.7,2.5)coordinate(h);
\draw (a)++(0.6,2)coordinate(i);
\draw (a)++(0.75,1.5)coordinate(j);
\draw (a)++(0.7,1)coordinate(k);
\draw (a)++(0.7,0.5)coordinate(l);
\draw (a)++(0.6,0)coordinate(m);
\draw plot [smooth cycle] coordinates {(a) (b) (c) (d) (e) (f) (g) (h) (i) (j) (k) (l) (m)};
\draw (h)coordinate(#1a);
\draw(l)coordinate(#1b);
}
%===================
%\rightBox[J]{3,2}   draws a box with lower end at (3,2) and the terminals called Ja and Jb
\newcommand{\boxRight}[2][p]{
\coordinate (aa) at (#2);
\draw (aa)++(0.025,0.5) coordinate(ba);
\draw (aa)++(0.04,1)coordinate(ca);
\draw (aa)++ (0.12,1.5)coordinate(da);
\draw (aa)++(0.13,2)coordinate(ea);
\draw (aa)++(0.1,2.5)coordinate(fa);
\draw (aa)++(-0.5,3)coordinate(ga);

\draw (aa)++(-0.8,2.5) coordinate(ha);
\draw (aa)++(-0.8,2) coordinate(ia);
\draw (aa)++ (-0.75,1.5) coordinate(ja);
\draw (aa)++(-0.7,1) coordinate(ka);
\draw (aa)++(-0.7,0.5) coordinate(la);
\draw (aa)++(-0.5,0) coordinate(ma);
\draw plot [smooth cycle] coordinates {(aa) (ba) (ca) (da) (ea) (fa) (ga) (ha) (ia) (ja) (ka) (la) (ma)};
\draw (ha)coordinate(#1a);
\draw(la)coordinate(#1b);
}
%===================
%writes text above matrix entries (outside the matrix bars)
\newcommand\bovermat[2]{%
  \makebox[0pt][r]{$\raisebox{16pt}[0pt][0pt]{\text{\RL{#1}}}$}#2}
\newcommand\covermat[2]{%
  \makebox[0pt][c]{$\raisebox{16pt}[0pt][0pt]{\text{\RL{#1}}}$}#2}
\newcommand\partialphantom{\vphantom{\frac{\partial e_{P,M}}{\partial w_{1,1}}}}

%=============================
%when a table is all math, instead of using $$ in each cell use the following.Text can be entered in a cell with \text{} 
%usage \begin{matrix}{C|L} ;not needed in array as array is $$ by default
\newcolumntype{L}{>{$}l<{$}}
\newcolumntype{C}{>{$}c<{$}}
\newcolumntype{R}{>{$}r<{$}}




%%still decided to format everything at the very end
%try to make a jump table so that a single command can be built

\newcommand*{\ksubRB}{\ensuremath{R_{\textup{B}}}}   %transistor
\newcommand*{\ksubRC}{\ensuremath{R_{\textup{C}}}}
\newcommand*{\ksubRE}{\ensuremath{R_{\textup{E}}}}

\newcommand*{\ksubRG}{\ensuremath{R_{\textup{G}}}}	%mosfet
\newcommand*{\ksubRD}{\ensuremath{R_{\textup{D}}}}
\newcommand*{\ksubRS}{\ensuremath{R_{\textup{S}}}}

\newcommand*{\ksubCB}{\ensuremath{C_{\textup{B}}}}	%transistor
\newcommand*{\ksubCC}{\ensuremath{C_{\textup{C}}}}
\newcommand*{\ksubCE}{\ensuremath{C_{\textup{E}}}}

\newcommand*{\ksubCG}{\ensuremath{C_{\textup{G}}}}	%mosfet
\newcommand*{\ksubCD}{\ensuremath{C_{\textup{D}}}}
\newcommand*{\ksubCS}{\ensuremath{C_{\textup{S}}}}

\newcommand*{\ksubRCB}{\ensuremath{R_{\textup{CB}}}}    %resistor used with base capacitor    (GET RID OF SUCH USAGE)
\newcommand*{\ksubRCC}{\ensuremath{R_{\textup{CC}}}}   %resistor used with collector capacitor
\newcommand*{\ksubRCE}{\ensuremath{R_{\textup{CE}}}}   %resistor used with emitter capacitor

\newcommand*{\ksubVBE}{\ensuremath{V_{\textup{BE}}}}  %npnTransistor
\newcommand*{\ksubVBC}{\ensuremath{V_{\textup{BC}}}}
\newcommand*{\ksubVCE}{\ensuremath{V_{\textup{CE}}}}

\newcommand*{\ksubVEB}{\ensuremath{V_{\textup{EB}}}}  %pnpTransistor
\newcommand*{\ksubVCB}{\ensuremath{V_{\textup{CB}}}}
\newcommand*{\ksubVEC}{\ensuremath{V_{\textup{EC}}}}

\newcommand*{\ksubVGS}{\ensuremath{V_{\textup{GS}}}}  %nMosfet
\newcommand*{\ksubVGD}{\ensuremath{V_{\textup{GD}}}}
\newcommand*{\ksubVDS}{\ensuremath{V_{\textup{DS}}}}

\newcommand*{\ksubVSG}{\ensuremath{V_{\textup{SG}}}} 	%pMosfet
\newcommand*{\ksubVDG}{\ensuremath{V_{\textup{DG}}}}
\newcommand*{\ksubVSD}{\ensuremath{V_{\textup{SD}}}}

\newcommand*{\ksubsubVRE}{\ensuremath{V_{R_{\textup{E}}}}}
\newcommand*{\ksubsubVRC}{\ensuremath{V_{R_{\textup{C}}}}}
\newcommand*{\ksubsubVRB}{\ensuremath{V_{R_{\textup{B}}}}}

\newcommand*{\ksub}[2]{\ensuremath{#1_{\textup{#2}}}}     %R_1    or V_1   or C_1     where numbers are used

%voltage sources
\newcommand*{\ksubVCC}{\ensuremath{V_{\textup{CC}}}} %transistor
\newcommand*{\ksubVBB}{\ensuremath{V_{\textup{BB}}}}
\newcommand*{\ksubVEE}{\ensuremath{V_{\textup{EE}}}}

\newcommand*{\ksubVDD}{\ensuremath{V_{\textup{DD}}}}	%mosfet
\newcommand*{\ksubVGG}{\ensuremath{V_{\textup{GG}}}}
\newcommand*{\ksubVSS}{\ensuremath{V_{\textup{SS}}}}

\newcommand*{\ksubVS}{\ensuremath{V_{\textup{S}}}}
\newcommand*{\ksubVs}{\ensuremath{V_{\textup{s}}}}

%gain
\newcommand*{\ksubAv}{\ensuremath{A_{\textup{v}}}}		%gains
\newcommand*{\ksubAi}{\ensuremath{A_{\textup{i}}}}
\newcommand*{\ksubGm}{\ensuremath{G_{\textup{m}}}}
\newcommand*{\ksubRm}{\ensuremath{R_{\textup{m}}}}
         %these are all tested. to use at the very end when book is finished

\graphicspath{{./fig/figFrontPage/}{./fig/figCalculusLimits/}{./fig/figCalculusDerivatives/}{./fig/figCalculusIntegration/}{./fig/figCalculusApplicationsOfIntegrals}{./fig/figCalculusTechniquesOfIntegration/}}%paths to figures


%
%\includeonly{./tex/prefaceFirstBook,./tex/physicsMeasurement}
%
%\includeonly{./tex/calculusPreface,./tex/prefaceFirstBook,./tex/calculusPreliminaries,./tex/calculusLimitsAndContinuity,./tex/calculusDerivatives,./tex/calculusApplicationsOfDerivatives,./tex/calculusIntegration,./tex/calculusApplicationsOfIntegrals,./tex/calculusTranscendentalFunctions,./tex/calculusTechniquesOfIntegration,./tex/calculusIntegrationTable}%
%
\includeonly{,./tex/prefaceFirstBook,./tex/physicsMeasurement}%


\author{
خالد خان یوسفزئی\\
\\
{\small {جامعہ کامسیٹ، اسلام آباد}}\\
\texttt{khalidyousafzai@hotmail.com}
}

%=========



\title{
طبیعیات  کے اصول
}
%\date{}                           %if absent gives date in arabic which is a rubbish


%\linenumbers

\makeindex

%==========
\begin{document}
\sloppy

\renewcommand*{\contentsname}{عنوان}    %this command has to be placed right here
%\renewcommand*{\proofname}{ثبوت}   %if placed before start of begin{urdufont}, it gets swept by the settings of the font environment
%\renewcommand*{\appendixname}{ضمیمہ}


\frontmatter                          %just added instead of \pagenumbering{roman}
%%\pagenumbering{roman}

\maketitle

\tableofcontents
\pagestyle{empty}
\newpage
\include{./tex/quantumMechanicsPreface}
\newpage
\باب{میری پہلی کتاب کا دیباچہ}
گزشتہ چند برسوں سے حکومتِ پاکستان اعلیٰ تعلیم کی طرف توجہ دے رہی ہے جس سے ملک کی تاریخ میں پہلی مرتبہ اعلیٰ تعلیمی اداروں میں تحقیق کا رجحان پیدا ہوا ہے۔امید کی جاتی ہے کہ یہ سلسلہ جاری رہے گا۔

پاکستان میں اعلٰی تعلیم کا نظام انگریزی زبان میں رائج ہے۔دنیا میں تحقیقی کام کا بیشتر حصہ انگریزی زبان میں ہی چھپتا ہے۔انگریزی زبان میں ہر موضوع پر لاتعداد کتابیں پائی جاتی ہیں جن سے طلبہ و طالبات استفادہ  کرتے ہیں۔

ہمارے ملک میں طلبہ و طالبات کی ایک بہت بڑی تعداد بنیادی تعلیم اردو زبان میں حاصل کرتی ہے۔ان کے لئے انگریزی زبان میں موجود مواد سے استفادہ  کرنا تو ایک طرف، انگریزی زبان ازخود ایک رکاوٹ کے طور پر ان کے سامنے آتی ہے۔یہ طلبہ و طالبات ذہین ہونے کے باوجود آگے بڑھنے اور قوم و ملک کی بھر پور خدمت کرنے کے قابل نہیں رہتے۔ایسے طلبہ و طالبات کو اردو زبان میں نصاب کی اچھی کتابیں درکار ہیں۔ہم نے قومی سطح پر ایسا کرنے کی کوئی خاطر خواہ کوشش نہیں کی۔ 

میں برسوں تک اس صورت حال کی وجہ سے پریشانی کا شکار رہا۔کچھ کرنے کی نیت رکھنے کے باوجود کچھ نہ کر سکتا تھا۔میرے لئے اردو میں ایک صفحہ بھی لکھنا ناممکن تھا۔آخر کار ایک دن میں نے اپنی اس کمزوری کو کتاب نہ لکھنے کا جواز بنانے سے انکار کر دیا اور یوں یہ کتاب وجود میں آئی۔

یہ کتاب اردو زبان میں تعلیم حاصل کرنے والے طلبہ و طالبات کے لئے نہایت آسان اردو میں لکھی گئی ہے۔کوشش کی گئی ہے کہ اسکول کی سطح پر نصاب میں استعمال ہونے والے تکنیکی الفاظ ہی استعمال کئے جائیں۔جہاں ایسے الفاظ موجود نہ تھے وہاں روز مرہ میں استعمال ہونے والے الفاظ چنے گئے۔تکنیکی الفاظ کی چنائی کے وقت اس بات کا دہان رکھا گیا کہ ان کا استعمال دیگر مضامین میں بھی ممکن ہو۔

کتاب میں بین الاقوامی نظامِ اکائی استعمال کی گئ ہے۔اہم متغیرات کی علامتیں وہی رکھی گئی ہیں جو موجودہ نظامِ تعلیم کی نصابی کتابوں میں رائج ہیں۔یوں اردو میں لکھی اس کتاب اور انگریزی میں اسی مضمون پر لکھی کتاب پڑھنے والے طلبہ و طالبات کو ساتھ کام کرنے میں دشواری نہیں ہو گی۔ 

امید کی جاتی ہے کہ یہ کتاب ایک دن خالصتاً اردو زبان میں انجنیئرنگ کی نصابی کتاب کے طور پر استعمال کی جائے گی۔اردو زبان میں برقی انجنیئرنگ کی مکمل نصاب کی طرف یہ پہلا قدم ہے۔ 

اس کتاب کے پڑھنے والوں سے گزارش کی جاتی ہے کہ اسے زیادہ سے زیادہ طلبہ و طالبات تک پہنچانے میں مدد دیں اور انہیں جہاں اس کتاب میں غلطی نظر آئے وہ اس کی نشاندہی میری ای-میل پر کریں۔میں ان کا نہایت شکر گزار ہوں گا۔

اس کتاب میں تمام غلطیاں مجھ سے ہی سر زد ہوئی ہیں البتہ انہیں درست کرنے میں بہت لوگوں کا ہاتھ ہے۔میں ان سب کا شکریہ ادا کرتا ہوں۔ یہ سلسلہ ابھی جاری ہے اور مکمل ہونے پر ان حضرات کے تاثرات یہاں شامل کئے جائیں گے۔  

میں یہاں کامسیٹ یونیورسٹی اور ہائر ایجوکیشن کمیشن کا شکریہ ادا کرنا چاہتا ہوں جن کی وجہ سے ایسی سرگرمیاں ممکن ہوئیں۔	
\vspace{5mm}

{\raggedleft{
خالد خان یوسفزئی

28 اکتوبر \سن{2011}}}

%\newpage
%\include{./tex/cktSymbols}


\mainmatter                      %added this
%\renewcommand*{\chaptername}{باب}
%%\pagenumbering{arabic}   %instead of this

\pagestyle{headings}

%\showthe\font

\باب{پیمائش}
\حصہء{طبیعیات کیا ہے؟}
	سائنس اور انجینئری پیمائش اور موازنہ پر مبنی ہے۔ یوں چیزوں کی پیمائش اور موازنہ کے لئیے ہمیں قواعد کی ضرورت پیش آتی ہے، اور ان پیمائش اور موازنوں کے بُعد تعین کرنے کے لئیے ہمیں تجربات کا سہارا لینا پڑھتا ہے۔ طبیعیات اور انجینئری کا ایک مقصد ان تجربات کی بناوٹ اور تجربہ کرنا ہے۔
	
\حصہء{چیزوں کی پیمائش}
	طبیعیات میں ملوث مقداروں کی پیمائش کی طریقے جان کر ہم طبیعیات دریافت کرتے ہیں۔ ان مقداروں میں لمبائی، وقت، کمیت، درجہ حرارت، دباؤ، اور برقی رو شامل ہیں۔
	
	ہم ہر طبی مقدار کا موازنہ ایک معیار کے ساتھ کرکے اسکو اپنی اکائیوں میں ناپتے ہیں۔ اس مقدار کی ناپ کو ایک منفرد نام دیا جاتا ہے جسے اکائی کہتے ہیں۔ مثلاً لمبائی کی ناپ کو میٹر میں ناپا جاتا ہے۔ معیار سے مراد مقدار کی ٹھیک ایک اکائی ہے۔ جیسا آپ دیکھیں گے لمبائی کا معیار جو ٹھیک ایک میٹر کے برابر ہے۔ اُس فاصلہ کو کہتے ہیں جو خلاء میں حرکت کرتے ہوئے کسی ایک مخصوص دورانیہ میں ایک شواع طے کرتا ہے۔ ہم ایک اکائی اور اسکے معیار کی تعریف جیسا چاہیں کر سکتے ہیں۔ تاہم، یہ ضروری ہے کہ دنیا کے باقی سائنسدان بھی اس تعریف کو معنی خیز اور قابلِ عمل کہیں۔
	
	ایک معیار مثلاً لمبائی کے لئیے طے کرنے کے بعد ہمیں وہ طریقہ کار واضح کرنے ہونگے جن سے ہم کسی بھی لمبائی چاہے وہ ہائڈروجن جوہر کا رداس ہو یا دور ستارے تک کا فاصلہ ہو اس معیار کی صورت میں ظاہر کر سکیں۔ ایسئ ایک ترکیب فیتہ کا استعمال ہے جو ہماری لمبائی کے معیار کو تخمینی طور پرظاہر کرتا ہے۔ بہرحال، بہت سارے موازنوں میں بِلا واسطہ طریقے استعمال کئیے جائیں گے۔ مثلاً ایک جوہر کا رداس یا قریبی ستارے تک کا فاصلہ فیتہ استعمال کرکے نہیں ناپا جا سکتا۔
	
\موٹا{اساسی مقداریں۔}
	اتنی زیادہ طبی مقداریں پائی جاتی ہیں کہ انہیں منظم کرنا ایک مسئلہ ہے۔ ہماری خوش قسمتی ہے کہ یہ تمام غیر تابع نہیں ہیں، مثلاً رفتار درحقیقت لمبائی اور وقت کا تناسب ہے۔ یوں بین الاقوامی متفقہ معاہدہ کے تحت چند طبی مقدار مثلاً لمبائی اور وقت منتخب کرکے صرف اِنہی کو معیار مختص کئیے جاتے ہیں۔ اس کے بعد باقی تمام طبی مقداروں کو انہی اساسی مقداروں اور اساسی معیاروں کے روپ میں ناپا جاتا ہے۔ مثال کی طور پر لمبائی اور وقت کی اساسی قیمتیں اور اِنکے اساسی معیار کی روپ میں رفتار تعین کیا جاتا ہے۔
	
	ضروری ہے کہ اساسی معیار قابلِ رسائی اور غیر متغیر ہوں۔ اگر ہم بازو کی لمبائی کو معیار لمبائی لیں تب یہ قابلِ رسائی ضرور ہوگا۔ البتہ ہر شخص کے لئیے یہ لمبائی مختلف ہوگی۔ سائنس اور انجینئرنگ میں زیادہ سے زیادہ درستگی مطلوب ہونے کے پیش نظر ہم پہلے غیرمتغیریت پر زور ڈالتے ہیں۔ اس کے بعد اساسی معیار کی بہتر سے بہتر نقل بنا کر اُنہیں فراہم کیا جاتا ہے جنہیں اِنکی ضرورت ہو۔


\حصہء{اکائیوں کا بین الاقوامی نظام}
	سن 1971 میںناپ و  تول  کے عمومی اجلاس میں سات مقداروں کو بطور اساسی مقدار منتخب کرکے بین الاقوامی نظامِ اکائی کے اساس چنے گئے۔ جدول  \حوالہ{جدول-پیمائژ_اساسی_اکائیاں} میں تین اساسی مقدار لمبائی، کمیت اور وقت دیکھائے گئے ہیں۔
	\begin{table}[h!]
\caption{بین القوامی نظام اکائی کے تین اساسی مقداروں کی اکئیاں}
\label{جدول-پیمائژ_اساسی_اکائیاں}
\centering
\begin{tabular}{r r c} 
\toprule
	مقدار & اکائی کا نام & اکائی کی علامت\\ 
	\midrule
	لمبائی & میٹر & \si{\meter} \\
	وقت & سیکنڈ & \si{\second} \\
	کمیت & کلوگرام &\si{\kilogram} \\
	\bottomrule
\end{tabular}
	\end{table}
	ان اکائیوں کی تعریف انسانی جسامت کو مدِ نظر رکھتے ہوئے کی گئی ہے۔
	
	کئی ماخوذ اکائیوں کی تعریف ان اساسی اکائیوں کی صورت میں کی جاتی ہے۔ مثلاً طاقت کی SI اکائی جسے واٹ کہتے ہیں۔ کمیت، لمبائی اور وقت کی اساسی اکائیوں کی صورت میں کی جاتی ہے۔ یوں جیسا باب 7 میں آپ دیکھیں گے درج ذیل ہوگا۔
	\begin{align}
\text{\RL{1 واٹ}}=\SI{1}{\watt}=\SI{1}{\kilogram}\cdot \si{\meter\squared\per\second\cubed}
	\end{align}
	جہاں آخر میں اکائیوں کو کلوگرام مربع میٹر فی مکعب سیکنڈ پڑھا جائے گا۔
	
	بہت بڑی یا بہت چھوٹی مقداروں کو جن سے ہمیں طبیعیات میں عموماً واسطہ پڑھتا ہے جن کو سائنسی علامتیت میں لکھا جاتا ہے، جو دس کی طاقت استعمال کرتا ہے۔ یوں درج ذیل ہوں گے۔
	\begin{align}
		\SI{3560000000}{\meter} = \SI{3.56e9}{\meter} 
	\end{align}
	\begin{align}
		\SI{0.000 000 492}{\second} = \SI{4.92e-7}{\second} 
	\end{align}
	کمیوٹرز پر سائنسی علامتیت اس سے بھی مختصر لکھی جاتی ہے۔ مثلاً  \عددی{3.56E9}  اور   \عددی{{4.92E-7}}  جہاں  دس کی طاقت کو  \عددی{E} سے ظاہر کیا جاتا ہے۔ کئی کیلکولیٹر میں اس سے بھی مختصر انداز میں لکھتے ہوئے    \عددی{E} کو خالی جگہ سے ظاہر کیا جاتا ہے۔
	
	ہم اپنی آسانی کے لئیے بہت بڑی یا بہت چھوٹی پیمائشوں کو جدول   \حوالہ{جدول_پیمائش_سابقے} میں دی گئی سابقہ کی مدد سے لکھتے ہیں۔
	\begin{table}[h!]
	\caption{بین الاقوامی نظام اکائی کے سابقے}
\label{جدول_پیمائش_سابقے}
\centering
\renewcommand{\arraystretch}{1.25}
\begin{tabular}{c c c} 
\toprule
	علامت & سابقہ &جزو ضربی\\
\midrule
	$\si{\yotta}$ & yotta- & $10^{24}$\\
	$\si{\zetta}$ & zetta- & $10^{21}$\\
	$\si{\exa}$ & exa- & $10^{18}$\\
	$\si{\peta}$ & peta- & $10^{15}$\\
	$\si{\tera}$ & tera- & $10^{12}$\\
	$\si{\giga}$ & giga- & $10^{9}$\\
	$\si{\mega}$ & mega- & $10^{6}$\\
	$\si{\kilo}$ & kilo- & $10^{3}$\\
	$\si{\hecto}$ & hecto- & $10^{2}$\\
	$\si{\deka}$ & deka- & $10^{1}$\\
	$\si{\deci}$ & deci- & $10^{-1}$\\
	$\si{\centi}$ & centi- & $10^{-2}$\\
	$\si{\milli}$ & milli- & $10^{-3}$\\
	$\si{\micro}$ & micro- & $10^{-6}$\\
	$\si{\nano}$ & nano- & $10^{-9}$\\
	$\si{\pico}$ & pico- & $10^{-12}$\\
	$\si{\femto}$ & femto- & $10^{-15}$\\
	$\si{\atto}$ & atto- & $10^{-18}$\\
	$\si{\zepto}$ & zepto- & $10^{-21}$\\
	$\si{\yocto}$ & yocto- & $10^{-24}$\\
	\bottomrule
\end{tabular}
	\end{table}
	جیسا آپ دیکھ سکتے ہیں ہر ایک سابقہ دس کی کسی مخصوص طاقت کو ظاہر کرتا ہے۔ جس کو بطور جذ ضربی استعمال کیا جاتا ہے۔ بین الاقوامی نظام اکائی کے ساتھ ایک سابقہ منسلک کرنے سے مراد اس اکائی کو مدابقتی جذ ضربی سے ضرب دینا ہے۔ یوں ہم کسی ایک مخصوص برقی طاقت کو
	\begin{align}
		 \text{\RL{واٹ}}\, 1.27\times 10^9 = \text{\RL{گیگا واٹ}} \, 1.27  = \SI{1.27}{\giga\watt}
	\end{align}
	یا کسی مخصوص وقتی دورانیہ کو  درج ذیل لکھ سکتے ہیں۔
	\begin{align}
		\text{\RL{سیکنڈ}} \, 2.35\times 10^{-9}  = \text{\RL{نینو سیکنڈ}} \, 2.35 
		= \SI{2.35}{\nano\second}.
	\end{align}
	چند سابقہ جو ملی لیٹر، سنٹی میٹر، کلوگرام یا میگابائٹ میں استعمال ہوتے ہیں اِن سے آپ ضرور واقف ہوں گے۔

\حصہء{اکائیوں کی تبدیلی}
	ہمیں بعض اوقات طبی مقداروں کی اکائی تبدیل کرنے کی ضرورت پیش آتی ہے۔ اس ترکیب میں ہم اصل پیمائش کو ایک تبادلی جذ جو اکائی کے برابر اکائیوں کا نسبت ہوتا ہے سے ضرب دیتے ہیں۔ مثال کے طور پر چونکہ ایک منٹ اور ساٹھ سیکنڈ مماثل دورانیہ کو ظاہر کرتے ہیں لحاظہ درج ذیل ہو گا۔
	\[\frac{1 min}{60 s} = 1\]
	یا
	\[\frac{60 s}{1 min} = 1\]
	یوں \(\frac{1 min}{60 s}\) یا \(\frac{60 s}{1 min}\) کے نسبت کو تبادلی جذ کے طور پر استعمال کیا جا سکتا ہے۔ ہم ہرگز \(\frac{1}{60}=1\) یا \(60 = 1\) نہیں لکھ سکتے۔ ہر عدد اور اسکی اکائی کو اکٹھے رکھنا ہو گا۔\\چونکہ اکائی سے ضرب دینے سے مقدار کی قیمت تبدیل نہیں ہوتی لحاظہ ہم جہاں چاہیں تبادلی جذ کا استعمال کر سکتے ہیں۔ ایسا کرتے ہوئے ہم غیر ضروری اکائیوں کو منسوخ کر سکتے ہیں۔ مثال کے طور پر دو منٹوں کو سیکنڈوں میں تبدیل کرتے ہوئے درج ذیل لکھا جائے گا۔
	\begin{align}
		2 min = (2 min)(1) = (2 min)(\frac{60 s}{1 min}) = 120 s
	\end{align}
	اگر تبادلہ جذ ضرب متعارف  کرنے سے غیر  ضروری  اکائیاں ایک دوسرے کے ساتھ منسوخ نہ ہوتی ہوں تب جذ ضربی کو اُلٹا کر دوبارہ کوشش کریں۔ اکائیوں کی تبادلہ میں اکائیوں پر متغیرات اور اعداد کے الجبرائی قواعد لاگو ہوں گے۔

\حصہء{لمبائی}
	سن 1972 میں فرانس کے نوزائیدہ جمہوریہ نے ناپ اور تول کا ایک نیا نظام قائم کیا۔ اسی کا سنگِ بنیاد میٹر تھا جو قطب شمال سے خط استوا کے فاصلے کا کڑوڑواں حصہ لیا گیا بعد میں عملی وجوہات کے بنا اس زمینی معیار کو ترک کرتے ہوئے پیرس شہر کے قریب ناپ اور تول کے ایک بین الاقوامی محکمہ میں رکھے گئے پلاٹینم، آئریڈئم کے ڈنڈے پر لگائے گئے دو باریک لکیروں کے بیچ فاصلے کو میٹر کہا گیا۔ اس ڈنڈے کے بہترین نقل پوری دنیا کے معیار سازی تجربہ گاہوں کو بیشے گئے۔ ان ثانوی معیاروں سے مزید زیادہ قابلے رسائی معیار تیار کئیے گئے حتہ کے آخر کار ہر پیمائشی آلہ اس معیاری میٹر ڈنڈے پر مبنی تھا۔
	
	کچھ عرصہ کے بعد ایک دھاتی ڈنڈے پر دو باریک لکیروں کے بیچ فاصلہ سے زیادہ بہتر معیار کی ضرورت در پیش آئی۔ سن 1960 میں شواع کی طولِ موج پر مبنی میٹر کے ایک نئے معیار پر اتقاو کیا گیا۔ یہ معیار کرپٹن 86 جو کرپٹن کا ایک مخصوص ہم جا ہے کے جوہروں سے خارج ایک مخصوص سرخ-نارنگی شواع کی 73.1650763 طولِ موج کے برابر فاصلہ لیا گیا۔ یہ شواع دنیا میں کہیں پر بھی گیس کے اخراگی نلی سے حاصل کی جا سکتی ہے۔ طولِ موج کی یہ تعداد اس لئیے منتخب کی گئی تا کہ نیا معیار پورانے میٹر کے قریب سے قریب تر ہو۔
	
	زیادہ سے زیادہ مطلوبہ درستگی کو آخر کار کرپٹن 86 کا معیار پورا نہیں کر سکتاتھا لحاظہ سن 1983 میں ایک نڈر فیصلہ کیا گیا۔ ناپ اور تول کے سترویں عمومی اجلاس میں درج ذیل تہ کیا گیا۔
	
	"خلاء میں ایک سیکنڈ کے \(\frac{1}{299792458}\) حصہ میں روشنی کے تہ شدہ فاصلہ کو ایک میٹر قرار دیا گیا"۔
	
	وقت کا یہ دورانیہ یوں منتخب کیا گیا کہ روشنی کی رفتار c ٹھیک ٹھیک درج ذیل ہو۔
	\[c = 299792458 m/s\]
	روشنی کی رفتار کی انتہائی درست پیمائش کرنا ممکن ہوا تھا لحاظہ روشنی کی رفتار کو استعمال کرتے ہوئے میٹر اخذ کرنا ایک بہتر قدم تھا۔
	
	جدول 3.1 میں لمبائیوں کی ایک بڑی ذات دیکھائی گئی ہے۔ جو کائنات سے لے کر انتہائی چھوٹی چیزوں کی لمبائیاں دیتا ہے۔
	\begin{table}[h!]
	\caption{چند تخمینی لمبائیاں}
\label{جدول_پیمائش_چند_تخمینی_لمائیاں}
\centering
\begin{tabular}{r l} 
	\toprule
  پیمائش& میٹروں میں لمبائی\\
\midrule
 اول ترین  پیدا کہکشاں تک فاصلہ& $2\times 10^{26}$ \\
  اندرومدا کہکشاں تک فاصلہ & $2\times 10^{22}$\\
  قریب ترین ستارہ تک فاصلہ & $4\times 10^{16}$\\
  پلوٹو تک فاصلہ & $6\times 10^{12}$\\
 زمین کا رداس & $6\times 10^{6}$ \\
  بلند ترین پہاڑی کی  اونچائی & $9\times 10^{3}$\\
 صفحہ کی موٹائی &  $1\times 10^{-4}$\\
  علامتی وائرس کی لمبائی & $1\times 10^{-8}$\\
  ہائیڈروجن  جوہر کا رداس & $5\times 10^{-11}$\\
  پروٹان کا رداس & $1\times 10^{-15}$\\
\bottomrule
		\end{tabular}
	\end{table}


\حصہء{بامعنی اعداد اور اشاریہ}
	مثال کے طور پر آپ ایک مسلے پر کام کر رہے ہیں جس میں ہر قیمت دو ہندسوں پر مشتمل ہوتی ہے۔ ان ہندسوں کو بامعنی ہندسے کہتے ہیں۔ اپنا جواب پیش کرتے ہوئے آپ اتنے ہی ہندسے استعمال کریں گے اگر آپ کا مواد دو ہندسوں میں دیا گیا ہو تب آپ کا جواب بھی دو ہندسوں پر مشتمل ہونا چاہئیے۔ اگر چہ آپ کے کیلکولیٹر میں زیادہ ہندسے نظر آئیں گے یہ ہندسے بے معنی ہوں گے۔
	
	اس کتاب میں دئیے گئے مواد میں کم سے کم با معنی ہندسوں کے برابر حساب کے اختتامی نتائج کو پورمپور کر کے پیش کیا جائے گا۔ ہاں بعض اوقات ایک اضافی ہندسہ بھی رکھا جائے گا۔ اگر ضائع کئیے جانے والے ہندسوں میں بایاں ترین ہندسہ پانچ کے برابر یا اس سے بڑا ہو تب آخری رہنے دیا گیا ہندسے کو اوپر جانب پورمپور کیا جاتا ہے۔ دیگر صورت اس کو اپنے حال میں ہی رکھا جاتا ہے۔ مثال کے طور پر 3516.11 کو تین با معنی ہندسوں تک پورمپور کرکے 4.11 جبکہ 3279.11 کو تین با معنی ہندسوں تک پورمپور کرتے ہوئے 3.11 لکھا جائے گا۔ اس کتاب میں نتائج پیش کرتے ہوئے پورمپور استعمال کئیے گئے ہونے کے باوجود $\approx$ کے بجائے = کی علامت استعمال کی جائے گی۔
	
	ایک عدد مثلاً 15.3 یا  $\times 10^{3}$15.3 میں با معنی ہندسوں کی تعداد صاف ظاہر ہے البتہ عدد 3000 میں با معنی ہندسے کتنے ہوں گے؟ کیا یہ صرف ایک با معنی ہندسے تک $\times10^{3}$3 معلوم ہے یا یہ ہمیں چار با معنی ہندسوں تک  $\times 10^{3}$000.3 تک معلوم ہے؟ اس کتاب میں 3000 کی طرح اعداد میں تمام صفروں کو بامعنی عدد تصور کیا جائے گا۔
	
	بامعنی ہندسے اور اشاریہ مقامات دو علیحدہ علیحدہ چیزیں ہیں۔ چرد ذیل لمبائیاں 6.35 ملی میٹر، 56.3 میٹر اور 00356.0 میٹر پر غور کریں۔ ان تمام میں تین با معنی ہندسے جبکہ بلترتیب ایک، دو اور پانچ اشاریہ مقامات پائے جاتے ہیں۔
	
\ابتدا{مثال}
 \موٹا{ دھاگے کا گیند، مقدار کا اندازاً رتبہ۔}
 
	دنیا میں دھاگے کے سب سے بڑے گیند کا رداس 2 میٹر ہے اس دھاگے کی کل لمبائی L کتنی ہوگی؟ اگر چہ ہم گیند سے دھاگہ کھول کر لمبائی L ناپ سکتے ہیں تاہم ہم ایسا نہیں کرنا چاہتے ہیں۔ ہم حساب کے ذریعہ اس کی لمبائی کا تخمینہ لگانا چاہتے ہیں۔
	
	\موٹا{حساب}
	
	فرض کریں یہ گیند کروی ہو جس کا رداس R=2 میٹر ہے۔ دھاگہ لپیٹتے ہوئے قریبی حصوں کے بیچ خالی جگہ پائی جاتی ہے۔ ان خالی جگہوں کو مدِ نظر رکھتے ہوئے ہم دھاگے کا عمودی تراش ذرا زیادہ تصور کرتے ہیں۔ ہم کہتے ہیں کہ دھاگے کا عمودی تراش چکور ہے جس کی زلی لمبائی d=4 ملی میٹر ہے۔ یوں اس کا رقبہ عمودی تراش $d^{2}$ اور لمبائی L ہوگا، دھاگے کا کل حجم درج ذیل ہوگا-
	\[V=(\text{\RL{رقبہ عمودی تراش}})(\text{\RL{لمبائی}})=d^{2}L\] 
	جو گیند کے حجم  \عددی{\tfrac{4}{3}\pi R^3} کے برابر ہوگا اور چونکہ $\pi$ تقریباً 3 کے برابر ہے لہٰذا  اس حجم کو $4R^{3}$ لکھا جا سکتا ہے۔ یوں درج ذیل ہوگا۔
	\[d^{2}L=4R^{3}\]
	یا
	\[L=\frac{4R^{3}}{d^{2}}\]
	\[L=\frac{4(2m)^3}{(4\times 10^{-3}m)^{2}}\]
	\[L=2\times 10^{6}m\approx 10^{6}m=10^{3}km\]
	(اتنے سادہ حساب کے لئیے کیلکولیٹر کی ضرورت پیش نہیں آنی چاہئیے۔)   قدر  کے  قریبی رتبہ تک اس گیند میں تقریباً 1000 کلو میٹر دھاگہ پایا جاتا ہے۔
	\انتہا{مثال}
	%%%KKKK AM HERE
	

\حصہ{وقت}
	وقت کے دو پہلو ہیں۔ روز مرہ زندگی میں ہم وقت جاننا چاہتے ہیں تاکہ دن کے کام کاج کو کسی ترتیب سے رکھنا ممکن ہو۔ سائنس کی دنیا میں ہم عموماً یہ جاننا چاہتے ہیں کہ ایک واقعہ کتنے دیر کے لیئے وقوع مذیر ہوا۔ یوں وقت کے کسی بھی معیار کو دو سوالات کا جواب دینا ہوگا: کب ہوا؟ اسکا دورانیہ کتنا تھا؟ جدول $\num{1.4}$ میں چند وقتی وقفوں کو پیش کیا گیا ہے۔
	
	\begin{table}[h!]
		\caption{چند تخمینی دورانیے}
	\label{جدول_پیمائش_تخمینی_دورانیے}
		\centering
		\begin{tabular}{r l}
\toprule
پیمائش & سیکنڈ میں دورانیہ \\
\midrule
پروٹان کا عرصہ  حیات (محض  اندازہ)  & $3\times 10^{40}$\\
کائنات کی عمر & $5\times 10^{17}$\\
ہرم  خوفو   (فرعون)  کی عمر & $1\times 10^{11}$\\
انسانی زندگی  (متوقع) & $2\times 10^{9}$ \\
ایک دن & $9\times 10^{4}$ \\
انسانی دل کی دھڑکنوں کے بیچ وقفہ & $8\times 10^{-1}$\\
میون کا عرصہ حیات & $2\times 10^{-6}$\\
تجربہ گاہ میں  مختصر ترین   شعاع کا دورانیہ & $1\times 10^{-16}$\\
  غیر مستحکم ترین  ذرے کا عرصہ حیات & $1\times 10^{-23}$\\
پلانک وقفہ\حاشیہد{ابتدائی دھماکہ کے اتنی دیر  بعد طبیعیات کے قواعد  قابل اطلاق ہیں۔}  & $1\times 10^{-43}$\\
\bottomrule
		\end{tabular}
	\end{table}
ایک ایسا مظہرجو اپنے آپ کو دہراتا ہو وقت کا ممکنہ معیار بن سکتا ہے۔ اپنے محور کے گرد زمین کا ایک چکر جو دن کی لمبائی تعین کرتا ہے کو یوں صدیوں تک استعمال کیا گیا۔ ایک کوارٹز گھڑی جس میں ایک کوارٹز چھلّا کو مسلسل ارتعاش پذیر رکھا جاتا ہے کی پیمانہ بندی زمین کے گھومنے کے ساتھ فلکیاتی مشاہدات کے ذریعہ تجربہ گاہ میں وقتی وقفوں کو ناپنے کے لیئے استعمال کیا جا سکتا ہے۔ تاہم جدید سائنس و انجینئرنگ میں درکار درستگی کی حد تک ایسی پیمانہ بندی ممکن نہیں ہے۔

بہتر معیارِ وقت کی ضرورت کے درپیش جوہری گھڑیاں تیار کی گئیں۔ سن 1967 میں ناپ و طول کے تیرویں عمومی اجلاس میں سیزیم گھڑی پر مبنی معیاری سیکنڈ پر اتفاق کیا گیا۔

سیزیم133 جوہر سے خارج ایک مخصوص طولِ موج کی شعاع کے $\num{9192631770}$ ارتعاش کو درکار وقت کو ایک سیکنڈ کہا گیا۔

جوہری گھڑیاں اتنی بلاتضاد ہوتی ہیں کہ دو سیزیم گھڑیوں کو چھ ہزار سال چلنا ہو گا تاکہ اِن میں ایک سیکنڈ کا فرق پیدا ہو۔ اِس وقت تیار کی جانے والی گھڑیوں کی درستگی $10^{18}$ میں ایک حصہ کے برابر ہے یعنی $10^{18}$ سیکنڈ (جو تقریباً \عددی{3\times 10^{10}} سال ہے)  میں صرف ایک سیکنڈ کا فرق ہو سکتا ہے۔
\حصہ{کمیت} 
\جزوحصہء{معیاری کلوگرام} 
فرانس کے شہر پیرس کے قریب ناپ و طول کے بین الاقوامی مہکمہ میں رکھے گئے پلاٹینم اور آئریڈیم کے ایک سلنڈر کو بین الاقوامی معاہدہ کے تحت ایک کلوگرام کمیت منتخب کیا گیا۔ اس کی بہترین نقلیں دنیا کی بیشتر معیار سازی تجربہ گاہوں کو بھیجی گئی ہیں جن کو استعمال کرتے ہوئے ترازو کی مدد سے کسی بھی جسم کی کمیت ناپی جاسکتی ہے۔ جدول 1.5 میں 83 قدری رطبہ تک کلوگرام کی صورت میں  چند کمیتیں پیش کی گئی ہیں۔
\begin{table}[h!]
	\centering
	\begin{tabular}{rl}
\toprule
چیز & کلوگرام میں کمیت\\
\midrule
معروف کائنات & $1\times 10^{53}$\\
ہماری کہکشاں & $2\times 10^{41}$\\
سورج & $2\times 10^{30}$\\
چاند & $7\times 10^{22}$\\
سیارچہ ایروز & $5\times 10^{15}$\\
چھوٹے پہاڑ & $1\times 10^{12}$\\
بحر لائنر & $7\times10^{7}$\\
ہاتھی & $5\times10^{3}$\\
انگور & $3\times10^{-3}$\\
دھول کی سپیک & $7\times10^{-10}$\\
پینسلن سالمہ & $5\times10^{-17}$\\
یورینیم جوہر & $4\times10^{-25}$\\
 پروٹان & $2\times10^{-27}$\\
 الیکٹران & $9\times10^{-31}$\\
 \bottomrule
	\end{tabular}
\caption{کچھ تخمینی کمیتیں}
\label{tab:my_label}
\end{table}
\جزوحصہء{دوم معیار کمیت} 
جوہروں کی کمیت کا موازنہ معیاری کلوگرام کی بجائے زیادہ درستگی کے ساتھ دیگر جوہروں کے ساتھ کیا جا سکتا ہے۔ اَسی کی بنا ہم دوم معیار کمیت بھی رکھتے ہیں۔ یہ کاربن 12 جوہر ہے جس کو بین الاقوامی معاہدہ کے تحت 12 جوہری کمیتی اکئیاں کی کمیت مختص کی گئی ہے۔ ان دو اکائیوں کے بیچ رشتہ درج ذیل ہے۔
\begin{align}
	\num{1} u = \SI{1.66053886e-27}{\kilogram}
\end{align}
جس کے آخری دو ہندسوں میں عدم یقینیت $\pm 10$ ہے۔ سائنس دان کافی درستگی کے ساتھ تجربہ کے ذریعہ کسی بھی جوہر کی کمیت کو کاربن 12 کی کمیت کی لحاظ سے تعین کر سکتے ہیں۔ اس وقت کمیت کی عام اکئیاں مثلاً کلوگرام کو استعمال کرتے ہوئے ہم اتنی درستگی حاصل کرنے سے قاصر ہیں۔
\جزوحصہ{کثافت} 
کثافت $\rho$ سے مراد  اکائی حجم میں کمیت ہے۔
\begin{align}
	\rho = \frac{m}{V}
\end{align}
اس پر باب 14 میں مزید تبصرہ کیا جائے گا۔ کثافت کو عام طور پر کلوگرام فی مربع میٹر یا گرام فی مربع سنٹی میٹر میں ناپا جاتا ہے۔ پانی کی کثافت ایک گرام فی مربع سنٹی میٹر یا ایک ہزار کلوگرام فی مربع میٹر کع عموماً موازنہ کے لیئے استعمال کیا جاتا ہے۔ پانی کی کثافت کے لحاظ سے پلاٹینم کی کثافت تقریباً اکیس گناہ جبکہ لکڑی کی کثافت صرف چونسٹھ فیصد ہوتی ہے۔



%\immediate\closeout\tempfileTerms

%=========================================================
%\addcontentsline{toc}{chapter}{حوالہ}     %it is standard not to include References in toc. 
%\renewcommand*{\bibname}{حوالہ}      %this command has to be placed right here
%\begin{thebibliography}{99}\label{حوالہ_بیرونی_مواد}
%\begin{otherlanguage}{english}
%%\cite{حوالہ_کریزگ_الف_گیارہ}         this is how it is referenced in the text
%\bibitem{حوالہ_کریزگ_الف_سات}
 %Coddington, E. A. and N. Levinson, Theory of
%Ordinary Differential Equations. Malabar, FL: Krieger,
%1984.
%\bibitem{حوالہ_کریزگ_الف_گیارہ}
%Ince, E. L., Ordinary Differential Equations. New
%York: Dover, 1956.
%\bibitem{حوالہ_کریزگ_ب_اٹھارہ}
%Watson, G. N., A Treatise on the Theory of Bessel Functions. 2nd ed. Cambridge: University Press, 1944.
%\end{otherlanguage}
%\end{thebibliography}
%============================================================

\chapter*{جوابات}
\addcontentsline{toc}{chapter}{جوابات}
\immediate\closeout\tempfile
\begin{multicols}{2}
{\small
\input{answer}
}
\end{multicols}

%=======================
\appendix
\renewcommand{\theequation}{\arabic{equation}}  %this ensures that the chapter number is not included
%\include{./tex/calculusAppendixA}
%

%\include{./tex/emtEndOfBookTableDivergenceCurlGradientLaplacian}
%\include{./tex/toDoList}
%\include{./tex/emtQuestions}

\backmatter

\cleardoublepage
%\include{./tex/emtDataTables}        %appendices
%\include{./tex/emtLinearAlgebra}
%\include{./tex/emtCoordinatesRelations}
%================

%=================

\renewcommand*{\indexname}{فرہنگ}      %this command has to be placed right here just before printindex command
\cleardoublepage
\addcontentsline{toc}{chapter}{فرہنگ}

%\printindex


\end{document}
