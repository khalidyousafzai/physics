HR_p8b
خلاصہ طبیعت میں پیمائش 
کبھی مقداروں کی پیمائش پر طبیعت مبنی ہے۔ کچھ طبی مقدار (مثلاً لمبائی، وقت، اور کمیت) کو بنیادی مقدار کے طور پر چنا گیا ہے؛ ہر ایک کی تعریف ایک معیار کے مطابق کی گئی ہے اور اس کو پیمائش کی اکائی (مثلاً 
\(\SI{\meter}\)
، 
\(\SI{\second}\)
، اور 
\(\SI{\kilo\gram}\))
مختص کی گئی ہے دیگر طبیعی  مقداروں کی تعریف ان بنیادی مقداروں اور ان کی معیار اور اکائیوں کی صورت میں کی جاتی ہے بین الاقوامی اکائی اس کتاب میں بین الاقوامی اکائی 
\(SI\) 
استعمال دی گئی ہے۔ جدول 
\(1.1\)
میں دکھائی گئی تین طبی مقدار ابتدائی بابوں میں استعمال کیے جائیں گے۔ بین الاقوامی معاہدوں کے تحت ان بنیادی مقداروں کے معیار طے کیے گئے ہیں جو ہر ایک کے لیے قابل رسائی اور غیر متغیر ہیں۔ بنیادی مقدار اور ان سے اخذ دیگر مقدار کی تمام طبی پیمائشیں انہی معیار کے تحت کی جاتی ہے۔ جدول 
\(1.2\)
میں پیش کی گئی علامتیں اور سابقہ استعمال کرتے ہوئے پیمائش کی سادہ علامتیت ملتی ہے۔ اکائیوں کی تبدیلی اکائیوں کی تبدیلی زنجیریں طریقہ استعمال کرتے ہوئے کی جا سکتی ہے۔
لمبائی ایک انتہائی زیادہ معین وقتی وقفہ کے دوران روشنی جتنا فاصلہ طے کرتی ہے، اسے 
\(\SI{meter}\)
کی تعریف لیا جاتا ہے وقت 
\(\SI{second}\)
کی تعریف ایک جوہر 
\(cesium-133\)
سے خارج روشنی کی صورت میں کی جاتی ہے۔ معیار برقرار رکھنے کی تجربات گاہوں میں موجود جوہری گڑیوں سے حاصل وقت کے درست اشارات کی تفصیل پوری دنیا میں کی جاتی ہے۔ کمیت پیرس شہر کے قریب رکھے گئے پلیٹینم اریڈیم کمیتی معیار 
 \(\SI{\kilo\gram}\)
کی تاریخ ہے۔ جوہری پیمانہ پر پیمائش کے لیے جوہری کمیٹی اکائی استعمال کی جاتی ہے جس کی تعریف کاربن 
\(12\)
جوہر کی صورت میں کی جاتی ہے۔کثافت کسی بھی چیز کی کثافت 
\(\SI{\rho}\)
سے مراد اکائی حجم میں اس کی کمیت ہے۔ 
\حوالہ{1.8} 
سوالات برائے حصہ 
\(1.1\)
بشمول لمبائی مختلف چیزوں کی پیمائش 
\ابتدا{سوال}
سوال 1 
زمین تخمینا ایک کرہ کی طرح ہے جس کا رداس 
\(6.37×10**6\SI{\meter}\)
ہے۔ 
(الف) اس کا محیط 
\(\SI{\kilo\meter}\)
میں، 
(ب) اس کا سطحی رقبہ مربع 
 \(\SI{\kilo\meter}\)
میں، اور اس کا حجم کھابی 
 \(\SI{\kilo\meter}\)
میں کتنا ہے؟ 
\انتہا{سوال}

\ابتدا{سوال}
سوال 2 
اشاعت قاری میں لمبائی کی مستمل اکائی پوائنٹ ہے 
\(pt\)
جس کی تعریف 
\(1/72\)\(\inch\)
ہے۔ 
\(0.1\)\(\inch**2\) 
رقبہ کو مربع پوائنٹ کی روپ میں لکھیں۔ 
\انتہا{سوال}

\ابتدا{سوال}
سوال 3
\(\SI{1}{\micro\meter}\)
کو عموما مائیکرون کہتے ہیں۔ 
(الف) کتنے مائیکرون 
\(\SI{1}{\kilo\meter}\)
کے برابر ہوں گے؟
(ب) 
\(\SI{1}{\centi\meter}\)
کا کتنا حصہ 
\(\SI{1}{\micro\meter}\) 
ہوگا؟ 
(ج) کتنے مائیکرونز ایک گز کے برابر ہوں گے؟
\انتہا{سوال}

\ابتدا{سوال}
سوال 4
اس کتاب میں فاصلوں کو پوائنٹ اور بیکاز دیکھائیوں میں نہ پا گیا ہے:
\(12\)
پوائنٹ مساوی ہے 
\(1\)
پیکا، اور 
\(6\)
پیکا مساوی ہے 
\(\SI{1}{\inch}\)
۔ اگر کتاب میں ایک شکل 
 \(\SI{0.80}{\centi\meter}\)
غلط رکھی گئی ہو، تب یہ 
(الف) پیکا کے اکائیوں میں اور 
(ب) پوائنٹ کی اکائیوں میں کتنا غلط رکھا گیا ہے؟ 
\انتہا{سوال}

\ابتدا{سول}
سوال 5
ایک مقابلے میں گھوڑے 
\(4.0\) 
فرلانگ کا فاصلہ دوڑ لگا کر طے کرتے ہیں۔ اس فاصلے کو 
(الف) اسا اور 
(ب) زنجیر 
کی صورت میں لکھیں۔
\(\text\RL{فرلانگ}=\SI{201.168}{\meter}\)
،
\(\text\RL{اسا}=\SI{5.0292}{\meter}\)
، اور 
\(\text\RL{زنجیر}=\SI{20.117}{\meter}\)
\انتہا{سوال}
