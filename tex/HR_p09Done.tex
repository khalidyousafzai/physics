%HR-p09
%edited. figures and external references pending
%---------------------------------
%Q6, p09
%Q6, Q7, Q8, Q10   below is different from that of the book. its answer must be changed accordingly.
%must work on Q6 and Q7  to get a better table, say havine سیر، کلو، دڑے

\ابتدا{سوال}
 جدول   \حوالہ{جدول_پیمائش_کلو_تا_ملی_گرام}مکمل کریں۔
\begin{table}
\caption{ملی گرام، گرام، اور کلو گرام کی چند قیمتیں۔}
\label{جدول_پیمائش_کلو_تا_ملی_گرام}
\begin{tabular}{CCCC}
\toprule
&\si{\milli\gram}&\si{\gram}&\si{\kilo\gram}\\
\midrule
\SI{300}{\milli\gram}&&&\\
\SI{0.50}{\gram}&&&\\
\SI{0.02}{\kilo\gram}&&&\\
\bottomrule
\end{tabular}
\end{table}
(ا) جدول مکمل کریں۔  
(ب)  \عددی{\SI{55}{\milli\gram}}  کتنے   \عددی{\si{\kilo\gram}} کے برابر  ہو گا؟
(ج)   \عددی{\SI{12}{\centi\meter\cubed}}   حجم کتنے     لٹر کے برابر ہو گا؟
\انتہا{سوال} 
%--------------------------------
%Q7 p09
\ابتدا{سوال} \شناخت{سوال_پیمائش_ایکڑ_فٹ}
ماقوائی معمار  پانی کا  حجم  عموماً    ایکڑ فٹ  میں ناپتے ہیں، جس سے مراد ایک  ایکڑ   رقبے پر ایک فٹ  گہرا پانی ہے۔ ایک شہر جس کا رقبہ   \عددی{\SI{26}{\kilo\meter\squared}} ہے میں   \عددی{30} منٹ  کی بارش \عددی{2} انچ پانی برساتی ہے۔  شہر پر کتنا ایکڑ فٹ پانی برستا ہے؟ 
\انتہا{سوال} 
%------------------------------
\ابتدا{سوال} 
ایک سڑک   \عددی{32} میل  اور   \عددی{5} فرلانگ لمبی ہے۔ اس کی لمبائی   \عددی{\si{\kilo\meter}} میں کتنی ہوگی؟ 
\انتہا{سوال} 
%----------------------------------
%Q9
\ابتدا{سوال}
\اصطلاح{بہر  منجمد جنوبی   }\فرہنگ{بحر منجمد جنوبی}\حاشیہب{antarctica}\فرہنگ{antarctica}    تقریباً نیم دائری ہے (شکل   \عددی{1.5})  جس کا رداس   \عددی{\SI{2000}{\kilo\meter}} ہے۔ اس میں برف کی اوسط موٹائی   \عددی{\SI{3000}{\meter}}  ہے ۔ بحر منجمد جنوبی میں کتنے   \عددی{\si{\cubic\centi\meter}} برف پایا جاتا ہے؟ (زمین کی سطح  مستوی تصور کریں۔)
\انتہا{سوال}
%----------------------
%Module 1.2 p09
 \حصہء{وقت}
 %---------------------
 %Q10
\ابتدا{سوال} 
بہت وسیع ممالک مثلاً روس میں مختلف مقامات پر گڑیوں کا وقت ایک دوسرے سے مختلف ہوتا ہے۔ (ا)   خط تول بلد کے   کتنے درجے چلنے کے بعد ایک گھنٹے کا فرق پایا جائے گا؟ (اشارہ: زمین   \عددی{24}
گھنٹے میں   \عددی{\SI{360}{\degree}} گھومتی ہے۔)  ایک      خط تول بلد کتنے منٹ کے برابر ہوگا؟
\انتہا{سوال}
%-------------------------------
\ابتدا{سوال} 
  فرانسیسی  انقلاب کے بعد تقریباً   \عددی{10}  سال تک حکومت کوشش کرتی رہی کہ وقت کی پیمائش مضرب   \عددی{10} رکھی جائے؟ ایک ہفتہ میں   \عددی{10}  دن، ایک  دن میں   \عددی{10} گھنٹے، ایک گھنٹہ میں   \عددی{100} منٹ، اور ایک منٹ میں   \عددی{\SI{100}{\second}} رکھے گئے ۔

(ا) فرانسیسی اعشاری  ہفتہ اور معیاری ہفتہ کی نسبت ، اور   (ب) فرانسیسی اعشاری  سیکنڈ اور معیاری سیکنڈ کی نسبت کیا ہے؟ 
\انتہا{سوال} 
%----------------------------------------
%Q12
\ابتدا{سوال} 
دنیا کا تیز ترین بڑھتا پودا  \قول{ہسپرویوکا } کہلاتا  ہے جو   \عددی{14} دن میں   \عددی{\SI{3.7}{\meter}} بڑا۔  پودے کے بڑھنے کی شرح   \عددی{\si{\micro\meter\per\second}}
کتنی ہے؟ 
\انتہا{سوال} 
%-----------------------------------
\ابتدا{سوال} 
تین گھڑیاں الف، ب، اور پ مختلف رفتار سے چلتی ہیں اور بیک وقت صفر نہیں دیتی۔ شکل   \حوالہء{1.6} میں چار موقوں  پر ان کی  بیک  وقت پیمائش  دکھائی  گئی ہے۔ (مثال کے طور پر جس لمحہ  گھڑی ب   \عددی{\SI{25}{\second}}  دیتی ہے  ، گھڑی پ   \عددی{\SI{92}{\second}} دیتی ہے۔) اگر دو واقعات گھڑی الف  پر   \عددی{\SI{600}{\second}}  فاصلے  پر  واقع ہوں، یہ (الف) گھڑی    ب  پر اور 
(ب) گھڑی    ج پر کتنے  فاصلے پر واقع ہوں گے؟   (ج) جس لمحہ گھڑی  الف \عددی{\SI{400}{\second}} دیتی ہے اس لمحہ گھڑی ب کیا دے گی؟  (د) جس وقت گھڑی     الف   \عددی{\SI{15}{\second}}
دیتی ہے، اس وقت گھڑی  ب کیا دے گی؟ ( قبل از صفر   اوقات منفی  علامت  کے تصور کریں۔)
\انتہا{سوال}
%----------------------------
%Q14
\ابتدا{سوال}
ایک درس (جو  \عددی{50} منٹ کا ہے)  تخمیناً  ایک خورد صدی  کا ہوگا۔

(ا)ایک خورد صدی    دورانیہ   کتنے منٹ ہوگا؟   (ب) درج ذیل کلیہ   استعمال کرتے ہوئے تخمین میں فی صد فرق تلاش کریں۔ 
\begin{align*}
\text{\RL{}}=\big(\frac{\text{\RL{اصل}}-\text{\RL{تخمیناً}}}{\text{\RL{اصل}}}\big)\, 100
\end{align*}
\انتہا{سوال} 
%----------------------------
\ابتدا{سوال} 
دو ہفتوں کا وقت کتنے   \عددی{\si{\micro\second}} ہوگا؟ 
\انتہا{سوال} 
%-------------------------
%Q16, p09
\ابتدا{سوال} 
معیاری وقت کا دارومدار  جوہری گھڑیوں پر ہے۔ اس سے بہتر معیار سیکنڈ  \اصطلاح{نابض }\فرہنگ{نابض}\حاشیہب{pulsars}\فرہنگ{pulsar} پر مبنی ہو سکتا ہے، جو  گھومتے  \اصطلاح{ نیوٹران تارے }\فرہنگ{تارہ!نیوٹران}\حاشیہب{neutron star}\فرہنگ{star!neutron} (انتہائی ٹھوس تارے جن میں صرف نیوٹران پائے جاتے ہیں)  ہیں۔ ان میں سے کئی انتہائی زیادہ مستحکم شرح سے گھومتے ہیں، اور ہر چکر کے دوران ایک مرتبہ زمین پر  شعاع ڈالتے ہیں  (سمندر کے کنارے \اصطلاح{  منارہ   نور }\فرہنگ{منارہ  نور}\حاشیہب{light house}\فرہنگ{light house} کی طرح جو سمندری جہاز کو خطرے سے آگاہ کرتا ہے)۔ نابض   \عددی{PSR\,1937\,\,21}  ایک ایسا تارہ ہے جو ایک  چکر   \عددی{\num{1.55780644887275}\pm 3 \, \si{\milli\second}}     میں پورا کرتا ہے، جہاں آخر  میں   \عددی{\pm 3}  آخری ہندسہ  میں عدم یقینیت دیتا ہے (اس کا    ہرگز \عددی{\SI{\pm 3}{\milli\second}}  مطلب نہیں)۔ 

(ا) یہ نابض    \عددی{7.00}  دنوں میں کتنے چکر کاٹتا ہے؟  (ب) یہ نابض    \عددی{10} لاکھ مرتبہ ٹھیک کتنے وقت میں چکر کاٹتا  ہے ، اور  (ج) اس سے وابستہ عدم یقینیت کیا ہوگا؟ 
\انتہا{سوال}
