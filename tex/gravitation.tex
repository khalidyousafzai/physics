%gravitation p354
\باب{تجاذب}
\موٹا{مقاصد}\\
اس حصہ کو پڑھنے کے بعد آپ ذیل کے قابل ہوں گے۔
\begin{enumerate}[1.]
\item
دو ذروں  کی کمیت اور ان کے بیچ فاصلے  کا ذروں کی باہمی تجاذبی  قوت  کے ساتھ تعلق نیوٹن کے   قانون تجاذب سے جان پائیں گے۔
\item
جان پائیں گے کہ مادے کا   یکساں کروی خول  ذرے کو   ، جو خول سے باہر ہو، بالکل اس طرح  کھینچتا ہے  جیسے  خول کی کمیت  خول کے مرکز پر واقع ہو۔
\item
ذرے پر دوسرے ذرے یا مادے کی  یکساں کروی تقسیم کی قوت تجاذب آزاد جسمی خاکہ   سے  ظاہر کر پائیں گے۔
\end{enumerate}

\موٹا{کلیدی تصورات}\\
\begin{itemize}
\item
کائنات میں ہر ذرہ دوسرے ذرے کو ذیل قدر کی  تجاذبی قوت سے اپنی طرف کھینچتا ہے:
\begin{align*}
F&=G\frac{m_1m_2}{r^2}\quad\quad\text{\RL{(نیوٹن کا قانون تجاذب)}}
\end{align*}
جہاں \عددی{m_1} اور \عددی{m_2} ذروں کی کمیتیں، \عددی{r} ان کے بیچ فاصلہ، اور \عددی{G=\SI{6.67e-11}{\newton\meter\squared\per\kilo\gram\squared}} تجاذبی مستقل  ہے۔
\item
وسیع اجسام کے بیچ تجاذبی قوت معلوم کرنے کی خاطر، جسم کے اندر تمام انفرادی  ذروں  پر انفرادی قوت کا مجموعہ (تکمل)  لینا ہو گا۔ تاہم، اگر  ایک جسم یکساں کروی خول  یا کروی تشاکل ٹھوس جسم ہو،بیرونی جسم پر اس کی  صافی تجاذبی  قوت معلوم کرتے وقت  خول یا ٹھوس جسم کی کمیت جسم کے مرکز پر تصور کی جا سکتی ہے۔
\end{itemize}

\حصہء{طبیعیات کیا ہے؟}
طبیعیات کا ایک مقصد ، قوت تجاذب کا    سمجھنا ہے۔ قوت تجاذب  ہمیں زمین پر رکھتی ہے، چاند کو  زمین کے گرد  مدار، اور زمین کو سورج کے گرد مدار میں رکھتی ہے۔اس کا اثر  ہماری \اصطلاح{ دودھیا کہکشاں }\فرہنگ{کہکشاں!دودھیا}\حاشیہب{milky way galaxy}\فرہنگ{galaxy!milky way} کے ہر کونے تک پہنچ کر، اربوں ستاروں، لاتعداد جوہر  اور ستاروں کے بیچ  دھول کے ذروں کو   کہکشاں میں  جکڑ کر رکھتا ہے۔ ہم  دودھیا  کہکشاں، جو ستاروں کا قرص نما جھرمٹ   ہے، کے کنارے کے قریب، کہکشاں کے مرکز سے \عددی{2.6e4}  نوری سال \عددی{(\SI{2.5e20}{\meter})}   فاصلے پر مرکز کے گرد آہستہ آہستہ طواف کرتے  ہوئے،  بستے ہیں۔

تجاذبی قوت  بین کہکشانی  فاصلے طے کر کے کہکشاں کے مقامی گروہ کو، جس میں دودھیا کہکشاں کے علاوہ \اصطلاح{ اندرومدا }\فرہنگ{کہکشاں!اندرومدا}\حاشیہب{Andromeda}\فرہنگ{galaxy!Andromeda}کہکشاں (شکل \حوالہء{13.1}) جو زمین  سے \عددی{2.3e6} نوری سال فاصلے پر ہے،   اور کئی \اصطلاح{  بالشتیا }\فرہنگ{کہکشاں!بالشتیا}\حاشیہب{dwarf}\فرہنگ{galaxy!dwarf} کہکشاں، جیسے \اصطلاح{  سحاب کبیر }\فرہنگ{سحاب کبیر}\حاشیہب{Large Magellanic Cloud}\فرہنگ{cloud!Large Magellanic}، شامل ہے۔ کہکشاں کا مقامی گروہ   از خود \اصطلاح{ مقامی  عظیم  خوشہ }\فرہنگ{مقامی عظیم خوشہ}\حاشیہب{Local Supercluster}\فرہنگ{Local Supercluster} کا حصہ ہے، جس کو تجاذبی قوت  انتہائی زیادہ کمیتی خطہ کی طرف، جو  \اصطلاح{عظیم  جالب }\فرہنگ{عظیم جالب}\حاشیہب{Great Attractor}\فرہنگ{Great Attractor} کہلاتا ہے،  کھنچ رہا ہے۔ یہ خطہ زمین سے  \عددی{3.0e8} نوری سال کے فاصلے پر، دودھیا کہکشاں کی دوسرے طرف ، واقع ہے۔ تجاذبی قوت اس سے بھی زیادہ دور رس ہے، چونکہ یہ پوری کائنات کو ، جس کا حجم بتدریج بڑھ  رہا ہے،  ایک ساتھ رکھتا ہے۔
%p355
