%gravitation p354
\باب{تجاذب}
\موٹا{مقاصد}\\
اس حصہ کو پڑھنے کے بعد آپ ذیل کے قابل ہوں گے۔
\begin{enumerate}[1.]
\item
دو ذروں  کی کمیت اور ان کے بیچ فاصلے  کا ذروں کی باہمی تجاذبی  قوت  کے ساتھ تعلق نیوٹن کے   قانون تجاذب سے جان پائیں گے۔
\item
جان پائیں گے کہ مادے کا   یکساں کروی خول  ذرے کو   ، جو خول سے باہر ہو، بالکل اس طرح  کھینچتا ہے  جیسے  خول کی کمیت  خول کے مرکز پر واقع ہو۔
\item
ذرے پر دوسرے ذرے یا مادے کی  یکساں کروی تقسیم کی قوت تجاذب آزاد جسمی خاکہ   سے  ظاہر کر پائیں گے۔
\end{enumerate}

\موٹا{کلیدی تصورات}\\
\begin{itemize}
\item
کائنات میں ہر ذرہ دوسرے ذرے کو ذیل قدر کی  تجاذبی قوت سے اپنی طرف کھینچتا ہے:
\begin{align*}
F&=G\frac{m_1m_2}{r^2}\quad\quad\text{\RL{(نیوٹن کا قانون تجاذب)}}
\end{align*}
جہاں \عددی{m_1} اور \عددی{m_2} ذروں کی کمیتیں، \عددی{r} ان کے بیچ فاصلہ، اور \عددی{G=\SI{6.67e-11}{\newton\meter\squared\per\kilo\gram\squared}} تجاذبی مستقل  ہے۔
\item
وسیع اجسام کے بیچ تجاذبی قوت معلوم کرنے کی خاطر، جسم کے اندر تمام انفرادی  ذروں  پر انفرادی قوت کا مجموعہ (تکمل)  لینا ہو گا۔ تاہم، اگر  ایک جسم یکساں کروی خول  یا کروی تشاکل ٹھوس جسم ہو،بیرونی جسم پر اس کی  صافی تجاذبی  قوت معلوم کرتے وقت  خول یا ٹھوس جسم کی کمیت جسم کے مرکز پر تصور کی جا سکتی ہے۔
\end{itemize}

\حصہء{طبیعیات کیا ہے؟}
طبیعیات کا ایک مقصد ، قوت تجاذب کا    سمجھنا ہے۔ قوت تجاذب  ہمیں زمین پر رکھتی ہے، چاند کو  زمین کے گرد  مدار، اور زمین کو سورج کے گرد مدار میں رکھتی ہے۔اس کا اثر  ہماری \اصطلاح{ دودھیا کہکشاں }\فرہنگ{کہکشاں!دودھیا}\حاشیہب{milky way galaxy}\فرہنگ{galaxy!milky way} کے ہر کونے تک پہنچ کر، اربوں ستاروں، لاتعداد جوہر  اور ستاروں کے بیچ  دھول کے ذروں کو   کہکشاں میں  جکڑ کر رکھتا ہے۔ ہم  دودھیا  کہکشاں، جو ستاروں کا قرص نما جھرمٹ   ہے، کے کنارے کے قریب، کہکشاں کے مرکز سے \عددی{\num{2.6e4}}  نوری سال \عددی{(\SI{2.5e20}{\meter})}   فاصلے پر مرکز کے گرد آہستہ آہستہ طواف کرتے  ہوئے،  بستے ہیں۔

تجاذبی قوت  بین کہکشانی  فاصلے طے کر کے کہکشاں کے مقامی گروہ کو، جس میں دودھیا کہکشاں کے علاوہ \اصطلاح{ اندرومدا }\فرہنگ{کہکشاں!اندرومدا}\حاشیہب{Andromeda}\فرہنگ{galaxy!Andromeda}کہکشاں (شکل \حوالہء{13.1}) جو زمین  سے \عددی{\num{2.3e6}} نوری سال فاصلے پر ہے،   اور کئی \اصطلاح{  بالشتیا }\فرہنگ{کہکشاں!بالشتیا}\حاشیہب{dwarf}\فرہنگ{galaxy!dwarf} کہکشاں، جیسے \اصطلاح{  سحاب کبیر }\فرہنگ{سحاب کبیر}\حاشیہب{Large Magellanic Cloud}\فرہنگ{cloud!Large Magellanic}، شامل ہے۔ کہکشاں کا مقامی گروہ   از خود \اصطلاح{ مقامی  عظیم  خوشہ }\فرہنگ{مقامی عظیم خوشہ}\حاشیہب{Local Supercluster}\فرہنگ{Local Supercluster} کا حصہ ہے، جس کو تجاذبی قوت  انتہائی زیادہ کمیتی خطہ کی طرف، جو  \اصطلاح{عظیم  جالب }\فرہنگ{عظیم جالب}\حاشیہب{Great Attractor}\فرہنگ{Great Attractor} کہلاتا ہے،  کھنچ رہا ہے۔ یہ خطہ زمین سے  \عددی{\num{3.0e8}} نوری سال کے فاصلے پر، دودھیا کہکشاں کی دوسرے طرف ، واقع ہے۔ تجاذبی قوت اس سے بھی زیادہ دور رس ہے، چونکہ یہ پوری کائنات کو ، جس کا حجم بتدریج بڑھ  رہا ہے،  ایک ساتھ رکھتا ہے۔
%p355
\اصطلاح{ثقب اسود  }\فرہنگ{ثقب اسود}\حاشیہب{black hole}\فرہنگ{black hole}   ، جو کائنات میں انتہائی پراسرار اجسام میں سے ایک ہے، کا دارومدار بھی اسی قوت پر ہے۔ جب  سورج  سے بڑا  ستارہ   زندگی کے اختتام کو پہنچتا ہے،  اس کے ذروں کے  بیچ تجاذبی قوت   ستارے کو   اپنے آپ پر  منہدم   کر کے ثقب اسود پیدا کرتی ہے۔ منہدم ستارے کی سطح پر تجاذبی قوت اتنی زیادہ   ہوتی ہے  کہ سطح سے   کوئی ذرہ نکل نہیں سکتا اور نا ہی  روشنی نکل سکتی ہے (اسی لئے اس کو \قول{ ثقب اسود }  یعنی \قول{ سیاہ  سوراخ } کہتے ہیں)۔ اگر کوئی ستارہ ثقب اسود کے زیادہ قریب پہنچے، ثقب اسود  کی توانا  تجاذبی قوت   ستارے کو   چیرپھاڑ  کر  ثقب (سوراخ) کے اندر کھینچ لیتی ہے۔ متعدد ستارے نوچنے   پر اس سے    \اصطلاح{بے پناہ کمیتی  ثقب اسود  }\فرہنگ{ثقب اسود!بے پناہ کمیتی}\حاشیہب{supermassive black hole}\فرہنگ{black hole!supermassive}بنتا ہے۔ ایسے  بھیانک  اور پراسرار اجسام سے کائنات  بھری نظر آتی ہے۔یقیناً  ہماری اپنی دودھیا کہکشاں  کے مرکز پر  ایک ثقب اسود پایا جاتا ہے، جو   القوس \عددی{A^*}  کہلاتا ہے، اور جس کی کمیت تقریباً  \عددی{\num{3.7e6}} شمسی  کمیت کے برابر   ہے۔ اس کی تجاذبی قوت اتنی توانا ہے کہ   قریبی ستارے مدار میں گھومتے  ہوئے صرف \عددی{15.2} سال میں القوس \عددی{A^*} کے گرد  چکر مکمل کرتے ہیں۔

اگرچہ  تجاذبی قوت  مکمل سمجھنے سے اب بھی ہم قاصر ہیں، اسے سمجھنے کا ابتدائی نقطہ  نیوٹن کا \ترچھا{ قانون تجاذب } ہے۔

%-------------------
%Newton's Law of Gravitation p355
\حصہء{نیوٹن کا قانون تجاذب}
مختلف مساوات پر بات کرنے سے قبل ذرہ سوچتے ہیں۔ہم زمین   کے ساتھ مس رہتے ہیں؛ مس رکھنے کی قوت اتنی زیادہ نہیں کہ ہمیں گھسیٹ کر چلنا پڑے اور نا ہی اتنی کم ہے کہ آئے دن سر چھت سے ٹکرائے۔ ساتھ ہی یہ قوت ہمیں زمین پر رکھتی ہے، تاہم اتنی طاقتور نہیں کہ ہم ایک دوسرے کے ساتھ جڑ جائیں۔ یقیناً  قوت کشش کا دارومدار  جسم میں مادے کی مقدار پر ہے۔ زمین میں مادے کی مقدار بہت  زیادہ ہے، لہٰذا زمین  کی کشش بھی بہت زیادہ ہے، جبکہ  انسان کے جسم میں مادے کی مقدار بہت کم ہے، اور اسی لئے دو انسانی اجسام کے بیچ قوت کشش  اتنی کم ہے کہ اسے محسوس کرنا ممکن نہیں۔مزید، ایک  مادہ  دوسرے مادہ کو کھینچتا ہے؛ ان کے بیچ قوت دفع نہیں پائی جاتی۔

ماضی میں انسان جانتا تھا کہ زمین ہمیں نشیب وار   کھینچتی ہے (خاص کر اس وقت جب زمین پر  گر نے سے ناک کی ہڈی ٹوٹتی تھی)، تاہم ان کا خیال تھا کہ یہ قوت صرف زمین کی خاصیت ہے جس کا آسمان میں  فلکیاتی اجسام کی  حرکت سے کوئی واسطہ نہیں۔ تاہم \سن{1665} میں، \عددی{23} سالہ،  اسحاق نیوٹن  کو اس بات کی سمجھ آئی کہ چاند کو یہی قوت زمین کے گرد گھومنے پر  مجبور کرتی ہے۔ بالخصوص، انہوں نے دکھایا کہ کائنات میں ہر جسم، ہر دوسرے جسم کو کھینچتا ہے۔  اجسام کا ایک دوسرے کی  طرف  حرکت کرنے کے  رجحان  کو\اصطلاح{ تجاذب }\فرہنگ{تجاذب}\حاشیہب{gravitation}\فرہنگ{gravitation} کہتے ہیں، اور اجسام میں وہ مادہ  جو  قوت کشش  پیدا کرتی ہے  \قول{کمیت } کہلاتا ہے۔ اگر  نیوٹن کے سر پر سیب گرنے کا  واقعہ درست ہو،  تب  سیب کی کمیت اور زمین کی کمیت کے بیچ کشش سیب کے گرنے کا سبب تھی۔ زمین کی کمیت اتنی زیادہ ہے کہ یہ قوت ، جو تقریباً \عددی{\SI{0.80}{\newton}} ہو گی، محسوس کی جا سکتی ہے۔ اس کے برعکس، دو قریب کھڑے  اشخاص کے بیچ قوت ، جو 
تقریباً \عددی{\SI{1}{\micro\newton}} ہو گی، محسوس نہیں کی جا سکتی۔

دو وسیع اجسام، جیسے دو شخص، کے بیچ تجاذبی توانائی کا حساب کرنا   مشکل ہو گا۔  ہم دو ذروں  (جن کی جسامت صفر تصور کی جائے گی) کے بیچ نیوٹن کے  قانون تجاذب   کے اطلاق پر غور کرتے ہیں۔ فرض کریں ذروں کی کمیت \عددی{m_1} اور \عددی{m_2} اور ان کے بیچ  فاصلہ  \عددی{r} ہے۔ ایک ذرے  کی دوسرے  ذرے پر قوت کشش کی قدر  ذیل ہو گی:
%eq 13.1
\begin{align}\label{مساوات_تجاذب_نیوٹن_کا_قانون_تجاذب_الف}
F=G\frac{m_1m_2}{r^2}\quad\quad\text{\RL{(نیوٹن کا قانون تجاذب)}}
\end{align}
جہاں \عددی{G}  \اصطلاح{تجاذبی مستقل }\فرہنگ{تجاذبی مستقل}\حاشیہب{gravitation constant}\فرہنگ{gravitation constant}ہے، جس کی قیمت ذیل ہے۔
%eq 13.2
\begin{gather}
\begin{aligned}
G&=\SI{6.67e-11}{\newton\meter\squared\per\kilo\gram\squared}\\
&=\SI{6.67e-11}{\meter\cubed\per\kilo\gram\per\second\squared}
\end{aligned}
\end{gather}
شکل \حوالہء{13.2a} میں  ذرہ \عددی{1} (کمیت \عددی{m_1}) پر ذرہ \عددی{2} (کمیت \عددی{m_2}) کی تجاذبی قوت \عددی{\vec{F}}  پیش ہے۔ قوت کا رخ ، ذرہ \عددی{2} کی طرف ہے اور چونکہ  ذرہ \عددی{2}  کی طرف ذرہ \عددی{1}  کشش محسوس کرتا ہے  لہٰذا  یہ\ترچھا{ قوت کشش  } کہلاتی ہے۔ قوت کی قدر مساوات \حوالہ{مساوات_تجاذب_نیوٹن_کا_قانون_تجاذب_الف} سے حاصل ہو گی۔ ذرہ \عددی{1} سے رداسی باہر نکلتی   محور \عددی{r}   ، جو ذرہ \عددی{2} سے  بھی گزرتی ہے، کے مثبت رخ کے ہمراہ  قوت \عددی{\vec{F}} عمل کرتی ہے (شکل \حوالہء{13.2b})۔ اکائی سمتیہ \عددی{\rhat}  ( جو بے بُعدی سمتیہ  ہے اور جس کی قدر \عددی{1} ہے) استعمال کرتے ہوئے، جو ذرہ \عددی{1}  سے دوری کے رخ  محور \عددی{r} کے ہمراہ  واقع ہے  (شکل \حوالہء{13.2c})، \عددی{\vec{F}} بیان کیا جا سکتا ہے۔ یوں، مساوات \حوالہ{مساوات_تجاذب_نیوٹن_کا_قانون_تجاذب_الف} استعمال کرتے ہوئے  ذرہ \عددی{1}  پر قوت     ذیل ہو گی۔
%eq 13.3
\begin{align}
\vec{F}=G\frac{m_1m_2}{r^2}\,\rhat
\end{align}

ذرہ \عددی{2} پر ذرہ \عددی{1} کی قوت کی قدر  وہی ہے  جو ذرہ \عددی{1} پر ذرہ \عددی{2} کی قدر  ہے، تاہم اس کا رخ  مخالف ہو گا۔ دونوں قوت مل کر  قانون سوم  جوڑی    قوت دیتے ہیں، اور ہم  دو ذروں کے بیچ قوت تجاذب  کی بات  کر سکتے ہیں جس کی قدر مساوات \حوالہ{مساوات_تجاذب_نیوٹن_کا_قانون_تجاذب_الف} دیتی ہے۔ دو ذروں کے بیچ قوت تجاذب  پر دیگر اجسام  کا اثر نہیں پایا جاتا، اگرچہ یہ اجسام ان ذروں کے  درمیان ہی  کیوں نہ ہوں۔ دوسرے لفظوں میں، کوئی  جسم ایک ذرے کو دوسرے ذرے کی تجاذبی قوت سے   بچا نہیں سکتا۔

دی گئی کمیت کے ذروں کی ، جو کسی مخصوص فاصلے پر ہوں،تجاذبی قوت کا زور  تجاذبی مستقل \عددی{G} پر منحصر ہے۔ اگر جادو  سے \عددی{G} کی قیمت دس گنّا کی جائے، زمین کی کشش آپ کو زمین پر گرا دیگی، اور \عددی{G} کی قیمت دس گنّا کم کر دی جائے، آپ چھلانگ  لگا کر  عمارت  پار کر پائیں گے۔

\موٹا{دیگر اجسام۔}\quad
اگرچہ، نیوٹن کا قانون تجاذب  ذروں کے لئے ہے،  جب تک اجسام کی جسامت ، اجسام کے بیچ فاصلے کے لحاظ سے بہت کم ہو،  قانون تجاذب  کا اطلاق وسیع اجسام  پر بھی  ممکن ہے۔ زمین اور چاند ایک دوسرے سے  اتنی دوری پر ہیں کہ انہیں تخمیناً ذرے تصور کیا جا سکتا ہے؛ لیکن سیب اور زمین  کا کیا ہو گا؟ سیب کے نقطہ نظر سے زمین اتنی وسیع ہے کہ اسے  ذرہ تصور  کرنا درست نہ ہو گا۔

نیوٹن نے (ذیل)    \ترچھا{مسئلہ خول }کے ذریعہ زمین اور سیب کا مسئلہ حل کیا۔

\ابتدا{قاعدہء}
مادے کا یکساں کروی خول ، کرہ سے باہر واقع  ذرے کو یوں کھینچتا ہے گویا  خول  کی تمام کمیت خول کے مرکز پر ہو۔
\انتہا{قاعدہء}
%-----------------------
 زمین کو خول کے اوپر خول چڑھا  جسم تصور کیا جا سکتا ہے۔ ہر خول بیرونی ذرے کو یوں کھینچتا ہے گویا خول کے مرکز پر خول کی تمام کمیت واقع ہو۔ یوں  سطح زمین   سے   اوپر ذرے کو تمام خول یوں کھینچتے ہیں گویا زمین کی پوری کمیت زمین کے مرکز پر واقع ہو۔ یوں، سیب کے نقطہ نظر سے زمین اس  ذرے کی مانند ہے جو زمین کے مرکز پر واقع ہے اور جس کی کمیت زمین کی کمیت کے برابر ہے۔

\موٹا{قانون سوم جوڑی قوت۔}\quad
فرض کریں، جیسا شکل \حوالہء{13.3} میں دکھایا گیا ہے، سیب کو زمین \عددی{\SI{0.80}{\newton}} قدر کی قوت سے نیچے کھینچتی ہے۔ اب  زمین کو سیب  \عددی{\SI{0.80}{\newton}} قدر کی  قوت سے اوپر کھینچے گا اور یہ قوت زمین کے مرکز پر عمل پیرا ہو گی۔ باب \حوالہء{5} کی زبان میں  یہ قوت مل کر نیوٹن کے قانون سوم  میں جوڑی قوت دیتی ہیں۔ اگرچہ، ان کی قدر ایک ہے، جب سیب رہا کیا جائے وہ قوت سیب پر مختلف اسراع عائد کرتی ہیں۔ سیب کے اسراع کی قدر \عددی{\SI{9.8}{\meter\per\second}} ہو گی، جو سطح زمین کے قریب  ، جانی پہچانی، آزاد جسمی اسراع ہے۔ سیب و زمین نظام  کے مرکز سے جڑی  حوالہ چوکھٹ  کے لحاظ سے زمین کے اسراع کی قدر صرف \عددی{\SI{1e-25}{\meter\per\second\squared}} ہو گی۔ یقیناً،  سیب کے گرنے کے دوران زمین ساکن نظر آتی ہے۔

%---------------------
%checkpoint 1 p356
\ابتدا{آزمائش}
ایک ذرہ باری باری  درج ذیل اجسام کے باہر رکھا جاتا ہے، جن کی انفرادی کمیت \عددی{m} ہے۔ (1) ایک بڑا یکساں ٹھوس کرہ،  (2) ایک بڑا یکساں کروی خول، (3) ایک چھوٹا یکساں ٹھوس کرہ، اور (4)  ایک چھوٹا  یکساں  خول۔ ہر مرتبہ ذرے اور کرہ کے مرکز کا فاصلہ  \عددی{d} رکھا جاتا ہے۔ ذرے پر قوت کی قدر کے لحاظ سے، اعظم قیمت اول رکھ کر، ان اجسام کی درجہ بندی کریں۔
\انتہا{آزمائش}
%--------------------

%13.2 gravitation and the principle of superposition p357
\حصہء{تجاذب اور اصول   انطباق}
\موٹا{مقاصد}\\
اس حصہ کو پڑھنے کے بعد آپ ذیل کے قابل ہوں گے۔
\begin{enumerate}[1.]
\item
جہاں ذرے پر ایک سے زیادہ    تجاذبی قوت عمل پیرا ہوں، قوتوں کو ظاہر کرنے والا   آزاد جسمی خاکہ بنا پائیں گے ، جس میں قوت سمتیات کی دم عین ذرے پر  ہو گی۔
\item
جہاں ایک سے زیادہ تجاذبی قوت ذرے پر عمل پیرا ہوں،  انفرادی قوت کو سمتیہ تصور کر کے تمام کا  مجموعہ لے کر صافی قوت تلاش کر پائیں گے۔
\end{enumerate}

\موٹا{کلیدی تصورات}\\
\begin{itemize}
\item
تجاذبی قوت   اصول   انطباق پر پورا اترتی ہے؛ یعنی، جہاں \عددی{n} ذرے باہم عمل کرتے ہوں، ایک ذرے پر ، جس کا  عرف \عددی{1} ہے، صافی 
قوت \عددی{\vec{F}_{1,\text{\RL{صافی}}}}  باقی تمام ذروں کو باری باری لے کر ،  انفرادی قوتوں کا  مجموعہ  ہو گا:
\begin{align*}
\vec{F}_{1,\text{\RL{صافی}}}=\sum_{i=2}^{n} \vec{F}_{1i}
\end{align*}
جہاں ذرہ \عددی{1} پر ذرہ \عددی{2}، \عددی{3}،\نقطے،  تا \عددی{n} کی قوتوں  کا سمتی مجموعہ لیا جائے گا۔
\item
وسیع جسم کی ذرے پر قوت \عددی{\vec{F}_1} معلوم کرنے کے لئے، وسیع جسم کو  تفریقی کمیت \عددی{\dif m} کے ٹکڑوں میں تقسیم کیا جاتا ہے، جو ذرے پر تفریقی قوت \عددی{\dif \vec{F}} پیدا کرتی ہیں؛   تمام  تفریقی قوتوں کا تکمل ان کا مجموعہ دیگا۔
\begin{align*}
\vec{F}_1=\int \dif\vec{F}
\end{align*}
\end{itemize}

%-----------------------
%gravitation and the principle of superpoition p357
\حصہء{تجاذب اور اصول انطباق}
\اصطلاح{اصول انطباق }\فرہنگ{اصول انطباق}\حاشیہب{principle of superposition}\فرہنگ{principle of superposition}استعمال کر  کے   ، ذروں کے  گروہ میں  ، ایک ذرے پر باقی ذروں   کی صافی (یا ماحصل) تجاذبی قوت معلوم کی جاتی ہے۔ یہ ایک عمومی اصول ہے، جو کہتا ہے انفرادی اثرات کا مجموعہ صافی اثر دیگا۔یہاں،اس  اصول   کے تحت منتخب ذرے پر باقی تمام ذروں کا (ایک ایک کر کے)  انفرادی قوت تجاذب حاصل کر کے ان کا  سمتی مجموعہ  لے کر صافی تجاذبی قوت حاصل کیا جائے گا۔  سمتیات  کا مجموعہ لینا ہم جانتے ہیں۔ قوت کے سمتیات کا مجموعہ بھی  اسی طرح حاصل کیا جائے گا۔

مذکورہ بالا آخری  دو جملوں میں پیش دو نقطوں  پر دوبارہ غور کرتے ہیں۔  (1)  قوت درحقیقت سمتیہ ہیں جن کے رخ مختلف ہو سکتے ہیں لہٰذا ان کا\ترچھا{ سمتی مجموعہ } لینا ضروری ہے، جو قوتوں کے رخ کا حساب بھی رکھے گا۔ (جب دو آدمی آپ کو مخالف رخ کھینچیں، ان کی صافی قوت یقیناً اس سے مختلف ہو گی جب دونوں آپ کو  ایک رخ  کھینچیں۔) (2) ہم انفرادی قوتوں کا مجموعہ لیتے ہیں۔ کتنا مشکل ہوتا اگر صافی قوت  کسی جزو ضربی پر منحصر ہوتی جس کی قیمت ہر قوت کے لئے  صورت حال کے مطابق    مختلف ہوتی، یا اگر ایک قوت کی موجودگی دوسری قوت پر اثر انداز ہوتی۔ ہماری خوش قسمتی  ہے کہ دنیا اتنی پیچیدہ نہیں؛ ہم قوتوں کا سادہ سمتی مجموعہ لیتے ہیں۔

جہاں \عددی{n} باہم عمل ذرے ہوں، وہاں ذرہ \عددی{1} پر باقی ذروں  کی تجاذبی قوتوں کا  اصول انطباق ذیل لکھا جا سکتا ہے۔
%eq 13.4 p357
\begin{align}
\vec{F}_{1,\text{\RL{صافی}}}=\vec{F}_{12}+\vec{F}_{13}+\vec{F}_{14}+\vec{F}_{15}+\cdots+\vec{F}_{1n}
\end{align}
یہاں ذرہ \عددی{1} پر صافی قوت \عددی{\vec{F}_{1,\text{\RL{صافی}}}} ہے اور ، مثال کے طور پر، ذرہ \عددی{1} پر ذرہ  \عددی{3} کی قوت \عددی{\vec{F}_{13}} ہے۔اس مساوات کو مختصراً  (ذیل) سمتی مجموعہ لکھا جا سکتا ہے۔
%eq 13.5 p357
\begin{align}\label{مساوات_تجاذب_سمتی_مجموعہ}
\vec{F}_{1,\text{\RL{صافی}}} =\sum_{i=2}^{n} \vec{F}_{1i}
\end{align}

\موٹا{حقیقی اجسام۔}\quad
ذرے پر حقیقی (وسیع) جسم کی تجاذبی قوت کیا ہو گی؟ ہم جسم کو اتنے چھوٹے چھوٹے ٹکڑوں میں تقسیم کرتے ہیں کہ ہر ٹکڑے کو ذرہ تصور کرنا ممکن ہو۔اس کے بعد  مساوات \حوالہ{مساوات_تجاذب_سمتی_مجموعہ} استعمال کر کے ذرے پر   تمام ٹکڑوں کی قوتوں کا  سمتی مجموعہ لیتے ہیں۔ تحدیدی صورت میں ہم وسیع جسم کو تفریقی ٹکڑوں  میں  تقسیم کرتے ہیں جن کی انفرادی کمیت \عددی{\dif m}  اور  انفرادی تفریقی  قوت \عددی{\dif\vec{F}} ہو گی، اور مساوات \حوالہ{مساوات_تجاذب_سمتی_مجموعہ} ذیل تکمل کا روپ اختیار کرتی ہے:
%eq 13.6 p358
\begin{align}\label{مساوات_تجاذب_سمتی_تکمل}
\vec{F}_1=\int \dif \vec{F}
\end{align}
جہاں تکمل پورے جسم پر لیا جاتا ہے اور ہم زیر نوشت   \قول{صافی}  لکھنا  بند کرتے  ہیں۔ اگر وسیع جسم ایک یکساں کرہ یا کروی خول ہو،  مساوات \حوالہ{مساوات_تجاذب_سمتی_تکمل} کے سمتی تکمل سے چھٹکارہ  حاصل کیا جا سکتا ہے؛ ہم جسم کی کمیت اس کے مرکز کمیت پر تصور کر کے مساوات \حوالہ{مساوات_تجاذب_نیوٹن_کا_قانون_تجاذب_الف} استعمال کرتے ہیں۔
%-------------------------
%sample problem 13.01 net gravitational force, 2D, three particles
\ابتدا{نمونی سوال}\موٹا{صافی تجاذبی قوت، دو ابعادی، تین ذروی}\\
شکل \حوالہء{13.4a} میں    ذرہ \عددی{1}، ذرہ \عددی{2}، اور ذرہ \عددی{3} پیش ہیں جن کی کمیت  \عددی{m_1=\SI{6.0}{\kilo\gram}}  اور 
\عددی{m_2=m_3=\SI{4.0}{\kilo\gram}} ہے، اور جہاں \عددی{a=\SI{2.0}{\centi\meter}} ہے۔ ذرہ \عددی{1} پر باقی ذروں کی صافی 
قوت \عددی{\vec{F}_{1,\text{\RL{صافی}}}} کیا ہے؟

\حصہء{کلیدی تصورات}
(1)  چونکہ  ہمیں ذروں سے واسطہ ہے، ذرہ \عددی{1} پر باقی ذروں کی  تجاذبی قوت کی قدر  مساوات \حوالہ{مساوات_تجاذب_نیوٹن_کا_قانون_تجاذب_الف}    \عددی{(F=Gm_1m_2/r^2)} سے حاصل ہو گی۔ (2)  ذرہ \عددی{1} پر تجاذبی قوت  اس ذرے  کے رخ ہو گی جو قوت پیدا کرتی ہے۔ (3)  یہ قوتیں ایک محور پر نہیں پائی جاتیں لہٰذا  ان  کی قدروں  کو جمع یا منفی نہیں کیا جا سکتا۔ انہیں سمتیات کی طرح جمع کرنا ہو گا۔

\موٹا{حساب:}\quad
مساوات \حوالہ{مساوات_تجاذب_نیوٹن_کا_قانون_تجاذب_الف} کے تحت ذرہ \عددی{1} پر ذرہ \عددی{2} کی قوت \عددی{\vec{F}_{12}} کی قدر ذیل ہو گی۔
%eq 13.7
\begin{gather}
\begin{aligned}
F_{12}&=\frac{Gm_1m_2}{a^2}\\
&=\frac{(\SI{6.67e-11}{\meter\cubed\per\kilo\gram\per\second\squared})(\SI{6.0}{\kilo\gram})(\SI{4.0}{\kilo\gram})}{(\SI{0.020}{\meter})^2}\\
&=\SI{4.00e-6}{\newton}
\end{aligned}
\end{gather}
اسی طرح ذرہ \عددی{1} پر ذرہ \عددی{3} کی قوت کی قدر ذیل ہو گی۔
%eq 13.8
\begin{gather}
\begin{aligned}
F_{13}&=\frac{Gm_1m_3}{(2a)^2}\\
&=\frac{(\SI{6.67e-11}{\meter\cubed\per\kilo\gram\per\second\squared})(\SI{6.0}{\kilo\gram})(\SI{4.0}{\kilo\gram})}{(\SI{0.040}{\meter})^2}\\
&=\SI{1.00e-6}{\newton}
\end{aligned}
\end{gather}
قوت \عددی{\vec{F}_{12}} مثبت محور \عددی{y}   رخ (شکل \حوالہء{13.4b})  ہے جس کا صرف \عددی{y} جزو \عددی{F_{12}} ہو گا۔ اسی طرح \عددی{\vec{F}_{13}}  منفی محور \عددی{x} رخ ہے جس کا صرف \عددی{x} جزو \عددی{-F_{13}} ہو گا (شکل \حوالہء{13.4c})۔ (یاد  رہے، ہم قوتوں کی دم اس ذرے پر رکھتے ہیں جو قوت محسوس کرتا ہو۔)

ذرہ \عددی{1} پر صافی قوت \عددی{\vec{F}_{1,\text{\RL{صافی}}}}  تلاش کرنے کے لئے  ہمیں دونوں قوت کو سمتیات کی طرح جمع کرنا ہو گا (شکل \حوالہء{13.4d} اور شکل \حوالہء{13.4e})۔ یہاں \عددی{-F_{13}} اور \عددی{F_{12}}  قوت \عددی{\vec{F}_{1,\text{\RL{صافی}}}} کے \عددی{x} اور \عددی{y}  جزو ہیں۔ یوں مساوات \حوالہء{3.6} استعمال کر کے قدر تلاش کر نے کے بعد \عددی{\vec{F}_{1,\text{\RL{صافی}}}}  کا رخ معلوم کرتے ہیں۔ قدر ذیل ہے۔
\begin{align*}
F_{1,\text{\RL{صافی}}}&=\sqrt{(F_{12})^+(-F_{13})^2}\\
&=\sqrt{(\SI{4.00e-6}{\newton})^2+(\SI{-1.00e-6}{\newton})^2}\\
&=\SI{4.1e-6}{\newton}\quad\quad\text{\RL{(جواب)}}
\end{align*}
مثبت محور \عددی{x} کے ساتھ    \عددی{\vec{F}_{1,\text{\RL{صافی}}}} کا زاویہ مساوات \حوالہء{3.6} کے تحت ذیل ہو گا۔
\begin{align*}
\theta=\tan^{-1}\frac{F_{12}}{-F_{13}}=\tan^{-1}\frac{\SI{4.00e-6}{\newton}}{\SI{-1.00e-6}{\newton}}=\SI{-76}{\degree}
\end{align*}
کیا یہ زاویہ درست معلوم ہوتا ہے (شکل \حوالہء{13.4f})؟ بالکل نہیں، چونکہ      \عددی{\vec{F}_{1,\text{\RL{صافی}}}}  کا زاویہ \عددی{\vec{F}_{12}} اور 
\عددی{\vec{F}_{13}} کے درمیان ہو گا۔ باب \حوالہء{3} سے یاد کریں، حساب کار ( کلکولیٹر ) \عددی{\tan^{-1}} کی  دو ممکنہ جوابات میں سے ایک دیگا۔ دوسرا جواب  جاننے کے لئے  زاویے  کے ساتھ  \عددی{\SI{180}{\degree}} جمع کرنا ہو گا:
\begin{align*}
\SI{-76}{\degree}+\SI{180}{\degree}=\SI{104}{\degree}\quad\quad\text{\RL{(جواب)}}
\end{align*}
جو      \عددی{\vec{F}_{1,\text{\RL{صافی}}}} کا درست زاویہ نظر آتا ہے (شکل \حوالہء{13.4g})۔ 
\انتہا{نمونی سوال}
%----------------------------

%checkpoint 2 p358
\ابتدا{آزمائش}
تین ذرے ، جن کی کمیت   برابر  ہے،  کو چار مختلف صورتوں میں رکھا گیا ہے (شکل \حوالہء{؟؟} )۔ (ا)  جس ذرے  کو \عددی{m} سے ظاہر کیا گیا ہے، اس پر صافی   تجاذبی قوت کی قدر کے لحاظ سے، اعظم قیمت اول رکھ کر، چار صورتوں کی درجہ بندی کریں۔ (ب) کیا  صورت \عددی{2} میں  صافی قوت کا رخ  اس لکیر  کے زیادہ قریب ہے جس کی لمبائی \عددی{d} ہے  یا  جس کی لمبائی \عددی{D} ہے۔؟
\انتہا{آزمائش}
%----------------

%13.3 gravitation near earth's surface p359
\حصہء{سطح زمین کے قریب تجاذب}
