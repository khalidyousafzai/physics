%gravitation p354
\باب{تجاذب}
%13.1 newton's law of gravitation p354
\حصہ{نیوٹن کا قانون تجاذب}
\موٹا{مقاصد}\\
اس حصہ کو پڑھنے کے بعد آپ ذیل کے قابل ہوں گے۔
\begin{enumerate}[1.]
\item
دو ذروں  کی کمیت اور ان کے بیچ فاصلے  کا ذروں کی باہمی تجاذبی  قوت  کے ساتھ تعلق نیوٹن کے   قانون تجاذب سے جان پائیں گے۔
\item
جان پائیں گے کہ مادے کا   یکساں کروی خول  ذرے کو   ، جو خول سے باہر ہو، بالکل اس طرح  کھینچتا ہے  جیسے  خول کی کمیت  خول کے مرکز پر واقع ہو۔
\item
ذرے پر دوسرے ذرے یا مادے کی  یکساں کروی تقسیم کی قوت تجاذب آزاد جسمی خاکہ   سے  ظاہر کر پائیں گے۔
\end{enumerate}

\موٹا{کلیدی تصورات}\\
\begin{itemize}
\item
کائنات میں ہر ذرہ دوسرے ذرے کو ذیل قدر کی  تجاذبی قوت سے اپنی طرف کھینچتا ہے:
\begin{align*}
F&=G\frac{m_1m_2}{r^2}\quad\quad\text{\RL{(نیوٹن کا قانون تجاذب)}}
\end{align*}
جہاں \عددی{m_1} اور \عددی{m_2} ذروں کی کمیتیں، \عددی{r} ان کے بیچ فاصلہ، اور \عددی{G=\SI{6.67e-11}{\newton\meter\squared\per\kilo\gram\squared}} تجاذبی مستقل  ہے۔
\item
وسیع اجسام کے بیچ تجاذبی قوت معلوم کرنے کی خاطر، جسم کے اندر تمام انفرادی  ذروں  پر انفرادی قوت کا مجموعہ (تکمل)  لینا ہو گا۔ تاہم، اگر  ایک جسم یکساں کروی خول  یا کروی تشاکل ٹھوس جسم ہو،بیرونی جسم پر اس کی  صافی تجاذبی  قوت معلوم کرتے وقت  خول یا ٹھوس جسم کی کمیت جسم کے مرکز پر تصور کی جا سکتی ہے۔
\end{itemize}

\جزوحصہء{طبیعیات کیا ہے؟}
طبیعیات کا ایک مقصد ، قوت تجاذب کا    سمجھنا ہے۔ قوت تجاذب  ہمیں زمین پر رکھتی ہے، چاند کو  زمین کے گرد  مدار، اور زمین کو سورج کے گرد مدار میں رکھتی ہے۔اس کا اثر  ہماری \اصطلاح{ دودھیا کہکشاں }\فرہنگ{کہکشاں!دودھیا}\حاشیہب{milky way galaxy}\فرہنگ{galaxy!milky way} کے ہر کونے تک پہنچ کر، اربوں ستاروں، لاتعداد جوہر  اور ستاروں کے بیچ  دھول کے ذروں کو   کہکشاں میں  جکڑ کر رکھتا ہے۔ ہم  دودھیا  کہکشاں، جو ستاروں کا قرص نما جھرمٹ   ہے، کے کنارے کے قریب، کہکشاں کے مرکز سے \عددی{\num{2.6e4}}  نوری سال \عددی{(\SI{2.5e20}{\meter})}   فاصلے پر مرکز کے گرد آہستہ آہستہ طواف کرتے  ہوئے،  بستے ہیں۔

تجاذبی قوت  بین کہکشانی  فاصلے طے کر کے کہکشاں کے مقامی گروہ کو، جس میں دودھیا کہکشاں کے علاوہ \اصطلاح{ اندرومدا }\فرہنگ{کہکشاں!اندرومدا}\حاشیہب{Andromeda}\فرہنگ{galaxy!Andromeda}کہکشاں (شکل \حوالہء{13.1}) جو زمین  سے \عددی{\num{2.3e6}} نوری سال فاصلے پر ہے،   اور کئی \اصطلاح{  بالشتیا }\فرہنگ{کہکشاں!بالشتیا}\حاشیہب{dwarf}\فرہنگ{galaxy!dwarf} کہکشاں، جیسے \اصطلاح{  سحاب کبیر }\فرہنگ{سحاب کبیر}\حاشیہب{Large Magellanic Cloud}\فرہنگ{cloud!Large Magellanic}، شامل ہے۔ کہکشاں کا مقامی گروہ   از خود \اصطلاح{ مقامی  عظیم  خوشہ }\فرہنگ{مقامی عظیم خوشہ}\حاشیہب{Local Supercluster}\فرہنگ{Local Supercluster} کا حصہ ہے، جس کو تجاذبی قوت  انتہائی زیادہ کمیتی خطہ کی طرف، جو  \اصطلاح{عظیم  جالب }\فرہنگ{عظیم جالب}\حاشیہب{Great Attractor}\فرہنگ{Great Attractor} کہلاتا ہے،  کھنچ رہا ہے۔ یہ خطہ زمین سے  \عددی{\num{3.0e8}} نوری سال کے فاصلے پر، دودھیا کہکشاں کی دوسرے طرف ، واقع ہے۔ تجاذبی قوت اس سے بھی زیادہ دور رس ہے، چونکہ یہ پوری کائنات کو ، جس کا حجم بتدریج بڑھ  رہا ہے،  ایک ساتھ رکھتا ہے۔
%p355
\اصطلاح{ثقب اسود  }\فرہنگ{ثقب اسود}\حاشیہب{black hole}\فرہنگ{black hole}   ، جو کائنات میں انتہائی پراسرار اجسام میں سے ایک ہے، کا دارومدار بھی اسی قوت پر ہے۔ جب  سورج  سے بڑا  ستارہ   زندگی کے اختتام کو پہنچتا ہے،  اس کے ذروں کے  بیچ تجاذبی قوت   ستارے کو   اپنے آپ پر  منہدم   کر کے ثقب اسود پیدا کرتی ہے۔ منہدم ستارے کی سطح پر تجاذبی قوت اتنی زیادہ   ہوتی ہے  کہ سطح سے   کوئی ذرہ  فرار نہیں  ہو سکتا اور نا ہی  روشنی نکل سکتی ہے (اسی لئے اس کو \قول{ ثقب اسود }  یعنی \قول{ سیاہ  سوراخ } کہتے ہیں)۔ اگر کوئی ستارہ ثقب اسود کے زیادہ قریب پہنچے، ثقب اسود  کی توانا  تجاذبی قوت   ستارے کو   چیرپھاڑ  کر  ثقب (سوراخ) کے اندر کھینچ لیتی ہے۔ متعدد ستارے نوچنے   پر اس سے    \اصطلاح{بے پناہ کمیتی  ثقب اسود  }\فرہنگ{ثقب اسود!بے پناہ کمیتی}\حاشیہب{supermassive black hole}\فرہنگ{black hole!supermassive}بنتا ہے۔ ایسے  بھیانک  اور پراسرار اجسام سے کائنات  بھری نظر آتی ہے۔یقیناً  ہماری اپنی دودھیا کہکشاں  کے مرکز پر  ایک ثقب اسود پایا جاتا ہے، جو   القوس \عددی{A^*}  کہلاتا ہے، اور جس کی کمیت تقریباً  \عددی{\num{3.7e6}} شمسی  کمیت کے برابر   ہے۔ اس کی تجاذبی قوت اتنی توانا ہے کہ   قریبی ستارے مدار میں گھومتے  ہوئے صرف \عددی{15.2} سال میں القوس \عددی{A^*} کے گرد  چکر مکمل کرتے ہیں۔

اگرچہ  تجاذبی قوت  مکمل سمجھنے سے اب بھی ہم قاصر ہیں، اسے سمجھنے کا ابتدائی نقطہ  نیوٹن کا \ترچھا{ قانون تجاذب } ہے۔

%-------------------
%Newton's Law of Gravitation p355
\جزوحصہء{نیوٹن کا قانون تجاذب}
مختلف مساوات پر بات کرنے سے قبل ذرہ سوچتے ہیں۔ہم زمین   کے ساتھ مس رہتے ہیں؛ مس رکھنے کی قوت اتنی زیادہ نہیں کہ ہمیں گھسیٹ کر چلنا پڑے اور نا ہی اتنی کم ہے کہ آئے دن سر چھت سے ٹکرائے۔ ساتھ ہی یہ قوت ہمیں زمین پر رکھتی ہے، تاہم اتنی طاقتور نہیں کہ ہم ایک دوسرے کے ساتھ جڑ جائیں۔ یقیناً  قوت کشش کا دارومدار  جسم میں مادے کی مقدار پر ہے۔ زمین میں مادے کی مقدار بہت  زیادہ ہے، لہٰذا زمین  کی کشش بھی بہت زیادہ ہے، جبکہ  انسان کے جسم میں مادے کی مقدار بہت کم ہے، اور اسی لئے دو انسانی اجسام کے بیچ قوت کشش  اتنی کم ہے کہ اسے محسوس کرنا ممکن نہیں۔مزید، ایک  مادہ  دوسرے مادہ کو کھینچتا ہے؛ ان کے بیچ قوت دفع نہیں پائی جاتی۔

ماضی میں انسان جانتا تھا کہ زمین ہمیں نشیب وار   کھینچتی ہے (خاص کر اس وقت جب زمین پر  گر نے سے ناک کی ہڈی ٹوٹتی تھی)، تاہم ان کا خیال تھا کہ یہ قوت صرف زمین کی خاصیت ہے جس کا آسمان میں  فلکیاتی اجسام کی  حرکت سے کوئی واسطہ نہیں۔ تاہم \سن{1665} میں، \عددی{23} سالہ،  اسحاق نیوٹن  کو اس بات کی سمجھ آئی کہ چاند کو یہی قوت زمین کے گرد گھومنے پر  مجبور کرتی ہے۔ بالخصوص، انہوں نے دکھایا کہ کائنات میں ہر جسم، ہر دوسرے جسم کو کھینچتا ہے۔  اجسام کا ایک دوسرے کی  طرف  حرکت کرنے کے  رجحان  کو\اصطلاح{ تجاذب }\فرہنگ{تجاذب}\حاشیہب{gravitation}\فرہنگ{gravitation} کہتے ہیں، اور اجسام میں وہ مادہ  جو  قوت کشش  پیدا کرتی ہے  \قول{کمیت } کہلاتا ہے۔ اگر  نیوٹن کے سر پر سیب گرنے کا  واقعہ درست ہو،  تب  سیب کی کمیت اور زمین کی کمیت کے بیچ کشش سیب کے گرنے کا سبب تھی۔ زمین کی کمیت اتنی زیادہ ہے کہ یہ قوت ، جو تقریباً \عددی{\SI{0.80}{\newton}} ہو گی، محسوس کی جا سکتی ہے۔ اس کے برعکس، دو قریب کھڑے  اشخاص کے بیچ قوت ، جو 
تقریباً \عددی{\SI{1}{\micro\newton}} ہو گی، محسوس نہیں کی جا سکتی۔

دو وسیع اجسام، جیسے دو شخص، کے بیچ تجاذبی توانائی کا حساب کرنا   مشکل ہو گا۔  ہم دو ذروں  (جن کی جسامت صفر تصور کی جائے گی) کے بیچ نیوٹن کے  قانون تجاذب   کے اطلاق پر غور کرتے ہیں۔ فرض کریں ذروں کی کمیت \عددی{m_1} اور \عددی{m_2} اور ان کے بیچ  فاصلہ  \عددی{r} ہے۔ ایک ذرے  کی دوسرے  ذرے پر قوت کشش کی قدر  ذیل ہو گی:
%eq 13.1
\begin{align}\label{مساوات_تجاذب_نیوٹن_کا_قانون_تجاذب_الف}
F=G\frac{m_1m_2}{r^2}\quad\quad\text{\RL{(نیوٹن کا قانون تجاذب)}}
\end{align}
جہاں \عددی{G}  \اصطلاح{تجاذبی مستقل }\فرہنگ{تجاذبی مستقل}\حاشیہب{gravitation constant}\فرہنگ{gravitation constant}ہے، جس کی قیمت ذیل ہے۔
%eq 13.2
\begin{gather}
\begin{aligned}
G&=\SI{6.67e-11}{\newton\meter\squared\per\kilo\gram\squared}\\
&=\SI{6.67e-11}{\meter\cubed\per\kilo\gram\per\second\squared}
\end{aligned}
\end{gather}
شکل \حوالہء{13.2a} میں  ذرہ \عددی{1} (کمیت \عددی{m_1}) پر ذرہ \عددی{2} (کمیت \عددی{m_2}) کی تجاذبی قوت \عددی{\vec{F}}  پیش ہے۔ قوت کا رخ ، ذرہ \عددی{2} کی طرف ہے اور چونکہ  ذرہ \عددی{2}  کی طرف ذرہ \عددی{1}  کشش محسوس کرتا ہے  لہٰذا  یہ\ترچھا{ قوت کشش  } کہلاتی ہے۔ قوت کی قدر مساوات \حوالہ{مساوات_تجاذب_نیوٹن_کا_قانون_تجاذب_الف} سے حاصل ہو گی۔ ذرہ \عددی{1} سے رداسی باہر نکلتی   محور \عددی{r}   ، جو ذرہ \عددی{2} سے  بھی گزرتی ہے، کے مثبت رخ کے ہمراہ  قوت \عددی{\vec{F}} عمل کرتی ہے (شکل \حوالہء{13.2b})۔ اکائی سمتیہ \عددی{\rhat}  ( جو بے بُعدی سمتیہ  ہے اور جس کی قدر \عددی{1} ہے) استعمال کرتے ہوئے، جو ذرہ \عددی{1}  سے دوری کے رخ  محور \عددی{r} کے ہمراہ  واقع ہے  (شکل \حوالہء{13.2c})، \عددی{\vec{F}} بیان کیا جا سکتا ہے۔ یوں، مساوات \حوالہ{مساوات_تجاذب_نیوٹن_کا_قانون_تجاذب_الف} استعمال کرتے ہوئے  ذرہ \عددی{1}  پر قوت     ذیل ہو گی۔
%eq 13.3
\begin{align}
\vec{F}=G\frac{m_1m_2}{r^2}\,\rhat
\end{align}

ذرہ \عددی{2} پر ذرہ \عددی{1} کی قوت کی قدر  وہی ہے  جو ذرہ \عددی{1} پر ذرہ \عددی{2} کی قدر  ہے، تاہم اس کا رخ  مخالف ہو گا۔ دونوں قوت مل کر  قانون سوم  جوڑی    قوت دیتے ہیں، اور ہم  دو ذروں کے بیچ قوت تجاذب  کی بات  کر سکتے ہیں جس کی قدر مساوات \حوالہ{مساوات_تجاذب_نیوٹن_کا_قانون_تجاذب_الف} دیتی ہے۔ دو ذروں کے بیچ قوت تجاذب  پر دیگر اجسام  کا اثر نہیں پایا جاتا، اگرچہ یہ اجسام ان ذروں کے  درمیان ہی  کیوں نہ ہوں۔ دوسرے لفظوں میں، کوئی  جسم ایک ذرے کو دوسرے ذرے کی تجاذبی قوت سے   بچا نہیں سکتا۔

دی گئی کمیت کے ذروں کی ، جو کسی مخصوص فاصلے پر ہوں،تجاذبی قوت کا زور  تجاذبی مستقل \عددی{G} پر منحصر ہے۔ اگر جادو  سے \عددی{G} کی قیمت دس گنّا کی جائے، زمین کی کشش آپ کو زمین پر گرا دیگی، اور \عددی{G} کی قیمت دس گنّا کم کر دی جائے، آپ چھلانگ  لگا کر  عمارت  پار کر پائیں گے۔

\موٹا{دیگر اجسام۔}\quad
اگرچہ، نیوٹن کا قانون تجاذب  ذروں کے لئے ہے،  جب تک اجسام کی جسامت ، اجسام کے بیچ فاصلے کے لحاظ سے بہت کم ہو،  قانون تجاذب  کا اطلاق وسیع اجسام  پر بھی  ممکن ہے۔ زمین اور چاند ایک دوسرے سے  اتنی دوری پر ہیں کہ انہیں تخمیناً ذرے تصور کیا جا سکتا ہے؛ لیکن سیب اور زمین  کا کیا ہو گا؟ سیب کے نقطہ نظر سے زمین اتنی وسیع ہے کہ اسے  ذرہ تصور  کرنا درست نہ ہو گا۔

نیوٹن نے (ذیل)    \ترچھا{مسئلہ خول }کے ذریعہ زمین اور سیب کا مسئلہ حل کیا۔

\ابتدا{قاعدہء}
مادے کا یکساں کروی خول ، کرہ سے باہر واقع  ذرے کو یوں کھینچتا ہے گویا  خول  کی تمام کمیت خول کے مرکز پر ہو۔
\انتہا{قاعدہء}
%-----------------------
 زمین کو خول کے اوپر خول چڑھا  جسم تصور کیا جا سکتا ہے۔ ہر خول بیرونی ذرے کو یوں کھینچتا ہے گویا خول کے مرکز پر خول کی تمام کمیت واقع ہو۔ یوں  سطح زمین   سے   اوپر ذرے کو تمام خول یوں کھینچتے ہیں گویا زمین کی پوری کمیت زمین کے مرکز پر واقع ہو۔ یوں، سیب کے نقطہ نظر سے زمین اس  ذرے کی مانند ہے جو زمین کے مرکز پر واقع ہے اور جس کی کمیت زمین کی کمیت کے برابر ہے۔

\موٹا{قانون سوم جوڑی قوت۔}\quad
فرض کریں، جیسا شکل \حوالہء{13.3} میں دکھایا گیا ہے، سیب کو زمین \عددی{\SI{0.80}{\newton}} قدر کی قوت سے نیچے کھینچتی ہے۔ اب  زمین کو سیب  \عددی{\SI{0.80}{\newton}} قدر کی  قوت سے اوپر کھینچے گا اور یہ قوت زمین کے مرکز پر عمل پیرا ہو گی۔ باب \حوالہء{5} کی زبان میں  یہ قوت مل کر نیوٹن کے قانون سوم  میں جوڑی قوت دیتی ہیں۔ اگرچہ، ان کی قدر ایک ہے، جب سیب رہا کیا جائے وہ قوت سیب پر مختلف اسراع عائد کرتی ہیں۔ سیب کے اسراع کی قدر \عددی{\SI{9.8}{\meter\per\second}} ہو گی، جو سطح زمین کے قریب گرتے جسم کا  اسراع ہے۔ سیب و زمین نظام  کے مرکز سے جڑی  حوالہ چوکھٹ  کے لحاظ سے زمین کے اسراع کی قدر صرف \عددی{\SI{1e-25}{\meter\per\second\squared}} ہو گی۔ یقیناً،  سیب کے گرنے کے دوران زمین ساکن نظر آتی ہے۔

%---------------------
%checkpoint 1 p356
\ابتدا{آزمائش}
ایک ذرہ باری باری  درج ذیل اجسام کے باہر رکھا جاتا ہے، جن کی انفرادی کمیت \عددی{m} ہے۔ (1) ایک بڑا یکساں ٹھوس کرہ،  (2) ایک بڑا یکساں کروی خول، (3) ایک چھوٹا یکساں ٹھوس کرہ، اور (4)  ایک چھوٹا  یکساں  خول۔ ہر مرتبہ ذرے اور کرہ کے مرکز کا فاصلہ  \عددی{d} رکھا جاتا ہے۔ ذرے پر قوت کی قدر کے لحاظ سے، اعظم قیمت اول رکھ کر، ان اجسام کی درجہ بندی کریں۔
\انتہا{آزمائش}
%--------------------

%13.2 gravitation and the principle of superposition p357
\حصہ{تجاذب اور اصول   انطباق}
\موٹا{مقاصد}\\
اس حصہ کو پڑھنے کے بعد آپ ذیل کے قابل ہوں گے۔
\begin{enumerate}[1.]
\item
جہاں ذرے پر ایک سے زیادہ    تجاذبی قوت عمل پیرا ہوں، قوتوں کو ظاہر کرنے والا   آزاد جسمی خاکہ بنا پائیں گے ، جس میں قوت سمتیات کی دم عین ذرے پر  ہو گی۔
\item
جہاں ایک سے زیادہ تجاذبی قوت ذرے پر عمل پیرا ہوں،  انفرادی قوت کو سمتیہ تصور کر کے تمام کا  مجموعہ لے کر صافی قوت تلاش کر پائیں گے۔
\end{enumerate}

\موٹا{کلیدی تصورات}\\
\begin{itemize}
\item
تجاذبی قوت   اصول   انطباق پر پورا اترتی ہے؛ یعنی، جہاں \عددی{n} ذرے باہم عمل کرتے ہوں، ایک ذرے پر ، جس کا  عرف \عددی{1} ہے، صافی 
قوت \عددی{\vec{F}_{1,\text{\RL{صافی}}}}  باقی تمام ذروں کو باری باری لے کر ،  انفرادی قوتوں کا  مجموعہ  ہو گا:
\begin{align*}
\vec{F}_{1,\text{\RL{صافی}}}=\sum_{i=2}^{n} \vec{F}_{1i}
\end{align*}
جہاں ذرہ \عددی{1} پر ذرہ \عددی{2}، \عددی{3}،\نقطے،  تا \عددی{n} کی قوتوں  کا سمتی مجموعہ لیا جائے گا۔
\item
وسیع جسم کی ذرے پر قوت \عددی{\vec{F}_1} معلوم کرنے کے لئے، وسیع جسم کو  تفریقی کمیت \عددی{\dif m} کے ٹکڑوں میں تقسیم کیا جاتا ہے، جو ذرے پر تفریقی قوت \عددی{\dif \vec{F}} پیدا کرتی ہیں؛   تمام  تفریقی قوتوں کا تکمل ان کا مجموعہ دیگا۔
\begin{align*}
\vec{F}_1=\int \dif\vec{F}
\end{align*}
\end{itemize}

%-----------------------
%gravitation and the principle of superpoition p357
\جزوحصہء{تجاذب اور اصول انطباق}
\اصطلاح{اصول انطباق }\فرہنگ{اصول انطباق}\حاشیہب{principle of superposition}\فرہنگ{principle of superposition}استعمال کر  کے   ، ذروں کے  گروہ میں  ، ایک ذرے پر باقی ذروں   کی صافی (یا ماحصل) تجاذبی قوت معلوم کی جاتی ہے۔ یہ ایک عمومی اصول ہے، جو کہتا ہے انفرادی اثرات کا مجموعہ صافی اثر دیگا۔یہاں،اس  اصول   کے تحت منتخب ذرے پر باقی تمام ذروں کا (ایک ایک کر کے)  انفرادی قوت تجاذب حاصل کر کے ان کا  سمتی مجموعہ  لے کر صافی تجاذبی قوت حاصل کیا جائے گا۔  سمتیات  کا مجموعہ لینا ہم جانتے ہیں۔ قوت کے سمتیات کا مجموعہ بھی  اسی طرح حاصل کیا جائے گا۔

مذکورہ بالا آخری  دو جملوں میں پیش دو نقطوں  پر دوبارہ غور کرتے ہیں۔  (1)  قوت درحقیقت سمتیہ ہیں جن کے رخ مختلف ہو سکتے ہیں لہٰذا ان کا\ترچھا{ سمتی مجموعہ } لینا ضروری ہے، جو قوتوں کے رخ کا حساب بھی رکھے گا۔ (جب دو آدمی آپ کو مخالف رخ کھینچیں، ان کی صافی قوت یقیناً اس سے مختلف ہو گی جب دونوں آپ کو  ایک رخ  کھینچیں۔) (2) ہم انفرادی قوتوں کا مجموعہ لیتے ہیں۔ کتنا مشکل ہوتا اگر صافی قوت  کسی جزو ضربی پر منحصر ہوتی جس کی قیمت ہر قوت کے لئے  صورت حال کے مطابق    مختلف ہوتی، یا اگر ایک قوت کی موجودگی دوسری قوت پر اثر انداز ہوتی۔ ہماری خوش قسمتی  ہے کہ دنیا اتنی پیچیدہ نہیں؛ ہم قوتوں کا سادہ سمتی مجموعہ لیتے ہیں۔

جہاں \عددی{n} باہم عمل ذرے ہوں، وہاں ذرہ \عددی{1} پر باقی ذروں  کی تجاذبی قوتوں کا  اصول انطباق ذیل لکھا جا سکتا ہے۔
%eq 13.4 p357
\begin{align}
\vec{F}_{1,\text{\RL{صافی}}}=\vec{F}_{12}+\vec{F}_{13}+\vec{F}_{14}+\vec{F}_{15}+\cdots+\vec{F}_{1n}
\end{align}
یہاں ذرہ \عددی{1} پر صافی قوت \عددی{\vec{F}_{1,\text{\RL{صافی}}}} ہے اور ، مثال کے طور پر، ذرہ \عددی{1} پر ذرہ  \عددی{3} کی قوت \عددی{\vec{F}_{13}} ہے۔اس مساوات کو مختصراً  (ذیل) سمتی مجموعہ لکھا جا سکتا ہے۔
%eq 13.5 p357
\begin{align}\label{مساوات_تجاذب_سمتی_مجموعہ}
\vec{F}_{1,\text{\RL{صافی}}} =\sum_{i=2}^{n} \vec{F}_{1i}
\end{align}

\موٹا{حقیقی اجسام۔}\quad
ذرے پر حقیقی (وسیع) جسم کی تجاذبی قوت کیا ہو گی؟ ہم جسم کو اتنے چھوٹے چھوٹے ٹکڑوں میں تقسیم کرتے ہیں کہ ہر ٹکڑے کو ذرہ تصور کرنا ممکن ہو۔اس کے بعد  مساوات \حوالہ{مساوات_تجاذب_سمتی_مجموعہ} استعمال کر کے ذرے پر   تمام ٹکڑوں کی قوتوں کا  سمتی مجموعہ لیتے ہیں۔ تحدیدی صورت میں ہم وسیع جسم کو تفریقی ٹکڑوں  میں  تقسیم کرتے ہیں جن کی انفرادی کمیت \عددی{\dif m}  اور  انفرادی تفریقی  قوت \عددی{\dif\vec{F}} ہو گی، اور مساوات \حوالہ{مساوات_تجاذب_سمتی_مجموعہ} ذیل تکمل کا روپ اختیار کرتی ہے:
%eq 13.6 p358
\begin{align}\label{مساوات_تجاذب_سمتی_تکمل}
\vec{F}_1=\int \dif \vec{F}
\end{align}
جہاں تکمل پورے جسم پر لیا جاتا ہے اور ہم زیر نوشت   \قول{صافی}  لکھنا  بند کرتے  ہیں۔ اگر وسیع جسم ایک یکساں کرہ یا کروی خول ہو،  مساوات \حوالہ{مساوات_تجاذب_سمتی_تکمل} کے سمتی تکمل سے چھٹکارہ  حاصل کیا جا سکتا ہے؛ ہم جسم کی کمیت اس کے مرکز کمیت پر تصور کر کے مساوات \حوالہ{مساوات_تجاذب_نیوٹن_کا_قانون_تجاذب_الف} استعمال کرتے ہیں۔
%-------------------------
%sample problem 13.01 net gravitational force, 2D, three particles
\ابتدا{نمونی سوال}\موٹا{صافی تجاذبی قوت، دو ابعادی، تین ذروی}\\
شکل \حوالہء{13.4a} میں    ذرہ \عددی{1}، ذرہ \عددی{2}، اور ذرہ \عددی{3} پیش ہیں جن کی کمیت  \عددی{m_1=\SI{6.0}{\kilo\gram}}  اور 
\عددی{m_2=m_3=\SI{4.0}{\kilo\gram}} ہے، اور جہاں \عددی{a=\SI{2.0}{\centi\meter}} ہے۔ ذرہ \عددی{1} پر باقی ذروں کی صافی 
قوت \عددی{\vec{F}_{1,\text{\RL{صافی}}}} کیا ہے؟

\موٹا{کلیدی تصورات}\\
(1)  چونکہ  ہمیں ذروں سے واسطہ ہے، ذرہ \عددی{1} پر باقی ذروں کی  تجاذبی قوت کی قدر  مساوات \حوالہ{مساوات_تجاذب_نیوٹن_کا_قانون_تجاذب_الف}    \عددی{(F=Gm_1m_2/r^2)} سے حاصل ہو گی۔ (2)  ذرہ \عددی{1} پر تجاذبی قوت  اس ذرے  کے رخ ہو گی جو قوت پیدا کرتی ہے۔ (3)  یہ قوتیں ایک محور پر نہیں پائی جاتیں لہٰذا  ان  کی قدروں  کو جمع یا منفی نہیں کیا جا سکتا۔ انہیں سمتیات کی طرح جمع کرنا ہو گا۔

\موٹا{حساب:}\quad
مساوات \حوالہ{مساوات_تجاذب_نیوٹن_کا_قانون_تجاذب_الف} کے تحت ذرہ \عددی{1} پر ذرہ \عددی{2} کی قوت \عددی{\vec{F}_{12}} کی قدر ذیل ہو گی۔
%eq 13.7
\begin{gather}
\begin{aligned}
F_{12}&=\frac{Gm_1m_2}{a^2}\\
&=\frac{(\SI{6.67e-11}{\meter\cubed\per\kilo\gram\per\second\squared})(\SI{6.0}{\kilo\gram})(\SI{4.0}{\kilo\gram})}{(\SI{0.020}{\meter})^2}\\
&=\SI{4.00e-6}{\newton}
\end{aligned}
\end{gather}
اسی طرح ذرہ \عددی{1} پر ذرہ \عددی{3} کی قوت کی قدر ذیل ہو گی۔
%eq 13.8
\begin{gather}
\begin{aligned}
F_{13}&=\frac{Gm_1m_3}{(2a)^2}\\
&=\frac{(\SI{6.67e-11}{\meter\cubed\per\kilo\gram\per\second\squared})(\SI{6.0}{\kilo\gram})(\SI{4.0}{\kilo\gram})}{(\SI{0.040}{\meter})^2}\\
&=\SI{1.00e-6}{\newton}
\end{aligned}
\end{gather}
قوت \عددی{\vec{F}_{12}} مثبت محور \عددی{y}   رخ (شکل \حوالہء{13.4b})  ہے جس کا صرف \عددی{y} جزو \عددی{F_{12}} ہو گا۔ اسی طرح \عددی{\vec{F}_{13}}  منفی محور \عددی{x} رخ ہے جس کا صرف \عددی{x} جزو \عددی{-F_{13}} ہو گا (شکل \حوالہء{13.4c})۔ (یاد  رہے، ہم قوتوں کی دم اس ذرے پر رکھتے ہیں جو قوت محسوس کرتا ہو۔)

ذرہ \عددی{1} پر صافی قوت \عددی{\vec{F}_{1,\text{\RL{صافی}}}}  تلاش کرنے کے لئے  ہمیں دونوں قوت کو سمتیات کی طرح جمع کرنا ہو گا (شکل \حوالہء{13.4d} اور شکل \حوالہء{13.4e})۔ یہاں \عددی{-F_{13}} اور \عددی{F_{12}}  قوت \عددی{\vec{F}_{1,\text{\RL{صافی}}}} کے \عددی{x} اور \عددی{y}  جزو ہیں۔ یوں مساوات \حوالہء{3.6} استعمال کر کے قدر تلاش کر نے کے بعد \عددی{\vec{F}_{1,\text{\RL{صافی}}}}  کا رخ معلوم کرتے ہیں۔ قدر ذیل ہے۔
\begin{align*}
F_{1,\text{\RL{صافی}}}&=\sqrt{(F_{12})^+(-F_{13})^2}\\
&=\sqrt{(\SI{4.00e-6}{\newton})^2+(\SI{-1.00e-6}{\newton})^2}\\
&=\SI{4.1e-6}{\newton}\quad\quad\text{\RL{(جواب)}}
\end{align*}
مثبت محور \عددی{x} کے ساتھ    \عددی{\vec{F}_{1,\text{\RL{صافی}}}} کا زاویہ مساوات \حوالہء{3.6} کے تحت ذیل ہو گا۔
\begin{align*}
\theta=\tan^{-1}\frac{F_{12}}{-F_{13}}=\tan^{-1}\frac{\SI{4.00e-6}{\newton}}{\SI{-1.00e-6}{\newton}}=\SI{-76}{\degree}
\end{align*}
کیا یہ زاویہ درست معلوم ہوتا ہے (شکل \حوالہء{13.4f})؟ بالکل نہیں، چونکہ      \عددی{\vec{F}_{1,\text{\RL{صافی}}}}  کا زاویہ \عددی{\vec{F}_{12}} اور 
\عددی{\vec{F}_{13}} کے درمیان ہو گا۔ باب \حوالہء{3} سے یاد کریں،   \اصطلاح{محاسب }\فرہنگ{محاسب}\حاشیہب{calculator}\فرہنگ{calculator} ( کلکولیٹر ) \عددی{\tan^{-1}} کی  دو ممکنہ جوابات میں سے ایک دیگا۔ دوسرا جواب  جاننے کے لئے  زاویے  کے ساتھ  \عددی{\SI{180}{\degree}} جمع کرنا ہو گا:
\begin{align*}
\SI{-76}{\degree}+\SI{180}{\degree}=\SI{104}{\degree}\quad\quad\text{\RL{(جواب)}}
\end{align*}
جو      \عددی{\vec{F}_{1,\text{\RL{صافی}}}} کا درست زاویہ نظر آتا ہے (شکل \حوالہء{13.4g})۔ 
\انتہا{نمونی سوال}
%----------------------------

%checkpoint 2 p358
\ابتدا{آزمائش}
تین ذرے ، جن کی کمیت   برابر  ہے،  کو چار مختلف صورتوں میں رکھا گیا ہے (شکل \حوالہء{؟؟} )۔ (ا)  جس ذرے  کو \عددی{m} سے ظاہر کیا گیا ہے، اس پر صافی   تجاذبی قوت کی قدر کے لحاظ سے، اعظم قیمت اول رکھ کر، چار صورتوں کی درجہ بندی کریں۔ (ب) کیا  صورت \عددی{2} میں  صافی قوت کا رخ  اس لکیر  کے زیادہ قریب ہے جس کی لمبائی \عددی{d} ہے  یا  جس کی لمبائی \عددی{D} ہے۔؟
\انتہا{آزمائش}
%----------------

%13.3 gravitation near earth's surface p359
\حصہ{سطح زمین کے قریب تجاذب}
\موٹا{مقاصد}\\
اس حصہ کو پڑھنے کے بعد آپ ذیل کے قابل ہوں گے۔
\begin{enumerate}[1.]
\item
 آزاد  گرنے کا اسراع اور تجاذبی اسراع میں  تمیز کر پائیں گے۔
\item
یکساں، کروی فلکیاتی جسم کے قریب لیکن اس  سے باہر تجاذبی اسراع کا حساب کر پائیں گے۔
\item
 پیمائشی وزن اور تجاذبی قوت کی قدر میں تمیز کر پائیں گے۔
\end{enumerate}

\موٹا{کلیدی تصورات}\\
\begin{itemize}
\item
ایک ذرے کا ( جس کی کمیت \عددی{m} ہے ) تجاذبی اسراع  \عددی{a_g}  صرف اور صرف ، ذرے پر عمل پیرا  ، تجاذبی قوت کی بنا ہے۔یکساں، کروی جسم، جس کی کمیت \عددی{M} ہے، کے مرکز  سے \عددی{r} فاصلے پر واقع ذرہ پر عمل پیرا  تجاذبی قوت کی قدر \عددی{F}  مساوات \حوالہ{مساوات_تجاذب_نیوٹن_کا_قانون_تجاذب_الف}   سے حاصل کی جا سکتی ہے۔ یوں، نیوٹن کے قانون دوم کے تحت ذیل ہو گا:
\begin{align*}
F=ma_g
\end{align*}
جس سے ذیل حاصل ہو گا۔
\begin{align*}
a_g=\frac{GM}{r^2}
\end{align*}
\item
زمین کی کمیتی تقسیم  یکساں نہیں،  زمین مکمل کروی نہیں، اور  زمین اپنے مرکز کے گرد  گھوم رہی ہے لہٰذا  سطح زمین کے قریب ذرے کا حقیقی آزاد  گرنے کا    اسراع  تجاذبی اسراع \عددی{a_g}  سے معمولی مختلف ہو گا، اور  ذرے کا وزن (جو \عددی{mg} کے برابر ہے) ذرے پر تجاذبی قوت  کی قدر سے مختلف ہو گا۔
\end{itemize}

%-------------------------------
%gravitataion near earth's surface p360
\جزوحصہء{سطح زمین کے قریب تجاذب}
آئیں فرض کریں زمین  یکساں  کروی ہے اور اس کی کمیت \عددی{M} ہے۔ زمین سے باہر ، زمین کے مرکز سے \عددی{r} فاصلے پر واقع ذرہ، جس کی کمیت \عددی{m} ہے، پر زمین کی تجاذبی قوت کی قدر   ،\عددی{F}، مساوات \حوالہ{مساوات_تجاذب_نیوٹن_کا_قانون_تجاذب_الف} دیتی ہے۔
%eq13.9
\begin{align}\label{مساوات_تجاذب_قوت_زمین}
F=G\frac{Mm}{r^2}
\end{align}
اگر ذرہ رہا  کیا جائے، تجاذبی قوت \عددی{\vec{F}} کی بنا ذرہ \اصطلاح{ تجاذبی اسراع }\فرہنگ{تجاذبی!اسراع}\حاشیہب{gravitational acceleration}\فرہنگ{gravitational!acceleration} \عددی{\vec{a}_g} کے ساتھ زمین  کے مرکز  کی طرف  گرے گا۔ نیوٹن کا قانون دوم  قوت کی قدر \عددی{F} اور اسراع \عددی{a_g} کا ذیل  تعلق دیتا ہے۔
%eq13.10
\begin{align}\label{مساوات_تجاذب_قوت_زمین_ب}
F=ma_g
\end{align}
مساوات \حوالہ{مساوات_تجاذب_قوت_زمین_ب} میں   \حوالہ{مساوات_تجاذب_قوت_زمین} سے \عددی{F} ڈال کر \عددی{a_g} کے لئے حل کر کے ذیل حاصل ہو گا۔
%eq13.11
\begin{align}\label{مساوات_تجاذب_حساب_اسراع}
a_g=\frac{GM}{r^2}
\end{align}
سطح زمین سے مختلف بلندیوں کے لئے \عددی{a_g} کی قیمتیں جدول \حوالہ{جدول_تجاذب_بلندی_اور_اسراع} میں  پیش ہیں۔دنیا کی بلند ترین چوٹی \اصطلاح{ سگرماتا }\فرہنگ{سگرماتا}\حاشیہب{Mount Everest}\فرہنگ{Mount Everest} پر \عددی{a_g=\SI{9.80}{\meter\per\second\squared}} ہے۔ آپ دیکھ سکتے ہیں  کہ \عددی{\SI{400}{\kilo\meter}} کی بلندی پر بھی \عددی{a_g} کی قیمت خاصی ہے۔
\begin{table}
\caption{بلندی کے ساتھ \عددی{a_g} کی تبدیلی}
\label{جدول_تجاذب_بلندی_اور_اسراع}
\centering
\begin{tabular}{ccr}
\toprule
بلندی& \(a_g\) &  \multicolumn{1}{c}{بلندی}\\
\((\si{\kilo\meter})\)&\((\si{\meter\per\second\squared})\)& \multicolumn{1}{c}{مثال}\\
\midrule
\(0\)& \(9.83\) & اوسط سطح زمین\\
\(8.8\)& \(9.80\) & سگرماتا\\
\(36.6\)& \(9.71\) & بلند ترین انسان بردار  غبارہ\\
\(400\)& \(8.70\) & خلائی جہاز کا مدار\\
\(35700\)& \(0.225\) & مواصلاتی سیارچے  کا مدار\\
\bottomrule
\end{tabular}
\end{table}

ہم حصہ \حوالہء{5.1} سے ، زمین  کا گھماو نظرانداز کر کے،  زمین  کو جمودی چوکھٹ  تصور کرتے رہے ہیں۔ اس تسہیل  کی بنا  ہم  فرض  کرتے  رہے ہیں  کہ  ذرے کے  آزاد گرنے  کا اسراع  \عددی{g} اور ذرے کا تجاذبی اسراع (جس کو اب ہم \عددی{a_g} کہتے ہیں)   ایک  ہیں۔ مزید، ہم فرض کرتے رہے ہیں کہ زمین پر ہر کہیں \عددی{g} کی قیمت مستقل اور  \عددی{\SI{9.8}{\meter\per\second\squared}} ہے۔ تاہم،   (درج ذیل) تین وجوہات کی بنا، کسی بھی مقام پر \عددی{g} کی پیمائشی قیمت،اس مقام پر  مساوات \حوالہ{مساوات_تجاذب_حساب_اسراع} سے حاصل  \عددی{a_g} کی قیمت سے مختلف ہو گی: (1)  زمین کی کمیتی تقسیم یکساں نہیں، (2) زمین مکمل کروی نہیں، اور (3)  زمین گھوم رہی ہے۔ ساتھ ہی، چونکہ \عددی{g}  کی قیمت \عددی{a_g} کی قیمت سے مختلف ہے، انہیں تین وجوہات کی بنا،  ذرے کی پیمائشی وزن \عددی{mg} ، مساوات \حوالہ{مساوات_تجاذب_قوت_زمین} سے حاصل ، ذرے پر تجاذبی قوت کی قدر سے مختلف ہو گی۔ آئیں ان وجوہات پر غور کریں۔

%--------------------------------------------------
\begin{enumerate}[1.]
\item
\موٹا{زمین کی کمیتی تقسیم یکساں نہیں۔}  زمین کے رداس کے ساتھ زمین  کی کثافت (کمیت فی اکائی حجم)     میں تبدیلی شکل \حوالہء{13.5} میں پیش ہے، اور  زمین کے قشر  (بیرونی پرت) کی کثافت   سطح زمین پر ایک مقام سے دوسرے مقام  مختلف ہے۔ یوں سطح زمین پر \عددی{g} کی قیمت ایک مقام سے دوسرے مقام مختلف ہو گی۔
\item
\موٹا{زمین مکمل کروی نہیں۔} زمین  تخمیناً   ترخیمی سطح ہے جو  قطبین پر   چپٹی   اور   خط استوا پر  ابھری ہے۔اس کا استوائی رداس  (زمین کے مرکز سے خط استوا تک فاصلہ) قطبی رداس (مرکز سے شمالی قطب یا جنوبی قطب تک فاصلہ)  سے \عددی{\SI{21}{\kilo\meter}}   زیادہ ہے۔ یوں، زمین کے  اندرونی کثیف  خطہ کو  ،  قطبین پر نقطہ  خط استوا  پر نقطہ    سے زیادہ قریب ہو گا۔یہ ایک وجہ ہے جس کی بنا، سطح سمندر پر  رہتے ہوئے خط استوا سے شمالی قطب  یا جنوبی قطب  کی طرف چلتے ہوئے آزاد گھرنے  کے اسراع  \عددی{g} کی پیمائشی قیمت بڑھتی ہے۔ جیسے جیسے آپ قطبین کی طرف بڑھتے ہیں، آپ  زمین کے مرکز کے قریب ہوتے جاتے ہیں، اور نیوٹن کے قانون تجاذب کے تحت، \عددی{g} بڑھتی ہے۔
\item
\موٹا{زمین گھوم رہی ہے۔}   زمین کی  محور گھماو  شمالی اور جنوبی قطب سے گزرتی ہے۔ قطبین کے علاوہ ، سطح زمین پر کسی بھی نقطہ پر   واقع جسم محور گھماو کے گرد  دائرے پر گھومتا ہے اور یوں  دائرے کے مرکز  کے رخ   مرکز مائل اسراع محسوس کرے گا۔ مرکز مائل اسراع کی بدولت  دائرے کے مرکز کے رخ مرکز مائل صافی قوت بھی ہو گی۔
\end{enumerate}
%-----------------------------------------------------

یہ سمجھنے کے لئے  کہ زمین کا گھماو \عددی{g} کو \عددی{a_g} سے کس طرح مختلف بناتا ہے، ہم ایک سادہ مثال پر غور کرتے ہیں۔ فرض کریں خط استوا پر کمیت \عددی{m} کا کریٹ  ترازو پر رکھا گیا ہے۔   شمالی قطب پر بلندی سے نظارہ  شکل \حوالہء{13.6a} میں دکھایا گیا ہے۔

شکل \حوالہء{13.6b} میں  کریٹ کا آزاد جسمی خاکہ، کریٹ  پر  عمل پیرا دو قوت  دکھاتا ہے۔ دونوں قوت محور  \عددی{r}  کے ہمراہ ہیں، جو مرکز زمین سے رداسی باہر رخ ہے۔ کریٹ  پر ترازو کی عمودی قوت \عددی{\vec{F}_N}  مثبت  محور \عددی{r} کے رخ، باہر کی طرف ہے۔ تجاذبی قوت، جو  \عددی{m\vec{a}_g} ہو گی، اندر کی طرف ہے۔چونکہ  کریٹ مرکز زمین کے گرد زمین کے ساتھ ساتھ گھومتا ہے، مرکز زمین کی طرف   کریٹ کا مرکز مائل  اسراع \عددی{\vec{a}}  ہو گا۔ مساوات \حوالہ{مساوات_گھماو_رداسی_اندر_اسراع_جزو} \عددی{(a_r=\omega^2r)} کے تحت یہ اسراع \عددی{\omega^2R} کے برابر ہو گا، جہاں \عددی{\omega} زمین کی زاوی رفتار اور \عددی{R}  دائرے کا ا رداس (جو تخمیناً زمین کے رداس کے برابر ہو گا) ہے ۔ یوں محور \عددی{r} پر قوتوں کے لئے نیوٹن کا قانون
 دوم \عددی{(F_{\text{\RL{صافی}},r}=ma_r)}  ذیل لکھا جائے گا۔
 %eq13.12
 \begin{align}\label{مساوات_تجاذب_ترازو_الف}
 F_N-ma_g=m(-\omega^2R)
 \end{align}
 عمودی قوت کی قدر \عددی{F_N} ترازو  پر  ناپا گیا وزن \عددی{mg}  ہے۔ مساوات \حوالہ{مساوات_تجاذب_ترازو_الف} میں \عددی{F_N} کی جگہ \عددی{mg} ڈال کر ذیل حاصل ہو گا:
 %eq 13.13
 \begin{align}\label{مساوات_تجاذب_ترازو_ب}
 mg=ma_g-m(\omega^2R)
 \end{align}
 جو ذیل کہتی ہے۔
 \begin{align*}
 (\text{\RL{ناپا گیا وزن}})= (\text{\RL{تجاذبی قوت کی قدر}})- (\text{\RL{کمیت ضرب مرکز مائل اسراع}})
 \end{align*}
 یوں زمین کے گھماو   کی بدولت کریٹ کا  ناپا گیا وزن تجاذبی قوت کی قدر سے کم ہو گا۔
 
 \موٹا{اسراع میں فرق۔}\quad
 مساوات \حوالہ{مساوات_تجاذب_ترازو_ب} میں دونوں اطراف \عددی{m} منسوخ کر کے اسراع  \عددی{g} اور  اسراع \عددی{a_g}  کا مطابقتی فقرہ حاصل کرتے ہیں:
 %eq13.14
 \begin{align}
 g=a_g-\omega^2R
 \end{align}
 جو  ذیل کہتا ہے۔
 \begin{align*}
 (\text{\RL{آزاد گرنے کا اسراع}})=(\text{\RL{تجاذبی اسراع}})-(\text{\RL{مرکز مائل اسراع}})
 \end{align*}
 یوں زمین کے گھماو کی بدولت آزاد گرنے کا اسراع تجاذبی اسراع سے کم ہو گا۔
 
 \موٹا{خط استوا۔}\quad
 اسراع \عددی{g} اور اسراع \عددی{a_g} میں فرق     \عددی{\omega^2R}  کے برابر ہے، جو خط استوا پر زیادہ سے زیادہ ہو گا(خط استوا پر کریٹ جس دائرے پر گھومتا ہے، اس کا رداس زیادہ سے زیادہ ہو گا)۔ اسراع کی قیمتوں میں فرق  جاننے کے لئے ہم   مساوات \حوالہ{مساوات_گھماو_اوسط_زاوی_سمتی_رفتار} \عددی{(\omega=\Delta \theta\!/\!\Delta t)}   اور زمین کا   رداس \عددی{R=\SI{6.37e6}{\meter}} استعمال کرتے ہیں۔ زمین کے ایک چکر کے لئے \عددی{\Delta \theta} کی قیمت \عددی{2\pi}  ریڈیئن   اور \عددی{\Delta t} کی قیمت تقریباً \عددی{24} گھنٹے ہے۔ ان قیمتوں کو استعمال کر کے (اور گھنٹوں کو سیکنڈ میں بدل کر)، ہم دیکھتے ہیں کہ \عددی{g} کی قیمت \عددی{a_g} کی قیمت   سے صرف \عددی{\SI{0.034}{\meter\per\second\squared}} کم ہے (جو \عددی{\SI{9.8}{\meter\per\second\squared}} کے لحاظ سے  کافی کم ہے)۔ اسی وجہ کی بنا، ہم \عددی{g} اور \عددی{a_g} میں فرق کو عموماً  نظرانداز کرتے ہیں۔ اسی طرح، وزن اور تجاذبی قوت کی قدر میں فرق بھی عموماً نظرانداز کیا جاتا ہے۔
  
  %-----------------------------------------
  %sample problem 13.02 difference in acceleration at head and feet p362
  \ابتدا{نمونی سوال}\موٹا{سر اور پیر پر اسراع میں فرق}\\
(ا)   ایک خلا باز ، جس کا قد \عددی{h=\SI{1.70}{\meter}} ہے، مرکز زمین سے \عددی{r=\SI{6.77e6}{\meter}} فاصلے پر خلائی طیارے میں \قول{پیر نیچے}   تیر رہا ہے۔ اس کے سر اور پیر پر  تجاذبی اسراع میں فرق کتنا ہو گا؟
  
  \موٹا{کلیدی تصورات}\\
ہم زمین کو یکساں کرہ تصور کر سکتے ہیں، جس کی کمیت \عددی{M_E}ہے۔ یوں، مرکز زمین سے \عددی{r} فاصلے پر تجاذبی اسراع مساوات  \حوالہ{مساوات_تجاذب_حساب_اسراع}    کے تحت ذیل ہو گا۔
%eq 13.15
\begin{align}\label{مساوات_تجاذب_نمونی_سر_پیر_الف}
a_g=\frac{GM_E}{r^2}
\end{align}
ہم  پیر  کا فاصلہ \عددی{r=\SI{6.77e6}{\meter}} اور سر کا فاصلہ \عددی{\SI{6.77e6}{\meter}+\SI{1.70}{\meter}} لے کر دونوں مقام پر اسراع معلوم کر کے ان کے بیچ فرق جان سکتے ہیں۔ تاہم ، \عددی{r} کے لحاظ سے \عددی{h} کی قیمت انتہائی  کم ہے،یوں اسراع کی  دو  قیمتوں میں فرق اتنا معمولی ہو گا  کہ عین ممکن ہے  محاسب (کیلکولیٹر) دونوں کی قیمت ایک دے اور یوں فرق صفر حاصل ہو گا، جو درست نہیں۔ بہتر طریقہ ذیل ہے: چونکہ خلا باز کے سر اور پیر کے بیچ \عددی{r} میں تفریقی فرق \عددی{\dif r} پایا جاتا ہے، ہمیں \عددی{r}   کے لحاظ سے مساوات \حوالہ{مساوات_تجاذب_نمونی_سر_پیر_الف}  کا تفرق لینا چاہیے۔

\موٹا{حساب:}\quad
رداس \عددی{r} کے لحاظ سے  مساوات \حوالہ{مساوات_تجاذب_نمونی_سر_پیر_الف} کا تفرق ذیل دیگا:
%eq 13.16
\begin{align}\label{مساوات_تجاذب_نمونی_سر_پیر_ب}
\dif a_g=-2\frac{GM_E}{r^3}\dif r
\end{align}
جہاں \عددی{r}  میں تفریقی تبدیلی \عددی{\dif r}  کی بنا تفریقی تجاذبی اسراع میں تفریقی تبدیلی \عددی{\dif a_g} ہے۔ خلا باز کے 
لئے، \عددی{\dif r=h} اور \عددی{r=\SI{6.77e6}{\meter}} ہے۔ مساوات \حوالہ{مساوات_تجاذب_نمونی_سر_پیر_ب} میں یہ معلومات ڈال کر ذیل حاصل ہو گا:
\begin{align*}
\dif a_g&=-2\frac{(\SI{6.67e-11}{\meter\cubed\per\kilo\gram\per\second\squared})(\SI{5.98e24}{\kilo\gram})}{(\SI{6.77e6}{\meter})^3} (\SI{1.70}{\meter})\\
&=\SI{-4.37e-6}{\meter\per\second\squared}\quad\quad\text{\RL{(جواب)}}
\end{align*}
جہاں زمین کی کمیت \عددی{M} ضمیمہ \حوالہء{C} سے لی گئی ہے۔ اس نتیجہ کے تحت زمین کی طرف  خلا باز کے پیر کا اسراع  اس کے سر کے اسراع سے معمولی زیادہ ہے۔ یہ فرق ( جو  \اصطلاح{مدوجزری اثر}\فرہنگ{مدوجزری اثر}\حاشیہب{tidal effect}\فرہنگ{tidal effect} کہلاتا ہے) خلا باز  کے جسم کو کھینچ کر لمبا کرنے کی کوشش کرتا ہے، تاہم یہ فرق اتنا معمولی ہے کہ وہ کبھی بھی اسے محسوس نہیں کر پائے گا۔

(ب)  اب فرض کریں خلا باز  \قول{پیر نیچے}  اسی رداس \عددی{r=\SI{6.77e6}{\meter}} پر  ثقب اسود کے گرد گھومتا ہے، جس کی کمیت \عددی{M_h=\SI{1.99e31}{\kilo\gram}}  (سورج کے \عددی{10} گنّا)ہے۔ اس کے سر اور پیر پر اسراع میں فرق کیا ہو گا؟ ثقب اسود کی ایک   ریاضی سطح (جو \اصطلاح{افق وقوعہ}\فرہنگ{افق وقوعہ}\حاشیہب{event horizon}\فرہنگ{event horizon} کہلاتی ہے) کا رداس \عددی{R_h=\SI{2.95e4}{\meter}} ہے۔ افق وقوعہ  سے، یا اس کے اندر سے، کوئی چیز باہر نہیں نکل سکتی؛ روشنی بھی اس سے نہیں نکل سکتی۔( اسی سے ثقب اسود یعنی سیاہ  روزن  کا نام نکلا ہے۔) خلا باز  اس  سطح سے کافی باہر (\عددی{r=229R_h} فاصلے پر) ہے۔

\موٹا{حساب:}\quad
یہاں بھی خلا باز کے سر اور پیر  میں \عددی{r}  کا تفریقی فرق \عددی{\dif r} پایا جاتا ہے، لہٰذا ہم دوبارہ  مساوات \حوالہ{مساوات_تجاذب_نمونی_سر_پیر_ب} استعمال کرتے ہیں۔ تاہم، اب \عددی{M_E} کی جگہ \عددی{M_h=\SI{1.99e31}{\kilo\gram}}   ہو گا۔ یوں ذیل حاصل ہو گا۔
\begin{align*}
\dif a_g&=-2\frac{(\SI{6.67e-11}{\meter\cubed\per\kilo\gram\per\second\squared})(\SI{1.99e31}{\kilo\gram})}{(\SI{6.77e6}{\meter})^3} (\SI{1.70}{\meter})\\
&=\SI{-14.5}{\meter\per\second\squared}\quad\quad\text{\RL{(جواب)}}
\end{align*}
خلا باز کے پیر پر اسراع اس کے سر پر اسراع سے  کافی زیادہ ہے۔ اگرچہ خلا باز اس کھینچ کو برداشت کر پائے گا لیکن  کافی درد   کے ساتھ۔ اگر وہ ثقب اسود کے مزید قریب جائے، اسراع میں فرق اتنا بڑھ سکتا ہے کہ اس کے جسم کو    چیرپھاڑ دے۔
  \انتہا{نمونی سوال}
  %-----------------------------------
  
  %13.4 gravitation inside earth p362
  \حصہ{زمین کے اندر تجاذب}
 \موٹا{مقاصد}\\
 اس حصہ کو پڑھنے کے بعد آپ ذیل کے قابل ہوں گے۔
 \begin{enumerate}[1.]
 \item
 جان پائیں گے کہ یکساں خول اس ذرے پر کوئی تجاذبی قوت  نہیں ڈالتی جو خول کے اندر ہو۔
 \item
مادہ کے نا گھومنے والے  یکساں کرہ  کے اندر واقع ذرے پر تجاذبی قوت کا حساب کر پائیں گے۔
 \end{enumerate}
 
 \موٹا{کلیدی تصورات}\\
 \begin{itemize}
 \item
مادے کا  یکساں خول اس ذرے پر کوئی صافی تجاذبی قوت لاگو نہیں کرتا جو اس  خول کے اندر ہو۔
\item
یکساں ٹھوس کرہ کے اندر، مرکز سے \عددی{r} فاصلے پر، واقع ذرے پر تجاذبی قوت \عددی{\vec{F}} صرف اس  کمیت \عددی{M_{\text{\RL{اندر}}}} کی بدولت ہو گی جو رداس \عددی{r}  کرہ کے اندر ہے:
\begin{align*}
M_{\text{\RL{اندر}}}=\frac{4}{3}\pi r^3 \rho =\frac{M}{R^3}r^3
\end{align*}
جہاں ٹھوس کرہ کی کثافت \عددی{\rho}، رداس \عددی{R}، اور کمیت \عددی{M} ہے۔ ہم  اندر کی کمیت  \عددی{M_{\text{\RL{اندر}}}}  کو ٹھوس کرہ کے مرکز پر ایک ذرے کی کمیت تصور کر کے نیوٹن کا  قانون تجاذب استعمال کرتے ہیں۔ یوں کمیت \عددی{m} پر ذیل قدر کی قوت عمل پیرا ہو گی۔
\begin{align*}
F=\frac{GmM}{R^3}r^3
\end{align*}
 \end{itemize}
 
 %-----------------------------------
 %gravitation inside earth p363
 \جزوحصہء{زمین کے اندر تجاذب}
 نیوٹن کا مسئلہ خول  یکساں خول کے اندر واقع ذرے پر لاگو کر کے ذیل حاصل ہو گا۔
 
 \ابتدا{قاعدہء}
مادے کے  یکساں خول کے اندر موجود ذرے پر خول صافی تجاذبی قوت لاگو نہیں کرتا۔
 \انتہا{قاعدہء}
 %----------------
 
 \ترچھا{انتباہ:} اس کا  ہرگز یہ  مطلب نہیں کہ  ذرے پر خول کے مختلف حصوں کا تجاذبی قوت  جادو سے غائب ہو جاتا ہے۔ درحقیقت،  خول کے تمام ٹکڑوں کے تجاذبی قوت کا مجموعہ صفر کے برابر ہو گا۔
 
 اگر زمین کی کمیت کی تقسیم یکساں ہوتی، سطح زمین پر موجود ذرے پر تجاذبی قوت زیادہ سے زیادہ ہوتی، اور  ذرے کو سطح سے  دور باہر کی طرف   لے جانے سے ذرے پر قوت کم  ہوتی۔ ذرے کو سطح کے اندر زمین کے مرکز کی طرف لے جانے سے تجاذبی قوت دو طرح اثر انداز ہو گی۔  (1)  چونکہ ذرہ زمین کے مرکز کے قریب ہو گا، اس پر تجاذبی قوت بڑھے گی۔ (2)   چونکہ  ذرے کے رداسی فاصلہ سے باہر خول ذرے  پر  صافی تجاذبی قوت پیدا نہیں کرتا لہٰذا  ذرے پر تجاذبی قوت گھٹے گی۔
 
 یکساں زمین کے اندر واقع کمیت \عددی{m}   کے جسم   پر تجاذبی قوت معلوم کرتے ہیں  ۔ شکل \حوالہء{13.7}  میں زمین کے آرپار، شمالی قطب سے جنوبی قطب تک،   سرنگ  دکھائی گئی  ہے ، جہاں  جنوبی قطب  سے سرنگ میں جسم  گر کر مرکز زمین سے \عددی{r} فاصلے پر پہنچا ہے۔ اس لمحے  جسم پر صرف \عددی{M_{\text{\RL{اندر}}}}  صافی تجاذبی قوت پیدا کرتی ہے، جو رداس \عددی{r} کے کرہ کے اندر  ( نقطہ دار لکیر  کے اندر) موجود   کمیت ہے ۔ نقطہ دار لکیر سے باہر خول جسم  پر صافی تجاذبی قوت پیدا نہیں کرتا۔ مزید، ہم   \عددی{M_{\text{\RL{اندر}}}} کو زمین کے مرکز پر  موجود ذرے کی کمیت تصور کر سکتے ہیں۔ یوں، مساوات \حوالہ{مساوات_تجاذب_نیوٹن_کا_قانون_تجاذب_الف} سے   جسم پر تجاذبی قوت کی قدر تلاش کرتے ہیں۔
 %eq13.17
 \begin{align}\label{مساوات_تجاذب_زمین_اندر_الف}
 F=\frac{GmM_{\text{\RL{اندر}}}}{r^2}
 \end{align}
 
 چونکہ ہم یکساں کثافت \عددی{\rho}  تصور کر رہے ہیں، ہم اس اندر کی کمیت کو  زمین کی کل کمیت  \عددی{M} اور رداس \عددی{R} کی صورت میں لکھ سکتے ہیں۔
 \begin{align*}
 \text{\RL{کثافت}}&=\frac{\text{\RL{اندر کی کمیت}}}{\text{\RL{اندر کا حجم}}}=\frac{\text{\RL{کل کمیت}}}{\text{\RL{کل حجم}}}\\
 \rho&=\frac{M_{\text{\RL{اندر}}}}{\tfrac{4}{3}\pi r^3}=\frac{M}{\tfrac{4}{3}\pi R^3}
 \end{align*}
  \عددی{M_{\text{\RL{اندر}}}} کے لئے حل کر کے ذیل حاصل ہو گا۔
  %eq13.18
  \begin{align}\label{مساوات_تجاذب_زمین_اندر_ب}
   M_{\text{\RL{اندر}}}=\frac{4}{3}\pi r^3 \rho=\frac{M}{R^3}r^3
  \end{align}
  مساوات \حوالہ{مساوات_تجاذب_زمین_اندر_الف} میں مساوات \حوالہ{مساوات_تجاذب_زمین_اندر_الف}    سے  \عددی{   M_{\text{\RL{اندر}}}} (کی دائیں ترین قیمت)   ڈال کر ذیل حاصل ہو گا۔
  %eq 13.19
  \begin{align}\label{مساوات_تجاذب_زمین_اندر_پ}
  F=\frac{GmM}{R^3}r
  \end{align}
  جیسا آپ دیکھ سکتے ہیں، زمین کے مرکز پر، جہاں \عددی{r=0} ہو گا، تجاذبی قوت صفر کے برابر ہے۔
  
  %--------------------------------
  %p364
  مساوات \حوالہ{مساوات_تجاذب_زمین_اندر_پ} کو سمتیہ  قوت  \عددی{\vec{F}} اور مرکز زمین سے باہر رخ رداسی محور کے ہمراہ جسم کے  تعین گر سمتیہ \عددی{\vec{r}} کی صورت میں  بھی لکھا جا سکتا ہے۔مساوات \حوالہ{مساوات_تجاذب_زمین_اندر_پ} میں تمام مستقلات  کو \عددی{K}  سے ظاہر کر کے ذیل لکھا جائے گا:
  %eq13.20
  \begin{align}\label{مساوات_تجاذب_زمین_اندر_ت}
  \vec{F}=-K\vec{r}
  \end{align}
  جہاں \عددی{\vec{F}} اور \عددی{\vec{r}} کی مخالف سمتوں کو  منفی کی علامت  شامل کر  کے ظاہر کیا گیا ہے۔ مساوات \حوالہ{مساوات_تجاذب_زمین_اندر_ت}  کا روپ قانون ہک (مساوات \حوالہء{7۔20}، \عددی{\vec{F}=-k\vec{d}})  کی طرح ہے۔ یوں (یکساں زمین کی صورت میں)،  اسپرنگ پر نسب جسم کی طرح، یہ جسم ارتعاش پذیر ہو گا، اور  مرکز زمین    ارتعاش کا مرکز ہو گا۔ جنوبی قطب سے سرنگ میں گر کر جسم مرکز زمین سے گزر کر شمالی قطب پہنچتا ہے۔وہاں سے جسم واپس مرکز زمین کی طرف گر کر مرکز زمین سے گزر کر  جنوبی قطب پہنچے گا، اور یہ سلسلہ اسی طرح چلتا رہے گا۔
  
  حقیقی دنیا  کی کمیتی تقسیم غیر  یکساں  ہے(شکل \حوالہء{13.5}) لہٰذا جیسے جیسے جسم زمین کی مرکز کی طرف گرتا ہے، اس پر ابتدائی طور قوت بڑھے گی۔ ایک مخصوص گہرائی پر قوت زیادہ سے زیادہ قیمت کو پہنچے گی اور اس کے بعد قوت کم ہونا شروع کرے گی۔
  
  %-----------------------------
  %13.5 gravitational potential energy p364
  \حصہ{تجاذبی مخفی توانائی}
  \موٹا{مقاصد}\\
  اس حصہ کو پڑھنے کے بعد آپ ذیل کے قابل ہوں گے۔
  \begin{enumerate}[1.]
  \item
  ذروں (یا یکساں کروں  جنہیں ذرے  تصور کیا جا سکے) کے نظام کی تجاذبی مخفی توانائی  کا حساب کر پائیں گے۔
  \item
  جان پائیں گے کہ  تجاذبی قوت محسوس کرتا  ذرہ  جب ابتدائی نقطہ سے اختتامی نقطہ  پہنچتا ہے، تجاذبی  قوت کا ذرے پر سرانجام کام (اور یوں تجاذبی مخفی توانائی میں تبدیلی) راہ پر منحصر نہیں۔
  \item
جب    جِرم فلکی  (یا     ساکن  جسم) کے قریب ایک ذرہ حرکت کرتا ہے،  ذرے پر   جِرم  کی تجاذبی قوت  استعمال کر کے  تجاذبی  قوت  کے سرانجام کام کا  حساب کر پائیں گے۔
\item
جِرم فلکی (یا  ساکن جسم) کے  لحاظ سے حرکت کرتے جسم پر (بشمول تجاذبی مخفی توانائی)     میکانی توانائی کی بقا کا اطلاق کر پائیں گے۔
\item
جِرم فلکی (جو عموماً یکساں کرہ تصور کیا جائے گا) سے   ذرے کی فراری   کے لئے  درکار توانائی کی وجوہات   سمجھ  پائیں گے۔
\item
جِرم فلکی سے   نکلنے  کے لئے ذرے کی   فرار کی  رفتار تلاش کر پائیں گے۔
  \end{enumerate}
  %---------------------------------------------
  
  \موٹا{کلیدی تصورات}\\
  \begin{itemize}
  \item
  دو ذروں، جن کی کمیتیں \عددی{M} اور \عددی{m} ہیں،  کے بیچ فاصلہ لامتناہی  سے  گھٹا کر \عددی{r} کرتے ہوئے، ایک ذرے پر دوسرے ذرے کی تجاذبی توانائی کا سرانجام کام ، دو ذروی     نظام، جس میں ذروں کے بیچ فاصل \عددی{r} ہو  ، کی مخفی توانائی \عددی{U(r)} کے منفی کے برابر ہو گا۔یہ توانائی ذیل ہو گی۔
  \begin{align*}
  U=-\frac{GMm}{r}\quad\quad\text{\RL{(تجاذبی مخفی توانائی)}}
  \end{align*}
  \item
  دو  سے زیادہ ذروں پر مشتمل نظام کی کل تجاذبی مخفی توانائی \عددی{U} تمام جوڑیوں کی مخفی توانائیوں کا مجموعہ ہو گا۔ مثال کے طور پر، تین ذروں کے لئے، جن کی کمیتیں \عددی{m_1}، \عددی{m_2}،اور \عددی{m_3} ہیں ذیل ہو گا۔
  \begin{align*}
  U=-\big(\frac{Gm_1m_2}{r_{12}}+\frac{Gm_1m_3}{r_{13}}+\frac{Gm_2m_3}{r_{23}}\big)
  \end{align*}
  \item
   کمیت \عددی{M} اور رداس \عددی{R}  کے  جِرم فلکی سے فرار (یعنی اس سے لامتناہی فاصلے پر پہنچنے)  کے لئے  جِرم فلکی کی سطح پر جسم کی رفتار کم از کم  (ذیل) فرار کی رفتار کے برابر ہونا لازم ہے۔
  \begin{align*}
  v=\sqrt{\frac{2GM}{R}}
  \end{align*}
  \end{itemize}
  
  %---------------------------------------
  %gravitational potential energy
  \جزوحصہء{تجاذبی مخفی توانائی}
  زمین و ذروی نظام کی تجاذبی مخفی توانائی  پر ہم حصہ \حوالہ{حصہ_مخفی_مخفی_توانائی} میں بات کر چکے ہیں۔ہم  نے ذرہ   سطح زمین کے قریب رکھا تا کہ  تجاذبی  قوت  مستقل تصور کرنا ممکن ہو۔اس کے بعد  ہم نے نظام  کے  کسی  حوالہ     تشکیل کے لئے تجاذبی مخفی توانائی صفر کے برابر رکھی۔عموماً، اس تشکیل میں ذرہ سطح زمین  پر ہوتا تھا۔ سطح زمین  سے دور ذرے کے لئے ذرے اور زمین کے بیچ فاصلہ گھٹانے سے تجاذبی مخفی توانائی گھٹتی  تھی۔
  
  یہاں، ہم اپنی سوچ کو وسعت دے کر کمیت \عددی{m} اور \عددی{M} کے دو ذروں، جن کے بیچ فاصلہ \عددی{r} ہے، کی تجاذبی مخفی توانائی \عددی{U} پر غور کرتے ہیں۔ اب بھی ہم ایک حوالہ تشکیل منتخب کرتے ہیں جس  کی \عددی{U} صفر رکھی جاتی ہے۔ تاہم،  مساوات  سادہ رکھنے کی خاطر، حوالہ تشکیل میں دوری کا فاصلہ \عددی{r} اتنا بڑا رکھا جاتا ہے کہ اس کو  \ترچھا{لامتناہی } تصور کرنا ممکن ہو۔ یہاں بھی،  دوری کا فاصلہ گھٹانے سے تجاذبی  مخفی  توانائی گھٹے گی۔ چونکہ \عددی{r=\infty} کے لئے \عددی{U=0} ہے،  کسی بھی متناہی فاصلے کے لئے تجاذبی توانائی منفی ہو گی، جو ذرے قریب کرنے سے مزید زیادہ منفی ہو گی۔
  
  ان حقائق کو مد نظر رکھ کر، ہم دو ذروی نظام کی تجاذبی مخفی توانائی  ذیل لیتے ہیں (اس کا جواز ہم جلد پیش کریں گے)۔
  %eq13.21
  \begin{align}\label{مساوات_تجاذب_مخفی_الف}
  U=-\frac{GMm}{r}\quad\quad\text{\RL{(تجاذبی مخفی توانائی)}}
  \end{align}
  یاد رہے، \عددی{r} لامتناہی  کے قریب تر  کرنے سے \عددی{U(r)} صفر  کے قریب تر پہنچتی ہے، اور \عددی{r} کی ہر متناہی قیمت کے لئے \عددی{U(r)} منفی ہو گا۔
  
\موٹا{بول چال کا انداز۔}\quad
مساوات \حوالہ{مساوات_تجاذب_مخفی_الف}  میں پیش مخفی توانائی دو ذروی نظام کی مخفی توانائی ہے، نا کہ ایک یا دوسرے ذرے کی۔ ایسا کوئی طریقہ موجود نہیں جس سے ہم توانائی کا کچھ حصہ ایک ذرے کو اور کچھ حصہ دوسرے ذرے کو مختص کریں۔ تاہم، \عددی{M\gg m} کی صورت میں، جو زمین (جس کی کمیت \عددی{M} ہے)  اور  گیند (جس کی کمیت \عددی{m} ہے) کے لئے درست ہو گا، ہم عموماً \قول{ گیند کی مخفی توانائی } کی بات کرتے ہیں۔ ہم یوں اس لئے کر سکتے ہیں کہ، جب زمین  کے قریب  گیند حرکت کرے زمین و گیند نظام کی مخفی توانائی میں تبدیلی  تقریباً مکمل طور پر گیند کی  حرکی توانائی میں تبدیلی  کے روپ میں نظر آتی ہے؛ زمین  کی حرکی توانائی میں تبدیلی اتنی کم ہو گی کہ اسے ناپنا ممکن نہیں ہو گا۔ اسی طرح حصہ \حوالہء{13.7} میں، مصنوعی سیارچہ کی کمیت زمینی کمیت سے اتنی کم ہے کہ  ہم زمین کی مدار میں \قول{ مصنوعی سیارچہ  کی مخفی توانائی } کی بات کریں گے۔ ہم پلہ کمیت کے اجسام  کی مخفی توانائی کی بات کرتے ہوئے، ہمیں محتاط  رہ کر انہیں ایک  نظام تصور کرنا ہو گا۔

\موٹا{متعدد ذرات۔}\quad
دو سے زیادہ ذروں پر مشتمل نظام کی صورت میں ہم     ذروں کی  ایک  ایک جوڑی   باری باری لے کر ان کی تجاذبی مخفی توانائی مساوات \حوالہ{مساوات_تجاذب_مخفی_الف} سے حاصل کر کے ، تمام کا الجبرائی مجموعہ لیتے ہیں۔ جوڑی کی تجاذبی مخفی توانائی حاصل کرتے وقت باقی ذروں کی موجودگی نظر انداز کی جاتی ہے۔ شکل \حوالہء{13.8}  کی  تینوں جوڑیوں پر مساوات \حوالہ{مساوات_تجاذب_مخفی_الف} کا اطلاق کر کے نظام کی (ذیل)  تجاذبی  مخفی توانائی حاصل ہو گی۔
%eq13.22
\begin{align}
  U=-\big(\frac{Gm_1m_2}{r_{12}}+\frac{Gm_1m_3}{r_{13}}+\frac{Gm_2m_3}{r_{23}}\big)
\end{align}

%proof of equation 13.21 p365
\حصہء{مساوات \حوالہ{مساوات_تجاذب_مخفی_الف} کا ثبوت}
آئیں شکل \حوالہء{13.9} میں دکھائی گئی راہ پر سیدھا  اوپر وار ، زمین سے دوری کے رخ، گیند پھینکتے ہیں۔  مرکز زمین سے \عددی{R} رداسی فاصلے پر، راہ  کے نقطہ \عددی{P} پر، گیند کی تجاذبی مخفی توانائی \عددی{U} کا الجبرائی فقرہ تلاش کرتے ہیں   ۔ گیند  کے نقطہ \عددی{P} سے روانا ہو کر زمین سے لامتناہی فاصلے تک  سفر کے دوران گیند پر تجاذبی قوت کا سرانجام کام \عددی{W} تلاش کر کے، ہم ایسا کر سکتے ہیں۔ تجاذبی قوت \عددی{\vec{F}(r)} تغیر پذیر قوت ہے (اس کی قدر \عددی{r} پر منحصر ہے)، لہٰذا  کام تلاش کرنے کے لئے  حصہ \حوالہء{7.5} کی تراکیب استعمال کرنی ہوں گی۔  سمتیہ ترقیم میں ذیل لکھا جا سکتا ہے۔
%eq13.23
\begin{align}\label{مساوات_تجاذب_ثبوت_الف}
W=\int_{R}^{\infty} \vec{F}(r)\cdot \dif \vec{r}
\end{align}
تکمل میں قوت \عددی{\vec{F}(r)} اور گیند کی راہ  پر تفریقی ہٹاو سمتیہ \عددی{\dif \vec{r}} کا غیر سمتی (نقطی) ضرب پایا جاتا ہے، جو ذیل لکھا جا سکتا ہے:
%eq13.24
\begin{align}\label{مساوات_تجاذب_ثبوت_ب}
\vec{F}(r)\cdot \dif \vec{r}=F(r)\dif r\cos\phi
\end{align}
جہاں \عددی{\vec{F}(r)} اور \عددی{\dif \vec{r}} کے بیچ زاویہ \عددی{\phi} ہے۔ زاویہ \عددی{\phi} کی جگہ \عددی{\SI{180}{\degree}} اور \عددی{F(r)} کی جگہ مساوات \حوالہ{مساوات_تجاذب_نیوٹن_کا_قانون_تجاذب_الف} ڈال کر مساوات \حوالہ{مساوات_تجاذب_ثبوت_ب} ذیل روپ اختیار کرتی ہے:
\begin{align*}
\vec{F}(r)\cdot \dif \vec{r}=-\frac{GMm}{r^2}\dif r
\end{align*}
جہاں زمین کی کمیت \عددی{M} اور گیند کی کمیت \عددی{m} ہے۔

اس کو مساوات \حوالہ{مساوات_تجاذب_ثبوت_الف} میں ڈال کر تکمل لیتے ہیں:
%eq13.25
\begin{gather}
\begin{aligned}\label{مساوات_تجاذب_ثبوت_پ}
W&=-GMm\int_{R}^{\infty} \frac{1}{r^2}\dif r=\big[\frac{GMm}{r}\big]_{R}^{\infty}\\
&=0-\frac{GMm}{R}=-\frac{GMm}{R}
\end{aligned}
\end{gather}
جہاں نقطہ \عددی{P} سے (جو فاصلہ \عددی{R} پر واقع ہے)  لامتناہی  فاصلے پر گیند  پہنچانے کے لئے درکار کام \عددی{W} ہو گا۔ مساوات \حوالہء{8.1} \عددی{\Delta U=-W} کہتی ہے ہم کام کو مخفی توانائی  کے روپ میں ذیل  لکھ سکتے ہیں۔
\begin{align*}
U_{\infty}-U=-W
\end{align*}
چونکہ لامتناہی پر مخفی توانائی \عددی{U_{\infty}} صفر ،نقطہ  \عددی{P}  پر مخفی توانائی \عددی{U}  ، اور \عددی{W} کام ہے جو مساوات \حوالہ{مساوات_تجاذب_ثبوت_پ}  دیتی  ہے، لہٰذا  یہ مساوات ذیل روپ اختیار کرتی ہے۔
\begin{align*}
U=W=-\frac{GMm}{R}
\end{align*}
اس میں \عددی{R} کی جگہ \عددی{r} لکھ کر مساوات \حوالہ{مساوات_تجاذب_مخفی_الف}  حاصل ہو گا۔ یوں ثبوت مکمل ہوتا ہے۔

\حصہء{راہ  پر غیر انحصار}
شکل \حوالہء{13.10} میں    تین رداسی لمبائیوں اور تین دائری قوس  (جن کا مرکز زمین ہے) پر   مشتمل  راہ پر چل کر  گیند  کو نقطہ \عددی{A} سے نقطہ \عددی{G}  منتقل کیا گیا ہے۔ نقطہ \عددی{A} سے نقطہ \عددی{G} تک  سفر کے دوران گیند پر زمین کی تجاذبی قوت \عددی{\vec{F}}  کے سرانجام کل  کام  \عددی{W} میں ہم دلچسپی رکھتے ہیں۔تینوں    دائری راہ  کے ہر نقطہ پر \عددی{\vec{F}} اور  راہ  آپس میں عمود دار ہیں لہٰذا ان قوسی راہوں پر سرانجام  کام صفر ہو گا۔ یوں \عددی{W} صرف تین  رداسی راہ پر  \عددی{\vec{F}} کے سرانجام کام کا مجموعہ ہو گا۔

فرض کریں، ہم لمحاتی طور  قوسی  راہوں  کی لمبائی گھٹا کر صفر کرتے ہیں۔ہم اب گیند کو ایک رداسی راہ پر \عددی{A} سے سیدھا \عددی{G} پہنچاتے ہیں۔ کیا ایسا کرنے سے \عددی{W} تبدیل ہو گا؟ جی نہیں۔ قوسی راہ پر سرانجام کام صفر ہے لہٰذا  ان راہ کو ہٹانے سے کام پر اثر نہیں پڑتا۔ نقطہ \عددی{A} سے نقطہ \عددی{G} تک راہ اب یقیناً پہلی راہ سے مختلف ہے، تاہم \عددی{\vec{F}} کا سرانجام کام وہی ہے۔

ہم  حصہ \حوالہء{8.1} میں اس طرح کے نتیجہ  پر عمومی بحث کر چکے ہیں۔یاد رہے، تجاذبی قوت    ایک بقائی  قوت ہے۔ یوں ابتدائی نقطہ \عددی{i} سے ذرہ  اختتامی نقطہ \عددی{f} منتقل کرتے ہوئے، ذرے پر تجاذبی قوت کا سرانجام کام ان نقطوں کے بیچ راہ پر منحصر نہیں ہو گا۔ مساوات \حوالہء{8.1} کے تحت، نقطہ \عددی{i} سے نقطہ \عددی{f}  تجاذبی مخفی توانائی میں تبدیلی \عددی{\Delta U} ذیل ہو گی۔
%eq13.26
\begin{align}
\Delta U=U_f-U_i=-W
\end{align}
چونکہ بقائی قوت  کا سرانجام کام \عددی{W}  اصل راہ پر منحصر نہیں،  تجاذبی مخفی توانائی میں تبدیلی \عددی{\Delta U} بھی راہ پر منحصر نہیں۔

\حصہء{مخفی توانائی اور قوت}
مساوات \حوالہ{مساوات_تجاذب_مخفی_الف} کے  ثبوت  میں،   ہم نے تفاعل قوت \عددی{\vec{F}(r)} سے مخفی توانائی کا تفاعل \عددی{U(r)} حاصل کیا۔ ہم توقع کرتے ہیں کہ  الٹ چل کر مخفی توانائی کے تفاعل سے آغاز کر کے  تفاعل قوت  حاصل  کرنا ممکن ہو گا۔ مساوات \حوالہء{8.22} \عددی{(F(x)=-\dif U(x)\!/\!\dif x)}  سے سبق  لیتے ہوئے ، ہم ذیل لکھتے ہیں۔
%eq13.27
\begin{gather}
\begin{aligned}
F&=-\frac{\dif U}{\dif r}=-\frac{\dif}{\dif r}\big(-\frac{GMm}{r}\big)\\
&=-\frac{GMm}{r^2}
\end{aligned}
\end{gather}
یہ نیوٹن کا قانون تجاذب (مساوات \حوالہ{مساوات_تجاذب_نیوٹن_کا_قانون_تجاذب_الف}) ہے ۔ منفی علامت کہتی ہے کمیت \عددی{m} پر قوت کا رخ رداسی اندر وار  ، کمیت \عددی{M} کے رخ  ، ہے۔

%----------------------------------
%escape velocity p367
\حصہء{فرار کی رفتار}
اگر آپ پتھر آسمان کی طرف  اوپر وار پھینکیں،   پتھر  کی رفتار بتدریج کم ہو گی حتٰی کہ یہ لمحاتی رک کر واپس زمین پر گرے گا۔ زیادہ ابتدائی رفتار کے ساتھ روانا کرنے پر پتھر زیادہ بلندی تک پہنچتا ہے۔ پتھر کی ابتدائی رفتار اتنی بڑھائی جا سکتی ہے کہ  اس کا لمحاتی رکھنے کا نقطہ لامتناہی پر واقع ہو۔ اس رفتار کو  (زمین سے)  \اصطلاح{فرار  کی رفتار }\فرہنگ{فرار کی رفتار}\حاشیہب{escape speed}\فرہنگ{escape speed} کہتے ہیں۔

فرض کریں کمیت \عددی{m} کا گولا  کسی سیارہ   (یا کسی جِرم فلکی  یا نظام)  کی سطح  سے فرار کی رفتار \عددی{v} سے روانا ہوتا ہے۔ گولے کی حرکی توانائی \عددی{K} کی قیمت \عددی{\tfrac{1}{2}mv^2} ، اور مخفی توانائی \عددی{U} ذیل  ہو گی جو مساوات \حوالہ{مساوات_تجاذب_مخفی_الف} سے حاصل کی جا سکتی ہے:
\begin{align*}
U=-\frac{GMm}{R}
\end{align*}
جہاں سیارہ کی کمیت \عددی{M} اور اس کا رداس \عددی{R} ہے۔

لامتناہی پہنچ کر، گولا لمحاتی رکھتا ہے لہٰذا اس کی حرکی توانائی صفر ہو گی۔ گولے کی مخفی توانائی بھی صفر ہو گی، چونکہ لامتناہی فاصلے پر دو جسم ہماری صفر مخفی توانائی تشکیل ہے۔ اس طرح، لامتناہی پر گولے کی کل  توانائی صفر ہو گا۔ توانائی کی بقا کے اصول کے تحت، سیارہ کی سطح پر بھی گولے کی کل توانائی صفر ہونی چاہیے۔ یوں ذیل ہو گا۔
\begin{align*}
K+U=\frac{1}{2}mv^2+\big(-\frac{GMm}{R}\big)=0
\end{align*}
اس سے ذیل حاصل ہو گا۔
%eq13.28
\begin{align}\label{مساوات_تجاذب_رفتار_فرار}
v=\sqrt{\frac{2GM}{R}}
\end{align}
یاد رہے، \عددی{v} کی قیمت اس پر منحصر نہیں کہ گولا سطح سیارہ سے کس رخ پھینکا جاتا ہے۔تاہم، اس رفتار  کا حصول زیادہ آسان ہو گا اگر گولا اس رخ پھینکا جائے جس رخ ، سیارہ کی اپنی محور کے  گرد گھماو کی بدولت، سطح  سیارہ  حرکت کرتی  ہو۔مثلاً، حیدرآباد  سے مصنوعی سیارہ   خلا میں بھیجنے کے لئے   ہوائی بان (راکٹ) مشرق  وار روانا کیا جائے گا، تا کہ   زمین کی گردش  کی بنا حیدرآباد  پر سطح زمین کی \عددی{\SI{1500}{\kilo\meter\per\hour}} مشرق وار رفتار سے فائدہ اٹھایا جا سکے۔

کسی بھی  جِرم فلکی کا رداس \عددی{R} اور کمیت \عددی{M}  مساوات \حوالہ{مساوات_تجاذب_رفتار_فرار} میں پُر کر کے اس سے جسم کی فرار کی رفتار معلوم کی جا سکتی ہے۔ جدول \حوالہ{جدول_تجاذب_رفتار_فرار} میں چند فرار کی  رفتاریں پیش ہیں۔

\begin{table}
\caption{چند فرار کی رفتاریں}
\label{جدول_تجاذب_رفتار_فرار}
\centering
\renewcommand{\arraystretch}{1.25}
\begin{tabular}{RRRC}
\toprule
\text{\RL{جِرم فلکی}}&(\si{\kilo\gram})\, \text{\RL{کمیت}}&(\si{\meter})\, \text{\RL{رداس}}&(\si{\kilo\meter\per\second})\, \text{\RL{فرار کی رفتار}} \\
\midrule
\text{\RL{سیرس سیارچہ}} &\num{1.17e21} &\num{3.8e5} &\num{0.64} \\
\text{\RL{زمین کا چاند}} & \num{7.36e22} &\num{1.74e6} &\num{2.38} \\
\text{\RL{زمین}} &\num{5.98e24} &\num{6.37e6} &\num{11.2} \\
\text{\RL{مشتری}} &\num{1.90e27} &\num{7.15e7} &\num{59.5} \\
\text{\RL{سورج}} &\num{1.99e30} &\num{6.96e8} &\num{618} \\
\text{\RL{الحبور B}} &\num{2e30} &\num{1e7} &\num{5200} \\
\text{\RL{نیوٹران تارا}} &\num{2e30} &\num{1e4} &\num{2e5} \\
\bottomrule
\end{tabular}
\end{table}

\اصطلاح{سیرس}\فرہنگ{سیارچہ!سیرس}\حاشیہب{Ceres}\فرہنگ{asteroid!Ceres}  سب سے بڑا  \اصطلاح{سیارچہ }\فرہنگ{سیارچہ}\حاشیہب{asteroid}\فرہنگ{asteroid} ہے؛ \اصطلاح{ العبور  با } \فرہنگ{Sirius B}\حاشیہب{Sirius B} ایک \اصطلاح{  سفید بالشتیا }\فرہنگ{بالشتیا!سفید}\حاشیہب{white dwarf}\فرہنگ{dwarf!white} ہے؛ اور  \اصطلاح{مستعر اعظم     }\فرہنگ{مستعر اعظم}\حاشیہب{supernova}\فرہنگ{supernova} میں دھماکے سے پھٹنے کے بعد تارے  کا  منہدم  قلب    نیوٹران تارا  کہلاتا ہے۔

%-------------------------------
%checkpoint 3 p367
\ابتدا{آزمائش}
کمیت \عددی{M} کے کرہ سے آپ کمیت \عددی{m} کا گیند دور لے جاتے ہیں۔ (ا)  کیا گیند و کرہ نظام کی تجاذبی مخفی توانائی بڑھتی ہے یا گھٹتی ہے؟ (ب)  کیا گیند اور کرہ کے بیچ تجاذبی قوت مثبت یا منفی کام کرتی ہے؟
\انتہا{آزمائش}
%------------------------------

%sample problem 13.03 asteroid falling from space, mechanical energy p368
\ابتدا{نمونی سوال}\موٹا{خلا سے سیارچے کا زمین پر گرنا، میکانی توانائی}\\
ایک سیارچہ   ،جو زمین سے \عددی{10} زمینی رداس  کے فاصلے پر ہے ، کی زمین کے لحاظ سے رفتار \عددی{\SI{12}{\kilo\meter\per\second}} ہے، اور وہ سیدھا زمین کے طرف گامزن ہے۔ \اصطلاح{  کرہ باد  }\فرہنگ{کرہ باد}\حاشیہب{atmosphere}\فرہنگ{atmosphere} کا سیارچے پر اثر نظرانداز کر کے سطح زمین پر پہنچ کر سیارچے کی رفتار \عددی{v_f}  تلاش کریں۔

\حصہء{کلیدی تصورات}
چونکہ سیارچے پر  کرہ باد کا  اثر نظرانداز کیا جاتا ہے، سیارچہ و زمین نظام کی میکانی توانائی کی بقا ہو گی۔ یوں ( سطح زمین پر پہنچ کر ) سیارچے کی اختتامی میکانی توانائی  اس کی ابتدائی میکانی توانائی کے برابر ہو گی۔ حرکی توانائی \عددی{K} اور تجاذبی مخفی توانائی \عددی{U} کی صورت میں یوں ذیل  لکھا جائے گا۔
%eq 13.29
\begin{align}\label{مساوات_تجاذب_نمونی_تین_الف}
K_f+U_f=K_i+U_i
\end{align}

ساتھ ہی، اگر ہم نظام کو جدا تصور کریں،  گرنے کے دوران  نظام کے ابتدائی معیار حرکت   کی بقا بھی ہو گی۔یوں، سیارچے اور زمین  کے معیار حرکت میں تبدیلی کی قدریں برابر اور علامت الٹ  ہوں گی۔ تاہم،سیارچے کی کمیت کے لحاظ سے  زمین کی کمیت  اتنی زیادہ ہے کہ زمین کی رفتار میں تبدیلی سیارچے کی رفتار میں تبدیلی کے لحاظ سے قابل نظرانداز ہو گی۔ اس طرح، زمین کی حرکی توانائی میں تبدیلی بھی قابل نظرانداز ہو گی۔یوں،  مساوات \حوالہ{مساوات_تجاذب_نمونی_تین_الف} میں  حرکی توانائیاں صرف  سیارچے کی تصور کی جا سکتی ہیں۔

\موٹا{حساب:}\quad
سیارچے کی کمیت \عددی{m} ، زمین کی کمیت \عددی{M}  \عددی{(\SI{5.98e24}{\kilo\gram})}  اور زمین کا
 رداس \عددی{R_E}   \عددی{(\SI{6.37e6}{\meter})} لیتے ہیں۔ سیارچہ ابتدائی طور \عددی{10R_E} فاصلے پر ہے۔ مساوات \حوالہ{مساوات_تجاذب_مخفی_الف} سے    \عددی{U} اور \عددی{K} کی جگہ \عددی{\tfrac{1}{2}mv^2} ڈال کر    مساوات \حوالہ{مساوات_تجاذب_نمونی_تین_الف}  لکھی جائے گی۔
 \begin{align*}
 \frac{1}{2}mv_f^2-\frac{GMm}{R_E}=\frac{1}{2}mv_i^2-\frac{GMm}{10R_E}
 \end{align*}
 ترتیب نو کے بعد معلومات پُر کر کے ذیل حاصل ہو گا:
 \begin{align*}
 v_f^2&=v_i^2+\frac{2GM}{R_E}\big(1-\frac{1}{10}\big)\\
 &=(\SI{12e3}{\meter\per\second})^2+\frac{2(\SI{6.67e-11}{\meter\cubed\per\kilo\gram\per\second\squared})(\SI{5.98e24}{\kilo\gram})}{\SI{6.37e6}{\meter}}\,0.9\\
 &=\SI{2.567e8}{\meter\squared\per\second\squared}
 \end{align*}
 جس سے اختتامی رفتار ذیل حاصل ہو گی۔
 \begin{align*}
 v_f=\SI{1.60e4}{\meter\per\second}=\SI{16}{\kilo\meter\per\second}\quad\quad\text{\RL{(جواب)}}
 \end{align*}
 
 اس رفتار پر، سیارچہ زیادہ بڑا نہ ہونے کی صورت میں بھی خاصہ نقصان کر سکتا ہے۔ صرف \عددی{\SI{5}{\meter}} قطر کا سیارچہ اس رفتار سے زمین پر گر کر  ہیروشیما  پر گرائے  گئے  ایٹمی بم جتنی  تباہی برپا کرے  گا۔ اس جسامت کے تقریباً \عددی{50} کروڑ  سیارچے زمینی مدار کے قریب  پائے جاتے ہیں، اور \سن{1994} میں  ان میں سے ایک   زمین کے کرہ باد میں داخل ہو کر  جنوبی بحرالکاہل  میں \عددی{\SI{20}{\kilo\meter}} کی بلندی پر دھماکے کے ساتھ  سے   پھٹا۔ 
\انتہا{نمونی سوال}
%-----------------------------------

%13.6 planets and satellites:Kepler's Laws p368
\حصہ{سیارے اور سیارچے:قوانین کپلر}
\موٹا{مقاصد}\\
اس حصہ کر پڑھنے کے بعد آپ ذیل کے قابل ہوں گے۔
\begin{enumerate}[1.]
\item
 \اصطلاح{ کپلر }\فرہنگ{قوانین کپلر}\حاشیہب{Kepler}\فرہنگ{Kepler's laws}  کے تین قوانین جان پائیں گے۔
\item
جان پائیں گے کپلر کا کونسا قانون زاوی معیار حرکت کی بقا  کے معادل ہے۔
\item
بیضوی مدار  کے خاکہ پر  نصف اکبر محور،  عن المرکزیت ،  حضیض شمسی،  اوج شمسی، اور   نقاط ماسکہ جان پائیں گے۔
\item
بیضوی مدار کے لئے  اکبر محور، عن المرکزیت، حضیض شمسی، اور اوج شمسی کے  تعلق  کا اطلاق کر پائیں گے۔
\item
مدار میں قدرتی یا مصنوعی سیارچہ  کے لئے،  مدار کا دوری عرصہ،  رداس ، اور اس   جِرم فلکی کی کمیت  کا کپلر رشتہ استعمال کر پائیں گے، جس کے گرد مدار ہو۔
\end{enumerate}

\موٹا{کلیدی تصورات}\\
\begin{itemize}
\item
قدرتی اور مصنوعی سیارچوں کی حرکت کپلر کے  (درج ذیل تین) قوانین   مطمئن کرتی ہے۔
\begin{enumerate}[1.]
\item
\ترچھا{مدار کا قانون۔}   تمام سیارات  بیضوی مدار میں گھومتے ہیں جن  کے ایک ماسکہ پر سورج واقع ہو گا۔
\item
\ترچھا{رقبوں کا قانون۔}  سیارہ  اور سورج کو ملانے والی لکیر برابر وقتی وقفوں میں  ایک جتنا رقبہ  جھاڑتی ہے۔
\item
\ترچھا{دوری عرصہ کا قانون۔}  سیارے کے  دوری عرصہ \عددی{T}  کا مربع اس کے مدار کے  نصف اکبر محور  \عددی{a} کے مکعب کے برابر ہو گا۔ دائری مدار  کے لئے، جس کا رداس \عددی{r} ہو ذیل ہو گا:
\begin{align*}
T^2=\left(\frac{4\pi^2}{GM}\right)r^3\quad\quad\text{\RL{(دوری عرصہ کا قانون)}}
\end{align*}
جہاں \عددی{M} کششی جسم کی کمیت ہو گی، جو  نظام شمسی میں سورج ہو گا۔ بیضوی سیاروی  مدار کے لئے \عددی{r} کی جگہ نصف اکبر محور \عددی{a}   رکھا جائے گا۔
\end{enumerate}
\end{itemize}

%---------------------------------
%Planets ans Satellites: Kepler's Laws p369
\جزوحصہء{سیارے اور سیارچے: قوانین کپلر}
ستاروں کے پس منظر  میں سیاروں کی بظاہر  بے مقصد  چلن  زمانہ قدیم سے انسان  کی حیرانی کا سبب بنا رہا۔ شکل \حوالہء{13.11} میں  مریخ  کی  گھیر در گھیر حرکت  بالخصوص حیران کن تھی۔یوہانس کپلر (\سن{1571} تا \سن{1630}) نے اپنی زندگی  ان حرکات کا   راز  فاش کرنے میں گزاری۔ انہوں نے زندگی بھر  کے مشاہدات  میں کثیر مواد جمع کر کے ان سے تجربی   قوانین اخذ کیے۔ بعد میں نیوٹن (\سن{1642} تا \سن{1727}) نے دکھایا کہ کپلر کے قوانین نیوٹن کے قانون  تجاذب سے اخذ کیے جا سکتے ہیں۔

اس حصہ میں کپلر کے تین قوانین پر غور کی جائے گی۔اگرچہ، یہاں ہم سورج کے گرد سیاروں  کی حرکت پر ان کا اطلاق کریں گے، ان کا  اطلاق  زمین یا کسی دوسرے وسطی جسیم  جسم کے مدار میں قدرتی اور مصنوعی   سیارچوں پر بھی ہو گا۔

%---------------------------------------
\ابتدا{قاعدہء}
\موٹا{قانون مدار:}  تمام سیارے بیضوی مدار پر حرکت کرتے ہیں اور سورج   بیضوی مدار کے ایک ماسکہ پر   ہو گا۔
\انتہا{قاعدہء}
%-------------------------------------
