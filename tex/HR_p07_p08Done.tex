%HR_p07_p08
%example 1.02
\ابتدا{مثال}\ترچھا{کثافت اور رقیق کاری }\\
ایسے زلزلہ کے دوران جس میں زمین کی \اصطلاح{رقیق کاری }\فرہنگ{رقیق کاری}\حاشیہب{liquefaction}\فرہنگ{liquefaction} ہو ،  بھاری جسم زمین میں دھنس سکتا ہے۔    رقت  کے دوران مٹی کے ذرے  نہایت  کم رگڑ محسوس کرتے ہوئے  ریلنا  شروع کرتے ہیں اور زمین  دلدل کی کیفیت اختیار کرتی   ہے ۔ ریتیلی زمین کی رقیق کاری کے ممکنات کی پیشنگوئی زمین کے نمونہ کی تناسب   خلا\عددی{e} کے روپ  میں کی جا سکتی ہے۔
\begin{align}\label{مساوات_پیمائش_خلا_تناسب}
e=\frac{V_{\text{\RL{خلا}}}}{V_{\text{\RL{دانے}}}} 
\end{align} 
 یہاں \عددی{V_{\text{\RL{دانے}}}}   نمونے میں ریت کے ذرات کا کل حجم جبکہ \عددی{V_{\text{\RL{خلا}}}}  ذروں  کے بیچ  خلا کا کل حجم ہے ۔اگر \عددی{e} فاصل قیمت \عددی{0.80} سے تجاوز کرتا ہو،  زلزلہ کے دوران رقیق کاری کا امکان ہوگا۔  مطابقتی  ریت  کی کثافت \عددی{\rho_{\text{\RL{ریت}}}}  کیا ہوگی؟  ٹھوس سلیکان ڈائی  اکسائیڈ  ، \عددی{(\ce{SiO2})} (  جو ریت کا بنیادی جزو ہے)   کی کثافت \عددی{\rho_{\ce{SiO2}}=\SI{ 2.6e3}{\kilo\gram\per\meter\cubed}} ہے ۔
 
\جزوحصہء{ کلیدی تصور}
 نمونے میں ریت کی کثافت\عددی{\rho_{\text{\RL{ریت}}}}  سے مراد اکائی حجم میں کمیت ہے ؛ جو ریت کے تمام ذروں کی کل کمیت \عددی{m_{\text{\RL{ریت}}}} اور نمونے کے کل حجم \عددی{V_{\text{\RL{کل}}}}  کا تناسب  :
\begin{align}\label{مساوات_پیمائش_کثافت_ریت}
\rho_{\text{\RL{ریت}}}=\frac{m_{\text{\RL{ریت}}}}{V_{\text{\RL{کل}}}}
\end{align}
ہے۔ 

\موٹا{ حساب: }\quad
نمونے کا کل حجم \عددی{V_{\text{\RL{کل}}}}  درج ذیل ہے
\begin{align*}
V_{\text{\RL{کل}}}=V_{\text{\RL{دانے}}}+V_{\text{\RL{خلا}}}
\end{align*}
 مساوات \حوالہ{مساوات_پیمائش_خلا_تناسب}  میں \عددی{V_{\text{\RL{خلا}}}} ڈال کر  \عددی{V_{\text{\RL{ریت}}}} کے لیے حل کر کے ذیل حاصل ہو گا ۔
\begin{align}\label{مساوات_پیمائش_حجم_دانے}
V_{\text{\RL{دانے}}}=\frac{V_{\text{\RL{کل}}}}{1+e}
\end{align}
مساوات  \حوالہء{1.8} کے تحت ریت کے ذرات کی کل کمیت \عددی{m_{\text{\RL{ریت}}}} سلیکان ڈائی اکسائیڈ کی کثافت ضرب ریت کے ذرات کا کل حجم :
\begin{align}
m_{\text{\RL{ریت}}}=\rho_{\ce{SiO2}}V_{\text{\RL{دانے}}}
\end{align}
 ہوگا ۔ اس کو مساوات  \حوالہ{مساوات_پیمائش_کثافت_ریت}  میں  ڈال کر کے مساوات \حوالہ{مساوات_پیمائش_حجم_دانے}  سے \عددی{V_{\text{\RL{ریت}}}} ڈال کر   ذیل حاصل ہوگا ۔
\begin{align}
\rho_{\text{\RL{ریت}}}=\frac{\rho_{\ce{SiO2}}}{V_{\text{\RL{کل}}}}\,\frac{V_{\text{\RL{کل}}}}{1+e}
\end{align}
فاصل قیمت \عددی{e = 0.80} اور    \عددی{\rho_{\ce{SiO2}}=\SI{2.600e3}{\kilo\gram\per\meter\cubed}} پر کر کے ہم دیکھتے ہیں کہ رقیق کاری اس صورت ہوگی جب ریت کی کثافت درج ذیل سے کم ہو۔
\begin{align*}
\rho_{\text{\RL{}}}=\frac{\SI{2.600e3}{\kilo\gram\per\meter\cubed}}{1.80}=\SI{1.4e3}{\kilo\gram\per\meter\cubed}
\end{align*} 
  رقیق   کاری میں عمارت کئی میٹر زمین میں دھنس سکتی  ہے۔
\انتہا{مثال}
