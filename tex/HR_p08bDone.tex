%HR_p08b
%edited. figures and external reference pending
\حصہء{نظر ثانی اور خلاصہ }
\جزوحصہء{طبیعیات میں پیمائش }
طبیعی  مقادیر  کی پیمائش پر طبیعیات مبنی ہے۔ کچھ طبیعی مقادیر (مثلاً لمبائی، وقت، اور کمیت) \اصطلاح{   اساسی  مقدار  }  منتخب کیے گئے؛ ہر ایک کی تعریف  \اصطلاح{ معیار } کے مطابق کی گئی  اور اس کو پیمائش کی  \اصطلاح{اکائی }(مثلاً 
\عددی{\si{\meter}} ،  \عددی{\si{\second}}، اور \عددی{\si{\kilo\gram}}) مختص کی گئی۔  دیگر طبیعی  مقادیر  کی تعریف ان اساسی  مقدار اور ان کے معیار اور اکائیوں کی صورت میں کی جاتی ہے ۔ 

\جزوحصہء{بین الاقوامی اکائی }
اس کتاب میں بین الاقوامی اکائی  \عددی{(SI)}  استعمال کی گئی ۔ جدول  \حوالہء{1.1} میں دکھائی گئی تین طبیعی  مقادیر ابتدائی بابوں میں استعمال کی جائیں گی۔ بین الاقوامی معاہدوں کے تحت  اساسی مقداروں کے معیار طے کیے گئے ، جو ہر ایک کے لیے قابل رسائی اور غیر تغیر ہیں۔ اساسی مقدار اور ان سے اخذ دیگر مقادیر  کی تمام طبیعی پیمائشیں انہی معیار کے تحت کی جاتی ہے۔ جدول  \حوالہء{1.2} میں پیش علامتیں اور سابقے استعمال کر کے پیمائشی ترقیم  کی سادہ صورت حاصل ہوتی  ہے۔ 

\جزوحصہء{اکائیوں کی باہم  تبدیلی }
اکائیوں کی تبدیلی  \ترچھا{زنجیری  طریقے } سے  جا سکتی ہے، جس میں اصل مواد   کو یک بعد دیگرے تبادلی ضربیوں  سے، جنہیں  ایک \عددی{(1)} کے روپ میں لکھا گیا ہو، ضرب دے  کر ، اکائیوں سے الجبرائی مقادیر  کی طرح  نپٹا جاتا ہے حتٰی کہ درکار اکائیاں رہ جائیں۔

\جزوحصہء{لمبائی }
وہ فاصلہ ہے جو انتہائی معین وقتی وقفے  کے دوران  بصری شعاع    طے کرتی ہے، میٹر کی تعریف  ہے۔

\جزوحصہ{وقت }
سیکنڈ کی تعریف  سیزیم \عددی{133}  جوہر سے خارج شعاع  کی صورت  میں کی جاتی ہے۔ معیار برقرار رکھتی  تجربہ  گاہوں میں موجود جوہری گھڑیوں کے  صحیح   وقتی اشارے پوری دنیا   میں نشر کیے جاتے ہیں۔

\جزوحصہء{ کمیت }
پیرس شہر کے قریب رکھے گئے  پلاٹینم  و  اریڈیم کمیتی معیار ،  کلوگرام  کی  تعریف  ہے۔ جوہری پیمانہ پر پیمائش کے لیے جوہری کمیتی  اکائی استعمال کی جاتی ہے جس کی تعریف کاربن  \عددی{12}  جوہر کی صورت میں کی جاتی ہے۔

\جزوحصہء{کثافت }
کسی بھی چیز کی کثافت  \عددی{\rho} سے مراد اکائی حجم میں اس کی کمیت ہے۔ 
\begin{align*}
\rho=\frac{m}{V} \tag{\arabicdigits{\setlatin1.8}}
\end{align*}
%======================================
\حصہء{سوالات }
\جزوحصہء{لمبائی اور دیگر اشیاء کی پیمائش}
%----------------------
%Q1 p8
\ابتدا{سوال}
زمین تخمیناً ایک کرہ  ہے جس کا رداس   \عددی{\SI{6.37e6}{\meter}} ہے۔اس کا   (ا)  محیط کلومیٹر میں،  (ب)  سطحی رقبہ مربع   مربع کلو میٹر میں،  اور  (ج) حجم  کعبی کلومیٹر  میں کتنا ہے؟ 
\انتہا{سوال}
%------------------------------
\ابتدا{سوال}
اشاعت کاری میں لمبائی کی مستمل اکائی\ترچھا{   نقطہ } کہلاتی  ہے  ،  جو  انچ کے   \عددی{\tfrac{1}{72}} حصے کے برابر  ہے۔\ترچھا{       مربع  نقطہ } کی صورت میں \عددی{0.1} مربع انچ  لکھیں۔ 
\انتہا{سوال}
%-------------------------
\ابتدا{سوال}
ایک مائیکرو میٹر  \عددی{(\SI{1}{\micro\meter})}  کو عموماً \ترچھا{  مائیکران } کہتے ہیں۔  (ا) کتنے مائیکران   \عددی{\SI{1}{\kilo\meter}} کے برابر ہیں؟ (ب) سنٹی میٹر کا کتنا حصہ \عددی{\SI{1}{\micro\meter}}  ہو گا؟  (ج) کتنے مائیکران ایک گز کے برابر ہوں گے؟
\انتہا{سوال}
%-----------------------------
%Q4
\ابتدا{سوال}
اس کتاب میں فاصلے \ترچھا{نقطہ} اور \ترچھا{  پیکا  } اکائی میں رکھے گئے ہیں: \عددی{12} نقطے \عددی{1} پیکا کے برابر ہے، اور  \عددی{6} پیکا    \عددی{1} انچ کے برابر۔  اگر کتاب میں ایک
 شکل   \عددی{\SI{0.80}{\centi\meter}} غلط رکھی گئی ہو، تب یہ  (ا) پیکا اکائیوں میں اور  (ب) نقطہ اکائیوں میں کتنی غلط رکھی گئی ہے؟ 
\انتہا{سوال}
%-------------------------------
%Q5 p08
\ابتدا{سوال}
ایک مقابلے میں گھوڑے \عددی{4.0}  فرلانگ  دوڑ لگا کر طے کرتے ہیں۔ اس فاصلے کو  (ا) عصا  اور  (ب) زنجیر  کی صورت میں لکھیں۔ (ایک فرلانگ \عددی{\SI{201.168}{\meter}} کے برابر ہے۔ ایک عصا \عددی{\SI{5.0292}{\meter}} اور ایک زنجیر \عددی{\SI{20.117}{\meter}} کے برابر ہے)
\انتہا{سوال}
