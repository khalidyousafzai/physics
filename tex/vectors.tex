\باب{سمتیات}

%not edited
%page 57
%HR_p57    
%Module 3.1 سمتیات اور ان کے اجزاء 
%Q1  
\ابتدا{سوال}
 ایک سمتیہ جس کا قدر  \عددی{\SI{7.3}{\meter}}   ہے مثبت x محور کے رخ سے گھڑی کی سوئی کے مخالف رخ  \عددی{\SI{250}{\degree}}   پر x y مستوی میں پایا جاتا ہے،  الف اس کا x جزو اور ب  وائی جزو تلاش کریں۔   
\انتہا{سوال}
 %----------------------------  
\ابتدا{سوال}
 سمتیہ ہٹاو r کا قدر  \عددی{\SI{15}{\meter}}   ہے اور یہ x y  مستوی میں زاویہ  \عددی{\theta=\SI{30}{\degree}}  کہ رخ ہے، شکل 3.26 دیکھیں اس سمتیہ کے الف x جزو اور  ب y جزو تلاش کریں۔  
\انتہا{سوال}
 %----------------------------  
\ابتدا{سوال}
  سمتیہ a کا x جزو -  \عددی{\SI{25}{\meter}}  اور y جزو +  \عددی{\SI{40}{\meter}}  ہے۔  الف سمتیہ a کا قدر کتنا ہے؟  ب سمتیہ a کے رخ اور محور x کے مثبت رخ کے بیچ زاویہ کتنا ہے؟   
\انتہا{سوال}
 %----------------------------  
\ابتدا{سوال}
  درج ذیل زاویوں کو ریڈیئن میں بیان کریں: الف  \عددی{\SI{20}{\degree}} ، ب \عددی{\SI{50}{\degree}} ، ج \عددی{\SI{100}{\degree}} ۔ درج ذیل زاویوں کو درجات  کی صورت میں پیش کریں: د 0.330 ریڈیئن،  ح 2.10 ریڈیئن، و 7.70 ریڈیئن۔  
\انتہا{سوال}
 %----------------------------  
\ابتدا{سوال}
  ایک بحری  جہاز شمال کے رخ  \عددی{\SI{120}{\kilo\meter}}  دور نقطہ کی جانب پہنچنا چاہتا ہے۔ سفر کے آغاز سے پہلے ہی ایک غیر متوقع آندھی اس کو نقطہ آغاز سے مشرق جانب  \عددی{\SI{100}{\kilo\meter}}   دور دکھیلتا ہے۔ اس جہاز کو اختتامی نقطہ پر پہنچنے کے لیے الف کتنا فاصلہ طے کرنا ہوگا طے کرنا ہوگا اور (ب) اسے کس رخ سفر کرنا ہوگا؟  
\انتہا{سوال}
 %----------------------------  
\ابتدا{سوال}
 شکل 3.27 میں ایک بھاری مشین کو افقی  رخ سے زاویہ  \عددی{\theta=\SI{20}{\degree}} پر رکھے گئے تختے پر  \عددی{d=\SI{12.5}{\meter}} فاصلے تک گھسیٹا جاتا ہے۔ اس مشین کو (الف) انتصابی روح اور  (ب)  افقی  رخ کتنا دور منتقل کیا گیا؟   
\انتہا{سوال}
 %----------------------------  
\ابتدا{سوال}
 ایک ہٹاو جس کا قدر  \عددی{\SI{3}{\meter}}  ہے اور دوسرا ہٹاو جس کا قدر  \عددی{\SI{4}{\meter}}  ہے پر غور کریں۔ دکھائیں کہ ان ہٹاو سمتیات کو استعمال کرتے ہوئے  (الف)  \عددی{\SI{7}{\meter}}  ، (ب)  \عددی{\SI{1}{\meter}} ، اور (ج) \عددی{\SI{5}{\meter}} قدر کے ہٹاو حاصل کیے جا سکتے ہیں۔ Module 3.2  اکائی سمتیات، سمتیات کی جمع بذریعہ اجزاء   
\انتہا{سوال}
 %----------------------------  
\ابتدا{سوال}
 ایک شخص  \عددی{\SI{3.1}{\kilo\meter}} شمال کی طرف چلنے کے بعد  \عددی{\SI{2.4}{\kilo\meter}} مغرب اور آخر  میں  \عددی{\SI{5.2}{\kilo\meter}}  جنوب کے رخ چلتا ہے۔ (الف) اس کے حرکت کو ظاہر کرنے کے لیے سمتی نقشہ بنائیں۔ ایک پرندہ اس نقطہ آغاز سے سیدھا نقطہ اختتام تک اڑتے ہوئے  (ب) کتنا فاصلہ طے کرے گا اور  (ج) کس رخ طے کرے گا؟   
\انتہا{سوال}
 %----------------------------  
\ابتدا{سوال}
 درج ذیل دو سمتیات دیے گئے ہیں
\begin{align*} 
   a=(\SI{4}{\meter})i - (\SI{3}{\meter})j + (\SI{1}{\meter})k 
\end{align*}
    اور 
\begin{align*}
     b=(\SI{-1}{\meter})i + (\SI{1}{\meter})j + (\SI{4}{\meter})k 
\end{align*} 
   اکائی سمتیہ علامتیت میں  (الف)  \عددی{\vec{a}+\vec{b}} ، (ب)  \عددی{\vec{a}-\vec{b}} اور (ج) ایک تیسرا سمتیہ  \عددی{c}  تلاش کریں جہاں  \عددی{\vec{a}-\vec{b}+\vec{c}=0}  ہے۔  
\انتہا{سوال}
 %----------------------------   
\ابتدا{سوال}
 ہٹاو  \عددی{c}  اور  \عددی{d} کے میٹروں میں اجزاء \عددی{c_x=7.4} ،  \عددی{c_y=-3.8}  ، \عددی{c_z=-6.1} ؛  \عددی{d_x=4.4} ، \عددی{d_y=-2.0} ، \عددی{d_z=3.3} ہیں۔ ان ہٹاو کہ مجموعہ  \عددی{\vec{r}}  کے  (الف) \عددی{x}، (ب) \عددی{y}، اور  (ج) \عددی{z} اجزاء  تلاش کریں  
\انتہا{سوال}
 %----------------------------   
\ابتدا{سوال}
  (الف) اگر  \ \عددی{\vec{a}=(\SI{4}{\meter})\hat{i}+(\SI{3}{\meter})\hat{j}}  اور  \عددی{\vec{b}=(\SI{-13}{\meter})\hat{i}+(\SI{7}{\meter})\hat{j}}  ہوں تب اکائی سمتیا علامتیت میں مجموعہ  \عددی{a+b} کیا ہوگا؟ اس مجموعے کا  (ب) قدر اور  (ج) رخ کیا ہوگا؟  
\انتہا{سوال}
 %----------------------------   
\ابتدا{سوال}
   ایک گاڑی کو مشرق کی طرف  \عددی{\SI{50}{\kilo\meter}}  ، اس کے بعد شمال کی طرف   \عددی{\SI{30}{\kilo\meter}} اور آخر میں شمال سے مشرق جانب   \عددی{\SI{30}{\degree}} کے رخ \عددی{\SI{25}{\kilo\meter}} چلایا جاتا ہے۔ اس کا سمتی نقشہ بنائیں۔ ابتدائی نقطہ سے گاڑی کی کل ہٹاو کا (الف) قدر اور  (ب) زاویہ تلاش۔   
\انتہا{سوال}
 %----------------------------   
%Q13
\ابتدا{سوال}
ایک شخص آپنے موجودہ مقام سے   \عددی{\SI{3.4}{\kilo\meter}} دور شمال سے مشرق جانب  \عددی{\SI{35}{\degree}} کے رخ مقام پر پہنچنا چاہتا ہے۔ تاہم اس کو مجبوراً ایسی گلیوں سے گزرنا ہوگا جو مشرق سے مغرب یا شمال سے جنوب ہیں۔ یہ شخص کتنا کم سے کم فاصلہ طے کر کے اس مقام تک پہنچ سکتا ہے؟   
\انتہا{سوال}
 %----------------------------   
\ابتدا{سوال}
 ہموار سہرا میں \عددی{xy} محددی نظام کے مبدا  سے آغاز کرتے ہوئے  \عددی{xy} محدد (مائنس 14 میٹر کم 30 میٹر) (\عددی{\SI{-140}{\meter}}, \عددی{\SI{30}{\meter}}) کہ مقام کو پہنچنا چاہتے ہیں۔ آپ کو صرف چار مرتبہ سیدھ میں چلنے کی اجازت ہے۔ آپ کی حرکت کے  \عددی{x} اور  \عددی{y}  اجزاء میٹروں میں بالترتیب درج ذیل ہیں:   (\عددی{60} اور \عددی{20}) ، اس کے بعد   (\عددی{-70} اور \عددی{bx}) ، اس کے بعد   (\عددی{c} اور \عددی{20-}) ، اور آخر میں  (\عددی{60-} اور \عددی{70-})۔  بتائیں  (الف) جوز  \عددی{bx} اور  (ب) جوز  \عددی{cy}  کیا ہوں گے؟ مجموعی ہٹاو کا  (ج) قدر  اور  (ح) مثبت  \عددی{x}  محور کے لحاظ سے زاویہ کیا ہوگا؟   
\انتہا{سوال}
 %----------------------------   
\ابتدا{سوال}
 شکل  \عددی{3.28} میں دکھائے گئے سمتیات  \عددی{a}  اور  \عددی{b} دونوں کے قدر \عددی{\SI{10}{\meter}} ہیں جبکہ ان کے زاویات  \عددی{\theta_1=\SI{30}{\degree}}  اور  \عددی{\theta_2=\SI{105}{\degree}} ہیں۔ ان کے سمتی مجموعہ  \عددی{r} کے  (الف)  \عددی{x} اور  (ب) \عددی{y} اجزاء تلاش کریں۔  (ج) سمتیہ  \عددی{r} کا قدر اور (ح) مثبت  \عددی{x}  محور کے رخ کے ساتھ  \عددی{r}  کا زاویہ تلاش کریں۔  
\انتہا{سوال}
 %----------------------------   
\ابتدا{سوال}
  ہٹاو سمتیات   \عددی{ \vec{a}=(\SI{3}{\meter})\hat{i} + (\SI{4}{\meter})\hat{j}}    اور     \عددی{ \vec{b}=(\SI{5}{\meter})\hat{i} + (\SI{-2}{\meter})\hat{j}}    کے لئے  (الف) اکائی سمتیہ علامتیت میں، اور  (ب) قدر  اور  (ج) سمتیہ  \عددی{i}  کے لحاظ سے زاویہ کی صورت میں  \عددی{a+b}  بیان کریں۔ اسی طرح  (د) اکائی سمتیہ علامتیت میں، اور  (ح)قدر  اور  (ط) زاویہ کی صورت میں  \عددی{b-a}  بیان کریں۔  
\انتہا{سوال}
 %----------------------------   
\ابتدا{سوال}
تین سمتیات  \عددی{a} ،  \عددی{b} ، اور  \عددی{c} مستوی  \عددی{xy} میں پائے جاتے ہیں اور ہر ایک کا قدر  \عددی{\SI{50}{\meter}} ہے۔ مثبت  \عددی{x} محور کے رخ کے لحاظ سے ان کے رخ بالترتیب  \عددی{\SI{30}{\degree}} ،  \عددی{\SI{195}{\degree}} اور  \عددی{\SI{315}{\degree}} ہیں۔ سمتیہ  \عددی{\vec{a}+\vec{b}+\vec{c}} کا  (الف) قدر  اور  (ب) زاویہ  کیا ہوگا اور سمتیہ  \عددی{\vec{a}-\vec{b}+\vec{c}}  کا  (ج) قدر  اور  (ح) زاویہ کیا ہوگا؟ ایک ایسے چوتھے سمتیہ  \عددی{\vec{d}}  کا  (و) قدر  اور  (ز) زاویہ  کیا ہوگا جو \عددی{(\vec{a}+\vec{b}) - (\vec{c}+\vec{d}) = 0}  کو مطمئن کرتا ہو؟  
\انتہا{سوال}
 %----------------------------   
\ابتدا{سوال}
مجموعہ  \عددی{\vec{A}+\vec{B}=\vec{C}} میں سمتیہ سمتیہ  \عددی{\vec{A}} کا قدر  \عددی{\SI{12}{\meter}}  اور مثبت  \عددی{x}  رخ سے خلاف  گھڑی  زاویہ  \عددی{\SI{40}{\degree}}  ہے، جبکہ سمتیہ سی کا قدر   \عددی{\SI{15}{\meter}} اور منفی  \عددی{x} رخ سے خلاف  گھڑی زاویہ   \عددی{\SI{20}{\degree}} ہے۔ سمتیہ  \عددی{\vec{B}} کا  (الف) قدر  اور  (ب) مثبت \عددی{x} محور کے لحاظ سے زاویہ کیا ہوگا؟  
\انتہا{سوال}
 %---------------------------- 
 %Q19  
\ابتدا{سوال}
 یک باغیچہ میں  \عددی{\SI{1}{\meter}} اطراف کے چوکور خانوں کا شطرنج کی کھیل کا میدان بنایا جاتا ہے ایک گھوڑا درج ذیل قدم لیتا ہے: \عددی{1} دو چوکور  (قدم) آگے، ایک چوکور دائیں؛   \عددی{2} دو چوکور بائیں، ایک چوکور آگے؛   \عددی{3} دو چوکور آگے، ایک چوکور بائیں۔ آگے چلنے کے رخ کے لحاظ سے گھوڑا کے مجموعی ہٹاو کا  (الف) قدر  اور  (ب) زاویہ  کیا ہوگا؟  
\انتہا{سوال}
 %----------------------------
