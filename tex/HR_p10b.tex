
آپ خط استوا پر پرسکون سمندر کے کنارے ریت پر لیٹ کر غروب ہوتے سورج کو دیکھ رہے ہیں جیسے ہی سورج کا بالائی سر سمندر کے پیچھے غروب ہوتا ہے اپ گھڑی میں وقت دیکھ کر قلم بند کرتے ہیں اس کے بعد اپ اتنی بلندی پر کھڑے ہوتے ہیں کہ اپ کی انک 
\(H=\SI{1.70}{\meter}\) 
زیادہ اونچائی پر ہوتی ہے، اور دوبارہ سورج کے بالائی سر کو غروب ہوتے دیکھ کر وقت قلم بند کرتے ہیں۔ کل دورانیہ
\(t=\SI{11.1}{\second}\)
ملتا ہے۔ زمین کا رداس 
\(r\) 
کتنا ہے؟ 
\انتہا{سوال}
برائے حصہ 
\(1.3\) 
کمیت 
\ابتدا{سوال} 
سوال 20 
سن 
\(1992\)
میں شیشے کی سب سے بڑی بوتل بنائی گئی جس کا حجم
\(193\)
گیلن تھا۔ 
(الف) یہ 
 \(\SI{1.0}{\mio\centi\meter\cubic}\)
سے کتنا کم ہے؟ 
(ب) 
اگر اس بوتل کو 
\(\SI{1.8}{\gram/min}\)
کی شرا سے پانی سے بھرا جائے، تو کتنی دیر لگے گی؟ پانی کی کثافت 
\(\SI{1000}{\kilo\gram/\meter**3}\)
ہے۔ 
\انتہا}سوال} 

\ابتدا{سوال} 
سوال 21 
زمین کی کمیت 
\(\SI{5.98×10*24}{\kilo\gram/\meter*3}\)
ہے۔ زمین کے ایٹموں کی اوسط کمیت تقریبا 
\(40\)
\(u\)
ہے۔ زمین میں کل کتنے ایٹم ہیں؟ 
\انتہا{سوال} 

\ابتدا{سوال} 
سوال 22 
سونے کی کثافت 
\(\SI{19.32}{\gram/\centi\meter**3}\)
ہے، اور یہ سب سے زیادہ تار پذیر دات ہے جس کو دبا کر باریک پتہ بنایا جا سکتا ہے یا کھینچ کر باریک تار بنائی جا سکتی ہے۔
(الف) اگر 
\(\SI{27.63}{\gram}\)
سونے سے 
\(\SI{1.000}{\micro\meter}\) 
موٹی چادر بنائی جائے، تو اس چادر کا رقبہ کتنا ہوگا؟ 
(ب) اس کے برعکس اگر اس سے 
\(\SI{2.500}{\micro\meter}\) 
رداس کی تار بنائی جائے، تو اس تار کی لمبائی کتنی ہوگی؟ 
\انتہا{سوال} 

\ابتدا{سوال} 
سوال 23 
(الف) پانی کی کثافت ٹھیک 
\(\SI{1}{\gram/\centi\meter**3}\)
فرض کرتے ہوئے، 
\(\SI{1}{\cubic\meter}\) 
پانی کی کمیت تلاش کریں۔ 
(ب) اگر ایک برتن سے 
\(\SI{5700}{\meter**3}\)
پانی کی نکاسی 
\(10.0\)
گھنٹوں میں ہوتی ہو، تو نکاسی کمیت کی شرح 
\(\SI{\kilo\gram/\second}\)
کتنا ہوگا؟ 
\انتہا{سوال} 

\ابتدا{سوال} 
سوال 24 
ساحل سمندر پر ریت تقریباً کروی سیلیکان ڈائی اکسائیڈ کے دانوں پر مشتمل ہے جن کی اوسط رداس 
\(\SI{50}{\micro\meter}\)
اور کثافت 
\(\SI{2600}{\kilo\gram/\meter**3}\) 
ہے۔ کتنی کمیت کہ ریتیلی دانوں کا کل سطحی رقبہ (تمام انفرادی کروں کا مجموعی رقبہ) 
\(\SI{1.00}{\meter}\)
کنارے والے موقب  کے ستھئ رقبہ کے برابر ہوگا؟ 
\انتہا{سوال}

\ابتدا{سول}
سوال 25 
تیز بارش کے دوران پہاڑی کا ایک حصہ جس کی اف لمبائی
\(\SI{2.5}{\kilo\meter}\)
، ڈھلوان کے ہمراہ لمبائی 
\(\SI{0.8}{\kilo\meter}\)
، اور موٹائی 
 \(\SI{2}{\\meter}\)
ہے نیچے گرتا ہے۔ یہ مٹی وادی میں
 \(\SI{0.4}{\kilo\meter}\)×\(\SI{0.4}{\kilo\meter}\)
کے رقبہ پر یکساں تقسیم ہو کر رکتی ہے۔ مٹی کی کثافت 
\(\SI{1900}{\kilo\gram/\meter**3}\)
لیں۔ وادی کے 
\(\SI{4}{\meter**2}\)
رقبہ پر مٹی کی کمیت کتنی ہوگی؟
 \انتہا{سوال}

\ابتدا{سوال} 
سوال 26 
\(14\)
ابر بادل کے  
 \(\SI{1}{\centi\meter**3}\)
میں عموما 
\(50\)
تا 
\(500\)
پانی کے قطرے پائے جاتے ہیں، جن کا عمومی رداس 
\(\SI{10}{\micro\meter}\)
ہوتا ہے۔ اس سات کے لیے، درج ذیل کی کمتر اور بلندتر قیمتیں کیا ہوںگی؟ 
(الف) نلقی شکل کے
%نلقی شکل کے_____ 
جس کا  رداس 
\(\SI{1}{\kilo\meter}\) 
اور قد 
 \(\SI{3}{\kilo\meter}\)
ہو میں کتنے 
\(\SI{\meter**3}\)
پانی پایا جائے گا؟ 
(ب) یہ پانی ایک 
\(litre\)
کے کتنے بوتل بھر سکتا ہے؟ 
(ج) پانی کی کثافت 
\(\SI{1000}{\kilo\gram/\meter**3}\) 
ہے۔ اس بادل میں پائے جانے والے پانی کی کمیت کتنی ہے؟ 
\انتہا{سوال}

\ابتدا{سوال} 
سوال 27 
لوہے کی کثافت 
\(\SI{7.87}{\gram/\centi\meter**3}\)
ہے جبکہ لوہے کے جوہر کی کمیت 
\(\SI{9.27×10**-26}{\kilo\gram}\)
ہے۔ فرض کریں ایٹم کروی ہے اور ان کے بیچ کوئی فاصلہ نہیں پایا جاتا۔ 
(الف) ایک لوہے کے جوہر کا حجم اور 
(ب) قریبی جوہر کے مراکز کے بیچ فاصلہ کیا ہے؟ 
\انتہا{سوال}

\ابتدا{سوال} 
سوال 28 
جوہر کے ایک مول سے مراد 
\(6.02×10**23\)
\(atoms\)
جوہر ہیں۔ ایک موٹی گھریلو بلی میں، مقدار کے قریبی رتبہ تک، جوہروں کے کتنے مول پائے جاتے ہیں؟ ہائڈروجن جوہر، اکسیجن جوہر، اور کاربن جوہر کی کمیتیں بالترتیب 
\(1.0\)
\(u\)
، 
\(16\)
\(u\)
، اور 
\(12\)
\(u\) 
ہیں۔ 
\انتہا{سوال} 

\ابتدا{سوال} 
سوال 29 
آپ ملیشیا کے مویشی منڈی میں ایک بیل خریدتے ہیں جس کا علاقائی اکائیوں میں وزن 
\(28.9\)
\(piculs\)
ہے: ایک
\(picul\)=\(100\)\(gins\) 
، کما ایک 
\(gin\)=\(16\)\(tahils\) 
، ایک 
\(tahil\)=\(10\)\(chees\)
، اور ایک 
\(chee\)=\(10\)\(hoons\)
۔ ایک 
\(hoon\)
کی کمیت 
\(\SI{0.3779}{\gram}\)
ہے۔ اس بیل کی کمیت 
\(\SI{\kilo\gram}\) 
میں کتنی ہے؟ 
\انتہا{سوال} 

\ابتدا{سوال} 
سوال 30 
پانی ایک ایسے برتن میں انڈیلا جاتا ہے جس سے معمولی معمولی پانی رستہ ہے۔ وقت 
\(t\)
کے لحاظ سے پانی کی کمیت 
\(m=5.00t**0.8 - 3.00t + 20.00\)
ہے، جہاں 
\(t is greater than=0\)
، 
\(m\)
کی اکائی 
\(\SI{\gram}\)
، اور 
\(t\) 
کی اکائی 
\(\SI{\second}\)
ہے۔ 
(الف) پانی کی کمیت کس وقت اعظم ہے، اور 
(ب) یہ اعظم کمیت کتنی ہے؟ کن کمیت میں تبدیلی کی شرح، 
\(\SI{\kilo\gram/\minute}\)
میں 
(ج) 
\(t\)=\(\SI{2.00}{\second}\)
اور 
(د) 
 \(t\)=\(\SI{5.00}{\second}\)
پر کتنی ہے؟ 
\انتہا{سوال}
\ابتدا{سوال} 
سوال 31 
ایک سیدھا کھڑا برتن جس کے تح کا رقبہ 
\(\SI{14}{\centi\meter}\)×\(\SI{17}{\centi\meter}\) 
ہے کو مٹھائی سے بھرا جاتا ہے۔ ہر ایک مٹھائی کی کمیت 
\(\SI{0.0200}{\gram}\)
اور حجم 
\(\SI{50.0}{\milli\meter**3}\)
ہے۔ مٹائیوں کے بیچ خالی جگہ کو نظر انداز کریں۔ اگر برتن میں مٹائیوں کی بلندی کا شرعہ  
\(\SI{0.250}{\centi\meter/\second}\)
ہو، تو (
\(\SI{\kilo\gram/\minute}\)
) کی مٹائیوں کمیت برتن میں کس شرح سے بڑھتی ہے؟ 
\انتہا{سوال} 
اضافی سوالات 
\ابتدا{سوال} 
سوال 32 
حقیقی گھر کے لحاظ سے 
\(1:12\) 
پیمانہ سے گڑیا کا گھر بنایا جاتا ہے (یعنی گڑیا کے گھر کا ہر ضلع حقیقی کر کے مطابقتی ضلعے کا 
\(1÷12\) 
ہوگا) جبکہ حقیقی گھر کے 
\(1:144\)
کے کے پیمانہ سے ایک چھوٹا گھر تعمیر کیا جاتا ہے۔ فرض کریں ایک حقیقی گھر (شکل 
\(1.7\)
) سامنے سے 
\(\SI{20}{\meter}\)
لمبا، 
\(\SI{12}{\meter}\) 
گھرا، اور 
\(\SI{6}{\meter}\)
اونچا ہے، جبکہ اس کا چھت ڈھلوانی ہے جو 
\(\SI{3.0}{\meter}\)
بلندی رکھتا ہے۔ 
\(\SI{\meter\cubed}\)
میں مطابقتی 
(الف) گڑیا گھر 
اور 
(ب) چھوٹے گھر کا حجم کتنا ہے؟ 
\انتہا{سوال}
