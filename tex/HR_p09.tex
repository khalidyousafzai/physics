%HR-p09
\ابتدا{سوال}
سوال 6
جدول 
\(1.6\) 
میں 
\(\SI{\milli\gram}\)
 \(\SI{\centi\gram}\)
\(\SI{\deci\gram}\)
\(\SI{\gram}\)
\(\SI{\deca\gram}\)
\(\SI{\hecto\gram}\)
اور 
\(\SI{\kilo\gram}\)
کی کچھ رقوم درج ہیں۔ 
(الف) جدول مکمل کریں۔ 
(ب)
\(\SI{55}{\milli\gram}\) 
کتنے 
\(\SI{\kilo\gram}\)
کے برابر ہوگا۔ 
(ج) 
\(\SI{12}{\centi\meter\cubed}\)
کا حجم کتنے 
\(litre\)
کے برابر ہوگا۔ 
\انتہا{سوال} 

\ابتدا{سوال} 
سوال 7 
ما کوائی معمار پانی کے حجم کو عموماً 
\(\SI{\kilo\meter\foot}\)
میں ناپتے ہیں، جس سے مراد ایک مربع
\(\SI{\kilo\meter}\)
کے رقبے پر ایک 
\(foot\)
گہرا پانی ہے۔ ایک شہر جس کا رقبہ 
\(\SI{26}{\kilo\meter\squared}\)
ہے میں 
\(30\)
منٹ کا بارش 
 \(\SI{2}{\inch}\)
کا پانی برساتا ہے۔ اس شہر پر کتنے
 \(\SI{\kilo\meter\foot}\)
کا پانی برسا؟ 
\انتہا{سوال} 

\ابتدا{سوال} 
سوال 8 
ایک سڑک 
\(32\)\(miles\)
اور 
\(5\)
فرلانگ لمبی ہے۔ اس کی لمبائی 
\(\SI{\kilo\meter}\)
میں کتنی ہوگی؟ 
\انتہا{سوال} 

\ابتدا{سول}
سوال 9 
بہرے منجمد جنوبی 
\(Antarctica\) 
کی شکل تقریباً نیم دائری ہے (شکل 
\(1.5\)) 
جس کا رداس 
 \(\SI{2000}{\kilo\meter}\)
ہے۔ اس میں برف کی اوسط موٹائی 
\(\SI{3000}{\merer}\) 
ہے بحر منجمد جنوبی میں کتنے 
\(\SI{\cubic\centi\meter}\)
برف پایا جاتا ہے؟ (زمین کی سطح کو مستوی تصور کریں۔)
\انتہا{سوال}برائے حصہ 
\(1.2\)
وقت 
\ابتدا{سوال} 
سوال 10 بہت وسیع ممالک مثلاً روس میں مختلف مقامات پر گڑیوں کا وقت ایک دوسرے سے مختلف ہوتا ہے۔ خط تول بلد پر اوسطاً کتنے درجے چلنے کے بعد ایک گھنٹے کا فرق پایا جائے گا؟ (اشارہ: زمین 
\(24\)
گھنٹے میں 
\(\SI{360}{\degree}\)
گھومتی ہے۔) ایک 
 \(\SI{\degree}\)
خط تول بلد کتنے منٹ کے برابر ہوگا؟\انتہا{سوال}

\ابتدا{سوال} 
سوال 11 
فرانسیسی انقلاب کے بعد تقریباً 
\(10\) 
سال تک حکومت کوشش کرتی رہی کہ وقت کی پیمائش مزرب 
\(10\)
رکھی جائے؟ ایک ہفتہ میں 
\(10\)
دن، ایک دن میں 
 \(10\)
گھنٹے، ایک گھنٹہ میں 
\(100\)
منٹ، اور ایک منٹ میں 
\(\SI{100}{\second}\)
رکھے گئے فل سٹاپ 
(الف) فرانسیسی اشاریہ ہفتہ اور معیاری ہفتہ کا تناسب، اور 
(ب) فرانسیسی اشاریہ سیکنڈ اور معیاری سیکنڈ کا تناسب کیا ہے؟ 
\انتہا{سوال} 

\ابتدا{سوال} 
سوال 12 
دنیا کا تیز ترین بڑھتا پودا ہسپر یوکا ہے جو 
\(14\)
دنوں میں 
\(\SI{3.7}{\meter}\)
بڑا۔ اس پودے کے بڑھنے کی شرح 
\(\SI{\micro\meter/\second}\)
کتنی ہے؟ 
\انتہا{سوال} 

\ابتدا{سوال} 
سوال 13 
تین گھڑیاں الف، ب، اور پ مختلف رفتار سے چلتی ہیں اور یہ بیک وقت صفر نہیں دیتی۔ شکل 
\(1.6\)
میں بیک وقت ان کے وقت دکھائے گئے ہیں۔ (مثال کے طور پر جس لمحہ ب 
\(\SI{25}{\second}\)
دیتا ہے اس وقت پ 
\(\SI{92}{\second}\)
دیتا ہے۔) اگر دو واقعات گڑی 
(الف) پر 
\(\SI{600}{\second}\)
دور واقع ہو، یہ (الف) گھڑی 
\(b\)
پر اور 
(ب) گھڑی 
\(c\)
پر کتنے دور واقعہ ہوں گے؟ 
(ج) جس لمحہ گھڑی 
(الف) 
\(\SI{400}{\second}\)
دیتی ہے اس لمحہ گھڑی 
(ب) کیا دے گی؟ 
(د) جس وقت گھڑی 
 \(a\)
 \(\SI{15}{\second}\)
دیتی ہے، اس وقت گھڑی (ب) کیا دے گی؟ (صفر سے قبل وقت کو منفی تصور کریں۔)
\انتہا{سوال}\ابتدا{سوال}
سوال 14 
ایک درس (جو 50 منٹ کا ہے) تقریباً 
\(\SI{1}{\micro\century}\)
کا ہوگا۔
(الف) 
 \(\SI{1}{\micro\century}\)
کا دورانیہ منٹوں میں کتنا ہوگا؟ 
(ب) درج ذیل کلیہ فیصد فرق مساوی ہے اصل منفی تخمین تقسیم اصل ضربِ 
\(100\)
استعمال کرتے ہوئے تخمین میں فیصد فرق تلاش کریں۔ 
\انتہا{سوال} 

\ابتدا{سوال} 
سوال 15 
دو ہفتوں کا وقت کتنے 
\(\SI{\micro\second}\)
کا ہوگا؟ 
\انتہا{سوال} 

\ابتدا{سوال} 
سوال 16 
مایاری وقت کا دارومدار اس وقت جوہری گڑیوں پر ہے۔ اس سے بہتر معیار سیکنڈ نابض پر مبنی ہو سکتا ہے، جو نیوٹران ستارے ہیں (انتہائی ٹھوس ستارے جن میں صرف نیوٹران پائے جاتے ہیں)۔ ان میں سے کئی انتہائی زیادہ مستحکم شرح سے گھومتے ہیں، اور ہر چکر کے دوران ایک مرتبہ زمین پر 
%{زمین پر ___ ڈالتے ہیں}
ڈالتے ہیں (سمندر کے کنارے روشنی مینار کی طرح جو سمندری جہاز کو خطرے سے اگاہ کرتے ہیں)۔ نابز پی ایس ار
 \(1937+21\)
ایک ایسی مثال ہے؛ یہ ایک چکر 
\(\SI{1.55780644887275+-3}{\milli\second}\)
میں پورا کرتا ہے، جہاں اخر میں 
\(+-3\)
اخری اشاریہ میں عدم یقینیت دیتی ہے (اس کا 
\(\SI{+-3}{\milli\second}\)
ہرگز مطلب نہیں)۔ 
(الف) یہ نابز 
\(7.00\)
دنوں میں کتنے چکر کاٹتا ہے؟ 
(ب) یہ نابیز 
\(10\)
لاکھ مرتبہ ٹھیک کتنے وقت میں گھومتا ہے اور 
(ج) اس سے وابستہ عدم یقینیت کیا ہوگی؟ 
\انتہا{سوال}
