\باب{پیمائش}
\حصہء{طبیعیات کیا ہے؟}
سائنس و  انجینئری پیمائش اور موازنے  پر مبنی ہے۔ چیزوں کی پیمائش اور موازنے کے  لئے قواعد کی ضرورت پیش آتی ہے؛  پیمائش اور موازنہ کے  بُعد تعین کرنے کے لئے تجربات کا سہارا لینا  ہو گا۔ طبیعیات اور انجینئری کا ایک مقصد ان تجربات کی بناوٹ اور تجربہ کرنا ہے۔

\حصہء{چیزوں کی پیمائش}
طبیعیات میں ملوث مقداروں کی پیمائش کی طریقے جان کر ہم طبیعیات دریافت کرتے ہیں۔ ان مقداروں میں لمبائی، وقت، کمیت، درجہ حرارت، دباو، اور برقی رو شامل ہیں۔

ہم ہر طبیعی  مقدار کا موازنہ ایک  \اصطلاح{معیار }\فرہنگ{معیار}\حاشیہب{standard}\فرہنگ{standard} کے ساتھ کرکے  طبیعی مقدار  کو اس کی اکائیوں  میں ناپتے ہیں۔ اس مقدار کی ناپ کو ایک منفرد نام دیا جاتا ہے جسے  \اصطلاح{ اکائی }\فرہنگ{اکائی}\حاشیہب{unit}\فرہنگ{unit} کہتے ہیں۔ مثلاً ، لمبائی کی پیمائش   میٹر  \عددی{(\si{\meter\relax})}میں  کی جاتی ہے۔ معیار سے مراد،  مقدار کی ٹھیک ایک اکائی ہے۔ جیسا آپ دیکھیں گے لمبائی کا معیار،  جو ٹھیک ایک میٹر کے برابر ہے، اُس فاصلہ کو کہتے ہیں جو خلاء میں  ،  ایک مخصوص دورانیہ میں  ، شعاع طے کرتی  ہے۔ ہم اکائی اور اس کے معیار کی تعریف جیسا چاہیں کر سکتے ہیں۔ تاہم، ضروری ہے کہ دنیا کے باقی سائنسدان بھی اس تعریف کو معنی خیز اور قابل استعمال  مانیں۔

ایک معیار ، مثلاً لمبائی کا معیار ،  طے کرنے کے بعد ہمیں وہ طریقہ کار وضع  کرنا  ہو گی جس سے   ہر  لمبائی ، چاہے وہ ہائیڈروجن جوہر کا رداس ہو یا دور  کسی ستارے تک فاصلہ ، اس معیار کی صورت میں ظاہر کی جا سکے۔ ایسی ایک ترکیب فیتے کا استعمال ہے ؛  لمبائی کے معیار کو  فیتہ تخمیناً ظاہر کرتا ہے۔ بہرحال، بہت سے  موازنوں میں بلا واسطہ طریقے استعمال کیے  جائیں گے۔ مثلاً ،  جوہر کا رداس یا قریبی ستارے تک فاصلہ فیتہ استعمال کرکے نہیں ناپا جا سکتا۔

\موٹا{اساسی مقادیر}
 طبیعی مقادیر کی تعداد اتنی زیادہ ہے  کہ انہیں منظم کرنا  ایک  مسئلہ ہے۔  خوش قسمتی سے تمام  مقادیر غیر تابع نہیں ہیں؛ مثلاً ، رفتار درحقیقت لمبائی اور وقت کی  تناسب کو کہتے ہیں۔  بین الاقوامی متفقہ  معاہدے کے تحت چند طبیعی مقادیر،  مثلاً ،  لمبائی ،  کمیت، اور وقت  کو\اصطلاح{ اساسی مقادیر }\فرہنگ{اساسی مقادیر}\حاشیہب{base quantities}\فرہنگ{base quantities} منتخب کرکے صرف انہی کو معیار مختص کیے گئے۔   باقی  طبیعی مقادیر ان \قول{ اساسی مقادیر } اور  انہیں کے  معیار (جنہیں \اصطلاح{اساسی معیار }\فرہنگ{اساسی معیار}\حاشیہب{base standards}\فرہنگ{base standards}  کہتے ہیں) کی صورت میں ناپے جاتے ہیں۔ مثلاً،  اساسی مقادیر  لمبائی اور وقت اور انکے اساسی معیار کی  شکل میں \قول{رفتار } تعین کیا جاتا ہے۔ 

اساسی معیار  کا قابل رسائی اور    غیر متغیر  ہونا  لازمی ہے۔ اگر ہم بازو کی لمبائی کو معیار لمبائی تسلیم  کریں تب یہ قابل رسائی ضرور ہو گی، البتہ ہر شخص کے لئے یہ لمبائی مختلف ہوگی لہٰذا یہ غیر  متغیر نہیں ہے۔ سائنس و  انجینئری میں زیادہ سے زیادہ درستگی مطلوب ہونے کی  پیش نظر ہم اساسی  معیار کی   غیر متغیریت  پر  خصوصی توجہ  دیتے  ہیں۔  اس کے بعد اساسی معیار کی بہتر سے بہتر نقل بنا کر    ان لوگوں کو  فراہم  کرتے ہیں جنہیں ضرورت ہو۔


\حصہء{اکائیوں کا بین الاقوامی نظام} 
\سن {1971 } میں ناپ و  تول  کے عمومی اجلاس میں سات مقادیر  کو بطور اساسی مقدار منتخب کرکے بین الاقوامی نظام اکائی کے اساس چنے گئے۔ بین الاقوامی نظام اکائی  کو مختصراً  \قول{\عددی{\textup{SI}} نظام}  کہتے ہیں۔جدول  \حوالہ{جدول_پیمائش_اساسی_اکائیاں} میں تین اساسی مقدار لمبائی، کمیت ، اور وقت دکھائے گئے ہیں۔
\begin{table}[h!]
\caption{بین الاقوامی نظام اکائی کی تین اساسی مقادیر کی اکائیاں}
\label{جدول_پیمائش_اساسی_اکائیاں}
\centering
\begin{tabular}{r r c} 
\toprule
مقدار & اکائی کا نام & اکائی کی علامت\\ 
\midrule
لمبائی & میٹر & \si{\meter} \\
کمیت & کلوگرام &\si{\kilogram} \\
وقت & سیکنڈ & \si{\second} \\
\bottomrule
\end{tabular}
\end{table}
ان اکائیوں کی تعریف انسانی جسامت مد نظر رکھتے ہوئے کی گئی۔

کئی \اصطلاح{ مشتق اکائیوں }\فرہنگ{اکائی!مشتق}\حاشیہب{derived units}\فرہنگ{unit!derived}کی تعریف ان اساسی اکائیوں کی صورت میں کی جاتی ہے۔ مثلاً ، طاقت کی  \عددی{\textup{SI}}  اکائی ، جو \اصطلاح{واٹ }\فرہنگ{واٹ}\حاشیہب{watt}\فرہنگ{watt}   \عددی{(\si{\watt})} کہلاتی ہے، کی تعریف   کمیت، لمبائی ، اور وقت کی اساسی اکائیوں کی صورت میں کی جاتی ہے۔ یوں ، جیسا  آپ باب\حوالہء{ 7 } میں دیکھیں گے ، درج ذیل ہوگا:
\begin{align}
\text{\RL{1 واٹ}}=\SI{1}{\watt}=\SI{1}{\kilogram}\cdot \si{\meter\squared\per\second\cubed}
\end{align}
جہاں آخر میں  کلوگرام مربع میٹر فی مکعب سیکنڈ پڑھا جائے گا۔

بہت بڑی یا بہت چھوٹی مقادیر، جن سے ہمیں طبیعیات میں عموماً واسطہ ہو گا،  \اصطلاح{سائنسی ترقیم }\فرہنگ{ترقیم!سائنسی}\حاشیہب{scientific notation}\فرہنگ{notation!scientific} میں لکھی جاتی ہیں، جو دس کی طاقت استعمال کرتی ہے۔ یوں درج ذیل  ہو گا۔
\begin{align}
\SI{3560000000}{\meter} = \SI{3.56e9}{\meter} 
\end{align}
\begin{align}
\SI{0.000 000 492}{\second} = \SI{4.92e-7}{\second} 
\end{align}
کمپیوٹر  میں  سائنسی ترقیم  مزید مختصر لکھی جاتی ہے؛ مثلاً،   \عددی{3.56E9}   اور   \عددی{{4.92E-7}}  ، جہاں   \عددی{E}\قول{ دس کی طاقت  } ظاہر کرتا  ہے۔ کئی\اصطلاح{  حساب کار }\فرہنگ{حساب کار}\حاشیہب{calculator}\فرہنگ{calculator}  (کلکولیٹر)  مزید مختصر انداز استعمال کرتے ہوئے  \عددی{E} کو خالی جگہ سے ظاہر کیا جاتا ہے۔

ہم اپنی آسانی کے لئے بہت بڑی یا بہت چھوٹی پیمائش  جدول   \حوالہ{جدول_پیمائش_سابقے} میں  پیش سابقے استعمال کر کے  لکھتے ہیں۔
\begin{table}[h!]
\caption{بین الاقوامی نظام اکائی کے سابقے}
\label{جدول_پیمائش_سابقے}
\centering
\renewcommand{\arraystretch}{1.25}
\newcommand*{\isEmpty}{\relax}
\begin{tabular}{crl} 
\toprule
علامت & سابقہ &جزو ضربی\\
\midrule
$\si{\yotta\isEmpty}$ & یوٹا & $10^{24}$\\
$\si{\zetta\isEmpty}$ & زیٹا & $10^{21}$\\
$\si{\exa\isEmpty}$ & اکسا & $10^{18}$\\
$\si{\peta\isEmpty}$ & پٹا & $10^{15}$\\
$\si{\tera\isEmpty}$ & ٹیرا & $10^{12}$\\
$\si{\giga\isEmpty}$ &\موٹا{گیگا} & $10^{9}$\\
$\si{\mega\isEmpty}$ & \موٹا{میگا} & $10^{6}$\\
$\si{\kilo\isEmpty}$ & \موٹا{کلو} & $10^{3}$\\
$\si{\hecto\isEmpty}$ & ہکٹو & $10^{2}$\\
$\si{\deka\isEmpty}$ & ڈیکا & $10^{1}$\\
$\si{\deci\isEmpty}$ & ڈسی & $10^{-1}$\\
$\si{\centi\isEmpty}$ &\موٹا{سنٹی} & $10^{-2}$\\
$\si{\milli\isEmpty}$ & \موٹا{ملی} & $10^{-3}$\\
$\si{\micro\isEmpty}$ & \موٹا{مائیکرو} & $10^{-6}$\\
$\si{\nano\isEmpty}$ & \موٹا{نینو} & $10^{-9}$\\
$\si{\pico\isEmpty}$ & \موٹا{پکو}& $10^{-12}$\\
$\si{\femto\isEmpty}$ &فیمٹو & $10^{-15}$\\
$\si{\atto\isEmpty}$ & اٹو & $10^{-18}$\\
$\si{\zepto\isEmpty}$ & زپٹو & $10^{-21}$\\
$\si{\yocto\isEmpty}$ & یکٹو & $10^{-24}$\\
\bottomrule
\end{tabular}
\end{table}

جیسا آپ دیکھ سکتے ہیں ہر  سابقہ دس کی  ایک مخصوص طاقت ظاہر کرتا ہے، جو بطور جزو ضربی استعمال کیا جاتا ہے۔ بین الاقوامی نظام اکائی کے ساتھ   سابقہ منسلک کرنے سے مراد اس اکائی کو مطابقتی  جزو ضربی سے ضرب دینا ہے۔ یوں ہم کسی ایک مخصوص برقی طاقت کو
\begin{align}
 \text{\RL{واٹ}}\, 1.27\times 10^9 = \text{\RL{گیگا واٹ}} \, 1.27  = \SI{1.27}{\giga\watt}
\end{align}
یا کسی مخصوص وقتی دورانیہ کو  درج ذیل لکھ سکتے ہیں۔
\begin{align}
\text{\RL{سیکنڈ}} \, 2.35\times 10^{-9}  = \text{\RL{نینو سیکنڈ}} \, 2.35 
= \SI{2.35}{\nano\second}.
\end{align}
چند سابقے ، جو ملی  لٹر ، سنٹی میٹر، کلوگرام یا میگا بائٹ میں استعمال ہوتے ہیں ، سے آپ ضرور واقف ہوں گے۔

\حصہء{اکائی کی تبدیلی}
بعض اوقات طبیعی  مقداروں کی اکائی تبدیل کرنے کی ضرورت پیش آتی ہے۔ ہم اصل پیمائش کو\قول{  تبادلی جزو }، جو  ایک \عددی{(1)} کے برابر اکائیوں کی نسبت ہو گی، سے ضرب دیتے ہیں۔ مثلاً ، ایک منٹ اور ساٹھ سیکنڈ مماثل دورانیہ ظاہر کرتے ہیں ، لہٰذا درج ذیل ہو گا۔
\begin{align*}
\frac{\SI{1}{\minute}}{\SI{60}{\second}} = 1
\end{align*}
یا
\begin{align*}
\frac{\SI{60}{\second}}{\SI{1}{\minute}} = 1
\end{align*}

یوں  \عددی{(\SI{1}{\minute})\!/\!(\SI{60}{\second})}   یا \ \عددی{(\SI{60}{\second})\!/\!(\SI{1}{\minute})}  تناسب  بطور\اصطلاح{  تبادلی جزو }\فرہنگ{تبادلی جزو}\حاشیہب{conversion factor}\فرہنگ{conversion factor}استعمال کیا جا سکتا ہے۔ ہم ہرگز \(\frac{1}{60}=1\) یا \(60 = 1\) نہیں لکھ سکتے؛  ہر عدد اور اسکی اکائی کو اکٹھا رکھنا ہو گا۔

ایک \عددی{(1)}  سے ضرب دینے سے مقدار کی قیمت تبدیل نہیں ہوتی لہٰذا ہم جب  چاہیں تبادلی جزو  استعمال کر سکتے ہیں۔ ایسا کرتے ہوئے ہم غیر ضروری اکائیوں کو منسوخ کر سکتے ہیں۔ مثال کے طور پر دو منٹ  کو سیکنڈ  میں تبدیل کرتے ہوئے درج ذیل لکھا جائے گا۔
\begin{align}
\SI{2}{\minute} = (\SI{2}{\minute})(1) = (\SI{2}{\bcancel{\mathrm{\minute}}})(\frac{\SI{60}{\second}}{\SI{1}{\bcancel{\mathrm{\minute}}}}) = \SI{120}{\second}
\end{align}
اگر تبادلہ جزو ضرب متعارف  کرنے سے غیر  ضروری  اکائیاں ایک دوسرے کے ساتھ منسوخ نہ ہوتی ہوں تب جزو ضربی کو اُلٹا کر دوبارہ کوشش کریں۔ اکائیوں کی تبادلہ میں اکائیوں پر متغیرات اور اعداد کے الجبرائی قواعد لاگو ہوں گے۔

\حصہء{لمبائی}
\سن {1792 } میں فرانس کی نوزائیدہ جمہوریہ نے ناپ اور تول کا ایک نیا نظام قائم کیا۔ میٹر اس کا سنگ بنیادی تھا،  جو قطب شمال سے خط استوا تک  فاصلے کا کڑوڑواں حصہ لیا گیا ۔ بعد میں عملی وجوہات کی  بنا    پر ا س زمینی معیار کو ترک کرتے ہوئے ،  \اصطلاح{پلاٹینم و    اریڈیم }\فرہنگ{پلاٹینم و اریڈیم}\حاشیہب{platinum-iridium}\فرہنگ{platinum-iridium} کی  ایک سلاخ  پر لگائے گئے دو باریک لکیروں کے بیچ فاصلہ  \اصطلاح{ میٹر }\فرہنگ{میٹر}\حاشیہب{meter}  قرار  پایا؛ یہ \اصطلاح{معیاری  میٹر  سلاخ }\فرہنگ{میٹر!معیاری سلاخ}\حاشیہب{standard meter bar}\فرہنگ{meter!standard bar} پیرس شہر کے قریب ناپ  و  تول کے   بین الاقوامی محکمہ میں رکھا گیا ہے۔  اس سلاخ کی  بہترین نقل،  دنیا کی معیار ساز تجربہ گاہوں کو   (بطور ثانوی معیار)  فراہم کی گئی۔ \اصطلاح{ثانوی معیار }\فرہنگ{معیار!ثانوی}\حاشیہب{secondary standards}\فرہنگ{standard!secondary} سے ، مزید  قابل رسائی معیار تیار کیے گئے    ، حتٰی کہ آخر کار ہر پیمائشی آلہ  معیاری میٹر سلاخ  پر مبنی تھا۔ 

کچھ عرصہ   بعد ، سلاخ  پر دو باریک لکیروں کے بیچ فاصلہ  کے معیاری میٹر سے بہتر معیار کی ضرورت پیش آئی۔ \سن {1960 } میں شعاع کے  طول موج پر مبنی میٹر کے   معیار پر اتفاق   کیا گیا۔ یہ معیار کرپٹن  \عددی{86}(  جو \اصطلاح{ کرپٹن }\فرہنگ{کرپٹن}\حاشیہب{krypton}\فرہنگ{krypton}کا ایک مخصوص ہم جا ہے )کے جوہروں سے خارج ایک مخصوص نارنجی  سرخ  شعاع کے \عددی{ 1650763.73}  طول موج کا فاصلہ  ٹہرایا گیا۔ یہ شعاع دنیا میں کہیں بھی  \اصطلاح{گیس خروج  نلی }\فرہنگ{گیس خروج نلی}\حاشیہب{gas discharge tube}\فرہنگ{gas discharge tube} سے حاصل کی جا سکتی ہے۔ نئے معیار  کو پرانے  معیار ( میٹر سلاخ  ) کے قریب سے قریب  رکھنے کی غرض سے تول موج کی  (مذکورہ بالا عجیب) تعداد منتخب کی گئی۔

کچھ عرصہ   تک یہ معیار سائنسی دنیا کی ضروریات  پوری  کر  پایا، تاہم سائنس کی دنیا    بہت جلد اتنی آگے بڑھ چکی کہ کرپٹن \عددی{86} کے طول موج پر مبنی معیار سائنسی ضروریات پوری کرنے   کے قابل نہیں رہا۔  آخر کار  \سن{1983} میں  ایک نڈر فیصلہ کیا گیا، اور میٹر وہ فاصلہ قرار پایا جو شعاع ایک مخصوص دورانیہ میں طے کرتی ہے۔ ناپ  و  تول کے سترھویں \عددی{(17)}  عمومی اجلاس میں درج ذیل  طے پایا۔

\ابتدا{تعریف}
خلاء میں ایک سیکنڈ کے \(\tfrac{1}{299792458}\) حصے میں شعاع کا طے کردہ فاصلہ  \اصطلاح{میٹر   }\فرہنگ{میٹر!تعریف}\حاشیہب{meter}\فرہنگ{meter!definition}کہلائے گا۔
\انتہا{تعریف}

وقت کا  (مذکورہ بالا)  دورانیہ یوں منتخب کیا گیا کہ شعاع  کی رفتار  \عددی{c} ٹھیک درج ذیل ہو۔
\begin{align*}
c = \SI{299792458}{\meter\per\second}
\end{align*}


شعاع کی رفتار اٹل ہے۔ یوں   شعاع کی رفتار  سے  میٹر اخذ کرنا ایک بہتر قدم تھا۔

جدول \حوالہ{جدول_پیمائش_چند_تخمینی_فاصلے} میں  فاصلوں کی  وسیع سعت  پیش ہے،  جو   کہکشانی فاصلوں  سے لے کر انتہائی چھوٹی چیزوں کی لمبائیاں دیتا ہے۔
\begin{table}[h!]
\caption{چند تخمینی فاصلے}
\label{جدول_پیمائش_چند_تخمینی_فاصلے}
\centering
\begin{tabular}{r l} 
\toprule
  پیمائش& میٹر  میں لمبائی\\
\midrule
 اول ترین  پیدا کہکشاں تک فاصلہ& $2\times 10^{26}$ \\
  اندرومدا کہکشاں تک فاصلہ & $2\times 10^{22}$\\
  قریب ترین تارے  تک فاصلہ & $4\times 10^{16}$\\
  پلوٹو تک فاصلہ & $6\times 10^{12}$\\
 زمین کا رداس & $6\times 10^{6}$ \\
  بلند ترین پہاڑی کی  اونچائی & $9\times 10^{3}$\\
 صفحے کی موٹائی &  $1\times 10^{-4}$\\
  علامتی وائرس کی لمبائی & $1\times 10^{-8}$\\
  ہائیڈروجن  جوہر کا رداس & $5\times 10^{-11}$\\
  پروٹان کا رداس & $1\times 10^{-15}$\\
\bottomrule
\end{tabular}
\end{table}


\حصہء{با معنی اعداد اور اشاریہ کے مقام}
فرض کریں  آپ ایک مسئلے پر کام کر رہے ہیں جس میں ہر قیمت دو ہندسوں پر مشتمل  ہے۔ ان ہندسوں کو\اصطلاح{ با معنی ہندسے }\فرہنگ{با معنی ہندسے}\حاشیہب{significant figures}\فرہنگ{significant figures} کہتے ہیں۔ اپنا جواب پیش کرتے ہوئے آپ اتنے ہندسے  ہی استعمال کریں گے ۔ اگر مواد دو ہندسوں میں دیا گیا ہو تب  جواب بھی دو ہندسوں پر مشتمل  ہو گا۔  اگرچہ  آپ کا   حساب کار   نتائج  زیادہ ہندسوں میں پیش کرتا ہے،  یہ (اضافی)  ہندسے بے معنی ہیں۔

اس کتاب میں ، دیے گئے مواد میں کم سے کم با معنی ہندسوں کے برابر ، حساب کے اختتامی نتائج   پورمپور کر کے پیش کیے جائیں گے۔ (ہاں ، بعض اوقات ایک اضافی با معنی  ہندسہ بھی رکھا جائے گا۔ ) اگر ضائع کیے جانے والے ہندسوں میں بایاں ترین ہندسہ  \عددی{5} کے برابر یا اس سے بڑا ہو تب آخری رہنے دیا گیا ہندسے کو  \قول{اوپر پورمپور } کیا جاتا ہے؛ دیگر صورت اس کو جوں کا توں  رکھا جاتا ہے۔ مثال کے طور پر \عددی{11.3516}  کو تین با معنی ہندسوں  میں  پورمپور کرکے  \عددی{11.4} جبکہ  \عددی{11.3279}  کو تین با معنی ہندسوں میں  پورمپور کرتے ہوئے  \عددی{11.3}  لکھا جائے گا۔( اس کتاب میں نتائج پیش کرتے ہوئے پورمپور کیے جانے کے باوجود \عددی{\approx} کی بجائے   عموماً \عددی{=} علامت استعمال کی جائے گی۔)

عدد  \عددی{3.15}  یا \عددی{3.15\times 10^3}  میں با معنی ہندسوں کی تعداد صاف ظاہر ہے؛ عدد  \عددی{3000}  میں با معنی ہندسے کتنے  ہیں؟ کیا یہ صرف ایک با معنی ہندسہ \عددی{3\times 10^3}   تک  معلوم ہے ،  یا یہ چار با معنی ہندسوں \عددی{3.000\times 10^3} تک معلوم ہے؟  اس کتاب میں \عددی{ 3000} کی  طرز  پر اعداد میں تمام صفروں کو با معنی  تصور کیا جائے گا۔

با معنی ہندسوں  اور اشاریہ کے  مقام دو  مختلف  باتیں ہیں۔ درج ذیل فاصلوں  \عددی{\SI{35.6}{\milli\meter}}، \عددی{\SI{3.56}{\meter}}، اور \عددی{\SI{0.00356}{\meter}} پر غور کریں۔  تمام میں تین با معنی ہندسے  ہیں، تاہم  ان میں اشاریہ کے مقام  بالترتیب ایک، دو ، اور پانچ  ہیں۔

%-----------------------------
\ابتدا{مثال}
 \موٹا{ دھاگے کا گیند؛  قدر  کے رتبہ  کی تخمین۔}
 
دنیا میں دھاگے کے سب سے بڑے  گیند کا رداس  \عددی{\SI{2}{\meter}} ہے۔  اس گیند میں  دھاگے کی کل لمبائی  \عددی{L} کتنی ہوگی؟اگرچہ ہم گیند سے دھاگہ کھول کر لمبائی  \عددی{L}  ناپ سکتے ہیں ، تاہم ہم ایسا نہیں کرنا چاہتے۔ ہم حساب کے ذریعہ اس کی لمبائی کا تخمینہ لگانا چاہتے ہیں۔ ہمیں فقط   قدر  کا قریبی  رتبہ درکار ہے۔

\موٹا{حساب}

ہم فرض کرتے ہیں گیند کروی ہے ؛ اس کا رداس  \عددی{R=\SI{2}{\meter}} ہے۔ دھاگہ لپیٹتے  ہوئے   دھاگے کے مختلف حصوں کے بیچ خالی جگہ ضرور ہو گی جس کے بارے میں جاننا نا ممکن بات ہے۔ ان خالی جگہوں کو مد نظر رکھتے ہوئے ہم دھاگے کا عمودی تراش ذرا زیادہ تصور کرتے ہیں۔ ہم کہتے ہیں کہ دھاگے کا عمودی تراش  (گول کی بجائے) چوکور ہے جس کا  ضلع \عددی{d=\SI{4}{\milli\meter}} ہے۔ یوں اس کا رقبہ عمودی تراش \عددی{d^2} ،  لمبائی  \عددی{L}  ، اور کل حجم درج ذیل ہوگا:
\begin{align*}
V=(\text{\RL{رقبہ عمودی تراش}})(\text{\RL{لمبائی}})=d^{2}L
\end{align*}
 
جو گیند کے حجم  \عددی{\tfrac{4}{3}\pi R^3} کے برابر ہوگا ؛ \عددی{\pi}  کو تخمیناً \عددی{3} لیتے ہوئے  یہ حجم \عددی{4R^{3}} لکھا جا سکتا ہے۔ یوں درج ذیل ہوگا
\begin{align*}
d^{2}L=4R^{3}
\end{align*}

جس سے درج ذیل حاصل ہو گا۔
\begin{align*}
L&=\frac{4R^{3}}{d^{2}}\\
&=\frac{4(\SI{2}{\meter})^3}{(\SI{4e-3}{\meter})^2}\\
&=\SI{2e6}{\meter}\approx \SI{e6}{\meter}\approx \SI{e3}{\kilo\meter}
\end{align*}

(اتنے  سادہ حساب کے لئے حساب کار ر کی ضرورت پیش نہیں ہونی  چاہئے۔)    قدر  کے  قریبی رتبہ تک  گیند میں تقریباً \عددی{\SI{1000}{\kilo\meter}} دھاگہ ہے۔
\انتہا{مثال}


\حصہ{وقت}
وقت کے دو پہلو ہیں۔ روز مرہ زندگی میں ہم  کام   کاج ترتیب سے رکھنے کی غرض سے وقت جاننا چاہتے ہیں۔ سائنس کی دنیا میں ہم عموماً جاننا چاہتے ہیں کہ ایک واقعہ کتنی  دیر وقوع پذیر ہوا۔ یوں وقت کے کسی بھی  معیار کو دو سوالات کا جواب دینا ہوگا: کب ہوا؟ اس کا دورانیہ کتنا تھا؟ جدول \حوالہ{جدول_پیمائش_تخمینی_دورانیے}  میں چند وقتی وقفے پیش ہیں، جہاں \اصطلاح{ پلانک وقت }\فرہنگ{پلانک!وقت}\حاشیہب{plank time}\فرہنگ{plank!time} سے مراد   \اصطلاح{ابتدائی دھماکے }\فرہنگ{ابتدائی دھماکہ}\حاشیہب{big bang}\فرہنگ{big bang} کے  بعد  وہ   اول ترین وقت ہے جب طبیعیات کے قواعد (جس طرح انہیں ہم اس وقت  جانتے ہیں)  قابل اطلاق ہوں گے۔

\begin{table}[h!]
\caption{چند تخمینی دورانیے}
\label{جدول_پیمائش_تخمینی_دورانیے}
\centering
\begin{tabular}{r l}
\toprule
پیمائش & سیکنڈ میں دورانیہ \\
\midrule
پروٹان کا عرصہ  حیات (محض  اندازہ)  & $3\times 10^{40}$\\
کائنات کی عمر & $5\times 10^{17}$\\
ہرم  خوفو      کی عمر & $1\times 10^{11}$\\
انسانی زندگی   (متوقع) & $2\times 10^{9}$ \\
ایک دن & $9\times 10^{4}$ \\
انسانی دل کی دھڑکنوں کے بیچ وقفہ & $8\times 10^{-1}$\\
میون کا عرصہ حیات & $2\times 10^{-6}$\\
تجربہ گاہ میں  مختصر ترین   شعاع کا دورانیہ & $1\times 10^{-16}$\\
  غیر مستحکم ترین  ذرے کا عرصہ حیات & $1\times 10^{-23}$\\
پلانک وقفہ  & $1\times 10^{-43}$\\
\bottomrule
\end{tabular}
\end{table}

 وہ  مظہر جو اپنے آپ کو دہراتا ہو وقت کا  معیار مقرر کیا جا سکتا ہے۔  محور کے گرد زمین کا ایک چکر ، جو دن کی لمبائی تعین کرتا ہے ، صدیوں تک بطور وقت کا  معیار  استعمال کیا گیا۔  \اصطلاح{سنگ مردہ }\فرہنگ{سنگ مردہ}\حاشیہب{quartz}\فرہنگ{quartz}  (کوارٹز) گھڑی،  جس میں  ایک سنگ مردہ  چھلے  کو مسلسل ارتعاش پذیر رکھا جاتا ہے ، کی پیمانہ بندی فلکیاتی مشاہدات کے ذریعہ، زمین کے گھومنے کے ساتھ  کر کے،  تجربہ گاہ میں وقتی وقفوں کی پیمائش کے لیے  استعمال کیا جا سکتا ہے۔ تاہم جدید سائنس و انجینئری کو  درکار درستگی ایسی پیمانہ بندی  سے ممکن نہیں۔

بہتر معیار وقت کی ضرورت کے درپیش \اصطلاح{ جوہری گھڑیاں  }\فرہنگ{گھڑی!جوہری}\حاشیہب{atomic clocks}\فرہنگ{clock!atomic} تیار کی گئیں۔\سن{1967 } میں ناپ و تول کے تیرھویں     عمومی اجلاس میں \اصطلاح{سیزیم گھڑی }\فرہنگ{گھڑی!سیزیم}\حاشیہب{cesium clock}\فرہنگ{clock!cesium} پر مبنی معیاری سیکنڈ پر اتفاق کیا گیا۔


\ابتدا{تعریف}
سیزیم \عددی{133}  جوہر سے خارج ایک مخصوص طول موج کی شعاع کے  \عددی{\num{9192631770}} ارتعاش کو درکار وقت ایک\اصطلاح{ سیکنڈ }\فرہنگ{سیکنڈ!تعریف}\حاشیہب{second}\فرہنگ{second!definition}  ٹہرایا گیا۔
\انتہا{تعریف}

جوہری گھڑیاں  انتہائی درست وقت بتاتی ہیں۔دو سیزیم گھڑیوں  میں ایک سیکنڈ فرق چھ ہزار سال چلنے کے بعد  پیدا ہو گا۔ اس وقت تیار کی جانے والی گھڑیوں کی درستگی  \عددی{10^{18}}  میں ایک حصے کے برابر ہے ، یعنی \عددی{10^{18}}  سیکنڈ (جو تقریباً \عددی{3\times 10^{10}} سال  کے برابر ہے)  میں صرف ایک سیکنڈ کا فرق ہو سکتا ہے۔

\حصہ{کمیت}  
\جزوحصہء{معیاری کلوگرام} 
فرانس کے شہر پیرس کے قریب ناپ و تول  کے بین الاقوامی محکمہ  میں رکھے گئے پلاٹینم و  اریڈیم کا ایک سلنڈر ، بین الاقوامی معاہدہ کے تحت ، ایک کلوگرام کمیت  ٹہرایا گیا۔ اس کی بہتر سے بہتر  نقل دنیا کے  بیشتر معیار ساز تجربہ گاہوں کو فراہم کی گئی  جن کو استعمال کرتے ہوئے ترازو کی مدد سے کسی بھی جسم کی کمیت ناپی جا سکتی ہے۔ جدول  \حوالہ{جدول_پیمائش_کمیت} میں   قدر کے \عددی{83}  رتبوں پر پھیلی کمیتوں کو  کلوگرام کی صورت میں  پیش کیا گیا ہے۔

\begin{table}[h!]
\caption{چند تخمینی کمیت}
\label{جدول_پیمائش_کمیت}
\centering
\begin{tabular}{rl}
\toprule
چیز & کلوگرام میں کمیت\\
\midrule
معروف کائنات & $1\times 10^{53}$\\
ہماری کہکشاں & $2\times 10^{41}$\\
سورج & $2\times 10^{30}$\\
چاند & $7\times 10^{22}$\\
سیارچہ ایراس& $5\times 10^{15}$\\
چھوٹا  پہاڑ & $1\times 10^{12}$\\
سمندری جہاز& $7\times10^{7}$\\
ہاتھی & $5\times10^{3}$\\
انگور & $3\times10^{-3}$\\
دھول کی ذرہ & $7\times10^{-10}$\\
پینسلین سالمہ & $5\times10^{-17}$\\
یورینیم جوہر & $4\times10^{-25}$\\
 پروٹان & $2\times10^{-27}$\\
 الیکٹران & $9\times10^{-31}$\\
 \bottomrule
\end{tabular}
\end{table}

\جزوحصہء{دوم معیار کمیت} 
جوہروں کی کمیت کا موازنہ معیاری کلوگرام کی بجائے ، زیادہ درستگی کے ساتھ ،  دیگر جوہروں کے ساتھ کیا جا سکتا ہے۔ اسی کی بنا ، ہم دوم معیار کمیت بھی رکھتے ہیں۔  کاربن \عددی{12} جوہر کو بین الاقوامی معاہدہ کے تحت  \عددی{12} \اصطلاح{جوہری کمیتی اکائیوں}\فرہنگ{جوہری کمیتی اکائی}\حاشیہب{atomic mass unit}\فرہنگ{atomic mass unit} کی کمیت مختص کی گئی۔ ان دو اکائیوں کے بیچ رشتہ درج ذیل ہے
\begin{align}
\SI{1}{\atomicmassunit} = \SI{1.66053886e-27}{\kilogram}
\end{align}
جہاں آخری دو ہندسوں میں عدم یقینیت \عددی{\pm10}  ہے۔ سائنس دان کافی درستگی کے ساتھ تجربہ کے ذریعہ کسی بھی جوہر کی کمیت کاربن  \عددی{12} کی کمیت کے لحاظ سے تعین کر سکتے ہیں۔ اس وقت،  کمیت کی روز مرہ زندگی میں مستعمل   اکائیاں، مثلاً کلوگرام ، استعمال کرتے ہوئے ہم اتنی درستگی حاصل کرنے سے قاصر ہیں۔

\جزوحصہ{کثافت} 
\اصطلاح{کثافت  }\فرہنگ{کثافت}\حاشیہب{density}\فرہنگ{density} \عددی{\rho} سے مراد  اکائی حجم میں کمیت ہے۔
\begin{align}
\rho = \frac{m}{V}
\end{align}
اس پر باب  \حوالہء{14} میں مزید تبصرہ کیا جائے گا۔ کثافت کو عام طور پر کلوگرام فی مربع میٹر یا گرام فی مربع سنٹی میٹر میں ناپا جاتا ہے۔ پانی کی کثافت ایک گرام فی مربع سنٹی میٹر یا ایک ہزار کلوگرام فی مربع میٹر ہے جس کو عموماً موازنہ کے لئے  استعمال کیا جاتا ہے۔ پانی کی کثافت کے لحاظ سے  تازہ برف  کی کثافت \عددی{\SI{10}{\percent}} اور پلاٹینم کی کثافت تقریباً  \عددی{21} گنّا  جبکہ لکڑی کی کثافت صرف  \عددی{\SI{64}{\percent}} ہے۔
