%starting from top of P239
%ch 9 "Elastic Collision In One Dimension"
\باب{یک بعدی لچکی تصادم}
حرکی توانائی کی بقا درج ذیل لکھی جائے گی۔
\begin{align}\label{مساوات_تصادم_حرکی_توانائی_کی_بقا}
\frac{1}{2}m_1v_{1i}^2+\frac{1}{2}m_2v_{2i}^2=\frac{1}{2}m_1v_{1f}^2+\frac{1}{2}m_2v_{2f}^2
\end{align}
ان ہمزاد  مساوات کو \عددی{v_{1f}} اور \عددی{v_{2f}} کے لئے حل کرنے کی خاطر  ہم مساوات \حوالہء{9.71} کو
\begin{align}\label{مساوات_تصادم_بقا_معار_دوم}
m_1(v_{1i}-v_{1f})=-m_2(v_{2i}-v_{2f})
\end{align}
اور مساوات \حوالہ{مساوات_تصادم_حرکی_توانائی_کی_بقا} درج ذیل صورت میں لکھتے ہیں۔
\begin{align}\label{مساوات_تصادم_بقا_دوم}
m_1(v_{1i}-v_{1f})(v_{1i}+v_{1f})=-m_2(v_{2i}-v_{2f})(v_{2i}+v_{2f})
\end{align}
مساوات \حوالہ{مساوات_تصادم_بقا_دوم} کو مساوات \حوالہ{مساوات_تصادم_بقا_معار_دوم} سے تقسیم کرنے  کے بعد کچھ الجبرا کے بعد درج ذیل حاصل ہوں گے۔
\begin{align}\label{مساوات_تصادم_اختتامی_الف}
v_{1f}=\frac{m_1-m_2}{m_1+m_2}v_{1i}+\frac{2m_2}{m_1+m_2}v_{2i}
\end{align}
اور
\begin{align}\label{مساوات_تصادم_اختتامی_ب}
v_{2f}=\frac{2m_1}{m_1+m_2}v_{1i}+\frac{m_2-m_1}{m_1+m_2}v_{2i}
\end{align}
یاد رہے، زیر نوشت \عددی{1} اور \عددی{2} کسی خاص ترتیب سے مختص نہیں کیے گئے۔  مساوات \حوالہء{9.19} میں  اور مساوات \حوالہ{مساوات_تصادم_اختتامی_الف} اور مساوات  \حوالہ{مساوات_تصادم_اختتامی_ب} میں  ان زیر نوشت کو آپس میں بدل کر لکھنے  مساوات کی وہی جوڑی ملتی ہے۔ اس پر بھی توجہ دیں کہ \عددی{v_{2i}=0}  لینے سے، شکل \حوالہء{9.18} میں جسم \عددی{2} ساکن ہدف ہو گا، اور مساوات \حوالہ{مساوات_تصادم_اختتامی_الف}  اور مساوات \حوالہ{مساوات_تصادم_اختتامی_ب} ہمیں  بالترتیب مساوات \حوالہء{9.67} اور مساوات \حوالہء{9.68} دیتی ہیں۔ 



\ابتدا{پڑتال}
شکل \حوالہء{9.18} میں گولے کا ابتدائی معیار حرکت \عددی{\SI{6}{\kilo\gram\meter\per\second}} اور اختتامی معیار حرکت (ا)  \عددی{\SI{2}{\kilo\gram\meter\per\second}} اور (ب) \عددی{\SI{-2}{\kilo\gram \meter\per\second}} ہونے کی صورت میں   ہدف کا  اختتامی خطی معیار حرکت کیا ہو گا؟ اگر گولے کی  ابتدائی اور  اختتامی حرکی توانائی بالترتیب  \عددی{\SI{5}{\joule}} اور \عددی{\SI{2}{\joule}} ہو، ہدف کی اختتامی حرکی توانائی کیا ہو گی؟
\انتہا{پڑتال}
%----------------------

\ابتدا{نمونی سوال} \quad \موٹا{لچکی تصادم در لچکی تصادم}
شکل \حوالہء{9.20a} میں \عددی{v_{1i}=\SI{10}{\meter\per\second}} سے چلتا ہوا سل 1 دو ساکن سلوں کی طرف بڑھتا  ہے۔تینوں سل ایک لکیر پر  ہیں۔ یہ سل  \عددی{2} سے ٹکراتا ہے جو آگے سل \عددی{3} سے  جا کر ٹکراتا ہے، جس کی کمیت \عددی{m_3=\SI{6.0}{\kilo\gram}} ہے۔ دوسرے  تصادم  کے بعد سل \عددی{2} دوبارہ ساکن ہے،  اور سل \عددی{3} کی رفتار  \عددی{v_{3f}=\SI{5.0}{\meter\per\second}} ہے (شکل \حوالہء{9.20b})۔ دونوں تصادم لچکی ہیں۔ سل \عددی{1} اور سل \عددی{2} کی  کمیتیں کیا ہیں؟ سل \عددی{1} کی اختتامی رفتار
 \عددی{v_{1f}} کیا ہے؟
 
 \موٹا{اہم تصورات}
 
 چونکہ ہم تصادم لچکدار تصور کرتے ہیں لہٰذا میکانی توانائی کی  بقا ہو گی (یوں  ٹکر کی آواز، گرمی، اور ارتعاش کی بدولت توانائی کا  ضیاع نظر انداز کیا جاتا ہے)۔ کوئی  بیرونی افقی قوت  سلوں پر عمل نہیں کرتی لہٰذا محور \عددی{x} پر خطی معیار حرکت کی بقا ہو گی۔ ان دو وجوہات کی بنا پر ہم دونوں تصادم پر  مساوات \حوالہء{9.67} اور مساوات \حوالہء{9.68} کا اطلاق کر سکتے ہیں۔
 
 \موٹا{حساب}\quad
 پہلے  تصادم  سے آغاز کرتے ہوئے ہمیں  اتنے زیادہ  نا معلوم متغیرات سے واسطہ ہو گا کہ آگے بڑھنا  مشکل ہو گا: ہم سلوں کی کمیت اور اختتامی سمتی رفتار نہیں جانتے۔ آئیں پہلے تصادم سے آغاز کریں، جس میں سل \عددی{3} کے ساتھ ٹکرانے کے بعد سل \عددی{2} رکتی ہے۔ مساوات \حوالہء{9.67} کا  اطلاق  اس تصادم پر کرتے ہیں جہاں ترقیم تبدیل کرتے ہوئے \عددی{v_{2i}}  تصادم سے قبل سل \عددی{2} کی  رفتار اور \عددی{v_{2f}}  تصادم کے بعد اس کی رفتار  دیتی ہیں۔یوں درج ذیل ہو گا۔
 \begin{align*}
 v_{2f}=\frac{m_2-m_3}{m_2+m_3}v_{2i}
 \end{align*}
 اس میں \عددی{v_{2f}=0} (سل \عددی{2}  رک جاتا ہے) ڈالنے کے بعد  \عددی{m_3=\SI{6.0}{\kilo\gram}}  ڈال کر درج ذیل حاصل ہو گا۔
 \begin{align*}
 m_2&=m_3=\SI{6.0}{\kilo\gram}&&\text{\RL{(جواب)}}
 \end{align*}
اسی طرح  ترقیم  تبدیل کر کے دوسرے تصادم کے لئے  مساوات \حوالہء{9.68} لکھتے ہیں
\begin{align*}
v_{3f}=\frac{2m_2}{m_2+m_3}v_{2i}
\end{align*}
جہاں \عددی{v_{3f}} تیسرے سل کی اختتامی سمتی رفتار ہے۔ اس میں \عددی{m_2=m_3} ڈالنے کے بعد \عددی{v_{3f}=\SI{5.0}{\meter\per\second}} ڈال کر درج ذیل حاصل ہو گا۔
\begin{align*}
v_{2i}=v_{3f}=\SI{5.0}{\meter\per\second}
\end{align*}
%???KKK am here example still continueing
\انتہا{نمونی سوال}
%--------------------
