%external references might be missing and figures are missing
%p295

\باب{لڑھکاو، قوت مروڑ، اور زاوی معیار حرکت}

\حصہ{مستقیم حرکت اور گھماو  مل کر لڑھکاو دیتے ہیں}
\موٹا{مقاصد}\\
اس حصے کو پڑھنے کے بعد آپ ذیل کے قابل ہوں گے۔
\begin{enumerate}[1.]
\item
جان پائیں گے کہ لڑھکاو خالص مستقیم حرکت اور خالص گھماو  کا مجموعہ ہے۔
\item
ہموار لڑھکاو  میں  مرکز کمیت  کی رفتار اور  جسم کی زاوی رفتار  کا تعلق استعمال کر پائیں گے۔
\end{enumerate}

\موٹا{کلیدی تصورات}\\
\begin{itemize}
\item
رداس \عددی{R} کے پہیا کے لئے جو  ہموار سطح پر لڑھک رہا ہو ذیل ہو گا:
\begin{align*}
v_{\text{\RL{مرکزکمیت}}}=\omega R
\end{align*}
جہاں \عددی{v_{\text{\RL{مرکزکمیت}}}} پہیے کے مرکز کمیت  کی خطی رفتار  اور \عددی{\omega} پہیے کے وسط پر پہیے کی زاوی رفتار ہے۔
\item
پہیے کو   نقطہ \عددی{P}  کے گرد، جو\قول{  فرش}  کے ساتھ مس ہے، لمحاتی   گھومتا تصور کیا جا سکتا ہے۔ مرکز کمیت کے گرد اور اس نقطہ کے گرد جسم کی زاوی رفتار برابر  ہے۔
\end{itemize}

\جزوحصہء{طبیعیات کیا ہے؟}
جیسا باب \حوالہ{باب_گھماو} میں ذکر کیا گیا، گھماو کا مطالعہ طبیعیات  میں شامل ہے۔ غالباً، اس مطالعے   کا اہم ترین     اطلاق پہیے اور پہیے نما اجسام کا لڑھکاو  ہے۔ یہ اطلاقی طبیعیات بہت عرصہ سے استعمال میں ہے۔ قدیم زمانے میں بھاری اجسام  لٹھا پر  لڑھکاتے ہوئے ایک جگہ سے دوسری جگہ منتقل کیے جاتے تھے۔آج کل ہم گاڑی میں سامان رکھ کر ایک جگہ سے دوسری جگہ لڑھکاتے ہیں۔

لڑھکاو  کی طبیعیات اور انجینئری  اتنی پرانی ہے کہ  اس میں نئے تصور ممکن  نظر نہیں  آتے۔ تاہم، \اصطلاح{ پہیے دار تختہ }\فرہنگ{پہیے دار تختہ}\حاشیہب{skateboards}\فرہنگ{skateboard}  زیادہ پرانا نہیں۔ ہمارا کام یہاں لڑھکاو کی حرکت  کو سادہ بنانا ہے۔

\جزوحصہء{مستقیم حرکت اور گھماو ت مل کر لڑھکاو دیتے ہیں}
سطح  پر  \ترچھا{ہمواری سے لڑھکتے } اجسام پر  یہاں غور کیا جائے گا؛ یعنی جسم بغیر اچھلے یا پھسلے سطح پر حرکت کرتا ہے۔  شکل \حوالہء{11.2} میں  ہموار لڑھکاو  کی پیچیدگی دکھائی گئی ہے: اگرچہ جسم کا مرکز کمیت سیدھی لکیر پر حرکت کرتا ہے، چکا پر نقطہ یقیناً ایسا نہیں کرتا۔بہرحال،  اس حرکت کو مرکز کمیت کی مستقیم حرکت اور  باقی جسم کا، اس مرکز پر ، گھماو تصور کیا جا سکتا ہے۔

اسے سمجھنے کے لئے، فرض کریں آپ سڑک کے کنارے کھڑے ہو کر،  گزرتے ہوئے سائیکل کے پہیے کا  مطالعہ کرتے ہیں (شکل \حوالہء{11.3})۔ جیسا شکل میں دکھایا گیا ہے، پہیے کا  مرکز  کمیت \عددی{O}  مستقل رفتار \عددی{v_{\text{\RL{مرکزکمیت}}}} سے آگے بڑھتا ہے۔ نقطہ \عددی{P} ، جہاں پہیا سڑک کو مس کرتا ہے،  بھی  \عددی{v_{\text{\RL{مرکزکمیت}}}}  رفتار سے آگے بڑھتا ہے، اور یوں \عددی{P} ہمیشہ \عددی{O} کے ٹھیک نیچے رہتا ہے۔

وقتی دورانیہ \عددی{t} کے دوران، \عددی{O} اور \عددی{P} دونوں فاصلہ \عددی{s} طے کرتے ہیں۔ سائیکل سوار  کے نقطہ نظر سے، پہیا زاویہ \عددی{\theta} طے کرتا ہے اور جو نقطہ \عددی{t} کے آغاز میں زمین پر تھا قوسی فاصلہ \عددی{s} طے کرتا ہے۔ مساوات \حوالہ{مساوات_گھماو_خطی_زاوی_تعلق_الف}  قوسی فاصلہ \عددی{s} اور زاویہ \عددی{\theta} کا تعلق دیتی ہے:
%eq 11.1
\begin{align}\label{مساوات_لڑھکاو_فاصلہ_زاویہ_الف}
s=\theta R
\end{align}
جہاں \عددی{R} پہیے کا رداس ہے۔ پہیے کے مرکز (یکساں پہیے کا مرکز کمیت) کی خطی رفتار \عددی{v_{\text{\RL{مرکزکمیت}}}} ہم \عددی{\dif s\!/\!\dif t} سے جان سکتے ہیں۔ پہیے کے مرکز پر پہیے کی زاوی رفتار \عددی{\dif \theta\!/\!\dif t} ہو گی۔ یوں \عددی{R} مستقل رکھتے ہوئے،  مساوات \حوالہ{مساوات_لڑھکاو_فاصلہ_زاویہ_الف} کا  وقت کے ساتھ تفرق ذیل دیگا۔
%eq 11.2
\begin{align}\label{مساوات_لڑھکاو_فاصلہ_زاویہ_ب}
v_{\text{\RL{مرکزکمیت}}}=\omega R\quad\quad\text{\RL{(ہموار لڑھکاو  حرکت)}}
\end{align}

\موٹا{دونوں کا ملاپ۔}
شکل \حوالہء{11.4} میں  دکھایا گیا ہے کہ پہیے کی لڑھکنی حرکت  خالص مستقیم حرکت اور خالص گھمیری حرکت کا مجموعہ ہے۔ شکل \حوالہء{11.4a} خالص گھمیری حرکت پیش کرتی ہے (جس میں مرکز پر محور گھماو ساکن تصور کیا جاتا ہے): پہیے کا ہر نقطہ ، مرکز پر ، زاوی رفتار \عددی{\omega} سے گھومتا ہے۔ (ایسی حرکت پر باب \حوالہ{باب_گھماو} میں غور کیا گیا۔) پہیے کے  باہری  کنارے (چکا)  پر ہر نقطے کی خطی رفتار \عددی{v_{\text{\RL{مرکزکمیت}}}}   مساوات \حوالہ{مساوات_لڑھکاو_فاصلہ_زاویہ_ب}  دیتی    ہے۔ شکل \حوالہء{11.4b} میں  خالص مستقیم  حرکت پیش ہے (جس میں تصور کیا جاتا ہے کہ پہیا گھوم نہیں رہا):  پہیے کا ہر نقطہ  \عددی{v_{\text{\RL{مرکزکمیت}}}} رفتار سے دائیں حرکت کرتا ہے۔

شکل \حوالہء{11.4a}  اور شکل \حوالہء{11.4b} مل کر ،   شکل \حوالہء{11.4c} میں پیش، پہیے کی  اصل لڑھکنی  حرکت دیتی ہیں۔ حرکات کے  ملاپ میں  پہیے کا   نچلا  نقطہ  (\عددی{P}) ساکن ہے جبکہ پہیے کا بالا  نقطہ (\عددی{T}) ، کسی بھی دوسرے نقطہ سے زیادہ تیز، \عددی{2v_{\text{\RL{مرکزکمیت}}}} رفتار سے حرکت کرتا ہے۔ شکل \حوالہء{11.5} میں ان نتائج  کا  اثباتی مظاہرہ کیا گیا ہے، جہاں سائیکل کے لڑھکنی پہیے  کا   \اصطلاح{وقتیہ افشا  }\فرہنگ{وقتیo!afxa}\حاشیہب{time exposure}\فرہنگ{time!exposure} پیش  ہے۔ آپ دیکھ کر  بتا سکتے ہیں کہ پہیے کا بالا حصہ زیادہ تیزی سے حرکت کرتا ہے، چونکہ اس حصے کی تیلیاں مدھم نظر آتی ہیں۔

سطح پر  دائری جسم کی ہموار لڑھکنی  حرکت  کو ، شکل  \حوالہء{11.4a} اور شکل \حوالہء{11.4b} کی طرح، خالص گھمیری حرکت اور خالص مستقیم حرکت میں علیحدہ  علیحدہ کیا جا سکتا ہے۔

\جزوجزوحصہء{لڑھکاو بطور خالص گھماو}
شکل \حوالہء{11.6} میں پہیے کا لڑھکاو   نئے انداز میں پیش کیا گیا ہے؛ جس نقطے پر پہیا سڑک مس کرتا ہے، اس نقطے  سے گزرتی محور پر پہیا گھومتا ہے؛ یہ محور \عددی{v_{\text{\RL{مرکزکمیت}}}}  رفتار سے حرکت میں ہو گی۔ہم  لڑھکاو کو  ،  شکل \حوالہء{11.4c} میں نقطہ \عددی{P} سے گزرتی  ، پہیے کو عموددار، محور پر خالص گھماو  تصور کرتے ہیں۔ یوں شکل \حوالہء{11.6} میں سمتیات ، لڑھکنی پہیے پر نقطوں کی لمحاتی سمتی رفتار  دیتے ہیں۔

\موٹا{سوال:}\quad
ساکن  مشاہدہ کار  اس محور پر سائیکل کے  لڑھکنی  پہیے کو کیا زاوی رفتار مختص کرے گا؟

\موٹا{جواب:}\quad
وہی زاوی رفتار \عددی{\omega} جو سائیکل  سوار  مرکز کمیت کے گرد خالص گھماو  کا مشاہدہ کرتے ہوئے پہیے کو مختص کرتا ہے۔

اس جواب کی تصدیق کرنے کی خاطر،  ہم ساکن مشاہدہ کار کے نقطہ نظر سے  لڑھکنی پہیے کے فراز  کی خطی رفتار تلاش کرتے ہیں۔ پہیے کا رداس \عددی{R} لیتے ہوئے، پہیے کا فراز  شکل \حوالہء{11.6} میں \عددی{P}  پر واقع محور سے \عددی{2R} فاصلے پر ہو گا، لہٰذا فراز کی خطی رفتار  (مساوات \حوالہ{مساوات_لڑھکاو_فاصلہ_زاویہ_ب} استعمال کر کے) ذیل ہو گی:
\begin{align*}
v_{\text{\RL{فراز}}}=(\omega)(2R)=2(\omega R)=2v_{\text{\RL{مرکزکمیت}}}
\end{align*}
جو شکل \حوالہء{11.4c} کے عین مطابق ہے۔آپ شکل \حوالہء{11.4c} میں  پیش   ، نقطہ  \عددی{O} اور \عددی{P} کی ، خطی رفتار کی تصدیق  بھی اس  طرح کر سکتے ہیں۔

%--------------------------
%checkpoint 1,  p297
\ابتدا{آزمائش}
ایک سائیکل کے پچھلے  پہیے کا رداس اگلے پہیے کے رداس کا دگنا ہے۔ (ا)  کیا چلنے کے دوران بڑے پہیے کے فراز کی خطی رفتار چھوٹے پہیے کے فراز کی خطی رفتار سے زیادہ ہے، کم ہے، یا اس کے برابر ہے؟ (ب)  کیا پچھلے پہیے کی زاوی رفتار اگلے پہیے کی زاوی رفتار سے زیادہ ہے، کم ہے، یا دونوں برابر ہیں؟
\انتہا{آزمائش}
%------------------------------

% 11.2  forces and kinetic energy of rolling   p298
\حصہ{لڑھکاو کی قوتیں اور حرکی توانائی}
\موٹا{مقاصد}\\
اس حصہ کو پڑھنے کے بعد آپ  ذیل کے قابل ہوں گے۔
\begin{enumerate}[1.]
\item
مرکز کمیت  کی  مستقیم حرکی توانائی اور مرکز کمیت   کے گرد گھمیری حرکی توانائی کا مجموعہ  حاصل کر کے جسم کی حرکی توانائی معلوم کر پائیں گے۔
\item
ہمواری کے ساتھ   لڑھکنی جسم کی حرکی توانائی میں تبدیلی اور  جسم پر سرانجام کام  کا تعلق استعمال کر پائیں گے۔
\item
ہموار لڑھکاو (لہٰذا  بغیر  پھسلن) کے لئے،  میکانی توانائی کی بقا استعمال کر کے ابتدائی توانائی  کی قیمتوں اور اختتامی توانائی  کی قیمتوں کا تعلق جان پائیں گے۔
\end{enumerate}

\موٹا{کلیدی تصورات}\\
\begin{itemize}
\item
ہموار لڑھکنی پہیے کی حرکی توانائی ذیل ہے،
\begin{align*}
K=\frac{1}{2}I_{\text{\RL{مرکزکمیت}}}\omega^2+\frac{1}{2}Mv_{\text{\RL{مرکزکمیت}}}^2
\end{align*}
جہاں  مرکز کمیت پر جسم کا گھمیری جمود \عددی{I_{\text{\RL{مرکزکمیت}}}}  اور پہیے کی کمیت \عددی{M} ہے۔
\item
اگر پہیا مسرع کیا جائے، اور پہیا اب بھی ہمواری کے ساتھ لڑھکتا  ہے ، مرکز کمیت  کے  اسراع \عددی{\vec{a}_{\text{\RL{مرکزکمیت}}}}  اور مرکز پر زاوی اسراع \عددی{\alpha}  کا تعلق ذیل ہو گا۔
\begin{align*}
a_{\text{\RL{مرکزکمیت}}}=\alpha R
\end{align*}
\item
اگر \عددی{\theta} زاویہ کے میلان پر  پہیا ہمواری کے ساتھ  اترتے ہوئے  لڑھکتا ہو، اس کا اسراع، میلان کے ہمراہ  اوپر رخ  محور  \عددی{x} پر،  ذیل ہو گا۔
\begin{align*}
a_{\text{\RL{مرکزکمیت}}}=-\frac{g\sin\theta}{1+I_{\text{\RL{مرکزکمیت}}}\!/\!{MR^2}}
\end{align*}
\end{itemize}

\جزوحصہء{لڑھکاو کی حرکی توانائی}
آئیں ساکن مشاہدہ کار  کے نقطہ نظر سے  لڑھکنی پہیے کی حرکی توانائی معلوم کریں۔ اگر ہم شکل \حوالہء{11.6} میں نقطہ \عددی{P} سے گزرتی محور  پر لڑھکاو کو خالص گھماو تصور کریں، تب مساوات \حوالہ{مساوات_گھماو_حرکی_گھمیری_تعریف} ذیل دیگی،
%eq 11.3
\begin{align}\label{مساوات_لڑھکاو_فاصلہ_زاویہ_پ}
K=\frac{1}{2}I_P\omega^2
\end{align}
جہاں  \عددی{P} پر واقع محور کے گرد پہیے کا گھمیری جمود \عددی{I_P} اور  پہیے کی زاوی رفتار \عددی{\omega} ہے۔ مساوات \حوالہ{مساوات_گھماو_مسئلہ_متوازی_محور}  کے مسئلہ متوازی محور (\عددی{I=I_{\text{\RL{مرکزکمیت}}}+Mh^2}) کے تحت ذیل ہو گا،
%eq 11.4
\begin{align}\label{مساوات_لڑھکاو_فاصلہ_زاویہ_ت}
I_P=I_{\text{\RL{مرکزکمیت}}}+MR^2
\end{align}
جہاں  \عددی{M} پہیے کی کمیت،   مرکز کمیت سے گزرتی محور  پر  گھمیری جمود  \عددی{I_{\text{\RL{مرکزکمیت}}}}، اور  \عددی{R} (پہیے کا رداس)  عموددار فاصلہ \عددی{h} ہے۔ مساوات \حوالہ{مساوات_لڑھکاو_فاصلہ_زاویہ_ت} کو مساوات \حوالہ{مساوات_لڑھکاو_فاصلہ_زاویہ_پ}  میں ڈال کر :
\begin{align*}
K=\frac{1}{2}I_{\text{\RL{مرکزکمیت}}}\omega^2+\frac{1}{2}MR^2\omega^2
\end{align*}
اور مساوات \حوالہ{مساوات_لڑھکاو_فاصلہ_زاویہ_ب}  (\عددی{v_{\text{\RL{مرکزکمیت}}}=\omega R})  استعمال کرکے ذیل حاصل ہو گا۔
%eq 11.5
\begin{align}\label{مساوات_لڑھکاو_مستقیم_گھمیری_الف}
K=\frac{1}{2}I_{\text{\RL{مرکزکمیت}}}\omega^2+\frac{1}{2}Mv_{\text{\RL{مرکزکمیت}}}^2
\end{align}

جزو \عددی{\tfrac{1}{2}I_{\text{\RL{مرکزکمیت}}}\omega^2}  کو  مرکز کمیت سے گزرتی محور پر پہیے کے لڑھکاو سے وابستہ حرکی توانائی تصور کیا جا سکتا ہے (شکل \حوالہء{11.4a})، اور جزو \عددی{\tfrac{1}{2}Mv_{\text{\RL{مرکزکمیت}}}^2} کو  پہیے کے مرکز کمیت کی مستقیم حرکت سے وابستہ حرکی توانائی تصور کیا جا سکتا ہے (شکل \حوالہء{11.4b})۔ یوں ذیل قاعدہ ابھرتا ہے۔

\ابتدا{قاعدہء}
لڑھکنی  جسم کی دو قسم کی حرکی توانائیاں ہوں گی: مرکز کمیت پر گھماو کی بدولت گھمیری حرکی توانائی \عددی{(\tfrac{1}{2}I_{\text{\RL{مرکزکمیت}}}\omega^2)} اور  مرکز کمیت کی مستقیم حرکت کی بدولت مستقیم حرکی توانائی \عددی{(\tfrac{1}{2}Mv_{\text{\RL{مرکزکمیت}}}^2)}۔
\انتہا{قاعدہء}

%----------------------------
%the forces of rolling p299
\جزوحصہء{لڑھکاو کی قوتیں}
\جزوجزوحصہء{رگڑ اور لڑھکاو}
اگر پہیا مستقل رفتار سے لڑھکتا ہو، جیسا شکل \حوالہء{11.3} میں دکھایا گیا ہے،  نقطہ  تماس \عددی{P} پر پہیا   ہرگز نہیں پھسلتا لہٰذا اس نقطے پر رگڑ نہیں ہو گی۔ تاہم، اگر صافی قوت پہیے کو تیز یا آہستہ  کرتی ہو، تب یہ صافی قوت  مرکز کمیت کو حرکت کے رخ  اسراع \عددی{\vec{a}_{\text{\RL{مرکزکمیت}}}} بخشے گی۔ ساتھ ہی  پہیا تیز یا آہستہ گھومے گا، لہٰذا زاوی اسراع \عددی{\alpha} بھی  ہو گا۔ ان اسراع کی بدولت پہیا \عددی{P} پر پھسل  سکتا ہے۔ یوں \عددی{P} پر رگڑی قوت عمل کرتی ہوئے  پہیے کو پھسلنے  سے روکتی ہے۔

اگر پہیا پھسلے نہیں، یہ قوت \ترچھا{سکونی } رگڑی قوت \عددی{\vec{f}_s} ہو گی اور حرکت ہموار لڑھکاو ہو گا۔ ایسی صورت میں،    (\عددی{R} مستقل رکھ کر)  وقت کے ساتھ مساوات \حوالہ{مساوات_لڑھکاو_فاصلہ_زاویہ_ب} کا تفرق  لے کر  خطی اسراع \عددی{\vec{a}_{\text{\RL{مرکزکمیت}}}} کی قدر اور زاوی اسراع کی قدر \عددی{\alpha} کا تعلق حاصل کر سکتے ہیں۔ بائیں ہاتھ \عددی{\dif v_{\text{\RL{مرکزکمیت}}}\!/\!\dif t} درحقیقت \عددی{a_{\text{\RL{مرکزکمیت}}}} اور دائیں ہاتھ  \عددی{\dif \omega\!/\!\dif t} درحقیقت \عددی{\alpha} ہے۔ یوں ہموار لڑھکاو کے لئے ذیل ہو گا۔
%eq 11.6
\begin{align}\label{مساوات_لڑھکاو_رگڑی_الف}
a_{\text{\RL{مرکزکمیت}}}=\alpha R\quad\quad\text{\RL{{(ہموار لڑھکنی حرکت)}}}
\end{align}

جب پہیے پر عمل پیرا  صافی قوت  کی بدولت   پہیا پھسلے ، تب   شکل \حوالہء{11.3} میں \عددی{P}   پر \ترچھا{حرکی } رگڑی قوت \عددی{\vec{f}_k}  عمل کرے گی؛ حرکت تب  ہموار  لڑھکاو نہیں ہو گی، اور مساوات \حوالہ{مساوات_لڑھکاو_رگڑی_الف} کا اطلاق نہیں ہو گا۔ اس باب میں صرف ہموار لڑھکنی حرکت پر بات کی جائے گی۔

شکل \حوالہء{11.7} میں،    افقی سطح پر دائیں   رخ لڑھکتے ہوئے   ،  سائیکل مقابلے کے آغاز کی طرح،  پہیا زیادہ تیز  گھمایا جاتا ہے۔ زیادہ تیز گھماو کی بدولت \عددی{P}  پر پہیا پھسل کر بائیں  جانا چاہتا ہے۔  نقطہ \عددی{P} پر  دائیں رخ رگڑی قوت  اس رجحان کا مقابلہ کرتی ہے۔ اگر پہیا پھسلے نہیں، یہ قوت سکونی رگڑی قوت \عددی{\vec{f}_s} ہو گی (جیسا دکھایا گیا ہے)، حرکت ہموار لڑھکاو ہو گی، اور مساوات \حوالہ{مساوات_لڑھکاو_رگڑی_الف} کا اطلاق ہو گا۔ (رگڑ کی غیر موجودگی میں سائیکل  مقابلہ ممکن نہیں ہو گا۔)

اگر شکل \حوالہء{11.7} میں پہیا آہستہ کیا جائے، ہمیں شکل دو طرح تبدیل کرنی ہو گی: مرکز کمیت کے اسراع \عددی{\vec{a}_{\text{\RL{مرکزکمیت}}}}   کا رخ   اور نقطہ \عددی{P}  پر رگڑی قوت \عددی{\vec{f}_s} کا رخ  اب بائیں  رخ  ہو گا۔

%rolling down a ramp p299
\جزوجزوحصہء{میلان  سے نیچے لڑھکاو}
شکل \حوالہء{11.8} میں گول یکساں جسم ، جس کی کمیت \عددی{M} اور رداس \عددی{R} ہے، زاویہ \عددی{\theta}  کے میلان پر  ہمواری سے ،  محور \عددی{x} کے ہمراہ، نیچے لڑھک رہا ہے۔ ہم میلان کے ہمراہ     اترائی کے  رخ جسم  کے  اسراع \عددی{a_{\text{\RL{مرکزکمیت}},x}} کا ریاضی فقرہ تلاش کرنا  چاہتے ہیں۔  نیوٹن کے  قانون دوم    کی  خطی صورت \عددی{(F_{\text{\RL{صافی}}}=Ma)} اور زاوی صورت \عددی{(\tau_{\text{\RL{صافی}}}=I\alpha)}  صورت دونوں  استعمال کر کے ایسا کرتے ہیں۔

جسم پر قوتوں کا خاکہ بنانے سے آغاز کرتے ہیں (شکل \حوالہء{11.8})۔
\begin{enumerate}[1.]
\item
جسم پر تجاذبی قوت \عددی{\vec{F}_g} نشیب وار ہے۔ اس سمتیہ کی دم جسم کے مرکز کمیت پر رکھی جاتی ہے۔ میلان کے ہمراہ جزو \عددی{F_g\sin\theta} ہے جو \عددی{Mg\sin\theta} کے برابر ہو گا۔
\item
میلان کو عموددار  جزو \عددی{\vec{F}_N} ہے۔ یہ جزو نقطہ تماس \عددی{P} پر عمل کرتا ہے، تاہم  شکل \حوالہء{11.8} میں ،  \عددی{\vec{F}_N} کا رخ تبدیل کیے بغیر،  اس کو یوں کھسکایا  کیا گیا ہے کہ اس کی دم جسم کے مرکز کمیت پر ہو۔
\item
نقطہ تماس \عددی{P} پر  عمل پیرا سکونی  رگڑی  قوت  \عددی{\vec{f}_s}  میلان کے ہمراہ چڑھائی کے رخ ہے۔ (کیا آپ بتا سکتے ہیں، کیوں؟ اگر  \عددی{P} پر جسم  پھسلے، وہ\ترچھا{  اترائی }کے رخ پھسلے گا۔ یوں مخالف  رگڑی قوت چڑھائی کے رخ ہو گی۔ )
\end{enumerate}

ہم شکل \حوالہء{11.8} میں  محور \عددی{x} کے ہمراہ  اجزاء کے لئے نیوٹن کا قانون دوم \عددی{(F_{\text{\RL{صافی}},x}=ma_x)}  لکھتے ہیں۔
%eq 11.7
\begin{align}\label{مساوات_لڑھکاو_ہمراہ}
f_s-Mg\sin\theta=Ma_{\text{\RL{صافی}},x}
\end{align}
اس مساوات میں دو نامعلوم متغیرات ، \عددی{f_s} اور \عددی{a_{\text{\RL{صافی}},x}}، پائے جاتے ہیں۔ (ہم \عددی{f_s} کی قیمت،  رگڑی قوت کی زیادہ سے زیادہ قیمت،  \عددی{f_{s,\text{\RL{بلندتر}}}} فرض نہیں کر سکتے۔ہم صرف اتنا جانتے ہیں کہ رگڑی قوت  اتنی ہے کہ جسم پھسلتا نہیں اور میلان پر ہمواری سے لڑھکتا اترتا ہے۔)

ہم اب جسم کے مرکز کمیت  پر  جسم کے گھماو  پر نیوٹن کے قانون دوم کا اطلاق کرتے ہیں۔ پہلے، مساوات \حوالہ{مساوات_گھماو_صافی_قوت_مروڑ_الف}
  \عددی{(\tau=r_{\perp}F)} استعمال کر کے  مرکز کمیت  کے لحاظ سے جسم پر قوت مروڑ لکھتے ہیں۔ رگڑی قوت \عددی{\vec{f}_s}  کے  معیار اثر کا بازو  \عددی{R} ہے، لہٰذا اس کی قوت مروڑ \عددی{Rf_s} ہو گی، جو  اس لئے مثبت ہے کہ شکل \حوالہء{11.8} میں یہ جسم کو خلاف گھڑی گھمانے کی کوشش کرتی ہے۔مرکز کمیت کے لحاظ سے  قوت \عددی{\vec{F}_g} اور \عددی{\vec{F}_N} کے معیار اثر بازو صفر ہیں، لہٰذا ان کی قوت مروڑ صفر ہوں گی۔ جسم کے مرکز کمیت سے گزرتی محور پر نیوٹن کا قانون دوم  زاوی روپ \عددی{(\tau_{\text{\RL{صافی}}}=I\alpha)} میں لکھتے ہیں۔
  %eq 11.8
  \begin{align}\label{مساوات_لڑھکاو_زاوی_روپ_الف}
  Rf_s=I_{\text{\RL{مرکزکمیت}}}\alpha
  \end{align}
  اس مساوات میں دو نامعلوم متغیرات، \عددی{f_s} اور \عددی{\alpha} ، پائے جاتے ہیں۔
  
  جسم ہموار لڑھکتا ہے لہٰذا مساوات \حوالہ{مساوات_لڑھکاو_رگڑی_الف} \عددی{(a_{\text{\RL{مرکزکمیت}}}=\alpha R)}  استعمال  کر کے نامعلوم 
  \عددی{a_{\text{\RL{مرکزکمیت}},x}} اور \عددی{\alpha} کا تعلق لکھا جا سکتا ہے۔ تاہم، ہمیں ہوشیاری سے کام لینا ہو گا، چونکہ یہاں     \عددی{a_{\text{\RL{مرکزکمیت}},x}} منفی  (محور \عددی{x} پر منفی رخ ہے)  اور \عددی{\alpha} مثبت  (خلاف گھڑی) ہے۔ یوں مساوات \حوالہ{مساوات_لڑھکاو_زاوی_روپ_الف} میں \عددی{\alpha} کی جگہ \عددی{-a_{\text{\RL{مرکزکمیت}},x}\!/\!R} ڈال کر  \عددی{f_s} کے لئے حل کر کے ذیل حاصل کرتے ہیں۔
  %eq 11.9
  \begin{align}\label{مساوات_لڑھکاو_زاوی_روپ_ب}
  f_s=-I_{\text{\RL{مرکزکمیت}}}\frac{a_{\text{\RL{مرکزکمیت}},x}}{R^2}
  \end{align}
  مساوات \حوالہ{مساوات_لڑھکاو_ہمراہ} میں \عددی{f_s} کی جگہ مساوات \حوالہ{مساوات_لڑھکاو_زاوی_روپ_ب} کا دایاں ہاتھ ڈال کر ذیل ملتا ہے۔
  %eq 11.10
  \begin{align}\label{مساوات_لڑھکاو_اسراع}
  a_{\text{\RL{مرکزکمیت}},x}=-\frac{g\sin\theta}{1+I_{\text{\RL{مرکزکمیت}}}\!/\!MR^2}
  \end{align}
  اس مساوات کو استعمال کر  کے ، افق کے ساتھ زاویہ \عددی{\theta}  کے میلان پر  کے ہمراہ لڑھکتے جسم  کا خطی اسراع \عددی{ a_{\text{\RL{مرکزکمیت}},x}} حاصل ہو گا۔
  
 یاد رہے، تجاذبی قوت جسم کو میلان پر اترنے پر مجبور کرتی ہے، تاہم جسم کو گھومنے اور یوں لڑھکنے پر رگڑی قوت مجبور کرتی ہے۔ اگر آپ رگڑ  خارج کر  دیں (مثلاً،  میلان کو  تیل سے چکنا کر کے ) یا \عددی{Mg\sin\theta} کو  \عددی{f_{s,\text{\RL{بلندتر}}}}  سے زیادہ کر دیں، ہموار لڑھکاو خارج ہو جائے گا اور جسم لڑھکنے کی بجائے میلان پر پھسل کر اترے گا۔
 
 %------------------------
 %checkpoint 2 p300
 \ابتدا{آزمائش}
 قرص \عددی{A} اور \عددی{B} ایک جیسے ہیں اور فرش پر ایک جتنی رفتار سے لڑھکتے ہیں۔ قرص \عددی{A} کے سامنے میلان آتا ہے جس پر یہ زیادہ سے زیادہ   \عددی{h} تک پہنچتا ہے۔ قرص \عددی{B} متماثل ، لیکن بلا رگڑ  ، میلان پر چڑھتا ہے ۔ کیا \عددی{h} سے زیادہ، کم، یا اس کے برابر بلندی تک \عددی{B} پہنچے گا؟
 \انتہا{آزمائش}
 
 %--------------------------------
 %sample problem 11.01 Ball rolling down a ramp  p310
 \ابتدا{نمونی سوال}
 یکساں گیند، جس کی کمیت \عددی{M=\SI{6.00}{\kilo\gram}} اور رداس \عددی{R} ہے،  زاویہ \عددی{\theta=\SI{30.0}{\degree}} میلان  سے،  ساکن حالت سے آغاز کر کے،  ہموار لڑھکتا اترتا ہے (شکل \حوالہء{11.8})۔
 
 (ا)  انتصابی \عددی{h=\SI{1.20}{\meter}} نیچے زمین کو  پہنچتا   کر  گیند کی رفتار  کیا ہو گی؟
 
 \جزوحصہء{کلیدی تصورات}
چونکہ صرف تجاذبی قوت، جو   بقائی قوت ہے،  گیند پر کام سرانجام  دیتی ہے، لہٰذا میلان پر لڑھک کر اترنے کے دوران گیند و زمین نظام کی میکانی توانائی \عددی{E}  کی بقا ہو گی۔میلان سے گیند پر عمود دار قوت  گیند کی راہ کو عمودی  ہونے کی وجہ سے کوئی کام سرانجام نہیں دیتی۔ گیند پھسلتا نہیں (\ترچھا{ہموار  لڑھکتا }ہے)  لہٰذا رگڑی قوت کوئی  توانائی حری توانائی میں تبدیلی نہیں کرتی۔

یوں میکانی توانائی کی بقا ہو گی \عددی{E_f=E_i}:
%eq 11.11
\begin{align}\label{مساوات_لڑھکاو_نمونی_گیند_الف}
K_f+U_f=K_i+U_i
\end{align}
جہاں زیر نوشت \عددی{f} اور \عددی{i} بالترتیب (زمین پر پہنچ کر) اختتامی اور  (ساکن حالت ) ابتدائی قیمتیں ظاہر کرتی ہیں۔تجاذبی مخفی توانائی کی  ابتدائی  قیمت \عددی{U_i=Mgh} (جہاں \عددی{M}  گیند کی کمیت ہے) اور  اختتامی قیمت \عددی{U_f=0} ہے۔  ابتدائی حرکی توانائی  \عددی{K_i=0} ہے اختتامی حرکی توانائی جاننے کے لئے  اضافی  تصور  درکار ہے:  چونکہ گیند لڑھکتا ہے اس کی  حرکی توانائی میں مستقیم اور گھمیری جزو شامل ہوں گے، جنہیں شامل کرنے کے لئے مساوات \حوالہ{مساوات_لڑھکاو_مستقیم_گھمیری_الف} کا دایاں ہاتھ استعمال کرتے ہیں۔

\موٹا{حساب:}\quad
مساوات \حوالہ{مساوات_لڑھکاو_نمونی_گیند_الف} میں  ڈالنے سے ذیل حاصل ہو گا:
%eq 11.12
\begin{align}\label{مساوات_لڑھکاو_نمونی_گیند_ب}
(\frac{1}{2}I_{\text{\RL{مرکزکمیت}}}\omega^2+\frac{1}{2}Mv_{\text{\RL{مرکزکمیت}}}^2)+0=0+Mgh
\end{align}
جہاں گیند کے مرکز کمیت سے گزرتی محور پر گیند کا گھمیری جمود \عددی{I_{\text{\RL{مرکزکمیت}}}} ، زمین پر پہنچ کر گیند کی رفتار  (جو ہم تلاش کرنا چاہتے ہیں)  \عددی{v_{\text{\RL{مرکزکمیت}}}} ، اور  زمین پر پہنچ کر زاوی رفتار \عددی{\omega} ہے۔

چونکہ گیند ہموار لڑھکتا ہے، ہم مساوات \حوالہ{مساوات_لڑھکاو_فاصلہ_زاویہ_ب} استعمال کر کے \عددی{\omega} کی جگہ \عددی{v_{\text{\RL{مرکزکمیت}}}\!/\!R} پُر کر کے مساوات \حوالہ{مساوات_لڑھکاو_نمونی_گیند_ب} میں نامعلوم متغیرات کی تعداد کم کر سکتے ہیں۔ ایسا  کر کے، اور جدول  \حوالہء{10.2f}  سے \عددی{I_{\text{\RL{مرکزکمیت}}}}  کی جگہ \عددی{\tfrac{2}{5}MR^2} ڈال کر  \عددی{v_{\text{\RL{مرکزکمیت}}}}  کے لئے حل کرنے سے ذیل حاصل ہو گا۔
\begin{align*}
v_{\text{\RL{}}}&=\sqrt{(\tfrac{10}{7})gh}=\sqrt{(\tfrac{10}{7})(\SI{9.8}{\meter\per\second\squared})(\SI{1.20}{\meter})}\\
&=\SI{4.10}{\meter\per\second}\quad\quad\text{\RL{(جواب)}}
\end{align*}
یاد رہے، جواب \عددی{M} اور \عددی{R} پر منحصر نہیں۔

(ب)  میلان پر لڑھک کر اترنے کے دوران گینف پر رگڑی قوت  کی قدر اور رخ کیا ہیں؟

\جزوحصہء{کلیدی تصور}
چونکہ گیند ہموار لڑھکتا ہے، مساوات \حوالہ{مساوات_لڑھکاو_زاوی_روپ_ب}  گیند پر رگڑی قوت دیگی۔

\موٹا{حساب:}\quad
مساوات \حوالہ{مساوات_لڑھکاو_زاوی_روپ_ب} استعمال کرنے سے قبل ہمیں  مساوات \حوالہ{مساوات_لڑھکاو_اسراع} سے گیند کا اسراع  معلوم کرنا ہو گا۔
\begin{align*}
a_{\text{\RL{مرکزکمیت}},x}&=-I_{\text{\RL{مرکزکمیت}}}\frac{a_{\text{\RL{مرکزکمیت}},x}}{R^2}
=-\frac{2}{5}MR^2\frac{a_{\text{\RL{مرکزکمیت}}},x}{R^2}=-\frac{2}{5}Ma_{\text{\RL{مرکزکمیت}},x}\\
&=-\frac{2}{5}(\SI{6.00}{\kilo\gram})(\SI{-3.50}{\meter\per\second\squared})=\SI{8.40}{\newton}\quad\quad\text{\RL{(جواب)}}
\end{align*}
یاد رہے ہمیں کمیت \عددی{M} درکار تھی  جبکہ رداس \عددی{R}   نہیں تھا۔ یوں، \عددی{\SI{30}{\degree}} میلان پر \عددی{\SI{6.00}{\kilo\gram}}  ہموار لڑھکتے گیند پر، گیند کے رداس سے قطع نظر،  رگڑی قوت \عددی{\SI{8.40}{\newton}} ہو گی، تاہم بڑی کمیت کی صورت میں رگڑی قوت زیادہ ہو گی۔
 \انتہا{نمونی سوال}
 %------------------------
 
 %section 11.3 YO-YO p301
 \حصہ{ڈوری دار لٹو}
 \موٹا{مقاصد}
 اس حصے کو پڑھنے کے بعد آپ  ذیل کے قابل ہوں گے۔
 \begin{enumerate}[1.]
 \item
 ڈوری پر اوپر نیچے حرکت کرتے\اصطلاح{ ڈوری دار لٹو }\فرہنگ{لٹو!ڈوری دار}\حاشیہب{Yo-Yo}\فرہنگ{Yo-Yo} کا آزاد جسمی خاکہ بنا پائیں گے۔
 \item
 جان پائیں گے کہ ڈوری دار لٹو  ، ایسا جسم ہے جو \عددی{\SI{90}{\degree}} زاویہ میلان پر ہموار  اوپر نیچے لڑھکتا ہے۔
 \item
 ڈوری پر اوپر نیچے حرکت کرتے ڈوری دار لٹو کے اسراع اور گھمیری جمود کا تعلق استعمال کر پائیں گے۔
 \item
ڈوری پر اوپر یا نیچے حرکت کے دوران ڈوری دار لٹو کی ڈور میں تناو تعین کر پائیں گے۔
 \end{enumerate}
 
 \موٹا{کلیدی تصور}
 \begin{itemize}
 \item
 ڈوری دار لٹو جو ڈور پر اوپر یا نیچے حرکت کرتا ہو کو \عددی{\SI{90}{\degree}} میلان پر  ہموار لڑھکتا جسم تصور کیا جا سکتا ہے۔
 \end{itemize}
 
 \جزوحصہء{ڈوری دار لٹو}
ڈوری پر \عددی{h} فاصلہ اتر کر ڈوری دار لٹو کی مخفی توانائی میں \عددی{mgh} کمی  واقع ہو گی جبکہ  اس کی حرکی توانائی   کے مستقیم جزو \عددی{(\tfrac{1}{2}Mv_{\text{\RL{مرکزکمیت}}}^2)} اور گھمیری جزو \عددی{(\tfrac{1}{2}I_{\text{\RL{مرکزکمیت}}}\omega^2)} میں  اضافہ ہو گا۔

ڈوری دار لٹو کی ایک نئی قسم میں ڈور  کو  دھرے کے ساتھ سخت  باندھنے  کے  بجائے   ڈور  کو دھرے کے گرد   ڈھیلا گھیرا دیا جاتا ہے۔ جب لٹو نیچے اترتے ہوئے   ڈور کے پیندا   کو\قول{  ٹکرا تا} ہے، دھرے پر ڈور  اوپر وار قوت لاگو کر کے  لٹو کی نشیبی حرکت روکتی ہے۔ اس کے بعد لٹو صرف گھمیری حرکی توانائی کے ساتھ (دھرا گھیر میں چکر کاٹتا ہوا) گھومتا ہے۔ لٹو      (\قول{ سوتے ہوئے } ) چکر  کاٹتا رہتا ہے ؛ ڈور  کو جھٹکا  دینے پر      ڈور دھرے  کو پکڑتی ہے ، \قول{لٹو  بیدار ہوتا  ہے}، اور  اوپر چڑھنا شروع کرتا ہے۔ ڈور کے  پیندا پر لٹو کی گھمیری حرکی توانائی (اور یوں سونے کا دورانیہ) بڑھانے کی خاطر  لٹو کو   ساکن حالت  سے روانا کرنے کی بجائے ابتدائی رفتار \عددی{v_{\text{\RL{مرکزکمیت}}}} اور \عددی{\omega} کے ساتھ نشیب وار پھینکا جاتا ہے۔

ڈور پر نشیب وار اترنے کے دوران لٹو کا خطی اسراع \عددی{a_{\text{\RL{مرکزکمیت}}}}   جاننے کے لئے ،شکل \حوالہء{11.8} میں  میلان پر اترتے لڑھکتے جسم کی طرح،  نیوٹن کا قانون دوم (خطی اور گھمیری روپ میں)  استعمال کیا جا سکتا ہے۔ ماسوائے ذیل ،  تجزیہ بالکل اسی طرح ہو گا۔
\begin{enumerate}[1.]
\item
افق کے ساتھ \عددی{\theta} زاویے کے میلان پر اترنے کے بجائے ڈوری دار لٹو افق کے ساتھ  \عددی{\SI{90}{\degree}}  زاویے کی ڈور پر اترتا ہے۔
\item
رداس \عددی{R} کی بیرونی سطح  پر لڑھکنے کے بجائے ڈوری دار لٹو رداس \عددی{R_0} کے دھرے پر لڑھکتا ہے (شکل \حوالہء{11.9a})۔
\item
رگڑی قوت \عددی{\vec{f}_s} کے بجائے، ڈوری دار لٹو کو ڈور کا تناو \عددی{\vec{T}} آہستہ کرتا ہے (شکل \حوالہء{11.9b})۔
\end{enumerate}

موجودہ تجزیہ بھی مساوات \حوالہ{مساوات_لڑھکاو_اسراع} دے گا۔ آئیں مساوات \حوالہ{مساوات_لڑھکاو_اسراع} کی ترقیم  تبدیل 
کر کے اور  \عددی{\theta=\SI{90}{\degree}}  ڈال کر خطی اسراع ذیل لکھتے ہیں:
%eq 11.13
\begin{align}
a_{\text{\RL{مرکزکمیت}}}=-\frac{g}{1+I_{\text{\RL{مرکزکمیت}}}\!/\!MR_0^2}
\end{align}
جہاں  لٹو کے مرکز کمیت پر لٹو کا گھمیری جمود \عددی{I_{\text{\RL{مرکزکمیت}}}} اور کمیت \عددی{M} ہے۔    ڈوری پر اوپر چڑھنے کے دوران   ڈوری دار لٹو کا اسراع یہی   نشیبی اسراع ہو گا۔

%section 11.4 Torque Revisited p302
\حصہ{قوت مروڑ پر نظر ثانی}
\موٹا{مقاصد}\\
اس حصہ کو پڑھنے کے بعد آپ ذیل کے قابل ہوں گے۔
\begin{enumerate}[1.]
\item
جان پائیں گے کہ قوت مروڑ ایک  سمتیہ  مقدار ہے۔
\item
جان پائیں گے کہ جس نقطہ پر قوت مروڑ  تعین کیا جائے اس  کا   ذکر صریحاً  کرنا   لازم ہے۔
\item
ذرے پر عمل پیرا قوت کی ذرے پر قوت مروڑ ، اکائی سمتیہ ترقیم یا قدر و زاویہ ترقیم  کے روپ میں،  ذرے کے تعین گر سمتیہ  اور قوت سمتیہ  کے صلیبی ضرب سے حاصل  کر پائیں گے۔
\item
صلیبی ضرب   کا دایاں ہاتھ قاعدہ استعمال کر کے قوت مروڑ کا رخ تعین کر پائیں گے۔
\end{enumerate}

\موٹا{کلیدی تصورات}\\
\begin{itemize}
\item
تین ابعاد میں، قوت مروڑ \عددی{\vec{\tau}}ایک  سمتیہ مقدار  ہو گی  ، جو کسی مقررہ نقطہ (عموماً مبدا)  کے لحاظ سے تعین کی جاتی ہے؛ اس کی تعریف ذیل ہے:
\begin{align*}
\vec{\tau}=\vec{r}\times \vec{F}
\end{align*}
جہاں \عددی{\vec{F}} ذرے پر لاگو قوت اور  \عددی{\vec{r}}  کسی مقررہ نقطے کے لحاظ سے ذرے کا تعین گر سمتیہ ہے، جو ذرے کا مقام دیتا ہے۔
\item
قوت مروڑ \عددی{\vec{\tau}} کی قدر  \عددی{\tau} ذیل ہو گی:
\begin{align*}
\tau=rF\sin\phi=rF_{\perp}=r_{\perp}F
\end{align*}
جہاں \عددی{\vec{F}} اور \عددی{\vec{r}} کے بیچ زاویہ \عددی{\phi} ہے، \عددی{\vec{r}} کو \عددی{\vec{F}} کا عمود دار جزو \عددی{F_{\perp}}، اور \عددی{\vec{F}}  کا معیار اثر کا بازو \عددی{r_{\perp}} ہے۔
\item
قوت مروڑ \عددی{\vec{\tau}} کا رخ صلیبی ضرب کا دایاں ہاتھ قاعدہ  دیگا۔
\end{itemize}

%Torque Revisited p303
\جزوحصہء{قوت مروڑ پر نظر ثانی}
باب \حوالہ{باب_گھماو} میں مقررہ محور کے گرد گھومنے کے قابل  استوار جسم   کے لئے قوت مروڑ \عددی{\tau} کی تعریف پیش کی گئی۔ ہم قوت مروڑ کی تعریف کو وسعت دے کر  (مقررہ محور کے بجائے)  مقررہ  \ترچھا{نقطے  } کے لحاظ سے کسی بھی راہ پر حرکت کرتے ہوئے  انفرادی ذرے   کے لئے استعمال کرتے ہیں۔ راہ کا دائری ہونا ضروری نہیں، اور ہم قوت مروڑ کو سمتیہ \عددی{\vec{\tau}} لکھتے ہیں جس کا رخ کچھ بھی ہو سکتا ہے۔قوت مروڑ کی قدر کلیہ سے اور رخ صلیبی ضرب کے دایاں ہاتھ قاعدہ سے حاصل  کیا جا سکتا ہے۔

 شکل \حوالہء{11.10a} میں ، نقطہ \عددی{A}   پر مستوی \عددی{xy} میں  ایسا  ایک ذرہ دکھایا گیا ہے۔  ذرے پر ، مستوی میں قوت،  \عددی{\vec{F}} عمل کرتی ہے، اور  مبدا  \عددی{O} کے لحاظ سے ذرے کا مقام   تعین گر سمتیہ \عددی{\vec{r}} دیتا ہے۔ مقررہ نقطہ \عددی{O} کے لحاظ سے  ذرے پر عمل پیرا \اصطلاح{ قوت مروڑ }\فرہنگ{قوت مروڑ!تعریف}\فرہنگ{torque!defined} \عددی{\vec{\tau}}  کی تعریف ذیل ہے۔
 %eq 11.14
 \begin{align}\label{مساوات_لڑھکاو_سمتیہ_زاوی_قوت_مروڑ}
 \vec{\tau}=\vec{r}\times \vec{F}\quad\quad\text{\RL{قوت مروڑ کی تعریف}}
 \end{align}
 
 قوت مروڑ \عددی{\vec{\tau}}  کی اس تعریف میں سمتی (صلیبی) ضرب کی تحسیب  حصہ \حوالہء{3.3}  کے قواعد سے کی جا سکتی ہے۔ \عددی{\vec{\tau}} کا رخ جاننے کے لئے ، سمتیہ   \عددی{\vec{F}}  کو( رخ تبدیل کیے بغیر)     کھسکا  کر  اس کی دم مبدا \عددی{O} پر  رکھی جاتی ہے؛یوں ، جیسا شکل \حوالہء{11.10b} میں دکھایا گیا ہے، سمتی ضرب کے دونوں سمتیات کی  دم ایک نقطے  پر  ہو گی۔ اب ہم شکل \حوالہء{3.19a}  میں پیش  دایاں ہاتھ قاعدہ   استعمال کرتے ہوئے، دائیں ہاتھ کی چار انگلیاں \عددی{\vec{r}} پر رکھ کر  (ضرب میں پہلا سمتیہ ہے) \عددی{\vec{F}}  کی طرف جکھاتے ہیں (جو ضرب میں دوسرا سمتیہ ہے)۔سیدھا کھڑا انگوٹھا \عددی{\vec{\tau}} کا رخ دیگا۔ شکل \حوالہء{11.10b} میں  \عددی{\vec{\tau}} کا رخ محور \عددی{z} کے مثبت رخ ہے۔
 
 \عددی{\vec{\tau}} کی قدر جاننے کے لئے، ہم مساوات \حوالہء{3.27}  \عددی{(c=ab\sin\phi)} کا عمومی نتیجہ بروئے کار لاتے ہیں، جو ذیل دیگا:
 %eq 11.15
 \begin{align}\label{مساوات_لڑھکاو_صلیبی_ضرب_قدر_الف}
 \tau=rF\sin\phi
 \end{align}
 جہاں \عددی{\vec{r}} اور \عددی{\vec{F}}   کے دم ایک نقطے  پر رکھ کر  سمتیات  کے بیچ چھوٹا زاویہ \عددی{\phi} ہے۔ شکل \حوالہء{11.10b} سے ہم دیکھ سکتے ہیں کہ مساوات \حوالہ{مساوات_لڑھکاو_صلیبی_ضرب_قدر_الف} ذیل لکھی جا سکتی ہے:
 %eq 11.16
 \begin{align}\label{مساوات_لڑھکاو_صلیبی_ضرب_قدر_ب}
 \tau=rF_{\perp}
 \end{align}
 جہاں \عددی{F_{\perp}} (جو \عددی{F\sin\phi} کے برابر ہے)  \عددی{\vec{r}} کا \عددی{\vec{F}} کا عمود دار جزو ہے۔ شکل \حوالہء{11.10c} کو دیکھ کر مساوات \حوالہ{مساوات_لڑھکاو_صلیبی_ضرب_قدر_الف} ذیل بھی لکھی جا سکتی ہے:
 %eq 11.17
 \begin{align}\label{مساوات_لڑھکاو_صلیبی_ضرب_قدر_پ}
 \tau=r_{\perp}F
 \end{align}
 جہاں \عددی{r_{\perp}} (جو \عددی{r\sin\phi} کے برابر ہے)  \عددی{\vec{F}} کا معیار اثر کا بازو   (\عددی{\vec{F}} کے  خط عمل اور \عددی{O} کے بیچ عمود دار فاصلہ) ہے۔
 
 %---------------------------------
 %Checkpoint 3 p303
 \ابتدا{آزمائش}
 ذرے  کا تعین گر سمتیہ \عددی{\vec{r}}،   مثبت محور \عددی{z}    کے ہمراہ  پایا جاتا ہے۔ اگر ذرے پر قوت مروڑ (ا)  صفر ہو، (ب) محور \عددی{x} کے منفی رخ ہو، اور (ج) محور \عددی{y} کے منفی رخ ہو، قوت مروڑ پیدا کرنے والی قوت کا رخ کیا ہو گا؟
 \انتہا{آزمائش}
 %-------------------------
 
 %Sample problem 11.02 Torque on a particle due to a force p304
 \ابتدا{نمونی سوال} \موٹا{قوت کی بدولت ذرے پر قوت مروڑ}\\
 شکل \حوالہء{11.11a} میں، \عددی{\SI{2.0}{\newton}} قدر کی تین قوت ذرے پر عمل کرتی ہیں۔ ذرہ ، مستوی \عددی{xy} میں ، نقطہ \عددی{A} پر ہے، جس کا تعین گر سمتیہ \عددی{\vec{r}}، جہاں \عددی{r=\SI{3.0}{\meter}} اور \عددی{\theta=\SI{30}{\degree}} ہے۔ مبدا \عددی{O} کے لحاظ سے  ہر   قوت کی  انفرادی قوت مروڑ  کیا  ہے؟
 
 \جزوحصہء{کلیدی تصور}
 چونکہ قوت ایک مستوی میں نہیں پائی جاتیں،  ہمیں صلیبی ضرب استعمال کرنا ہو گی، جس کی قدر مساوات \حوالہ{مساوات_لڑھکاو_صلیبی_ضرب_قدر_الف} \عددی{(\tau=rF\sin\phi)}   دیگی اور رخ دایاں ہاتھ قاعدہ  دیگا۔
 
 \موٹا{حساب:}\quad
 ہم  مبدا  \عددی{O}  کے لحاظ سے قوت مروڑ جاننا چاہتے ہیں لہٰذا  دیا گیا تعین گر سمتیہ     صلیبی ضرب میں درکار سمتیہ \عددی{\vec{r}}ہو گا۔ قوت اور \عددی{\vec{r}} کے بیچ زاویہ \عددی{\phi} جاننے کے لئے ہم  شکل \حوالہء{11.11a} میں دیے گئے سمتیہ   قوت باری باری  یوں  کھسکاتے  ہیں کہ ان کی دم \عددی{O} پر ہو۔ انتقال کے بعد قوت \عددی{\vec{F}_1}، \عددی{\vec{F}_2}، اور \عددی{\vec{F}_3}  بالترتیب شکل \حوالہء{11.11b}،  شکل \حوالہء{11.11c}، اور  شکل \حوالہء{11.11d} میں، جو مستوی \عددی{xz}  کا نظارہ دیتی ہیں، دکھائی گئی ہیں (جن میں سمتیہ قوت اور تعین گر سمتیہ کے بیچ زاویے   باآسانی نظر آتے ہیں)۔ شکل \حوالہء{11.11d} میں  \عددی{\vec{r}} اور \عددی{\vec{F}_3} کے رخ کے بیچ زاویہ \عددی{\SI{90}{\degree}} ہے اور علامت  \عددی{\otimes} کہتی ہے \عددی{\vec{F}_3} صفحہ میں عمود دار  اندر رخ ہے۔ (صفحہ سے عمود دار نکلنے کی صورت میں \عددی{\odot} علامت استعمال کی جاتی ہے۔)
 
 مساوات \حوالہ{مساوات_لڑھکاو_صلیبی_ضرب_قدر_الف}  استعمال کر ذیل حاصل ہو گا۔
 \begin{align*}
 \tau_1&=rF_1\sin\phi_1=(\SI{3.0}{\meter})(\SI{2.0}{\newton})(\sin \SI{150}{\degree})=\SI{3.0}{\newton\meter}\\
 \tau_2&=rF_2\sin\phi_2=(\SI{3.0}{\meter})(\SI{2.0}{\newton})(\sin \SI{120}{\degree})=\SI{5.2}{\newton\meter}\\
 \tau_3&=rF_3\sin\phi_3=(\SI{3.0}{\meter})(\SI{2.0}{\newton})(\sin \SI{90}{\degree})=\SI{6.0}{\newton\meter}\quad \quad\text{\RL{(جواب)}}
 \end{align*}
 
اب دائیں ہاتھ قاعدہ  استعمال کرتے ہوئے،دائیں  ہاتھ کی چار انگلیاں   \عددی{\vec{r}} کے رخ    رکھ کر \عددی{\vec{F}}     کے رخ  (سمتیات کے رخ کے بیچ چھوٹے زاویے)گھماتے ہیں۔ دائیں ہاتھ کا انگوٹھا   ، جو چار انگلیوں کو عمود دار رکھا گیا ہے، قوت مروڑ کا رخ دیگا۔ یوں   \عددی{\vec{\tau}_1} شکل \حوالہء{11.11b} میں  صفحے کے اندر   جانے  کے رخ   ہو گا؛   \عددی{\vec{\tau}_2} شکل \حوالہء{11.11c} میں صفحہ سے باہر نکلنے کے رخ ہو گا؛ اور  \عددی{\vec{\tau}_3}  کا رخ شکل \حوالہء{11.11d}  میں دکھایا گیا ہے۔ تینوں قوت مروڑ سمتیات شکل \حوالہء{11.11e} میں پیش ہیں۔
 \انتہا{نمونی سوال}
 %-----------------------
 
 
 %section 11.5 Angular Momentum 0305
 \حصہ{زاوی معیار حرکت}
 \موٹا{مقاصد}\\
 اس حصہ کو پڑھنے کے بعد آپ ذیل کے قابل ہوں گے۔
 \begin{enumerate}[1.]
 \item
 جان پائیں گے کہ زاوی معیار حرکت ایک سمتیہ مقدار ہے۔
 \item
 جان پائیں گے کہ جس مقررہ نقطے کے لحاظ سے زاوی معیار حرکت   تعین کیا جائے اس  کا   ذکر صریحاً  کرنا   لازم ہے۔
 \item
 اکائی سمتیہ ترقیم  یا قدر و زاویہ ترقیم میں، ذرے کے تعین گر سمتیہ اور  معیار حرکت سمتیہ کا صلیبی ضرب لے کر ذرے کا زاوی معیار حرکت تعین کر پائیں گے۔
 \item
 صلیبی ضرب کا دایاں ہاتھ قاعدہ استعمال کر کے زاوی معیار حرکت کا رخ تعین کر پائیں گے۔
 \end{enumerate}
 
 \موٹا{کلیدی تصورات}\\
 \begin{itemize}
 \item
  ایک ذرہ، جس کا خطی معیار حرکت \عددی{\vec{p}} ، کمیت \عددی{m}، اور خطی سمتی رفتار \عددی{\vec{v}} ہو، کا مقررہ نقطے کے لحاظ سے (جو عموماً مبدا ہو گا)  زاوی معیار حرکت \عددی{\vec{\ell}} کی تعریف ذیل سمتی  مقدار ہے۔
  \begin{align*}
  \vec{\ell}=\vec{r}\times \vec{p}=m(\vec{r}\times \vec{v})
  \end{align*}
  \item
  زاوی معیار حرکت \عددی{\vec{\ell}}   کی قدر \عددی{\ell} ذیل ہو گی:
  \begin{align*}
  \ell&=rmv\sin\phi\\
  &=rp_{\perp}=rmv_{\perp}\\
  &=r_{\perp}p=r_{\perp}mv
  \end{align*}
  جہاں \عددی{\vec{r}} اور \عددی{\vec{p}} کے بیچ زاویہ \عددی{\phi} ہے، \عددی{\vec{r}} کو \عددی{\vec{p}} اور \عددی{\vec{v}} کے عمود دار
   جزو \عددی{p_{\perp}} اور \عددی{v_{\perp}} ہیں، اور مقررہ نقطے سے مبسوط  \عددی{\vec{p}}  تک عمود دار فاصلہ \عددی{r_{\perp}} ہے۔
   \item
   دایاں ہاتھ قاعدہ \عددی{\vec{\ell}} کا رخ دیگا: دائیں ہاتھ کی چار انگلیاں \عددی{\vec{r}} کے رخ پر (ابتدائی طور)   رکھ کر  انہیں گھما کر  \عددی{\vec{p}} کے رخ    پر  رکھیں۔دائیں ہاتھ کا سیدھا کھڑا انگوٹھا \عددی{\vec{\ell}} کا رخ دیگا۔
 \end{itemize}
 
 %Angular Momentum p305
 \جزوحصہء{زاوی معیار حرکت}
 یاد کریں، خطی معیار حرکت \عددی{\vec{p}} اور خطی معیار حرکت کی بقا کا اصول انتہائی طاقتور  اوزار ہیں۔انہیں استعمال کر کے نتائج  کی  ، مثلاً دو گاڑیوں کے تصادم کی تفصیل جانے بغیر  تصادم کی ،  پیشنگوئی کی جا سکتی ہے۔یہاں ہم \عددی{\vec{p}} کے زاوی  مدمقابل  پر تبصرہ  شروع کرتے ہیں جس کا اختتام حصہ \حوالہء{11.8} میں  بقائی اصول کے مدمقابل پر ہو گا۔
 
 شکل \حوالہء{11.12} میں  مستوی \عددی{xy} میں نقطہ \عددی{A} سے  کمیت \عددی{m}   اور خطی معیار حرکت \عددی{\vec{p}} \عددی{(m\vec{v}=)}  کا ذرہ گزرتا دکھایا گیا ہے۔ مبدا \عددی{O} کے لحاظ سے ذرے کا \اصطلاح{ زاوی معیار حرکت}\فرہنگ{معیار حرکت!زاوی، تعریف}\حاشیہب{angular momentum}\فرہنگ{momentum!angular, defined}  \عددی{\vec{\ell}}  سمتیہ مقدار ہو گا جس کی تعریف ذیل ہے،
 %eq 11.18
 \begin{align}\label{مساوات_لڑھکاو_سمتیہ_زاوی_قوت_مروڑ_دوم}
 \vec{\ell}=\vec{r}\times \vec{p}=m(\vec{r}\times \vec{v})\quad\quad\text{\RL{(زاوی معیار حرکت کی تعریف)}}
 \end{align}
 جہاں  \عددی{O} کے لحاظ سے ذرے کا تعین گر سمتیہ \عددی{\vec{r}} ہے۔مبدا \عددی{O}  کے لحاظ سے جب  ذرہ معیار حرکت  \عددی{\vec{p}} \عددی{(m\vec{v}=)}  کے رخ   کرتا ہے،  اس کا تعین گر سمتیہ  \عددی{\vec{r}} مبدا \عددی{O} کے گرد گھمیری حرکت کرتا  ہے۔ غور کریں، \عددی{O}   پر زاوی معیار حرکت کے لئے ضروری نہیں کہ ذرہ خود  \عددی{O} کے گرد گھومتا ہو۔ مساوت \حوالہ{مساوات_لڑھکاو_سمتیہ_زاوی_قوت_مروڑ} اور مساوات \حوالہ{مساوات_لڑھکاو_سمتیہ_زاوی_قوت_مروڑ_دوم}  کا موازنہ کرنے سے معلوم ہو گا کہ زاوی معیار حرکت اور خطی معیار حرکت کا آپس میں وہی رشتہ ہے جو قوت مروڑ کا قوت  کے ساتھ ہے۔ بین الاقوامی نظام اکائی میں زاوی معیار حرکت کی اکائی کلوگرام مربع   میٹر  فی سیکنڈ \عددی{(\si{\kilo\gram\meter\squared\per\second})} ہے، جو جاول سیکنڈ \عددی{(\si{\joule\second})} کا معادل ہے۔
 
 \موٹا{رخ۔}\quad
 شکل \حوالہء{11.12} میں  زاوی معیار حرکت سمتیہ \عددی{\vec{\ell}} کا رخ جاننے کے لئے، ہم  سمتیہ \عددی{\vec{p}} کو کھسکا  کر کے اس کی دم مبدا \عددی{O}  پر رکھتے ہیں۔اس کے بعد صلیبی ضرب کا  دایاں ہاتھ قاعدہ استعمال کر کے انگلیوں  کو \عددی{\vec{r}} سے \عددی{\vec{p}}  لپیٹتے ہیں۔ سیدھا کھڑا انگوٹھا \عددی{\vec{\ell}} کا رخ ، شکل \حوالہء{11.12} میں ، محور \عددی{z}   کا مثبت رخ دیتا ہے۔ یہ مثبت رخ،  محور \عددی{z} پر تعین گر سمتیہ  \عددی{\vec{r}} کے خلاف گھڑی گھماو کے عین مطابق ہے، جو ذرے کی حرکت سے پیدا ہوتی ہے۔ (\عددی{\vec{\ell}} کی منفی قیمت محور \عددی{z} پر گھڑی وار  گھماو ظاہر کرے گی۔)
 
 \موٹا{قدر۔}\quad
     زاوی معیار حرکت \عددی{\vec{\ell}}  کی قدر  معلوم  کرنے کے لئے ہم   مساوات \حوالہء{3.27} کا عمومی نتیجہ  ذیل لکھتے ہیں:
 %eq 11.19
 \begin{align}\label{مساوات_لڑھکاو_زاوی_قدر_معیار_حرکت}
 \ell=rmv\sin\phi
 \end{align}
 جہاں \عددی{\vec{r}} اور \عددی{\vec{p}}  کی دم ایک نقطہ پر رکھ کر  سمتیات کے بیچ چھوٹا زاویہ \عددی{\phi} ہے۔ شکل \حوالہء{11.12a} دیکھ کر مساوات \حوالہ{مساوات_لڑھکاو_زاوی_قدر_معیار_حرکت} ذیل لکھی جا سکتی ہے:
 %eq 11.20
 \begin{align}
 \ell=rp_{\perp}=rmv_{\perp}
 \end{align}
 جہاں \عددی{\vec{r}} کو  \عددی{\vec{p}} کا عمود دار  جزو \عددی{p_{\perp}} ہے، اور  \عددی{\vec{r}} کو  \عددی{\vec{v}} کا عمود دار  جزو \عددی{v_{\perp}} ہے۔ شکل \حوالہء{11.12b} دیکھ کر مساوات \حوالہ{مساوات_لڑھکاو_زاوی_قدر_معیار_حرکت} ذیل بھی لکھی جا سکتی ہے:
 %eq 11.21
 \begin{align}\label{مساوات_لڑھکاو_سمتیہ_زاوی_قوت_مروڑ_چہارم}
 \ell=r_{\perp}p=r_{\perp}mv
 \end{align}
 جہاں مبسوط   \عددی{\vec{p}}  سے \عددی{O} کا  عمود دار فاصلہ \عددی{r_{\perp}} ہے۔
 
 \موٹا{اہم۔}\quad
 دو  پہلو پر غور کریں: (1)  زاوی معیار حرکت صرف  کسی مخصوص مبدا کے لحاظ سے معنی خیز ہے اور (2)  اس کا رخ ہر صورت اس مستوی کو عمودی ہو گا جو   تعین گر سمتیہ \عددی{\vec{r}} اور  خطی معیار حرکت  سمتیہ  \عددی{\vec{p}} مل کر بناتے ہیں۔
 
 %------------------------------
 %Checkpoint 4 p305
 \ابتدا{آزمائش}
 شکل \حوالہء{؟؟}    ا میں  ذرہ \عددی{1} اور \عددی{2} نقطہ \عددی{O} کے گر بالترتیب د رداس \عددی{\SI{2}{\meter}} اور \عددی{\SI{4}{\meter}} کے دائروں پر حرکت کرتے ہیں۔ شکل ب میں ذرہ \عددی{3} اور \عددی{4}  نقطہ \عددی{O} سے بالترتیب  \عددی{\SI{4}{\meter}} اور \عددی{\SI{2}{\meter}}  عمود دار فاصلوں پر  خط مستقیم پر حرکت کرتے ہیں۔ ذرہ \عددی{5} نقطہ \عددی{O} سے باہری رخ حرکت کرتا ہے۔ تمام  ذروں کی کمیت اور رفتار برابر ہیں۔ (ا)  نقطہ \عددی{O} پر زاوی معیار حرکت کے لحاظ سے ، اعظم اول رکھ کر، ذروں کی درجہ بندی کریں۔ (ب)  نقطہ \عددی{O} پر کن ذروں کا زاوی معیار حرکت منفی ہے؟
 \انتہا{آزمائش}
 %---------------------------
 
 %Sample Problem 11.03 Angular momentum of a two-particle system p306
 \ابتدا{نمونی سوال}\موٹا{دو ذروی نظام کا زاوی معیار حرکت}\\
 افقی راہوں پر  دو ذرے مستقل معیار حرکت  کے ساتھ  حرکت کرتے ہیں ۔شکل \حوالہء{11.13} میں ان کا فضائی جائزہ پیش ہے۔ ذرہ \عددی{1} ، جس کے معیار حرکت کی قدر \عددی{p_1=\SI{5.0}{\kilo\gram\meter\per\second}} اور تعین گر سمتیہ \عددی{\vec{r}_1} ہے، نقطہ \عددی{O} سے \عددی{\SI{2.0}{\meter}}  فاصلے پر گزرے گا۔  ذرہ \عددی{2} ، جس کے معیار حرکت کی قدر \عددی{p_2=\SI{2.0}{\kilo\gram\meter\per\second}} اور تعین گر 
 سمتیہ \عددی{\vec{r}_2} ہے، نقطہ \عددی{O} سے \عددی{\SI{4.0}{\meter}}  فاصلے پر گزرے گا۔ دو ذروی نظام کا نقطہ \عددی{O} پر صافی زاوی معیار حرکت \عددی{\vec{L}} کیا ہو گا؟
 
 \جزوحصہء{کلیدی تصور}
 انفرادی زاوی معیار حرکت \عددی{\vec{\ell}_1} اور \عددی{\vec{\ell}_2} معلوم کر نے کے بعد  جمع  کر کے   ہم صافی معیار حرکت \عددی{\vec{L}} تلاش کر  سکتے ہیں۔ ان کی قدریں  مساوات \حوالہ{مساوات_لڑھکاو_سمتیہ_زاوی_قوت_مروڑ_دوم}  تا مساوات \حوالہ{مساوات_لڑھکاو_سمتیہ_زاوی_قوت_مروڑ_چہارم} میں  ہر ایک سے تعین کی جا سکتی ہیں۔ البتہ، ہمیں عمود دار فاصلے \عددی{r_{\perp 1}} \عددی{(\SI{2.0}{\meter}=)} اور
  \عددی{r_{\perp 2}} \عددی{(\SI{4.0}{\meter}=)} اور معیار حرکت کی قدریں \عددی{p_1} اور \عددی{p_2} دی گئی ہیں لہٰذا   مساوات \حوالہ{مساوات_لڑھکاو_سمتیہ_زاوی_قوت_مروڑ_چہارم} کا استعمال زیادہ آسان ہو گا۔
  
  \موٹا{حساب:}\quad
  ذرہ \عددی{1} کے لئے مساوات \حوالہ{مساوات_لڑھکاو_سمتیہ_زاوی_قوت_مروڑ_چہارم} ذیل دیگی۔
  \begin{align*}
  \ell_1&=r_{\perp 1}p_1=(\SI{2.0}{\meter})(\SI{5.0}{\kilo\gram\meter\per\second})\\
  &=\SI{10}{\kilo\gram\meter\squared\per\second} 
  \end{align*}
  سمتیہ \عددی{\vec{\ell}_1} کا رخ مساوات \حوالہ{مساوات_لڑھکاو_سمتیہ_زاوی_قوت_مروڑ_دوم}   اور سمتیات کے صلیبی ضرب کا دایاں ہاتھ قاعدہ  دے گا۔ صلیبی ضرب \عددی{\vec{r}_1\times \vec{p}_1} صفحہ سے باہر  نکلنے کے رخ، شکل \حوالہء{11.13}  کے مستوی کو عمود دار ہو گا۔یہ مثبت رخ ہے، جو ذرے  کی حرکت  کے دوران ذرے کے تعین گر سمتیہ  \عددی{\vec{r}_1} کا نقطہ \عددی{O} کے گرد خلاف گھڑی گھماو کے عین مطابق ہے۔  یوں ذرہ \عددی{1} کا زاوی معیار حرکت سمتیہ ذیل ہو گا۔
  \begin{align*}
  \ell_1=+\SI{10}{\kilo\gram\meter\squared\per\second}
  \end{align*}
  اسی طرح \عددی{\vec{\ell}_2} کی قدر  ذیل
  \begin{align*}
  \ell_2&=r_{\perp 2}p_2=(\SI{4.0}{\meter})(\SI{2.0}{\kilo\gram\meter\per\second})\\
  &=\SI{8.0}{\kilo\gram\meter\squared\per\second}
  \end{align*}
  اور  \عددی{\vec{r}_2\times \vec{p}_2}  سمتیہ  حاصل ضرب صفحہ سے باہر رخ ہے، جو منفی رخ ہے، اور جو ذرہ \عددی{2}  کی حرکت کے دوران \عددی{O} کے گرد \عددی{\vec{r}_2} کے گھڑی وار حرکت کے عین مطابق ہے۔ یوں ذرہ \عددی{2} کا زاوی معیار حرکت سمتیہ ذیل ہو گا۔
  \begin{align*}
  \ell_2=\SI{-8.0}{\kilo\gram\meter\squared\per\second}
  \end{align*}
  دو ذروی نظام کا صافی زاوی معیار حرکت  ذیل ہو گا۔
  \begin{align*}
  L&=\ell_1+\ell_2=+\SI{10}{\kilo\gram\meter\squared\per\second}+(\SI{-8.0}{\kilo\gram\meter\squared\per\second})\\
  &=+\SI{2.0}{\kilo\gram\meter\squared\per\second}\quad\quad\quad\text{\RL{(جواب)}}
  \end{align*}
  مثبت علامت کہتی ہے  \عددی{O} پر  نظام کا صافی معیار حرکت صفحہ سے باہر نکلنے کے رخ ہے۔
 \انتہا{نمونی سوال}
 %---------------------
 
 %section 11.6 newton's second law in  angular form p307
 \حصہ{نیوٹن کا قانون دوم، زاوی روپ}
 \موٹا{مقاصد}\\
 اس حصے کو پڑھنے کے بعد آپ ذیل کے قابل ہوں گے۔
 \begin{enumerate}[1.]
 \item
زاوی روپ میں   نیوٹن کا قانون دوم استعمال کر کے  ،   کسی  مخصوص نقطہ کے لحاظ سے، ذرے پر عمل پیرا قوت مروڑ اور   ذرے کے زاوی معیار حرکت میں پیدا  تبدیلی  کا رشتہ جان پائیں گے۔
 \end{enumerate}
 
 \موٹا{کلیدی تصور}\\
 \begin{itemize}
 \item
 نیوٹن کا قانون دوم  کا زاوی روپ ذیل ہے:
 \begin{align*}
 \vec{\tau}_{\text{\RL{صافی}}}=\frac{\dif \vec{\ell}}{\dif t}
 \end{align*}
 جہاں \عددی{\vec{\tau}_{\text{\RL{صافی}}}}  ذرے پر صافی قوت مروڑ اور \عددی{\vec{\ell}} ذرے کا زاوی معیار حرکت ہے۔
 \end{itemize}
 
 %-------------------------
 %Newton's Second Law in Angular Form p307
 \جزوحصہء{نیوٹن کا قانون دوم، زاوی روپ}
 نیوٹن کا قانون دوم ذیل روپ میں:
 %eq 11.22
 \begin{align}\label{مساوات_لڑھکاو_قانون_دوم_خطی}
 \vec{F}_{\text{\RL{}}}=\frac{\dif \vec{p}}{\dif t}\quad\quad \text{\RL{(واحد ذرہ)}}
 \end{align}
واحد ذرے کے لئے،  قوت اور خطی معیار حرکت کے بیچ قریبی رشتہ اجاگر کرتا ہے۔ ہم خطی اور زاوی مقادیر  کی متوازیت  دیکھ چکے ہیں اور توقع کر سکتے ہیں کہ قوت مروڑ اور زاوی معیار حرکت کے بیچ بھی قریبی تعلق ہو گا۔ مساوات \حوالہ{مساوات_لڑھکاو_قانون_دوم_خطی} کو دیکھ کر ہم  ذیل تعلق کی توقع کرتے ہیں۔
%eq 11.23
\begin{align}\label{مساوات_لڑھکاو_قانون_دوم_زاوی}
\vec{\tau}_{\text{\RL{}}}=\frac{\dif \vec{\ell}}{\dif t}\quad\quad \text{\RL{(واحد ذرہ)}}
\end{align}
یقیناً ، مساوات \حوالہ{مساوات_لڑھکاو_قانون_دوم_زاوی} واحد ذرے کے لئے نیوٹن کے قانون دوم کا زاوی روپ ہے:

%---------------------
\ابتدا{قانون}
ذرے پر  تمام قوت مروڑ کا (سمتی) مجموعہ  ذرے کے زاوی معیار حرکت میں تبدیلی کے برابر ہو گا۔
\انتہا{قانون}
%--------------------------

کسی مخصوص نقطہ کے لحاظ سے، جو عموماً  محددی نظام کا مبدا ہو گا،  قوت مروڑ \عددی{\vec{\tau}} اور زاوی معیار حرکت \عددی{\vec{\ell}}  تعین کیے  بغیر مساوات \حوالہ{مساوات_لڑھکاو_قانون_دوم_زاوی} بے معنی ہو گی۔
%-----------------

\جزوجزوحصہء{مساوات \حوالہ{مساوات_لڑھکاو_قانون_دوم_زاوی} کا ثبوت}
ہم مساوات \حوالہ{مساوات_لڑھکاو_سمتیہ_زاوی_قوت_مروڑ_دوم} سے آغاز کرتے ہیں، جو ذرے کے زاوی معیار حرکت کی تعریف ہے:
\begin{align*}
\vec{\ell}=m(\vec{r}\times \vec{v})
\end{align*}
جہاں \عددی{\vec{r}} ذرے کا تعین گر سمتیہ اور \عددی{\vec{v}}  ذرے کی سمتی  رفتار ہے۔ دونوں اطراف  کا تفرق\حاشیہد{سمتی حاصل ضرب کا تفرق لیتے ہوئے مستعمل   مقادیر کی ترتیب برقرار رکھیں۔یوں یہاں   \عددی{\vec{r}} ہمیشہ \عددی{\vec{v}}  سے پہلے ہو گا۔} \عددی{t} کے لحاظ سے لیتے ہیں۔
%eq 11.24
\begin{align}\label{مساوات_لڑھکاو_نیوٹن_دوم_ثبوت_الف}
\frac{\dif \vec{\ell}}{\dif t}=m\big(\vec{r}\times \frac{\dif \vec{v}}{\dif t}+\frac{\dif \vec{r}}{\dif t}\times \vec{v}\big)
\end{align}
البتہ،  \عددی{\dif\vec{v}\!/\!\dif t} ذرے کا  اسراع \عددی{\vec{a}} ، اور \عددی{\dif\vec{r}\!/\!\dif t} ذرے کی سمتی رفتار ہے۔ یوں مساوات \حوالہ{مساوات_لڑھکاو_نیوٹن_دوم_ثبوت_الف}  ذیل لکھی جا سکتی ہے۔
\begin{align*}
\frac{\dif\vec{\ell}}{\dif t}=m(\vec{r}\times \vec{a}+\vec{v}\times \vec{v})
\end{align*}
اب \عددی{\vec{v}\times \vec{v}=0} ہے (چونکہ سمتیہ کا اپنے ساتھ زاویہ صفر ہے لہٰذا  سمتیہ کا  اپنے ساتھ سمتی ضرب ہمیشہ صفر کے برابر ہو گا۔) یوں آخری جزو  خارج  ہو گا اور ذیل رہ جائے گا۔
\begin{align*}
\frac{\dif\vec{\ell}}{\dif t}=m(\vec{r}\times \vec{a})=\vec{r}\times m\vec{a}
\end{align*}
ہم نیوٹن کا قانون دوم \عددی{(\vec{F}_{\text{\RL{صافی}}}=m\vec{a})} استعمال کر کے  \عددی{m\vec{a}} کی جگہ \عددی{\vec{F}_{\text{\RL{صافی}}}}  ڈال کر ذیل حاصل کرتے ہیں۔
%eq 11.25
\begin{align}\label{مساوات_لڑھکاو_نیوٹن_دوم_ثبوت_ب}
\frac{\dif \vec{\ell}}{\dif t}=\vec{r}\times \vec{F}_{\text{\RL{صافی}}}=\sum(\vec{r}\times \vec{F})
\end{align}
یہاں علامت \عددی{\sum} کہتی ہے تمام قوتوں کے  سمتی ضرب \عددی{\vec{r}\times \vec{F}}  کا مجموعہ لینا ہو گا۔ البتہ، مساوات \حوالہ{مساوات_لڑھکاو_سمتیہ_زاوی_قوت_مروڑ} سے ہم جانتے ہیں   (درج بالا)ہر  سمتی ضرب کسی ایک  قوت  سے  وابستہ قوت مروڑ ہو گا۔ یوں، مساوات \حوالہ{مساوات_لڑھکاو_نیوٹن_دوم_ثبوت_ب} ذیل کہتی ہے:
\begin{align*}
\vec{\tau}_{\text{\RL{صافی}}}=\frac{\dif \vec{\ell}}{\dif t}
\end{align*}
جو مساوات \حوالہ{مساوات_لڑھکاو_قانون_دوم_زاوی} ہے، جسے  ہم ثابت کرنا چاہتے تھے۔

%----------------------
%checkpoint 5 p308
\ابتدا{آزمائش}
کل \حوالہء{؟؟} میں کسی ایک لمحے پر   ذرے کا تعین گر سمتیہ \عددی{\vec{r}}، اور  ذرے کو مسرع کرنے والی قوتوں کے چار ممکنہ رخ دیے گئے ہیں۔ تمام قوت سطح \عددی{xy} میں ہیں۔ (ا)  نقطہ \عددی{O} پر ذرے کے زاوی معیار حرکت میں تبدیلی  \عددی{(\dif\vec{\ell}\!/\!\dif t)} کی قدر  کے لحاظ سے، اعظم قیمت  اول رکھ کر،  قوتوں کی درجہ بندی کریں۔ (ب)  نقطہ \عددی{O} پر کونسی قوت  تبدیلی کی منفی شرح دیتی ہے؟
\انتہا{آزمائش}
%--------------------------

%Sample problem 11.04 Torque and the time derivative of angular momentum
\ابتدا{نمونی سوال}\موٹا{قوت مروڑ اور زاوی معیار حرکت کا وقتی تفرق}\\
ایک ذرہ جس کی کمیت  \عددی{\SI{0.500}{\kilo\gram}}   ہے اور جس کا تعین گر سمتیہ ذیل ہے، مستقیم خط پر حرکت میں ہے (شکل \حوالہء{11.14a}):
\begin{align*}
\vec{r}=(-2.00t^2-t) \ihat+5.00\jhat
\end{align*}
جہاں \عددی{\vec{r}} میٹر میں اور \عددی{t} سیکنڈ میں ہے، اور آغاز \عددی{t=0} پر ہوتا ہے۔ تعین گر سمتیہ مبدا سے ذرے  کے مرکز  کی نشاندہی کرتا ہے۔ اکائی سمتیہ ترقیم میں،  ذرے کا  زاوی معیار حرکت \عددی{\vec{\ell}} اور  ذرے پر قوت مروڑ \عددی{\vec{\tau}}  مبدا کے لحاظ سے  (یا مبدا پر) تلاش کریں۔ ذرے کی حرکت کو مد نظر رکھتے ہوئے ان مقادیر کی الجبرائی علامت  کی وجہ پیش کریں۔

\جزوحصہء{کلیدی تصورات}
(1)  جس نقطہ پر ذرے کا زاوی معیار حرکت  تلاش کرنا ہو اس کی نشاندہی کرنا لازم ہے۔ یہاں وہ نقطہ مبدا پر واقع ہے۔ (2)   مساوات \حوالہ{مساوات_لڑھکاو_سمتیہ_زاوی_قوت_مروڑ_دوم} \عددی{(\vec{\ell}=\vec{r}\times \vec{p}=m(\vec{r}\times \vec{v}))}  ذرے کا زاوی معیار حرکت  \عددی{\vec{\ell}}   دیتی ہے۔  (3)  ذرے کے زاوی معیار حرکت سے  وابستہ  علامت (\عددی{+} یا \عددی{-}) ، ذرے کی حرکت کے دوران ذرے کے تعین گر سمتیہ کے (محور گھماو کے گرد)   گھماو کی سمت دیتی ہے:گھڑی وار منفی اور خلاف گھڑی مثبت ہو گا۔ (4)  اگر ذرے پر قوت مروڑ اور ذرے کا  زاوی معیار حرکت \ترچھا{ ایک } نقطہ  پر حاصل کیے گئے ہوں،  تب قوت مروڑ  اور زاوی معیار حرکت کا تعلق مساوات \حوالہ{مساوات_لڑھکاو_قانون_دوم_زاوی}  
\عددی{(\vec{\tau}=\dif\vec{\ell}\!/\!\dif t)} دیگی۔

\موٹا{حساب:}\quad
مساوات \حوالہ{مساوات_لڑھکاو_سمتیہ_زاوی_قوت_مروڑ_دوم} استعمال کر کے مبدا پر زاوی معیار حرکت تلاش کرنے کے لئے ضروری ہے کہ پہلے  تعین گر سمتیہ کا وقتی تفرق لے کر ذرے کی سمتی رفتار  کا الجبرائی فقرہ حاصل کیا جائے۔ مساوات \حوالہء{4.10}  \عددی{(\vec{v}=\dif\vec{r}\!/\!\dif t)}  کو دیکھ کر ہم ذیل لکھتے ہیں:
\begin{align*}
\vec{v}&=\frac{\dif}{\dif t}((-2.00t^2-t)\ihat+5.00\jhat\,)\\
&=(-4.00t-1.00)\ihat
\end{align*}
جہاں \عددی{\vec{v}} میٹر فی سیکنڈ میں ہے۔

اس کے بعد مساوات \حوالہء{3.27} میں صلیبی ضرب کا  دکھایا گیا ڈھانچہ   استعمال  کر  کے \عددی{\vec{r}} اور \عددی{\vec{v}} کا صلیبی ضرب معلوم کرتے ہیں۔
\begin{align*}
\vec{a}\times \vec{b}=(a_yb_z-b_ya_z)\ihat+(a_zb_x-b_za_x)\jhat+(a_xb_y-b_xa_y)\khat
\end{align*}
یہاں \عددی{\vec{r}}  کو  عمومی سمتیہ  \عددی{\vec{a}} اور \عددی{\vec{v}}  کو  عمومی سمتیہ  \عددی{\vec{b}}  ظاہر کرتا ہے۔ چونکہ، ہم ضرورت سے زیادہ کام نہیں کرنا چاہتے، آئیں عمومی   صلیبی ضرب میں پُر کردہ بدل پر غور کرتے ہیں۔ چونکہ  \عددی{\vec{r}} میں  \عددی{z} جزو اور \عددی{\vec{v}} میں \عددی{y} اور \عددی{z} اجزاء نہیں پائے جاتے، اس عمومی صلیبی ضرب  کا  صرف آخری  جزو \عددی{(-b_xa_y)\khat} غیر صفر ہے۔ یوں، زیادہ الجبرائی دوڑ کے بغیر ذیل لکھتے ہیں۔
\begin{align*}
\vec{r}\times \vec{v}=-(-4.00t-1.00)(5.00)\khat=(20.0t+5.00)\khat\,\,\si{\meter\squared\per\second}
\end{align*}
یاد رہے، ہمیشہ کی طرح صلیبی ضرب جو سمتیہ دیتی ہے وہ ابتدائی سمتیات کو عمود دار ہو گا۔

مساوات \حوالہ{مساوات_لڑھکاو_سمتیہ_زاوی_قوت_مروڑ_دوم}   پوری کرنے کے لئے ، کمیت سے ضرب دے کر ذیل حاصل کرتے ہیں۔
\begin{align*}
\vec{\ell}&=(\SI{0.500}{\kilo\gram})[(20.0t+5.00)\khat\,\,\si{\meter\squared\per\second}]\\
&=(10.0t+2.50)\khat\,\,\si{\kilo\gram\meter\squared\per\second}\quad\quad\text{\RL{(جواب)}}
\end{align*}
مبدا  پر قوت مروڑ اب مساوات \حوالہ{مساوات_لڑھکاو_قانون_دوم_زاوی}   سے فوراً  حاصل ہو گا:
\begin{align*}
\vec{\tau}&=\frac{\dif}{\dif t}(10.0t+2.50)\khat\,\,\si{\kilo\gram\meter\squared\per\second}\\
&=10.0\khat\,\, \si{\kilo\gram\meter\squared\per\second\squared}=10.0\khat\,\,\si{\newton\meter}\quad\quad\text{\RL{(جواب)}}
\end{align*}
جو محور \عددی{z}  کے مثبت رخ ہے۔

ہمارا \عددی{\vec{\ell}} کا نتیجہ کہتا ہے زاوی معیار حرکت محور \عددی{z} کے مثبت رخ ہے۔ تعین گر سمتیہ کے گھماو  کی صورت میں \قول{ مثبت } نتیجے کا مطلب سمجھنے  کے لئے  اس سمتیہ کی قیمت مختلف اوقات پر معلوم کرتے ہیں۔
\begin{align*}
t&=0, &&\vec{r}_0=\phantom{-3.00\ihat+\,}5.00\jhat\,\,\si{\meter}\\
t&=\SI{1.00}{\second}, &&\vec{r}_1=-3.00\ihat+5.00\jhat\,\,\si{\meter}\\
t&=\SI{2.00}{\second}, &&\vec{r}_2=-10.0\ihat+5.00\jhat\,\,\si{\meter}\\
\end{align*}
یہ  نتائج شکل \حوالہء{11.14b} میں  پیش ہیں؛ ہم دیکھتے ہیں کہ ذرے کے ساتھ ساتھ چلنے کے لئے \عددی{\vec{r}}  خلاف گھڑی گھومتا ہے۔ یہی گھماو کا مثبت رخ ہے۔ یوں، اگرچہ ذرہ خود سیدھی لکیر پر حرکت کرتا ہے، مبدا کے لحاظ سے یہ اس کی حرکت خلاف گھڑی ہے، اور یوں اس کا زاوی معیار حرکت مثبت ہے۔

 ہم \عددی{\vec{\ell}} کے رخ کا مطلب ، صلیبی ضرب ( یہاں \عددی{\vec{r}\times\vec{v}} یا  آپ  چاہیں \عددی{m\vec{r}\times\vec{v}} ، جو ایک رخ دیتے ہیں)  کا دایاں ہاتھ قاعدہ استعمال کر کے  سمجھ  سکتے ہیں۔ذرے کی حرکت کے دوران  کسی بھی معیار اثر کے لئے، دائیں ہاتھ کی چار انگلیاں  صلیبی ضرب  کے اول سمتیہ \عددی{\vec{r}} کے رخ  رکھی جاتی ہیں (شکل \حوالہء{11.14c})۔ ہاتھ یوں سمت بند کیا جاتا ہے کہ ہتھیلی  کے گرد انگلیاں باآسانی گھما کر صلیبی ضرب کے دوسرے سمتیہ \عددی{\vec{v}} کے رخ کی جائیں (شکل \حوالہء{11.14d})۔ اس پورے عمل کے دوران انگوٹھے کو چار انگلیوں کے لحاظ سے عمود دار رکھا جاتا ہے۔انگوٹھا  صلیبی ضرب  کے نتیجے کا رخ دیگا۔ جیسا شکل \حوالہء{11.14e} میں  دکھایا گیا ہے، ماحصل سمتیہ محور \عددی{z} کے مثبت رخ  (جو شکل کے مستوی سے سیدھا باہر نکلتا ہے) اور گزشتہ نتیجے کے عین مطابق ہے۔ شکل \حوالہء{11.14e} میں  \عددی{\vec{\tau}} کا رخ بھی دیا گیا ہے، جو محور \عددی{z} کے مثبت رخ ہے؛چونکہ،  زاوی معیار حرکت اسی رخ ہے اور اس کی قدر بڑھ رہی ہے۔
\انتہا{نمونی سوال}
%---------------------------

%section 11.7 Angular Momentum of a Rigid Body p310
\حصہ{استوار جسم کا زاوی معیار حرکت}
\موٹا{مقاصد}\\
اس حصہ کو پڑھنے کے بعد آپ ذیل کے قابل ہوں گے۔
\begin{enumerate}[1.]
\item
ذروں پر مشتمل نظام کے لئے،  نیوٹن کا قانون دوم زاوی روپ میں استعمال کر کے  نظام پر صافی قوت  مروڑ اور   نظام کے زاوی معیار  حرکت میں پیدا  تبدیلی کی شرح  کا تعلق  جان پائیں گے۔
\item
مقررہ محور کے گرد گھومتے استوار جسم   کے زاوی معیار حرکت اور  اسی محور  کے گرد جسم کے  گھمیری جمود اور زاوی رفتار کا  تعلق استعمال کر پائیں گے۔
\item
اگر دو جسم ایک ہی محور گھماو کے گرد گھومتے ہوں، ان کے کل زاوی معیار حرکت   کا حساب  کر پائیں گے۔
\end{enumerate}

\موٹا{کلیدی تصورات}\\
\begin{itemize}
\item
ذروں پر مشتمل نظام، کا زاوی معیار حرکت \عددی{\vec{L}} انفرادی ذروں کے زاوی معیار حرکت کا مجموعہ ہو گا۔
\begin{align*}
\vec{L}=\vec{\ell}_1+\vec{\ell}_2+\vec{\ell}_3+\cdots+\vec{\ell}_n=\sum_{i=1}^{n} \vec{\ell}_i
\end{align*}
\item
اس زاوی معیار حرکت کی تبدیلی کی شرح نظام پر  صافی بیرونی  قوت مروڑ  کے برابر ہو گی (جو نظام کے اندرونی  ذروں اور   نظام  کے باہر ذروں کے باہم عمل سے پیدا قوت مروڑ کا سمتی مجموعہ ہو گا)۔
\begin{align*}
\vec{\tau}_{\text{\RL{صافی}}}=\frac{\dif \vec{L}}{\dif t}\quad\quad\text{\RL{(ذروں پر مشتمل نظام)}}
\end{align*}
\item
مقررہ محور پر گھومتے استوار جسم کے لئے، ،محور گھماو کے متوازی زاوی معیار حرکت کا جزو ذیل ہو گا۔
\begin{align*}
L=I\omega\quad\quad\text{\RL{(استوار جسم، مقررہ محور)}}
\end{align*}
\end{itemize}

%----------------------------
%The Angular Momentum of a System of Particles
\جزوحصہء{ذروں پر مشتمل نظام کا زاوی معیار حرکت}
مبدا کے لحاظ سے ذروں پر مشتمل نظام کے زاوی معیار حرکت پر غور کرتے ہیں۔ نظام کا کل زاوی معیار حرکت \عددی{\vec{L}}  انفرادی ذروں کے زاوی معیار حرکت \عددی{\vec{\ell}}   کا (سمتی) مجموعہ ہو گا۔
%eq 11.26
\begin{align}\label{مساوات_لڑھکاو_ذروں_نظام_الف}
\vec{L}=\vec{\ell}_1+\vec{\ell}_2+\vec{\ell}_3+\cdots+\vec{\ell}_n=\sum_{i=1}^{n} \vec{\ell}_i
\end{align}
انفرادی زاوی معیار حرکت  کو زیر نوشت  \عددی{i}   سے ظاہر کیا گیا ہے۔

 دیگر ذروں کے ساتھ یا نظام کے بیرون  کے ساتھ باہم عمل کی  بنا انفرادی ذرے کا زاوی معیار حرکت وقت کے ساتھ  تبدیل ہو سکتا ہے۔ ہم \عددی{\vec{L}} میں  پیدا تبدیلی  مساوات \حوالہ{مساوات_لڑھکاو_ذروں_نظام_الف} کا (ذیل)  وقتی تفرق  معلوم کر سکتے ہیں۔
 %eq 11.27
 \begin{align}\label{مساوات_لڑھکاو_ذروں_نظام_ب}
 \frac{\dif\vec{L}}{\dif t}=\sum_{i=1}^{n}\frac{\dif \vec{\ell}_i}{\dif t}
 \end{align}
 مساوات \حوالہ{مساوات_لڑھکاو_قانون_دوم_زاوی} سے ہم دیکھتے ہیں کہ \عددی{i} ویں ذرے پر صافی قوت مروڑ \عددی{\dif\vec{\ell}\!/\!\dif t} ہو گی۔مساوات \حوالہ{مساوات_لڑھکاو_ذروں_نظام_ب} ذیل لکھی جا سکتی ہے۔
 %eq 11.28
 \begin{align}\label{مساوات_لڑھکاو_ذروں_نظام_پ}
 \frac{\dif\vec{L}}{\dif t}=\sum_{i=1}^{n}\vec{\tau}_{\text{\RL{صافی}},i}
 \end{align}
 یعنی، نظام کے زاوی معیار حرکت \عددی{\vec{L}}  کی  تبدیلی کی شرح  انفرادی ذروں پر قوت مروڑ کے سمتی مجموعہ کے برابر ہو گا۔ ان قوت مروڑ میں (ذروں کے بیچ قوتوں کی بنا)   \ترچھا{اندرونی قوت مروڑ}  اور (ذروں پر نظام سے باہر اجسام  کی قوت کی بنا)  \ترچھا{بیرونی قوت مروڑ } شامل ہیں۔ تاہم، ذروں کے بیچ قوت (نیوٹن کے قانون سوم کی بنا)   جوڑیوں کے روپ میں  ہو گی لہٰذا ان کی مجموعی قوت مروڑ  صفر ہو گی۔ یوں،  نظام کے کل زاوی معیار حرکت \عددی{\vec{L}} کو صرف نظام پر عمل پیرا  بیرونی قوت مروڑ تبدیلی کرتی ہیں۔
 
 \موٹا{صافی بیرونی  قوت مروڑ۔}\quad
 نظام  میں تمام ذروں  پر بیرونی قوت مروڑ کا سمتی مجموعہ \عددی{\vec{\tau}_{\text{\RL{صافی}}}}  صافی بیرونی قوت مروڑ کو ظاہر کرتا ہے۔ یوں  مساوات \حوالہ{مساوات_لڑھکاو_ذروں_نظام_پ} ذیل لکھی جا سکتی ہے:
 %eq 11.29
 \begin{align}\label{مساوات_لڑھکاو_ذروں_نظام_ت}
 \vec{\tau}_{\text{\RL{صافی}}}=\frac{\dif\vec{L}}{\dif t}\quad\quad\text{\RL{(ذروں پر مشتمل نظام)}}
 \end{align}
 جو نیوٹن کے قانون دوم کا زاوی روپ ہے۔ اس کے تحت ذیل ہو گا۔
 
 %-----------------------------------
 \ابتدا{قاعدہء}
 ذروں پر مشتمل نظام  پر صافی بیرونی قوت مروڑ \عددی{\vec{\tau}_{\text{\RL{صافی}}}}    نظام کے کل زاوی معیار حرکت \عددی{\vec{L}} کی تبدیلی کی شرح کے برابر ہو گی۔
 \انتہا{قاعدہء}
 %------------------------------------
 
 مساوات \حوالہ{مساوات_لڑھکاو_ذروں_نظام_ت} اور  \عددی{\vec{F}_{\text{\RL{صافی}}}=\dif \vec{P}\!/\!\dif t} (مساوات \حوالہء{9.27})   مماثل ہیں  تاہم  اول الذکر  زیادہ احتیاط مانگتی  ہے:قوت مروڑ اور نظام کا زاوی معیار حرکت ایک  مبدا کے لحاظ سے   ناپنا لازمی ہے۔ اگر  اندرونی جمودی چھوکٹ  کے لحاظ سے نظام کا مرکز کمیت  مسرع نہ ہو، یہ مبدا کسی بھی نقطے پر ہو سکتا ہے؛ اگر مسرع ہو، تب\ترچھا{ لازم }ہے کہ   مرکز کمیت پر  ہو۔ مثال کے طور پر،  پہیے کو ذروں پر مشتمل نظام تصور کریں۔ اگر زمین کے لحاظ سے ساکن  محور پر پہیا   گھومتا ہو، تب مساوات \حوالہ{مساوات_لڑھکاو_ذروں_نظام_ت}  استعمال  کرتے وقت زمین کے لحاظ سے کوئی بھی ساکن نقطہ بطور مبدا تسلیم کیا جا سکتا ہے۔البتہ، اگر پہیا مسرع محور کے گرد گھومتا ہو (جیسے جب پہیا میلان پر لڑھکتا نیچے آتا ہے)،  تب صرف پہیے کا مرکز کمیت مبدا  تسلیم کیا جا سکتا ہے۔
 %-------------------------------
 
 %The Angular Momentum of a Rigid Body Rotating About a Fixed Axis p311
 
