%external references might be missing and figures are missing
%p295

\باب{لڑھکاو، قوت مروڑ، اور زاوی معیار حرکت}

\حصہ{مستقیم حرکت اور گھماو  مل کر لڑھکاو دیتے ہیں}
\موٹا{مقاصد}\\
اس حصے کو پڑھنے کے بعد آپ ذیل کے قابل ہوں گے۔
\begin{enumerate}[1.]
\item
جان پائیں گے کہ لڑھکاو خالص مستقیم حرکت اور خالص گھماو  کا مجموعہ ہے۔
\item
ہموار لڑھکاو  میں  مرکز کمیت  کی رفتار اور  جسم کی زاوی رفتار  کا تعلق استعمال کر پائیں گے۔
\end{enumerate}

\موٹا{کلیدی تصورات}\\
\begin{itemize}
\item
رداس \عددی{R} کے پہیا کے لئے جو  ہموار سطح پر لڑھک رہا ہو ذیل ہو گا:
\begin{align*}
v_{\text{\RL{مرکزکمیت}}}=\omega R
\end{align*}
جہاں \عددی{v_{\text{\RL{مرکزکمیت}}}} پہیے کے مرکز کمیت  کی خطی رفتار  اور \عددی{\omega} پہیے کے وسط پر پہیے کی زاوی رفتار ہے۔
\item
پہیے کو   نقطہ \عددی{P}  کے گرد، جو\قول{  فرش}  کے ساتھ مس ہے، لمحاتی   گھومتا تصور کیا جا سکتا ہے۔ مرکز کمیت کے گرد اور اس نقطہ کے گرد جسم کی زاوی رفتار برابر  ہے۔
\end{itemize}

\جزوحصہء{طبیعیات کیا ہے؟}
جیسا باب \حوالہ{باب_گھماو} میں ذکر کیا گیا، گھماو کا مطالعہ طبیعیات  میں شامل ہے۔ غالباً، اس مطالعے   کا اہم ترین     اطلاق پہیے اور پہیے نما اجسام کا لڑھکاو  ہے۔ یہ اطلاقی طبیعیات بہت عرصہ سے استعمال میں ہے۔ قدیم زمانے میں بھاری اجسام  لٹھا پر  لڑھکاتے ہوئے ایک جگہ سے دوسری جگہ منتقل کیے جاتے تھے۔آج کل ہم گاڑی میں سامان رکھ کر ایک جگہ سے دوسری جگہ لڑھکاتے ہیں۔

لڑھکاو  کی طبیعیات اور انجینئری  اتنی پرانی ہے کہ  اس میں نئے تصور ممکن  نظر نہیں  آتے۔ تاہم، \اصطلاح{ پہیے دار تختہ }\فرہنگ{پہیے دار تختہ}\حاشیہب{skateboards}\فرہنگ{skateboard}  زیادہ پرانا نہیں۔ ہمارا کام یہاں لڑھکاو کی حرکت  کو سادہ بنانا ہے۔

\جزوحصہء{مستقیم حرکت اور گھماو ت مل کر لڑھکاو دیتے ہیں}
سطح  پر  \ترچھا{ہمواری سے لڑھکتے } اجسام پر  یہاں غور کیا جائے گا؛ یعنی جسم بغیر اچھلے یا پھسلے سطح پر حرکت کرتا ہے۔  شکل \حوالہء{11.2} میں  ہموار لڑھکاو  کی پیچیدگی دکھائی گئی ہے: اگرچہ جسم کا مرکز کمیت سیدھی لکیر پر حرکت کرتا ہے، چکا پر نقطہ یقیناً ایسا نہیں کرتا۔بہرحال،  اس حرکت کو مرکز کمیت کی مستقیم حرکت اور  باقی جسم کا، اس مرکز پر ، گھماو تصور کیا جا سکتا ہے۔

اسے سمجھنے کے لئے، فرض کریں آپ سڑک کے کنارے کھڑے ہو کر،  گزرتے ہوئے سائیکل کے پہیے کا  مطالعہ کرتے ہیں (شکل \حوالہء{11.3})۔ جیسا شکل میں دکھایا گیا ہے، پہیے کا  مرکز  کمیت \عددی{O}  مستقل رفتار \عددی{v_{\text{\RL{مرکزکمیت}}}} سے آگے بڑھتا ہے۔ نقطہ \عددی{P} ، جہاں پہیا سڑک کو مس کرتا ہے،  بھی  \عددی{v_{\text{\RL{مرکزکمیت}}}}  رفتار سے آگے بڑھتا ہے، اور یوں \عددی{P} ہمیشہ \عددی{O} کے ٹھیک نیچے رہتا ہے۔

وقتی دورانیہ \عددی{t} کے دوران، \عددی{O} اور \عددی{P} دونوں فاصلہ \عددی{s} طے کرتے ہیں۔ سائیکل سوار  کے نقطہ نظر سے، پہیا زاویہ \عددی{\theta} طے کرتا ہے اور جو نقطہ \عددی{t} کے آغاز میں زمین پر تھا قوسی فاصلہ \عددی{s} طے کرتا ہے۔ مساوات \حوالہ{مساوات_گھماو_خطی_زاوی_تعلق_الف}  قوسی فاصلہ \عددی{s} اور زاویہ \عددی{\theta} کا تعلق دیتی ہے:
%eq 11.1
\begin{align}\label{مساوات_لڑھکاو_فاصلہ_زاویہ_الف}
s=\theta R
\end{align}
جہاں \عددی{R} پہیے کا رداس ہے۔ پہیے کے مرکز (یکساں پہیے کا مرکز کمیت) کی خطی رفتار \عددی{v_{\text{\RL{مرکزکمیت}}}} ہم \عددی{\dif s\!/\!\dif t} سے جان سکتے ہیں۔ پہیے کے مرکز پر پہیے کی زاوی رفتار \عددی{\dif \theta\!/\!\dif t} ہو گی۔ یوں \عددی{R} مستقل رکھتے ہوئے،  مساوات \حوالہ{مساوات_لڑھکاو_فاصلہ_زاویہ_الف} کا  وقت کے ساتھ تفرق ذیل دیگا۔
%eq 11.2
\begin{align}\label{مساوات_لڑھکاو_فاصلہ_زاویہ_ب}
v_{\text{\RL{مرکزکمیت}}}=\omega R\quad\quad\text{\RL{(ہموار لڑھکاو  حرکت)}}
\end{align}

\موٹا{دونوں کا ملاپ۔}
شکل \حوالہء{11.4} میں  دکھایا گیا ہے کہ پہیے کی لڑھکنی حرکت  خالص مستقیم حرکت اور خالص گھمیری حرکت کا مجموعہ ہے۔ شکل \حوالہء{11.4a} خالص گھمیری حرکت پیش کرتی ہے (جس میں مرکز پر محور گھماو ساکن تصور کیا جاتا ہے): پہیے کا ہر نقطہ ، مرکز پر ، زاوی رفتار \عددی{\omega} سے گھومتا ہے۔ (ایسی حرکت پر باب \حوالہ{باب_گھماو} میں غور کیا گیا۔) پہیے کے  باہری  کنارے (چکا)  پر ہر نقطے کی خطی رفتار \عددی{v_{\text{\RL{مرکزکمیت}}}}   مساوات \حوالہ{مساوات_لڑھکاو_فاصلہ_زاویہ_ب}  دیتی    ہے۔ شکل \حوالہء{11.4b} میں  خالص مستقیم  حرکت پیش ہے (جس میں تصور کیا جاتا ہے کہ پہیا گھوم نہیں رہا):  پہیے کا ہر نقطہ  \عددی{v_{\text{\RL{مرکزکمیت}}}} رفتار سے دائیں حرکت کرتا ہے۔

شکل \حوالہء{11.4a}  اور شکل \حوالہء{11.4b} مل کر ،   شکل \حوالہء{11.4c} میں پیش، پہیے کی  اصل لڑھکنی  حرکت دیتی ہیں۔ حرکات کے  ملاپ میں  پہیے کا   نچلا  نقطہ  (\عددی{P}) ساکن ہے جبکہ پہیے کا بالا  نقطہ (\عددی{T}) ، کسی بھی دوسرے نقطہ سے زیادہ تیز، \عددی{2v_{\text{\RL{مرکزکمیت}}}} رفتار سے حرکت کرتا ہے۔ شکل \حوالہء{11.5} میں ان نتائج  کا  اثباتی مظاہرہ کیا گیا ہے، جہاں سائیکل کے لڑھکنی پہیے  کا   \اصطلاح{وقتیہ افشا  }\فرہنگ{وقتیo!afxa}\حاشیہب{time exposure}\فرہنگ{time!exposure} پیش  ہے۔ آپ دیکھ کر  بتا سکتے ہیں کہ پہیے کا بالا حصہ زیادہ تیزی سے حرکت کرتا ہے، چونکہ اس حصے کی تیلیاں مدھم نظر آتی ہیں۔

سطح پر  دائری جسم کی ہموار لڑھکنی  حرکت  کو ، شکل  \حوالہء{11.4a} اور شکل \حوالہء{11.4b} کی طرح، خالص گھمیری حرکت اور خالص مستقیم حرکت میں علیحدہ  علیحدہ کیا جا سکتا ہے۔

\جزوجزوحصہء{لڑھکاو بطور خالص گھماو}
شکل \حوالہء{11.6} میں پہیے کا لڑھکاو   نئے انداز میں پیش کیا گیا ہے؛ جس نقطے پر پہیا سڑک مس کرتا ہے، اس نقطے  سے گزرتی محور پر پہیا گھومتا ہے؛ یہ محور \عددی{v_{\text{\RL{مرکزکمیت}}}}  رفتار سے حرکت میں ہو گی۔ہم  لڑھکاو کو  ،  شکل \حوالہء{11.4c} میں نقطہ \عددی{P} سے گزرتی  ، پہیے کو عمود دار، محور پر خالص گھماو  تصور کرتے ہیں۔ یوں شکل \حوالہء{11.6} میں سمتیات ، لڑھکنی پہیے پر نقطوں کی لمحاتی سمتی رفتار  دیتے ہیں۔

\موٹا{سوال:}\quad
ساکن  مشاہدہ کار  اس محور پر سائیکل کے  لڑھکنی  پہیے کو کیا زاوی رفتار مختص کرے گا؟

\موٹا{جواب:}\quad
وہی زاوی رفتار \عددی{\omega} جو سائیکل  سوار  مرکز کمیت کے گرد خالص گھماو  کا مشاہدہ کرتے ہوئے پہیے کو مختص کرتا ہے۔

اس جواب کی تصدیق کرنے کی خاطر،  ہم ساکن مشاہدہ کار کے نقطہ نظر سے  لڑھکنی پہیے کے فراز  کی خطی رفتار تلاش کرتے ہیں۔ پہیے کا رداس \عددی{R} لیتے ہوئے، پہیے کا فراز  شکل \حوالہء{11.6} میں \عددی{P}  پر واقع محور سے \عددی{2R} فاصلے پر ہو گا، لہٰذا فراز کی خطی رفتار  (مساوات \حوالہ{مساوات_لڑھکاو_فاصلہ_زاویہ_ب} استعمال کر کے) ذیل ہو گی:
\begin{align*}
v_{\text{\RL{فراز}}}=(\omega)(2R)=2(\omega R)=2v_{\text{\RL{مرکزکمیت}}}
\end{align*}
جو شکل \حوالہء{11.4c} کے عین مطابق ہے۔آپ شکل \حوالہء{11.4c} میں  پیش   ، نقطہ  \عددی{O} اور \عددی{P} کی ، خطی رفتار کی تصدیق  بھی اس  طرح کر سکتے ہیں۔

%--------------------------
%checkpoint 1,  p297
\ابتدا{آزمائش}
ایک سائیکل کے پچھلے  پہیے کا رداس اگلے پہیے کے رداس کا دگنا ہے۔ (ا)  کیا چلنے کے دوران بڑے پہیے کے فراز کی خطی رفتار چھوٹے پہیے کے فراز کی خطی رفتار سے زیادہ ہے، کم ہے، یا اس کے برابر ہے؟ (ب)  کیا پچھلے پہیے کی زاوی رفتار اگلے پہیے کی زاوی رفتار سے زیادہ ہے، کم ہے، یا دونوں برابر ہیں؟
\انتہا{آزمائش}
%------------------------------

% 11.2  forces and kinetic energy of rolling   p298
\حصہ{لڑھکاو کی قوتیں اور حرکی توانائی}
\موٹا{مقاصد}\\
اس حصہ کو پڑھنے کے بعد آپ  ذیل کے قابل ہوں گے۔
\begin{enumerate}[1.]
\item
مرکز کمیت  کی  مستقیم حرکی توانائی اور مرکز کمیت   کے گرد گھمیری حرکی توانائی کا مجموعہ  حاصل کر کے جسم کی حرکی توانائی معلوم کر پائیں گے۔
\item
ہمواری کے ساتھ   لڑھکنی جسم کی حرکی توانائی میں تبدیلی اور  جسم پر سرانجام کام  کا تعلق استعمال کر پائیں گے۔
\item
ہموار لڑھکاو (لہٰذا  بغیر  پھسلن) کے لئے،  میکانی توانائی کی بقا استعمال کر کے ابتدائی توانائی  کی قیمتوں اور اختتامی توانائی  کی قیمتوں کا تعلق جان پائیں گے۔
\end{enumerate}

\موٹا{کلیدی تصورات}\\
\begin{itemize}
\item
ہموار لڑھکنی پہیے کی حرکی توانائی ذیل ہے،
\begin{align*}
K=\frac{1}{2}I_{\text{\RL{مرکزکمیت}}}\omega^2+\frac{1}{2}Mv_{\text{\RL{مرکزکمیت}}}^2
\end{align*}
جہاں  مرکز کمیت پر جسم کا گھمیری جمود \عددی{I_{\text{\RL{مرکزکمیت}}}}  اور پہیے کی کمیت \عددی{M} ہے۔
\item
اگر پہیا مسرع کیا جائے، اور پہیا اب بھی ہمواری کے ساتھ لڑھکتا  ہے ، مرکز کمیت  کے  اسراع \عددی{\vec{a}_{\text{\RL{مرکزکمیت}}}}  اور مرکز پر زاوی اسراع \عددی{\alpha}  کا تعلق ذیل ہو گا۔
\begin{align*}
a_{\text{\RL{مرکزکمیت}}}=\alpha R
\end{align*}
\item
اگر \عددی{\theta} زاویہ کے میلان پر  پہیا ہمواری کے ساتھ نیچے لڑھکتا ہو، اس کا اسراع، میلان کے ہمراہ  اوپر رخ  محور  \عددی{x} پر،  ذیل ہو گا۔
\begin{align*}
a_{\text{\RL{مرکزکمیت}}}=-\frac{g\sin\theta}{1+I_{\text{\RL{مرکزکمیت}}}\!/\!{MR^2}}
\end{align*}
\end{itemize}

\جزوحصہء{لڑھکاو کی حرکی توانائی}
آئیں ساکن مشاہدہ کار  کے نقطہ نظر سے  لڑھکنی پہیے کی حرکی توانائی معلوم کریں۔ اگر ہم شکل \حوالہء{11.6} میں نقطہ \عددی{P} سے گزرتی محور  پر لڑھکاو کو خالص گھماو تصور کریں، تب مساوات \حوالہ{مساوات_گھماو_حرکی_گھمیری_تعریف} ذیل دیگی،
%eq 11.3
\begin{align}\label{مساوات_لڑھکاو_فاصلہ_زاویہ_پ}
K=\frac{1}{2}I_P\omega^2
\end{align}
جہاں  \عددی{P} پر واقع محور کے گرد پہیے کا گھمیری جمود \عددی{I_P} اور  پہیے کی زاوی رفتار \عددی{\omega} ہے۔ مساوات \حوالہ{مساوات_گھماو_مسئلہ_متوازی_محور}  کے مسئلہ متوازی محور (\عددی{I=I_{\text{\RL{مرکزکمیت}}}+Mh^2}) کے تحت ذیل ہو گا،
%eq 11.4
\begin{align}\label{مساوات_لڑھکاو_فاصلہ_زاویہ_ت}
I_P=I_{\text{\RL{مرکزکمیت}}}+MR^2
\end{align}
جہاں  \عددی{M} پہیے کی کمیت،   مرکز کمیت سے گزرتی محور  پر  گھمیری جمود  \عددی{I_{\text{\RL{مرکزکمیت}}}}، اور  \عددی{R} (پہیے کا رداس)  عمود دار فاصلہ \عددی{h} ہے۔ مساوات \حوالہ{مساوات_لڑھکاو_فاصلہ_زاویہ_ت} کو مساوات \حوالہ{مساوات_لڑھکاو_فاصلہ_زاویہ_پ}  میں ڈال کر :
\begin{align*}
K=\frac{1}{2}I_{\text{\RL{مرکزکمیت}}}\omega^2+\frac{1}{2}MR^2\omega^2
\end{align*}
اور مساوات \حوالہ{مساوات_لڑھکاو_فاصلہ_زاویہ_ب}  (\عددی{v_{\text{\RL{مرکزکمیت}}}=\omega R})  استعمال کرکے ذیل حاصل ہو گا۔
%eq 11.5
\begin{align}
K=\frac{1}{2}I_{\text{\RL{مرکزکمیت}}}\omega^2+\frac{1}{2}Mv_{\text{\RL{مرکزکمیت}}}^2
\end{align}

جزو \عددی{\tfrac{1}{2}I_{\text{\RL{مرکزکمیت}}}\omega^2}  کو  مرکز کمیت سے گزرتی محور پر پہیے کے لڑھکاو سے وابستہ حرکی توانائی تصور کیا جا سکتا ہے (شکل \حوالہء{11.4a})، اور جزو \عددی{\tfrac{1}{2}Mv_{\text{\RL{مرکزکمیت}}}^2} کو  پہیے کے مرکز کمیت کی مستقیم حرکت سے وابستہ حرکی توانائی تصور کیا جا سکتا ہے (شکل \حوالہء{11.4b})۔ یوں ذیل قاعدہ ابھرتا ہے۔

\ابتدا{قاعدہء}
لڑھکنی  جسم کی دو قسم کی حرکی توانائیاں ہوں گی: مرکز کمیت پر گھماو کی بدولت گھمیری حرکی توانائی \عددی{(\tfrac{1}{2}I_{\text{\RL{مرکزکمیت}}}\omega^2)} اور  مرکز کمیت کی مستقیم حرکت کی بدولت مستقیم حرکی توانائی \عددی{(\tfrac{1}{2}Mv_{\text{\RL{مرکزکمیت}}}^2)}۔
\انتہا{قاعدہء}

%----------------------------
%the forces of rolling p299
\جزوحصہء{لڑھکاو کی قوتیں}
\جزوجزوحصہء{رگڑ اور لڑھکاو}
اگر پہیا مستقل رفتار سے لڑھکتا ہو، جیسا شکل \حوالہء{11.3} میں دکھایا گیا ہے،  نقطہ  تماس \عددی{P} پر پہیا   ہرگز نہیں پھسلتا لہٰذا اس نقطے پر رگڑ نہیں ہو گی۔ تاہم، اگر صافی قوت پہیے کو تیز یا آہستہ  کرتی ہو، تب یہ صافی قوت  مرکز کمیت کو حرکت کے رخ  اسراع \عددی{\vec{a}_{\text{\RL{مرکزکمیت}}}} بخشے گی۔ ساتھ ہی  پہیا تیز یا آہستہ گھومے گا، لہٰذا زاوی اسراع \عددی{\alpha} بھی  ہو گا۔ ان اسراع کی بدولت پہیا \عددی{P} پر پھسل  سکتا ہے۔ یوں \عددی{P} پر رگڑی قوت عمل کرتی ہوئے  پہیے کو پھسلنے  سے روکتی ہے۔

اگر پہیا پھسلے نہیں، یہ قوت \ترچھا{سکونی } رگڑی قوت \عددی{\vec{f}_s} ہو گی اور حرکت ہموار لڑھکاو ہو گا۔ ایسی صورت میں،    (\عددی{R} مستقل رکھ کر)  وقت کے ساتھ مساوات \حوالہ{مساوات_لڑھکاو_فاصلہ_زاویہ_ب} کا تفرق  لے کر  خطی اسراع \عددی{\vec{a}_{\text{\RL{مرکزکمیت}}}} کی قدر اور زاوی اسراع کی قدر \عددی{\alpha} کا تعلق حاصل کر سکتے ہیں۔ بائیں ہاتھ \عددی{\dif v_{\text{\RL{مرکزکمیت}}}\!/\!\dif t} درحقیقت \عددی{a_{\text{\RL{مرکزکمیت}}}} اور دائیں ہاتھ  \عددی{\dif \omega\!/\!\dif t} درحقیقت \عددی{\alpha} ہے۔ یوں ہموار لڑھکاو کے لئے ذیل ہو گا۔
%eq 11.6
\begin{align}\label{مساوات_لڑھکاو_رگڑی_الف}
a_{\text{\RL{مرکزکمیت}}}=\alpha R\quad\quad\text{\RL{{(ہموار لڑھکنی حرکت)}}}
\end{align}

جب پہیے پر عمل پیرا  صافی قوت  کی بدولت   پہیا پھسلے ، تب   شکل \حوالہء{11.3} میں \عددی{P}   پر \ترچھا{حرکی } رگڑی قوت \عددی{\vec{f}_k}  عمل کرے گی؛ حرکت تب  ہموار  لڑھکاو نہیں ہو گی، اور مساوات \حوالہ{مساوات_لڑھکاو_رگڑی_الف} کا اطلاق نہیں ہو گا۔ اس باب میں صرف ہموار لڑھکنی حرکت پر بات کی جائے گی۔

شکل \حوالہء{11.7} میں،    افقی سطح پر دائیں   رخ لڑھکتے ہوئے   ،  سائیکل مقابلے کے آغاز کی طرح،  پہیا زیادہ تیز  گھمایا جاتا ہے۔ زیادہ تیز گھماو کی بدولت \عددی{P}  پر پہیا پھسل کر بائیں  جانا چاہتا ہے۔  نقطہ \عددی{P} پر  دائیں رخ رگڑی قوت  اس رجحان کا مقابلہ کرتی ہے۔ اگر پہیا پھسلے نہیں، یہ قوت سکونی رگڑی قوت \عددی{\vec{f}_s} ہو گی (جیسا دکھایا گیا ہے)، حرکت ہموار لڑھکاو ہو گی، اور مساوات \حوالہ{مساوات_لڑھکاو_رگڑی_الف} کا اطلاق ہو گا۔ (رگڑ کی غیر موجودگی میں سائیکل  مقابلہ ممکن نہیں ہو گا۔)

اگر شکل \حوالہء{11.7} میں پہیا آہستہ کیا جائے، ہمیں شکل دو طرح تبدیل کرنی ہو گی: مرکز کمیت کے اسراع \عددی{\vec{a}_{\text{\RL{مرکزکمیت}}}}   کا رخ   اور نقطہ \عددی{P}  پر رگڑی قوت \عددی{\vec{f}_s} کا رخ  اب بائیں  رخ  ہو گا۔

%rolling down a ramp p299
\جزوجزوحصہء{میلان  سے نیچے لڑھکاو}
