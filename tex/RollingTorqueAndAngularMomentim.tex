%external references might be missing and figures are missing
%p295

\باب{لڑھکاو، قوت مروڑ، اور زاوی معیار حرکت}

\حصہ{مستقیم حرکت اور گھماو  مل کر لڑھکاو دیتے ہیں}
\موٹا{مقاصد}\\
اس حصے کو پڑھنے کے بعد آپ ذیل کے قابل ہوں گے۔
\begin{enumerate}[1.]
\item
جان پائیں گے کہ لڑھکاو خالص مستقیم حرکت اور خالص گھماو  کا مجموعہ ہے۔
\item
ہموار لڑھکاو  میں  مرکز کمیت  کی رفتار اور  جسم کی زاوی رفتار  کا تعلق استعمال کر پائیں گے۔
\end{enumerate}

\موٹا{کلیدی تصورات}\\
\begin{itemize}
\item
رداس \عددی{R} کے پہیا کے لئے جو  ہموار سطح پر لڑھک رہا ہو ذیل ہو گا:
\begin{align*}
v_{\text{\RL{مرکزکمیت}}}=\omega R
\end{align*}
جہاں \عددی{v_{\text{\RL{مرکزکمیت}}}} پہیے کے مرکز کمیت  کی خطی رفتار  اور \عددی{\omega} پہیے کے وسط پر پہیے کی زاوی رفتار ہے۔
\item
پہیے کو   نقطہ \عددی{P}  کے گرد، جو\قول{  فرش}  کے ساتھ مس ہے، لمحاتی   گھومتا تصور کیا جا سکتا ہے۔ مرکز کمیت کے گرد اور اس نقطہ کے گرد جسم کی زاوی رفتار برابر  ہے۔
\end{itemize}

\جزوحصہء{طبیعیات کیا ہے؟}
جیسا باب \حوالہ{باب_گھماو} میں ذکر کیا گیا، گھماو کا مطالعہ طبیعیات  میں شامل ہے۔ غالباً، اس مطالعے   کا اہم ترین     اطلاق پہیے اور پہیے نما اجسام کا لڑھکاو  ہے۔ یہ اطلاقی طبیعیات بہت عرصہ سے استعمال میں ہے۔ قدیم زمانے میں بھاری اجسام  لٹھا پر  لڑھکاتے ہوئے ایک جگہ سے دوسری جگہ منتقل کیے جاتے تھے۔آج کل ہم گاڑی میں سامان رکھ کر ایک جگہ سے دوسری جگہ لڑھکاتے ہیں۔

لڑھکاو  کی طبیعیات اور انجینئری  اتنی پرانی ہے کہ  اس میں نئے تصور ممکن  نظر نہیں  آتے۔ تاہم، \اصطلاح{ پہیے دار تختہ }\فرہنگ{پہیے دار تختہ}\حاشیہب{skateboards}\فرہنگ{skateboard}  زیادہ پرانا نہیں۔ ہمارا کام یہاں لڑھکاو کی حرکت  کو سادہ بنانا ہے۔

\جزوحصہء{مستقیم حرکت اور گھماو ت مل کر لڑھکاو دیتے ہیں}
سطح  پر  \ترچھا{ہمواری سے لڑھکتے } اجسام پر  یہاں غور کیا جائے گا؛ یعنی جسم بغیر اچھلے یا پھسلے سطح پر حرکت کرتا ہے۔  شکل \حوالہء{11.2} میں  ہموار لڑھکاو  کی پیچیدگی دکھائی گئی ہے: اگرچہ جسم کا مرکز کمیت سیدھی لکیر پر حرکت کرتا ہے، چکا پر نقطہ یقیناً ایسا نہیں کرتا۔بہرحال،  اس حرکت کو مرکز کمیت کی مستقیم حرکت اور  باقی جسم کا، اس مرکز پر ، گھماو تصور کیا جا سکتا ہے۔

اسے سمجھنے کے لئے، فرض کریں آپ سڑک کے کنارے کھڑے ہو کر،  گزرتے ہوئے سائیکل کے پہیے کا  مطالعہ کرتے ہیں (شکل \حوالہء{11.3})۔ جیسا شکل میں دکھایا گیا ہے، پہیے کا  مرکز  کمیت \عددی{O}  مستقل رفتار \عددی{v_{\text{\RL{مرکزکمیت}}}} سے آگے بڑھتا ہے۔ نقطہ \عددی{P} ، جہاں پہیا سڑک کو مس کرتا ہے،  بھی  \عددی{v_{\text{\RL{مرکزکمیت}}}}  رفتار سے آگے بڑھتا ہے، اور یوں \عددی{P} ہمیشہ \عددی{O} کے ٹھیک نیچے رہتا ہے۔

وقتی دورانیہ \عددی{t} کے دوران، \عددی{O} اور \عددی{P} دونوں فاصلہ \عددی{s} طے کرتے ہیں۔ سائیکل سوار  کے نقطہ نظر سے، پہیا زاویہ \عددی{\theta} طے کرتا ہے اور جو نقطہ \عددی{t} کے آغاز میں زمین پر تھا قوسی فاصلہ \عددی{s} طے کرتا ہے۔ مساوات \حوالہ{مساوات_گھماو_خطی_زاوی_تعلق_الف}  قوسی فاصلہ \عددی{s} اور زاویہ \عددی{\theta} کا تعلق دیتی ہے:
%eq 11.1
\begin{align}\label{مساوات_لڑھکاو_فاصلہ_زاویہ_الف}
s=\theta R
\end{align}
جہاں \عددی{R} پہیے کا رداس ہے۔ پہیے کے مرکز (یکساں پہیے کا مرکز کمیت) کی خطی رفتار \عددی{v_{\text{\RL{مرکزکمیت}}}} ہم \عددی{\dif s\!/\!\dif t} سے جان سکتے ہیں۔ پہیے کے مرکز پر پہیے کی زاوی رفتار \عددی{\dif \theta\!/\!\dif t} ہو گی۔ یوں \عددی{R} مستقل رکھتے ہوئے،  مساوات \حوالہ{مساوات_لڑھکاو_فاصلہ_زاویہ_الف} کا  وقت کے ساتھ تفرق ذیل دیگا۔
%eq 11.2
\begin{align}\label{مساوات_لڑھکاو_فاصلہ_زاویہ_ب}
v_{\text{\RL{مرکزکمیت}}}=\omega R\quad\quad\text{\RL{(ہموار لڑھکاو  حرکت)}}
\end{align}

\موٹا{دونوں کا ملاپ۔}
شکل \حوالہء{11.4} میں  دکھایا گیا ہے کہ پہیے کی لڑھکنی حرکت  خالص مستقیم حرکت اور خالص گھمیری حرکت کا مجموعہ ہے۔ شکل \حوالہء{11.4a} خالص گھمیری حرکت پیش کرتی ہے (جس میں مرکز پر محور گھماو ساکن تصور کیا جاتا ہے): پہیے کا ہر نقطہ ، مرکز پر ، زاوی رفتار \عددی{\omega} سے گھومتا ہے۔ (ایسی حرکت پر باب \حوالہ{باب_گھماو} میں غور کیا گیا۔) پہیے کے  باہری  کنارے (چکا)  پر ہر نقطے کی خطی رفتار \عددی{v_{\text{\RL{مرکزکمیت}}}}   مساوات \حوالہ{مساوات_لڑھکاو_فاصلہ_زاویہ_ب}  دیتی    ہے۔ شکل \حوالہء{11.4b} میں  خالص مستقیم  حرکت پیش ہے (جس میں تصور کیا جاتا ہے کہ پہیا گھوم نہیں رہا):  پہیے کا ہر نقطہ  \عددی{v_{\text{\RL{مرکزکمیت}}}} رفتار سے دائیں حرکت کرتا ہے۔

شکل \حوالہء{11.4a}  اور شکل \حوالہء{11.4b} مل کر ،   شکل \حوالہء{11.4c} میں پیش، پہیے کی  اصل لڑھکنی  حرکت دیتی ہیں۔ حرکات کے  ملاپ میں  پہیے کا   نچلا  نقطہ  (\عددی{P}) ساکن ہے جبکہ پہیے کا بالا  نقطہ (\عددی{T}) ، کسی بھی دوسرے نقطہ سے زیادہ تیز، \عددی{2v_{\text{\RL{مرکزکمیت}}}} رفتار سے حرکت کرتا ہے۔ شکل \حوالہء{11.5} میں ان نتائج  کا  اثباتی مظاہرہ کیا گیا ہے، جہاں سائیکل کے لڑھکنی پہیے  کا   \اصطلاح{وقتیہ افشا  }\فرہنگ{وقتیo!afxa}\حاشیہب{time exposure}\فرہنگ{time!exposure} پیش  ہے۔ آپ دیکھ کر  بتا سکتے ہیں کہ پہیے کا بالا حصہ زیادہ تیزی سے حرکت کرتا ہے، چونکہ اس حصے کی تیلیاں مدھم نظر آتی ہیں۔

سطح پر  دائری جسم کی ہموار لڑھکنی  حرکت  کو ، شکل  \حوالہء{11.4a} اور شکل \حوالہء{11.4b} کی طرح، خالص گھمیری حرکت اور خالص مستقیم حرکت میں علیحدہ  علیحدہ کیا جا سکتا ہے۔

\جزوجزوحصہء{لڑھکاو بطور خالص گھماو}
شکل \حوالہء{11.6} میں پہیے کا لڑھکاو   نئے انداز میں پیش کیا گیا ہے؛ جس نقطے پر پہیا سڑک مس کرتا ہے، اس نقطے  سے گزرتی محور پر پہیا گھومتا ہے؛ یہ محور \عددی{v_{\text{\RL{مرکزکمیت}}}}  رفتار سے حرکت میں ہو گی۔ہم  لڑھکاو کو  ،  شکل \حوالہء{11.4c} میں نقطہ \عددی{P} سے گزرتی  ، پہیے کو عموددار، محور پر خالص گھماو  تصور کرتے ہیں۔ یوں شکل \حوالہء{11.6} میں سمتیات ، لڑھکنی پہیے پر نقطوں کی لمحاتی سمتی رفتار  دیتے ہیں۔

\موٹا{سوال:}\quad
ساکن  مشاہدہ کار  اس محور پر سائیکل کے  لڑھکنی  پہیے کو کیا زاوی رفتار مختص کرے گا؟

\موٹا{جواب:}\quad
وہی زاوی رفتار \عددی{\omega} جو سائیکل  سوار  مرکز کمیت کے گرد خالص گھماو  کا مشاہدہ کرتے ہوئے پہیے کو مختص کرتا ہے۔

اس جواب کی تصدیق کرنے کی خاطر،  ہم ساکن مشاہدہ کار کے نقطہ نظر سے  لڑھکنی پہیے کے فراز  کی خطی رفتار تلاش کرتے ہیں۔ پہیے کا رداس \عددی{R} لیتے ہوئے، پہیے کا فراز  شکل \حوالہء{11.6} میں \عددی{P}  پر واقع محور سے \عددی{2R} فاصلے پر ہو گا، لہٰذا فراز کی خطی رفتار  (مساوات \حوالہ{مساوات_لڑھکاو_فاصلہ_زاویہ_ب} استعمال کر کے) ذیل ہو گی:
\begin{align*}
v_{\text{\RL{فراز}}}=(\omega)(2R)=2(\omega R)=2v_{\text{\RL{مرکزکمیت}}}
\end{align*}
جو شکل \حوالہء{11.4c} کے عین مطابق ہے۔آپ شکل \حوالہء{11.4c} میں  پیش   ، نقطہ  \عددی{O} اور \عددی{P} کی ، خطی رفتار کی تصدیق  بھی اس  طرح کر سکتے ہیں۔

%--------------------------
%checkpoint 1,  p297
\ابتدا{آزمائش}
ایک سائیکل کے پچھلے  پہیے کا رداس اگلے پہیے کے رداس کا دگنا ہے۔ (ا)  کیا چلنے کے دوران بڑے پہیے کے فراز کی خطی رفتار چھوٹے پہیے کے فراز کی خطی رفتار سے زیادہ ہے، کم ہے، یا اس کے برابر ہے؟ (ب)  کیا پچھلے پہیے کی زاوی رفتار اگلے پہیے کی زاوی رفتار سے زیادہ ہے، کم ہے، یا دونوں برابر ہیں؟
\انتہا{آزمائش}
%------------------------------

% 11.2  forces and kinetic energy of rolling   p298
\حصہ{لڑھکاو کی قوتیں اور حرکی توانائی}
\موٹا{مقاصد}\\
اس حصہ کو پڑھنے کے بعد آپ  ذیل کے قابل ہوں گے۔
\begin{enumerate}[1.]
\item
مرکز کمیت  کی  مستقیم حرکی توانائی اور مرکز کمیت   کے گرد گھمیری حرکی توانائی کا مجموعہ  حاصل کر کے جسم کی حرکی توانائی معلوم کر پائیں گے۔
\item
ہمواری کے ساتھ   لڑھکنی جسم کی حرکی توانائی میں تبدیلی اور  جسم پر سرانجام کام  کا تعلق استعمال کر پائیں گے۔
\item
ہموار لڑھکاو (لہٰذا  بغیر  پھسلن) کے لئے،  میکانی توانائی کی بقا استعمال کر کے ابتدائی توانائی  کی قیمتوں اور اختتامی توانائی  کی قیمتوں کا تعلق جان پائیں گے۔
\end{enumerate}

\موٹا{کلیدی تصورات}\\
\begin{itemize}
\item
ہموار لڑھکنی پہیے کی حرکی توانائی ذیل ہے،
\begin{align*}
K=\frac{1}{2}I_{\text{\RL{مرکزکمیت}}}\omega^2+\frac{1}{2}Mv_{\text{\RL{مرکزکمیت}}}^2
\end{align*}
جہاں  مرکز کمیت پر جسم کا گھمیری جمود \عددی{I_{\text{\RL{مرکزکمیت}}}}  اور پہیے کی کمیت \عددی{M} ہے۔
\item
اگر پہیا مسرع کیا جائے، اور پہیا اب بھی ہمواری کے ساتھ لڑھکتا  ہے ، مرکز کمیت  کے  اسراع \عددی{\vec{a}_{\text{\RL{مرکزکمیت}}}}  اور مرکز پر زاوی اسراع \عددی{\alpha}  کا تعلق ذیل ہو گا۔
\begin{align*}
a_{\text{\RL{مرکزکمیت}}}=\alpha R
\end{align*}
\item
اگر \عددی{\theta} زاویہ کے میلان پر  پہیا ہمواری کے ساتھ  اترتے ہوئے  لڑھکتا ہو، اس کا اسراع، میلان کے ہمراہ  اوپر رخ  محور  \عددی{x} پر،  ذیل ہو گا۔
\begin{align*}
a_{\text{\RL{مرکزکمیت}}}=-\frac{g\sin\theta}{1+I_{\text{\RL{مرکزکمیت}}}\!/\!{MR^2}}
\end{align*}
\end{itemize}

\جزوحصہء{لڑھکاو کی حرکی توانائی}
آئیں ساکن مشاہدہ کار  کے نقطہ نظر سے  لڑھکنی پہیے کی حرکی توانائی معلوم کریں۔ اگر ہم شکل \حوالہء{11.6} میں نقطہ \عددی{P} سے گزرتی محور  پر لڑھکاو کو خالص گھماو تصور کریں، تب مساوات \حوالہ{مساوات_گھماو_حرکی_گھمیری_تعریف} ذیل دیگی،
%eq 11.3
\begin{align}\label{مساوات_لڑھکاو_فاصلہ_زاویہ_پ}
K=\frac{1}{2}I_P\omega^2
\end{align}
جہاں  \عددی{P} پر واقع محور کے گرد پہیے کا گھمیری جمود \عددی{I_P} اور  پہیے کی زاوی رفتار \عددی{\omega} ہے۔ مساوات \حوالہ{مساوات_گھماو_مسئلہ_متوازی_محور}  کے مسئلہ متوازی محور (\عددی{I=I_{\text{\RL{مرکزکمیت}}}+Mh^2}) کے تحت ذیل ہو گا،
%eq 11.4
\begin{align}\label{مساوات_لڑھکاو_فاصلہ_زاویہ_ت}
I_P=I_{\text{\RL{مرکزکمیت}}}+MR^2
\end{align}
جہاں  \عددی{M} پہیے کی کمیت،   مرکز کمیت سے گزرتی محور  پر  گھمیری جمود  \عددی{I_{\text{\RL{مرکزکمیت}}}}، اور  \عددی{R} (پہیے کا رداس)  عموددار فاصلہ \عددی{h} ہے۔ مساوات \حوالہ{مساوات_لڑھکاو_فاصلہ_زاویہ_ت} کو مساوات \حوالہ{مساوات_لڑھکاو_فاصلہ_زاویہ_پ}  میں ڈال کر :
\begin{align*}
K=\frac{1}{2}I_{\text{\RL{مرکزکمیت}}}\omega^2+\frac{1}{2}MR^2\omega^2
\end{align*}
اور مساوات \حوالہ{مساوات_لڑھکاو_فاصلہ_زاویہ_ب}  (\عددی{v_{\text{\RL{مرکزکمیت}}}=\omega R})  استعمال کرکے ذیل حاصل ہو گا۔
%eq 11.5
\begin{align}\label{مساوات_لڑھکاو_مستقیم_گھمیری_الف}
K=\frac{1}{2}I_{\text{\RL{مرکزکمیت}}}\omega^2+\frac{1}{2}Mv_{\text{\RL{مرکزکمیت}}}^2
\end{align}

جزو \عددی{\tfrac{1}{2}I_{\text{\RL{مرکزکمیت}}}\omega^2}  کو  مرکز کمیت سے گزرتی محور پر پہیے کے لڑھکاو سے وابستہ حرکی توانائی تصور کیا جا سکتا ہے (شکل \حوالہء{11.4a})، اور جزو \عددی{\tfrac{1}{2}Mv_{\text{\RL{مرکزکمیت}}}^2} کو  پہیے کے مرکز کمیت کی مستقیم حرکت سے وابستہ حرکی توانائی تصور کیا جا سکتا ہے (شکل \حوالہء{11.4b})۔ یوں ذیل قاعدہ ابھرتا ہے۔

\ابتدا{قاعدہء}
لڑھکنی  جسم کی دو قسم کی حرکی توانائیاں ہوں گی: مرکز کمیت پر گھماو کی بدولت گھمیری حرکی توانائی \عددی{(\tfrac{1}{2}I_{\text{\RL{مرکزکمیت}}}\omega^2)} اور  مرکز کمیت کی مستقیم حرکت کی بدولت مستقیم حرکی توانائی \عددی{(\tfrac{1}{2}Mv_{\text{\RL{مرکزکمیت}}}^2)}۔
\انتہا{قاعدہء}

%----------------------------
%the forces of rolling p299
\جزوحصہء{لڑھکاو کی قوتیں}
\جزوجزوحصہء{رگڑ اور لڑھکاو}
اگر پہیا مستقل رفتار سے لڑھکتا ہو، جیسا شکل \حوالہء{11.3} میں دکھایا گیا ہے،  نقطہ  تماس \عددی{P} پر پہیا   ہرگز نہیں پھسلتا لہٰذا اس نقطے پر رگڑ نہیں ہو گی۔ تاہم، اگر صافی قوت پہیے کو تیز یا آہستہ  کرتی ہو، تب یہ صافی قوت  مرکز کمیت کو حرکت کے رخ  اسراع \عددی{\vec{a}_{\text{\RL{مرکزکمیت}}}} بخشے گی۔ ساتھ ہی  پہیا تیز یا آہستہ گھومے گا، لہٰذا زاوی اسراع \عددی{\alpha} بھی  ہو گا۔ ان اسراع کی بدولت پہیا \عددی{P} پر پھسل  سکتا ہے۔ یوں \عددی{P} پر رگڑی قوت عمل کرتی ہوئے  پہیے کو پھسلنے  سے روکتی ہے۔

اگر پہیا پھسلے نہیں، یہ قوت \ترچھا{سکونی } رگڑی قوت \عددی{\vec{f}_s} ہو گی اور حرکت ہموار لڑھکاو ہو گا۔ ایسی صورت میں،    (\عددی{R} مستقل رکھ کر)  وقت کے ساتھ مساوات \حوالہ{مساوات_لڑھکاو_فاصلہ_زاویہ_ب} کا تفرق  لے کر  خطی اسراع \عددی{\vec{a}_{\text{\RL{مرکزکمیت}}}} کی قدر اور زاوی اسراع کی قدر \عددی{\alpha} کا تعلق حاصل کر سکتے ہیں۔ بائیں ہاتھ \عددی{\dif v_{\text{\RL{مرکزکمیت}}}\!/\!\dif t} درحقیقت \عددی{a_{\text{\RL{مرکزکمیت}}}} اور دائیں ہاتھ  \عددی{\dif \omega\!/\!\dif t} درحقیقت \عددی{\alpha} ہے۔ یوں ہموار لڑھکاو کے لئے ذیل ہو گا۔
%eq 11.6
\begin{align}\label{مساوات_لڑھکاو_رگڑی_الف}
a_{\text{\RL{مرکزکمیت}}}=\alpha R\quad\quad\text{\RL{{(ہموار لڑھکنی حرکت)}}}
\end{align}

جب پہیے پر عمل پیرا  صافی قوت  کی بدولت   پہیا پھسلے ، تب   شکل \حوالہء{11.3} میں \عددی{P}   پر \ترچھا{حرکی } رگڑی قوت \عددی{\vec{f}_k}  عمل کرے گی؛ حرکت تب  ہموار  لڑھکاو نہیں ہو گی، اور مساوات \حوالہ{مساوات_لڑھکاو_رگڑی_الف} کا اطلاق نہیں ہو گا۔ اس باب میں صرف ہموار لڑھکنی حرکت پر بات کی جائے گی۔

شکل \حوالہء{11.7} میں،    افقی سطح پر دائیں   رخ لڑھکتے ہوئے   ،  سائیکل مقابلے کے آغاز کی طرح،  پہیا زیادہ تیز  گھمایا جاتا ہے۔ زیادہ تیز گھماو کی بدولت \عددی{P}  پر پہیا پھسل کر بائیں  جانا چاہتا ہے۔  نقطہ \عددی{P} پر  دائیں رخ رگڑی قوت  اس رجحان کا مقابلہ کرتی ہے۔ اگر پہیا پھسلے نہیں، یہ قوت سکونی رگڑی قوت \عددی{\vec{f}_s} ہو گی (جیسا دکھایا گیا ہے)، حرکت ہموار لڑھکاو ہو گی، اور مساوات \حوالہ{مساوات_لڑھکاو_رگڑی_الف} کا اطلاق ہو گا۔ (رگڑ کی غیر موجودگی میں سائیکل  مقابلہ ممکن نہیں ہو گا۔)

اگر شکل \حوالہء{11.7} میں پہیا آہستہ کیا جائے، ہمیں شکل دو طرح تبدیل کرنی ہو گی: مرکز کمیت کے اسراع \عددی{\vec{a}_{\text{\RL{مرکزکمیت}}}}   کا رخ   اور نقطہ \عددی{P}  پر رگڑی قوت \عددی{\vec{f}_s} کا رخ  اب بائیں  رخ  ہو گا۔

%rolling down a ramp p299
\جزوجزوحصہء{میلان  سے نیچے لڑھکاو}
شکل \حوالہء{11.8} میں گول یکساں جسم ، جس کی کمیت \عددی{M} اور رداس \عددی{R} ہے، زاویہ \عددی{\theta}  کے میلان پر  ہمواری سے ،  محور \عددی{x} کے ہمراہ، نیچے لڑھک رہا ہے۔ ہم میلان کے ہمراہ     اترائی کے  رخ جسم  کے  اسراع \عددی{a_{\text{\RL{مرکزکمیت}},x}} کا ریاضی فقرہ تلاش کرنا  چاہتے ہیں۔  نیوٹن کے  قانون دوم    کی  خطی صورت \عددی{(F_{\text{\RL{صافی}}}=Ma)} اور زاوی صورت \عددی{(\tau_{\text{\RL{صافی}}}=I\alpha)}  صورت دونوں  استعمال کر کے ایسا کرتے ہیں۔

جسم پر قوتوں کا خاکہ بنانے سے آغاز کرتے ہیں (شکل \حوالہء{11.8})۔
\begin{enumerate}[1.]
\item
جسم پر تجاذبی قوت \عددی{\vec{F}_g} نشیب وار ہے۔ اس سمتیہ کی دم جسم کے مرکز کمیت پر رکھی جاتی ہے۔ میلان کے ہمراہ جزو \عددی{F_g\sin\theta} ہے جو \عددی{Mg\sin\theta} کے برابر ہو گا۔
\item
میلان کو عموددار  جزو \عددی{\vec{F}_N} ہے۔ یہ جزو نقطہ تماس \عددی{P} پر عمل کرتا ہے، تاہم  شکل \حوالہء{11.8} میں ،  \عددی{\vec{F}_N} کا رخ تبدیل کیے بغیر،  اس کو یوں کھسکایا  کیا گیا ہے کہ اس کی دم جسم کے مرکز کمیت پر ہو۔
\item
نقطہ تماس \عددی{P} پر  عمل پیرا سکونی  رگڑی  قوت  \عددی{\vec{f}_s}  میلان کے ہمراہ چڑھائی کے رخ ہے۔ (کیا آپ بتا سکتے ہیں، کیوں؟ اگر  \عددی{P} پر جسم  پھسلے، وہ\ترچھا{  اترائی }کے رخ پھسلے گا۔ یوں مخالف  رگڑی قوت چڑھائی کے رخ ہو گی۔ )
\end{enumerate}

ہم شکل \حوالہء{11.8} میں  محور \عددی{x} کے ہمراہ  اجزاء کے لئے نیوٹن کا قانون دوم \عددی{(F_{\text{\RL{صافی}},x}=ma_x)}  لکھتے ہیں۔
%eq 11.7
\begin{align}\label{مساوات_لڑھکاو_ہمراہ}
f_s-Mg\sin\theta=Ma_{\text{\RL{صافی}},x}
\end{align}
اس مساوات میں دو نامعلوم متغیرات ، \عددی{f_s} اور \عددی{a_{\text{\RL{صافی}},x}}، پائے جاتے ہیں۔ (ہم \عددی{f_s} کی قیمت،  رگڑی قوت کی زیادہ سے زیادہ قیمت،  \عددی{f_{s,\text{\RL{بلندتر}}}} فرض نہیں کر سکتے۔ہم صرف اتنا جانتے ہیں کہ رگڑی قوت  اتنی ہے کہ جسم پھسلتا نہیں اور میلان پر ہمواری سے لڑھکتا اترتا ہے۔)

ہم اب جسم کے مرکز کمیت  پر  جسم کے گھماو  پر نیوٹن کے قانون دوم کا اطلاق کرتے ہیں۔ پہلے، مساوات \حوالہ{مساوات_گھماو_صافی_قوت_مروڑ_الف}
  \عددی{(\tau=r_{\perp}F)} استعمال کر کے  مرکز کمیت  کے لحاظ سے جسم پر قوت مروڑ لکھتے ہیں۔ رگڑی قوت \عددی{\vec{f}_s}  کے  معیار اثر کا بازو  \عددی{R} ہے، لہٰذا اس کی قوت مروڑ \عددی{Rf_s} ہو گی، جو  اس لئے مثبت ہے کہ شکل \حوالہء{11.8} میں یہ جسم کو خلاف گھڑی گھمانے کی کوشش کرتی ہے۔مرکز کمیت کے لحاظ سے  قوت \عددی{\vec{F}_g} اور \عددی{\vec{F}_N} کے معیار اثر بازو صفر ہیں، لہٰذا ان کی قوت مروڑ صفر ہوں گی۔ جسم کے مرکز کمیت سے گزرتی محور پر نیوٹن کا قانون دوم  زاوی روپ \عددی{(\tau_{\text{\RL{صافی}}}=I\alpha)} میں لکھتے ہیں۔
  %eq 11.8
  \begin{align}\label{مساوات_لڑھکاو_زاوی_روپ_الف}
  Rf_s=I_{\text{\RL{مرکزکمیت}}}\alpha
  \end{align}
  اس مساوات میں دو نامعلوم متغیرات، \عددی{f_s} اور \عددی{\alpha} ، پائے جاتے ہیں۔
  
  جسم ہموار لڑھکتا ہے لہٰذا مساوات \حوالہ{مساوات_لڑھکاو_رگڑی_الف} \عددی{(a_{\text{\RL{مرکزکمیت}}}=\alpha R)}  استعمال  کر کے نامعلوم 
  \عددی{a_{\text{\RL{مرکزکمیت}},x}} اور \عددی{\alpha} کا تعلق لکھا جا سکتا ہے۔ تاہم، ہمیں ہوشیاری سے کام لینا ہو گا، چونکہ یہاں     \عددی{a_{\text{\RL{مرکزکمیت}},x}} منفی  (محور \عددی{x} پر منفی رخ ہے)  اور \عددی{\alpha} مثبت  (خلاف گھڑی) ہے۔ یوں مساوات \حوالہ{مساوات_لڑھکاو_زاوی_روپ_الف} میں \عددی{\alpha} کی جگہ \عددی{-a_{\text{\RL{مرکزکمیت}},x}\!/\!R} ڈال کر  \عددی{f_s} کے لئے حل کر کے ذیل حاصل کرتے ہیں۔
  %eq 11.9
  \begin{align}\label{مساوات_لڑھکاو_زاوی_روپ_ب}
  f_s=-I_{\text{\RL{مرکزکمیت}}}\frac{a_{\text{\RL{مرکزکمیت}},x}}{R^2}
  \end{align}
  مساوات \حوالہ{مساوات_لڑھکاو_ہمراہ} میں \عددی{f_s} کی جگہ مساوات \حوالہ{مساوات_لڑھکاو_زاوی_روپ_ب} کا دایاں ہاتھ ڈال کر ذیل ملتا ہے۔
  %eq 11.10
  \begin{align}\label{مساوات_لڑھکاو_اسراع}
  a_{\text{\RL{مرکزکمیت}},x}=-\frac{g\sin\theta}{1+I_{\text{\RL{مرکزکمیت}}}\!/\!MR^2}
  \end{align}
  اس مساوات کو استعمال کر  کے ، افق کے ساتھ زاویہ \عددی{\theta}  کے میلان پر  کے ہمراہ لڑھکتے جسم  کا خطی اسراع \عددی{ a_{\text{\RL{مرکزکمیت}},x}} حاصل ہو گا۔
  
 یاد رہے، تجاذبی قوت جسم کو میلان پر اترنے پر مجبور کرتی ہے، تاہم جسم کو گھومنے اور یوں لڑھکنے پر رگڑی قوت مجبور کرتی ہے۔ اگر آپ رگڑ  خارج کر  دیں (مثلاً،  میلان کو  تیل سے چکنا کر کے ) یا \عددی{Mg\sin\theta} کو  \عددی{f_{s,\text{\RL{بلندتر}}}}  سے زیادہ کر دیں، ہموار لڑھکاو خارج ہو جائے گا اور جسم لڑھکنے کی بجائے میلان پر پھسل کر اترے گا۔
 
 %------------------------
 %checkpoint 2 p300
 \ابتدا{آزمائش}
 قرص \عددی{A} اور \عددی{B} ایک جیسے ہیں اور فرش پر ایک جتنی رفتار سے لڑھکتے ہیں۔ قرص \عددی{A} کے سامنے میلان آتا ہے جس پر یہ زیادہ سے زیادہ   \عددی{h} تک پہنچتا ہے۔ قرص \عددی{B} متماثل ، لیکن بلا رگڑ  ، میلان پر چڑھتا ہے ۔ کیا \عددی{h} سے زیادہ، کم، یا اس کے برابر بلندی تک \عددی{B} پہنچے گا؟
 \انتہا{آزمائش}
 
 %--------------------------------
 %sample problem 11.01 Ball rolling down a ramp  p310
 \ابتدا{نمونی سوال}
 یکساں گیند، جس کی کمیت \عددی{M=\SI{6.00}{\kilo\gram}} اور رداس \عددی{R} ہے،  زاویہ \عددی{\theta=\SI{30.0}{\degree}} میلان  سے،  ساکن حالت سے آغاز کر کے،  ہموار لڑھکتا اترتا ہے (شکل \حوالہء{11.8})۔
 
 (ا)  انتصابی \عددی{h=\SI{1.20}{\meter}} نیچے زمین کو  پہنچتا   کر  گیند کی رفتار  کیا ہو گی؟
 
 \جزوحصہء{کلیدی تصورات}
چونکہ صرف تجاذبی قوت، جو   بقائی قوت ہے،  گیند پر کام سرانجام  دیتی ہے، لہٰذا میلان پر لڑھک کر اترنے کے دوران گیند و زمین نظام کی میکانی توانائی \عددی{E}  کی بقا ہو گی۔میلان سے گیند پر عمود دار قوت  گیند کی راہ کو عمودی  ہونے کی وجہ سے کوئی کام سرانجام نہیں دیتی۔ گیند پھسلتا نہیں (\ترچھا{ہموار  لڑھکتا }ہے)  لہٰذا رگڑی قوت کوئی  توانائی حری توانائی میں تبدیلی نہیں کرتی۔

یوں میکانی توانائی کی بقا ہو گی \عددی{E_f=E_i}:
%eq 11.11
\begin{align}\label{مساوات_لڑھکاو_نمونی_گیند_الف}
K_f+U_f=K_i+U_i
\end{align}
جہاں زیر نوشت \عددی{f} اور \عددی{i} بالترتیب (زمین پر پہنچ کر) اختتامی اور  (ساکن حالت ) ابتدائی قیمتیں ظاہر کرتی ہیں۔تجاذبی مخفی توانائی کی  ابتدائی  قیمت \عددی{U_i=Mgh} (جہاں \عددی{M}  گیند کی کمیت ہے) اور  اختتامی قیمت \عددی{U_f=0} ہے۔  ابتدائی حرکی توانائی  \عددی{K_i=0} ہے اختتامی حرکی توانائی جاننے کے لئے  اضافی  تصور  درکار ہے:  چونکہ گیند لڑھکتا ہے اس کی  حرکی توانائی میں مستقیم اور گھمیری جزو شامل ہوں گے، جنہیں شامل کرنے کے لئے مساوات \حوالہ{مساوات_لڑھکاو_مستقیم_گھمیری_الف} کا دایاں ہاتھ استعمال کرتے ہیں۔

\موٹا{حساب:}\quad
مساوات \حوالہ{مساوات_لڑھکاو_نمونی_گیند_الف} میں  ڈالنے سے ذیل حاصل ہو گا:
%eq 11.12
\begin{align}\label{مساوات_لڑھکاو_نمونی_گیند_ب}
(\frac{1}{2}I_{\text{\RL{مرکزکمیت}}}\omega^2+\frac{1}{2}Mv_{\text{\RL{مرکزکمیت}}}^2)+0=0+Mgh
\end{align}
جہاں گیند کے مرکز کمیت سے گزرتی محور پر گیند کا گھمیری جمود \عددی{I_{\text{\RL{مرکزکمیت}}}} ، زمین پر پہنچ کر گیند کی رفتار  (جو ہم تلاش کرنا چاہتے ہیں)  \عددی{v_{\text{\RL{مرکزکمیت}}}} ، اور  زمین پر پہنچ کر زاوی رفتار \عددی{\omega} ہے۔

چونکہ گیند ہموار لڑھکتا ہے، ہم مساوات \حوالہ{مساوات_لڑھکاو_فاصلہ_زاویہ_ب} استعمال کر کے \عددی{\omega} کی جگہ \عددی{v_{\text{\RL{مرکزکمیت}}}\!/\!R} پُر کر کے مساوات \حوالہ{مساوات_لڑھکاو_نمونی_گیند_ب} میں نامعلوم متغیرات کی تعداد کم کر سکتے ہیں۔ ایسا  کر کے، اور جدول  \حوالہء{10.2f}  سے \عددی{I_{\text{\RL{مرکزکمیت}}}}  کی جگہ \عددی{\tfrac{2}{5}MR^2} ڈال کر  \عددی{v_{\text{\RL{مرکزکمیت}}}}  کے لئے حل کرنے سے ذیل حاصل ہو گا۔
\begin{align*}
v_{\text{\RL{}}}&=\sqrt{(\tfrac{10}{7})gh}=\sqrt{(\tfrac{10}{7})(\SI{9.8}{\meter\per\second\squared})(\SI{1.20}{\meter})}\\
&=\SI{4.10}{\meter\per\second}\quad\quad\text{\RL{(جواب)}}
\end{align*}
یاد رہے، جواب \عددی{M} اور \عددی{R} پر منحصر نہیں۔

(ب)  میلان پر لڑھک کر اترنے کے دوران گینف پر رگڑی قوت  کی قدر اور رخ کیا ہیں؟

\جزوحصہء{کلیدی تصور}
چونکہ گیند ہموار لڑھکتا ہے، مساوات \حوالہ{مساوات_لڑھکاو_زاوی_روپ_ب}  گیند پر رگڑی قوت دیگی۔

\موٹا{حساب:}\quad
مساوات \حوالہ{مساوات_لڑھکاو_زاوی_روپ_ب} استعمال کرنے سے قبل ہمیں  مساوات \حوالہ{مساوات_لڑھکاو_اسراع} سے گیند کا اسراع  معلوم کرنا ہو گا۔
\begin{align*}
a_{\text{\RL{مرکزکمیت}},x}&=-I_{\text{\RL{مرکزکمیت}}}\frac{a_{\text{\RL{مرکزکمیت}},x}}{R^2}
=-\frac{2}{5}MR^2\frac{a_{\text{\RL{مرکزکمیت}}},x}{R^2}=-\frac{2}{5}Ma_{\text{\RL{مرکزکمیت}},x}\\
&=-\frac{2}{5}(\SI{6.00}{\kilo\gram})(\SI{-3.50}{\meter\per\second\squared})=\SI{8.40}{\newton}\quad\quad\text{\RL{(جواب)}}
\end{align*}
یاد رہے ہمیں کمیت \عددی{M} درکار تھی  جبکہ رداس \عددی{R}   نہیں تھا۔ یوں، \عددی{\SI{30}{\degree}} میلان پر \عددی{\SI{6.00}{\kilo\gram}}  ہموار لڑھکتے گیند پر، گیند کے رداس سے قطع نظر،  رگڑی قوت \عددی{\SI{8.40}{\newton}} ہو گی، تاہم بڑی کمیت کی صورت میں رگڑی قوت زیادہ ہو گی۔
 \انتہا{نمونی سوال}
 %------------------------
 
 %section 11.3 YO-YO p301
 \حصہ{ڈوری دار لٹو}
 \موٹا{مقاصد}
 اس حصے کو پڑھنے کے بعد آپ  ذیل کے قابل ہوں گے۔
 \begin{enumerate}[1.]
 \item
 ڈوری پر اوپر نیچے حرکت کرتے\اصطلاح{ ڈوری دار لٹو }\فرہنگ{لٹو!ڈوری دار}\حاشیہب{Yo-Yo}\فرہنگ{Yo-Yo} کا آزاد جسمی خاکہ بنا پائیں گے۔
 \item
 جان پائیں گے کہ ڈوری دار لٹو  ، ایسا جسم ہے جو \عددی{\SI{90}{\degree}} زاویہ میلان پر ہموار  اوپر نیچے لڑھکتا ہے۔
 \item
 ڈوری پر اوپر نیچے حرکت کرتے ڈوری دار لٹو کے اسراع اور گھمیری جمود کا تعلق استعمال کر پائیں گے۔
 \item
ڈوری پر اوپر یا نیچے حرکت کے دوران ڈوری دار لٹو کی ڈور میں تناو تعین کر پائیں گے۔
 \end{enumerate}
 
 \موٹا{کلیدی تصور}
 \begin{itemize}
 \item
 ڈوری دار لٹو جو ڈور پر اوپر یا نیچے حرکت کرتا ہو کو \عددی{\SI{90}{\degree}} میلان پر  ہموار لڑھکتا جسم تصور کیا جا سکتا ہے۔
 \end{itemize}
 
 \جزوحصہء{ڈوری دار لٹو}
ڈوری پر \عددی{h} فاصلہ اتر کر ڈوری دار لٹو کی مخفی توانائی میں \عددی{mgh} کمی  واقع ہو گی جبکہ  اس کی حرکی توانائی   کے مستقیم جزو \عددی{(\tfrac{1}{2}Mv_{\text{\RL{مرکزکمیت}}}^2)} اور گھمیری جزو \عددی{(\tfrac{1}{2}I_{\text{\RL{مرکزکمیت}}}\omega^2)} میں  اضافہ ہو گا۔

ڈوری دار لٹو کی ایک نئی قسم میں ڈور  کو  دھرے کے ساتھ سخت  باندھنے  کے  بجائے   ڈور  کو دھرے کے گرد   ڈھیلا گھیرا دیا جاتا ہے۔ جب لٹو نیچے اترتے ہوئے   ڈور کے پیندا   کو\قول{  ٹکرا تا} ہے، دھرے پر ڈور  اوپر وار قوت لاگو کر کے  لٹو کی نشیبی حرکت روکتی ہے۔ اس کے بعد لٹو صرف گھمیری حرکی توانائی کے ساتھ (دھرا گھیر میں چکر کاٹتا ہوا) گھومتا ہے۔ لٹو      (\قول{ سوتے ہوئے } ) چکر  کاٹتا رہتا ہے ؛ ڈور  کو جھٹکا  دینے پر      ڈور دھرے  کو پکڑتی ہے ، \قول{لٹو  بیدار ہوتا  ہے}، اور  اوپر چڑھنا شروع کرتا ہے۔ ڈور کے  پیندا پر لٹو کی گھمیری حرکی توانائی (اور یوں سونے کا دورانیہ) بڑھانے کی خاطر  لٹو کو   ساکن حالت  سے روانا کرنے کی بجائے ابتدائی رفتار \عددی{v_{\text{\RL{مرکزکمیت}}}} اور \عددی{\omega} کے ساتھ نشیب وار پھینکا جاتا ہے۔

ڈور پر نشیب وار اترنے کے دوران لٹو کا خطی اسراع \عددی{a_{\text{\RL{مرکزکمیت}}}}   جاننے کے لئے ،شکل \حوالہء{11.8} میں  میلان پر اترتے لڑھکتے جسم کی طرح،  نیوٹن کا قانون دوم (خطی اور گھمیری روپ میں)  استعمال کیا جا سکتا ہے۔ ماسوائے ذیل ،  تجزیہ بالکل اسی طرح ہو گا۔
\begin{enumerate}[1.]
\item
افق کے ساتھ \عددی{\theta} زاویے کے میلان پر اترنے کے بجائے ڈوری دار لٹو افق کے ساتھ  \عددی{\SI{90}{\degree}}  زاویے کی ڈور پر اترتا ہے۔
\item
رداس \عددی{R} کی بیرونی سطح  پر لڑھکنے کے بجائے ڈوری دار لٹو رداس \عددی{R_0} کے دھرے پر لڑھکتا ہے (شکل \حوالہء{11.9a})۔
\item
رگڑی قوت \عددی{\vec{f}_s} کے بجائے، ڈوری دار لٹو کو ڈور کا تناو \عددی{\vec{T}} آہستہ کرتا ہے (شکل \حوالہء{11.9b})۔
\end{enumerate}

موجودہ تجزیہ بھی مساوات \حوالہ{مساوات_لڑھکاو_اسراع} دے گا۔ آئیں مساوات \حوالہ{مساوات_لڑھکاو_اسراع} کی ترقیم  تبدیل 
کر کے اور  \عددی{\theta=\SI{90}{\degree}}  ڈال کر خطی اسراع ذیل لکھتے ہیں:
%eq 11.13
\begin{align}
a_{\text{\RL{مرکزکمیت}}}=-\frac{g}{1+I_{\text{\RL{مرکزکمیت}}}\!/\!MR_0^2}
\end{align}
جہاں  لٹو کے مرکز کمیت پر لٹو کا گھمیری جمود \عددی{I_{\text{\RL{مرکزکمیت}}}} اور کمیت \عددی{M} ہے۔    ڈوری پر اوپر چڑھنے کے دوران   ڈوری دار لٹو کا اسراع یہی   نشیبی اسراع ہو گا۔

%section 11.4 Torque Revisited p302
\حصہ{قوت مروڑ پر نظر ثانی}
\موٹا{مقاصد}\\
اس حصہ کو پڑھنے کے بعد آپ ذیل کے قابل ہوں گے۔
\begin{enumerate}[1.]
\item
جان پائیں گے کہ قوت مروڑ ایک  سمتیہ  مقدار ہے۔
\item
جان پائیں گے کہ جس نقطہ پر قوت مروڑ  تعین کیا جائے اس  کا   ذکر صریحاً  کرنا   لازم ہے۔
\item
ذرے پر عمل پیرا قوت کی ذرے پر قوت مروڑ ، اکائی سمتیہ ترقیم یا قدر و زاویہ ترقیم  کے روپ میں،  ذرے کے تعین گر سمتیہ  اور قوت سمتیہ  کے صلیبی ضرب سے حاصل  کر پائیں گے۔
\item
صلیبی ضرب   کا دایاں ہاتھ قاعدہ استعمال کر کے قوت مروڑ کا رخ تعین کر پائیں گے۔
\end{enumerate}

\موٹا{کلیدی تصورات}\\
\begin{itemize}
\item
تین ابعاد میں، قوت مروڑ \عددی{\vec{\tau}}ایک  سمتیہ مقدار  ہو گی  ، جو کسی مقررہ نقطہ (عموماً مبدا)  کے لحاظ سے تعین کی جاتی ہے؛ اس کی تعریف ذیل ہے:
\begin{align*}
\vec{\tau}=\vec{r}\times \vec{F}
\end{align*}
جہاں \عددی{\vec{F}} ذرے پر لاگو قوت اور  \عددی{\vec{r}}  کسی مقررہ نقطے کے لحاظ سے ذرے کا تعین گر سمتیہ ہے، جو ذرے کا مقام دیتا ہے۔
\item
قوت مروڑ \عددی{\vec{\tau}} کی قدر  \عددی{\tau} ذیل ہو گی:
\begin{align*}
\tau=rF\sin\phi=rF_{\perp}=r_{\perp}F
\end{align*}
جہاں \عددی{\vec{F}} اور \عددی{\vec{r}} کے بیچ زاویہ \عددی{\phi} ہے، \عددی{\vec{r}} کو \عددی{\vec{F}} کا عمود دار جزو \عددی{F_{\perp}}، اور \عددی{\vec{F}}  کا معیار اثر کا بازو \عددی{r_{\perp}} ہے۔
\item
قوت مروڑ \عددی{\vec{\tau}} کا رخ صلیبی ضرب کا دایاں ہاتھ قاعدہ  دیگا۔
\end{itemize}

%Torque Revisited p303
\جزوحصہء{قوت مروڑ پر نظر ثانی}
باب \حوالہ{باب_گھماو} میں مقررہ محور کے گرد گھومنے کے قابل  استوار جسم   کے لئے قوت مروڑ \عددی{\tau} کی تعریف پیش کی گئی۔ ہم قوت مروڑ کی تعریف کو وسعت دے کر  (مقررہ محور کے بجائے)  مقررہ  \ترچھا{نقطے  } کے لحاظ سے کسی بھی راہ پر حرکت کرتے ہوئے  انفرادی ذرے   کے لئے استعمال کرتے ہیں۔ راہ کا دائری ہونا ضروری نہیں، اور ہم قوت مروڑ کو سمتیہ \عددی{\vec{\tau}} لکھتے ہیں جس کا رخ کچھ بھی ہو سکتا ہے۔قوت مروڑ کی قدر کلیہ سے اور رخ صلیبی ضرب کے دایاں ہاتھ قاعدہ سے حاصل  کیا جا سکتا ہے۔

 شکل \حوالہء{11.10a} میں ، نقطہ \عددی{A}   پر مستوی \عددی{xy} میں  ایسا  ایک ذرہ دکھایا گیا ہے۔  ذرے پر ، مستوی میں قوت،  \عددی{\vec{F}} عمل کرتی ہے، اور  مبدا  \عددی{O} کے لحاظ سے ذرے کا مقام   تعین گر سمتیہ \عددی{\vec{r}} دیتا ہے۔ مقررہ نقطہ \عددی{O} کے لحاظ سے  ذرے پر عمل پیرا \اصطلاح{ قوت مروڑ }\فرہنگ{قوت مروڑ!تعریف}\فرہنگ{torque!defined} \عددی{\vec{\tau}}  کی تعریف ذیل ہے۔
 %eq 11.14
 \begin{align}\label{مساوات_لڑھکاو_سمتیہ_زاوی_قوت_مروڑ}
 \vec{\tau}=\vec{r}\times \vec{F}\quad\quad\text{\RL{قوت مروڑ کی تعریف}}
 \end{align}
 
 قوت مروڑ \عددی{\vec{\tau}}  کی اس تعریف میں سمتی (صلیبی) ضرب کی تحسیب  حصہ \حوالہء{3.3}  کے قواعد سے کی جا سکتی ہے۔ \عددی{\vec{\tau}} کا رخ جاننے کے لئے ، سمتیہ   \عددی{\vec{F}}  کو( رخ تبدیل کیے بغیر)     کھسکا  کر  اس کی دم مبدا \عددی{O} پر  رکھی جاتی ہے؛یوں ، جیسا شکل \حوالہء{11.10b} میں دکھایا گیا ہے، سمتی ضرب کے دونوں سمتیات کی  دم ایک نقطے  پر  ہو گی۔ اب ہم شکل \حوالہء{3.19a}  میں پیش  دایاں ہاتھ قاعدہ   استعمال کرتے ہوئے، دائیں ہاتھ کی چار انگلیاں \عددی{\vec{r}} پر رکھ کر  (ضرب میں پہلا سمتیہ ہے) \عددی{\vec{F}}  کی طرف جکھاتے ہیں (جو ضرب میں دوسرا سمتیہ ہے)۔سیدھا کھڑا انگوٹھا \عددی{\vec{\tau}} کا رخ دیگا۔ شکل \حوالہء{11.10b} میں  \عددی{\vec{\tau}} کا رخ محور \عددی{z} کے مثبت رخ ہے۔
 
 \عددی{\vec{\tau}} کی قدر جاننے کے لئے، ہم مساوات \حوالہء{3.27}  \عددی{(c=ab\sin\phi)} کا عمومی نتیجہ بروئے کار لاتے ہیں، جو ذیل دیگا:
 %eq 11.15
 \begin{align}\label{مساوات_لڑھکاو_صلیبی_ضرب_قدر_الف}
 \tau=rF\sin\phi
 \end{align}
 جہاں \عددی{\vec{r}} اور \عددی{\vec{F}}   کے دم ایک نقطے  پر رکھ کر  سمتیات  کے بیچ چھوٹا زاویہ \عددی{\phi} ہے۔ شکل \حوالہء{11.10b} سے ہم دیکھ سکتے ہیں کہ مساوات \حوالہ{مساوات_لڑھکاو_صلیبی_ضرب_قدر_الف} ذیل لکھی جا سکتی ہے:
 %eq 11.16
 \begin{align}\label{مساوات_لڑھکاو_صلیبی_ضرب_قدر_ب}
 \tau=rF_{\perp}
 \end{align}
 جہاں \عددی{F_{\perp}} (جو \عددی{F\sin\phi} کے برابر ہے)  \عددی{\vec{r}} کا \عددی{\vec{F}} کا عمود دار جزو ہے۔ شکل \حوالہء{11.10c} کو دیکھ کر مساوات \حوالہ{مساوات_لڑھکاو_صلیبی_ضرب_قدر_الف} ذیل بھی لکھی جا سکتی ہے:
 %eq 11.17
 \begin{align}\label{مساوات_لڑھکاو_صلیبی_ضرب_قدر_پ}
 \tau=r_{\perp}F
 \end{align}
 جہاں \عددی{r_{\perp}} (جو \عددی{r\sin\phi} کے برابر ہے)  \عددی{\vec{F}} کا معیار اثر کا بازو   (\عددی{\vec{F}} کے  خط عمل اور \عددی{O} کے بیچ عمود دار فاصلہ) ہے۔
 
 %---------------------------------
 %Checkpoint 3 p303
 \ابتدا{آزمائش}
 ذرے  کا تعین گر سمتیہ \عددی{\vec{r}}،   مثبت محور \عددی{z}    کے ہمراہ  پایا جاتا ہے۔ اگر ذرے پر قوت مروڑ (ا)  صفر ہو، (ب) محور \عددی{x} کے منفی رخ ہو، اور (ج) محور \عددی{y} کے منفی رخ ہو، قوت مروڑ پیدا کرنے والی قوت کا رخ کیا ہو گا؟
 \انتہا{آزمائش}
 %-------------------------
 
 %Sample problem 11.02 Torque on a particle due to a force p304
 \ابتدا{نمونی سوال} \موٹا{قوت کی بدولت ذرے پر قوت مروڑ}\\
 شکل \حوالہء{11.11a} میں، \عددی{\SI{2.0}{\newton}} قدر کی تین قوت ذرے پر عمل کرتی ہیں۔ ذرہ ، مستوی \عددی{xy} میں ، نقطہ \عددی{A} پر ہے، جس کا تعین گر سمتیہ \عددی{\vec{r}}، جہاں \عددی{r=\SI{3.0}{\meter}} اور \عددی{\theta=\SI{30}{\degree}} ہے۔ مبدا \عددی{O} کے لحاظ سے  ہر   قوت کی  انفرادی قوت مروڑ  کیا  ہے؟
 
 \جزوحصہء{کلیدی تصور}
 چونکہ قوت ایک مستوی میں نہیں پائی جاتیں،  ہمیں صلیبی ضرب استعمال کرنا ہو گی، جس کی قدر مساوات \حوالہ{مساوات_لڑھکاو_صلیبی_ضرب_قدر_الف} \عددی{(\tau=rF\sin\phi)}   دیگی اور رخ دایاں ہاتھ قاعدہ  دیگا۔
 
 \موٹا{حساب:}\quad
 ہم  مبدا  \عددی{O}  کے لحاظ سے قوت مروڑ جاننا چاہتے ہیں لہٰذا  دیا گیا تعین گر سمتیہ     صلیبی ضرب میں درکار سمتیہ \عددی{\vec{r}}ہو گا۔ قوت اور \عددی{\vec{r}} کے بیچ زاویہ \عددی{\phi} جاننے کے لئے ہم  شکل \حوالہء{11.11a} میں دیے گئے سمتیہ   قوت باری باری  یوں  کھسکاتے  ہیں کہ ان کی دم \عددی{O} پر ہو۔ انتقال کے بعد قوت \عددی{\vec{F}_1}، \عددی{\vec{F}_2}، اور \عددی{\vec{F}_3}  بالترتیب شکل \حوالہء{11.11b}،  شکل \حوالہء{11.11c}، اور  شکل \حوالہء{11.11d} میں، جو مستوی \عددی{xz}  کا نظارہ دیتی ہیں، دکھائی گئی ہیں (جن میں سمتیہ قوت اور تعین گر سمتیہ کے بیچ زاویے   باآسانی نظر آتے ہیں)۔ شکل \حوالہء{11.11d} میں  \عددی{\vec{r}} اور \عددی{\vec{F}_3} کے رخ کے بیچ زاویہ \عددی{\SI{90}{\degree}} ہے اور علامت  \عددی{\otimes} کہتی ہے \عددی{\vec{F}_3} صفحہ میں عمود دار  اندر رخ ہے۔ (صفحہ سے عمود دار نکلنے کی صورت میں \عددی{\odot} علامت استعمال کی جاتی ہے۔)
 
 مساوات \حوالہ{مساوات_لڑھکاو_صلیبی_ضرب_قدر_الف}  استعمال کر ذیل حاصل ہو گا۔
 \begin{align*}
 \tau_1&=rF_1\sin\phi_1=(\SI{3.0}{\meter})(\SI{2.0}{\newton})(\sin \SI{150}{\degree})=\SI{3.0}{\newton\meter}\\
 \tau_2&=rF_2\sin\phi_2=(\SI{3.0}{\meter})(\SI{2.0}{\newton})(\sin \SI{120}{\degree})=\SI{5.2}{\newton\meter}\\
 \tau_3&=rF_3\sin\phi_3=(\SI{3.0}{\meter})(\SI{2.0}{\newton})(\sin \SI{90}{\degree})=\SI{6.0}{\newton\meter}\quad \quad\text{\RL{(جواب)}}
 \end{align*}
 
اب دائیں ہاتھ قاعدہ  استعمال کرتے ہوئے،دائیں  ہاتھ کی چار انگلیاں   \عددی{\vec{r}} کے رخ    رکھ کر \عددی{\vec{F}}     کے رخ  (سمتیات کے رخ کے بیچ چھوٹے زاویے)گھماتے ہیں۔ دائیں ہاتھ کا انگوٹھا   ، جو چار انگلیوں کو عمود دار رکھا گیا ہے، قوت مروڑ کا رخ دیگا۔ یوں   \عددی{\vec{\tau}_1} شکل \حوالہء{11.11b} میں  صفحے کے اندر   جانے  کے رخ   ہو گا؛   \عددی{\vec{\tau}_2} شکل \حوالہء{11.11c} میں صفحہ سے باہر نکلنے کے رخ ہو گا؛ اور  \عددی{\vec{\tau}_3}  کا رخ شکل \حوالہء{11.11d}  میں دکھایا گیا ہے۔ تینوں قوت مروڑ سمتیات شکل \حوالہء{11.11e} میں پیش ہیں۔
 \انتہا{نمونی سوال}
 %-----------------------
 
 
 %section 11.5 Angular Momentum 0305
 \حصہ{زاوی معیار حرکت}
 \موٹا{مقاصد}\\
 اس حصہ کو پڑھنے کے بعد آپ ذیل کے قابل ہوں گے۔
 \begin{enumerate}[1.]
 \item
 جان پائیں گے کہ زاوی معیار حرکت ایک سمتیہ مقدار ہے۔
 \item
 جان پائیں گے کہ جس مقررہ نقطے کے لحاظ سے زاوی معیار حرکت   تعین کیا جائے اس  کا   ذکر صریحاً  کرنا   لازم ہے۔
 \item
 اکائی سمتیہ ترقیم  یا قدر و زاویہ ترقیم میں، ذرے کے تعین گر سمتیہ اور  معیار حرکت سمتیہ کا صلیبی ضرب لے کر ذرے کا زاوی معیار حرکت تعین کر پائیں گے۔
 \item
 صلیبی ضرب کا دایاں ہاتھ قاعدہ استعمال کر کے زاوی معیار حرکت کا رخ تعین کر پائیں گے۔
 \end{enumerate}
 
 \موٹا{کلیدی تصورات}\\
 \begin{itemize}
 \item
  ایک ذرہ، جس کا خطی معیار حرکت \عددی{\vec{p}} ، کمیت \عددی{m}، اور خطی سمتی رفتار \عددی{\vec{v}} ہو، کا مقررہ نقطے کے لحاظ سے (جو عموماً مبدا ہو گا)  زاوی معیار حرکت \عددی{\vec{\ell}} کی تعریف ذیل سمتی  مقدار ہے۔
  \begin{align*}
  \vec{\ell}=\vec{r}\times \vec{p}=m(\vec{r}\times \vec{v})
  \end{align*}
  \item
  زاوی معیار حرکت \عددی{\vec{\ell}}   کی قدر \عددی{\ell} ذیل ہو گی:
  \begin{align*}
  \ell&=rmv\sin\phi\\
  &=rp_{\perp}=rmv_{\perp}\\
  &=r_{\perp}p=r_{\perp}mv
  \end{align*}
  جہاں \عددی{\vec{r}} اور \عددی{\vec{p}} کے بیچ زاویہ \عددی{\phi} ہے، \عددی{\vec{r}} کو \عددی{\vec{p}} اور \عددی{\vec{v}} کے عمود دار
   جزو \عددی{p_{\perp}} اور \عددی{v_{\perp}} ہیں، اور مقررہ نقطے سے مبسوط  \عددی{\vec{p}}  تک عمود دار فاصلہ \عددی{r_{\perp}} ہے۔
   \item
   دایاں ہاتھ قاعدہ \عددی{\vec{\ell}} کا رخ دیگا: دائیں ہاتھ کی چار انگلیاں \عددی{\vec{r}} کے رخ پر (ابتدائی طور)   رکھ کر  انہیں گھما کر  \عددی{\vec{p}} کے رخ    پر  رکھیں۔دائیں ہاتھ کا سیدھا کھڑا انگوٹھا \عددی{\vec{\ell}} کا رخ دیگا۔
 \end{itemize}
 
 %Angular Momentum p305
 \جزوحصہء{زاوی معیار حرکت}
 یاد کریں، خطی معیار حرکت \عددی{\vec{p}} اور خطی معیار حرکت کی بقا کا اصول انتہائی طاقتور  اوزار ہیں۔انہیں استعمال کر کے نتائج  کی  ، مثلاً دو گاڑیوں کے تصادم کی تفصیل جانے بغیر  تصادم کی ،  پیشنگوئی کی جا سکتی ہے۔یہاں ہم \عددی{\vec{p}} کے زاوی  مدمقابل  پر تبصرہ  شروع کرتے ہیں جس کا اختتام حصہ \حوالہء{11.8} میں  بقائی اصول کے مدمقابل پر ہو گا۔
 
 شکل \حوالہء{11.12} میں  مستوی \عددی{xy} میں نقطہ \عددی{A} سے  کمیت \عددی{m}   اور خطی معیار حرکت \عددی{\vec{p}} \عددی{(m\vec{v}=)}  کا ذرہ گزرتا دکھایا گیا ہے۔ مبدا \عددی{O} کے لحاظ سے ذرے کا \اصطلاح{ زاوی معیار حرکت}\فرہنگ{معیار حرکت!زاوی، تعریف}\حاشیہب{angular momentum}\فرہنگ{momentum!angular, defined}  \عددی{\vec{\ell}}  سمتیہ مقدار ہو گا جس کی تعریف ذیل ہے،
 %eq 11.18
 \begin{align}\label{مساوات_لڑھکاو_سمتیہ_زاوی_قوت_مروڑ_دوم}
 \vec{\ell}=\vec{r}\times \vec{p}=m(\vec{r}\times \vec{v})\quad\quad\text{\RL{(زاوی معیار حرکت کی تعریف)}}
 \end{align}
 جہاں  \عددی{O} کے لحاظ سے ذرے کا تعین گر سمتیہ \عددی{\vec{r}} ہے۔مبدا \عددی{O}  کے لحاظ سے جب  ذرہ معیار حرکت  \عددی{\vec{p}} \عددی{(m\vec{v}=)}  کے رخ   کرتا ہے،  اس کا تعین گر سمتیہ  \عددی{\vec{r}} مبدا \عددی{O} کے گرد گھمیری حرکت کرتا  ہے۔ غور کریں، \عددی{O}   پر زاوی معیار حرکت کے لئے ضروری نہیں کہ ذرہ خود  \عددی{O} کے گرد گھومتا ہو۔ مساوت \حوالہ{مساوات_لڑھکاو_سمتیہ_زاوی_قوت_مروڑ} اور مساوات \حوالہ{مساوات_لڑھکاو_سمتیہ_زاوی_قوت_مروڑ_دوم}  کا موازنہ کرنے سے معلوم ہو گا کہ زاوی معیار حرکت اور خطی معیار حرکت کا آپس میں وہی رشتہ ہے جو قوت مروڑ کا قوت  کے ساتھ ہے۔ بین الاقوامی نظام اکائی میں زاوی معیار حرکت کی اکائی کلوگرام مربع   میٹر  فی سیکنڈ \عددی{(\si{\kilo\gram\meter\squared\per\second})} ہے، جو جاول سیکنڈ \عددی{(\si{\joule\second})} کا معادل ہے۔
 
 \موٹا{رخ۔}\quad
 شکل \حوالہء{11.12} میں  زاوی معیار حرکت سمتیہ \عددی{\vec{\ell}} کا رخ جاننے کے لئے، ہم  سمتیہ \عددی{\vec{p}} کو کھسکا  کر کے اس کی دم مبدا \عددی{O}  پر رکھتے ہیں۔اس کے بعد صلیبی ضرب کا  دایاں ہاتھ قاعدہ استعمال کر کے انگلیوں  کو \عددی{\vec{r}} سے \عددی{\vec{p}}  لپیٹتے ہیں۔ سیدھا کھڑا انگوٹھا \عددی{\vec{\ell}} کا رخ ، شکل \حوالہء{11.12} میں ، محور \عددی{z}   کا مثبت رخ دیتا ہے۔ یہ مثبت رخ،  محور \عددی{z} پر تعین گر سمتیہ  \عددی{\vec{r}} کے خلاف گھڑی گھماو کے عین مطابق ہے، جو ذرے کی حرکت سے پیدا ہوتی ہے۔ (\عددی{\vec{\ell}} کی منفی قیمت محور \عددی{z} پر گھڑی وار  گھماو ظاہر کرے گی۔)
 
 \موٹا{قدر۔}\quad
     زاوی معیار حرکت \عددی{\vec{\ell}}  کی قدر  معلوم  کرنے کے لئے ہم   مساوات \حوالہء{3.27} کا عمومی نتیجہ  ذیل لکھتے ہیں:
 %eq 11.19
 \begin{align}\label{مساوات_لڑھکاو_زاوی_قدر_معیار_حرکت}
 \ell=rmv\sin\phi
 \end{align}
 جہاں \عددی{\vec{r}} اور \عددی{\vec{p}}  کی دم ایک نقطہ پر رکھ کر  سمتیات کے بیچ چھوٹا زاویہ \عددی{\phi} ہے۔ شکل \حوالہء{11.12a} دیکھ کر مساوات \حوالہ{مساوات_لڑھکاو_زاوی_قدر_معیار_حرکت} ذیل لکھی جا سکتی ہے:
 %eq 11.20
 \begin{align}
 \ell=rp_{\perp}=rmv_{\perp}
 \end{align}
 جہاں \عددی{\vec{r}} کو  \عددی{\vec{p}} کا عمود دار  جزو \عددی{p_{\perp}} ہے، اور  \عددی{\vec{r}} کو  \عددی{\vec{v}} کا عمود دار  جزو \عددی{v_{\perp}} ہے۔ شکل \حوالہء{11.12b} دیکھ کر مساوات \حوالہ{مساوات_لڑھکاو_زاوی_قدر_معیار_حرکت} ذیل بھی لکھی جا سکتی ہے:
 %eq 11.21
 \begin{align}
 \ell=r_{\perp}p=r_{\perp}mv
 \end{align}
 جہاں مبسوط   \عددی{\vec{p}}  سے \عددی{O} کا  عمود دار فاصلہ \عددی{r_{\perp}} ہے۔
 
 \موٹا{اہم۔}\quad
 دو  پہلو پر غور کریں: (1)  زاوی معیار حرکت صرف  کسی مخصوص مبدا کے لحاظ سے معنی خیز ہے اور (2)  اس کا رخ ہر صورت اس مستوی کو عمودی ہو گا جو   تعین گر سمتیہ \عددی{\vec{r}} اور  خطی معیار حرکت  سمتیہ  \عددی{\vec{p}} مل کر بناتے ہیں۔
 
 %Checkpoint 4 p305
 
