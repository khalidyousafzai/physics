%starting from top of P239
%ch 9 "Elastic Collision In One Dimension"
\باب{مرکز کمیت اور خطی معیار حرکت}
\حصہ{ایک بُعد میں  لچکی تصادم}
حرکی توانائی کی بقا درج ذیل لکھی جائے گی۔
\begin{align}\label{مساوات_مرکز_کمیت_حرکی_توانائی_کی_بقا}
\frac{1}{2}m_1v_{1i}^2+\frac{1}{2}m_2v_{2i}^2=\frac{1}{2}m_1v_{1f}^2+\frac{1}{2}m_2v_{2f}^2
\end{align}
ان ہمزاد  مساوات کو \عددی{v_{1f}} اور \عددی{v_{2f}} کے لئے حل کرنے کی خاطر  ہم مساوات \حوالہء{9.71} کو
\begin{align}\label{مساوات_مرکز_کمیت_بقا_معار_دوم}
m_1(v_{1i}-v_{1f})=-m_2(v_{2i}-v_{2f})
\end{align}
اور مساوات \حوالہ{مساوات_مرکز_کمیت_حرکی_توانائی_کی_بقا} درج ذیل صورت میں لکھتے ہیں۔
\begin{align}\label{مساوات_مرکز_کمیت_بقا_دوم}
m_1(v_{1i}-v_{1f})(v_{1i}+v_{1f})=-m_2(v_{2i}-v_{2f})(v_{2i}+v_{2f})
\end{align}
مساوات \حوالہ{مساوات_مرکز_کمیت_بقا_دوم} کو مساوات \حوالہ{مساوات_مرکز_کمیت_بقا_معار_دوم} سے تقسیم کر کے کچھ الجبرا کے بعد درج ذیل حاصل ہوں گے۔
\begin{align}\label{مساوات_مرکز_کمیت_اختتامی_الف}
v_{1f}=\frac{m_1-m_2}{m_1+m_2}v_{1i}+\frac{2m_2}{m_1+m_2}v_{2i}
\end{align}
اور
\begin{align}\label{مساوات_مرکز_کمیت_اختتامی_ب}
v_{2f}=\frac{2m_1}{m_1+m_2}v_{1i}+\frac{m_2-m_1}{m_1+m_2}v_{2i}
\end{align}
یاد رہے، زیر نوشت \عددی{1} اور \عددی{2} کسی خاص ترتیب سے مختص نہیں کیے گئے۔  مساوات \حوالہء{9.19} میں  اور مساوات \حوالہ{مساوات_مرکز_کمیت_اختتامی_الف} اور مساوات  \حوالہ{مساوات_مرکز_کمیت_اختتامی_ب} میں  ان زیر نوشت کو آپس میں بدل کر لکھنے  مساوات کی وہی جوڑی ملتی ہے۔ اس پر بھی توجہ دیں کہ \عددی{v_{2i}=0}  لینے سے، شکل \حوالہء{9.18} میں جسم \عددی{2} ساکن ہدف ہو گا، اور مساوات \حوالہ{مساوات_مرکز_کمیت_اختتامی_الف}  اور مساوات \حوالہ{مساوات_مرکز_کمیت_اختتامی_ب} ہمیں  بالترتیب مساوات \حوالہء{9.67} اور مساوات \حوالہء{9.68} دیتی ہیں۔ 



\ابتدا{آزمائش}
شکل \حوالہء{9.18} میں گولے کا ابتدائی معیار حرکت \عددی{\SI{6}{\kilo\gram\meter\per\second}} اور اختتامی معیار حرکت (ا)  \عددی{\SI{2}{\kilo\gram\meter\per\second}} اور (ب) \عددی{\SI{-2}{\kilo\gram \meter\per\second}} ہونے کی صورت میں   ہدف کا  اختتامی خطی معیار حرکت کیا ہو گا؟ اگر گولے کی  ابتدائی اور  اختتامی حرکی توانائی بالترتیب  \عددی{\SI{5}{\joule}} اور \عددی{\SI{2}{\joule}} ہو، ہدف کی اختتامی حرکی توانائی کیا ہو گی؟
\انتہا{آزمائش}
%----------------------

\ابتدا{نمونی سوال} \quad \موٹا{لچکی تصادم در لچکی تصادم}
شکل \حوالہء{9.20a} میں \عددی{v_{1i}=\SI{10}{\meter\per\second}} سے چلتا ہوا سل 1 دو ساکن سلوں کی طرف بڑھتا  ہے۔تینوں سل ایک لکیر پر  ہیں۔ یہ سل  \عددی{2} سے ٹکراتا ہے جو آگے سل \عددی{3} سے  جا کر ٹکراتا ہے، جس کی کمیت \عددی{m_3=\SI{6.0}{\kilo\gram}} ہے۔ دوسرے  تصادم  کے بعد سل \عددی{2} دوبارہ ساکن ہے،  اور سل \عددی{3} کی رفتار  \عددی{v_{3f}=\SI{5.0}{\meter\per\second}} ہے (شکل \حوالہء{9.20b})۔ دونوں تصادم لچکی ہیں۔ سل \عددی{1} اور سل \عددی{2} کی  کمیتیں کیا ہیں؟ سل \عددی{1} کی اختتامی رفتار
 \عددی{v_{1f}} کیا ہے؟
 
 \موٹا{کلیدی تصورات}
 
 چونکہ ہم تصادم لچکدار تصور کرتے ہیں لہٰذا میکانی توانائی کی  بقا ہو گی (یوں  ٹکر کی آواز، گرمی، اور ارتعاش کی بدولت توانائی کا  ضیاع نظر انداز کیا جاتا ہے)۔ کوئی  بیرونی افقی قوت  سلوں پر عمل نہیں کرتی لہٰذا محور \عددی{x} پر خطی معیار حرکت کی بقا ہو گی۔ ان دو وجوہات کی بنا پر ہم دونوں تصادم پر  مساوات \حوالہء{9.67} اور مساوات \حوالہء{9.68} کا اطلاق کر سکتے ہیں۔
 
 \موٹا{حساب}\quad
 پہلے  تصادم  سے آغاز کرتے ہوئے ہمیں  اتنے زیادہ  نا معلوم متغیرات سے واسطہ ہو گا کہ آگے بڑھنا  مشکل ہو گا: ہم سلوں کی کمیت اور اختتامی سمتی رفتار نہیں جانتے۔ آئیں پہلے تصادم سے آغاز کریں، جس میں سل \عددی{3} کے ساتھ ٹکرانے کے بعد سل \عددی{2} رکتی ہے۔ مساوات \حوالہء{9.67} کا  اطلاق  اس تصادم پر کرتے ہیں جہاں ترقیم تبدیل کرتے ہوئے \عددی{v_{2i}}  تصادم سے قبل سل \عددی{2} کی  رفتار اور \عددی{v_{2f}}  تصادم کے بعد اس کی رفتار  دیتی ہیں۔یوں درج ذیل ہو گا۔
 \begin{align*}
 v_{2f}=\frac{m_2-m_3}{m_2+m_3}v_{2i}
 \end{align*}
 اس میں \عددی{v_{2f}=0} (سل \عددی{2}  رک جاتا ہے) ڈالنے کے بعد  \عددی{m_3=\SI{6.0}{\kilo\gram}}  ڈال کر درج ذیل حاصل ہو گا۔
 \begin{align*}
 m_2&=m_3=\SI{6.0}{\kilo\gram}&&\text{\RL{(جواب)}}
 \end{align*}
اسی طرح  ترقیم  تبدیل کر کے دوسرے تصادم کے لئے  مساوات \حوالہء{9.68} لکھتے ہیں
\begin{align*}
v_{3f}=\frac{2m_2}{m_2+m_3}v_{2i}
\end{align*}
جہاں \عددی{v_{3f}} تیسرے سل کی اختتامی سمتی رفتار ہے۔ اس میں \عددی{m_2=m_3} ڈالنے کے بعد \عددی{v_{3f}=\SI{5.0}{\meter\per\second}} ڈال کر درج ذیل حاصل ہو گا۔
\begin{align*}
v_{2i}=v_{3f}=\SI{5.0}{\meter\per\second}
\end{align*}

%p240

آئیں اب پہلے تصادم پر غور کریں؛ ہمیں سل \عددی{2} کے لئے مستعمل ترقیم پر توجہ دینی ہو گی: تصادم کے بعد سل \عددی{2}  کی سمتی رفتار \عددی{v_{2f}} وہی ہے جو تصادم سے قبل  اس کی سمتی رفتار \عددی{v_{2i}=\SI{5.0}{\meter\per\second}} تھی۔ پہلے تصادم پر مساوات \حوالہء{68} کا اطلاق  کر کے دی گئی  \عددی{v_{1i}=\SI{10}{\meter\per\second}}
ڈال کر   ذیل    ہو گا
\begin{align*}
v_{2f}&=\frac{2m_1}{m_1+m_2}v_{1i}\\
\SI{5.0}{\meter\per\second}&=\frac{2m_1}{m_1+m_2}(\SI{10}{\meter\per\second})
\end{align*}
جو ذیل دیگا۔
\begin{align*}
m_1&=\frac{1}{3}m_2=\frac{1}{3}(\SI{6.0}{\kilo\gram})=\SI{2.0}{\kilo\gram}&&\text{\RL{(جواب)}}
\end{align*}
یہ  نتیجہ اور دی گئی \عددی{v_{1i}}  استعمال کرتے ہوئے  پہلے تصادم  پر مساوات \حوالہء{9.67} کا اطلاق کر کے درج ذیل لکھا جا سکتا ہے۔
\begin{align*}
v_{1f}&=\frac{m_1-m_2}{m_1+m_2}v_{1i}\\
&=\frac{\tfrac{1}{3}m_2-m_2}{\tfrac{1}{3}m_2+m_2}(\SI{10}{\meter\per\second})=\SI{-5.0}{\meter\per\second}&&\text{\RL{(جواب)}}
\end{align*}
\انتہا{نمونی سوال}
%--------------------

\حصہ{دو ابعاد میں تصادم}
\جزوحصہء{مقاصد}
اس حصہ کو پڑھنے کے بعد آپ درج ذیل کے قابل ہوں گے۔

جدا نظام کے لئے جس میں دو بُعدی تصادم  واقع ہو ، ہر ایک محور پر   معیار حرکت کی بقا  کا اطلاق کرتے ہوئے  ، تصادم  کے بُعد محور پر معیار حرکت کے  اجزاء  کا   اسی محور پر تصادم سے قبل معیار حرکت کے   اجزاء کے ساتھ رشتہ جان سکیں۔

جدا نظام کے لئے جس میں دو بُعدی لچکی تصادم واقع ہو، (ا)  ، ہر ایک محور پر   معیار حرکت کی بقا  کا اطلاق کرتے ہوئے  ، تصادم  کے بعد محور پر معیار حرکت کے  اجزاء  کا   اسی محور پر تصادم سے قبل معیار حرکت کے   اجزاء کے ساتھ رشتہ جان سکیں اور (ب)کل   حرکی توانائی  کی بقا کا اطلاق کر کے تصادم سے قبل اور تصادم کے بعد حرکی توانائیوں کا رشتہ جان سکیں۔

\جزوحصہء{کلیدی  تصور}
اگر دو جسم ٹکرائیں اور ان کی حرکت ایک محور پر نہ ہو (تصادم  آمنے سامنے سے  نہیں ہے)، تصادم دو بُعدی ہو گا۔ اگر دو جسمی نظام بند اور جدا ہو،تصادم پر  معیار حرکت کی بقا کے   قانون کا اطلاق ہو گا لہٰذا درج  ہو گا۔
\begin{align*}
\vec{P}_{1i}+\vec{P}_{2i}=\vec{P}_{1f}+\vec{P}_{2f}
\end{align*}
یہ قانون اجزاء کی صورت میں دو مساوات   (ہر  بُعد کے لئے ایک مساوات) دیگا جو تصادم کو بیان کرتی ہیں۔ اگر تصادم لچکی بھی ہو (جو  ایک خصوصی صورت ہے)، تصادم کے دوران حرکی توانائی کی بقا (ذیل)  تیسری مساوات دیگی۔
\begin{align*}
K_{1i}+K_{2i}=K_{1f}+K_{2f}
\end{align*}

\حصہء{دو بُعد میں تصادم}
جب دو اجسام کا تصادم  ہو، اجسام کس  رخ حرکت    کرتے ہیں ، اس کا تعین ان کے بیچ ضرب (جھٹکا ) کرتی ہے۔ بالخصوص، جب تصادم آمنے سامنے سے نہ ہو، اجسام اپنے اپنے   ابتدائی محور پر نہیں رہتے۔ ایسے دو بُعدی تصادم میں  جو بند، اور جدا نظام میں واقع ہو،  کل خطی معیار حرکت کی بقا  ہو گی۔
\begin{align}\label{مساوات_مرکز_کمیت_خطی_معیار_حرکت_بقا}
\vec{P}_{1i}+\vec{P}_{2i}=\vec{P}_{1f}+\vec{P}_{2f}
\end{align}
اگر تصادم لچکی بھی ہو (جو  ایک خصوصی صورت ہے)، تب کل حرکی توانائی کی بقا بھی ہو گی۔
\begin{align}\label{مساوات_مرکز_کمیت_حرکی_توانائی_بقا}
K_{1i}+K_{2i}=K_{1f}+K_{2f}
\end{align}

دو بُعدی تصادم  کا تجزیہ کرنے کے لئے مساوات \حوالہ{مساوات_مرکز_کمیت_خطی_معیار_حرکت_بقا} کو \عددی{xy} محددی نظام کے اجزاء کی صورت میں لکھنا زیادہ مفید ثابت ہوتا ہے۔ مثال کے طور پر، شکل \حوالہء{9.21} میں  ساکن ہدف  کو  گولا بغلی ( آمنے سامنے سے نہیں )  ٹکراتا ہے۔  ان  کے بیچ ضرب،  اجسام کو محور \عددی{x}، جس پر گولا ابتدائی طور حرکت میں تھا، کے لحاظ سے \عددی{\theta_1} اور \عددی{\theta_2}  زاویوں پر بھیجتی ہے۔ یہاں ہم مساوات \حوالہ{مساوات_مرکز_کمیت_خطی_معیار_حرکت_بقا} کو محور \عددی{x} کے ہمراہ ذیل
\begin{align}\label{مساوات_مرکز_کمیت_معیار_ایکس_جزو}
m_1v_{1i}=m_1v_{1f}\cos\theta_1+m_2v_{2f}\cos\theta_2
\end{align}
اور محور \عددی{y} کے ہمراہ ذیل لکھیں گے۔
\begin{align}\label{مساوات_مرکز_کمیت_معیار_وائے_جزو}
0=-m_1v_{1f}\sin \theta_1+m_2v_{2f}\sin\theta_2
\end{align}
ہم مساوات \حوالہ{مساوات_مرکز_کمیت_حرکی_توانائی_بقا} کو  (اس خصوصی صورت کے لئے) رفتار کے روپ میں لکھ سکتے ہیں۔
\begin{align}\label{مساوات_مرکز_حرکی_بصورت_رفتار}
\frac{1}{2}m_1v_{1i}^2&=\frac{1}{2}m_1v_{1f}^2+\frac{1}{2}m_2v_{2f}^2&&\text{\RL{(حرکی توانائی)}}
\end{align}
مساوات \حوالہ{مساوات_مرکز_کمیت_معیار_ایکس_جزو} تا مساوات \حوالہ{مساوات_مرکز_حرکی_بصورت_رفتار} میں سات متغیر ہیں: دو کمیت، \عددی{m_1} اور \عددی{m_2}؛ تین رفتار، \عددی{v_{1i}}، \عددی{v_{1f}}، اور \عددی{v_{2f}}؛ اور دو زاویے، \عددی{\theta_1} اور \عددی{\theta_2}۔اگر ہم  ان میں سے کوئی بھی چار متغیرات جانتے ہوں،  باقی  تین متغیرات ان تین مساوات کو حل کر کے   معلوم کیے جا سکتے ہیں۔

%---------------------------
\ابتدا{نمونی سوال}
فرض کریں شکل \حوالہء{9.21} میں گولے کا  ابتدائی معیار حرکت  \عددی{\SI{6}{\kilo\gram\meter\per\second}} ،  جبکہ  اختتامی  معیار حرکت کا \عددی{x} جزو \عددی{\SI{4}{\kilo\gram\meter\per\second}} اور اختتامی معیار حرکت کا \عددی{y} جزو \عددی{\SI{-3}{\kilo\gram\meter\per\second}} ہے۔ ہدف کے (ا) اختتامی معیار  حرکت کا \عددی{x} جزو اور (ب) اختتامی معیار حرکت کا \عددی{y} جزو کیا ہوں گے؟
\انتہا{نمونی سوال}
%-----------------------------

\حصہ{متغیر کمیت کے  نظام: ہوائی بان}
\جزوحصہء{مقاصد}
اس حصہ کو پڑھنے کے بعد آپ  ذیل کے قابل ہوں گے۔

\اصطلاح{ہوائی بان }\فرہنگ{ہوائی بان}\حاشیہب{rocket}\فرہنگ{rocket} کی پہلی مساوات استعمال کر کے ہوائی بان کی کمیت میں کمی کی شرح، ہوائی بان کے لحاظ سے اخراجی مادے کی اضافی رفتار، ہوائی بان کی کمیت، اور ہوائی بان کی اسراع کا رشتہ جان پائیں گے۔

ہوائی بان کی دوسری مساوات استعمال کر کے اخراجی مادے کی اضافی رفتار کے لحاظ سے ہوائی بان کی رفتار ، اور ہوائی بان  کی ابتدائی اور اختتامی کمیت کا رشتہ جان پائیں گے۔

ایک ایسا حرکت پذیر  نظام  جس کی کمیت دی گئی شرح سے تبدیل ہوتی ہو کے لئے  اس شرح    اور معیار حرکت میں تبدیلی  کا رشتہ جان پائیں گے۔

\جزوحصہء{کلیدی تصورات}
بیرونی قوتوں کی غیر موجودگی میں ہوائی بان درج ذیل لمحاتی شرح سے  اسراع پذیر ہو گا،
\begin{align*}
Rv_{\text{\RL{اضافی}}}&=Ma &&\text{\RL{(ہوائی بان کی پہلی مساوات)}}
\end{align*}
جہاں \عددی{M} ہوائی بان کی لمحاتی کمیت (بشمول غیر استعمال شدہ ایندھن) ، \عددی{R} ایندھن  کے استعمال کی شرح، اور \عددی{v_{\text{\RL{اضافی}}}} ہوائی بان کے لحاظ سے    اخراج کی اضافی رفتار ہے۔ جزو \عددی{Rv_{\text{\RL{اضافی}}}} ہوائی بان انجن کا دھکا ہے۔

مستقل \عددی{R} اور \عددی{v_{\text{\RL{اضافی}}}} کی صورت میں اگر  ہوائی بان  کی رفتار \عددی{v_i} سے تبدیل ہو کر    \عددی{v_f}  ہو جائے، اور کمیت \عددی{M_i} سے تبدیل ہو کر \عددی{M_f} ہو جائے تب درج ذیل ہو گا۔
\begin{align*}
v_f-v_i=&v_{\text{\RL{اضافی}}}\ln\frac{M_i}{M_f}&&\text{\RL{(ہوائی بان کی دوسری مساوات)}}
\end{align*}

\جزوحصہء{متغیر کمیت کے  نظام: ہوائی بان}
اب تک ہم فرض کرتے رہے ہیں کہ نظام کی کل  کمیت اٹل ہے۔ بعض اوقات، مثلاً ہوائی بان میں، ایسا نہیں ہو گا۔اڑان سے قبل  \اصطلاح{چبوترہ روانگی   }\فرہنگ{چبوترہ!روانگی}\حاشیہب{launching pad}\فرہنگ{launching pad} پر کھڑے ہوائی بان کی زیادہ تر کمیت دراصل ایندھن ہو گی، جو  آخر کار جل کر ہوائی بان کے انجن کی  \اصطلاح{ٹونٹی }\فرہنگ{ٹونٹی}\حاشیہب{nozzle}\فرہنگ{nozzle} سے   دھویں کی شکل میں  خارج ہو گا۔  اسراع پذیر ہوائی بان کی متغیر کمیت سے نپٹنے کی خاطر نیوٹن کے   دوسرے قاعدے کا اطلاق، صرف  ہوائی بان  کی بجائے،  ہوائی بان اور  خارجی مواد  دونوں  کو اکٹھا  لیتے ہوئے کیا جاتا ہے۔ہوائی بان کی اسراع کے دوران  اس نظام کی کمیت  تبدیل نہیں ہو گی۔

\جزوحصہء{اسراع کی تلاش}
فرض کریں ہم  جمودی حوالہ  چھوکٹ کے لحاظ سے ساکن بیٹھے گہری فضا میں، جہاں کوئی تجاذبی یا ہوائی  کی  رگڑ  ی قوت موجود نہیں،  ہوائی بان کو اسراع کرتا دیکھ رہے ہیں۔ اس یک بُعدی حرکت  کے لئے   ہم ، اختیاری لمحہ \عددی{t} پر، ہوائی بان کی کمیت \عددی{M} اور سمتی رفتار \عددی{v }  فرض کرتے ہیں (شکل \حوالہء{9.22a})۔

شکل \حوالہء{9.22b}  وقتی دورانیہ \عددی{\dif t} کے بعد صورت حال پیش کرتی ہے۔ ہوائی بان کی سمتی رفتار \عددی{v+\dif v} اور کمیت \عددی{M+\dif M} ہیں، جہاں کمیت میں تبدیلی \عددی{\dif M}\ترچھا{ منفی مقدار } ہے۔ وقفہ \عددی{\dif t} کے دوران ہوائی بان سے  اخراجی مواد کی کمیت \عددی{-\dif M}  اور جمودی  حوالہ چھوکٹ کے لحاظ  سے    مواد کی سمتی  رفتار \عددی{U} ہے۔

\جزوحصہء{معیار حرکت کی بقا ہو گی}
ہمارا  نظام  ہوائی بان اور وقفہ \عددی{\dif t} میں اخراجی مواد پر مشتمل ہے۔ نظام بند اور  جدا ہے لہٰذا وقفہ \عددی{\dif t} کے دوران نظام کی خطی معیار حرکت کی بقا لازمی ہے۔ یوں ذیل ہو گا
\begin{align}\label{مساوات_مرکز_کمیت_معیار_بقا_لازمی}
P_i=P_f
\end{align}
جہاں زیر نوشت  \عددی{i} اور \عددی{f} بالترتیب  وقفہ  \عددی{\dif t} کے آغاز میں اور  اس کے اختتام پر قیمتیں ظاہر کرتی ہیں۔ مساوات \حوالہ{مساوات_مرکز_کمیت_معیار_بقا_لازمی} درج ذیل لکھی جا سکتی ہے
\begin{align}\label{مساوات_مرکز_کمیت_وقفہ_کے_دوران}
Mv=-\dif M \,U+(M+\dif M)(v+\dif v)
\end{align}
جہاں دائیں ہاتھ پہلا جزو وقفہ  \عددی{\dif t} کے دوران خارج کردہ مواد کا  خطی معیار حرکت اور  دوسرا جزو وقفہ  \عددی{\dif t} کے اختتام  پر ہوائی بان کا خطی معیار حرکت  ہے۔

\جزوحصہء{اضافی رفتار کا  استعمال}
مساوات \حوالہ{مساوات_مرکز_کمیت_وقفہ_کے_دوران} کی سادہ صورت  ہوائی بان اور اخراجی مواد کے بیچ اضافی رفتار \عددی{v_{\text{\RL{اضافی}}}}  استعمال کرکے  حاصل کی جا سکتی ہے۔اضافی رفتار اور چھوکٹ کے لحاظ سے سمتی رفتاروں  کے بیچ درج ذیل تعلق پایا جاتا ہے۔
\begin{align*}
\left(\parbox{3cm}{\centering چھوکٹ کے لحاظ سے ہوائی بان کی سمتی رفتار}\right)=\left(\parbox{3cm}{\centering اخراجی مواد کے لحاظ سے ہوائی بان کی سمتی رفتار}\right)+\left(\parbox{3cm}{\centering چھوکٹ کے لحاظ سے اخراجی مواد کی سمتی رفتار}\right)
\end{align*}
اس کو علامتی روپ میں لکھتے ہیں۔
\begin{align}
(v+\dif v)&=v_{\text{\RL{اضافی}}}+U\notag\\
U&=v+\dif v-v_{\text{\RL{اضافی}}}\hspace{3cm}\text{یعنی}
\end{align}
اس نتیجہ کو مساوات \حوالہ{مساوات_مرکز_کمیت_وقفہ_کے_دوران} میں \عددی{U} کی جگہ ڈال کر کچھ الجبرا کے بعد ذیل حاصل ہو گا۔
\begin{align}\label{مساوات_مرکز_کمیت_تعلق_بان}
-\dif M\,v_{\text{\RL{اضافی}}}=M\dif v
\end{align}
دونوں اطراف \عددی{\dif t} سے تقسیم کرتے ہیں۔
\begin{align}\label{مساوات_مرکز_کمیت_اضافی_رفتار_الف}
-\frac{\dif M}{\dif t}v_{\text{\RL{اضافی}}}=M\frac{\dif v}{\dif t}
\end{align}
ہم \عددی{\dif M\!/\!\dif t}  (جو ہوائی بان کی کمیت میں کمی کی شرح ہے)  کو \عددی{-R} لکھتے ہیں، جہاں \عددی{R} ایندھن  جلنے کی (مثبت) شرح ہے، اور \عددی{\dif v\!/\!\dif t} ہوائی بان کی اسراع ہے۔ ان تبدیلیوں کے ساتھ مساوات \حوالہ{مساوات_مرکز_کمیت_اضافی_رفتار_الف} ذیل روپ اختیار کرتی ہے۔
\begin{align}\label{مساوات_مرکز_کمیت_ہوائی_بان_پہلی}
Rv_{\text{\RL{اضافی}}}=Ma  \quad \text{\RL{(ہوائی بان کی پہلی مساوات)}}
\end{align}
ہر   لمحے پر مقادیر کی قیمتیں مساوات \حوالہ{مساوات_مرکز_کمیت_ہوائی_بان_پہلی}     مطمئن  کرتی ہیں۔

مساوات \حوالہ{مساوات_مرکز_کمیت_ہوائی_بان_پہلی} کا بایاں  ہاتھ  قوت کا بُعد \عددی{(\si{\kilo\gram\per\second}\cdot\si{\meter\per\second}=\si{\kilo\gram}\cdot\si{\meter\per\second\squared}=\si{\newton})}  رکھتا ہے اور  صِرف ہوائی بان کی بناوٹ پر منحصر ہے؛ یعنی، شرح \عددی{R} پر ، جس سے ایندھن (کمیت ) صَرف کیا جاتا ہے ، اور  رفتار \عددی{v_{\text{\RL{اضافی}}}} پر،  جس سے   یہ کمیت ہوائی بان سے خارج کی جاتی ہے۔ہم اس جزو \عددی{Rv_{\text{\RL{اضافی}}}}  کو ہوائی بان کی\اصطلاح{   قوت  دھکیل }\فرہنگ{قوت دھکیل}\حاشیہب{thrust}\فرہنگ{thrust} کہتے   اور  \عددی{T} سے ظاہر کرتے ہیں۔ مساوات \حوالہ{مساوات_مرکز_کمیت_ہوائی_بان_پہلی} کو \عددی{T=Ma} لکھ کر نیوٹن کا دوسرا قانون حاصل ہوتا ہے، جہاں اس لمحے پر جب ہوائی بان کی کمیت \عددی{M} ہے اس کی اسراع \عددی{a} ہے۔

\جزوحصہء{سمتی رفتار کی تلاش}
ہم جاننا چاہتے ہیں کہ جیسے جیسے ہوائی بان  ایندھن صَرف کرتا ہے اس کی  سمتی رفتار کیسے تبدیل ہو گی۔ مساوات \حوالہ{مساوات_مرکز_کمیت_تعلق_بان}  ذیل کہتی ہے۔
\begin{align*}
\dif v=-v_{\text{\RL{اضافی}}}\frac{\dif M}{M}
\end{align*}
اس کے تکمل
\begin{align*}
\int_{v_i}^{v_f}\dif v=-v_{\text{\RL{اضافی}}}\int_{M_i}^{M_f}\frac{\dif M}{M}
\end{align*}
میں \عددی{M_i}ہوائی بان کی  ابتدائی کمیت اور \عددی{M_f} اختتامی کمیت ہے۔ تکمل لینے سے ذیل حاصل ہو گا
\begin{align}\label{مساوات_مرکز_کمیت_ہوائی_بان_دوم_مساوات}
v_f-v_i=v_{\text{\RL{اضافی}}}\ln\frac{M_i}{M_f}\quad \text{\RL{(ہوائی بان کی دوسری مساوات)}}
\end{align}
جو  ہوائی بان کی کمیت \عددی{M_i} سے گھٹ کر  \عددی{M_f} ہونے کی صورت میں ہوائی بان  کی رفتار میں اضافہ دیتی ہے۔ (مساوات \حوالہ{مساوات_مرکز_کمیت_ہوائی_بان_دوم_مساوات} میں علامت \عددی{\ln}\اصطلاح{ قدرتی لوگارتھم }\فرہنگ{لوگارتھم!قدرتی}\حاشیہب{natural logarithm}\فرہنگ{logarithm!natural} ظاہر کرتی ہے۔) ہم یہاں \اصطلاح{  کثیرالمراحل }\فرہنگ{کثیرالمراحل}\حاشیہب{multistage}\فرہنگ{multistage} ہوائی بان کی افادیت   جان سکتے ہیں  جو   ایندھن ختم ہونے پر خالی  ٹینکی سے چھٹکارا حاصل کر کے \عددی{M_f} گھٹاتا ہے۔ مثالی ہوائی بان  مطلوبہ مقام پر  صرف ضروری ساز و سامان کے ساتھ پہنچے گا۔

%----------------------------
\ابتدا{نمونی سوال}\موٹا{ہوائی بان کا انجن، قوت دھکیل، اسراع}
اس باب کی تمام گزشتہ مثالوں میں نظام کی کمیت اٹل تھی۔ یہاں ہم ایسے نظام ( ہوائی بان ) کی بات کرتے ہیں جس کی کمیت بتدریج کم ہوتی ہے۔ ایک  ہوائی بان  جس کی ابتدائی کمیت \عددی{M_i=\SI{850}{\kilo\gram}} ہے  \عددی{R=\SI{2.3}{\kilo\gram\per\second}}  شرح سے   ایندھن  صَرف کرتا ہے۔ ہوائی بان کے لحاظ سے اخراجی مواد کی
 رفتار \عددی{v_{\text{\RL{اضافی}}}=\SI{2800}{\meter\per\second}} ہے۔ (ا)  ہوائی بان کا انجن کتنی قوت دھکیل پیدا کرتا ہے؟
 
 \جزوحصہء{کلیدی تصور}
 مساوات \حوالہ{مساوات_مرکز_کمیت_ہوائی_بان_پہلی} کے تحت ایندھن صَرف کرنے کی شرح \عددی{R}  کو اخراجی مواد کی اضافی رفتار \عددی{v_{\text{\RL{اضافی}}}} سے ضرب دینے سے قوت دھکیل  \عددی{T} حاصل ہو گی۔
 
حساب:\quad
یوں درج ذیل ہو گا۔
\begin{align*}
T&=Rv_{\text{\RL{اضافی}}}=(\SI{2.3}{\kilo\gram\per\second})(\SI{2800}{\meter\per\second})\\
&=\SI{6440}{\newton}\approx \SI{6400}{\newton}\quad \quad \text{\RL{(جواب)}}
\end{align*}
(ب) ہوائی بان کی ابتدائی اسراع کیا ہو گی؟

\جزوحصہء{کلیدی  تصور}
ہم ہوائی بان کی قوت دھکیل \عددی{T} اور اس کی اسراع کی قدر \عددی{a} کا رشتہ \عددی{T=Ma} جانتے ہیں، جہاں  \عددی{M} ہوائی بان کی کمیت ہے۔ لیکن، جیسے جیسے ایندھن صَرف ہوتا ہے \عددی{M} گھٹتی اور \عددی{a} بڑھتا ہے۔ ہمیں ابتدائی اسراع درکار ہے لہٰذا ہم ہوائی بان کی ابتدائی کمیت  \عددی{M_i} لیں گے۔

حساب:\quad
ان معلومات سے ذیل حاصل ہو گا۔
\begin{align*}
a=\frac{T}{M}=\frac{\SI{6440}{\newton}}{\SI{850}{\kilo\gram}}=\SI{7.6}{\meter\per\second\squared}\quad \quad \text{\RL{(جواب)}}
\end{align*}
سطح زمین سے  سیدھا اوپر اڑان کے لئے ضروری ہے کہ ابتدائی اسراع \عددی{g=\SI{9.8}{\meter\per\second\squared}} سے  زیادہ  ہو۔ یعنی، ابتدائی اسراع کو سطح زمین پر تجاذبی اسراع سے زیادہ ہونا ہو گا۔دوسرے لفظوں میں،  ہوائی  بان پر  ابتدائی تجاذبی قوت  ، جس کی قدر \عددی{M_ig} ہے
\begin{align*}
(\SI{850}{\kilo\gram})(\SI{9.8}{\meter\per\second\squared})=\SI{8330}{\newton}
\end{align*}
  سے   قوت دھکیل \عددی{T} کا زیادہ ہونا لازمی ہے، ورنہ  ہوائی بان زمین سے اٹھنے کے قابل نہیں ہو گا۔ چونکہ اس ہوائی بان کی قوت دھکیل  (جو یہاں \عددی{T=\SI{6440}{\newton}} ہے) درکار قدر سے کم ہے لہٰذا یہ ہوائی بان اڑ نہیں پائے گا؛یہاں زیادہ طاقتور ہوائی بان کی ضرورت ہے۔
\انتہا{نمونی سوال}
%---------------------------------

\حصہء{نظر ثانی اور خلاصہ}
\جزوحصہء{مرکز کمیت}
ایک نظام جو \عددی{n} ذرات پر مشتمل ہو کے مرکز کمیت  کی تعریف  وہ نقطہ ہے جس کے محدد درج ذیل ہوں۔
\begin{align*}
x_{\text{\RL{مرکزکمیت}}}&=\frac{1}{M}\sum_{i=1}^{n} m_ix_i\\
y_{\text{\RL{مرکزکمیت}}}&=\frac{1}{M}\sum_{i=1}^{n} m_iy_i\\
z_{\text{\RL{مرکزکمیت}}}&=\frac{1}{M}\sum_{i=1}^{n} m_i z_i \tag{\arabicdigits{\setlatin9.5}}
\end{align*}
اس کو مختصراً ذیل لکھا جا سکتا ہے، جہاں \عددی{M} نظام کی کل کمیت  \عددی{\sum_{i=1}^{n} m_i} ہے۔
\begin{align*}
\vec{r}_{\text{\RL{مرکزکمیت}}}=\frac{1}{M}\sum_{i=1}^{n} m_i\vec{r}_i\tag{\arabicdigits{\setlatin9.8}}
\end{align*}

\جزوحصہء{نیوٹن کا دوسرا قانون برائے ذرات کا نظام}
ایک نظام ، جو ذرات پر مشتمل ہو، کے مرکز کمیت کی حرکت \موٹا{ نیوٹن کے دوسرے قانون برائے ذرات پر مشتمل نظام }کے تحت ہو گی، جو ذیل کہتا ہے۔
\begin{align*}
\vec{F}_{\text{\RL{صافی}}}=M\vec{a}_{\text{\RL{مرکزکمیت}}}\tag{\arabicdigits{\setlatin9.14}}
\end{align*}
یہاں نظام پر لاگو   تمام \ترچھا{بیرونی }قوتیں مل کر صافی قوت \عددی{\vec{F}_{\text{\RL{صافی}}}} دیتی ہیں۔ نظام کی کل کمیت \عددی{M}، اور  نظام کے مرکز کمیت 
کی اسراع \عددی{\vec{a}_{\text{\RL{مرکزکمیت}}}} ہے۔

\جزوحصہء{خطی معیار حرکت اور نیوٹن کا دوسرا قانون}
تنہا ذرے کے لئے،   مقدار  \عددی{\vec{p}} متعارف کر کے ،  جو  اس ذرے کا \موٹا{ خطی  معیار حرکت } کہلاتا ہے اور جس کی تعریف ذیل ہے،
\begin{align*}
\vec{p}=m\vec{v}\tag{\arabicdigits{\setlatin9.22}}
\end{align*}
ہم نیوٹن کا دوسرا قانون اس معیار حرکت کی صورت میں  لکھ سکتے ہیں۔
\begin{align*}
\vec{F}_{\text{\RL{صافی}}}=\frac{\dif \vec{p}}{\dif t}\tag{\arabicdigits{\setlatin9.23}}
\end{align*}
ذرات پر مشتمل نظام کے لئے   مذکورہ بالا دو تعلق   ذیل  لکھا جائیں گے۔
\begin{align*}
 \vec{F}_{\text{\RL{صافی}}}=\frac{\dif \vec{P}}{\dif t} \quad \text{\RL{اور}}\quad  \vec{P}=M\vec{v}_{\text{\RL{مرکزکمیت}}}
 \tag{\arabicdigits{\setlatin9.25،\,9.27}}
\end{align*}

\جزوحصہء{تصادم اور ضرب}
تصادم میں ملوث ذرہ نما جسم پر معیار حرکت کے روپ میں نیوٹن کے دوسرے قانون کا اطلاق  \موٹا{ضرب و خطی معیار حرکت کا  مسئلہ }دیگا:
\begin{align*}
\vec{p}_f-\vec{p}_i=\Delta \vec{p}=\vec{J} \tag{\arabicdigits{\setlatin9.31،\,9.32}}
\end{align*}
جہاں  جسم کے خطی معیار حرکت میں تبدیلی \عددی{\vec{p}_f-\vec{p}_i=\Delta \vec{p}} ہے ، اور\موٹا{ ضرب}     \عددی{\vec{J}} وہ  قوت  \عددی{\vec{F}(t)} ہے جو تصادم کے دوران دوسرا جسم اس (پہلے جسم)  پر لاگو کرتا ہے۔ 
\begin{align*}
\vec{J}=\int_{t_i}^{t_f} \vec{F}(t)\dif t   \tag{\arabicdigits{\setlatin9.30}}
\end{align*}
اگر تصادم کا دورانیہ \عددی{\Delta t} اور اس دوران  \عددی{\vec{F}(t)}  کی اوسط  قیمت   \عددی{F_{\text{\RL{اوسط}}}} ہو تب یک بُعدی حرکت کے لئے ذیل ہو گا۔
\begin{align*}
J=F_{\text{\RL{اوسط}}}\Delta t     \tag{\arabicdigits{\setlatin9.35}}
\end{align*}
 ساکن  جسم  پر  کمیت \عددی{m}   کے ذرے،  جن کی    رفتار \عددی{v}  ہے ، برس کر   ذیل اوسط قوت   پیدا کرتے ہیں
\begin{align*}
F_{\text{\RL{اوسط}}}=-\frac{n}{\Delta t}\Delta p=-\frac{n}{\Delta t}m\Delta v  \tag{\arabicdigits{\setlatin9.37}}
\end{align*}
جہاں ساکن جسم سے ذروں کے  تصادم کی شرح \عددی{n\!/\!\Delta t} ، اور ہر ایک ذرے کی رفتار میں تبدیلی \عددی{\Delta v} ہے   ( جسم ساکن رہتا ہے)۔ یہ اوسط قوت ذیل بھی لکھی جا سکتی ہے
\begin{align*}
F_{\text{\RL{اوسط}}}=-\frac{\Delta M}{\Delta t}\Delta v    \tag{\arabicdigits{\setlatin9.40}}
\end{align*}
جہاں \عددی{\Delta M\!/\!\Delta t} وہ شرح ہے جس سے  کمیت ساکن جسم سے ٹکراتی ہے۔درج بالا دو مساوات میں اگر ذرے تصادم کے بعد رک جاتے ہوں تب \عددی{\Delta v=-v} ہو گا، اور اگر  ذرے  جسم پر ٹپکی  کھا کر  رفتار میں تبدیلی کے بغیر واپس لوٹیں  تب  \عددی{\Delta v=-2v} ہو گا۔

\جزوحصہء{خطی معیار حرکت کی بقا}
  جدا    نظام   پر  بیرونی قوت عمل نہیں کرتی، لہٰذا  اس نظام کا خطی معیار حرکت  تبدیل نہیں ہو گا۔
  \begin{align*}
  \vec{P}=\text{\RL{مستقل}}\quad \quad \text{\RL{(بند، جدا نظام)}}   \tag{\arabicdigits{\setlatin9.42}}
  \end{align*}
  اس کو ذیل بھی لکھ سکتے ہیں جہاں زیر نوشت کسی ابتدائی لمحہ اور   اختتامی لمحہ کو ظاہر کرتی ہیں۔
    \begin{align*}
  \vec{P}_i=\vec{P}_f\quad \quad \text{\RL{(بند، جدا نظام)}}   \tag{\arabicdigits{\setlatin9.43}}
  \end{align*}
  مذکورہ بالا دونوں مساوات \موٹا{خطی معیار حرکت کی بقا}   کو بیان کرتی ہیں۔
  
  \جزوحصہء{ایک بُعد میں غیر لچکی تصادم}
دو اجسام کی\ترچھا{ غیر لچکی } تصادم میں دو جسمی نظام کی حرکی توانائی کی بقا نہیں ہو گی (حرکی توانائی  مستقل نہیں ہو گی)۔ اگر نظام بند اور جدا ہو ، نظام کے کل خطی معیار حرکت کی بقا لازماً  ہو گی (یہ مستقل  ہو گا)، جس کو سمتیہ روپ میں ذیل لکھا جا سکتا ہے، جہاں زیر نوشت \عددی{i} اور \عددی{j} بالترتیب تصادم سے عین  قبل اور اس کے عین  بعد لمحات ظاہر کرتی ہیں۔
\begin{align*}
\vec{p}_{1i}+\vec{p}_{2i}=\vec{p}_{1f}+\vec{p}_{2f}   \tag{\arabicdigits{\setlatin9.50}}
\end{align*}

ذروں  کی حرکت ایک محور پر ہونے کی صورت میں تصادم یک بُعدی ہو گا اور ہم مذکورہ بالا مساوات کو  محور کے ہمراہ  سمتی رفتار اجزاء کی صورت میں  ذیل لکھ سکتے ہیں۔
\begin{align*}
m_1v_{1i}+m_2v_{2i}=m_1v_{1f}+m_2v_{2f}    \tag{\arabicdigits{\setlatin9.51}}
\end{align*}

اگر  دو جسم آپس میں چپک جائیں، تصادم \ترچھا{مکمل غیر لچکی }ہو گا اور  دونوں  اجسام کی اختتامی سمتی رفتار \عددی{V} ہو گی (کیونکہ یہ آپس میں جڑے ہیں)۔

\جزوحصہء{مرکز کمیت کی حرکت}
دو متصادم  اجسام  کے  بند،  جدا نظام  کے   مرکز کمیت  پر تصادم  اثر  انداز نہیں ہو گا۔ بالخصوص، مرکز کمیت کی سمتی رفتار \عددی{\vec{v}_{\text{\RL{مرکزکمیت}}}} کو  تصادم  تبدیل نہیں کرتا۔

\جزوحصہء{ایک بُعد میں لچکی تصادم}
\ترچھا{لچکی تصادم}  ایک خاص قسم کا تصادم ہے جس میں متصادم اجسام کے نظام کی حرکی توانائی  برقرار رہتی ہے۔اگر نظام بند اور جدا بھی ہو، اس کا خطی معیار حرکت بھی برقرار رہے گا۔یک بُعدی تصادم کے لئے، جس میں جسم \عددی{2} ہدف اور جسم \عددی{1} گولا ہے، حرکی توانائی اور خطی معیار حرکت کی بقا، تصادم کے  عین بعد   سمتی رفتاروں کے لئے درج ذیل مساوات دیتی ہیں۔
\begin{align*}
v_{1f}&=\frac{m_1-m_2}{m_1+m_2}v_{1i}    \tag{\arabicdigits{\setlatin9.67}}\\
v_{2f}&=\frac{2m_1}{m_1+m_2}v_{1i}      \tag{\arabicdigits{\setlatin9.68}}
\end{align*}
\جزوحصہء{دو ابعاد میں تصادم}
اگر دو جسم یوں ٹکرائیں کہ   ان  کی حرکت ایک  ہی محور پر نہ ہو (ٹکر آمنے سامنے سے نہیں)، تصادم دو بُعدی ہو گا۔اگر دو جسمی نظام بند اور جدا ہو، معیار حرکت کی بقا کے قانون  کا اطلاق تصادم پر ہو گا جو ذیل لکھا جائے گا۔
\begin{align*}
\vec{P}_{1i}+\vec{P}_{2i}=\vec{P}_{1f}+\vec{P}_{2f}       \tag{\arabicdigits{\setlatin9.77}}
\end{align*}
اجزاء کے روپ میں یہ قانون دو مساوات دے گا جو تصادم کو بیان کریں گی   (دو ابعاد میں ہر بُعد کے لئے ایک مساوات) ۔ اگر تصادم لچکی بھی ہو (خصوصی صورت)، تصادم کے دوران حرکی توانائی کی بقا تیسری مساوات دیگی۔
\begin{align*}
K_{1i}+K_{2i}=K_{1f}+K_{2f}         \tag{\arabicdigits{\setlatin9.78}}
\end{align*}

\جزوحصہء{متغیر کمیتی نظام}
بیرونی قوتوں کی عدم موجودگی میں ہوائی بان ذیل لمحاتی شرح سے اسراع پذیر ہو گا
\begin{align*}
Rv_{\text{\RL{اضافی}}}=Ma\quad\quad\text{\RL{(ہوائی بان کی پہلی مساوات )}}       \tag{\arabicdigits{\setlatin9.87}}
\end{align*}
جہاں \عددی{M} ہوائی بان کی لمحاتی کمیت  (جس میں غیر استعمال شدہ   ایندھن شامل ہے)، \عددی{R} ایندھن کے اصراف کی شرح، اور \عددی{v_{\text{\RL{اضافی}}}}  ہوائی بان کے لحاظ سے اخراج کی اضافی   رفتار ہے۔جزو \عددی{Rv_{\text{\RL{اضافی}}}}  ہوائی بان  کی انجن کی \موٹا{قوت دھکیل }ہے۔  جب  ایک ہوائی بان  کی،جس کی \عددی{R} اور \عددی{v_{\text{\RL{اضافی}}}}   اٹل ہو،    کمیت \عددی{M_i} سے \عددی{M_f} ہونے پر اس کی رفتار \عددی{v_i} سے \عددی{v_f} ہو،   درج ذیل ہو گا۔
\begin{align*}
v_f-v_i=v_{\text{\RL{اضافی}}}\ln\frac{M_i}{M_f}\quad \quad \text{\RL{(ہوائی بان کی دوسری مساوات)}}       \tag{\arabicdigits{\setlatin9.88}}
\end{align*}

%=======================================
%p245
\حصہء{سوالات}
%Q1
\ابتدا{سوال}
تین ذرات جن پر بیرونی قوتیں عمل کرتی ہیں  کا فضائی جائزہ  شکل \حوالہء{9۔23}  میں  پیش  ہے۔ دو ذروں پر قوتوں کی قدریں اور سمتیں دی گئی ہیں۔ تین ذروی نظام  کا مرکز کمیت (ا) ساکن، (ب) دائیں رخ مستقل  سمتی رفتار سے، اور (ج) اوپر وار اسراع پذیر ہونے کی صورت میں تیسری قوت کی قدر اور سمت تلاش کریں۔
\انتہا{سوال}
%-----------------------
\ابتدا{سوال}
بلا رگڑ  مستوی  پر مستقل سمتی   رفتاروں سے  حرکت کرتے  ہوئے ایک برابر کمیت کے چار ذروں کا فضائی جائزہ    شکل \حوالہء{9.24} میں پیش ہے۔ سمتی رفتاروں کے رخ دیے گئے ہیں؛ ان کی قدریں برابر ہیں۔ ذروں کی جوڑیاں بنائیں۔ کون  سی جوڑی  ایسا نظام دیتی ہے جس کا مرکز کمیت (ساکن ہے، (ب)  ساکن ہے اور مبدا پر ہے، اور (ج) مبدا سے گزرتا ہے؟
\انتہا{سوال}
%----------------------
%Q3
\ابتدا{سوال}
فرض کریں ایک ڈبہ،   جو \عددی{x} محور پر  مستقل   مثبت سمتی رفتار سے حرکت میں ہو، دھماکے سے دو  ٹکڑوں  میں تقسیم ہوتا ہے۔  ایک  ٹکڑا ، جس کی کمیت \عددی{m_1} ہے ،  مثبت سمتی رفتار \عددی{\vec{v}_1}  سے حرکت کرتا ہے۔ دوسرا ٹکڑا جس کی کمیت \عددی{m_2} ہے  (ا)  مثبت سمتی رفتار  \عددی{\vec{v}_2}  (شکل \حوالہء{9.25a})، (ب) منفی سمتی رفتار \عددی{\vec{v}_2} (شکل \حوالہء{9.25b})، یا (ج)  صفر سمتی رفتار  (شکل \حوالہء{ 9.25c}) رکھ سکتا ہے۔  ان ممکن نتائج کی درجہ بندی  مطابقتی \عددی{\vec{v}_1} کی قدر  کے لحاظ سے ،اعظم اول  رکھ کر، کریں۔ 
\انتہا{سوال}
%----------------------
\ابتدا{سوال}
تصادم میں ملوث  جسم کے لئے  قوت کی قدر بالمقابل وقت کی ترسیمات شکل \حوالہء{9.26} میں پیش ہیں۔ ترسیمات کی درجہ بندی  جسم پر قوت دھکیل کی قدر  کے لحاظ سے، اعظم اول رکھ کر،  کریں۔
\انتہا{سوال}
%--------------------------
%Q5
\ابتدا{سوال}
بلا رگڑ مستوی پر حرکت کرتے    تین ڈبوں پر عمل پیرا قوت  کا  فضائی نظارہ شکل \حوالہء{9.27} میں پیش ہے۔ ہر ایک ڈبہ کے لئے ،  کیا   محور \عددی{x} اور محور \عددی{y} کے ہمراہ خطی معیار حرکت کی بقا ہو گی؟
\انتہا{سوال}
%------------------------------
\ابتدا{سوال}
تین یا  چار یکساں ذروں کا گروہ ، جو محور \عددی{x} یا محور \عددی{y} کے متوازی  ایک رفتار سے حرکت کرتے ہوں، شکل \حوالہء{9.28} میں دکھایا گیا ہے۔ مرکز کمیت کی رفتار کے لحاظ سے ان کی درجہ بندی، اعظم اول رکھ کر،  کریں۔
\انتہا{سوال}
%----------------------------
%Q7
\ابتدا{سوال}
ایک سل بلا رگڑ فرش  پر حرکت کر کے اس جتنی کمیت کی دوسری سل سے  ٹکراتی ہے۔ شکل \حوالہء{9.29} میں سلوں کی حرکی توانائی \عددی{K}  کی چار  ممکنہ ترسیم  پیش ہیں۔ (ا) ان میں سے کون سی طبیعی  وجوہات کی بنا پر  ممکن نہیں؟ باقی میں سے  کونسی (ب) لچکی تصادم اور (ج) غیر لچکی تصادم بہتر ظاہر کرتی ہے؟
\انتہا{سوال}
%--------------------------
\ابتدا{سوال}
بلا رگڑ فرش پر محور \عددی{x} کے ہمراہ   سل  \عددی{1}   ساکن  سل \عددی{2} کی طرف بڑھتا ہے۔عین لچکی تصادم سے  قبل لمحہ پر   ان کی تصویر کشی شکل \حوالہء{9.30} میں  کی گئی  ہے۔ اس لمحہ پر دو  سل نظام کے مرکز کمیت کے تین ممکن مقام بھی پیش  ہیں۔ (نقطہ \عددی{B}     سلوں کے مراکز کے  درمیان  نصف فاصلے پر ہے۔)  اگر  تصادم کے بعد نظام کا مرکز کمیت  (ا) \عددی{A} پر، (ب) \عددی{B} پر، اور (ج) \عددی{C} پر ہو، کیا سل \عددی{1} ساکن  ہو گا؟ آگے  کی طرف گامزن ہو گا؟     پیچھے کی طرف گامزن ہو گا؟
\انتہا{سوال}
%--------------------------
%Q9
\ابتدا{سوال}
دو  اجسام   محور \عددی{x}  کے ہمراہ یک بُعدی  لچکی تصادم  کا شکار ہوتے ہیں۔ شکل \حوالہء{9.31} میں  اجسام  اور     مرکز کمیت کے  مقام بالمقابل وقت  ترسیمات پیش ہیں۔ (ا) کیا دونوں جسم ابتدائی طور پر حرکت میں تھی، یا ان میں سے ایک ساکن تھا؟ کونسا لکیری قطع (ب)  تصادم سے قبل اور (ج) تصادم کے بعد  مرکز کمیت دیتا ہے؟ (د)  کیا تصادم سے قبل زیادہ تیز  حرکت کرتے جسم کی کمیت دوسرے جسم  کی کمیت سے زیادہ ہے، کم ہے، یا اس کے برابر ہے؟
\انتہا{سوال}
%--------------------------------------
\ابتدا{سوال}
افقی فرش پر  سل ابتدائی طور ساکن ، محور \عددی{x} کے ہمراہ مثبت رخ ، یا  محور کے منفی رخ حرکت میں ہے۔ سل دھماکے سے دو ٹکڑوں میں تقسیم ہوتا ہے جو اسی محور پر حرکت کرتے ہیں۔ فرض کریں سل اور اس کے دو ٹکڑے ایک بند اور جدا  نظام دیتے ہیں۔سل اور   ٹکڑوں کے معیار حرکت بالمقابل وقت \عددی{t}   کی  چھ ترسیمات شکل \حوالہء{9.32} میں پیش ہیں۔ کونسی ترسیمات طبیعی نا ممکن ہیں؟ وجوہات پیش کریں۔
\انتہا{سوال}
%----------------------------
\ابتدا{سوال}
محور \عددی{x} پر کمیت \عددی{m_1} کا سل بلا رگڑ فرش پر  چلتا ہوا کمیت \عددی{m_2} کے ساکن سل سے لچکی   متصادم ہوتا  ہے۔ شکل \حوالہء{9.33}  میں  سل \عددی{1} کا مقام \عددی{x}  بالمقابل وقت \عددی{t}  ٹھوس لکیر سے پیش  کیا گیا ہے ، جس پر  نقطہ    تصادم \عددی{x_c} اور وقت تصادم \عددی{t_c}    کی نشاندہی کی گئی ہے۔ اگر (ا) \عددی{m_1<m_2}   اور (ب) \عددی{m_1>m_2} ہو، تصادم کے بعد  سل \عددی{1} نقطہ دار راہ \عددی{A}، \عددی{B}، \عددی{C} ، اور \عددی{D} میں کس      پر   گامزن ہو گا؟ (ج)  اگر \عددی{m_1=m_2} ہو یہ راہ   \عددی{1}، \عددی{2}، \عددی{3}، \عددی{4}، اور  \عددی{4} میں کس  پر گامزن ہو گا؟
\انتہا{سوال}
%-----------------------------
%Q12
\ابتدا{سوال}
دو  جسم  اور ان کے مرکز کمیت کی  مقام بالمقابل وقت  کی  چار ترسیمات پیش ہیں۔ یہ جسم بند اور جدا نظام دیتے ہیں اور محور \عددی{x} پر چلتے ہوئے یک بُعدی  مکمل غیر لچکی تصادم کا شکار ہوتے ہیں۔ کیا ترسیم \عددی{1} میں (ا) دو جسم اور (ب) مرکز کمیت محور \عددی{x} پر مثبت رخ یا منفی رخ حرکت  کرتے ہیں؟ (ج) کونسی ترسیم طبیعی ناممکن ہے؟ وجہ پیش کریں۔
\انتہا{سوال}
%--------------------------------
% p246, module 9-1   Center of Mass
\جزوحصہء{مرکز کمیت}
%---------------------------------
%Q1, p246
\ابتدا{سوال}
کمیت \عددی{\SI{2.00}{\kilo\gram}}  ذرے کا \عددی{xy} محدد \عددی{(\SI{-1.20}{\meter},\SI{0.500}{\meter})}، اور کمیت \عددی{\SI{4.00}{\kilo\gram}} ذرے کا \عددی{xy} محدد  \عددی{(\SI{0.600}{\meter},\SI{-0.750}{\meter})} ہے۔ دونوں افقی مستوی پر پائے جاتے ہیں۔ کمیت \عددی{\SI{3.00}{\kilo\gram}} کا تیسرا ذرہ  کس (ا) \عددی{x} اور (ب) \عددی{y} پر رکھ کر تین ذروی نظام کا مرکز کمیت \عددی{(\SI{-0.500}{\meter},\SI{-0.700}{\meter})} پر ہو گا؟
\انتہا{سوال}
%----------------------------------------
\ابتدا{سوال}
تین ذروی نظام جس میں \عددی{m_1=\SI{3.0}{\kilo\gram}}، \عددی{m_2=\SI{4.0}{\kilo\gram}}، اور \عددی{m_3=\SI{8.0}{\kilo\gram}} ہے شکل \حوالہء{9.35} میں پیش ہے۔ محور کے  پیما \عددی{x_s=\SI{2.0}{\meter}} اور \عددی{y_s=\SI{2.0}{\meter}}کے لحاظ سے رکھے گئے ہیں۔ نظام کے مرکز کمیت کا (ا) \عددی{x} محدد اور (ب) \عددی{y} محدد کیا ہو گا؟ (ج)  کیا \عددی{m_3} بتدریج بڑھانے سے مرکز کمیت  اس ذرے کی جانب منتقل ہو گا، اس سے دور منتقل ہو گا ، یا ساکن رہے گا؟
\انتہا{سوال}
%--------------------------
\ابتدا{سوال}
ایک سل جس  کے اضلاع \عددی{d_1=\SI{11.0}{\centi\meter}}، \عددی{d_2=\SI{2.80}{\centi\meter}}، اور \عددی{d_3=\SI{13.0}{\centi\meter}} ہیں شکل \حوالہء{9.36} میں دکھایا گیا ہے۔ اس کا نصف حصہ  المونیم    (کثافت \عددی{\SI{2.70}{\gram\per\centi\meter\cubed}} ) اور آدھا لوہے (کثافت \عددی{\SI{7.85}{\gram\per\centi\meter\cubed}}) کا  ہے۔  سل کے مرکز کمیت کا (ا) \عددی{x} محدد، (ب) \عددی{y} محدد، اور (ج) \عددی{z} محدد کیا ہو گا؟
\انتہا{سوال}
%---------------------------
%Q4 p 247
\ابتدا{سوال}
تین  یکساں پیکر   ڈنڈیاں جن میں ہر ایک کی لمبائی \عددی{L=\SI{22}{\centi\meter}}  ہے  مل کر  الٹ نون غُنّہ بناتی ہیں (شکل \حوالہء{9.37})۔ انتصابی ڈنڈی  کی کمیت \عددی{\SI{14}{\gram}}  اور افقی ڈنڈی کی کمیت \عددی{\SI{42}{\gram}} ہے۔ نظام کے مرکز کمیت کا  (ا)  \عددی{x} محدد اور (ب) \عددی{y} محدد کیا ہو گا؟
\انتہا{سوال}
%---------------------------
\ابتدا{سوال}
یکساں موٹائی کا  چادر شکل \حوالہء{9.38} میں پیش ہے۔ اگر \عددی{L=\SI{5.0}{\centi\meter}} ہو چادر کے مرکز کمیت  کا (ا) \عددی{x} محدد اور (ب) \عددی{y} محدد کیا ہو گا؟
\انتہا{سوال}
%--------------------------------
\ابتدا{سوال}
قابل نظر انداز موٹائی کی  یکساں  دھاتی چادر  سے بنایا گیا مکعب  شکل \حوالہء{9.39} میں پیش ہے۔ مکعب اوپر سے کھلا ہے اور اس کا کنارہ \عددی{L=\SI{40}{\centi\meter}} لمبا  ہے۔ مکعب کے مرکز کمیت  کا  (ا) \عددی{x} محدد، (ب) \عددی{y} محدد، اور (ج) \عددی{z} محدد تلاش کریں۔
\انتہا{سوال}
%------------------------------
%Q7
\ابتدا{سوال}
ایمونیا سالمہ   \عددی{(\ce{NH3})} ، جس میں ہائیڈروجن جوہر     \عددی{(\ce{H})} متساوی الاضلاع  مثلث  بناتے ہیں ، شکل \حوالہء{9.40} میں پیش ہے۔ مثلث کا مرکز ہر   \عددی{\ce{H}} جوہر سے \عددی{d=\SI{9.40e-11}{\meter}}  فاصلے پر ہے۔ نائیٹروجن جوہر \عددی{\ce{N}} اس ہرم کی چوٹی پر واقع ہے جس کا   تل تین  \عددی{\ce{H}} جوہر بناتے ہیں۔ نائیٹروجن  اور  ہائیڈروجن  کی جوہری کمیت نسبت  \عددی{13.9} ، اور نائیٹروجن تا ہائیڈروجن فاصلہ \عددی{L=\SI{10.14e-11}{\meter}} ہے۔ سالمہ کے مرکز کمیت کا (ا) \عددی{x} محدد اور (ب) \عددی{y} محدد کیا ہو گا؟
\انتہا{سوال}
%---------------------------------
\ابتدا{سوال}
 یکساں پیکر کی   بوتل  جس کی کمیت \عددی{\SI{0.140}{\kilo\gram}} اور  لمبائی \عددی{\SI{12.0}{\centi\meter}}  ہے، میں  \عددی{\SI{0.354}{\kilo\gram}} مشروب بھری  ہے (شکل \حوالہء{9.41})۔ بوتل کے سر اور تل میں  ، مشروب  خارج کرنے کی غرض سے، باریک سوراخ (جو بوتل کی کمیت پر اثر انداز نہیں ہوتے)  کیے جاتے ہیں۔ (ا)  مکمل بھری بوتل (بمع مشروب) کے مرکز  کمیت   کی  اور  (ب) مکمل خالی بوتل کے مرکز کمیت کی بلندی \عددی{h}  کیا ہو گی؟ (ج)  جیسے  جیسے مشروب  خارج ہوتا ہے، \عددی{h} کو کیا ہو گا؟ (د)  مرکز کمیت کے  لمحاتی بلندی کو \عددی{x} کہہ کر اس کی کمتر قیمت تلاش کریں۔
\انتہا{سوال}
%-----------------------------
%Q9 p247, Module 9-2, Newton's Second Law for a System of Particles
\جزوحصہء{نیوٹن کا دوسرا قاعدہ برائے ذرات کا  نظام }
%-----------------------------------------------
\ابتدا{سوال}
ایک پتھر \عددی{t=0} پر گرنے دیا جاتا ہے۔ دوسرا پتھر جس کی کمیت دگنی ہے، اسی بلندی سے، \عددی{t=\SI{100}{\milli\second}} پر گرنے دیا جاتا ہے۔ (ا) نقطہ رہائی سے ، \عددی{t=\SI{300}{\milli\second}}  پر،   دو پتھر نظام کا مرکز کمیت  کتنا نیچے ہو گا؟ (دونوں پتھر اس لمحے تک ہوا میں ہیں۔) (ب)  اس لمحے پر دو پتھر نظام کا مرکز کمیت کس رفتار سے حرکت کرتا ہے؟
\انتہا{سوال}
%---------------------------------
\ابتدا{سوال}
چوراہا بتی پر \عددی{\SI{1000}{\kilo\gram}} کمیت کی  گاڑی  کھڑی ہے۔ جیسے ہی بتی سبز ہوتی ہے گاڑی \عددی{\SI{4.0}{\meter\per\second\squared}}  مستقل اسراع سے  حرکت میں آتی ہے۔ عین اسی لمحے ایک ٹرک جس کی کمیت \عددی{\SI{2000}{\kilo\gram}} اور جو \عددی{\SI{8.0}{\meter\per\second}} رفتار سے چل رہا ہے گاڑی سے آگے نکلتا ہے۔ (ا) گاڑی و ٹرک نظام کا مرکز کمیت \عددی{t=\SI{3.0}{\second}} بعد  بتی سے کتنا دور اور (ب) اس کی  رفتار  کیا ہو گی؟
\انتہا{سوال}
%------------------------
\ابتدا{سوال}
زیتون  کا ایک بڑا پھل     \عددی{(m=\SI{0.50}{\kilo\gram})}  \عددی{xy} محددی نظام کے  مرکز پر، اور  جوز برازیل   \عددی{M=\SI{1.5}{\kilo\gram}} نقطہ
 \عددی{(\SI{1.0}{\meter},\SI{2.0}{\meter})}  پر پڑا ہے۔لمحہ \عددی{t=0} پر قوت \عددی{\vec{F}_z=(2.0\hat{i}+3.0\hat{j})\,\si{\newton}} زیتون کے پھل  پر اور \عددی{\vec{F}_j=(-3.0\hat{i}-2.0\hat{j})\,\si{\newton}} جوز برازیل  پر  عمل کرنا  شروع کرتی ہیں۔ لمحہ \عددی{t=0} کے لحاظ سے \عددی{t=\SI{4.0}{\second}} پر  زیتون و جوز نظام کے  مرکز کمیت  کا  ہٹاو  اکائی سمتی ترقیم میں کیا ہو گا؟
\انتہا{سوال}
%---------------------------
\ابتدا{سوال}
دو  پھسلن باز ، جن میں سے ایک کی کمیت \عددی{\SI{65}{\kilo\gram}} اور دوسرے کی \عددی{\SI{40}{\kilo\gram}} ہے، \عددی{\SI{10}{\meter}} لمبا ڈنڈا ، جس کی کمیت قابل نظر انداز ہے  ، تھامے  برف پر  کھڑے ہیں۔ ڈنڈے کے  سروں سے آغاز کرتے ہوئے پھسلن باز ڈنڈا کھینچ کر ، ڈنڈے کے ہمراہ حرکت  کرتے ہوئے  قریب آ کر ، ملتے ہیں۔  کم کمیتی شخص کتنا فاصلہ طے کرتا ہے؟
\انتہا{سوال}
%-----------------------------
%Q13
\ابتدا{سوال}
ایک گولا  \عددی{\SI{20}{\meter\per\second}} کی ابتدائی   سمتی رفتار   \عددی{\vec{v}_0} کے ساتھ افق سے  \عددی{\theta_0=\SI{60}{\degree}} زاویہ  اوپر پھینکا جاتا ہے۔ خط حرکت کے   بلند تر نقطہ پر  گولا دھماکے سے دو برابر  ٹکڑوں میں تقسیم ہوتا ہے (شکل \حوالہء{9-42})۔ ایک ٹکڑا جس کا رفتار دھماکے کے عین بعد صفر ہے سیدھا نیچے گرتا ہے۔ دوسرا ٹکڑا  توپ سے  کتنے فاصلے پر گرتا ہے؟ (ہوائی رگڑ نظر انداز کریں اور زمین  ہموار تصور کریں۔)
\انتہا{سوال}
%--------------------------
\ابتدا{سوال}
وقت \عددی{t=0} پر دو ذرے محددی نظام کے مبدا سے  پھینکے  جاتے ہیں (شکل \حوالہء{9-43})۔ ذرہ \عددی{1} جس کی کمیت \عددی{m_1=\SI{5.00}{\gram}} ہے بلا رگڑ  افقی زمین  پر محور \عددی{x} کے ہمراہ \عددی{\SI{10.0}{\meter\per\second}} رفتار سے   روانا  کیا جاتا ہے۔  ذرہ \عددی{2} جس کی کمیت \عددی{m_2=\SI{3.00}{\gram}} ہے \عددی{\SI{20}{\meter\per\second}} سے  اوپری زاویے پر یوں پھینکا جاتا ہے کہ  یہ ہر لمحہ ذرہ \عددی{1}کے ٹھیک  اوپر  رہتا ہے۔ (ا) دو ذروی نظام کا مرکز کمیت  کتنی زیادہ سے زیادہ بلندی \عددی{H_{\text{\RL{بلندتر}}}} کو پہنچتا ہے؟ اکائی سمتی ترقیم میں  مرکز کمیت کی (ب) سمتی رفتار اور (ج) اسراع اس لمحے کیا ہو گی جب مرکز کمیت  \عددی{H_{\text{\RL{بلندتر}}}} پر  ہو؟
\انتہا{سوال}
%----------------------------------
%Q15 p248
\ابتدا{سوال}
ایک ریڑھی جو ہوائی ڈگر پر چلتی ہے    رسی کے ذریعہ اینٹ سے منسلک  ہے جو لٹک رہی ہے (شکل \حوالہء{9.44})۔ ریڑھی کی کمیت \عددی{m_1=\SI{0.600}{\kilo\gram}} اور اس کا مرکز کمیت ابتدائی طور پر \عددی{(\SI{-0.500}{\meter},\SI{0}{\meter})}   محدد  پر ہے۔ اینٹ کی کمیت \عددی{m_2=\SI{0.400}{\kilo\gram}}  اور اس کا مرکز کمیت ابتدائی طور پر \عددی{(\SI{0}{\meter},\SI{-0.100}{\meter})} محدد   پر ہے۔ رسی اور چرخی کی کمیت قابل نظر انداز ہے۔ریڑھی ساکن حالت سے رہا کی جاتی ہے۔ ریڑھی اور اینٹ حرکت کرتی  ہیں  حتٰی کہ ریڑھی چرخی سے ٹکراتی ہے۔ ریڑھی اور  ہوائی ڈگر کے بیچ رگڑ  ، اور  چرخی اور دھرے  کے  بیچ  رگڑ قابل نظر انداز ہے۔ (ا) ریڑھی و اینٹ نظام کے مرکز کمیت کی اسراع اکائی سمتی ترقیم میں کیا  ہو   گی؟  (ب)  مرکز کمیت کی سمتی رفتار بطور وقت \عددی{t} کا تفاعل  کیا ہو گی؟ (ج) مرکز کمیت کی راہ ترسیم کریں۔ (د)  اگر راہ قوسی ہو، کیا یہ دائیں اوپر  جانب یا بائیں  نیچے جانب  ابھرتی ہے، اور اگر راہ سیدھی ہو، \عددی{x} محور اور راہ کے بیچ زاویہ کیا ہو گا؟
\انتہا{سوال}
%--------------------------
\ابتدا{سوال}
زریاب  جس کی کمیت \عددی{\SI{80}{\kilo\gram}} ہے اور اسد جو ہلکا ہے \عددی{\SI{30}{\kilo\gram}} ساکن  کشتی میں بیٹھ  (ناران میں) کر سیف  الملوک  جھیل کا نظارہ  کر رہے ہیں۔ ان   کی  نشستیں  \عددی{\SI{3.0}{\meter}} فاصلے پر، اور کشتی کے مرکز کمیت کے لحاظ سے متشاکلی واقع ہیں۔ دونوں  آپس میں نشست  تبدیل کرتے ہیں۔ اگر کشتی کا مرکز کمیت  گھاٹ کے لحاظ سے \عددی{\SI{40}{\centi\meter}} افقی حرکت کرے، اسد کی کمیت کیا ہو گی؟
\انتہا{سوال}
%------------------------
\ابتدا{سوال}
کنارے سے \عددی{D=\SI{6.1}{\meter}} فاصلے پر \عددی{\SI{4.5}{\kilo\gram}} کتّا \عددی{\SI{18}{\kilo\gram}} کشتی میں کھڑا ہے (شکل \حوالہء{9.45}-الف)۔ یہ کنارے کی طرف \عددی{\SI{2.4}{\meter}} چل کر رکتا ہے۔ کتّا  اب کنارے سے کتنا دور ہو گا؟   کشتی اور پانی کے بیچ رگڑ نظر انداز کریں۔ (اشارہ: شکل-ب دیکھیں۔)
\انتہا{سوال}
%------------------------------
%Q18 p248, Module 9-3, Linear Momentum
\جزوحصہء{خطی معیار حرکت}
%----------------------------------
\ابتدا{سوال}
ایک گیند جس کی کمیت \عددی{\SI{0.70}{\kilo\gram}} ہے \عددی{\SI{5.0}{\meter\per\second}} افقی حرکت کر کے انتصابی دیوار سے ٹکرا کر \عددی{\SI{2.0}{\meter\per\second}} رفتار سے واپس  پلٹتا ہے۔ گیند کے خطی معیار حرکت میں تبدیلی کیا ہو گی؟
\انتہا{سوال}
%---------------------------
\ابتدا{سوال}
ایک  ٹرک ، جس کی کمیت \عددی{\SI{2100}{\kilo\gram}} ہے، شمال کی طرف \عددی{\SI{41}{\kilo\meter\per\hour}} چلتے ہوئے مشرق  کو  مڑ کر  \عددی{\SI{51}{\kilo\meter\per\hour}} اسراع پذیر ہوتا ہے۔ (ا) ٹرک کے حرکی توانائی میں تبدیلی کیا ہو گی؟   ٹرک کے معیار حرکت میں تبدیلی کی (ب) قدر اور (ج)  تبدیلی کا رخ کیا ہو گا؟
\انتہا{سوال}
%-------------------------
\ابتدا{سوال}
ہم سطح زمین پر  رکھا گیند   وقت \عددی{t=0} پر  سطح  زمین سے مار کر  روانا کیا جاتا ہے۔گیند کا معیار حرکت \عددی{p} بالمقابل وقت \عددی{t} شکل \حوالہء{9.46} میں پیش ہے ( جہاں \عددی{p_0=\SI{6.0}{\kilo\gram\meter\per\second}} اور \عددی{p_1=\SI{4.0}{\kilo\gram\meter\per\second}} ہے)۔ گیند کا ابتدائی زاویہ کیا ہے؟ (اشارہ:وہ حل تلاش کریں جس میں ترسیم کا زیریں ترین نقطہ پڑھنے کی ضرورت پیش نہ آئے۔)
\انتہا{سوال}
%------------------------
%Q21
\ابتدا{سوال}
بلا سے ٹکرانے سے عین قبل  \عددی{\SI{0.30}{\kilo\gram}} کمیت  کا گیند \عددی{\SI{15}{\meter\per\second}} سمتی رفتار سے افق سے نیچے \عددی{\SI{35}{\degree}} زاویے کے ساتھ گامزن ہے۔ بلے کے ساتھ تماس کے دوران گیند کے معیار حرکت میں تبدیلی کی قدر  کیا ہو گی   اگر  گیند (ا) سیدھا انتصابی  نیچے  رخ \عددی{\SI{20}{\meter\per\second}}، اور (ب) افقی واپس \عددی{\SI{20}{\meter\per\second}} کی رفتار سے لوٹے؟
\انتہا{سوال}
%-----------------------------
\ابتدا{سوال}
شکل \حوالہء{9.47} میں   \عددی{\SI{0.165}{\kilo\gram}} کمیت  گیند  کا   فضائی جائزہ پیش ہے۔ گیند  اطرافی  دیوار  سے  ٹپکی کھاتا   دکھایا گیا ہے۔ گیند کی ابتدائی رفتار \عددی{\SI{2.00}{\meter\per\second}} اور زاویہ \عددی{\theta_1=\SI{30}{\degree}} ہے۔ ٹپکی گیند  کے سمتی رفتار کا \عددی{y} جزو الٹ کرتا ہے جبکہ \عددی{x} جزو اثر انداز نہیں ہوتا۔ (ا) زاویہ \عددی{\theta_2} کیا ہو گا؟ (ب) گیند کے خطی معیار حرکت میں تبدیلی اکائی سمتی ترقیم میں کیا ہو گی؟ (گیند کے   لڑھکاو  کا یہاں کوئی کردار نہیں۔)
\انتہا{سوال}
%-----------------------------
%module 9-4, Collision and Impulse
\جزوحصہء{تصادم اور  ضرب}
%Q23, p248
\ابتدا{سوال}
ایک مسخرہ \عددی{\SI{12}{\meter}} بلندی سے \عددی{\SI{30}{\centi\meter}} گہرے پانی میں  پیٹ کے بل گر کر لوگوں کا دات لیتا ہے۔فرض کریں، عین  پانی کی تہہ کو پہنچ کر یہ شخص   رکتا ہے۔ اس کی کمیت فرض کر کے اس پر پانی کی  ضرب کی قدر تلاش کریں۔
\انتہا{سوال}
%--------------------------
\ابتدا{سوال}
 چھتر سپاہی  \عددی{\SI{370}{\meter}} بلندی  پر پرواز کرتے  ہوئے طیارے سے کودتا ہے۔بدقسمتی سے اس کی چھتری نہیں کھل پاتی۔ وہ برف میں گر کر معمولی زخمی ہوتا ہے۔فرض کریں زمین پر پہنچ کر  اس کی (اخیر) رفتار  \عددی{\SI{56}{\meter\per\second}} اور   کمیت  ( بمع ساز و سامان) \عددی{\SI{85}{\kilo\gram}} ہے، اور  اس پر برف کی قوت کی قدر \عددی{\SI{1.2e5}{\newton}} ہے (جس پر انسان مشکل سے زندہ رہ پاتا ہے)۔ (ا) برف کی تہہ کم سے کم کتنی موٹی ہے؟ (ب)  اس پر برف کی ضرب کی قدر کیا ہے؟
\انتہا{سوال}
%-------------------------
%Q25 p 249
