%starting from top of P239
%ch 9 "Elastic Collision In One Dimension"
\باب{مرکز کمیت اور خطی معیار حرکت}
\حصہ{ایک بُعد میں  لچکی تصادم}
حرکی توانائی کی بقا درج ذیل لکھی جائے گی۔
\begin{align}\label{مساوات_مرکز_کمیت_حرکی_توانائی_کی_بقا}
\frac{1}{2}m_1v_{1i}^2+\frac{1}{2}m_2v_{2i}^2=\frac{1}{2}m_1v_{1f}^2+\frac{1}{2}m_2v_{2f}^2
\end{align}
ان ہمزاد  مساوات کو \عددی{v_{1f}} اور \عددی{v_{2f}} کے لئے حل کرنے کی خاطر  ہم مساوات \حوالہء{9.71} کو
\begin{align}\label{مساوات_مرکز_کمیت_بقا_معار_دوم}
m_1(v_{1i}-v_{1f})=-m_2(v_{2i}-v_{2f})
\end{align}
اور مساوات \حوالہ{مساوات_مرکز_کمیت_حرکی_توانائی_کی_بقا} درج ذیل صورت میں لکھتے ہیں۔
\begin{align}\label{مساوات_مرکز_کمیت_بقا_دوم}
m_1(v_{1i}-v_{1f})(v_{1i}+v_{1f})=-m_2(v_{2i}-v_{2f})(v_{2i}+v_{2f})
\end{align}
مساوات \حوالہ{مساوات_مرکز_کمیت_بقا_دوم} کو مساوات \حوالہ{مساوات_مرکز_کمیت_بقا_معار_دوم} سے تقسیم کر کے کچھ الجبرا کے بعد درج ذیل حاصل ہوں گے۔
\begin{align}\label{مساوات_مرکز_کمیت_اختتامی_الف}
v_{1f}=\frac{m_1-m_2}{m_1+m_2}v_{1i}+\frac{2m_2}{m_1+m_2}v_{2i}
\end{align}
اور
\begin{align}\label{مساوات_مرکز_کمیت_اختتامی_ب}
v_{2f}=\frac{2m_1}{m_1+m_2}v_{1i}+\frac{m_2-m_1}{m_1+m_2}v_{2i}
\end{align}
یاد رہے، زیر نوشت \عددی{1} اور \عددی{2} کسی خاص ترتیب سے مختص نہیں کیے گئے۔  مساوات \حوالہء{9.19} میں  اور مساوات \حوالہ{مساوات_مرکز_کمیت_اختتامی_الف} اور مساوات  \حوالہ{مساوات_مرکز_کمیت_اختتامی_ب} میں  ان زیر نوشت کو آپس میں بدل کر لکھنے  مساوات کی وہی جوڑی ملتی ہے۔ اس پر بھی توجہ دیں کہ \عددی{v_{2i}=0}  لینے سے، شکل \حوالہء{9.18} میں جسم \عددی{2} ساکن ہدف ہو گا، اور مساوات \حوالہ{مساوات_مرکز_کمیت_اختتامی_الف}  اور مساوات \حوالہ{مساوات_مرکز_کمیت_اختتامی_ب} ہمیں  بالترتیب مساوات \حوالہء{9.67} اور مساوات \حوالہء{9.68} دیتی ہیں۔ 



\ابتدا{آزمائش}
شکل \حوالہء{9.18} میں گولے کا ابتدائی معیار حرکت \عددی{\SI{6}{\kilo\gram\meter\per\second}} اور اختتامی معیار حرکت (ا)  \عددی{\SI{2}{\kilo\gram\meter\per\second}} اور (ب) \عددی{\SI{-2}{\kilo\gram \meter\per\second}} ہونے کی صورت میں   ہدف کا  اختتامی خطی معیار حرکت کیا ہو گا؟ اگر گولے کی  ابتدائی اور  اختتامی حرکی توانائی بالترتیب  \عددی{\SI{5}{\joule}} اور \عددی{\SI{2}{\joule}} ہو، ہدف کی اختتامی حرکی توانائی کیا ہو گی؟
\انتہا{آزمائش}
%----------------------

\ابتدا{نمونی سوال} \quad \موٹا{لچکی تصادم در لچکی تصادم}
شکل \حوالہء{9.20a} میں \عددی{v_{1i}=\SI{10}{\meter\per\second}} سے چلتا ہوا سل 1 دو ساکن سلوں کی طرف بڑھتا  ہے۔تینوں سل ایک لکیر پر  ہیں۔ یہ سل  \عددی{2} سے ٹکراتا ہے جو آگے سل \عددی{3} سے  جا کر ٹکراتا ہے، جس کی کمیت \عددی{m_3=\SI{6.0}{\kilo\gram}} ہے۔ دوسرے  تصادم  کے بعد سل \عددی{2} دوبارہ ساکن ہے،  اور سل \عددی{3} کی رفتار  \عددی{v_{3f}=\SI{5.0}{\meter\per\second}} ہے (شکل \حوالہء{9.20b})۔ دونوں تصادم لچکی ہیں۔ سل \عددی{1} اور سل \عددی{2} کی  کمیتیں کیا ہیں؟ سل \عددی{1} کی اختتامی رفتار
 \عددی{v_{1f}} کیا ہے؟
 
 \موٹا{کلیدی تصورات}
 
 چونکہ ہم تصادم لچکدار تصور کرتے ہیں لہٰذا میکانی توانائی کی  بقا ہو گی (یوں  ٹکر کی آواز، گرمی، اور ارتعاش کی بدولت توانائی کا  ضیاع نظر انداز کیا جاتا ہے)۔ کوئی  بیرونی افقی قوت  سلوں پر عمل نہیں کرتی لہٰذا محور \عددی{x} پر خطی معیار حرکت کی بقا ہو گی۔ ان دو وجوہات کی بنا پر ہم دونوں تصادم پر  مساوات \حوالہء{9.67} اور مساوات \حوالہء{9.68} کا اطلاق کر سکتے ہیں۔
 
 \موٹا{حساب}\quad
 پہلے  تصادم  سے آغاز کرتے ہوئے ہمیں  اتنے زیادہ  نا معلوم متغیرات سے واسطہ ہو گا کہ آگے بڑھنا  مشکل ہو گا: ہم سلوں کی کمیت اور اختتامی سمتی رفتار نہیں جانتے۔ آئیں پہلے تصادم سے آغاز کریں، جس میں سل \عددی{3} کے ساتھ ٹکرانے کے بعد سل \عددی{2} رکتی ہے۔ مساوات \حوالہء{9.67} کا  اطلاق  اس تصادم پر کرتے ہیں جہاں ترقیم تبدیل کرتے ہوئے \عددی{v_{2i}}  تصادم سے قبل سل \عددی{2} کی  رفتار اور \عددی{v_{2f}}  تصادم کے بعد اس کی رفتار  دیتی ہیں۔یوں درج ذیل ہو گا۔
 \begin{align*}
 v_{2f}=\frac{m_2-m_3}{m_2+m_3}v_{2i}
 \end{align*}
 اس میں \عددی{v_{2f}=0} (سل \عددی{2}  رک جاتا ہے) ڈالنے کے بعد  \عددی{m_3=\SI{6.0}{\kilo\gram}}  ڈال کر درج ذیل حاصل ہو گا۔
 \begin{align*}
 m_2&=m_3=\SI{6.0}{\kilo\gram}&&\text{\RL{(جواب)}}
 \end{align*}
اسی طرح  ترقیم  تبدیل کر کے دوسرے تصادم کے لئے  مساوات \حوالہء{9.68} لکھتے ہیں
\begin{align*}
v_{3f}=\frac{2m_2}{m_2+m_3}v_{2i}
\end{align*}
جہاں \عددی{v_{3f}} تیسرے سل کی اختتامی سمتی رفتار ہے۔ اس میں \عددی{m_2=m_3} ڈالنے کے بعد \عددی{v_{3f}=\SI{5.0}{\meter\per\second}} ڈال کر درج ذیل حاصل ہو گا۔
\begin{align*}
v_{2i}=v_{3f}=\SI{5.0}{\meter\per\second}
\end{align*}

%p240

آئیں اب پہلے تصادم پر غور کریں؛ ہمیں سل \عددی{2} کے لئے مستعمل ترقیم پر توجہ دینی ہو گی: تصادم کے بعد سل \عددی{2}  کی سمتی رفتار \عددی{v_{2f}} وہی ہے جو تصادم سے قبل  اس کی سمتی رفتار \عددی{v_{2i}=\SI{5.0}{\meter\per\second}} تھی۔ پہلے تصادم پر مساوات \حوالہء{68} کا اطلاق  کر کے دی گئی  \عددی{v_{1i}=\SI{10}{\meter\per\second}}
ڈال کر   ذیل    ہو گا
\begin{align*}
v_{2f}&=\frac{2m_1}{m_1+m_2}v_{1i}\\
\SI{5.0}{\meter\per\second}&=\frac{2m_1}{m_1+m_2}(\SI{10}{\meter\per\second})
\end{align*}
جو ذیل دیگا۔
\begin{align*}
m_1&=\frac{1}{3}m_2=\frac{1}{3}(\SI{6.0}{\kilo\gram})=\SI{2.0}{\kilo\gram}&&\text{\RL{(جواب)}}
\end{align*}
یہ  نتیجہ اور دی گئی \عددی{v_{1i}}  استعمال کرتے ہوئے  پہلے تصادم  پر مساوات \حوالہء{9.67} کا اطلاق کر کے درج ذیل لکھا جا سکتا ہے۔
\begin{align*}
v_{1f}&=\frac{m_1-m_2}{m_1+m_2}v_{1i}\\
&=\frac{\tfrac{1}{3}m_2-m_2}{\tfrac{1}{3}m_2+m_2}(\SI{10}{\meter\per\second})=\SI{-5.0}{\meter\per\second}&&\text{\RL{(جواب)}}
\end{align*}
\انتہا{نمونی سوال}
%--------------------

\حصہ{دو ابعاد میں تصادم}
\جزوحصہء{مقاصد}
اس حصہ کو پڑھنے کے بعد آپ درج ذیل کے قابل ہوں گے۔

جدا نظام کے لئے جس میں دو بُعدی تصادم  واقع ہو ، ہر ایک محور پر   معیار حرکت کی بقا  کا اطلاق کرتے ہوئے  ، تصادم  کے بُعد محور پر معیار حرکت کے  اجزاء  کا   اسی محور پر تصادم سے قبل معیار حرکت کے   اجزاء کے ساتھ رشتہ جان سکیں۔

جدا نظام کے لئے جس میں دو بُعدی لچکی تصادم واقع ہو، (ا)  ، ہر ایک محور پر   معیار حرکت کی بقا  کا اطلاق کرتے ہوئے  ، تصادم  کے بعد محور پر معیار حرکت کے  اجزاء  کا   اسی محور پر تصادم سے قبل معیار حرکت کے   اجزاء کے ساتھ رشتہ جان سکیں اور (ب)کل   حرکی توانائی  کی بقا کا اطلاق کر کے تصادم سے قبل اور تصادم کے بعد حرکی توانائیوں کا رشتہ جان سکیں۔

\جزوحصہء{کلیدی  تصور}
اگر دو جسم ٹکرائیں اور ان کی حرکت ایک محور پر نہ ہو (تصادم  آمنے سامنے سے  نہیں ہے)، تصادم دو بُعدی ہو گا۔ اگر دو جسمی نظام بند اور جدا ہو،تصادم پر  معیار حرکت کی بقا کے   قانون کا اطلاق ہو گا لہٰذا درج  ہو گا۔
\begin{align*}
\vec{P}_{1i}+\vec{P}_{2i}=\vec{P}_{1f}+\vec{P}_{2f}
\end{align*}
یہ قانون اجزاء کی صورت میں دو مساوات   (ہر  بُعد کے لئے ایک مساوات) دیگا جو تصادم کو بیان کرتی ہیں۔ اگر تصادم لچکی بھی ہو (جو  ایک خصوصی صورت ہے)، تصادم کے دوران حرکی توانائی کی بقا (ذیل)  تیسری مساوات دیگی۔
\begin{align*}
K_{1i}+K_{2i}=K_{1f}+K_{2f}
\end{align*}

\حصہء{دو بُعد میں تصادم}
جب دو اجسام کا تصادم  ہو، اجسام کس  رخ حرکت    کرتے ہیں ، اس کا تعین ان کے بیچ ضرب (جھٹکا ) کرتی ہے۔ بالخصوص، جب تصادم آمنے سامنے سے نہ ہو، اجسام اپنے اپنے   ابتدائی محور پر نہیں رہتے۔ ایسے دو بُعدی تصادم میں  جو بند، اور جدا نظام میں واقع ہو،  کل خطی معیار حرکت کی بقا  ہو گی۔
\begin{align}\label{مساوات_مرکز_کمیت_خطی_معیار_حرکت_بقا}
\vec{P}_{1i}+\vec{P}_{2i}=\vec{P}_{1f}+\vec{P}_{2f}
\end{align}
اگر تصادم لچکی بھی ہو (جو  ایک خصوصی صورت ہے)، تب کل حرکی توانائی کی بقا بھی ہو گی۔
\begin{align}\label{مساوات_مرکز_کمیت_حرکی_توانائی_بقا}
K_{1i}+K_{2i}=K_{1f}+K_{2f}
\end{align}

دو بُعدی تصادم  کا تجزیہ کرنے کے لئے مساوات \حوالہ{مساوات_مرکز_کمیت_خطی_معیار_حرکت_بقا} کو \عددی{xy} محددی نظام کے اجزاء کی صورت میں لکھنا زیادہ مفید ثابت ہوتا ہے۔ مثال کے طور پر، شکل \حوالہء{9.21} میں  ساکن ہدف  کو  گولا بغلی ( آمنے سامنے سے نہیں )  ٹکراتا ہے۔  ان  کے بیچ ضرب،  اجسام کو محور \عددی{x}، جس پر گولا ابتدائی طور حرکت میں تھا، کے لحاظ سے \عددی{\theta_1} اور \عددی{\theta_2}  زاویوں پر بھیجتی ہے۔ یہاں ہم مساوات \حوالہ{مساوات_مرکز_کمیت_خطی_معیار_حرکت_بقا} کو محور \عددی{x} کے ہمراہ ذیل
\begin{align}\label{مساوات_مرکز_کمیت_معیار_ایکس_جزو}
m_1v_{1i}=m_1v_{1f}\cos\theta_1+m_2v_{2f}\cos\theta_2
\end{align}
اور محور \عددی{y} کے ہمراہ ذیل لکھیں گے۔
\begin{align}\label{مساوات_مرکز_کمیت_معیار_وائے_جزو}
0=-m_1v_{1f}\sin \theta_1+m_2v_{2f}\sin\theta_2
\end{align}
ہم مساوات \حوالہ{مساوات_مرکز_کمیت_حرکی_توانائی_بقا} کو  (اس خصوصی صورت کے لئے) رفتار کے روپ میں لکھ سکتے ہیں۔
\begin{align}\label{مساوات_مرکز_حرکی_بصورت_رفتار}
\frac{1}{2}m_1v_{1i}^2&=\frac{1}{2}m_1v_{1f}^2+\frac{1}{2}m_2v_{2f}^2&&\text{\RL{(حرکی توانائی)}}
\end{align}
مساوات \حوالہ{مساوات_مرکز_کمیت_معیار_ایکس_جزو} تا مساوات \حوالہ{مساوات_مرکز_حرکی_بصورت_رفتار} میں سات متغیر ہیں: دو کمیت، \عددی{m_1} اور \عددی{m_2}؛ تین رفتار، \عددی{v_{1i}}، \عددی{v_{1f}}، اور \عددی{v_{2f}}؛ اور دو زاویے، \عددی{\theta_1} اور \عددی{\theta_2}۔اگر ہم  ان میں سے کوئی بھی چار متغیرات جانتے ہوں،  باقی  تین متغیرات ان تین مساوات کو حل کر کے   معلوم کیے جا سکتے ہیں۔

%---------------------------
\ابتدا{نمونی سوال}
فرض کریں شکل \حوالہء{9.21} میں گولے کا  ابتدائی معیار حرکت  \عددی{\SI{6}{\kilo\gram\meter\per\second}} ،  جبکہ  اختتامی  معیار حرکت کا \عددی{x} جزو \عددی{\SI{4}{\kilo\gram\meter\per\second}} اور اختتامی معیار حرکت کا \عددی{y} جزو \عددی{\SI{-3}{\kilo\gram\meter\per\second}} ہے۔ ہدف کے (ا) اختتامی معیار  حرکت کا \عددی{x} جزو اور (ب) اختتامی معیار حرکت کا \عددی{y} جزو کیا ہوں گے؟
\انتہا{نمونی سوال}
%-----------------------------

\حصہ{متغیر کمیت کے  نظام: ہوائی بان}
\جزوحصہء{مقاصد}
اس حصہ کو پڑھنے کے بعد آپ  ذیل کے قابل ہوں گے۔

\اصطلاح{ہوائی بان }\فرہنگ{ہوائی بان}\حاشیہب{rocket}\فرہنگ{rocket} کی پہلی مساوات استعمال کر کے ہوائی بان کی کمیت میں کمی کی شرح، ہوائی بان کے لحاظ سے اخراجی مادے کی اضافی رفتار، ہوائی بان کی کمیت، اور ہوائی بان کی اسراع کا رشتہ جان پائیں گے۔

ہوائی بان کی دوسری مساوات استعمال کر کے اخراجی مادے کی اضافی رفتار کے لحاظ سے ہوائی بان کی رفتار ، اور ہوائی بان  کی ابتدائی اور اختتامی کمیت کا رشتہ جان پائیں گے۔

ایک ایسا حرکت پذیر  نظام  جس کی کمیت دی گئی شرح سے تبدیل ہوتی ہو کے لئے  اس شرح    اور معیار حرکت میں تبدیلی  کا رشتہ جان پائیں گے۔

\جزوحصہء{کلیدی تصورات}
بیرونی قوتوں کی غیر موجودگی میں ہوائی بان درج ذیل لمحاتی شرح سے  اسراع پذیر ہو گا،
\begin{align*}
Rv_{\text{\RL{اضافی}}}&=Ma &&\text{\RL{(ہوائی بان کی پہلی مساوات)}}
\end{align*}
جہاں \عددی{M} ہوائی بان کی لمحاتی کمیت (بشمول غیر استعمال شدہ ایندھن) ، \عددی{R} ایندھن  کے استعمال کی شرح، اور \عددی{v_{\text{\RL{اضافی}}}} ہوائی بان کے لحاظ سے    اخراج کی اضافی رفتار ہے۔ جزو \عددی{Rv_{\text{\RL{اضافی}}}} ہوائی بان انجن کا دھکا ہے۔

مستقل \عددی{R} اور \عددی{v_{\text{\RL{اضافی}}}} کی صورت میں اگر  ہوائی بان  کی رفتار \عددی{v_i} سے تبدیل ہو کر    \عددی{v_f}  ہو جائے، اور کمیت \عددی{M_i} سے تبدیل ہو کر \عددی{M_f} ہو جائے تب درج ذیل ہو گا۔
\begin{align*}
v_f-v_i=&v_{\text{\RL{اضافی}}}\ln\frac{M_i}{M_f}&&\text{\RL{(ہوائی بان کی دوسری مساوات)}}
\end{align*}

\جزوحصہء{متغیر کمیت کے  نظام: ہوائی بان}
اب تک ہم فرض کرتے رہے ہیں کہ نظام کی کل  کمیت اٹل ہے۔ بعض اوقات، مثلاً ہوائی بان میں، ایسا نہیں ہو گا۔اڑان سے قبل  \اصطلاح{چبوترہ روانگی   }\فرہنگ{چبوترہ!روانگی}\حاشیہب{launching pad}\فرہنگ{launching pad} پر کھڑے ہوائی بان کی زیادہ تر کمیت دراصل ایندھن ہو گی، جو  آخر کار جل کر ہوائی بان کے انجن کی  \اصطلاح{ٹونٹی }\فرہنگ{ٹونٹی}\حاشیہب{nozzle}\فرہنگ{nozzle} سے   دھویں کی شکل میں  خارج ہو گا۔  اسراع پذیر ہوائی بان کی متغیر کمیت سے نپٹنے کی خاطر نیوٹن کے   دوسرے قاعدے کا اطلاق، صرف  ہوائی بان  کی بجائے،  ہوائی بان اور  خارجی مواد  دونوں  کو اکٹھا  لیتے ہوئے کیا جاتا ہے۔ہوائی بان کی اسراع کے دوران  اس نظام کی کمیت  تبدیل نہیں ہو گی۔

\جزوحصہء{اسراع کی تلاش}
فرض کریں ہم  جمودی حوالہ  چھوکٹ کے لحاظ سے ساکن بیٹھے گہری فضا میں، جہاں کوئی تجاذبی یا ہوائی  کی  رگڑ  ی قوت موجود نہیں،  ہوائی بان کو اسراع کرتا دیکھ رہے ہیں۔ اس یک بُعدی حرکت  کے لئے   ہم ، اختیاری لمحہ \عددی{t} پر، ہوائی بان کی کمیت \عددی{M} اور سمتی رفتار \عددی{v }  فرض کرتے ہیں (شکل \حوالہء{9.22a})۔

شکل \حوالہء{9.22b}  وقتی دورانیہ \عددی{\dif t} کے بعد صورت حال پیش کرتی ہے۔ ہوائی بان کی سمتی رفتار \عددی{v+\dif v} اور کمیت \عددی{M+\dif M} ہیں، جہاں کمیت میں تبدیلی \عددی{\dif M}\ترچھا{ منفی مقدار } ہے۔ وقفہ \عددی{\dif t} کے دوران ہوائی بان سے  اخراجی مواد کی کمیت \عددی{-\dif M}  اور جمودی  حوالہ چھوکٹ کے لحاظ  سے    مواد کی سمتی  رفتار \عددی{U} ہے۔

\جزوحصہء{معیار حرکت کی بقا ہو گی}
ہمارا  نظام  ہوائی بان اور وقفہ \عددی{\dif t} میں اخراجی مواد پر مشتمل ہے۔ نظام بند اور  جدا ہے لہٰذا وقفہ \عددی{\dif t} کے دوران نظام کی خطی معیار حرکت کی بقا لازمی ہے۔ یوں ذیل ہو گا
\begin{align}\label{مساوات_مرکز_کمیت_معیار_بقا_لازمی}
P_i=P_f
\end{align}
جہاں زیر نوشت  \عددی{i} اور \عددی{f} بالترتیب  وقفہ  \عددی{\dif t} کے آغاز میں اور  اس کے اختتام پر قیمتیں ظاہر کرتی ہیں۔ مساوات \حوالہ{مساوات_مرکز_کمیت_معیار_بقا_لازمی} درج ذیل لکھی جا سکتی ہے
\begin{align}\label{مساوات_مرکز_کمیت_وقفہ_کے_دوران}
Mv=-\dif M \,U+(M+\dif M)(v+\dif v)
\end{align}
جہاں دائیں ہاتھ پہلا جزو وقفہ  \عددی{\dif t} کے دوران خارج کردہ مواد کا  خطی معیار حرکت اور  دوسرا جزو وقفہ  \عددی{\dif t} کے اختتام  پر ہوائی بان کا خطی معیار حرکت  ہے۔

\جزوحصہء{اضافی رفتار کا  استعمال}
مساوات \حوالہ{مساوات_مرکز_کمیت_وقفہ_کے_دوران} کی سادہ صورت  ہوائی بان اور اخراجی مواد کے بیچ اضافی رفتار \عددی{v_{\text{\RL{اضافی}}}}  استعمال کرکے  حاصل کی جا سکتی ہے۔اضافی رفتار اور چھوکٹ کے لحاظ سے سمتی رفتاروں  کے بیچ درج ذیل تعلق پایا جاتا ہے۔
\begin{align*}
\left(\parbox{3cm}{\centering چھوکٹ کے لحاظ سے ہوائی بان کی سمتی رفتار}\right)=\left(\parbox{3cm}{\centering اخراجی مواد کے لحاظ سے ہوائی بان کی سمتی رفتار}\right)+\left(\parbox{3cm}{\centering چھوکٹ کے لحاظ سے اخراجی مواد کی سمتی رفتار}\right)
\end{align*}
اس کو علامتی روپ میں لکھتے ہیں۔
\begin{align}
(v+\dif v)&=v_{\text{\RL{اضافی}}}+U\notag\\
U&=v+\dif v-v_{\text{\RL{اضافی}}}\hspace{3cm}\text{یعنی}
\end{align}
اس نتیجہ کو مساوات \حوالہ{مساوات_مرکز_کمیت_وقفہ_کے_دوران} میں \عددی{U} کی جگہ ڈال کر کچھ الجبرا کے بعد ذیل حاصل ہو گا۔
\begin{align}\label{مساوات_مرکز_کمیت_تعلق_بان}
-\dif M\,v_{\text{\RL{اضافی}}}=M\dif v
\end{align}
دونوں اطراف \عددی{\dif t} سے تقسیم کرتے ہیں۔
\begin{align}\label{مساوات_مرکز_کمیت_اضافی_رفتار_الف}
-\frac{\dif M}{\dif t}v_{\text{\RL{اضافی}}}=M\frac{\dif v}{\dif t}
\end{align}
ہم \عددی{\dif M\!/\!\dif t}  (جو ہوائی بان کی کمیت میں کمی کی شرح ہے)  کو \عددی{-R} لکھتے ہیں، جہاں \عددی{R} ایندھن  جلنے کی (مثبت) شرح ہے، اور \عددی{\dif v\!/\!\dif t} ہوائی بان کی اسراع ہے۔ ان تبدیلیوں کے ساتھ مساوات \حوالہ{مساوات_مرکز_کمیت_اضافی_رفتار_الف} ذیل روپ اختیار کرتی ہے۔
\begin{align}\label{مساوات_مرکز_کمیت_ہوائی_بان_پہلی}
Rv_{\text{\RL{اضافی}}}=Ma  \quad \text{\RL{(ہوائی بان کی پہلی مساوات)}}
\end{align}
ہر   لمحے پر مقادیر کی قیمتیں مساوات \حوالہ{مساوات_مرکز_کمیت_ہوائی_بان_پہلی}     مطمئن  کرتی ہیں۔

مساوات \حوالہ{مساوات_مرکز_کمیت_ہوائی_بان_پہلی} کا بایاں  ہاتھ  قوت کا بُعد \عددی{(\si{\kilo\gram\per\second}\cdot\si{\meter\per\second}=\si{\kilo\gram}\cdot\si{\meter\per\second\squared}=\si{\newton})}  رکھتا ہے اور  صِرف ہوائی بان کی بناوٹ پر منحصر ہے؛ یعنی، شرح \عددی{R} پر ، جس سے ایندھن (کمیت ) صَرف کیا جاتا ہے ، اور  رفتار \عددی{v_{\text{\RL{اضافی}}}} پر،  جس سے   یہ کمیت ہوائی بان سے خارج کی جاتی ہے۔ہم اس جزو \عددی{Rv_{\text{\RL{اضافی}}}}  کو ہوائی بان کی\اصطلاح{   قوت  دھکیل }\فرہنگ{قوت دھکیل}\حاشیہب{thrust}\فرہنگ{thrust} کہتے   اور  \عددی{T} سے ظاہر کرتے ہیں۔ مساوات \حوالہ{مساوات_مرکز_کمیت_ہوائی_بان_پہلی} کو \عددی{T=Ma} لکھ کر نیوٹن کا دوسرا قانون حاصل ہوتا ہے، جہاں اس لمحے پر جب ہوائی بان کی کمیت \عددی{M} ہے اس کی اسراع \عددی{a} ہے۔

\جزوحصہء{سمتی رفتار کی تلاش}
ہم جاننا چاہتے ہیں کہ جیسے جیسے ہوائی بان  ایندھن صَرف کرتا ہے اس کی  سمتی رفتار کیسے تبدیل ہو گی۔ مساوات \حوالہ{مساوات_مرکز_کمیت_تعلق_بان}  ذیل کہتی ہے۔
\begin{align*}
\dif v=-v_{\text{\RL{اضافی}}}\frac{\dif M}{M}
\end{align*}
اس کے تکمل
\begin{align*}
\int_{v_i}^{v_f}\dif v=-v_{\text{\RL{اضافی}}}\int_{M_i}^{M_f}\frac{\dif M}{M}
\end{align*}
میں \عددی{M_i}ہوائی بان کی  ابتدائی کمیت اور \عددی{M_f} اختتامی کمیت ہے۔ تکمل لینے سے ذیل حاصل ہو گا
\begin{align}\label{مساوات_مرکز_کمیت_ہوائی_بان_دوم_مساوات}
v_f-v_i=v_{\text{\RL{اضافی}}}\ln\frac{M_i}{M_f}\quad \text{\RL{(ہوائی بان کی دوسری مساوات)}}
\end{align}
جو  ہوائی بان کی کمیت \عددی{M_i} سے گھٹ کر  \عددی{M_f} ہونے کی صورت میں ہوائی بان  کی رفتار میں اضافہ دیتی ہے۔ (مساوات \حوالہ{مساوات_مرکز_کمیت_ہوائی_بان_دوم_مساوات} میں علامت \عددی{\ln}\اصطلاح{ قدرتی لوگارتھم }\فرہنگ{لوگارتھم!قدرتی}\حاشیہب{natural logarithm}\فرہنگ{logarithm!natural} ظاہر کرتی ہے۔) ہم یہاں \اصطلاح{  کثیرالمراحل }\فرہنگ{کثیرالمراحل}\حاشیہب{multistage}\فرہنگ{multistage} ہوائی بان کی افادیت   جان سکتے ہیں  جو   ایندھن ختم ہونے پر خالی  ٹینکی سے چھٹکارا حاصل کر کے \عددی{M_f} گھٹاتا ہے۔ مثالی ہوائی بان  مطلوبہ مقام پر  صرف ضروری ساز و سامان کے ساتھ پہنچے گا۔

%----------------------------
\ابتدا{نمونی سوال}\موٹا{ہوائی بان کا انجن، قوت دھکیل، اسراع}
اس باب کی تمام گزشتہ مثالوں میں نظام کی کمیت اٹل تھی۔ یہاں ہم ایسے نظام ( ہوائی بان ) کی بات کرتے ہیں جس کی کمیت بتدریج کم ہوتی ہے۔ ایک  ہوائی بان  جس کی ابتدائی کمیت \عددی{M_i=\SI{850}{\kilo\gram}} ہے  \عددی{R=\SI{2.3}{\kilo\gram\per\second}}  شرح سے   ایندھن  صَرف کرتا ہے۔ ہوائی بان کے لحاظ سے اخراجی مواد کی
 رفتار \عددی{v_{\text{\RL{اضافی}}}=\SI{2800}{\meter\per\second}} ہے۔ (ا)  ہوائی بان کا انجن کتنی قوت دھکیل پیدا کرتا ہے؟
 
 \جزوحصہء{کلیدی تصور}
 مساوات \حوالہ{مساوات_مرکز_کمیت_ہوائی_بان_پہلی} کے تحت ایندھن صَرف کرنے کی شرح \عددی{R}  کو اخراجی مواد کی اضافی رفتار \عددی{v_{\text{\RL{اضافی}}}} سے ضرب دینے سے قوت دھکیل  \عددی{T} حاصل ہو گی۔
 
حساب:\quad
یوں درج ذیل ہو گا۔
\begin{align*}
T&=Rv_{\text{\RL{اضافی}}}=(\SI{2.3}{\kilo\gram\per\second})(\SI{2800}{\meter\per\second})\\
&=\SI{6440}{\newton}\approx \SI{6400}{\newton}\quad \quad \text{\RL{(جواب)}}
\end{align*}
(ب) ہوائی بان کی ابتدائی اسراع کیا ہو گی؟

\جزوحصہء{کلیدی  تصور}
ہم ہوائی بان کی قوت دھکیل \عددی{T} اور اس کی اسراع کی قدر \عددی{a} کا رشتہ \عددی{T=Ma} جانتے ہیں، جہاں  \عددی{M} ہوائی بان کی کمیت ہے۔ لیکن، جیسے جیسے ایندھن صَرف ہوتا ہے \عددی{M} گھٹتی اور \عددی{a} بڑھتا ہے۔ ہمیں ابتدائی اسراع درکار ہے لہٰذا ہم ہوائی بان کی ابتدائی کمیت  \عددی{M_i} لیں گے۔

حساب:\quad
ان معلومات سے ذیل حاصل ہو گا۔
\begin{align*}
a=\frac{T}{M}=\frac{\SI{6440}{\newton}}{\SI{850}{\kilo\gram}}=\SI{7.6}{\meter\per\second\squared}\quad \quad \text{\RL{(جواب)}}
\end{align*}
سطح زمین سے  سیدھا اوپر اڑان کے لئے ضروری ہے کہ ابتدائی اسراع \عددی{g=\SI{9.8}{\meter\per\second\squared}} سے  زیادہ  ہو۔ یعنی، ابتدائی اسراع کو سطح زمین پر تجاذبی اسراع سے زیادہ ہونا ہو گا۔دوسرے لفظوں میں،  ہوائی  بان پر  ابتدائی تجاذبی قوت  ، جس کی قدر \عددی{M_ig} ہے
\begin{align*}
(\SI{850}{\kilo\gram})(\SI{9.8}{\meter\per\second\squared})=\SI{8330}{\newton}
\end{align*}
  سے   قوت دھکیل \عددی{T} کا زیادہ ہونا لازمی ہے، ورنہ  ہوائی بان زمین سے اٹھنے کے قابل نہیں ہو گا۔ چونکہ اس ہوائی بان کی قوت دھکیل  (جو یہاں \عددی{T=\SI{6440}{\newton}} ہے) درکار قدر سے کم ہے لہٰذا یہ ہوائی بان اڑ نہیں پائے گا؛یہاں زیادہ طاقتور ہوائی بان کی ضرورت ہے۔
\انتہا{نمونی سوال}
%---------------------------------

\حصہء{نظر ثانی اور خلاصہ}
\جزوحصہء{مرکز کمیت}
ایک نظام جو \عددی{n} ذرات پر مشتمل ہو کے مرکز کمیت  کی تعریف  وہ نقطہ ہے جس کے محدد درج ذیل ہوں۔
\begin{align*}
x_{\text{\RL{مرکزکمیت}}}&=\frac{1}{M}\sum_{i=1}^{n} m_ix_i\\
y_{\text{\RL{مرکزکمیت}}}&=\frac{1}{M}\sum_{i=1}^{n} m_iy_i\\
z_{\text{\RL{مرکزکمیت}}}&=\frac{1}{M}\sum_{i=1}^{n} m_i z_i \tag{\arabicdigits{\setlatin9.5}}
\end{align*}
اس کو مختصراً ذیل لکھا جا سکتا ہے، جہاں \عددی{M} نظام کی کل کمیت  \عددی{\sum_{i=1}^{n} m_i} ہے۔
\begin{align*}
\vec{r}_{\text{\RL{مرکزکمیت}}}=\frac{1}{M}\sum_{i=1}^{n} m_i\vec{r}_i\tag{\arabicdigits{\setlatin9.8}}
\end{align*}

\جزوحصہء{نیوٹن کا دوسرا قانون برائے ذرات کا نظام}
ایک نظام ، جو ذرات پر مشتمل ہو، کے مرکز کمیت کی حرکت \موٹا{ نیوٹن کے دوسرے قانون برائے ذرات پر مشتمل نظام }کے تحت ہو گی، جو ذیل کہتا ہے۔
\begin{align*}
\vec{F}_{\text{\RL{صافی}}}=M\vec{a}_{\text{\RL{مرکزکمیت}}}\tag{\arabicdigits{\setlatin9.14}}
\end{align*}
یہاں نظام پر لاگو   تمام \ترچھا{بیرونی }قوتیں مل کر صافی قوت \عددی{\vec{F}_{\text{\RL{صافی}}}} دیتی ہیں۔ نظام کی کل کمیت \عددی{M}، اور  نظام کے مرکز کمیت 
کی اسراع \عددی{\vec{a}_{\text{\RL{مرکزکمیت}}}} ہے۔

\جزوحصہء{خطی معیار حرکت اور نیوٹن کا دوسرا قانون}
تنہا ذرے کے لئے،   مقدار  \عددی{\vec{p}} متعارف کر کے ،  جو  اس ذرے کا \موٹا{ خطی  معیار حرکت } کہلاتا ہے اور جس کی تعریف ذیل ہے،
\begin{align*}
\vec{p}=m\vec{v}\tag{\arabicdigits{\setlatin9.22}}
\end{align*}
ہم نیوٹن کا دوسرا قانون اس معیار حرکت کی صورت میں  لکھ سکتے ہیں۔
\begin{align*}
\vec{F}_{\text{\RL{صافی}}}=\frac{\dif \vec{p}}{\dif t}\tag{\arabicdigits{\setlatin9.23}}
\end{align*}
ذرات پر مشتمل نظام کے لئے   مذکورہ بالا دو تعلق   ذیل  لکھا جائیں گے۔
\begin{align*}
 \vec{F}_{\text{\RL{صافی}}}=\frac{\dif \vec{P}}{\dif t} \quad \text{\RL{اور}}\quad  \vec{P}=M\vec{v}_{\text{\RL{مرکزکمیت}}}
 \tag{\arabicdigits{\setlatin9.25،\,9.27}}
\end{align*}

\جزوحصہء{تصادم اور ضرب}
تصادم میں ملوث ذرہ نما جسم پر معیار حرکت کے روپ میں نیوٹن کے دوسرے قانون کا اطلاق  \موٹا{ضرب و خطی معیار حرکت کا  مسئلہ }دیگا:
\begin{align*}
\vec{p}_f-\vec{p}_i=\Delta \vec{p}=\vec{J} \tag{\arabicdigits{\setlatin9.31،\,9.32}}
\end{align*}
جہاں  جسم کے خطی معیار حرکت میں تبدیلی \عددی{\vec{p}_f-\vec{p}_i=\Delta \vec{p}} ہے ، اور\موٹا{ ضرب}     \عددی{\vec{J}} وہ  قوت  \عددی{\vec{F}(t)} ہے جو تصادم کے دوران دوسرا جسم اس (پہلے جسم)  پر لاگو کرتا ہے۔ 
\begin{align*}
\vec{J}=\int_{t_i}^{t_f} \vec{F}(t)\dif t   \tag{\arabicdigits{\setlatin9.30}}
\end{align*}
اگر تصادم کا دورانیہ \عددی{\Delta t} اور اس دوران  \عددی{\vec{F}(t)}  کی اوسط  قیمت   \عددی{F_{\text{\RL{اوسط}}}} ہو تب یک بُعدی حرکت کے لئے ذیل ہو گا۔
\begin{align*}
J=F_{\text{\RL{اوسط}}}\Delta t     \tag{\arabicdigits{\setlatin9.35}}
\end{align*}
 ساکن  جسم  پر  کمیت \عددی{m}   کے ذرے،  جن کی    رفتار \عددی{v}  ہے ، برس کر   ذیل اوسط قوت   پیدا کرتے ہیں
\begin{align*}
F_{\text{\RL{اوسط}}}=-\frac{n}{\Delta t}\Delta p=-\frac{n}{\Delta t}m\Delta v  \tag{\arabicdigits{\setlatin9.37}}
\end{align*}
جہاں ساکن جسم سے ذروں کے  تصادم کی شرح \عددی{n\!/\!\Delta t} ، اور ہر ایک ذرے کی رفتار میں تبدیلی \عددی{\Delta v} ہے   ( جسم ساکن رہتا ہے)۔ یہ اوسط قوت ذیل بھی لکھی جا سکتی ہے
\begin{align*}
F_{\text{\RL{اوسط}}}=-\frac{\Delta M}{\Delta t}\Delta v    \tag{\arabicdigits{\setlatin9.40}}
\end{align*}
جہاں \عددی{\Delta M\!/\!\Delta t} وہ شرح ہے جس سے  کمیت ساکن جسم سے ٹکراتی ہے۔درج بالا دو مساوات میں اگر ذرے تصادم کے بعد رک جاتے ہوں تب \عددی{\Delta v=-v} ہو گا، اور اگر  ذرے  جسم پر ٹپکی  کھا کر  رفتار میں تبدیلی کے بغیر واپس لوٹیں  تب  \عددی{\Delta v=-2v} ہو گا۔

\جزوحصہء{خطی معیار حرکت کی بقا}
  جدا    نظام   پر  بیرونی قوت عمل نہیں کرتی، لہٰذا  اس نظام کا خطی معیار حرکت  تبدیل نہیں ہو گا۔
  \begin{align*}
  \vec{P}=\text{\RL{مستقل}}\quad \quad \text{\RL{(بند، جدا نظام)}}   \tag{\arabicdigits{\setlatin9.42}}
  \end{align*}
  اس کو ذیل بھی لکھ سکتے ہیں جہاں زیر نوشت کسی ابتدائی لمحہ اور   اختتامی لمحہ کو ظاہر کرتی ہیں۔
    \begin{align*}
  \vec{P}_i=\vec{P}_f\quad \quad \text{\RL{(بند، جدا نظام)}}   \tag{\arabicdigits{\setlatin9.43}}
  \end{align*}
  مذکورہ بالا دونوں مساوات \موٹا{خطی معیار حرکت کی بقا}   کو بیان کرتی ہیں۔
  
  \جزوحصہء{ایک بُعد میں غیر لچکی تصادم}
دو اجسام کی\ترچھا{ غیر لچکی } تصادم میں دو جسمی نظام کی حرکی توانائی کی بقا نہیں ہو گی (حرکی توانائی  مستقل نہیں ہو گی)۔ اگر نظام بند اور جدا ہو ، نظام کے کل خطی معیار حرکت کی بقا لازماً  ہو گی (یہ مستقل  ہو گا)، جس کو سمتیہ روپ میں ذیل لکھا جا سکتا ہے، جہاں زیر نوشت \عددی{i} اور \عددی{j} بالترتیب تصادم سے عین  قبل اور اس کے عین  بعد لمحات ظاہر کرتی ہیں۔
\begin{align*}
\vec{p}_{1i}+\vec{p}_{2i}=\vec{p}_{1f}+\vec{p}_{2f}   \tag{\arabicdigits{\setlatin9.50}}
\end{align*}

ذروں  کی حرکت ایک محور پر ہونے کی صورت میں تصادم یک بُعدی ہو گا اور ہم مذکورہ بالا مساوات کو  محور کے ہمراہ  سمتی رفتار اجزاء کی صورت میں  ذیل لکھ سکتے ہیں۔
\begin{align*}
m_1v_{1i}+m_2v_{2i}=m_1v_{1f}+m_2v_{2f}    \tag{\arabicdigits{\setlatin9.51}}
\end{align*}

اگر  دو جسم آپس میں چپک جائیں، تصادم \ترچھا{مکمل غیر لچکی }ہو گا اور  دونوں  اجسام کی اختتامی سمتی رفتار \عددی{V} ہو گی (کیونکہ یہ آپس میں جڑے ہیں)۔

\جزوحصہء{مرکز کمیت کی حرکت}
دو متصادم  اجسام  کے  بند،  جدا نظام  کے   مرکز کمیت  پر تصادم  اثر  انداز نہیں ہو گا۔ بالخصوص، مرکز کمیت کی سمتی رفتار \عددی{\vec{v}_{\text{\RL{مرکزکمیت}}}} کو  تصادم  تبدیل نہیں کرتا۔

\جزوحصہء{ایک بُعد میں لچکی تصادم}
\ترچھا{لچکی تصادم}  ایک خاص قسم کا تصادم ہے جس میں متصادم اجسام کے نظام کی حرکی توانائی  برقرار رہتی ہے۔اگر نظام بند اور جدا بھی ہو، اس کا خطی معیار حرکت بھی برقرار رہے گا۔یک بُعدی تصادم کے لئے، جس میں جسم \عددی{2} ہدف اور جسم \عددی{1} گولا ہے، حرکی توانائی اور خطی معیار حرکت کی بقا، تصادم کے  عین بعد   سمتی رفتاروں کے لئے درج ذیل مساوات دیتی ہیں۔
\begin{align*}
v_{1f}&=\frac{m_1-m_2}{m_1+m_2}v_{1i}    \tag{\arabicdigits{\setlatin9.67}}\\
v_{2f}&=\frac{2m_1}{m_1+m_2}v_{1i}      \tag{\arabicdigits{\setlatin9.68}}
\end{align*}
\جزوحصہء{دو ابعاد میں تصادم}
اگر دو جسم یوں ٹکرائیں کہ   ان  کی حرکت ایک  ہی محور پر نہ ہو (ٹکر آمنے سامنے سے نہیں)، تصادم دو بُعدی ہو گا۔اگر دو جسمی نظام بند اور جدا ہو، معیار حرکت کی بقا کے قانون  کا اطلاق تصادم پر ہو گا جو ذیل لکھا جائے گا۔
\begin{align*}
\vec{P}_{1i}+\vec{P}_{2i}=\vec{P}_{1f}+\vec{P}_{2f}       \tag{\arabicdigits{\setlatin9.77}}
\end{align*}
اجزاء کے روپ میں یہ قانون دو مساوات دے گا جو تصادم کو بیان کریں گی   (دو ابعاد میں ہر بُعد کے لئے ایک مساوات) ۔ اگر تصادم لچکی بھی ہو (خصوصی صورت)، تصادم کے دوران حرکی توانائی کی بقا تیسری مساوات دیگی۔
\begin{align*}
K_{1i}+K_{2i}=K_{1f}+K_{2f}         \tag{\arabicdigits{\setlatin9.78}}
\end{align*}

\جزوحصہء{متغیر کمیتی نظام}
بیرونی قوتوں کی عدم موجودگی میں ہوائی بان ذیل لمحاتی شرح سے اسراع پذیر ہو گا
\begin{align*}
Rv_{\text{\RL{اضافی}}}=Ma\quad\quad\text{\RL{(ہوائی بان کی پہلی مساوات )}}       \tag{\arabicdigits{\setlatin9.87}}
\end{align*}
جہاں \عددی{M} ہوائی بان کی لمحاتی کمیت  (جس میں غیر استعمال شدہ   ایندھن شامل ہے)، \عددی{R} ایندھن کے اصراف کی شرح، اور \عددی{v_{\text{\RL{اضافی}}}}  ہوائی بان کے لحاظ سے اخراج کی اضافی   رفتار ہے۔جزو \عددی{Rv_{\text{\RL{اضافی}}}}  ہوائی بان  کی انجن کی \موٹا{قوت دھکیل }ہے۔  جب  ایک ہوائی بان  کی،جس کی \عددی{R} اور \عددی{v_{\text{\RL{اضافی}}}}   اٹل ہو،    کمیت \عددی{M_i} سے \عددی{M_f} ہونے پر اس کی رفتار \عددی{v_i} سے \عددی{v_f} ہو،   درج ذیل ہو گا۔
\begin{align*}
v_f-v_i=v_{\text{\RL{اضافی}}}\ln\frac{M_i}{M_f}\quad \quad \text{\RL{(ہوائی بان کی دوسری مساوات)}}       \tag{\arabicdigits{\setlatin9.88}}
\end{align*}

%=======================================
%p245
\حصہء{سوالات}
%Q1
\ابتدا{سوال}
تین ذرات جن پر بیرونی قوتیں عمل کرتی ہیں  کا فضائی جائزہ  شکل \حوالہء{9۔23}  میں  پیش  ہے۔ دو ذروں پر قوتوں کی قدریں اور سمتیں دی گئی ہیں۔ تین ذروی نظام  کا مرکز کمیت (ا) ساکن، (ب) دائیں رخ مستقل  سمتی رفتار سے، اور (ج) اوپر وار اسراع پذیر ہونے کی صورت میں تیسری قوت کی قدر اور سمت تلاش کریں۔
\انتہا{سوال}
%-----------------------
\ابتدا{سوال}
بلا رگڑ  مستوی  پر مستقل سمتی   رفتاروں سے  حرکت کرتے  ہوئے ایک برابر کمیت کے چار ذروں کا فضائی جائزہ    شکل \حوالہء{9.24} میں پیش ہے۔ سمتی رفتاروں کے رخ دیے گئے ہیں؛ ان کی قدریں برابر ہیں۔ ذروں کی جوڑیاں بنائیں۔ کون  سی جوڑی  ایسا نظام دیتی ہے جس کا مرکز کمیت (ساکن ہے، (ب)  ساکن ہے اور مبدا پر ہے، اور (ج) مبدا سے گزرتا ہے؟
\انتہا{سوال}
%----------------------
%Q3
\ابتدا{سوال}
فرض کریں ایک ڈبہ،   جو \عددی{x} محور پر  مستقل   مثبت سمتی رفتار سے حرکت میں ہو، دھماکے سے دو  ٹکڑوں  میں تقسیم ہوتا ہے۔  ایک  ٹکڑا ، جس کی کمیت \عددی{m_1} ہے ،  مثبت سمتی رفتار \عددی{\vec{v}_1}  سے حرکت کرتا ہے۔ دوسرا ٹکڑا جس کی کمیت \عددی{m_2} ہے  (ا)  مثبت سمتی رفتار  \عددی{\vec{v}_2}  (شکل \حوالہء{9.25a})، (ب) منفی سمتی رفتار \عددی{\vec{v}_2} (شکل \حوالہء{9.25b})، یا (ج)  صفر سمتی رفتار  (شکل \حوالہء{ 9.25c}) رکھ سکتا ہے۔  ان ممکن نتائج کی درجہ بندی  مطابقتی \عددی{\vec{v}_1} کی قدر  کے لحاظ سے ،اعظم اول  رکھ کر، کریں۔ 
\انتہا{سوال}
%----------------------
\ابتدا{سوال}
تصادم میں ملوث  جسم کے لئے  قوت کی قدر بالمقابل وقت کی ترسیمات شکل \حوالہء{9.26} میں پیش ہیں۔ ترسیمات کی درجہ بندی  جسم پر قوت دھکیل کی قدر  کے لحاظ سے، اعظم اول رکھ کر،  کریں۔
\انتہا{سوال}
%--------------------------
%Q5
\ابتدا{سوال}
بلا رگڑ مستوی پر حرکت کرتے    تین ڈبوں پر عمل پیرا قوت  کا  فضائی نظارہ شکل \حوالہء{9.27} میں پیش ہے۔ ہر ایک ڈبہ کے لئے ،  کیا   محور \عددی{x} اور محور \عددی{y} کے ہمراہ خطی معیار حرکت کی بقا ہو گی؟
\انتہا{سوال}
%------------------------------
\ابتدا{سوال}
تین یا  چار یکساں ذروں کا گروہ ، جو محور \عددی{x} یا محور \عددی{y} کے متوازی  ایک رفتار سے حرکت کرتے ہوں، شکل \حوالہء{9.28} میں دکھایا گیا ہے۔ مرکز کمیت کی رفتار کے لحاظ سے ان کی درجہ بندی، اعظم اول رکھ کر،  کریں۔
\انتہا{سوال}
%----------------------------
%Q7
\ابتدا{سوال}
ایک سل بلا رگڑ فرش  پر حرکت کر کے اس جتنی کمیت کی دوسری سل سے  ٹکراتی ہے۔ شکل \حوالہء{9.29} میں سلوں کی حرکی توانائی \عددی{K}  کی چار  ممکنہ ترسیم  پیش ہیں۔ (ا) ان میں سے کون سی طبیعی  وجوہات کی بنا پر  ممکن نہیں؟ باقی میں سے  کونسی (ب) لچکی تصادم اور (ج) غیر لچکی تصادم بہتر ظاہر کرتی ہے؟
\انتہا{سوال}
%--------------------------
\ابتدا{سوال}
بلا رگڑ فرش پر محور \عددی{x} کے ہمراہ   سل  \عددی{1}   ساکن  سل \عددی{2} کی طرف بڑھتا ہے۔عین لچکی تصادم سے  قبل لمحہ پر   ان کی تصویر کشی شکل \حوالہء{9.30} میں  کی گئی  ہے۔ اس لمحہ پر دو  سل نظام کے مرکز کمیت کے تین ممکن مقام بھی پیش  ہیں۔ (نقطہ \عددی{B}     سلوں کے مراکز کے  درمیان  نصف فاصلے پر ہے۔)  اگر  تصادم کے بعد نظام کا مرکز کمیت  (ا) \عددی{A} پر، (ب) \عددی{B} پر، اور (ج) \عددی{C} پر ہو، کیا سل \عددی{1} ساکن  ہو گا؟ آگے  کی طرف گامزن ہو گا؟     پیچھے کی طرف گامزن ہو گا؟
\انتہا{سوال}
%--------------------------
%Q9
\ابتدا{سوال}
دو  اجسام   محور \عددی{x}  کے ہمراہ یک بُعدی  لچکی تصادم  کا شکار ہوتے ہیں۔ شکل \حوالہء{9.31} میں  اجسام  اور     مرکز کمیت کے  مقام بالمقابل وقت  ترسیمات پیش ہیں۔ (ا) کیا دونوں جسم ابتدائی طور پر حرکت میں تھی، یا ان میں سے ایک ساکن تھا؟ کونسا لکیری قطع (ب)  تصادم سے قبل اور (ج) تصادم کے بعد  مرکز کمیت دیتا ہے؟ (د)  کیا تصادم سے قبل زیادہ تیز  حرکت کرتے جسم کی کمیت دوسرے جسم  کی کمیت سے زیادہ ہے، کم ہے، یا اس کے برابر ہے؟
\انتہا{سوال}
%--------------------------------------
\ابتدا{سوال}
افقی فرش پر  سل ابتدائی طور ساکن ، محور \عددی{x} کے ہمراہ مثبت رخ ، یا  محور کے منفی رخ حرکت میں ہے۔ سل دھماکے سے دو ٹکڑوں میں تقسیم ہوتا ہے جو اسی محور پر حرکت کرتے ہیں۔ فرض کریں سل اور اس کے دو ٹکڑے ایک بند اور جدا  نظام دیتے ہیں۔سل اور   ٹکڑوں کے معیار حرکت بالمقابل وقت \عددی{t}   کی  چھ ترسیمات شکل \حوالہء{9.32} میں پیش ہیں۔ کونسی ترسیمات طبیعی نا ممکن ہیں؟ وجوہات پیش کریں۔
\انتہا{سوال}
%----------------------------
\ابتدا{سوال}
محور \عددی{x} پر کمیت \عددی{m_1} کا سل بلا رگڑ فرش پر  چلتا ہوا کمیت \عددی{m_2} کے ساکن سل سے لچکی   متصادم ہوتا  ہے۔ شکل \حوالہء{9.33}  میں  سل \عددی{1} کا مقام \عددی{x}  بالمقابل وقت \عددی{t}  ٹھوس لکیر سے پیش  کیا گیا ہے ، جس پر  نقطہ    تصادم \عددی{x_c} اور وقت تصادم \عددی{t_c}    کی نشاندہی کی گئی ہے۔ اگر (ا) \عددی{m_1<m_2}   اور (ب) \عددی{m_1>m_2} ہو، تصادم کے بعد  سل \عددی{1} نقطہ دار راہ \عددی{A}، \عددی{B}، \عددی{C} ، اور \عددی{D} میں کس      پر   گامزن ہو گا؟ (ج)  اگر \عددی{m_1=m_2} ہو یہ راہ   \عددی{1}، \عددی{2}، \عددی{3}، \عددی{4}، اور  \عددی{4} میں کس  پر گامزن ہو گا؟
\انتہا{سوال}
%-----------------------------
%Q12
\ابتدا{سوال}
دو  جسم  اور ان کے مرکز کمیت کی  مقام بالمقابل وقت  کی  چار ترسیمات پیش ہیں۔ یہ جسم بند اور جدا نظام دیتے ہیں اور محور \عددی{x} پر چلتے ہوئے یک بُعدی  مکمل غیر لچکی تصادم کا شکار ہوتے ہیں۔ کیا ترسیم \عددی{1} میں (ا) دو جسم اور (ب) مرکز کمیت محور \عددی{x} پر مثبت رخ یا منفی رخ حرکت  کرتے ہیں؟ (ج) کونسی ترسیم طبیعی ناممکن ہے؟ وجہ پیش کریں۔
\انتہا{سوال}
%--------------------------------
% p246, module 9-1   Center of Mass
\جزوحصہء{مرکز کمیت}
%---------------------------------
%Q1, p246
\ابتدا{سوال}
کمیت \عددی{\SI{2.00}{\kilo\gram}}  ذرے کا \عددی{xy} محدد \عددی{(\SI{-1.20}{\meter},\SI{0.500}{\meter})}، اور کمیت \عددی{\SI{4.00}{\kilo\gram}} ذرے کا \عددی{xy} محدد  \عددی{(\SI{0.600}{\meter},\SI{-0.750}{\meter})} ہے۔ دونوں افقی مستوی پر پائے جاتے ہیں۔ کمیت \عددی{\SI{3.00}{\kilo\gram}} کا تیسرا ذرہ  کس (ا) \عددی{x} اور (ب) \عددی{y} پر رکھ کر تین ذروی نظام کا مرکز کمیت \عددی{(\SI{-0.500}{\meter},\SI{-0.700}{\meter})} پر ہو گا؟
\انتہا{سوال}
%----------------------------------------
\ابتدا{سوال}
تین ذروی نظام جس میں \عددی{m_1=\SI{3.0}{\kilo\gram}}، \عددی{m_2=\SI{4.0}{\kilo\gram}}، اور \عددی{m_3=\SI{8.0}{\kilo\gram}} ہے شکل \حوالہء{9.35} میں پیش ہے۔ محور کے  پیما \عددی{x_s=\SI{2.0}{\meter}} اور \عددی{y_s=\SI{2.0}{\meter}}کے لحاظ سے رکھے گئے ہیں۔ نظام کے مرکز کمیت کا (ا) \عددی{x} محدد اور (ب) \عددی{y} محدد کیا ہو گا؟ (ج)  کیا \عددی{m_3} بتدریج بڑھانے سے مرکز کمیت  اس ذرے کی جانب منتقل ہو گا، اس سے دور منتقل ہو گا ، یا ساکن رہے گا؟
\انتہا{سوال}
%--------------------------
\ابتدا{سوال}
ایک سل جس  کے اضلاع \عددی{d_1=\SI{11.0}{\centi\meter}}، \عددی{d_2=\SI{2.80}{\centi\meter}}، اور \عددی{d_3=\SI{13.0}{\centi\meter}} ہیں شکل \حوالہء{9.36} میں دکھایا گیا ہے۔ اس کا نصف حصہ  المونیم    (کثافت \عددی{\SI{2.70}{\gram\per\centi\meter\cubed}} ) اور آدھا لوہے (کثافت \عددی{\SI{7.85}{\gram\per\centi\meter\cubed}}) کا  ہے۔  سل کے مرکز کمیت کا (ا) \عددی{x} محدد، (ب) \عددی{y} محدد، اور (ج) \عددی{z} محدد کیا ہو گا؟
\انتہا{سوال}
%---------------------------
%Q4 p 247
\ابتدا{سوال}
تین  یکساں پیکر   ڈنڈیاں جن میں ہر ایک کی لمبائی \عددی{L=\SI{22}{\centi\meter}}  ہے  مل کر  الٹ نون غُنّہ بناتی ہیں (شکل \حوالہء{9.37})۔ انتصابی ڈنڈی  کی کمیت \عددی{\SI{14}{\gram}}  اور افقی ڈنڈی کی کمیت \عددی{\SI{42}{\gram}} ہے۔ نظام کے مرکز کمیت کا  (ا)  \عددی{x} محدد اور (ب) \عددی{y} محدد کیا ہو گا؟
\انتہا{سوال}
%---------------------------
\ابتدا{سوال}
یکساں موٹائی کا  چادر شکل \حوالہء{9.38} میں پیش ہے۔ اگر \عددی{L=\SI{5.0}{\centi\meter}} ہو چادر کے مرکز کمیت  کا (ا) \عددی{x} محدد اور (ب) \عددی{y} محدد کیا ہو گا؟
\انتہا{سوال}
%--------------------------------
\ابتدا{سوال}
قابل نظر انداز موٹائی کی  یکساں  دھاتی چادر  سے بنایا گیا مکعب  شکل \حوالہء{9.39} میں پیش ہے۔ مکعب اوپر سے کھلا ہے اور اس کا کنارہ \عددی{L=\SI{40}{\centi\meter}} لمبا  ہے۔ مکعب کے مرکز کمیت  کا  (ا) \عددی{x} محدد، (ب) \عددی{y} محدد، اور (ج) \عددی{z} محدد تلاش کریں۔
\انتہا{سوال}
%------------------------------
%Q7
\ابتدا{سوال}
ایمونیا سالمہ   \عددی{(\ce{NH3})} ، جس میں ہائیڈروجن جوہر     \عددی{(\ce{H})} متساوی الاضلاع  مثلث  بناتے ہیں ، شکل \حوالہء{9.40} میں پیش ہے۔ مثلث کا مرکز ہر   \عددی{\ce{H}} جوہر سے \عددی{d=\SI{9.40e-11}{\meter}}  فاصلے پر ہے۔ نائیٹروجن جوہر \عددی{\ce{N}} اس ہرم کی چوٹی پر واقع ہے جس کا   تل تین  \عددی{\ce{H}} جوہر بناتے ہیں۔ نائیٹروجن  اور  ہائیڈروجن  کی جوہری کمیت نسبت  \عددی{13.9} ، اور نائیٹروجن تا ہائیڈروجن فاصلہ \عددی{L=\SI{10.14e-11}{\meter}} ہے۔ سالمہ کے مرکز کمیت کا (ا) \عددی{x} محدد اور (ب) \عددی{y} محدد کیا ہو گا؟
\انتہا{سوال}
%---------------------------------
\ابتدا{سوال}
 یکساں پیکر کی   بوتل  جس کی کمیت \عددی{\SI{0.140}{\kilo\gram}} اور  لمبائی \عددی{\SI{12.0}{\centi\meter}}  ہے، میں  \عددی{\SI{0.354}{\kilo\gram}} مشروب بھری  ہے (شکل \حوالہء{9.41})۔ بوتل کے سر اور تل میں  ، مشروب  خارج کرنے کی غرض سے، باریک سوراخ (جو بوتل کی کمیت پر اثر انداز نہیں ہوتے)  کیے جاتے ہیں۔ (ا)  مکمل بھری بوتل (بمع مشروب) کے مرکز  کمیت   کی  اور  (ب) مکمل خالی بوتل کے مرکز کمیت کی بلندی \عددی{h}  کیا ہو گی؟ (ج)  جیسے  جیسے مشروب  خارج ہوتا ہے، \عددی{h} کو کیا ہو گا؟ (د)  مرکز کمیت کے  لمحاتی بلندی کو \عددی{x} کہہ کر اس کی کمتر قیمت تلاش کریں۔
\انتہا{سوال}
%-----------------------------
%Q9 p247, Module 9-2, Newton's Second Law for a System of Particles
\جزوحصہء{نیوٹن کا دوسرا قاعدہ برائے ذرات کا  نظام }
%-----------------------------------------------
\ابتدا{سوال}
ایک پتھر \عددی{t=0} پر گرنے دیا جاتا ہے۔ دوسرا پتھر جس کی کمیت دگنی ہے، اسی بلندی سے، \عددی{t=\SI{100}{\milli\second}} پر گرنے دیا جاتا ہے۔ (ا) نقطہ رہائی سے ، \عددی{t=\SI{300}{\milli\second}}  پر،   دو پتھر نظام کا مرکز کمیت  کتنا نیچے ہو گا؟ (دونوں پتھر اس لمحے تک ہوا میں ہیں۔) (ب)  اس لمحے پر دو پتھر نظام کا مرکز کمیت کس رفتار سے حرکت کرتا ہے؟
\انتہا{سوال}
%---------------------------------
\ابتدا{سوال}
چوراہا بتی پر \عددی{\SI{1000}{\kilo\gram}} کمیت کی  گاڑی  کھڑی ہے۔ جیسے ہی بتی سبز ہوتی ہے گاڑی \عددی{\SI{4.0}{\meter\per\second\squared}}  مستقل اسراع سے  حرکت میں آتی ہے۔ عین اسی لمحے ایک ٹرک جس کی کمیت \عددی{\SI{2000}{\kilo\gram}} اور جو \عددی{\SI{8.0}{\meter\per\second}} رفتار سے چل رہا ہے گاڑی سے آگے نکلتا ہے۔ (ا) گاڑی و ٹرک نظام کا مرکز کمیت \عددی{t=\SI{3.0}{\second}} بعد  بتی سے کتنا دور اور (ب) اس کی  رفتار  کیا ہو گی؟
\انتہا{سوال}
%------------------------
\ابتدا{سوال}
زیتون  کا ایک بڑا پھل     \عددی{(m=\SI{0.50}{\kilo\gram})}  \عددی{xy} محددی نظام کے  مرکز پر، اور  جوز برازیل   \عددی{M=\SI{1.5}{\kilo\gram}} نقطہ
 \عددی{(\SI{1.0}{\meter},\SI{2.0}{\meter})}  پر پڑا ہے۔لمحہ \عددی{t=0} پر قوت \عددی{\vec{F}_z=(2.0\hat{i}+3.0\hat{j})\,\si{\newton}} زیتون کے پھل  پر اور \عددی{\vec{F}_j=(-3.0\hat{i}-2.0\hat{j})\,\si{\newton}} جوز برازیل  پر  عمل کرنا  شروع کرتی ہیں۔ لمحہ \عددی{t=0} کے لحاظ سے \عددی{t=\SI{4.0}{\second}} پر  زیتون و جوز نظام کے  مرکز کمیت  کا  ہٹاو  اکائی سمتی ترقیم میں کیا ہو گا؟
\انتہا{سوال}
%---------------------------
\ابتدا{سوال}
دو  پھسلن باز ، جن میں سے ایک کی کمیت \عددی{\SI{65}{\kilo\gram}} اور دوسرے کی \عددی{\SI{40}{\kilo\gram}} ہے، \عددی{\SI{10}{\meter}} لمبا ڈنڈا ، جس کی کمیت قابل نظر انداز ہے  ، تھامے  برف پر  کھڑے ہیں۔ ڈنڈے کے  سروں سے آغاز کرتے ہوئے پھسلن باز ڈنڈا کھینچ کر ، ڈنڈے کے ہمراہ حرکت  کرتے ہوئے  قریب آ کر ، ملتے ہیں۔  کم کمیتی شخص کتنا فاصلہ طے کرتا ہے؟
\انتہا{سوال}
%-----------------------------
%Q13
\ابتدا{سوال}
ایک گولا  \عددی{\SI{20}{\meter\per\second}} کی ابتدائی   سمتی رفتار   \عددی{\vec{v}_0} کے ساتھ افق سے  \عددی{\theta_0=\SI{60}{\degree}} زاویہ  اوپر پھینکا جاتا ہے۔ خط حرکت کے   بلند تر نقطہ پر  گولا دھماکے سے دو برابر  ٹکڑوں میں تقسیم ہوتا ہے (شکل \حوالہء{9-42})۔ ایک ٹکڑا جس کا رفتار دھماکے کے عین بعد صفر ہے سیدھا نیچے گرتا ہے۔ دوسرا ٹکڑا  توپ سے  کتنے فاصلے پر گرتا ہے؟ (ہوائی رگڑ نظر انداز کریں اور زمین  ہموار تصور کریں۔)
\انتہا{سوال}
%--------------------------
\ابتدا{سوال}
وقت \عددی{t=0} پر دو ذرے محددی نظام کے مبدا سے  پھینکے  جاتے ہیں (شکل \حوالہء{9-43})۔ ذرہ \عددی{1} جس کی کمیت \عددی{m_1=\SI{5.00}{\gram}} ہے بلا رگڑ  افقی زمین  پر محور \عددی{x} کے ہمراہ \عددی{\SI{10.0}{\meter\per\second}} رفتار سے   روانا  کیا جاتا ہے۔  ذرہ \عددی{2} جس کی کمیت \عددی{m_2=\SI{3.00}{\gram}} ہے \عددی{\SI{20}{\meter\per\second}} سے  اوپری زاویے پر یوں پھینکا جاتا ہے کہ  یہ ہر لمحہ ذرہ \عددی{1}کے ٹھیک  اوپر  رہتا ہے۔ (ا) دو ذروی نظام کا مرکز کمیت  کتنی زیادہ سے زیادہ بلندی \عددی{H_{\text{\RL{بلندتر}}}} کو پہنچتا ہے؟ اکائی سمتی ترقیم میں  مرکز کمیت کی (ب) سمتی رفتار اور (ج) اسراع اس لمحے کیا ہو گی جب مرکز کمیت  \عددی{H_{\text{\RL{بلندتر}}}} پر  ہو؟
\انتہا{سوال}
%----------------------------------
%Q15 p248
\ابتدا{سوال}
ایک ریڑھی جو ہوائی ڈگر پر چلتی ہے    رسی کے ذریعہ اینٹ سے منسلک  ہے جو لٹک رہی ہے (شکل \حوالہء{9.44})۔ ریڑھی کی کمیت \عددی{m_1=\SI{0.600}{\kilo\gram}} اور اس کا مرکز کمیت ابتدائی طور پر \عددی{(\SI{-0.500}{\meter},\SI{0}{\meter})}   محدد  پر ہے۔ اینٹ کی کمیت \عددی{m_2=\SI{0.400}{\kilo\gram}}  اور اس کا مرکز کمیت ابتدائی طور پر \عددی{(\SI{0}{\meter},\SI{-0.100}{\meter})} محدد   پر ہے۔ رسی اور چرخی کی کمیت قابل نظر انداز ہے۔ریڑھی ساکن حالت سے رہا کی جاتی ہے۔ ریڑھی اور اینٹ حرکت کرتی  ہیں  حتٰی کہ ریڑھی چرخی سے ٹکراتی ہے۔ ریڑھی اور  ہوائی ڈگر کے بیچ رگڑ  ، اور  چرخی اور دھرے  کے  بیچ  رگڑ قابل نظر انداز ہے۔ (ا) ریڑھی و اینٹ نظام کے مرکز کمیت کی اسراع اکائی سمتی ترقیم میں کیا  ہو   گی؟  (ب)  مرکز کمیت کی سمتی رفتار بطور وقت \عددی{t} کا تفاعل  کیا ہو گی؟ (ج) مرکز کمیت کی راہ ترسیم کریں۔ (د)  اگر راہ قوسی ہو، کیا یہ دائیں اوپر  جانب یا بائیں  نیچے جانب  ابھرتی ہے، اور اگر راہ سیدھی ہو، \عددی{x} محور اور راہ کے بیچ زاویہ کیا ہو گا؟
\انتہا{سوال}
%--------------------------
\ابتدا{سوال}
زریاب  جس کی کمیت \عددی{\SI{80}{\kilo\gram}} ہے اور اسد جو ہلکا ہے \عددی{\SI{30}{\kilo\gram}} ساکن  کشتی میں بیٹھ  (ناران میں) کر سیف  الملوک  جھیل کا نظارہ  کر رہے ہیں۔ ان   کی  نشستیں  \عددی{\SI{3.0}{\meter}} فاصلے پر، اور کشتی کے مرکز کمیت کے لحاظ سے متشاکلی واقع ہیں۔ دونوں  آپس میں نشست  تبدیل کرتے ہیں۔ اگر کشتی کا مرکز کمیت  گھاٹ کے لحاظ سے \عددی{\SI{40}{\centi\meter}} افقی حرکت کرے، اسد کی کمیت کیا ہو گی؟
\انتہا{سوال}
%------------------------
\ابتدا{سوال}
کنارے سے \عددی{D=\SI{6.1}{\meter}} فاصلے پر \عددی{\SI{4.5}{\kilo\gram}} کتّا \عددی{\SI{18}{\kilo\gram}} کشتی میں کھڑا ہے (شکل \حوالہء{9.45}-الف)۔ یہ کنارے کی طرف \عددی{\SI{2.4}{\meter}} چل کر رکتا ہے۔ کتّا  اب کنارے سے کتنا دور ہو گا؟   کشتی اور پانی کے بیچ رگڑ نظر انداز کریں۔ (اشارہ: شکل-ب دیکھیں۔)
\انتہا{سوال}
%------------------------------
%Q18 p248, Module 9-3, Linear Momentum
\جزوحصہء{خطی معیار حرکت}
%----------------------------------
\ابتدا{سوال}
ایک گیند جس کی کمیت \عددی{\SI{0.70}{\kilo\gram}} ہے \عددی{\SI{5.0}{\meter\per\second}} افقی حرکت کر کے انتصابی دیوار سے ٹکرا کر \عددی{\SI{2.0}{\meter\per\second}} رفتار سے واپس  پلٹتا ہے۔ گیند کے خطی معیار حرکت میں تبدیلی کیا ہو گی؟
\انتہا{سوال}
%---------------------------
\ابتدا{سوال}
ایک  ٹرک ، جس کی کمیت \عددی{\SI{2100}{\kilo\gram}} ہے، شمال کی طرف \عددی{\SI{41}{\kilo\meter\per\hour}} چلتے ہوئے مشرق  کو  مڑ کر  \عددی{\SI{51}{\kilo\meter\per\hour}} اسراع پذیر ہوتا ہے۔ (ا) ٹرک کے حرکی توانائی میں تبدیلی کیا ہو گی؟   ٹرک کے معیار حرکت میں تبدیلی کی (ب) قدر اور (ج)  تبدیلی کا رخ کیا ہو گا؟
\انتہا{سوال}
%-------------------------
\ابتدا{سوال}
ہم سطح زمین پر  رکھا گیند   وقت \عددی{t=0} پر  سطح  زمین سے مار کر  روانا کیا جاتا ہے۔گیند کا معیار حرکت \عددی{p} بالمقابل وقت \عددی{t} شکل \حوالہء{9.46} میں پیش ہے ( جہاں \عددی{p_0=\SI{6.0}{\kilo\gram\meter\per\second}} اور \عددی{p_1=\SI{4.0}{\kilo\gram\meter\per\second}} ہے)۔ گیند کا ابتدائی زاویہ کیا ہے؟ (اشارہ:وہ حل تلاش کریں جس میں ترسیم کا زیریں ترین نقطہ پڑھنے کی ضرورت پیش نہ آئے۔)
\انتہا{سوال}
%------------------------
%Q21
\ابتدا{سوال}
بلا سے ٹکرانے سے عین قبل  \عددی{\SI{0.30}{\kilo\gram}} کمیت  کا گیند \عددی{\SI{15}{\meter\per\second}} سمتی رفتار سے افق سے نیچے \عددی{\SI{35}{\degree}} زاویے کے ساتھ گامزن ہے۔ بلے کے ساتھ تماس کے دوران گیند کے معیار حرکت میں تبدیلی کی قدر  کیا ہو گی   اگر  گیند (ا) سیدھا انتصابی  نیچے  رخ \عددی{\SI{20}{\meter\per\second}}، اور (ب) افقی واپس \عددی{\SI{20}{\meter\per\second}} کی رفتار سے لوٹے؟
\انتہا{سوال}
%-----------------------------
\ابتدا{سوال}
شکل \حوالہء{9.47} میں   \عددی{\SI{0.165}{\kilo\gram}} کمیت  گیند  کا   فضائی جائزہ پیش ہے۔ گیند  اطرافی  دیوار  سے  ٹپکی کھاتا   دکھایا گیا ہے۔ گیند کی ابتدائی رفتار \عددی{\SI{2.00}{\meter\per\second}} اور زاویہ \عددی{\theta_1=\SI{30}{\degree}} ہے۔ ٹپکی گیند  کے سمتی رفتار کا \عددی{y} جزو الٹ کرتا ہے جبکہ \عددی{x} جزو اثر انداز نہیں ہوتا۔ (ا) زاویہ \عددی{\theta_2} کیا ہو گا؟ (ب) گیند کے خطی معیار حرکت میں تبدیلی اکائی سمتی ترقیم میں کیا ہو گی؟ (گیند کے   لڑھکاو  کا یہاں کوئی کردار نہیں۔)
\انتہا{سوال}
%-----------------------------
%module 9-4, Collision and Impulse
\جزوحصہء{تصادم اور  ضرب}
%Q23, p248
\ابتدا{سوال}
ایک مسخرہ \عددی{\SI{12}{\meter}} بلندی سے \عددی{\SI{30}{\centi\meter}} گہرے پانی میں  پیٹ کے بل گر کر لوگوں کا دات لیتا ہے۔فرض کریں، عین  پانی کی تہہ کو پہنچ کر یہ شخص   رکتا ہے۔ اس کی کمیت فرض کر کے اس پر پانی کی  ضرب کی قدر تلاش کریں۔
\انتہا{سوال}
%--------------------------
\ابتدا{سوال}
 چھتر سپاہی  \عددی{\SI{370}{\meter}} بلندی  پر پرواز کرتے  ہوئے طیارے سے کودتا ہے۔بدقسمتی سے اس کی چھتری نہیں کھل پاتی۔ وہ برف میں گر کر معمولی زخمی ہوتا ہے۔فرض کریں زمین پر پہنچ کر  اس کی (اخیر) رفتار  \عددی{\SI{56}{\meter\per\second}} اور   کمیت  ( بمع ساز و سامان) \عددی{\SI{85}{\kilo\gram}} ہے، اور  اس پر برف کی قوت کی قدر \عددی{\SI{1.2e5}{\newton}} ہے (جس پر انسان مشکل سے زندہ رہ پاتا ہے)۔ (ا) برف کی تہہ کم سے کم کتنی موٹی ہے؟ (ب)  اس پر برف کی ضرب کی قدر کیا ہے؟
\انتہا{سوال}
%-------------------------
%Q25 p 249
\ابتدا{سوال}
زمین پر \عددی{\SI{1.2}{\kilo\gram}} کا  گیند   \عددی{\SI{25}{\meter\per\second}} رفتار سے انتصابی  گرتا ہے۔ٹپکی کے بعد اس کی ابتدائی رفتار \عددی{\SI{10}{\meter\per\second}} ہے۔ (ا)  تماس کے دوران گیند پر کتنی ضرب  عمل کرتی ہے؟ (ب)  اگر گیند \عددی{\SI{0.020}{\second}} کے لئے زمین کے ساتھ مس ہو، زمین پر گیند کی اوسط قوت کتنی ہو گی؟
\انتہا{سوال}
%-----------------------------------
\ابتدا{سوال}
عین  اس وقت جب ایک شخص ، جس کی کمیت \عددی{\SI{70}{\kilo\gram}} ہے  ، کرسی پر بیٹھتا  ہے اس کا شرارتی دوست کرسی کھینچ لیتا ہے، جس کی بدولت  پہلا شخص \عددی{\SI{0.50}{\meter}} نیچے   زمین پر گرتا ہے۔ اگر زمین کے ساتھ تصادم کا دورانیہ  \عددی{\SI{0.082}{\second}}  ہو، تصادم کے دوران شخص پر زمین    (ا)  کی ضرب  اور (ب) اوسط قوت کتنی ہو گی؟
\انتہا{سوال}
%------------------------
\ابتدا{سوال}
محور \عددی{x}  پر ابتدائی طور پر  مثبت رخ  \عددی{\SI{14}{\meter\per\second}}  سے حرکت کرتے ہوئے \عددی{\SI{0.40}{\kilo\gram}} گیند پر \عددی{\SI{27}{\milli\second}} کے لئے  محدد کے منفی رخ  قوت عمل کرتی ہے۔ قوت کی قدر میں  تبدیل پائی جاتی ہے اور  ضرب کی قدر \عددی{\SI{32.4}{\newton\second}} ہے۔ قوت لاگو کرنے کے عین  بعد گیند کی (ا)  رفتار اور (ب) اس کا  رخ کیا ہو گا؟ (ج) قوت کی اوسط قدر اور (د)  گیند پر  ضرب کا رخ کیا ہو گا؟
\انتہا{سوال}
%-------------------------------
%Q28
\ابتدا{سوال}
ایک پہلوان    میز پر   \عددی{\SI{13}{\meter\per\second}} رفتار سے  تھپڑ  مارتا ہے۔ اس کا ہاتھ \عددی{\SI{5.0}{\milli\second}}  کے تصادم میں رکتا  ہے۔ فرض کریں تصادم  کے دوران ہاتھ اور بازو  ایک دوسرے پر اثر انداز نہیں ہوتے  اور ہاتھ کی کمیت \عددی{\SI{0.70}{\kilo\gram}} ہے۔ ہاتھ پر  میز کی (ا) ضرب  کی قدر اور (ب) اوسط قوت کی قدر کیا ہو گی؟
\انتہا{سوال}
%--------------------------------
\ابتدا{سوال}
 ہدف پر  \عددی{\SI{3}{\gram}} کی   \عددی{100} گولیاں فی سیکنڈ شرح سے  چلائی جاتی ہیں۔ گولی  کی رفتار \عددی{\SI{500}{\meter\per\second}} ہے۔فرض کریں گولیاں  اسی رفتار سے ٹپک کر واپس لوٹتی ہیں۔ ہدف پر   اوسط قوت   کی قدر کیا ہو گی؟
\انتہا{سوال}
%-----------------------------------
\ابتدا{سوال}\ترچھا{دو اوسط قوت}
دیوار پر \عددی{\SI{0.250}{\kilo\gram}}  کے برف گولے \عددی{\SI{4.00}{\meter\per\second}}   رفتار سے لگاتار عمودی  مارے جاتے ہیں۔ہر گولا  دیوار  سے چپکتا ہے۔ شکل \حوالہء{9.49} میں دیوار پر  دو  متواتر تصادم  کی قوت  کی قدر \عددی{F} بالمقابل وقت \عددی{t} پیش ہے۔ تصادم کی  تکرار کا وقفہ  \عددی{\Delta t_r=\SI{50.0}{\milli\second}} اور دورانیہ \عددی{\Delta t_d=\SI{10}{\milli\second}} ہے، جو ترسیم پر مساوی الساقین مثلث بناتے ہیں۔   ہر تصادم   کی قوت کی زیادہ سے زیادہ قدر \عددی{F_{\text{\RL{بلندتر}}}=\SI{200}{\newton}}  ہے۔ ہر تصادم کے دوران دیوار پر  (ا) ضرب اور (ب) اوسط قوت کی قدریں کیا ہوں گی؟ (ج)  کئی تصادم کے دوران دیوار پر اوسط قوت  کی قدر کیا ہو گی؟
\انتہا{سوال}
%--------------------------------
%Q31, p249
\ابتدا{سوال}\ترچھا{بلند کن کی ٹکر سے پہلے اچھلنا}
بلند کن  کا رسا ٹوٹتا ہے اور بدقسمتی سے  اس کا حفاظتی نظام بھی ناکارہ ہوتا ہے، جس کی بدولت یہ \عددی{\SI{36}{\meter}} بلندی سے گرتا ہے۔زمین پر پہنچ کر \عددی{\SI{90}{\kilo\gram}} سوار  \عددی{\SI{5.0}{\milli\second}} کے تصادم   میں رکتا ہے۔ (فرض کریں نہ بلند کن اور نہ یہ شخص ٹپکی کھاتے ہیں۔) تصادم کے دوران شخص پر  (ا) ضرب   اور (ب)  اوسط قوت کی قدریں کیا ہو ں گی؟ اگر  عین تصادم سے قبل ، بلند کن کے لحاظ سے شخص \عددی{\SI{7.0}{\meter\per\second}} کی رفتار سے   اوپر  چھلانگ  لگائے (ج) ضرب اور (د) اوسط قوت کی قدریں کیا ہوں گی (رکنے کا دورانیہ وہی تصور کریں)؟
\انتہا{سوال}
%--------------------
\ابتدا{سوال}
بچوں کا  کھلونا  جس کی کمیت \عددی{\SI{5.0}{\kilo\gram}} ہے محور \عددی{x} پر حرکت کر سکتا ہے ۔شکل \حوالہء{9.50}  اس قوت کا جزو \عددی{F_x} دیتی ہے جو  کھلونے پر  ، جو ساکن حالت سے لمحہ \عددی{t=0} پر  روانا ہوتا ہے، عمل کرتی ہے۔ محور \عددی{F_x} کا پیمانہ  \عددی{F_{xs}=\SI{5.0}{\newton}} تعین کرتی ہے۔ اکائی سمتی ترقیم میں (ا)  لمحہ \عددی{t=\SI{4.0}{\second}} اور (ب) \عددی{t=\SI{7.0}{\second}} پر \عددی{\vec{p}} کیا ہو گا، اور \عددی{t=\SI{9.0}{\second}} پر \عددی{\vec{v}} کیا ہو گی؟
\انتہا{سوال}
%-----------------------------
\ابتدا{سوال}
عین تصادم  سے قبل اور  عین تصادم کے بعد  \عددی{\SI{0.300}{\kilo\gram}}  گیند بلے سے ٹکراتا ہوا شکل \حوالہء{9.51} میں دکھایا گیا ہے۔  عین تصادم سے قبل گیند کی  سمتی رفتار  \عددی{\vec{v}_1} کی قدر \عددی{\SI{12.0}{\meter\per\second}} اور زاویہ \عددی{\theta_1=\SI{35.0}{\degree}} ہے۔ تصادم کے عین بعد گیند کی سمتی رفتار \عددی{\vec{v}_2} کی قدر \عددی{\SI{10.0}{\meter\per\second}} ہے اور یہ سیدھا اوپر رخ حرکت کرتا ہے۔ تصادم کا دورانیہ \عددی{\SI{2.00}{\milli\second}} ہے۔ گیند پر بلے کی ضرب  (ا)  کی قدر اور (ب)   مثبت \عددی{x} محور کے لحاظ سے رخ کیا ہیں؟ گیند پر بلے کی اوسط قوت  کی (ج) قدر اور (د) رخ کیا ہیں؟
\انتہا{سوال}
%-----------------------------
%Q34, p249
\ابتدا{سوال}
براعظم امریکہ کے  وسطی   اور جنوبی علاقوں میں \اصطلاح{ افعی چھپکلی }\فرہنگ{چھپکلی!افعی}\حاشیہب{basilisk lizard}\فرہنگ{lizard!basilisk} پائی جاتی ہے جو پانی کی سطح پر  پچھلی  دو  ٹانگوں  کی مدد سے دوڑ سکتی ہے۔  قدم لیتے ہوئے چھپکلی پہلے     زور سے    پانی کی سطح پر  پاوں سے تھپڑ مارتی ہے، اور اس کے بعد  پاوں  کو پانی میں   اس تیزی سے نیچے دھکیلتی ہے   کہ پاوں کے اوپر  ہوا کا غبارہ بن جاتا ہے۔اس سے قبل کہ ہوا کے غبارے میں  اطراف سے پانی بھر آئے  چھپکلی اسی پھرتی سے پاوں واپس اوپر کھینچ کر پانی کی  قوت  گھساٹ سے بچ پاتی ہے۔ ڈوبنے سے  بچنے کے لئے ضروری ہے کہ تھپڑ، نیچے دھکیل اور پاوں  واپس اٹھانے  کے دوران اوپری اوسط  ضرب،  تجاذبی  قوت کی  نشیب وار ضرب کے برابر ہو۔ فرض کریں افعی چھپکلی کی کمیت \عددی{\SI{90.0}{\gram}}، ہر  پاوں کی کمیت \عددی{\SI{3.00}{\gram}}،   تھپڑ کے وقت پاوں کی رفتار \عددی{\SI{1.50}{\meter\per\second}}، اور ایک قدم کا اوسط دورانیہ \عددی{\SI{0.600}{\second}} ہے۔ (ا) تھپڑ کے دوران  چھپکلی پر ضرب کی قدر کیا ہے؟ (تصور کریں یہ ضرب سیدھی اوپر رخ ہے۔) (ب) ایک قدم کے   \عددی{\SI{0.600}{\second}} دورانیہ میں تجاذبی قوت کی چھپکلی پر نشین وار ضرب کتنی ہے؟ (ج)   کیا چھپکلی کو سہارا تھپڑ دیتا ہے ، نیچے دھکیل دیتی ہے،  یا دونوں کا حصہ تقریباً برابر ہے؟
 
\انتہا{سوال}
%-------------------------------
%Q35, p 250
\ابتدا{سوال}
دیوار کے ساتھ \عددی{\SI{58}{\gram}} کمیت کا  گیند  ٹکراتا ہے۔ شکل \حوالہء{9.53} میں تصادم کی قوت کی قدر \عددی{F} بالمقابل وقت \عددی{t} ترسیم کی گئی ہے۔ گیند کی  ، دیوار کو عمودی ، ابتدائی رفتار   \عددی{\SI{34}{\meter\per\second}} ہے؛ گیند  ٹپکی کے بعد تقریباً اسی رفتار سے، دیوار کو عمودی،  واپس  لوٹتا ہے۔ تصادم کے دوران  گیند پر  دیوار کی  قوت کی زیادہ سے زیادہ قدر \عددی{F_{\text{\RL{بلندتر}}}} کیا ہو گی؟
\انتہا{سوال}
%----------------------------------
\ابتدا{سوال}
بلا رگڑ برفانی  سطح پر \عددی{\SI{0.25}{\kilo\gram}} قرص ساکن پڑا ہے۔ لمحہ \عددی{t=0} پر    \عددی{\vec{F}=(12.0-3.00t^2)\hat{i}} افقی قوت، جہاں قوت  نیوٹن میں اور وقت   سیکنڈ میں  ہے، قرص کو  حرکت دیتی ہے۔ قوت  کی قدر صفر ہونے تک  یہ قرص پر عمل کرتی ہے۔ (ا) لمحہ \عددی{t=\SI{0.500}{\second}}  اور \عددی{t=\SI{1.25}{\second}} کے بیچ قرص پر   قوت کی ضرب کی قدر کیا ہو گی؟ (ب) وقت \عددی{t=0} سے اس  لمحے تک جب \عددی{F=0} ہو، قرص کے معیار حرکت میں تبدیلی کیا ہو گی؟
\انتہا{سوال}
%-----------------------------------
\ابتدا{سوال}
کھلاڑی \عددی{\SI{0.45}{\kilo\gram}} گیند کو، جو ساکن ہے،  لات مارتا ہے۔ کھلاڑی کا پاوں گیند کے ساتھ \عددی{\SI{3.0e-3}{\second}} کے لئے مس ہے اور لات کی  قوت  درج ذیل ہے ، جہاں \عددی{0\le t\le \SI{3.0e-3}{\second}} اور  \عددی{t} سیکنڈوں میں ہے۔
\[F(t)=[(6.0\times 10^6)t-(2.0\times 10^9)t^2]\,\si{\newton}\]
تماس کے دوران (ا) لات  سے گیند پر ضرب کی قدر   ،  (ب) گیند پر   اوسط قوت کی قدر،    (ج) گیند پر  زیادہ سے زیادہ قوت  کی قدر    ، اور  (د) عین  اس لمحے گیند کی سمتی رفتار  کی قدر جس لمحے گیند لات سے علیحدہ ہوتا ہے  تلاش کریں۔
\انتہا{سوال}
%----------------------------------------
%Q38, p250
\ابتدا{سوال}
ایک گیند جس کی کمیت \عددی{\SI{300}{\gram}} اور رفتار \عددی{v=\SI{6.0}{\meter\per\second}} ہے ،دیوار کے ساتھ  زاویہ \عددی{\theta=\SI{30}{\degree}} سے   ٹکرا کر  اسی رفتار اور زاویہ سے ٹپکی کے بعد   واپس ہوتا  ہے۔ شکل \حوالہء{9.54} میں  فضائی جائزہ دکھایا گیا ہے۔ گیند اور دیوار آپس میں \عددی{\SI{10}{\milli\second}} کے لئے  مس   رہتے  ہیں۔ اکائی  سمتی ترقیم میں   (ا) گیند پر دیوار کی  ضرب اور (ب) دیوار پر گیند کی ضرب  کیا ہو گی، اور (ج)  دیوار پر گیند کی اوسط قوت کیا ہو گی؟
\انتہا{سوال}
%--------------------------------------
%Module 9-5, Conservation of Linear Momentum, p250
\جزوحصہء{خطی معیار حرکت کی بقا}
%----------------------------
%Q39
\ابتدا{سوال}
بلا رگڑ سطح پر  \عددی{\SI{91}{\kilo\gram}} کمیت کا لیٹا ہوا  شخص \عددی{\SI{68}{\gram}}   پتھر کو \عددی{\SI{4.0}{\meter\per\second}} رفتار سے  سطح پر روانا کرتا ہے۔ یہ شخص نتیجتاً کتنی رفتار حاصل کرتا ہے؟
\انتہا{سوال}
%--------------------------
\ابتدا{سوال}
زمین کے لحاظ سے \عددی{\SI{43000}{\kilo\meter\per\hour}} رفتار سے پرواز کرتا فضائی طیارہ استعمال شدہ  ہوائی بان موٹر (کمیت \عددی{4m})    کو قابو کار  مقیاسہ  (کمیت  \عددی{m})  سے علیحدہ کر کے   مقیاسہ کے لحاظ سے \عددی{\SI{82}{\kilo\meter\per\hour}} رفتار سے پیچھے  پھینکتا ہے۔علیحدگی کے فوراً بعد قابو کار مقیاسہ کی رفتار زمین کے لحاظ سے کیا ہو گی؟
\انتہا{سوال}
%------------------------
\ابتدا{سوال}
دو طرفہ ہوائی بان شکل \حوالہء{9.55} میں دکھایا گیا ہے، جس کا وسطی حصہ \عددی{C} (جس کی کمیت   \عددی{M=\SI{6.00}{\kilo\gram}} ہے)   اور اطرافی حصے \عددی{L} اور \عددی{R}  (جن کی   انفرادی  کمیت \عددی{\SI{2.00}{\kilo\gram}} ہے) ہیں۔ ہوائی بان بلا رگڑ فرش پر \عددی{x} محور کے مبدا پر ابتدائی طور ساکن پڑا ہے۔ چھوٹے دھماکوں  سے اطراف کے حصے علیحدہ کر کے \عددی{x} محور پر وسطی حصہ سے دور روانا کیے جا سکتے ہیں۔کچھ یوں کیا جاتا ہے: (1)  وقت \عددی{t=0} پر  حصہ \عددی{L} کو   ،   باقی حصے کو منتقل سمتی  رفتار کے لحاظ سے ، \عددی{\SI{3.00}{\meter\per\second}} رفتار سے بائیں پھینکا جاتا ہے۔ (2) اس کے بعد، وقت \عددی{t=\SI{0.80}{\second}} پر   حصہ \عددی{R} کو ، \عددی{\SI{3.00}{\meter\per\second}} رفتار سے ،  حصہ \عددی{C}  کو منتقل  سمتی رفتار کے لحاظ سے، دائیں پھینکا جاتا ہے۔ وقت \عددی{t=\SI{2.80}{\second}} پر (ا)  حصہ \عددی{C} کی سمتی رفتار کیا ہو گی  اور (ب)  اس کے مرکز کا مقام کیا ہو گا؟
\انتہا{سوال}
%--------------------------
%Q42 p250
\ابتدا{سوال}
  ایک جسم جس کی کمیت \عددی{m}  اور مشاہدہ کار کے لحاظ سے رفتار \عددی{v} ہے، دھماکے سے دو حصوں میں تقسیم ہوتا ہے، جہاں ایک ٹکڑے کی کمیت دوسرے ٹکڑے کی کمیت کی تین گنّا ہے؛ دھماکہ گہری فضا میں واقع ہوتا ہے جہاں تجاذبی قوت نہیں پایا جاتا۔ کم کمیتی ٹکڑا مشاہدہ کار کے لحاظ سے رک جاتا ہے۔ مشاہدہ کار کی حوالہ چھوکٹ  میں ناپتے ہوئے دھماکہ نظام کو کتنی حرکی توانائی منتقل کرتا ہے؟
\انتہا{سوال}
%---------------------------
\ابتدا{سوال}
زیادہ بلندی تک پہنچنے   کی غرض سے   ، عین  چھلانگ     سے قبل ، کھلاڑی دو    وزن  اوپر اٹھاتا اور چھلانگ کے بعد  ، پرواز کے دوران ، نیچے  زور سے  پھینکتا ہے۔  فرض کریں  ایک  کھلاڑی کی کمیت \عددی{\SI{78}{\kilo\gram}} اور   ایک وزن کی کمیت \عددی{\SI{5.50}{\kilo\gram}} ہے۔ یہ کھلاڑی بلند چھلانگ کی بجائے لمبی چھلانگ لگانا چاہتا ہے۔اس غرض سے چھلانگ کے دوران بلند ترین نقطہ پر پہنچ کر کھلاڑی وزن افقی یوں  پیچھے  پھینکتا ہے کہ زمین کے لحاظ سے ان کی افقی سمتی  رفتار صفر ہوتی ہے۔لمحہ  اٹھان پر کھلاڑی کی سمتی رفتار ، بغیر وزن اور بمع وزن دونوں صورتوں میں، \عددی{\vec{v}=(9.5\hat{i}+4.0\hat{j})\,\si{\meter\per\second}} ہے اور فرض کریں زمین ہم سطح ہے۔ وزن کا استعمال اس کو کتنا  اضافی فاصل طے کراتا ہے؟
\انتہا{سوال}
%------------------------
\ابتدا{سوال}
ساکن جسم دھماکے سے دو ٹکڑوں \عددی{R} اور \عددی{L} میں تقسیم ہوتا ہے، جو بلا رگڑ سطح  پر  گزرنے کے بعد  رگڑ کے خطوں میں داخل ہو کر آخر کار رکتے ہیں (شکل \حوالہء{9.57})۔ ٹکڑا \عددی{L}، جس کی کمیت \عددی{\SI{2.0}{\kilo\gram}} ، اور جس   کا  سامنا \عددی{\mu_L=0.40} حرکی رگڑ کے مستقل سے  ہے، \عددی{d_L=\SI{0.15}{\meter}} فاصلے میں رکتا ہے۔  ٹکڑا \عددی{R}، جس   کا  سامنا \عددی{\mu_R=0.50} حرکی رگڑ کے مستقل سے  ہے، \عددی{d_R=\SI{0.25}{\meter}} فاصلے میں رکتا ہے۔ اس ٹکڑے کی کمیت کیا ہے؟
\انتہا{سوال}
%-----------------------
\ابتدا{سوال}
ایک جسم جس کی  کمیت \عددی{\SI{20.0}{\kilo\gram}} ہے فضا میں \عددی{x} محور کے مثبت رخ \عددی{\SI{200}{\meter\per\second}} رفتار سے حرکت   کے دوران اندرونی دھماکے کی وجہ سے تین ٹکڑوں میں تقسیم ہوتا  ہے۔ ایک ٹکڑا جس کی کمیت \عددی{\SI{10.0}{\kilo\gram}} ہے  ، نقطہ دھماکہ سے مثبت  \عددی{y}   محور کے رخ   \عددی{\SI{100}{\meter\per\second}} رفتار سے روانا ہوتا ہے۔ دوسرا ٹکڑا ، جس کی کمیت \عددی{\SI{4.0}{\kilo\gram}} ہے، منفی \عددی{x} محور پر \عددی{\SI{500}{\meter\per\second}} سے روانا  ہوتا  ہے۔ (ا) اکائی سمتی ترقیم میں  تیسرے ٹکڑے کی سمتی رفتار تلاش کریں۔ (ب)  دھماکے میں کتنی توانائی رہا ہوتی ہے؟ تجاذبی قوت کے  اثرات نظرانداز کریں۔
\انتہا{سوال}
%-----------------------------
\ابتدا{سوال}
ایک  جسم ، جس کی کمیت \عددی{\SI{4.0}{\kilo\gram}} ہے ، بلا رگڑ سطح پر  حرکت کرتے ہوئے دھماکے سے دو \عددی{\SI{2.0}{\kilo\gram}} ٹکڑوں میں تقسیم ہوتا ہے۔ ایک ٹکڑا \عددی{\SI{3.0}{\meter\per\second}}  شمال  کو اور دوسرا \عددی{\SI{5.0}{\meter\per\second}} مشرق سے \عددی{\SI{30}{\degree}} شمال  کی طرف روانا ہوتا ہے۔ جسم کی ابتدائی رفتار کیا ہے؟
\انتہا{سوال}
%--------------------------
%Q47, p251
\ابتدا{سوال}
ایک جسم جو \عددی{xy} محددی نظام کے مبدا پر ساکن پڑا ہے دھماکے سے تین ٹکڑوں میں تقسیم ہوتا ہے۔عین  دھماکے  کے بعد ایک ٹکڑا، جس کی کمیت \عددی{m} ہے، \عددی{(\SI{-30}{\meter\per\second})\hat{i}} سمتی  رفتار سے  اور دوسرا ٹکڑا، جس کی کمیت   بھی \عددی{m} ہے،  \عددی{(\SI{-30}{\meter\per\second})\hat{j}} سمتی رفتار سے  حرکت کرتے ہیں۔ تیسرے ٹکڑے کی کمیت \عددی{3m} ہے۔ عین دھماکے کے بعد تیسرے ٹکڑے کی سمتی رفتار کی (ا) قدر  کیا ہو گی اور (ب) رخ کیا ہو گا؟
\انتہا{سوال}
%----------------------------
\ابتدا{سوال}
ذرہ \عددی{A} اور ذرہ \عددی{B}   جن کے بیچ دبا ہوا اسپرنگ ہے کو زبردستی اکٹھے   پکڑ کر رکھا گیا ہے۔رہا کرنے پر  اسپرنگ انہیں مخالف رخوں  دھکیل کر ان سے علیحدہ ہوتا ہے۔ ذرہ \عددی{A} کی کمیت ذرہ \عددی{B} کی کمیت کی \عددی{2.00} گنّا ہے، اور دبے   اسپرنگ میں ذخیرہ مخفی توانائی \عددی{\SI{60}{\joule}} ہے۔ فرض کریں اسپرنگ کی کمیت قابل نظر انداز ہے اور اس کی توانائی مکمل طور پر ذروں کو منتقل ہوتی ہے۔توانائی کا  انتقال  مکمل ہونے پر (ا) ذرہ \عددی{A} اور (ب) ذرہ \عددی{B} کی حرکی توانائی کیا ہو گی؟
\انتہا{سوال}
%------------------------------
%Module 9-6 Momentum and Kinetic Energy in Collisions
\جزوحصہء{معیار حرکت اور تصادم میں حرکی توانائی}
%Q49, p251
\ابتدا{سوال}
منجنیقی رقاص جس کی کمیت \عددی{\SI{2.0}{\kilo\gram}} ہے، پر \عددی{\SI{10}{\gram}} گولی  چلائی جاتی ہے۔ رقاص کا مرکز کمیت \عددی{\SI{12}{\centi\meter}}  بلندی  تک پہنچتا ہے۔ فرض کریں گولی رقاص میں دھنس جاتی ہے۔گولی کی ابتدائی رفتار کیا ہے؟
\انتہا{سوال}
%---------------------------
\ابتدا{سوال}
بلا رگڑ فرش پر  لکڑی کا تختہ جس کی کمیت \عددی{\SI{700}{\gram}} ہے ساکن  پڑا  ہے۔اس پر \عددی{\SI{5.20}{\gram}}  گولی چلائی جاتی ہے جو \عددی{\SI{672}{\meter\per\second}} سے حرکت کرتے ہوئے تختہ کو مار کر اس  سے پار \عددی{\SI{428}{\meter\per\second}} رفتار سے   خارج ہوتی ہے۔ (ا) تختے  کی رفتار کیا ہو گی؟ (ب)  تختہ و گولی نظام کے مرکز کمیت کی رفتار کیا ہو گی؟
\انتہا{سوال}
%----------------------------
\ابتدا{سوال}
بلا رگڑ فرش پر پڑے  دو ساکن  جسم پر \عددی{\SI{3.50}{\gram}} گولی افقی ماری جاتی ہے (شکل \حوالہء{9.58}-الف)۔ گولی جسم \عددی{1}، جس کی کمیت \عددی{\SI{1.20}{\kilo\gram}} ہے، سے گزر کر دوسرے جسم، جس کی کمیت \عددی{\SI{1.80}{\kilo\gram}} ہے، میں دھنس جاتی ہے جس کی بدولت  جسم \عددی{1} کی رفتار \عددی{v_1=\SI{0.630}{\meter\per\second}} اور جسم \عددی{2} کی رفتار  \عددی{v_2=\SI{1.40}{\meter\per\second}} حاصل کرتے ہیں (شکل \حوالہء{9.58}-ب)۔ جسم \عددی{1} سے نکالا گیا مواد نظر انداز کرتے ہوئے، گولہ کی رفتار اس لمحے تلاش کریں جب یہ جسم \عددی{1} سے (ا) نکلتی اور (ب) داخلی ہوتی ہے۔
\انتہا{سوال}
%----------------------
%Q52, p251
\ابتدا{سوال}
ایک گولی جس کی کمیت \عددی{\SI{10}{\gram}} ہے  سیدھا اوپر \عددی{\SI{1000}{\meter\per\second}} رفتار سے  حرکت  کرتے ہوئے  ابتدائی طور ساکن  \عددی{\SI{5.0}{\kilo\gram}} سل کے مرکز کمیت سے گزرتی ہے۔گولی سل سے گزر کر \عددی{\SI{400}{\meter\per\second}} رفتار سے خارج ہو کر اوپر وار حرکت کرتی ہے۔سل ابتدائی مقام سے کتنی بلندی تک اٹھتا ہے؟
\انتہا{سوال}
%--------------------
\ابتدا{سوال}
ایلاسکا  میں گاڑی اور  بارہ سنگا  کے تصادم عام بات ہے۔ فرض کریں \عددی{\SI{1000}{\kilo\gram}} گاڑی  \عددی{\SI{500}{\kilo\gram}} ساکن  بارہ سنگا سے ٹکراتی ہے۔ (ا)  حرکی توانائی کا کتنا فی صد حصہ توانائی   کے دیگر  صورتوں میں تبدیل ہو گا؟ اس قسم کا مسئلہ  عرب  ممالک میں پایا جاتا ہے جہاں گاڑی اور اونٹ کا ٹکر عام ہے۔ (ب) اگر یہی گاڑی اونٹ کے  ساکن بچے سے ٹکرائے  جس کی کمیت  \عددی{\SI{300}{\kilo\gram}}  ہے تب  کتنی فی صد  حرکی توانائی  ضائع ہو گی؟ (ج) کیا جانور کی کمیت بڑھنے سے  فی صد توانائی کا ضیاع بڑھتا ہے یا گھٹتا ہے؟
\انتہا{سوال}
%------------------------
\ابتدا{سوال}
انتصابی محور پر مخالف رخ حرکت کرتے  لبدی  کے دو گولوں کے بیچ مکمل غیر لچکی تصادم ہوتا ہے۔عین تصادم سے قبل ایک گولا ، جس کی کمیت  \عددی{\SI{3.0}{\kilo\gram}} ہے، \عددی{\SI{20}{\meter\per\second}}   اوپر وار  اور دوسرا گولا، جس کی کمیت \عددی{\SI{2.0}{\kilo\gram}} ہے، \عددی{\SI{12}{\meter\per\second}} سے  نشیب وار حرکت کرتا ہے۔ نقطہ تصادم سے دونوں گولوں  کا مجموعہ کتنی بلندی تک پہنچتا ہے؟(ہوائی رگڑ نظر انداز کریں۔)
\انتہا{سوال}
%---------------------
\ابتدا{سوال}
ایک سل جس کی کمیت \عددی{\SI{5.0}{\kilo\gram}} اور رفتار \عددی{\SI{3.0}{\meter\per\second}}  دوسری  سل جس کی کمیت \عددی{\SI{10}{\kilo\gram}} اور اسی رخ رفتار \عددی{\SI{2.0}{\meter\per\second}} ہے سے ٹکراتا ہے۔تصادم کے بعد \عددی{\SI{10}{\kilo\gram}} سل اسی رخ \عددی{\SI{2.5}{\meter\per\second}} سے حرکت کرتی ہے۔ (ا)  تصادم کے بعد دوسری سل کی رفتار کیا ہو گی؟ (ب)  تصادم کی وجہ سے دو  سل نظام کی کل  حرکی توانائی میں  کتنی تبدیلی رونما ہوتی ہے؟ (ج) اس کے برعکس، اگر \عددی{\SI{10}{\kilo\gram}}  سل  کی اسی رخ رفتار \عددی{\SI{4.0}{\meter\per\second}} ہو، تب کل حرکی توانائی میں تبدیلی کتنی ہو گی؟ (د)  جزو ج میں حاصل جواب کی وجہ پیش کریں۔
\انتہا{سوال}
%----------------------
\ابتدا{سوال}
سرخ اشارے پر کھڑی گاڑی   \عددی{A}   (کمیت \عددی{\SI{1100}{\kilo\gram}}) کو پیچھے سے گاڑی \عددی{B} (کمیت \عددی{\SI{1400}{\kilo\gram}}) ٹکر مارتی ہے (شکل \حوالہء{9.60})۔ دونوں گاڑی  نم سڑک پر (جس کی \عددی{\mu_k=0.13} کافی کم ہے)   پھسل  کر آخر کار
  \عددی{d_A=\SI{8.2}{\meter}}
   اور 
   \عددی{d_B=\SI{6.1}{\meter}}
    فاصلے طے کرنے کے بعد رکتی  ہیں۔ عین  تصادم  کے بعد  (ا) گاڑی \عددی{A} اور (ب) گاڑی  \عددی{B} کی رفتار کیا ہے؟ (ج)  فرض کریں تصادم کے دوران خطی معیار حرکت کی بقا ہوتی ہے۔ عین تصادم سے قبل گاڑی \عددی{B}  کی رفتار کیا  ہو گی؟ (د)  بتائیں یہ مفروضہ کیوں غلط ہو سکتا ہے۔
\انتہا{سوال}
%-----------------------
%Q57, p251
\ابتدا{سوال}
بلا رگڑ فرش پر  ساکن  اسپرنگ بندوق، جس کی کمیت \عددی{M=\SI{240}{\gram}} ہے،  کی نالی میں \عددی{m=\SI{60}{\gram}} گیند \عددی{v_i=\SI{22}{\meter\per\second}} رفتار سے  پھینکی جاتی ہے (شکل \حوالہء{9.61})۔ گیند نالی میں اس مقام پر اڑ جاتا ہے جہاں اسپرنگ زیادہ سے زیادہ دبا ہے۔ گیند اور نالی کے بیچ رگڑ کی بنا حر توانائی میں اضافہ قابل نظرانداز ہے۔ (ا) اس لمحے  بندوق کی رفتار کیا ہو گی جب گیند نالی میں رکتا ہے؟ (ب)  گیند کی ابتدائی حرکی توانائی کا کتنا حصہ اسپرنگ میں ذخیرہ ہو گا؟
\انتہا{سوال}
%---------------------
\ابتدا{سوال}
سل \عددی{2} جس کی کمیت \عددی{\SI{1.0}{\kilo\gram}} ہے بلا رگڑ فرش پر ساکن ڈھیلے  اسپرنگ (جس کا مقیاس لچک \عددی{\SI{200}{\newton\per\meter}} ہے) کے ایک سر کے ساتھ مس  ہے۔اسپرنگ کا  دوسرا سر دیوار کے ساتھ پکا جڑا ے۔ سل \عددی{1}  ، جس کی کمیت \عددی{\SI{2.0}{\kilo\gram}} ہے، \عددی{v_2=\SI{4.0}{\meter\per\second}} رفتار سے سل \عددی{2} سے ٹکرا کر اس کے ساتھ جڑ جاتا ہے۔جب  سل  لمحاتی رکتے ہیں، اس لمحے اسپرنگ کتنا دبا ہو گا؟
\انتہا{سوال}
%-----------------------
%Q59, p252
\ابتدا{سوال}
سل \عددی{1}(کمیت \عددی{\SI{2.0}{\kilo\gram}})   دائیں رخ \عددی{\SI{10}{\meter\per\second}} اور سل \عددی{2}(کمیت \عددی{\SI{5.0}{\kilo\gram}})   دائیں رخ \عددی{\SI{3}{\meter\per\second}} حرکت میں ہیں (شکل \حوالہء{9.63})۔ فرش بلا رگڑ ہے اور سل \عددی{2} کے ساتھ اسپرنگ پکا   جڑا ہے جس کی اسپرنگ مستقل \عددی{\SI{1120}{\newton\per\meter}} ہے۔ تصادم کے دوران اسپرنگ کا داب  اس وقت زیادہ سے زیادہ ہو گا جب  دونوں سل کی سمتی  رفتار ایک ہو۔ زیادہ سے زیادہ  داب تلاش کریں۔
\انتہا{سوال}
%--------------------
%Module 9-7 Elastic Collisions in One Dimension, p252
\جزوحصہء{یک بعدی لچکی تصادم }
%Q60
\ابتدا{سوال}
بلا رگڑ فرش پر  سل \عددی{A} (کمیت \عددی{\SI{1.6}{\kilo\gram}})  حرکت کرتا ہوا سل  \عددی{B}  (کمیت \عددی{\SI{2.4}{\kilo\gram}}) سے ٹکراتا ہے (شکل \حوالہء{9.64})۔ تصادم سے قبل تین سمتی رفتار  \عددی{(i)} اور تصادم کے بعد تین سمتی رفتار \عددی{(f)} بھی پیش ہیں؛ مطابقتی رفتار \عددی{v_{Ai}=\SI{5.5}{\meter\per\second}}، \عددی{v_{Bi}=\SI{2.5}{\meter\per\second}}، اور \عددی{v_{Bf}=\SI{4.9}{\meter\per\second}} ہیں۔  سمتی رفتار \عددی{\vec{v}_{Af}} کی (ا) رفتار  اور (ب) رخ (دائیں یا بائیں) کیا ہیں؟ (ج) کیا تصادم لچکی ہے؟
\انتہا{سوال}
%---------------------------
\ابتدا{سوال}
بلا رگڑ   خطی ہوائی ڈگر  پر \عددی{\SI{340}{\gram}} ریڑھی  \عددی{\SI{1.2}{\meter\per\second}} ابتدائی رفتار سے چل کر نامعلوم  کمیت  کی  ساکن ریڑھی  سے ٹکراتی ہے۔ تصادم کے بعد پہلی ریڑھی رخ برقرار رکھ کر \عددی{\SI{0.66}{\meter\per\second}} سے حرکت کرتی ہے۔ (ا) دوسری ریڑھی کی کمیت کیا ہے؟ (ب) تصادم کے بعد اس کی رفتار کیا ہو گی؟ (ج)  دو ریڑھی نظام کے مرکز کمیت کی رفتار کیا ہو گی؟
\انتہا{سوال}
%---------------------------
\ابتدا{سوال}
\اصطلاح{طیطانیم }\فرہنگ{طیطانیم}\حاشیہب{titanium}\فرہنگ{titanium} کے دو کرہ  ایک رفتار سے چل کر آمنے سامنے سے لچکی تصادم کا شکار ہوتے ہیں۔ تصادم کے بعد ایک کرہ، جس کی کمیت \عددی{\SI{300}{\gram}} ہے، رک جاتا ہے۔ (ا) دوسرے کرہ کی کمیت کیا ہے؟ (ب)  اگر دونوں کرہ  کی ابتدائی رفتار \عددی{\SI{2.00}{\meter\per\second}} ہو، دو کرہ  نظام کے مرکز کمیت کی رفتار کیا ہو گی؟
\انتہا{سوال}
%-------------------------
%Q63 p252
\ابتدا{سوال}
بلا رگڑ فرش پر  \عددی{m_1} کمیت کی سل چل کر   ساکن سل، جس کی کمیت \عددی{m_2=3m_1} ہے ، سے  یک بعدی لچکی تصادم میں مبتلا ہوتی ہے۔ تصادم سے قبل دو جسمی نظام کے مرکز کمیت کی رفتار 
\عددی{\SI{3.00}{\meter\per\second}} ہے۔ تصادم کے بعد  (ا) مرکز کمیت اور (ب) سل \عددی{2} کی رفتار کیا ہو گی؟
\انتہا{سوال}
%---------------------------
\ابتدا{سوال}
 کمیت \عددی{\SI{0.500}{\kilo\gram}} کا فولادی گیند  \عددی{\SI{70.0}{\centi\meter}} ڈور  سے لٹک رہا ہے (شکل \حوالہء{9.65})۔گیند کو ایک جانب  اٹھایا جاتا ہے اور جب ڈور افقی ہو اسے رہا کیا جاتا ہے۔نچلے ترین نقطہ پر پہنچ کر یہ \عددی{\SI{2.5}{\kilo\gram}} کی فولادی سل سے ٹکراتا ہے جو بلا رگڑ فرش پر ساکن پڑا ہے۔ تصادم لچکی ہے۔ عین تصادم کے بعد (ا) گیند کی رفتار اور (ب)  سل کی رفتار تلاش کریں۔
\انتہا{سوال}
%-------------------------
\ابتدا{سوال}
ایک جسم،  جس کی کمیت \عددی{\SI{2.0}{\kilo\gram}} ہے،  دوسرے ساکن جسم سے لچکی ٹکر کے بعد ،  رخ برقرار رکھ کر ، ایک چوتھائی  رفتار سے  حرکت کرتا ہے۔ (ا) دوسرے جسم کی کمیت تلاش کریں۔ (ب)  اگر \عددی{\SI{2.0}{\kilo\gram}}  کی ابتدائی رفتار \عددی{\SI{4.0}{\meter\per\second}} ہو، دو جسمی نظام کے مرکز کمیت کی رفتار کیا ہو گی؟
\انتہا{سوال}
%-------------------------
%Q66 p252
\ابتدا{سوال}
بلا رگڑ فرش پر \عددی{m_1} کمیت کا سل \عددی{1} محور \عددی{x} پر \عددی{\SI{4.0}{\meter\per\second}} سے حرکت کرتے ہوئے  سل \عددی{2} ، جو ساکن ہے اور جس کی کمیت \عددی{m_2=0.40m_1} ہے ، سے یک بعدی لچکی تصادم کرتی ہے۔اس کے بعد اجسام پھسلتے ہوئے  ایسے خطہ میں داخل ہو کر آخر کار رکتے ہیں جہاں  حرکی رگڑ کا  عددی سر  \عددی{0.50} ہے۔ اس خطہ میں (ا)  سل \عددی{1} اور (ب)  سل\عددی{2} کتنا فاصلہ طے کرتی ہے؟
\انتہا{سوال}
%-------------------------------
\ابتدا{سوال}
بلا رگڑ فرش پر  ذرہ \عددی{1} جس کی کمیت \عددی{m_1=\SI{0.30}{\kilo\gram}} ہے  دائیں رخ محور \عددی{x}   پر   \عددی{\SI{2.0}{\meter\per\second}} رفتار سے پھسل  کر چلتا ہے  ( شکل \حوالہء{9.66} )۔  نقطہ \عددی{x=0} پر اس کا یک بعدی لچکی تصادم ذرہ \عددی{2} کے ساتھ ہوتا ہے، جو ساکن ہے اور جس کی کمیت \عددی{m_2=\SI{0.40}{\kilo\gram}} ہے۔ تصادم کے بعد ذرہ \عددی{2}  دیوار سے،  جو\عددی{x_w=\SI{70}{\centi\meter}} پر واقع ہے،   ٹپکی کھا کر  ، رفتار  میں تبدیلی کے بغیر ، واپس لوٹتا ہے۔ محور \عددی{x} پر ذروں کا  آپس میں دوسرا تصادم کس نقطہ پر ہو گا؟
\انتہا{سوال}
%----------------------
\ابتدا{سوال}
سل \عددی{1}، جس کی کمیت \عددی{m_1} ہے، ساکن حالت سے میلان پر \عددی{h=\SI{2.50}{\meter}} بلندی سے روانا ہو کر ساکن  سل \عددی{2}  کے ساتھ ٹکراتی ہے، جس کی کمیت \عددی{m_2=2.00m_1} ہے(شکل \حوالہء{9.67})۔ تصادم کے بعد سل \عددی{2} ایسے خطہ میں داخل ہو کر، جہاں حرکی رگڑ کا عددی سر \عددی{\mu_k=0.500} ہے،  فاصلہ \عددی{d} طے کرنے کے بعد رکتی ہے۔ (ا) لچکی اور (ب)مکمل  غیر لچکی  تصادم کی صورت میں \عددی{d} کی قیمت تلاش کریں۔
\انتہا{سوال}
%---------------------------------
%Q69, p252
\ابتدا{سوال}
چھوٹے گیند کو بڑے گیند کے ٹھیک اوپر  معمولی  بلندی پر  رکھ کر دونوں کو بیکوقت \عددی{h=\SI{1.8}{\meter}} بلندی سے  گرنے دیا جاتا ہے (گیندوں کے رداس  \عددی{h} کے لحاظ سے قابل نظرانداز ہیں)۔ ان کی کمیت بالترتیب \عددی{m} اور \عددی{M=\SI{0.63}{\kilo\gram}} ہے (شکل \حوالہء{9.68}-الف)۔ (ا) اگر   بڑا گیند زمین سے لچکی ٹپکی کھائے اور  اس کے بعد چھوٹا گیند  بڑے گیند سے لچکی ٹپکی کھائے ، تو چھوٹے گیند کی کمیت  \عددی{m}کتنی  ہونی چاہیے کہ بڑا گیند  چھوٹے گیند سے ٹکرا کر رک جائے؟ (ب)ایسی صورت میں   چھوٹا گیند  کتنی بلندی تک جائے گا  (شکل \حوالہء{9.68}-ب)؟
\انتہا{سوال}
%---------------------
\ابتدا{سوال}
قرص \عددی{1}، جس کی کمیت \عددی{m_1=\SI{0.20}{\kilo\gram}} ہے،  بلا رگڑ میز پر پھسلتا ہوا  ساکن قرص \عددی{2} سے یک بعدی لچکی تصادم کا شکار ہوتا  ہے (شکل \حوالہء{9.69})۔ قرص \عددی{2} میز سے زمین پر کنارہ سے \عددی{d} فاصلہ دور گرتا ہے۔ قرص \عددی{1}   تصادم کے بعد واپس  ہو کر  میز کے مخالف کنارے سے \عددی{2d} فاصلہ دور زمین پر گرتا ہے۔ قرص \عددی{2} کی کمیت کیا ہے؟ (اشارہ: علامتوں پر نظر رکھیں۔)
\انتہا{سوال}
%------------------
%Module 9-8 Collisions in Two Dimensions, p253
\جزوحصہء{دو ابعاد میں تصادم}
%Q71, p253

