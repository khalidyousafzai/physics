%starting from top of P239
%ch 9 "Elastic Collision In One Dimension"
\باب{مرکز کمیت اور خطی معیار حرکت}
\حصہ{ایک بعد میں  لچکی تصادم}
حرکی توانائی کی بقا درج ذیل لکھی جائے گی۔
\begin{align}\label{مساوات_مرکز_کمیت_حرکی_توانائی_کی_بقا}
\frac{1}{2}m_1v_{1i}^2+\frac{1}{2}m_2v_{2i}^2=\frac{1}{2}m_1v_{1f}^2+\frac{1}{2}m_2v_{2f}^2
\end{align}
ان ہمزاد  مساوات کو \عددی{v_{1f}} اور \عددی{v_{2f}} کے لئے حل کرنے کی خاطر  ہم مساوات \حوالہء{9.71} کو
\begin{align}\label{مساوات_مرکز_کمیت_بقا_معار_دوم}
m_1(v_{1i}-v_{1f})=-m_2(v_{2i}-v_{2f})
\end{align}
اور مساوات \حوالہ{مساوات_مرکز_کمیت_حرکی_توانائی_کی_بقا} درج ذیل صورت میں لکھتے ہیں۔
\begin{align}\label{مساوات_مرکز_کمیت_بقا_دوم}
m_1(v_{1i}-v_{1f})(v_{1i}+v_{1f})=-m_2(v_{2i}-v_{2f})(v_{2i}+v_{2f})
\end{align}
مساوات \حوالہ{مساوات_مرکز_کمیت_بقا_دوم} کو مساوات \حوالہ{مساوات_مرکز_کمیت_بقا_معار_دوم} سے تقسیم کرنے  کے بعد کچھ الجبرا کے بعد درج ذیل حاصل ہوں گے۔
\begin{align}\label{مساوات_مرکز_کمیت_اختتامی_الف}
v_{1f}=\frac{m_1-m_2}{m_1+m_2}v_{1i}+\frac{2m_2}{m_1+m_2}v_{2i}
\end{align}
اور
\begin{align}\label{مساوات_مرکز_کمیت_اختتامی_ب}
v_{2f}=\frac{2m_1}{m_1+m_2}v_{1i}+\frac{m_2-m_1}{m_1+m_2}v_{2i}
\end{align}
یاد رہے، زیر نوشت \عددی{1} اور \عددی{2} کسی خاص ترتیب سے مختص نہیں کیے گئے۔  مساوات \حوالہء{9.19} میں  اور مساوات \حوالہ{مساوات_مرکز_کمیت_اختتامی_الف} اور مساوات  \حوالہ{مساوات_مرکز_کمیت_اختتامی_ب} میں  ان زیر نوشت کو آپس میں بدل کر لکھنے  مساوات کی وہی جوڑی ملتی ہے۔ اس پر بھی توجہ دیں کہ \عددی{v_{2i}=0}  لینے سے، شکل \حوالہء{9.18} میں جسم \عددی{2} ساکن ہدف ہو گا، اور مساوات \حوالہ{مساوات_مرکز_کمیت_اختتامی_الف}  اور مساوات \حوالہ{مساوات_مرکز_کمیت_اختتامی_ب} ہمیں  بالترتیب مساوات \حوالہء{9.67} اور مساوات \حوالہء{9.68} دیتی ہیں۔ 



\ابتدا{آزمائش}
شکل \حوالہء{9.18} میں گولے کا ابتدائی معیار حرکت \عددی{\SI{6}{\kilo\gram\meter\per\second}} اور اختتامی معیار حرکت (ا)  \عددی{\SI{2}{\kilo\gram\meter\per\second}} اور (ب) \عددی{\SI{-2}{\kilo\gram \meter\per\second}} ہونے کی صورت میں   ہدف کا  اختتامی خطی معیار حرکت کیا ہو گا؟ اگر گولے کی  ابتدائی اور  اختتامی حرکی توانائی بالترتیب  \عددی{\SI{5}{\joule}} اور \عددی{\SI{2}{\joule}} ہو، ہدف کی اختتامی حرکی توانائی کیا ہو گی؟
\انتہا{آزمائش}
%----------------------

\ابتدا{نمونی سوال} \quad \موٹا{لچکی تصادم در لچکی تصادم}
شکل \حوالہء{9.20a} میں \عددی{v_{1i}=\SI{10}{\meter\per\second}} سے چلتا ہوا سل 1 دو ساکن سلوں کی طرف بڑھتا  ہے۔تینوں سل ایک لکیر پر  ہیں۔ یہ سل  \عددی{2} سے ٹکراتا ہے جو آگے سل \عددی{3} سے  جا کر ٹکراتا ہے، جس کی کمیت \عددی{m_3=\SI{6.0}{\kilo\gram}} ہے۔ دوسرے  تصادم  کے بعد سل \عددی{2} دوبارہ ساکن ہے،  اور سل \عددی{3} کی رفتار  \عددی{v_{3f}=\SI{5.0}{\meter\per\second}} ہے (شکل \حوالہء{9.20b})۔ دونوں تصادم لچکی ہیں۔ سل \عددی{1} اور سل \عددی{2} کی  کمیتیں کیا ہیں؟ سل \عددی{1} کی اختتامی رفتار
 \عددی{v_{1f}} کیا ہے؟
 
 \موٹا{کلیدی تصورات}
 
 چونکہ ہم تصادم لچکدار تصور کرتے ہیں لہٰذا میکانی توانائی کی  بقا ہو گی (یوں  ٹکر کی آواز، گرمی، اور ارتعاش کی بدولت توانائی کا  ضیاع نظر انداز کیا جاتا ہے)۔ کوئی  بیرونی افقی قوت  سلوں پر عمل نہیں کرتی لہٰذا محور \عددی{x} پر خطی معیار حرکت کی بقا ہو گی۔ ان دو وجوہات کی بنا پر ہم دونوں تصادم پر  مساوات \حوالہء{9.67} اور مساوات \حوالہء{9.68} کا اطلاق کر سکتے ہیں۔
 
 \موٹا{حساب}\quad
 پہلے  تصادم  سے آغاز کرتے ہوئے ہمیں  اتنے زیادہ  نا معلوم متغیرات سے واسطہ ہو گا کہ آگے بڑھنا  مشکل ہو گا: ہم سلوں کی کمیت اور اختتامی سمتی رفتار نہیں جانتے۔ آئیں پہلے تصادم سے آغاز کریں، جس میں سل \عددی{3} کے ساتھ ٹکرانے کے بعد سل \عددی{2} رکتی ہے۔ مساوات \حوالہء{9.67} کا  اطلاق  اس تصادم پر کرتے ہیں جہاں ترقیم تبدیل کرتے ہوئے \عددی{v_{2i}}  تصادم سے قبل سل \عددی{2} کی  رفتار اور \عددی{v_{2f}}  تصادم کے بعد اس کی رفتار  دیتی ہیں۔یوں درج ذیل ہو گا۔
 \begin{align*}
 v_{2f}=\frac{m_2-m_3}{m_2+m_3}v_{2i}
 \end{align*}
 اس میں \عددی{v_{2f}=0} (سل \عددی{2}  رک جاتا ہے) ڈالنے کے بعد  \عددی{m_3=\SI{6.0}{\kilo\gram}}  ڈال کر درج ذیل حاصل ہو گا۔
 \begin{align*}
 m_2&=m_3=\SI{6.0}{\kilo\gram}&&\text{\RL{(جواب)}}
 \end{align*}
اسی طرح  ترقیم  تبدیل کر کے دوسرے تصادم کے لئے  مساوات \حوالہء{9.68} لکھتے ہیں
\begin{align*}
v_{3f}=\frac{2m_2}{m_2+m_3}v_{2i}
\end{align*}
جہاں \عددی{v_{3f}} تیسرے سل کی اختتامی سمتی رفتار ہے۔ اس میں \عددی{m_2=m_3} ڈالنے کے بعد \عددی{v_{3f}=\SI{5.0}{\meter\per\second}} ڈال کر درج ذیل حاصل ہو گا۔
\begin{align*}
v_{2i}=v_{3f}=\SI{5.0}{\meter\per\second}
\end{align*}

%p240

آئیں اب پہلے تصادم پر غور کریں؛ ہمیں سل \عددی{2} کے لئے مستعمل ترقیم پر توجہ دینی ہو گی: تصادم کے بعد سل \عددی{2}  کی سمتی رفتار \عددی{v_{2f}} وہی ہے جو تصادم سے قبل  اس کی سمتی رفتار \عددی{v_{2i}=\SI{5.0}{\meter\per\second}} تھی۔ پہلے تصادم پر مساوات \حوالہء{68} کا اطلاق  کر کے دی گئی  \عددی{v_{1i}=\SI{10}{\meter\per\second}}
ڈال کر   ذیل    ہو گا
\begin{align*}
v_{2f}&=\frac{2m_1}{m_1+m_2}v_{1i}\\
\SI{5.0}{\meter\per\second}&=\frac{2m_1}{m_1+m_2}(\SI{10}{\meter\per\second})
\end{align*}
جو ذیل دیگا۔
\begin{align*}
m_1&=\frac{1}{3}m_2=\frac{1}{3}(\SI{6.0}{\kilo\gram})=\SI{2.0}{\kilo\gram}&&\text{\RL{(جواب)}}
\end{align*}
یہ  نتیجہ اور دی گئی \عددی{v_{1i}}  استعمال کرتے ہوئے  پہلے تصادم  پر مساوات \حوالہء{9.67} کا اطلاق کر کے درج ذیل لکھا جا سکتا ہے۔
\begin{align*}
v_{1f}&=\frac{m_1-m_2}{m_1+m_2}v_{1i}\\
&=\frac{\tfrac{1}{3}m_2-m_2}{\tfrac{1}{3}m_2+m_2}(\SI{10}{\meter\per\second})=\SI{-5.0}{\meter\per\second}&&\text{\RL{(جواب)}}
\end{align*}
\انتہا{نمونی سوال}
%--------------------

\حصہ{دو ابعاد میں تصادم}
\جزوحصہء{مقاصد}
اس حصہ کو پڑھنے کے بعد آپ درج ذیل کے قابل ہوں گے۔

جدا نظام کے لئے جس میں دو بعدی تصادم  واقع ہو ، ہر ایک محور پر   معیار حرکت کی بقا  کا اطلاق کرتے ہوئے  ، تصادم  کے بعد محور پر معیار حرکت کے  اجزاء  کا   اسی محور پر تصادم سے قبل معیار حرکت کے   اجزاء کے ساتھ رشتہ جان سکیں۔

جدا نظام کے لئے جس میں دو بعدی لچکی تصادم واقع ہو، (ا)  ، ہر ایک محور پر   معیار حرکت کی بقا  کا اطلاق کرتے ہوئے  ، تصادم  کے بعد محور پر معیار حرکت کے  اجزاء  کا   اسی محور پر تصادم سے قبل معیار حرکت کے   اجزاء کے ساتھ رشتہ جان سکیں اور (ب)کل   حرکی توانائی  کی بقا کا اطلاق کر کے تصادم سے قبل اور تصادم کے بعد حرکی توانائیوں کا رشتہ جان سکیں۔

\جزوحصہء{کلیدی  تصور}
اگر دو جسم ٹکرائیں اور ان کی حرکت ایک محور پر نہ ہو (تصادم  آمنے سامنے سے  نہیں ہے)، تصادم دو بعدی ہو گا۔ اگر دو جسمی نظام بند اور جدا ہو،تصادم پر  معیار حرکت کی بقا کے   قانون کا اطلاق ہو گا لہٰذا درج  ہو گا۔
\begin{align*}
\vec{P}_{1i}+\vec{P}_{2i}=\vec{P}_{1f}+\vec{P}_{2f}
\end{align*}
یہ قانون اجزاء کی صورت میں دو مساوات   (ہر  بعد کے لئے ایک مساوات) دیگا جو تصادم کو بیان کرتی ہیں۔ اگر تصادم لچکی بھی ہو (جو  ایک خصوصی صورت ہے)، تصادم کے دوران حرکی توانائی کی بقا (ذیل)  تیسری مساوات دیگی۔
\begin{align*}
K_{1i}+K_{2i}=K_{1f}+K_{2f}
\end{align*}

\حصہء{دو بعد میں تصادم}
جب دو اجسام کا تصادم  ہو، اجسام کس  رخ حرکت    کرتے ہیں ، اس کا تعین ان کے بیچ ضرب (جھٹکا ) کرتی ہے۔ بالخصوص، جب تصادم آمنے سامنے سے نہ ہو، اجسام اپنے اپنے   ابتدائی محور پر نہیں رہتے۔ ایسے دو بعدی تصادم میں  جو بند، اور جدا نظام میں واقع ہو،  کل خطی معیار حرکت کی بقا  ہو گی۔
\begin{align}\label{مساوات_مرکز_کمیت_خطی_معیار_حرکت_بقا}
\vec{P}_{1i}+\vec{P}_{2i}=\vec{P}_{1f}+\vec{P}_{2f}
\end{align}
اگر تصادم لچکی بھی ہو (جو  ایک خصوصی صورت ہے)، تب کل حرکی توانائی کی بقا بھی ہو گی۔
\begin{align}\label{مساوات_مرکز_کمیت_حرکی_توانائی_بقا}
K_{1i}+K_{2i}=K_{1f}+K_{2f}
\end{align}

دو بعدی تصادم  کا تجزیہ کرنے کے لئے مساوات \حوالہ{مساوات_مرکز_کمیت_خطی_معیار_حرکت_بقا} کو \عددی{xy} محددی نظام کے اجزاء کی صورت میں لکھنا زیادہ مفید ثابت ہوتا ہے۔ مثال کے طور پر، شکل \حوالہء{9.21} میں  ساکن ہدف  کو  گولا بغلی (ر آمنے سامنے سے نہیں )  ٹکراتا ہے۔  ان  کے بیچ ضرب،  اجسام کو محور \عددی{x}، جس پر گولا ابتدائی طور حرکت میں تھا، کے لحاظ سے \عددی{\theta_1} اور \عددی{\theta_2}  زاویوں پر بھیجتی ہے۔ یہاں ہم مساوات \حوالہ{مساوات_مرکز_کمیت_خطی_معیار_حرکت_بقا} کو محور \عددی{x} کے ہمراہ ذیل
\begin{align}\label{مساوات_مرکز_کمیت_معیار_ایکس_جزو}
m_1v_{1i}=m_1v_{1f}\cos\theta_1+m_2v_{2f}\cos\theta_2
\end{align}
اور محور \عددی{y} کے ہمراہ ذیل لکھیں گے۔
\begin{align}\label{مساوات_مرکز_کمیت_معیار_وائے_جزو}
0=-m_1v_{1f}\sin \theta_1+m_2v_{2f}\sin\theta_2
\end{align}
ہم مساوات \حوالہ{مساوات_مرکز_کمیت_حرکی_توانائی_بقا} کو  (اس خصوصی صورت کے لئے) رفتار کے روپ میں لکھ سکتے ہیں۔
\begin{align}\label{مساوات_مرکز_حرکی_بصورت_رفتار}
\frac{1}{2}m_1v_{1i}^2&=\frac{1}{2}m_1v_{1f}^2+\frac{1}{2}m_2v_{2f}^2&&\text{\RL{(حرکی توانائی)}}
\end{align}
مساوات \حوالہ{مساوات_مرکز_کمیت_معیار_ایکس_جزو} تا مساوات \حوالہ{مساوات_مرکز_حرکی_بصورت_رفتار} میں سات متغیر ہیں: دو کمیت، \عددی{m_1} اور \عددی{m_2}؛ تین رفتار، \عددی{v_{1i}}، \عددی{v_{1f}}، اور \عددی{v_{2f}}؛ اور دو زاویے، \عددی{\theta_1} اور \عددی{\theta_2}۔اگر ہم  ان میں سے کوئی بھی چار متغیرات جانتے ہوں،  باقی  تین متغیرات ان تین مساوات کو حل کر کے   معلوم کیے جا سکتے ہیں۔

%---------------------------
\ابتدا{نمونی سوال}
فرض کریں شکل \حوالہء{9.21} میں گولے کا  ابتدائی معیار حرکت  \عددی{\SI{6}{\kilo\gram\meter\per\second}} ،  جبکہ  اختتامی  معیار حرکت کا \عددی{x} جزو \عددی{\SI{4}{\kilo\gram\meter\per\second}} اور اختتامی معیار حرکت کا \عددی{y} جزو \عددی{\SI{-3}{\kilo\gram\meter\per\second}} ہے۔ ہدف کے (ا) اختتامی معیار  حرکت کا \عددی{x} جزو اور (ب) اختتامی معیار حرکت کا \عددی{y} جزو کیا ہوں گے؟
\انتہا{نمونی سوال}
%-----------------------------

\حصہ{متغیر کمیت کے  نظام: ہوائی بان}
\جزوحصہء{مقاصد}
اس حصہ کو پڑھنے کے بعد آپ  ذیل کے قابل ہوں گے۔

\اصطلاح{ہوائی بان }\فرہنگ{ہوائی بان}\حاشیہب{rocket}\فرہنگ{rocket} کی پہلی مساوات استعمال کر کے ہوائی بان کی کمیت میں کمی کی شرح، ہوائی بان کے لحاظ سے اخراجی مادے کی اضافی رفتار، ہوائی بان کی کمیت، اور ہوائی بان کی اسراع کا رشتہ جان پائیں گے۔

ہوائی بان کی دوسری مساوات استعمال کر کے اخراجی مادے کی اضافی رفتار کے لحاظ سے ہوائی بان کی رفتار ، اور ہوائی بان  کی ابتدائی اور اختتامی کمیت کا رشتہ جان پائیں گے۔

ایک ایسا حرکت پذیر  نظام  جس کی کمیت دی گئی شرح سے تبدیل ہوتی ہو کے لئے  اس شرح    اور معیار حرکت میں تبدیلی  کا رشتہ جان پائیں گے۔

\جزوحصہء{کلیدی تصورات}
بیرونی قوتوں کی غیر موجودگی میں ہوائی بان درج ذیل لمحاتی شرح سے  اسراع پذیر ہو گا،
\begin{align*}
Rv_{\text{\RL{اضافی}}}&=Ma &&\text{\RL{(ہوائی بان کی پہلی مساوات)}}
\end{align*}
جہاں \عددی{M} ہوائی بان کی لمحاتی کمیت (بشمول غیر استعمال شدہ ایندھن) ، \عددی{R} ایندھن  کے استعمال کی شرح، اور \عددی{v_{\text{\RL{اضافی}}}} ہوائی بان کے لحاظ سے    اخراج کی اضافی رفتار ہے۔ جزو \عددی{Rv_{\text{\RL{اضافی}}}} ہوائی بان انجن کا دھکا ہے۔

مستقل \عددی{R} اور \عددی{v_{\text{\RL{اضافی}}}} کی صورت میں اگر  ہوائی بان  کی رفتار \عددی{v_i} سے تبدیل ہو کر    \عددی{v_f}  ہو جائے، اور کمیت \عددی{M_i} سے تبدیل ہو کر \عددی{M_f} ہو جائے تب درج ذیل ہو گا۔
\begin{align*}
v_f-v_i=&v_{\text{\RL{اضافی}}}\ln\frac{M_i}{M_f}&&\text{\RL{(ہوائی بان کی دوسری مساوات)}}
\end{align*}

\جزوحصہء{متغیر کمیت کے  نظام: ہوائی بان}
اب تک ہم فرض کرتے رہے ہیں کہ نظام کی کل  کمیت اٹل ہے۔ بعض اوقات، مثلاً ہوائی بان میں، ایسا نہیں ہو گا۔اڑان سے قبل  \اصطلاح{چبوترہ روانگی   }\فرہنگ{چبوترہ!روانگی}\حاشیہب{launching pad}\فرہنگ{launching pad} پر کھڑے ہوائی بان کی زیادہ تر کمیت دراصل ایندھن ہو گی، جو  آخر کار جل کر ہوائی بان کے انجن کی  \اصطلاح{ٹونٹی }\فرہنگ{ٹونٹی}\حاشیہب{nozzle}\فرہنگ{nozzle} سے   دھویں کی شکل میں  خارج ہو گا۔  اسراع پذیر ہوائی بان کی متغیر کمیت سے نپٹنے کی خاطر نیوٹن کے   دوسرے قاعدے کا اطلاق، صرف  ہوائی بان  کی بجائے،  ہوائی بان اور  خارجی مواد  دونوں  کو اکٹھا  لیتے ہوئے کیا جاتا ہے۔ہوائی بان کی اسراع کے دوران  اس نظام کی کمیت  تبدیل نہیں ہو گی۔

\جزوحصہء{اسراع کی تلاش}
فرض کریں ہم  جمودی حوالہ  چھوکٹ کے لحاظ سے ساکن بیٹھے گہری فضا میں، جہاں کوئی تجاذبی یا ہوائی  کی  رگڑ  ی قوت موجود نہیں،  ہوائی بان کو اسراع کرتا دیکھ رہے ہیں۔ اس یک بعدی حرکت  کے لئے   ہم ، اختیاری لمحہ \عددی{t} پر، ہوائی بان کی کمیت \عددی{M} اور سمتی رفتار \عددی{v }  فرض کرتے ہیں (شکل \حوالہء{9.22a})۔

شکل \حوالہء{9.22b}  وقتی دورانیہ \عددی{\dif t} کے بعد صورت حال پیش کرتی ہے۔ ہوائی بان کی سمتی رفتار \عددی{v+\dif v} اور کمیت \عددی{M+\dif M} ہیں، جہاں کمیت میں تبدیلی \عددی{\dif M}\ترچھا{ منفی مقدار } ہے۔ وقفہ \عددی{\dif t} کے دوران ہوائی بان سے  اخراجی مواد کی کمیت \عددی{-\dif M}  اور جمودی  حوالہ چھوکٹ کے لحاظ  سے    مواد کی سمتی  رفتار \عددی{U} ہے۔

\جزوحصہء{معیار حرکت کی بقا ہو گی}
ہمارا  نظام  ہوائی بان اور وقفہ \عددی{\dif t} میں اخراجی مواد پر مشتمل ہے۔ نظام بند اور  جدا ہے لہٰذا وقفہ \عددی{\dif t} کے دوران نظام کی خطی معیار حرکت کی بقا لازمی ہے۔ یوں ذیل ہو گا
\begin{align}\label{مساوات_مرکز_کمیت_معیار_بقا_لازمی}
P_i=P_f
\end{align}
جہاں زیر نوشت  \عددی{i} اور \عددی{f} بالترتیب  وقفہ  \عددی{\dif t} کے آغاز میں اور  اس کے اختتام پر قیمتیں ظاہر کرتی ہیں۔ مساوات \حوالہ{مساوات_مرکز_کمیت_معیار_بقا_لازمی} درج ذیل لکھی جا سکتی ہے
%???KKK am just below eq 9.82 p 242
