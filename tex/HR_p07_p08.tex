%HR_p07_p08
مثال  
\(1.02\)
کثافت اور رقیق کاری 
ایسے زلزلہ کے دوران جس میں زمین کی رقیق کاری ہوتی ہو میں بھاری جسم زمین میں دھنس سکتا ہے ترکیک کے دوران مٹی کے ذرے بہت کم رگڑ محسوس کرتے ہوئے کھسکنا شروع کرتے ہیں اور زمین ایک دلدل میں تبدیل ہو جاتی ہے ریتلی زمین کی رقیق کاری کے ممکنات کی پیشنگوئی زمین کے نمونہ کی تناسب خلا 
\(e\)
\حوالہ{1.9} 
کی صورت میں کی جا سکتی ہے یہاں
\(V\)
اس نمونہ میں ریت کے ذرات کا کل حجم جبکہ 
\(V\)
خالی زراعت کے بیچ کل خالی حجم کو ظاہر کرتا ہے اگر 
\(e\)
فاصل قیمت 
\(0.80\)
سے تجاوز کرتا ہو تب زلزلہ کے دوران رقیق کاری کا امکان ہوگا مطابقتی کثافت ریت 
\(\SI{\rho}\)
ریت کیا ہوگا ٹھوس سلیکون ڈائی اکسائیڈ جو ریت کا بنیادی جز ہے کی کثافت 
\(\SI{\rho_{SiO_2}} = 2.6×10*3\(\SI{\kilo\gram/\meter*3}\)\)
ہے مرکزی خیال نمونہ میں ریت کی کثافت
\(\rho\)
ریت سے مراد اکائی حجم میں کمیت ہے جو ریت کے تمام ذرات کی کل کمیت 
\(m\)
ریت اور نمونے کے کل حجم 
\(V\)
کل کا تناسب 
\حوالہ{1.10} 
ہے حساب نمونے کا کل حجم 
\(V\)
کل درج ذیل ہے 
\(V_کل\) = \(V_ریت\) + \(V_خلا\)
مساوات 
\(1.9\)
میں 
\(V_خلا\)
ڈالتے ہوئے 
\(V_ریت\)
کے لیے حل کرتے ہیں 
\حوالہ{1.11} 
مساوات
 \(1.8\)
کے تحت ریت کے ذرات کی کل کمیت 
\(m_ریت\)
سلیکون ڈائی اکسائیڈ کی کثافت ضرب ریت کے ذرات کا کل حجم 
\حوالہ{1.12}
ہوگا اس کو مساوات 
\(1.10\)
میں پر کر کے مساوات 
\(1.11\) 
سے 
\(V_ریت\)
ڈال کر درج ذیل حاصل ہوگا 
\حوالہ{1.13} 
فاصل قیمت 
\(e = 0.80\)
اور
 \(\SI{\rho_{SiO_2}} = 2.6×10*3\(\SI{\kilo\gram/\meter*3}\)\)
پر کر کے ہم دیکھتے ہیں کہ رقیق کاری اس صورت ہوگی جب ریت کی کثافت درج ذیل سے کم ہو 
\(\SI{\rho_{SiO_2}} = 2.6×10*3\(\SI{\kilo\gram/\meter3}\)÷\(1.80\) = 1.4×103\(\SI{\kilo\gram/\meter*3}\)\)
ایسی ترکیک میں ایک عمارت کئی میٹر زمین میں دھنس سکتا ہے
