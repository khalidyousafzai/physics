\باب{مخفی توانائی اور توانائی کی بقا}
%P199 right after eq 8-43
اختتامی حال میں اسپرنگ  ڈھیلے حال میں  ہو گا اور  ہوا باز ساکن  زمینی سطح پر ہو گا، لہٰذا نظام کی   اختتامی  میکانی  توانائی درج ذیل ہو گی۔ 
\begin{gather}
\begin{aligned}
E_{\text{\RL{میکانی،2}}}&=K_2+U_{e2}+U_{g2}\\
&=0+0+0
\end{aligned}
\end{gather}
آئیں اب زمینی سطح  راہ اور تیراک  کی حراری توانائی  میں تبدیلی \عددی{\Delta E_{\text{حر}}} کی بات کرتے ہیں۔ مساوات \حوالہء{8.31} سے \عددی{\Delta E_{\text{حر}}} کے لئے (رگڑی قوت  قدر ضرب رگڑ کا فاصلہ) \عددی{f_kL} ڈالا  جا سکتا ہے۔ مساوات \حوالہء{6.2} سے ہم جانتے ہیں  \عددی{f_k=\mu_kF_N} ہو گا،  جہاں \عددی{F_N} عمودی قوت ہے۔خطہ میں تیراک رگڑ کے ساتھ افقی  حرکت کرتا ہے لہٰذا  \عددی{F_N} کی قدر \عددی{mg} کے برابر ہو گی (اوپر وار اور نشیب وار قوت  برابر ہوں گی)۔ یوں میکانی توانائی سے رگڑ درج ذیل  مقدار کٹوتی کرے  گی۔
\begin{align}\label{مساوات_مخفی_حری_ضیاع}
\Delta E_{\text{حر}}=\mu_kmgL
\end{align}
(مزید تجربہ کیے  بغیر یہ جاننا ممکن نہیں اس توانائی کا کتنا حصہ تیراک کو اور کتنا راہ کو منتقل ہو گا۔ہم صرف کل مقدار  جانتے ہیں۔)

مساوات \حوالہء{8.43} تا مساوات \حوالہ{مساوات_مخفی_حری_ضیاع}  کو مساوات \حوالہء{8.42} میں پر کرنے سے  
\begin{align}
0&=\frac{1}{2}kd^2+mgh-\mu_k mg L
\end{align}
ملتا ہے ، لہٰذا درج ذیل حاصل ہو گا۔
\begin{align*}
L&=\frac{kd^2}{2\mu_kmg}+\frac{h}{\mu_k}\\
&=\frac{(\SI{3.2e3}{\newton\per\meter})(\SI{5}{\meter})^2}{2(0.800)(\SI{200}{\kilo\gram})(\SI{9.8}{\meter\per\second\squared)}}+\frac{\SI{35}{\meter}}{0.800}\\
&=\SI{69.3}{\meter}&&\text{\RL{جواب}}
\end{align*}

آخر میں اس بات پر توجہ  دیں کہ ریاضی حل کتنا آسان تھا۔سوچ سمجھ کر  نظام  تعین کر کے یاد رکھتے ہوئے کہ یہ   جدا   نظام ہے، ہم توانائی کی بقا کا قانون استعمال کر پاتے ہیں۔ یوں نظام کے    ابتدائی اور اختتامی حال  توانائیوں کو ،  درمیانے  حال جانے بغیر، برابر رکھا جا سکتا ہے۔ بالخصوص،  غیر ہموار راہ پر تیراک کی حرکت پر غور کرنے کی ضرورت پیش  نہیں آئی۔ اس کی بجائے، اگر ہم قوانین نیوٹن استعمال کریں، ہمیں راہ کی مکمل معلومات جاننا ہو گا اور حساب بھی مشکل ہوتا۔

\حصہء{نظر ثانی اور خلاصہ}
\جزوحصہء{بقائی قوت}
وہ قوت،  جو کسی بند راہ  پر حرکت کرتے ہوئے ذرہ پر ، کسی ابتدائی نقطہ سے چل کر اسی نقطہ پر واپس پہنچ کر ، صفر صافی  کام کرتی ہو\موٹا{ بقائی قوت }ہو گی۔ ہم یوں بھی کہہ سکتے ہیں کہ  اگر ایک قوت  دو نقطوں کے بیچ حرکت کرتے ہوئے ذرے پر  جو صافی    کام کرے وہ  راہ پر منحصر نہ ہو تب  قوت بقائی ہو گی۔ تجاذبی قوت اور اسپرنگ قوت بقائی ہیں؛ حرکی رگڑی قوت  \موٹا{غیر بقائی }ہے۔

\جزوحصہء{مخفی توانائی}
وہ توانائی جو  ایسے نظام کی تشکیل کے ساتھ  وابستہ ہو جس میں بقائی قوت عمل پیرا ہو \موٹا{ مخفی توانائی } کہلاتی ہے۔ جب نظام کے اندر ذرے پر بقائی قوت   کام \عددی{W}  کرے، نظام کی مخفی توانائی میں  تبدیلی  \عددی{\Delta U} درج ذیل ہو گی۔
\begin{align*}
\Delta U&=-W &&(8.1)
\end{align*}
نقطہ \عددی{x_i} سے نقطہ \عددی{x_f}  پہنچنے پر، نظام کی مخفی توانائی میں تبدیلی درج ذیل ہو گی۔
\begin{align*}
\Delta U&=-\int_{x_i}^{x_f}F(x)\dif x&&(8.6)
\end{align*}

\جزوحصہء{تجاذبی مخفی توانائی}
زمین اور اس کے قریب ذرے کے نظام سے وابستہ  مخفی توانائی کو\موٹا{ تجاذبی مخفی توانائی} کہتے ہیں۔اگر ذرہ \عددی{y_i}  بلندی سے \عددی{y_f} بلندی منتقل ہو، زمین و ذرہ نظام کی تجاذبی مخفی توانائی میں رونما ہونے والی تبدیلی درج ذیل ہو گی۔
\begin{align*}
\Delta U&=mg(y_f-y_i)=mg\Delta y && (8.7)
\end{align*}
\موٹا{حوالہ نقطہ } \عددی{y_i} پر رکھ کر اور اس نقطہ پر تجاذبی مخفی توانائی \عددی{U_i=0} رکھ کر کسی بھی بلندی \عددی{y} پر ذرے کی تجاذبی مخفی توانائی درج ذیل ہو گی۔
\begin{align*}
U(y)&=mgy && (8.9)
\end{align*}

\جزوحصہء{لچکی مخفی توانائی}
لچکدار جسم کی  حالت کھینچ یا حالت داب  سے وابستہ توانائی کو\موٹا{ لچکی مخفی توانائی }کہتے ہیں۔ایک   اسپرنگ    ، جو اس وقت  قوت \عددی{F=-kx} پیدا کرتا ہے جب اس کے آزاد سر کا ہٹاو \عددی{x} ہو، کی لچکی مخفی توانائی درج ذیل ہو گی۔
\begin{align*}
U(x)&=\frac{1}{2}kx^2 && (8.11)
\end{align*}
\موٹا{حوالہ تنظیم }وہ ہو گا جب اسپرنگ ڈھیلا ہو، \عددی{x=0}  اور \عددی{U=0} ہو۔

\جزوحصہء{میکانی توانائی}
حرکی توانائی \عددی{K} اور مخفی توانائی \عددی{U} کا مجموعہ نظام کی میکانی توانائی \عددی{E_{\text{میکانی}}} ہو گا۔
\begin{align*}
E_{\text{میکانی}}=K+U && (8.12)
\end{align*}
\موٹا{جدا نظام }سے مراد وہ نظام ہے جس میں\قول{   بیرونی قوت } توانائی کی تبدیلی کا سبب نہیں بنتی۔ اگر صرف تجاذبی قوتیں جدا نظام کے  اندرون  کام کرتی ہوں، تب نظام کی میکانی توانائی \عددی{E_{\text{میکانی}}} تبدیل نہیں  ہو سکتی۔\موٹا{میکانی توانائی کی بقا کا اصول } درج ذیل لکھا جا سکتا ہے، جہاں زیر نوشت  توانائی کے انتقال  کے دوران  مختلف   لمحات ظاہر کرتی ہیں۔
\begin{align*}
K_2+U_2&=K_1+U_1 && (8.17)
\end{align*}
یہ اصول درج ذیل بھی لکھا جا سکتا ہے۔
\begin{align*}
\Delta E_{\text{میکانی}}&=\Delta K+\Delta U=0 && (8.18)
\end{align*}

\جزوحصہء{مخفی توانائی منحنیات}
ایک نظام،  جس میں یک بعدی  قوت  \عددی{F(x)} ذرے پر  عمل پیرا ہو، کی مخفی توانائی تفاعل \عددی{U(x)} جانتے ہوئے ہم یہ قوت تلاش کر سکتے ہیں۔
\begin{align*}
F(x)&=-\frac{\dif U}{\dif x}&&(8.22)
\end{align*}
اگر تفاعل  \عددی{U(x)} کی ترسیم دی گئی ہو، تب کسی بھی  نقطہ \عددی{x} پر ، ترسیم کی  ڈھال  کی   نفی  اس نقطہ پر  قوت \عددی{F(x)} ہو گی اور  ذرے کی حرکی توانائی درج ذیل ہو گی،  جہاں \عددی{E_{\text{میکانی}}} نظام  کی میکانی توانائی ہے۔
\begin{align*}
K(x)&=E_{\text{میکانی}}-U(x)&&(8.24)
\end{align*}
موٹا{واپسیں نقطہ } سے مراد وہ نقطہ ہے جس پر ذرہ  حرکت کا رخ تبدیل کرتا ہے؛ اس نقطہ پر \عددی{K=0} ہو گا۔ جن نقطوں پر \عددی{U(x)} کی ترسیم  کی ڈھال صفر ہو ان نقطوں پر ذرہ\موٹا{  توازن } میں ہو گا؛ ان نقطوں پر \عددی{F(x)=0} ہو گا۔

\جزوحصہء{نظام پر بیرونی قوت کا کردہ کام}
