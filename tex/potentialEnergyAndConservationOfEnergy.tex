%chapter 8  "Potential Energy And Conservation Of Energy"
%partial work
%p199 to p204 done
\باب{مخفی توانائی اور توانائی کی بقا}
%top of P199 (right after eq 8-43 at the end of p198)
اختتامی حال میں اسپرنگ  ڈھیلے حال میں  ہو گا اور  ہوا باز ساکن  زمینی سطح پر ہو گا، لہٰذا نظام کی   اختتامی  میکانی  توانائی  ذیل ہو گی۔ 
\begin{gather}
\begin{aligned}
E_{\text{\RL{میکانی،2}}}&=K_2+U_{e2}+U_{g2}\\
&=0+0+0
\end{aligned}
\end{gather}
آئیں اب زمینی سطح  راہ اور تیراک  کی حراری توانائی  میں تبدیلی \عددی{\Delta E_{\text{حر}}} کی بات کرتے ہیں۔ مساوات \حوالہء{8.31} سے \عددی{\Delta E_{\text{حر}}} کے لئے (رگڑی قوت  قدر ضرب رگڑ کا فاصلہ) \عددی{f_kL} ڈالا  جا سکتا ہے۔ مساوات \حوالہء{6.2} سے ہم جانتے ہیں  \عددی{f_k=\mu_kF_N} ہو گا،  جہاں \عددی{F_N} عمودی قوت ہے۔خطہ میں تیراک رگڑ کے ساتھ افقی  حرکت کرتا ہے لہٰذا  \عددی{F_N} کی قدر \عددی{mg} کے برابر ہو گی (اوپر وار اور نشیب وار قوت  برابر ہوں گی)۔ یوں میکانی توانائی سے رگڑ درج ذیل  مقدار کٹوتی کرے  گی۔
\begin{align}\label{مساوات_مخفی_حری_ضیاع}
\Delta E_{\text{حر}}=\mu_kmgL
\end{align}
(مزید تجربہ کیے  بغیر یہ جاننا ممکن نہیں اس توانائی کا کتنا حصہ تیراک کو اور کتنا راہ کو منتقل ہو گا۔ہم صرف کل مقدار  جانتے ہیں۔)

مساوات \حوالہء{8.43} تا مساوات \حوالہ{مساوات_مخفی_حری_ضیاع}  کو مساوات \حوالہء{8.42} میں پر کرنے سے  
\begin{align}
0&=\frac{1}{2}kd^2+mgh-\mu_k mg L
\end{align}
ملتا ہے ، لہٰذا درج ذیل حاصل ہو گا۔
\begin{align*}
L&=\frac{kd^2}{2\mu_kmg}+\frac{h}{\mu_k}\\
&=\frac{(\SI{3.2e3}{\newton\per\meter})(\SI{5}{\meter})^2}{2(0.800)(\SI{200}{\kilo\gram})(\SI{9.8}{\meter\per\second\squared)}}+\frac{\SI{35}{\meter}}{0.800}\\
&=\SI{69.3}{\meter}&&\text{\RL{جواب}}
\end{align*}

آخر میں اس بات پر توجہ  دیں کہ ریاضی حل کتنا آسان تھا۔سوچ سمجھ کر  نظام  تعین کر کے یاد رکھتے ہوئے کہ یہ   جدا   نظام ہے، ہم توانائی کی بقا کا قانون استعمال کر پاتے ہیں۔ یوں نظام کے    ابتدائی اور اختتامی حال  توانائیوں کو ،  درمیانے  حال جانے بغیر، برابر رکھا جا سکتا ہے۔ بالخصوص،  غیر ہموار راہ پر تیراک کی حرکت پر غور کرنے کی ضرورت پیش  نہیں آئی۔ اس کی بجائے، اگر ہم قوانین نیوٹن استعمال کریں، ہمیں راہ کی مکمل معلومات جاننا ہو گا اور حساب بھی مشکل ہوتا۔

\حصہء{نظر ثانی اور خلاصہ}
\جزوحصہء{بقائی قوت}
وہ قوت،  جو کسی بند راہ  پر حرکت کرتے ہوئے ذرہ پر ، کسی ابتدائی نقطہ سے چل کر اسی نقطہ پر واپس پہنچ کر ، صفر صافی  کام کرتی ہو\موٹا{ بقائی قوت }ہو گی۔ ہم یوں بھی کہہ سکتے ہیں کہ  اگر ایک قوت  دو نقطوں کے بیچ حرکت کرتے ہوئے ذرے پر  جو صافی    کام کرے وہ  راہ پر منحصر نہ ہو تب  قوت بقائی ہو گی۔ تجاذبی قوت اور اسپرنگ قوت بقائی ہیں؛ حرکی رگڑی قوت  \موٹا{غیر بقائی }ہے۔

\جزوحصہء{مخفی توانائی}
وہ توانائی جو  ایسے نظام کی تشکیل کے ساتھ  وابستہ ہو جس میں بقائی قوت عمل پیرا ہو \موٹا{ مخفی توانائی } کہلاتی ہے۔ جب نظام کے اندر ذرے پر بقائی قوت   کام \عددی{W}  کرے، نظام کی مخفی توانائی میں  تبدیلی  \عددی{\Delta U}   ذیل ہو گی۔
\begin{align*}
\Delta U&=-W &&(8.1)
\end{align*}
نقطہ \عددی{x_i} سے نقطہ \عددی{x_f}  پہنچنے پر، نظام کی مخفی توانائی میں تبدیلی درج ذیل ہو گی۔
\begin{align*}
\Delta U&=-\int_{x_i}^{x_f}F(x)\dif x&&(8.6)
\end{align*}

\جزوحصہء{تجاذبی مخفی توانائی}
زمین اور اس کے قریب ذرے کے نظام سے وابستہ  مخفی توانائی کو\موٹا{ تجاذبی مخفی توانائی} کہتے ہیں۔اگر ذرہ \عددی{y_i}  بلندی سے \عددی{y_f} بلندی منتقل ہو، زمین و ذرہ نظام کی تجاذبی مخفی توانائی میں رونما ہونے والی تبدیلی ذیل ہو گی۔
\begin{align*}
\Delta U&=mg(y_f-y_i)=mg\Delta y && (8.7)
\end{align*}
\موٹا{حوالہ نقطہ } \عددی{y_i} پر رکھ کر اور اس نقطہ پر تجاذبی مخفی توانائی \عددی{U_i=0} رکھ کر کسی بھی بلندی \عددی{y} پر ذرے کی تجاذبی مخفی توانائی درج ذیل ہو گی۔
\begin{align*}
U(y)&=mgy && (8.9)
\end{align*}

\جزوحصہء{لچکی مخفی توانائی}
لچکدار جسم کی  حالت کھینچ یا حالت داب  سے وابستہ توانائی کو\موٹا{ لچکی مخفی توانائی }کہتے ہیں۔ایک   اسپرنگ    ، جو اس وقت  قوت \عددی{F=-kx} پیدا کرتا ہے جب اس کے آزاد سر کا ہٹاو \عددی{x} ہو، کی لچکی مخفی توانائی  ذیل ہو گی۔
\begin{align*}
U(x)&=\frac{1}{2}kx^2 && (8.11)
\end{align*}
\موٹا{حوالہ تنظیم }وہ ہو گا جب اسپرنگ ڈھیلا ہو، \عددی{x=0}  اور \عددی{U=0} ہو۔

\جزوحصہء{میکانی توانائی}
حرکی توانائی \عددی{K} اور مخفی توانائی \عددی{U} کا مجموعہ نظام کی میکانی توانائی \عددی{E_{\text{میکانی}}} ہو گا۔
\begin{align*}
E_{\text{میکانی}}=K+U && (8.12)
\end{align*}
\موٹا{جدا نظام }سے مراد وہ نظام ہے جس میں\قول{   بیرونی قوت } توانائی کی تبدیلی کا سبب نہیں بنتی۔ اگر صرف تجاذبی قوتیں جدا نظام کے  اندرون  کام کرتی ہوں، تب نظام کی میکانی توانائی \عددی{E_{\text{میکانی}}} تبدیل نہیں  ہو سکتی۔\موٹا{میکانی توانائی کی بقا کا اصول } درج ذیل لکھا جا سکتا ہے، جہاں زیر نوشت  توانائی کے انتقال  کے دوران  مختلف   لمحات ظاہر کرتی ہیں۔
\begin{align*}
K_2+U_2&=K_1+U_1 && (8.17)
\end{align*}
یہ اصول درج ذیل بھی لکھا جا سکتا ہے۔
\begin{align*}
\Delta E_{\text{میکانی}}&=\Delta K+\Delta U=0 && (8.18)
\end{align*}

\جزوحصہء{مخفی توانائی منحنیات}
ایک نظام،  جس میں یک بعدی  قوت  \عددی{F(x)} ذرے پر  عمل پیرا ہو، کی مخفی توانائی تفاعل \عددی{U(x)} جانتے ہوئے ہم یہ قوت تلاش کر سکتے ہیں۔
\begin{align*}
F(x)&=-\frac{\dif U}{\dif x}&&(8.22)
\end{align*}
اگر تفاعل  \عددی{U(x)} کی ترسیم دی گئی ہو، کسی بھی  نقطہ \عددی{x} پر ، ترسیم کی  ڈھال  کی   نفی  اس نقطہ پر  قوت \عددی{F(x)} ہو گی اور  ذرے کی حرکی توانائی درج ذیل ہو گی،  جہاں \عددی{E_{\text{میکانی}}} نظام  کی میکانی توانائی ہے۔
\begin{align*}
K(x)&=E_{\text{میکانی}}-U(x)&&(8.24)
\end{align*}
موٹا{واپسیں نقطہ } سے مراد وہ نقطہ ہے جس پر ذرہ  حرکت کا رخ تبدیل کرتا ہے؛ اس نقطہ پر \عددی{K=0} ہو گا۔ جن نقطوں پر \عددی{U(x)} کی ترسیم  کی ڈھال صفر ہو ان نقطوں پر ذرہ\موٹا{  توازن } میں ہو گا؛ ان نقطوں پر \عددی{F(x)=0} ہو گا۔

\جزوحصہء{نظام پر بیرونی قوت کا کردہ کام}
کام \عددی{W} سے مراد وہ توانائی ہے  جو نظام پر بیرونی قوت کے عمل   کی بنا نظام سے  باہر  یا نظام کے اندر منتقل ہو۔ جہاں ایک سے زیادہ قوتیں عمل پیرا ہوں وہاں منتقل توانائی   ان کا مجموعی\موٹا{ صافی کام  } ہو گی۔   رگڑ کی غیر موجودگی میں  نظام پر کیا گیا کام اور نظام کی میکانی توانائی میں تبدیلی \عددی{\Delta E_{\text{میکانی}}} برابر ہو گی۔
\begin{align*}
W&=E_{\text{میکانی}}=\Delta K+\Delta U(x)&&(8.26,\, 8.25)
\end{align*}
نظام کے اندر حرکی رگڑی قوت کی موجودگی میں میں نظام کی حری توانائی \عددی{E_{\text{حر}}} تبدیل ہو گی۔ (حری توانائی نظام میں  جوہر  اور سالموں  کی بلا منصوبہ حرکت سے وابستہ ہے۔) ایسی صورت میں نظام پر کیا گیا کام درج ذیل ہو گا۔
\begin{align*}
W&=E_{\text{میکانی}}+\Delta E_{\text{حر}}&&(8.33)
\end{align*}
یہ تبدیلی \عددی{\Delta E_{\text{حر}}} بیرونی قوت  سے پیدا  ہٹاو کی قدر \عددی{d}  اور  رگڑی قوت کی قدر \عددی{f_k}  پر منحصر ہے۔
\begin{align*}
E_{\text{حر}}&=f_k d&&(8.31)
\end{align*}

\جزوحصہء{توانائی کی بقا}
نظام کی\موٹا{ کل توانائی } (جو میکانی توانائی اور  اندرونی توانائیوں    ، بشمول حری توانائی  ، کا مجموعہ ہو گا) میں تبدیلی  اس  توانائی  کے برابر ہو گی جو نظام سے باہر یا نظام کے اندر منتقل کی جائے۔ اس تجرباتی حقیقت کو\موٹا{ توانائی کی بقا } کہتے ہیں۔ نظام پر کیا کام \عددی{W} ہونے کی صورت میں  ذیل ہو گا۔
\begin{align*}
W&=\Delta E=E_{\text{میکانی}}+E_{\text{حر}}+E_{\text{اندرونی}}&&(8.35)
\end{align*}
جدا  نظام \عددی{W=0} کے لئے اس سے 
\begin{align*}
&E_{\text{میکانی}}+E_{\text{حر}}+E_{\text{اندرونی}} = 0 
&&(8.36)
\end{align*}
اور
\begin{align*}
E_{\text{میکانی،2}}&=E_{\text{میکانی،1}}-\Delta E_{\text{حر}}-\Delta E_{\text{اندرونی}} &&(8.37)
\end{align*}
حاصل ہوں گے، جہاں زیر نوشت ، \عددی{1} اور \عددی{2} ، دو مختلف لمحات ظاہر کرتی ہیں۔

\جزوحصہ{طاقت}
قوت کی بنا طاقت،    اس  توانائی کے انتقال کی شرح کو کہتے ہیں ، جو قوت منتقل کرتی ہے۔ یوں \عددی{\Delta t} دورانیہ میں اگر قوت  توانائی \عددی{\Delta E}  منتقل کرتی ہو تب اس قوت کی  اوسط  طاقت  درج ذیل ہو گی۔
\begin{align*}
P_{\text{اوسط}}&=\frac{\Delta E}{\Delta t} &&(8.40)
\end{align*}
قوت کی لمحاتی طاقت  ذیل ہو گی۔
\begin{align*}
P&=\frac{\dif E}{\dif t} &&(8.41)
\end{align*}

%================================
\حصہء{سوالات}
\setcounter{questioncounter}{0}
\ابتدا{سوال}
شکل \حوالہء{8.18} میں  افقی حرکت کرتا ہوا   جسم  نقطہ دار اختتامی لکیر تک تین بلا رگڑ راستوں سے پہنچ سکتا ہے، جن میں فقط بلندی کا فرق ہے۔ ان راہ کی درجہ بندی (ا) اختتامی لکیر پر جسم کی  رفتار کے لحاظ سے  اور (ب) اختتامی لکیر تک جسم کے  پہنچنے کے دورانیہ حرکت کے لحاظ سے کریں؛ زیادہ قیمت  کا نتیجہ  اول رکھیں۔
\انتہا{سوال}
%--------------------------
\ابتدا{سوال}
ایک ذرے کی مخفی توانائی تفاعل شکل \حوالہء{8.19} میں پیش ہے۔ (ا)  ذرے پر قوت کی قدر کے لحاظ سے خطہ \عددی{AB}، \عددی{BC}، \عددی{CD}، اور \عددی{DE} کی درجہ بندی کریں۔ زیادہ قیمت  کا نتیجہ  اول رکھیں۔ (ب)  بائیں مخفی توانائی کنواں  میں  پھنس جانے  کے لئے ذرے کی میکانی توانائی \عددی{E_{\text{میکانی}}}  کو کس  قیمت  سے تجاوز  کرنے کی اجازت نہیں؟ (ج) دائیں کنواں میں پھنسنے کے لئے یہ قیمت کیا ہو گی؟ (د) دونوں کنووں میں حرکت کر سکنے لیکن نقطہ \عددی{H} سے دائیں نکلنے کی صلاحیت نہ رکھنے کی صورت میں وہ قیمت کیا ہو گی؟ جزو  د کی صورت میں  \عددی{BC}، \عددی{DE}، اور \عددی{FG} میں سے کس خطہ  میں  ذرے کی  حرکی توانائی (ہ) زیادہ سے زیادہ، (و) کم سے کم ہو گی؟  
\انتہا{سوال}
%-------------------------
\ابتدا{سوال}
نقطہ \عددی{i} سے نقطہ \عددی{f} تک ایک  براہ راست  راستہ اور چار  راستے گھوم کر جاتے  ہیں۔ براہ راست راستے پر اور تین  گھوم کر جانے  والے راستوں پر  ذرے پر  بقائی قوت \عددی{F_\text{بقائی}} عمل کرتی ہے۔ چوتھے راستے پر  ذرے پر بقائی قوت \عددی{F_{\text{بقائی}}} اور غیر بقائی قوت \عددی{F_{\text{غیر بقائی}}}  عمل کرتی ہیں۔ نقطہ \عددی{i} سے نقطہ \عددی{j}   جاتے ہوئے ذرے کی  میکانی توانائی میں  تبدیلی \عددی{\Delta E_{میکانی}}  ، گھوم کر جانے والی راہوں کے ہر  سیدھے حصے پر     (جاول میں)درج  ہے۔ (ا)  براہ راست راستے پر \عددی{i} سے \عددی{j}  تک \عددی{\Delta E_{\text{میکانی}}} کیا ہو گی؟ (ب) اس ایک راہ پر جس پر \عددی{E_{\text{غیر بقائی}}} عمل پیرا ہے،  \عددی{E_{\text{غیر بقائی}}}   کی بدولت  \عددی{\Delta E_{\text{میکانی}}}کیا ہو گی؟
\انتہا{سوال}
%----------------
\ابتدا{سوال}
ایک جسم جسم      \عددی{\SI{3}{\meter}} بلندی سے   بلا رگڑ راہ پر   رہا کیا جاتا ہے (شکل \حوالہء{8.21})۔  چوٹیوں کی بلندیاں شکل میں  دی گئی ہیں۔تمام چوٹیاں  ایک جیسی دائری ہیں، اور جسم کسی بھی چوٹی سے اڑ کر نہیں  گرتا۔ (ا)  وہ  کونسی  پہلی چوٹی ہے جسے جسم پار کرنے سے قاصر ہو گا؟  (ب)  اس چوٹی کو پار نہ کرنے کے بعد جسم کیا کرے گا؟ جن چوٹیوں کو جسم پار کر پاتا ہے،  کس چوٹی پر جسم کی  (ج) مرکز مائل قوت  زیادہ سے زیادہ ہو گی، اور (د) کس چوٹی پر اس کی عمودی قوت کم سے کم ہو گی؟
\انتہا{سوال}
%------------------------------
\ابتدا{سوال}
ایک جسم  بلا رگڑ  میلان  پر \عددی{A} تا \عددی{C} حرکت کر نے کے بعد افقی خطہ \عددی{CD} سے گزرتا ہے، جہاں رگڑی قوت عمل پیرا ہے۔ کیا جسم کی حرکی توانائی(ا)  خطہ  \عددی{AB}، (ب) خطہ \عددی{BC}، اور (ج) خطہ \عددی{CD}  میں بڑھتی ہے، گھٹتی ہے، یا  مستقل رہتی ہے؟ (د) کیا ان خطوں میں جسم کی میکانی توانائی بڑھتی ہے، گھٹتی ہے، یا مستقل رہتی ہے؟
\انتہا{سوال}
%--------------------------------
\ابتدا{سوال}\شناخت{سوال_مخفی_نظم_و_ضبط}
ایک  بیلن کو،   جو  انتصابی  سلاخ  پر  چڑھا ہوا  ہے، رسی  سے اوپر کھینچا جاتا ہے (شکل \حوالہء{8.23a})۔   تنگ  سوراخ  کی بدولت یہ سلاخ پر چست بیٹھا ہے لہٰذا  رگڑی قوت کافی زیادہ   ہے۔ آپ کی قوت بیلن و سلاخ و زمین نظام پر \عددی{W=\SI{100}{\joule}} کام کرتی ہے (شکل \حوالہء{8.23b})۔نظام کی توانائیوں کو   شکل \حوالہء{8.23c} میں  \قول{فقرہ بند}  کیا گیا ہے: حرکی توانائی  \عددی{K} میں اضافہ \عددی{\SI{50}{\joule}}،  اور تجاذبی توانائی  \عددی{U_g} میں اضافہ \عددی{\SI{20}{\joule}} ہے۔ ان کے علاوہ نظام میں صرف حری توانائی \عددی{E_{\text{حر}}}  تبدیل ہوتی ہے۔ حری توانائی میں تبدیلی \عددی{\Delta E_{\text{حر}}} کیا ہو گی؟
\انتہا{سوال}
%-----------------------------
\ابتدا{سوال}
شکل \حوالہء{8.24} میں   دکھایا نظام   سوال \حوالہ{سوال_مخفی_نظم_و_ضبط}  میں پیش نظام کی طرح ہے۔ یہاں  بیلن سے بندھی   رسی آپ نیچے کھینچتے ہیں۔  نیچے جاتے ہوئے بیلن میز پر رکھے جسم کو دوسری رسی کی مدد سے کھینچتا ہے۔یہاں بھی بیلن و  سلاخ و زمین نظام کو  شکل \حوالہء{8.23b} میں پیش نظام کی طرح تصور کریں۔ آپ نظام پر \عددی{\SI{200}{\joule}} کام کرتے ہیں۔ نظام جسم پر \عددی{\SI{60}{\joule}} کام کرتا ہے۔ نظام کے اندرون میں حرکی توانائی میں \عددی{\SI{130}{\joule}} اضافہ ، اور تجاذبی توانائی میں \عددی{\SI{20}{\joule}} کمی رونما ہوئی۔ (ا)  شکل \حوالہء{8.23c} کی طرز پر نظام کی توانائی کو \قول{فقرہ بند } کریں۔ (ب) نظام کے اندر حری توانائی میں تبدیلی کتنی ہو گی؟
\انتہا{سوال}
%-----------------------------
\ابتدا{سوال}
ایک جسم شکل \حوالہء{8.25} میں  راہ پر چلتے ہوئے \عددی{h}  بلندی سے اترتا ہے۔ماسوائے نچلی افقی حصہ  کے،  جس میں جسم \عددی{D} فاصلہ  کرنے کے بعد رک جاتا ہے ،   راہ بلا رگڑ ہے۔ (ا)  بلند \عددی{h} کم کرنے سے جسم \عددی{D} سے زیادہ، کم، یا اس کے  برابر فاصلہ طے کرے گا؟ (ب) اس کے برعکس، جسم کی کمیت بڑھانے سے جسم \عددی{D} سے زیادہ، کم، یا اس کے برابر فاصلہ طے کرے گا؟
\انتہا{سوال}
%--------------------------------
\ابتدا{سوال}
ایک جسم میلان پر اترتا ہے۔ شکل \حوالہء{8.26} میں تین صورتیں پیش کی گئی ہیں، جہاں میلان بلا رگڑ نہیں ہیں۔ تینوں صورتوں میں جسم ایک جتنی بلندی سے آغاز کرتے ہوئے حرکت کرتا ہے حتٰی کہ  حرکی رگڑی قوت اسے روک پاتی ہے۔ ان صورتوں کی درجہ بندی حر توانائی میں اضافہ کے لحاظ سے کریں۔ زیادہ قیمت اول رکھیں۔
\انتہا{سوال}
%-------------------------------
\ابتدا{سوال}
تین  گیند  ایک بلندی اور ایک رفتار سے پھینکے جاتے ہیں (شکل \حوالہء{8.27})۔ ایک گیند سیدھا اوپر پھینکا جاتا ہے۔ دوسرا انتصابی لکیر سے معمولی زاویہ پر پھینکا جاتا ہے۔ تیسرا بلا رگڑ  میلان  پر روانا کیا جاتا ہے۔گیندوں کی درجہ بندی ،  نقطہ دار لکیر پر  پہنچ کر  ان کی  رفتار کے لحاظ سے کریں۔ زیادہ قیمت اول رکھیں۔
\انتہا{سوال}
%---------------------------------------
\ابتدا{سوال}
جب ایک ذرہ \عددی{f} سے \عددی{i} اور \عددی{j} سے \عددی{i}شکل \حوالہء{8.28} میں دکھائے راستوں  پر دکھائے رخ حرکت کرتا ہے، ایک بقائی قوت \عددی{\vec{F}} اس پر عمل کر کے، شکل میں   پیش کام کرتی ہے۔ نقطہ \عددی{f} سے  براہ راست  \عددی{j} منتقل ہونے کی صورت میں ذرے پر  \عددی{\vec{F}} کتنا کام   کرے  گا؟
\انتہا{سوال}
%-----------------------------------------
%module 8.1, potential energy
\حصہ{مخفی توانائی}\شناخت{حصہ_مخفی_مخفی_توانائی}
%Q1, p202
\ابتدا{سوال}
ایک اسپرنگ جو \عددی{\SI{7.5}{\centi\meter}} دبی حالت میں \عددی{\SI{25}{\joule}}  لچکی مخفی توانائی ذخیرہ کرتا ہو کا  مقیاس لچک کیا ہو گا؟
\انتہا{سوال}
%-----------------------------
\ابتدا{سوال}\شناخت{سوال_مخفی_توانائی_تفریحی_گاڑی}
پہلی چوٹی جس کی بلندی \عددی{h=\SI{42}{\meter}}  کو سر کر کے، بلا رگڑ  تفریحی  گاڑی  جس کی کمیت  \عددی{m=\SI{825}{\kilo\gram}} ہے، کی رفتار \عددی{v_0=\SI{17}{\meter\per\second}}  ہے (شکل \حوالہء{8.29})۔ اس نقطہ سے  (ا) نقطہ \عددی{A}، (ب) نقطہ \عددی{B}، اور (ج) نقطہ \عددی{C} تک  تجاذبی قوت گاڑی پر کتنا کام کرتی ہے؟ نقطہ \عددی{C} پر گاڑی و زمین نظام کی تجاذبی مخفی توانائی صفر لیتے ہوئے اس کی قیمت اس وقت کیا ہو گی جب گاڑی   (د) نقطہ \عددی{B} اور (ہ) نقطہ \عددی{A} پر ہو؟ (و)  کمیت \عددی{m} دگنی  کرنے سے  نقطہ \عددی{A} اور نقطہ \عددی{B} کے بیچ نظام کی تجاذبی  مخفی توانائی میں تبدیلی بڑھے گی، گھٹے گی، یا تبدیل نہیں ہو گی؟
\انتہا{سوال}
%------------------------
\ابتدا{سوال}\شناخت{سوال_مخفی_توانائی_کتاب}
آپ \عددی{\SI{2}{\kilo\gram}} کمیت کی   کتاب \عددی{D=\SI{10}{\meter}} بلندی  سے کھڑکی سے  نیچے    دوست کو گراتے ہو۔   آپ کے دوست کے ہاتھ زمین سے  \عددی{d=\SI{1.5}{\meter}} بلندی  (شکل \حوالہء{8.30}) پر ہیں۔ (ا)   آپ کے دوست کے  ہاتھوں تک پہنچتے ہوئے کتاب پر تجاذبی قوت کتنا کام  \عددی{W_g} کرے گی؟ (ب)   گرنے کے دوران کتاب و زمین نظام کی  تجاذبی مخفی توانائی  میں  تبدیلی  \عددی{\Delta U} کتنی ہو گی؟اگر  زمین پر نظام کی تجاذبی مخفی توانائی \عددی{U}  صفر ہو،  (ج)   پوری بلندی پر  \عددی{U} کیا ہو گی؟ (د)آپ کے  دوست کے  ہاتھوں میں پہنچ کر \عددی{U} کیا ہو گی؟ اب  زمینی سطح پر \عددی{U=\SI{100}{\joule}} لیں اور دوبارہ   (ہ)  \عددی{W_g}، (و) \عددی{\Delta U}، (ز) پوری بلندی پر \عددی{U}، اور (ح) دوست کے ہاتھوں میں \عددی{U} تلاش کریں۔
\انتہا{سوال}
%------------------
%Q4, p202
\ابتدا{سوال}\شناخت{سوال_مخفی_گھمومتا_سلاخ}
ایک گیند جس کی کمیت \عددی{m=\SI{0.341}{\kilo\gram}}  ہے بلا کمیت سلاخ جس کی لمبائی \عددی{L=\SI{0.452}{\meter}} ہے کے ایک سر کے ساتھ باندھا ہوا ہے۔ سلاخ کا دوسرا سر چول دار ہے، جو گیند  کو انتصابی  دائرے  میں حرکت کی اجازت دیتا ہے۔سلاخ کو افقی رکھ کر نیچے رخ اتنا  دھکا دیا جاتا ہے کہ گیند  جھول کر انتصابی بالا مقام تک بمشکل  پہنچ پاتا ہے، جہاں اس کی رفتار صفر  ہوتی ہے۔ تجاذبی قوت  گیند پر ابتدائی نقطہ سے (ا)  نچلے ترین نقطہ تک، (ب) بالا ترین نقطہ تک، (ج) ابتدائی  نقطہ کے ہم بلند  دائیں ہاتھ نقطہ تک  کتنا کام کرتی ہے؟ ابتدائی نقطہ پر گیند و زمین نظام کی تجاذبی مخفی توانائی صفر لیتے ہوئے،  اس کی قیمت  اس وقت کیا ہو گی جب گیند (د) نچلے ترین نقطہ، (ہ) بالا ترین نقطہ، اور ابتدائی نقطہ کے ہم بلند  دائیں ہاتھ نقطہ پر  ہو؟ (ز)  فرض کریں گیند کو  اتنی ابتدائی دھکیل دی جاتی ہے کہ یہ بالا ترین نقطہ پر غیر صفر رفتار سے پہنچتا ہے۔ کیا اس مرتبہ   نچلے ترین نقطہ سے بالا ترین نقطہ تک \عددی{\Delta U} پہلے کے لحاظ سے زیادہ ، کم، یا   وہی ہو گا؟
\انتہا{سوال}
%--------------------------
\ابتدا{سوال}\شناخت{سوال_مخفی_کروی_برتن}
 نصف کروی برتن، جس کا رداس \عددی{r=\SI{22}{\centi\meter}}   ہے،   کے کنارہ سے  \عددی{\SI{2}{\gram}}    برفانی پرت پھسلنے دی جاتی ہے۔ پرت اور برتن کا  تماس بے رگڑ ہے۔  (ا)  برتن کی تہہ تک  اترتے ہوئے پرت پر تجاذبی مخفی توانائی کتنا کام کرتی ہے؟ (ب)  پرت و زمین نظام  کی مخفی توانائی میں اس  اترنے کے دوران  کتنی تبدیلی رونما ہو گی؟ (ج)  اگر یہ مخفی توانائی برتن کی تہہ  میں صفر لی جائے، تب برتن کے کنارے پر اس کی قیمت کیا ہو گی؟ (د)  اس کے برعکس، اگر برتن کے کنارے پر جہاں پرت رہا کی گئی، مخفی توانائی صفر لی جائے تب برتن کی تہہ میں اس کی قیمت کیا ہو گی؟ (ہ)  پرت کی کمیت دگنی  کرنے سے کیا  جزو ا تا جزو د  کے جوابات  میں اضافہ ہو گا، کمی ہو گی، یا  نتائج تبدیل نہیں ہوں گے؟
\انتہا{سوال}
%----------------------------
%Q6
\ابتدا{سوال}\شناخت{سوال_مخفی_گھیر_در_گھیر}
ایک سل جس کی کمیت \عددی{m=\SI{0.032}{\kilo\gram}} ہے شکل \حوالہء{8.33} کے بے رگڑ   گھیر در  گھیر پر حرکت کرتی ہے، جہاں  گھیر کا رداس \عددی{R=\SI{12}{\centi\meter}} ہے۔  گھیر  کے نچلے حصہ سے \عددی{h=5.0\,R} بلند   نقطہ \عددی{P} سے  ساکن  سل رہا کی جاتی ہے۔ تجاذبی قوت سل پر نقطہ \عددی{P} سے نقطہ (ا) \عددی{Q} تک ، (ب)  گھیر کی چوٹی تک،  کتنا کام کرتی ہے؟ سل  و زمین نظام کی تجاذبی  مخفی توانائی  گھیر کے  تل  پر صفر لیتے ہوئے،  مخفی توانائی  اس وقت کیا ہو گی جب سل  (ج) نقطہ \عددی{P} پر، (د) نقطہ \عددی{Q}   پر،  اور (ہ)  گھیر کی چوٹی پر ہو؟ (و)  سل  محض رہا کرنے کی بجائے اسے راہ کے ہمراہ  نیچے رخ دھکا دیا جاتا ہے۔ کیا جزو ا تا جزو ہ کے جواب   میں اضافہ ہو گا، کمی ہو گی، یا  کوئی تبدیلی نہیں ہو گی؟
\انتہا{سوال}
%------------------------------------
\ابتدا{سوال}\شناخت{سوال_مخفی_چول_دار}
ایک پتلی  سلاخ  جس کی کمیت قابل نظر انداز  اور لمبائی \عددی{L=\SI{2}{\meter}} ہے کا یک سر چول دار ہے جو سلاخ کو انتصابی  دائرے  میں چکر کی اجازت دیتا ہے۔ سلاخ کے دوسرے سر کے ساتھ \عددی{m=\SI{5}{\kilo\gram}} کمیت کا گیند باندھا گیا ہے۔ سلاخ کو  ایک طرف \عددی{\theta_0=\SI{30}{\degree}}  زاویہ تک کھینچ کر   \عددی{\vec{v}_0=0} ابتدائی سمتی رفتار کے ساتھ رہا کیا جاتا ہے۔ نچلے ترین نقطے تک اترنے پر ، (ا) تجاذبی قوت  گیند پر کتنا کام کرتی ہے اور (ب) گیند و زمین نظام کی تجاذبی مخفی توانائی میں  تبدیلی کیا ہو گی؟ (ج)  نچلے نقطہ پر تجاذبی مخفی توانائی صفر لیتے ہوئے اس کی قیمت نقطہ رہائی  پر کیا ہو گی؟ (د) زاویہ \عددی{\theta_0} بڑھانے سے کیا جزو ا تا جزو ج کے جواب میں اضافہ ہو گا، کمی ہو گی، یا ان میں کوئی تبدیلی نہیں ہو گی؟
\انتہا{سوال}
%-----------------------------------------
\ابتدا{سوال}\شناخت{سوال_مخفی_چٹان}
کھڑی چٹان جس کی بلندی \عددی{\SI{12.5}{\meter}} ہے،   کی چوٹی سے  افق کے ساتھ \عددی{\SI{41}{\degree}} اوپر رخ \عددی{\SI{14}{\meter\per\second}} ابتدائی سمتی رفتار کے ساتھ \عددی{\SI{1.50}{\kilo\gram}} کا برف گولا  پھینکا جاتا ہے۔ (ا)  چٹان   کے سر سے نیچے  ہموار زمین تک پرواز کے دوران    برف گولا پر تجاذبی قوت کتنا کام کرتی ہے؟ (ب) پرواز کے دوران گولا و زمین نظام کی تجاذبی مخفی توانائی میں کتنی تبدیلی رونما ہوتی ہے؟ (ج) چٹان کی چوٹی پر  تجاذبی مخفی توانائی کی قیمت صفر لیتے ہوئے، اس کی قیمت اس وقت کیا ہو گی جب گولا نیچے زمین پر ہو؟
\انتہا{سوال}
%-----------------------------------------

%module 8.2,  conservation of mechanical energy
\حصہ{میکانی توانائی کی بقا}
%Q9
\ابتدا{سوال}
تفریحی گاڑی کی رفتار سوال \حوالہ{سوال_مخفی_توانائی_تفریحی_گاڑی} میں (ا)  نقطہ  \عددی{A} پر، (ب)  نقطہ \عددی{B} پر، اور (ج) نقطہ \عددی{C} پر کیا ہو گی؟ (د) آخری پہاڑ، جس کو گاڑی سر کرنے سے قاصر ہے،  پر گاڑی کس بلند تک پہنچ پائے گی؟ (ہ) گاڑی کی کمیت دگنی کرنے سے جزو ا تا جزو د کے جوابات کیا ہوں گے؟
\انتہا{سوال}
%--------------------------------
\ابتدا{سوال}
(ا) ہاتھوں کو پہنچ کر کتاب کی رفتار سوال \حوالہ{سوال_مخفی_توانائی_کتاب} میں کیا ہو گی؟ (ب)  کتاب کی کمیت دگنی کرنے سے یہ  رفتار کیا ہو گی؟ (ج) اس کے برعکس، اگر کتاب نیچے پھینکی جائے، کیا جزو ا کے  جواب  میں اضافہ ہو گا، کمی ہو گی، یا اس میں کوئی تبدیلی  نہیں ہو گی؟
\انتہا{سوال}
%----------------------------------
\ابتدا{سوال}
(ا)  برتن کی تہہ کو پہنچ کر سوال \حوالہ{سوال_مخفی_کروی_برتن} میں برفانی پرت کی رفتار کیا ہو گی؟ (ب)  پرت کی کمیت دگنی کرنے سے یہ  رفتار  کی ہو گی؟ (ج)  اس کے برعکس، اگر پرت کو برتن کے ہمراہ  ابتدائی  نیچے رفتار  دی جائے، کیا جزو ا کے جواب میں اضافہ ہو گا، کمی ہو گی، یا اس میں کوئی تبدیلی نہیں ہو گی؟
\انتہا{سوال}
%--------------------------
\ابتدا{سوال}
(ا)  توانائی کے تراکیب ، نا کہ باب \حوالہء{4}  کے تراکیب،  استعمال کرتے ہوئے  سوال \حوالہ{سوال_مخفی_چٹان} میں کھڑی چٹان کی چوٹی سے نیچے زمین پر پہنچ کر  برف گولے کی رفتار تلاش کریں۔ (ب)   زاویہ پھینک  افق سے   \عددی{\SI{41}{\degree}}  نیچے  رکھنے سے رفتار کیا ہو گی؟     (ج)   کمیت \عددی{\SI{2.5}{\kilo\gram}} کرنے سے  رفتار کیا ہو گی؟
\انتہا{سوال}
%-------------------------
%Q13
\ابتدا{سوال}
اسپرنگ بندوق سے \عددی{\SI{5.0}{\gram}} چھرا سیدھا اوپر مارا جاتا ہے۔ دبے اسپرنگ  پر چھرے کے مقام سے \عددی{\SI{20}{\meter}} بلندی  تک  پہنچنے کے لئے اسپرنگ کو \عددی{\SI{8.0}{\centi\meter}} دبانا ہو گا۔ (ا)  چھرا و زمین نظام کی تجاذبی مخفی توانائی میں چھرے  کے \عددی{\SI{20}{\meter}}  صعود  کے دوران کتنی تبدیلی   \عددی{\Delta U_g} ہو گی؟ (ب)  چھرا  پھینکنے کے دوران  اسپرنگ کی لچکی مخفی توانائی  میں تبدیلی \عددی{\Delta U_s} کیا ہو گی؟ (ج)  اسپرنگ کا مقیاس لچک کیا ہے؟
\انتہا{سوال}
%--------------------------------------
%Q14
\ابتدا{سوال}
(ا) انتصابی بالا نقطہ تک صفر رفتار کے ساتھ پہنچنے کے لئے سوال \حوالہ{سوال_مخفی_گھمومتا_سلاخ} میں  گیند کی  ابتدائی  رفتار کیا ہو گی؟ ایسی صورت میں   گیند کی رفتار (ب) زیریں ترین نقطہ پر اور (ج)ابتدائی مقام کے ہم بلند  دائیں نقطہ پر  کیا ہو گی؟ (د) کیا   گیند کی کمیت دگنی کرنے سے جزو ا تا جزو ج کے جواب میں اضافہ  ہو گا، کمی ہو گی، یا ان میں کوئی تبدیلی نہیں ہو گی؟
\انتہا{سوال}
%---------------------
\ابتدا{سوال}
ایک ٹرک جس کے بریک ناکارہ ہو چکے ہیں \عددی{\SI{130}{\kilo\meter\per\hour}} رفتار  کے ساتھ سوات    \اصطلاح{تیز رو شاہراہ }\فرہنگ{تیز رو شاہراہ}\حاشیہب{motorway}\فرہنگ{motorway} پر  پہاڑی  سے اتر رہا ہے جب ڈرائیور اس  کو حفاظتی  \اصطلاح{ روک میلان }\فرہنگ{روک میلان}\حاشیہب{escape ramp}\فرہنگ{escape ramp} پر ڈالتا ہے جس کا زاویہ میلان \عددی{\theta=\SI{15}{\degree}} ہے ( شکل \حوالہء{8.35} )۔ ٹرک کی کمیت \عددی{\SI{1.2e4}{\kilo\gram}} ہے۔ (ا)  ٹرک کو روک پانے  کے لئے میلان کی کم سے کم لمبائی \عددی{L} کیا ہے؟ (ٹرک کو ایک ذرہ تصور کریں اور اس مفروضے کا جواز پیش کریں۔)  (ب)  ٹرک کی کمیت  کم کرنے سے اور (ج) اس کی رفتار بڑھانے سے، کیا کم سے کم درکار  لمبائی \عددی{L} بڑھے گی، کم ہو گی، یا اس میں کوئی تبدیلی نہیں آئے گی؟
\انتہا{سوال}
%------------------------ 
\ابتدا{سوال}
ایک سل جس  کی کمیت \عددی{\SI{700}{\gram}} ہے ، انتصابی اسپرنگ  جس کا مقیاس لچک \عددی{k=\SI{400}{\newton\per\meter}}  اور کمیت قابل نظر انداز ہے،  کے اوپر   \عددی{h_0} بلندی سے (ساکن حالت سے) گرنے دیا جاتا ہے۔ سل اور اسپرنگ  آپس میں جڑ جاتے  ہیں اور اس وقت لمحاتی رکتے  ہیں جب اسپرنگ \عددی{\SI{19.0}{\centi\meter}} دب جائے۔ رکنے تک  (ا) اسپرنگ پر سل کتنا کام کرتی ہے اور (ب) سل پر اسپرنگ کتنا کام کرتا ہے۔ (ج)   \عددی{h_0} کی قیمت کیا ہے؟ (د) سل کو \عددی{2h_0} بلندی سے رہا کرنے کی صورت میں اسپرنگ کتنا دبے گا؟
\انتہا{سوال}
%--------------------------
%Q17
\ابتدا{سوال}
سل پر  سوال \حوالہ{سوال_مخفی_گھیر_در_گھیر} میں   نقطہ \عددی{Q} پر صافی  عمل پیرا قوت  کی قدر  کا (ا) افقی جزو اور (ب) انتصابی جزو کیا ہوں گے؟ (ج)  سل  کس بلندی \عددی{h} سے رہا کرتی ہو گی اگر ہم چاہتے ہوں کہ   یہ  گھیر کی چوٹی پر   راہ سے اٹھنے لگے۔ (راہ سے سل اس وقت اٹھنے لگے گی جب   سل  پر راہ کی  عمودی قوت صفر ہو۔) (د)  ابتدائی بلند ی کی سعت \عددی{h=0} تا \عددی{h=6R}  کے لئے چوٹی پر  پہنچ کر   سل پر عمودی قوت  کی قدر ترسیم کریں۔
\انتہا{سوال}
%---------------------------------
\ابتدا{سوال}
(ا) گیند کی  رفتار  زیریں تر نقطہ پر سوال \حوالہ{سوال_مخفی_چول_دار} میں کیا ہو گی؟ (ب)   گیند کی کمیت بڑھانے سے کیا رفتار بڑھتی ہے، گھٹتی ہے، یا تبدیل نہیں ہوتی؟
\انتہا{سوال}
%-----------------------------
%Q19
\ابتدا{سوال}
ایک پتھر جس کی کمیت \عددی{\SI{8.00}{\kilo\gram}} ہے، اسپرنگ پر ساکن پڑا ہے (شکل \حوالہء{8.36})۔ اسپرنگ کو پتھر \عددی{\SI{10.0}{\centi\meter}} دباتا ہے۔ (ا) اسپرنگ کا مقیاس لچک کیا ہے؟ (ب)  پتھر کو مزید \عددی{\SI{30.0}{\centi\meter}} دبا کر رہا کیا جاتا ہے۔ رہا کرنے سے قبل دبے اسپرنگ کی لچکی مخفی توانائی کیا ہو گی؟ (ج)  نقطہ رہائی سے بلند تر نقطہ پہنچ کر  پتھر و زمین نظام کی تجاذبی مخفی توانائی میں کتنی تبدیلی رونما ہو گی؟ (د)  نقطہ رہائی سے یہ بلند تر نقطہ کتنی اونچائی پر ہے؟
\انتہا{سوال}
%------------------------------------
\ابتدا{سوال}
قابل نظر انداز کمیت کے \عددی{\SI{4.0}{\meter}} لمبے دھاگے  کے ساتھ   \عددی{\SI{2.0}{\kilo\gram}}  پتھر باندھ  کر ایک رقاص حاصل کیا جاتا ہے۔ زیریں تر نقطہ سے گزرتے وقت پتھر کی رفتار \عددی{\SI{8.0}{\meter\per\second}} ہے۔ (ا)  اس کی رفتار اس وقت کیا ہو گی جب دھاگا انتصاب کے ساتھ \عددی{\SI{60}{\degree}} زاویہ بناتا ہو؟  (ب)   پتھر کی حرکت کے دوران  انتصاب کے ساتھ دھاگا زیادہ سے زیادہ کتنا زاویہ بنائے گا؟ (ج)  اگر  پتھر کے  زیریں تر نقطہ  پر رقاص و زمین نظام کی مخفی توانائی صفر  رکھی جائے، نظام کی کل میکانی توانائی کیا ہو گی؟
\انتہا{سوال}
%-------------------------------
\ابتدا{سوال}
ایک رقاص جس کی لمبائی \عددی{L=\SI{1.25}{\meter}} ہے شکل \حوالہء{8.34} میں دکھایا گیا ہے۔  اس کے بلور (جس میں  عملاً رقاص کی پوری کمیت   سموتی ہے) کی رفتار اس وقت \عددی{v_0} ہو گی جب  رقاص کا دھاگا انتصاب کے ساتھ \عددی{\theta_0=\SI{40.0}{\degree}} زاویے پر ہو۔ (ا) اگر \عددی{v_0=\SI{8.00}{\meter\per\second}} ہو، زیریں تر نقطہ پر بلور کی رفتار کیا ہو گی؟    اگر  نیچے جانے کے بعد  دھاگا  سیدھا رکھتے ہوئے   (ب) رقاص   افقی حالت  ، اور (ج) انتصابی حالت اختیار  پائے، \عددی{v_0}  کی کم سے کم قیمت کیا ہو گی؟  (د)  زاویہ \عددی{\theta_0} چند درجے بڑھانے سے کیا جزو ب اور جزو ج کے جواب میں اضافہ ہو گا، کمی ہو گی، یا ان میں کوئی تبدیلی نہیں ہو گی؟
\انتہا{سوال}
%-----------------------
%Q22
\ابتدا{سوال}
ایک  سکی باز  جس کی کمیت \عددی{\SI{60}{\kilo\gram}} ہے، ساکن حالت سے  سکی  اچھال میلان کے اختتام   سے \عددی{H=\SI{20}{\meter}} بلند نقطہ سے آغاز کر کے(شکل \حوالہء{8.37})  زاویہ \عددی{\theta=\SI{28}{\degree}} پر میلان چھوڑتا ہے۔ ہوائی رگڑ نظر انداز  کریں اور  میلان بلا رگڑ تصور کریں۔ (ا)  میلان کے اختتام سے کتنی زیادہ سے زیادہ  بلندی  \عددی{h} تک  یہ پہنچے گا؟ (ب)  اگر سکی باز   ساز و   سامان  اٹھا کر روانا ہو، کیا \عددی{h} کی قیمت میں اضافہ ہو گا، کمی ہو گی، یا وہی رہے گی؟
\انتہا{سوال}
%---------------------------------
\ابتدا{سوال}
ایک دھاگا جس کی لمبائی \عددی{L=\SI{120}{\centi\meter}} ہے کا یک سر بندھا ہوا  جبکہ دوسرے  سے گیند لٹکائی گئی ہے۔بندھے سر  سے \عددی{d=\SI{75.0}{\centi\meter}} فاصلے پر  دیوار میں نقطہ \عددی{P} پر    ایک  میخ موجود ہے۔   دھاگا افقی رکھتے ہوئے (جیسا شکل \حوالہء{8.38} میں دکھایا گیا ہے) ساکن گیند رہا  کیا جاتا ہے، جو نقطہ دار قوس پر چلے گا۔ (ا) زیریں ترین نقطہ پر، اور (ب) میخ میں دھاگا  پھنسنے کے بعد بلند ترین نقطہ پر گیند کی رفتار کیا ہو گی؟
\انتہا{سوال}
%-----------------------------
\ابتدا{سوال}
ایک سل جس کی کمیت \عددی{m=\SI{2.0}{\kilo\gram}} ہے اسپرنگ پر \عددی{h=\SI{40}{\centi\meter}} بلندی سے گرنے دیا جاتا ہے (شکل \حوالہء{8.39})۔ اسپرنگ کا مقیاس لچک \عددی{k=\SI{1960}{\newton\per\meter}} ہے۔ اسپرنگ زیادہ سے زیادہ کتنا دبے گا؟
\انتہا{سوال}
%-------------------------------
%Q25
\ابتدا{سوال}
لمحہ \عددی{t=0} پر \عددی{\SI{1.0}{\kilo\gram}}  گیند     بلند  کھمبے   سے \عددی{\vec{v}=(\SI{18}{\meter\per\second})\vec{i}+(\SI{24}{\meter\per\second})\vec{j}} کے ساتھ روانا کیا جاتا ہے۔  گیند و زمین نظام کی \عددی{\Delta U} لمحہ \عددی{t=0} تا  \عددی{t=\SI{6.0}{\second}} کیا ہو گی (آزادانہ گرنا تصور کریں)؟
\انتہا{سوال}
%-------------------------
\ابتدا{سوال} 
محور \عددی{x} پر حرکت کرتے ہوئے  ذرے پر بقائی قوت \عددی{\vec{F}=(6.0x-12)\hat{i}\,\si{\newton}} عمل   کرتی ہے، جہاں \عددی{x} میٹروں میں ہے۔ اس قوت کے ساتھ  وابستہ مخفی توانائی \عددی{U} نقطہ \عددی{x=0} پر \عددی{\SI{27}{\joule}} ہے۔ (ا)  مخفی توانائی \عددی{U} کا تفاعل \عددی{x} کی صورت میں لکھیں جہاں \عددی{x}  میٹروں میں ہے۔ (ب)  زیادہ سے زیادہ مثبت مخفی توانائی کیا ہے؟  \عددی{x} کی  کس (ج) مثبت قیمت اور (د) منفی قیمت پر مخفی توانائی صفر ہے؟
\انتہا{سوال}
%--------------------------
%Q27
\ابتدا{سوال}
کھڑی چٹان سے  \عددی{\SI{688}{\newton}}  وزن کا شخص  \عددی{\SI{18}{\meter}} لمبی رسی  سے جھولتا ہے (شکل \حوالہء{8.40})۔  چٹان کی چوٹی سے     زیریں ترین نقطہ تک نشیب  \عددی{\SI{3.2}{\meter}}  ہے۔ رسی اس وقت ٹوٹے  گی جب اس  کو   \عددی{\SI{950}{\newton}} سے زیادہ قوت  کھینچے۔ (ا) کیا رسی ٹوٹے گی؟ (ب)  رسی نہ ٹوٹنے کی صورت میں نشیب کے دوران رسی پر زیادہ سے زیادہ قوت کتنی ہو گی؟  رسی ٹوٹنے کی صورت میں، ٹوٹتے وقت رسی انتصاب کے ساتھ کس زاویے پر ہو گی؟
\انتہا{سوال}
%------------------------
\ابتدا{سوال}
ہوائی بندوق میں  نصب اسپرنگ شکل \حوالہء{8.41a} پر پورا  اترتا ہے؛ جو قوت بالمقابل اسپرنگ کا داب یا   دراضی   دیتا ہے۔ اسپرنگ کو \عددی{\SI{5.5}{\centi\meter}} دبا کر \عددی{\SI{3.8}{\gram}} چھرا بندوق سے  مارا جاتا ہے۔ (ا)  اگر   چھرا اس  لمحے   رہا  ہو جب اسپرنگ اپنے ڈھیلے  حالت کو پہنچے، چھرے کی رفتار اس  لمحے کیا ہو گی؟ (ب)  اس کے برعکس،  تصور کریں چھرا  اسپرنگ  کو پکڑے رکھتا ہے اور    اسپرنگ  کو  کھینچ کر  \عددی{\SI{1.5}{\centi\meter}}  لمبا کرنے کے بعد اس سے علیحدہ ہوتا ہے۔چھرے کی رفتار اس لمحے کیا ہو گی جب یہ اسپرنگ سے علیحدہ ہوتا ہے؟
\انتہا{سوال}
%-------------------
%Q29
\ابتدا{سوال}
ایک سل جس کی کمیت \عددی{m=\SI{12}{\kilo\gram}} ہے ساکن حالت سے \عددی{\theta=\SI{30}{\degree}} بلا رگڑ  میلان پر رہا کیا جاتا ہے (شکل \حوالہء{8.42})۔ میلان پر سل سے نیچے  ایک اسپرنگ ہے جس کو \عددی{\SI{270}{\newton}} قوت \عددی{\SI{2.0}{\centi\meter}} دبا سکتی ہے۔ اسپرنگ کو \عددی{\SI{5.5}{\centi\meter}} دبا کر سل لمحاتی رکتی ہے۔ (ا)    نقطہ رہائی سے رکنے کے نقطہ تک میلان پر سل کتنا فاصلہ طے کرتی ہے؟ (ب)سل کی رفتار اس لمحے کیا ہو گی جب وہ   اسپرنگ کو چھوتی ہے؟
\انتہا{سوال}
%-----------------------------
%Q30 p204
\ابتدا{سوال}
بلا رگڑ میلان جس کا  زاویہ \عددی{\theta=\SI{40}{\degree}} ہے پر  رکھا \عددی{\SI{2.0}{\kilo\gram}} ڈبہ    ایک ڈوری کے ذریعہ،  جو چرخی سے گزرتی ہے،   اسپرنگ سے باندھا گیا ہے۔ اسپرنگ کا مقیاس لچک \عددی{k=\SI{120}{\newton\per\meter}} ہے (شکل \حوالہء{8.43})۔   ڈور  میں جھول  نہیں  اور  اسپرنگ  ڈھیلا ہے۔ ڈبہ ساکن حالت سے رہا کیا جاتا ہے۔ چرخی کو بلا رگڑ اور بلا کمیت تصور کریں۔ (ا)  میلان پر \عددی{\SI{10}{\centi\meter}}  نیچے  رخ چل کر ڈبے کی رفتار کیا ہو گی؟ (ب) نقطہ رہائی سے میلان  پر ڈبہ کتنا فاصلہ طے کرنے کے بعد لمحاتی رکتا ہے، اور  اس لمحے  پر ڈبے  کے  اسراع کی (ج)  قدر اور (د) رخ (میلان پر اوپر یا نیچے رخ) کیا ہوں گے؟
\انتہا{سوال}
%---------------
%Q31 p204
\ابتدا{سوال}
بلا رگڑ میلان جس کا زاویہ \عددی{\theta=\SI{30.0}{\degree}} ہے پر \عددی{m=\SI{2.00}{\kilo\gram}}  سل    \عددی{k=\SI{19.6}{\newton\per\centi\meter}} مقیاس لچک اسپرنگ کے ساتھ  ملا کر رکھی  جاتی ہے، تاہم یہ ایک دوسرے کے ساتھ جڑے نہیں ہیں  (شکل \حوالہء{8.44})۔ اسپرنگ کو \عددی{\SI{20.0}{\centi\meter}} دبا کر رہا کیا جاتا ہے۔ (ا)  دبے اسپرنگ کی لچکی مخفی توانائی کیا ہو گی؟ (ب)     سل و  زمین کی تجاذبی مخفی توانائی میں تبدیلی ، نقطہ رہائی سے  میلان پر بلند تر  نقطہ تک سل کے   پہنچنے تک ، کتنی  ہو گی؟ (ج)  نقطہ رہائی سے سل میلان پر بلند تر نقطہ تک کتنا فاصلہ طے کرتی ہے۔
\انتہا{سوال}
%------------------------
