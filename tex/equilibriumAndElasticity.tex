%in Q5 change penguin پینگوین to sheep

%equilibrium and elasticity p327
\باب{توازن اور لچک}
%12.1 equilibrium p327
\حصہ{توازن}
\موٹا{مقاصد}\\
اس حصہ کو پڑھ کر آپ ذیل کے قابل ہوں گے۔
\begin{enumerate}[1.]
\item
توازن اور سکونی توازن میں فرق کر پائیں گے۔
\item
سکونی توازن کے چار شرائط جان پائیں گے۔
\item
مرکز  ثقل    اور  اس کا مرکز کمیت  سے تعلق   سمجھا پائیں گے۔
\item
ذروں کی  دی گئی تقسیم کے لئے  مرکز ثقل اور مرکز کمیت کے محدد  کا حساب کر پائیں گے۔
\end{enumerate}

\موٹا{کلیدی تصور}\\
\begin{itemize}
\item
استوار جسم جب ساکن ہو، وہ سکونی توازن میں ہو گا۔ ایسے جسم کے لئے، جسم پر بیرونی قوتوں کا مجموعہ صفر ہو گا۔
\begin{align*}
\vec{F}_{\text{\RL{صافی}}}=0 \quad\quad\text{\RL{(قوتوں کا توازن)}}
\end{align*}
اگر تمام قوت \عددی{xy} مستوی میں ہوں، یہ مساوات   ذیل دو جزوی مساوات کی معادل ہو گی.
\begin{align*}
F_{\text{\RL{صافی}},y}&=0 \quad \text{\RL{اور}}\quad F_{\text{\RL{صافی}},x}=0  \quad\quad\text{\RL{(قوتوں کا توازن)}}
\end{align*}
\item
سکونی توازن سے مراد یہ بھی ہے کہ کسی بھی نقطے  کے لحاظ سے جسم پر   بیرونی  قوت مروڑ  کا مجموعہ صفر ہو گا:
\begin{align*}
\vec{\tau}_{\text{\RL{صافی}}}=0\quad\quad\text{\RL{(قوت مروڑ کا توازن)}}
\end{align*}
اور اگر تمام قوت \عددی{xy} مستوی میں ہوں تب تمام قوت مروڑ سمتیات محور \عددی{z} کو متوازی ہوں گے، اور  قوت مروڑ کے توازن کی مساوات ذیل  یک جزوی  مساوات کی معادل ہو گی۔
\begin{align*}
\tau_{\text{\RL{صافی}},z}=0\quad\quad\text{\RL{(قوت مروڑ کا توازن)}}
\end{align*}
\item
تجاذبی قوت جسم کے ہر ذرے پر انفرادی عمل  کرتی ہے۔تمام انفرادی  اعمال کا صافی  اثر  جاننے کے لئے مرکز کمیت پر معادل تجاذبی قوت \عددی{\vec{F}_g} فرض  کرنی  ہو گی۔اگر جسم کے تمام ٹکڑوں پر ثقلی اسراع \عددی{\vec{g}} ایک ہو، ثقلی مرکز جسم کے مرکز کمیت پر ہو گا۔
\end{itemize}

\جزوحصہ{طبیعیات کیا ہے؟}
انسانی بنائی چیزیں، لاگو قوتوں سے قطع نظر، مستحکم   تصور کی جاتی ہیں۔  تجاذبی قوت اور ہوائی قوتوں کے باوجود ہم توقع کرتے ہیں کہ عمارت کھڑی رہے گی، اور پُل سمندر میں  گرے گا نہیں۔

طبیعیات کے مرکز توجہ  وہ حقیقت ہے جو عمل پیرا قوتوں کے باوجود  جسم کو  مستحکم رکھتا ہے۔ اس باب میں استحکام  کے دو نقطہ نظر پر غور کیا جائے گا: استوار جسم پر عمل پیرا قوت اور قوت مروڑ کا \ترچھا{ توازن } اور نا  استوار اجسام کی  \ترچھا{لچک} ، جس پر اجسام کا  مسخ ہونا منحصر ہے۔ اگر  طبیعیات درست کی جائے،اس پر   انجینئری اور طبیعیات کے جریدوں  میں لاتعداد  مضامین  لکھے جائیں گے؛ اگر غلط کی جائے، اخبار   کا سرنامہ بنے گا اور قانونی کارروائی ہو گی۔

\جزوحصہء{توازن}
ذیل اجسام پر غور کریں: (1) میز پر  پڑی ساکن کتاب، (2) بلا رگڑ سطح پر مستقل سمتی رفتار سے حرکت پذیر قرص، (3)  چھت کے پنکھے کے چکر کھاتے پَر، اور (4)  سیدھی راہ پر چلتے سائیکل کا پہیا۔ ان چار اجسام کے لئے
\begin{enumerate}[1.]
\item
مرکز کمیت کا خطی معیار حرکت \عددی{\vec{P}} ایک مستقل ہے۔
\item
مرکز کمیت یا کسی دوسرے نقطہ کے لحاظ سے ان کا زاوی معیار حرکت \عددی{\vec{L}} بھی ایک مستقل ہے۔
\end{enumerate}

ہم کہتے ہیں یہ جسم\اصطلاح{ توازن }\فرہنگ{توازن}\حاشیہب{equilibrium}\فرہنگ{equilibrium} میں ہیں۔ یوں توازن کے  دو  شرائط ذیل ہیں۔
%eq 12.1
\begin{align}\label{مساوات_توازن_تعریف_الف}
\vec{L}=\text{\RL{مستقل}}\quad \text{\RL{اور}}\quad \vec{P}=\text{\RL{مستقل}}
\end{align}

اس باب میں ہم صرف  ان صورتوں پر غور کرتے ہیں جہاں  مساوات  \حوالہ{مساوات_توازن_تعریف_الف} میں مستقل کی قیمت صفر ہو؛ یعنی   ہم ان اجسام میں دلچسپی رکھتے ہیں جو حوالہ چوکھٹ کے لحاظ سے  ساکن ہوں؛  خطی سکون اور گھمیری سکون میں ہم دلچسپی رکھتے ہیں۔ ایسے اجسام\اصطلاح{ سکونی توازن }\فرہنگ{توازن!سکونی}\حاشیہب{static equilibrium}\فرہنگ{equilibrium!static} میں ہوں گے۔ باب کے آغاز میں چار  اجسام میں صرف میز پر پڑی کتاب سکونی توازن میں ہے۔

شکل \حوالہء{12.1} میں   دکھائی گئی چٹان  ، فی الحال ، سکونی توازن میں ہے۔مساجد،  پُل، گھر، وغیرہ بھی سکونی توازن میں ہیں؛ یہ وقت  گزرنے کے باوجود  ساتھ  ساکن رہتے ہیں۔

جیسا ہم حصہ \حوالہء{8.3} میں ذکر کر چکے ، اگر  سکونی توازن سے قوت کے بل بوتے  پر   نکالے جانے  کے بعد جسم واپس  سکونی توازن  کو لوٹے،ہم کہتے ہیں یہ  جسم\ترچھا{ مستحکم } سکونی توازن میں ہے۔ نصف کرہ کے تل میں رکھا گیا  کنچا اس کی ایک مثال ہے۔ اس کے برعکس، اگر  چھوٹی قوت جسم کو ہلا کر  توازن ختم کر پائے، جسم \ترچھا{ غیر مستحکم } سکونی توازن میں ہو گا۔

\موٹا{زنجیری اثر۔}\quad
فرض کریں ہم  ایک اینٹ یوں کھڑی کریں کہ اس کا مرکز کمیت عین ایک  کنارے کے اوپر ہو (شکل \حوالہء{12.2a})۔  تجاذبی قوت \عددی{\vec{F}_g} کا خط عمل  اسی کنارے  سے گزرتا ہے لہٰذا  اس کنارے  پر \عددی{\vec{F}_g} کی قوت مروڑ  صفر ہو گی۔ اینٹ توازن میں ہے۔معمولی   اضطراب  اس توازن کو برباد کر دیگا۔ جیسے ہی \عددی{\vec{F}_g} کا خط عمل  کنارے  سے  ایک  طرف ہو (شکل \حوالہء{12.2b})، \عددی{\vec{F}_g} کی پیدا کردہ قوت مروڑ اینٹ کو اس طرف گھمائے گی۔ یوں شکل \حوالہء{12.2a} میں اینٹ غیر مستحکم توازن میں ہے۔

شکل \حوالہء{12.2c} میں اینٹ اتنی غیر مستحکم نہیں۔ اینٹ   گرانے کے لئے ضروری ہے کہ  قوت   اینٹ  اتنی گھمائے کہ اینٹ کا مرکز کمیت کنارے کو پار کر جائے۔ معمولی قوت اس اینٹ کو نہیں گرا سکتی، تاہم  انگلی سے جھٹکا دے کر اسے گرایا جا سکتا ہے۔(اینٹوں کو قطار میں کھڑا کر کے ، پہلی اینٹ کو جھٹکا دے کر گرانے سے تمام اینٹیں گرائی جا سکتی ہیں۔)

\موٹا{سل۔}\quad
شکل \حوالہء{12.2d} میں دکھایا گیا سل مزید زیادہ مستحکم ہے۔  مرکز کمیت کو سل کے کنارے کی دوسری طرف لی جانے کے لئے مرکز کمیت کو  کافی زیادہ  دور لے جانا ہو گا۔ انگلی کا جھٹکا سل کا   پاسا  نہیں پلٹ سکتا۔ (اسی لئے سل قطار میں رکھ کر زنجیری اثر  پیدا  نہیں کیا جا سکتا۔) شکل \حوالہء{12.3} میں  شہتیر  پر بیٹھا مزدور سل کی مانند جبکہ اس پر کھڑا مزدور اینٹ کی مانند ہو گا (جس کو ہوا کا جھٹکا نیچے لا سکتا ہے)۔

سکونی توازن اطلاقی انجینئری   کے لئے  بہت  ضروری ہے۔ تخلیق کار  تمام بیرونی قوت اور قوت مروڑ کی نشاندہی کر کے، بہتر تراکیب اور  مواد  استعمال کر کے، یقینی بناتا ہے کہ ان کی موجودگی کے باوجود عمارت یا مشین مستحکم رہے۔ یوں پُل  کا نقشہ تیار کرتے وقت  تخلیق کار تفصیلی تجزیہ کر کے   یقینی بناتا ہے  کہ پُل پر  آمد و رفت اور ہوائی قوتوں   کو پُل   سہ سکے۔

\جزوحصہء{توازن کے شرائط}
جسم کی   مستقیم  حرکت ، خطی معیار حرکت کے روپ میں   نیوٹن کے قانون دوم کو ، جو (ذیل)  مساوات \حوالہء{9.27} دیتی ہے،    مطمئن کرتی ہے۔
%eq 12.2
\begin{align}
\vec{F}_{\text{\RL{صافی}}}=\frac{\dif \vec{P}}{\dif t}
\end{align}
اگر جسم مستقیم  توازن  میں ہو؛ یعنی اگر  \عددی{\vec{P}} ایک مستقل ہو،  تب \عددی{\dif\vec{P}\!/\!\dif t=0} ہو گا لہٰذا لازماً درج ذیل ہو گا۔
%eq 12.3
\begin{align}\label{مساوات_توازن_شرط_ایک}
\vec{F}_{\text{\RL{صافی}}}=0\quad\quad\text{\RL{(متوازن قوت}}
\end{align}

جسم کی    گھمیری   حرکت ، زاوی  معیار حرکت کے روپ میں   نیوٹن کے قانون دوم کو ، جو (ذیل)  مساوات \حوالہ{مساوات_لڑھکاو_ذروں_نظام_ت} دیتی ہے،    مطمئن کرتی ہے۔
%eq 12.4
\begin{align}
\vec{\tau}_{\text{\RL{صافی}}}=\frac{\dif \vec{L}}{\dif t}
\end{align}
اگر جسم  گھمیری   توازن میں ہو؛ یعنی اگر  \عددی{\vec{L}} ایک مستقل ہو،  تب \عددی{\dif\vec{L}\!/\!\dif t=0} ہو گا لہٰذا لازماً درج ذیل ہو گا۔
%eq 12.5
\begin{align}\label{مساوات_توازن_شرط_دو}
\vec{\tau}_{\text{\RL{صافی}}}=0\quad\quad\text{\RL{(متوازن قوت مروڑ)}}
\end{align}
یوں  جسم کا توازن میں ہونے کے لئے ذیل دو شرائط   ہیں۔

\ابتدا{قاعدہء}
\begin{enumerate}[1.]
\item
جسم پر تمام بیرونی قوتوں کا سمتی مجموعہ صفر  ہونا لازم ہے۔
\item
ہر ممکنہ  نقطہ  کے لحاظ سے، جسم پر بیرونی قوت مروڑ  کا سمتی مجموعہ صفر ہونا   لازم ہے۔
\end{enumerate}
\انتہا{قاعدہء}
%------------------------------------


ہاں یہ شرائط\ترچھا{ سکونی } توازن کے لئے بھی ہیں۔ یہ شرائط عمومی صورت کے لئے بھی درست ہیں، جہاں \عددی{\vec{P}} اور \عددی{\vec{L}} مستقل ضرور لیکن غیر صفر ہوں۔

مساوات \حوالہ{مساوات_توازن_شرط_ایک} اور مساوات \حوالہ{مساوات_توازن_شرط_دو}، بطور سمتی مساوات، درحقیقت (ذیل)  تین تین جزوی مساوات  کی معادل ہیں۔
%eq 12.6
\begin{gather}
\begin{aligned}\label{مساوات_توازن_چہ}
&\text{\RL{متوازن قوت}}  & \text{\RL{متوازن قوت مروڑ}}\\
&F_{\text{\RL{صافی}},x}=0   &\tau_{\text{\RL{صافی}},x}=0\\
&F_{\text{\RL{صافی}},y}=0  &\tau_{\text{\RL{صافی}},y}=0\\
&F_{\text{\RL{صافی}},z}=0  &\tau_{\text{\RL{صافی}},z}=0
\end{aligned}
\end{gather}

\موٹا{اصل مساوات۔}\quad
ہم صرف  ان صورتوں پر غور کرتے ہیں جس میں جسم پر لاگو قوت \عددی{xy} مستوی میں پائے جاتے ہیں۔ یوں مسئلہ کم  پیچیدہ ہو گا۔ اس طرح جسم پر عمل پیرا قوت صرف محور \عددی{z} کی  متوازی محور   کے گرد  جسم  گھما سکتے ہیں۔ اس مفروضے کے ساتھ  مساوات \حوالہ{مساوات_توازن_چہ} میں سے قوت کی ایک مساوات اور قوت مروڑ کی دو مساوات سے چھٹکارا  حاصل ہو گا۔ یوں ذیل باقی رہتی ہیں۔
%eq 12.7, 12.8, 12.9
\begin{align}
F_{\text{\RL{صافی}},x}&=0 \label{مساوات_توازن_شرائط_الف} \\
F_{\text{\RL{صافی}},y}&=0  \label{مساوات_توازن_شرائط_ب} \\
\tau_{\text{\RL{صافی}},z}&=0 \label{مساوات_توازن_شرائط_پ}  
\end{align}
یہاں، \عددی{\tau_{\text{\RL{صافی}},z}}  وہ صافی قوت مروڑ ہے جو محور \عددی{z} یا اس کے متوازی کسی محور پر  بیرونی  قوت پیدا کرتی ہیں۔

جمی ہوئی   برف  پر مستقل سمتی رفتار سے حرکت کرتا قرص مساوات \حوالہ{مساوات_توازن_شرائط_الف}، مساوات \حوالہ{مساوات_توازن_شرائط_ب}، اور مساوات \حوالہ{مساوات_توازن_شرائط_پ}   مطمئن کرتا ہے، لہٰذا یہ توازن میں ہو گا،\ترچھا{ تاہم یہ سکونی توازن میں  ہرگز نہیں}۔ سکونی توازن کے  لئے قرص کا خطی معیار حرکت \عددی{\vec{P}} ایک مستقل ہونے کے ساتھ ساتھ  صفر ہونا  لازم  ہے؛ قرص کا جمی ہوئی برف پر ساکن ہونا لازم ہے۔ یوں، سکونی توازن کے لئے درج ذیل  شرط  بھی لازم   ہے۔

\ابتدا{قاعدہء}
جسم کے  خطی معیار حرکت \عددی{\vec{P}}  کا صفر ہونا لازم ہے۔
\انتہا{قاعدہء}

%----------------------
%Checkpoint 1 p330
\ابتدا{آزمائش}
یکساں سلاخ ، جس پر سلاخ کو عمود دار دو یا دو سے زیادہ قوت عمل کرتی ہیں،  کے چھ فضائی نظارے شکل \حوالہء{؟؟} میں پیش ہیں۔ قوتوں کی قدریں      (غیر صفر رکھ کر اور)   تبدیل کر کے  کون کونسی سلاخ سکونی توازن میں لائی جا سکتی ہیں؟
\انتہا{آزمائش}
%------------------

%the center of gravity p330
\جزوحصہء{مرکز ثقل}
جسم پر تجاذبی قوت  ، جسم کے انفرادی ٹکڑوں (جوہر) پر تجاذبی قوتوں کا سمتی مجموعہ ہو گا۔ انفرادی ٹکڑوں کی بات کرتے ہوئے ہم ذیل کہتے ہیں۔

\ابتدا{قاعدہء}
جسم پر تجاذبی قوت \عددی{\vec{F}_g} \قول{    عملاً}   جسم کے \اصطلاح{ مرکز ثقل }\فرہنگ{مرکز ثقل}\حاشیہب{center of gravity}\فرہنگ{center of gravity} پر    عمل کرتی ہے۔
\انتہا{قاعدہء}
%---------------------------

یہاں لفظ \قول{عملاً} کا مطلب یہ ہے کہ  اگر کسی طرح انفرادی ٹکڑوں  پر تجاذبی قوت ختم کر دی جائے اور تجاذبی قوت  \عددی{\vec{F}_g} جسم کے مرکز ثقل پر پیدا کر دی جائے، جسم پر صافی قوت اور  (کسی بھی محور کے لحاظ سے)  جسم پر صافی قوت مروڑ تبدیل نہیں ہوں گی۔

اب تک،  ہم فرض کرتے رہے ہیں کہ تجاذبی قوت  \عددی{\vec{F}_g} جسم کے مرکز کمیت پر عمل کرتی ہے، جو   اس  کے  مترادف  ہے  کہ ہم کہیں جسم کا مرکز ثقل  جسم کے مرکز کمیت  پر پایا جاتا ہے۔ یاد  کریں، کمیت \عددی{M}   جسم پر تجاذبی  قوت \عددی{\vec{F}_g=M\vec{g}}     عمل کرتی ہے، جہاں \عددی{\vec{g}}   جسم کا وہ اسراع ہے جو جسم پر \عددی{\vec{F}_g} لاگو کرنے سے پیدا ہو گا۔نیچے دیے گئے  ثبوت میں ہم ذیل ثابت کریں گے۔

%----------------------
%p331
\ابتدا{قاعدہء}
اگر جسم کے تمام ٹکڑوں کے لئے \عددی{\vec{g}}  ایک ہو، جسم کا مرکز ثقل اور جسم کا مرکز کمیت ایک نقطے پر ہوں گے۔
\انتہا{قاعدہء}
%-------------------------------

سطح زمین پر \عددی{\vec{g}} کی قدر بہت کم تبدیل ہوتی ہے اور  (عام زندگی میں جن بلندیوں سے  ہمیں واسطہ پڑتا ہے ان) بلندی کے ساتھ   \عددی{\vec{g}}کی قدر زیادہ تبدیل نہیں ہوتی  لہٰذا روز مرہ اشیاء  کے لئے  درج بالا تخمیناً درست ہو گا۔ یوں  چوہے یا بھینس  کے لئے تجاذبی قوت کا   ان کے مرکز کمیت پر عمل پیرا   ہونا  فرض کرنا درست ہو گا۔ ذیل ثبوت کے بعد ہم اسی مفروضے پر چلیں گے۔

\جزوجزوحصہء{ثبوت}
ہم  جسم کے انفرادی ٹکڑوں پر  پہلے غور کرتے ہیں۔ شکل \حوالہء{12.4a} میں  وسیع جسم  ، جس کی کمیت \عددی{M} ، اور   جسم کا ایک چھوٹا ٹکڑا جس کی کمیت \عددی{m_i} ہے، پیش ہے۔ ہر ٹکڑے پر تجاذبی قوت \عددی{\vec{F}_{gi}}،   جو \عددی{m_i\vec{g}_i} کے برابر ہے، عمل کرتی ہے۔ \عددی{\vec{g}_i} میں زیرنوشت کہتی ہے \عددی{\vec{g}_i}   ٹکڑا \عددی{i}کے مقام پر ثقلی اسراع ہے  (دیگر ٹکڑوں کے لئے اس کی قیمت مختلف ہو سکتی ہے)۔

شکل \حوالہء{12.4a} میں  ہر ایک    ٹکڑے  پر قوت \عددی{\vec{F}_{gi}} عمل کر کے،  مبدا \عددی{O}  کے لحاظ سے ٹکڑے پر  قوت مروڑ \عددی{\tau_i} ، جس کا معیار اثر  کا بازو \عددی{x_i} ہے، پیدا کرتی ہے۔ مساوات \حوالہ{مساوات_گھماو_صافی_قوت_مروڑ_الف}  \عددی{(\tau=r_{\perp}F)}  کی راہ نمائی  میں ہم ہر ایک قوت مروڑ \عددی{\tau_i}   ذیل لکھ سکتے ہیں۔
%eq12.10
\begin{align}
\tau_i=x_iF_{gi}
\end{align}
یوں، جسم کے تمام ٹکڑوں پر صافی قوت مروڑ ذیل ہو گی۔
%eq12.11
\begin{align}\label{مساوات_توازن_ثبوت_صفر}
\tau_{\text{\RL{صافی}}}=\sum \tau_i=\sum x_i F_{gi}
\end{align}

اب، پورا  جسم   لیتے ہیں۔شکل \حوالہء{12.4b} میں جسم کے مرکز  ثقل  پر تجاذبی  قوت \عددی{\vec{F}_g} عمل کرتا دکھایا گیا ہے۔مبدا     \عددی{O} کے لحاظ سے   اس  قوت  کا معیار اثر کا بازو \عددی{x_{\text{\RL{مرکزکمیت}}}}  اور جسم پر  پیدا قوت  مروڑ \عددی{\tau}  ہے۔ مساوات \حوالہ{مساوات_گھماو_صافی_قوت_مروڑ_الف} دوبارہ استعمال کر کے یہ قوت مروڑ ذیل لکھی جا سکتی ہے۔
%eq1.212
\begin{align}\label{مساوات_توازن_ثبوت_الف}
\tau=x_{\text{\RL{مرکزثقل}}} F_g
\end{align}
جسم پر تجاذبی قوت \عددی{\vec{F}_g} ، جسم کے تمام ٹکڑوں پر تجاذبی  قوت \عددی{\vec{F}_{gi}} کا  مجموعہ ہو گا۔ یوں  مساوات \حوالہ{مساوات_توازن_ثبوت_الف} میں \عددی{F_g} کی جگہ \عددی{\sum F_{gi}} ڈال کر ذیل لکھا جائے گا۔
%eq12.13
\begin{align}\label{مساوات_توازن_ثبوت_ب}
\tau=x_{\text{\RL{مرکزثقل}}}\sum F_{gi}
\end{align}

یاد کریں، مرکز ثقل پر عمل پیرا  قوت \عددی{\vec{F}_g} سے پیدا قوت مروڑ  اس صافی  قوت مروڑ کے برابر ہو گا جو جسم کے تمام ٹکڑوں پر عمل پیرا قوت \عددی{\vec{F}_{g}} پیدا کرتی ہیں۔ (مرکز ثقل کی تعریف یہی ہے۔) یوں مساوات \حوالہ{مساوات_توازن_ثبوت_ب}  کا \عددی{\tau}، مساوات \حوالہ{مساوات_توازن_ثبوت_صفر}  کے  \عددی{\tau_{\text{\RL{صافی}}}}  کے برابر ہے۔ دونوں مساوات کو برابر رکھ کر ذیل لکھا جا سکتا ہے۔
\begin{align*}
x_{\text{\RL{مرکزثقل}}}\sum F_{gi}=\sum x_iF_{gi}
\end{align*}
\عددی{F_{gi}} کی جگہ \عددی{m_ig_i} ڈال کر ذیل حاصل ہو گا۔
%eq12.14
\begin{align}
x_{\text{\RL{مرکزثقل}}}\sum m_ig_i=\sum x_im_ig_i
\end{align}
اب کلیدی تصور پیش کرتے ہیں: اگر  ٹکڑوں کا مقامات   پر اسراع \عددی{g_i} ایک ہو، ہم \عددی{g_i} منسوخ کرکے ذیل لکھ سکتے ہیں۔
%eq12.15
\begin{align}\label{مساوات_توازن_ثبوت_ج}
x_{\text{\RL{مرکزثقل}}}\sum m_i=\sum x_im_i
\end{align}
تمام ٹکڑوں کی کمیتوں کا مجموعہ \عددی{\sum m_i} جسم کی کمیت \عددی{M} دیتا ہے۔ یوں  مساوات \حوالہ{مساوات_توازن_ثبوت_ج} ذیل لکھی جا سکتی ہے۔
%eq12.16
\begin{align}
x_{\text{\RL{مرکزثقل}}}=\frac{1}{M}\sum x_im_i
\end{align}
اس مساوات کا دایاں ہاتھ جسم کے مرکز ثقل (مساوات \حوالہء{9.4}) کا محدد \عددی{x_{\text{\RL{مرکزثقل}}}}  دیتی ہے۔ یوں  ثبوت مکمل ہوتا ہے۔ اگر جسم کے تمام ٹکڑوں کے مقام  پر تجاذبی اسراع ایک ہو، جسم کا مرکز ثقل اور مرکز کمیت  مماثل ہوں گے۔
%eq12.17
\begin{align}
x_{\text{\RL{مرکزثقل}}}=x_{\text{\RL{مرکزکمیت}}}
\end{align}

%-------------------------
%12.2 some examples of static equilibrium p332
\حصہ{سکونی توازن کی چند مثالیں}
\موٹا{مقاصد}\\
اس حصہ کو پڑھنے کے بعد آپ ذیل کے قابل ہوں گے۔
\begin{enumerate}[1.]
\item
سکونی توازن کے لئے قوت اور قوت مروڑ کی شرائط   کا اطلاق کر پائیں گے۔
\item
سمجھ پائیں گے کہ مبدا (جس کے لحاظ سے قوت مروڑ کا حساب کیا جائے گا)  کا مقام سوچ سمجھ کر منتخب کرنے سے ایک یا ایک سے زیادہ نا معلوم قوت کو قوت مروڑ کی مساوات سے خارج کرنا ممکن ہو گا، جس سے قوت مروڑ کا حساب آسان ہو گا۔
\end{enumerate}

\موٹا{کلیدی تصور}\\
\begin{itemize}
\item
جب استوار جسم   ساکن حالت میں ہو ہم کہتے ہیں وہ سکونی توازن  میں ہے۔ایسے جسم کے لئے، جسم پر بیرونی قوتوں کا سمتی مجموعہ صفر کے برابر ہو گا۔
\begin{align*}
\vec{F}_{\text{\RL{صافی}}}=0 \quad\quad\text{\RL{(متوازن قوت)}}
\end{align*}
اگر تما قوت \عددی{xy} مستوی میں ہوں،درج بالا  سمتی مساوات   ذیل دو جزوی مساوات کے  مترادف ہو گی۔
\begin{align*}
F_{\text{\RL{صافی}},y}=0\quad \text{\RL{اور}}\quad F_{\text{\RL{صافی}},x}=0 \quad\quad\text{\RL{(متوازن قوت)}}
\end{align*}
\item
سکونی توازن  سے یہ بھی مراد ہے کہ، \ترچھا{ کسی بھی } نقطہ کے لحاظ سے،   جسم پر بیرونی قوت مروڑ کا سمتی مجموعہ  صفر کے برابر ہو گا۔
\begin{align*}
\vec{\tau}_{\text{\RL{صافی}}}=0\quad\quad\text{\RL{(متوازن قوت مروڑ)}}
\end{align*}
اگر بیرونی قوت \عددی{xy} مستوی میں ہوں، تمام قوت مروڑ محور \عددی{z} کے متوازی ہوں گی، اور درج بالا سمتی مساوات  ذیل  جزوی مساوات کی  مماثل ہو گی۔
\begin{align*}
\tau_{\text{\RL{صافی}},z}=0\quad\quad\text{\RL{(متوازن قوت مروڑ)}}
\end{align*}
\end{itemize}

%--------------------
%Some examples of static equilibrium p332
\جزوحصہء{سکونی توازن کی چند مثالیں}
یہاں ہم سکونی توازن کے کئی نمونی مسائل  پر  غور کریں گے۔ہر مسئلے میں ایک یا ایک سے زیادہ اجسام پر مبنی نظام منتخب کر کے توازن کی مساوات (مساوات \حوالہ{مساوات_توازن_شرائط_الف}، مساوات \حوالہ{مساوات_توازن_شرائط_ب}، اور مساوات \حوالہ{مساوات_توازن_شرائط_پ}) کا اطلاق کریں گے۔  تمام قوت \عددی{xy} مستوی میں ہیں لہٰذا قوت مروڑ \عددی{z} محور کو متوازی ہوں گے۔ یوں، مساوات \حوالہ{مساوات_توازن_شرائط_پ} کا اطلاق  کرتے ہوئے ، ہم  محور \عددی{z} کے متوازی  قوت مروڑ کی محور منتخب کرتے ہیں۔اگرچہ محور \عددی{z}  کے متوازی ہر محور پر مساوات  \حوالہ{مساوات_توازن_شرائط_پ}  کا اطلاق ممکن ہے، جیسا آپ دیکھیں گے، بعض محور کے انتخاب کی صورت میں   ایک یا ایک سے زیادہ نا معلوم قوت   خارج  ہوں گی، جس کی بدولت  مساوات  \حوالہ{مساوات_توازن_شرائط_پ} کا حل نسبتاً آسان ہو گا۔

%--------------
%Checkpoint 2 p332
\ابتدا{آزمائش}
یکساں سلاخ ، جو  سکونی توازن میں ہے، کا فضائی جائزہ شکل \حوالہء{؟؟} میں پیش ہے۔ (ا)  کیا قوتوں کو متوازن کر کے آپ \عددی{\vec{F}_1} اور \عددی{\vec{F}_2} کی قدریں  تلاش کر سکتے ہیں؟ (ب)  \عددی{\vec{F}_2} کی قدر تلاش کرنے کے لئے، محور گھماو کس نقطہ پر رکھ کر \عددی{\vec{F}_1} کو مساوات سے خارج کیا جا سکتا ہے؟ (ج)  \عددی{\vec{F}_2} کی قدر \عددی{\SI{65}{\newton}} حاصل ہو گی۔ \عددی{\vec{F}_1} کی قدر کیا ہے؟
\انتہا{آزمائش}
%-----------------------

%sample problem 12.01 balancing a horizontal beam p333
\ابتدا{نمونی سوال}\موٹا{افقی شہتیری   متوازن بنانا}\\
شکل \حوالہء{12.5a} میں، کمیت \عددی{m=\SI{1.8}{\kilo\gram}} کی یکساں  شہتیری ، جس کی لمبائی \عددی{L} ہے، دو ترازو پر رکھی گئی ہے۔ کمیت \عددی{M=\SI{2.7}{\kilo\gram}}  کی یکساں سل شہتیری پر رکھی گئی ہے۔سل کا مرکز شہتیری کے  بائیں سر سے \عددی{L\!/\!4}  فاصلے پر ہے۔ ترازو کیا وزن دیں گے؟

\جزوحصہء{کلیدی تصورات}
سکونی توازن کا کوئی بھی مسئلہ حل کرنے سے پہلے ذیل کرنا ہو گا: نظام کی نشاندہی  کریں اور اس کا آزاد جسمی  خاکہ بنائیں، جس پر تمام قوتوں  کی نشاندہی   ہو۔ یہاں ہم شہتیری اور سل کو  نظام مانتے ہیں۔ اس کے بعد ، نظام پر قوت دکھائیں، جیسا شکل \حوالہء{12.5b}  کے آزاد جسمی خاکہ میں کیا گیا ہے۔(نظام کے  انتخاب کے لئے تجربہ درکار ہے، اور عموماً ایک سے زیادہ  ممکنات ہوں گے۔) نظام سکونی توازن میں ہے ، لہٰذا قوتوں کے توازن  کی مساوات  (مساوات \حوالہ{مساوات_توازن_شرائط_الف} اور مساوات \حوالہ{مساوات_توازن_شرائط_ب}) اور قوت مروڑ کے توازن کی مساوات (مساوات \حوالہ{مساوات_توازن_شرائط_پ}) کا اطلاق کیا جا سکتا ہے۔

\موٹا{حساب:}\quad
بائیں ترازو سے شہتیری پر عمودی قوت \عددی{\vec{F}_l} اور  دائیں ترازو سے عمودی  قوت  \عددی{\vec{F}_r} ہے۔ہم ان قوت کی قدریں جاننا چاہتے ہیں۔ تجاذبی قوت \عددی{\vec{F}_{g,\text{\RL{شہتیر}}}}،    جو  \عددی{m\vec{g}} کے برابر ہے، شہتیری  کے مرکز کمیت پر عمل کرتی ہے۔ اسی طرح،  سل  پر تجاذبی قوت \عددی{\vec{F}_{g,\text{\RL{سل}}}}، جو \عددی{M\vec{g}} کے برابر ہے، سل کے مرکز کمیت پر عمل کرتی ہے۔تاہم،  شکل \حوالہء{12.5b} سادہ بنانے کی غرض سے، سل کو نقطہ سے ظاہر کیا گیا ہے، اور سمتیہ \عددی{\vec{F}_{g,\text{\RL{سل}}}}  کی دم اس نقطہ پر رکھی گئی ہے۔ (سمتیہ  \عددی{\vec{F}_{g,\text{\RL{سل}}}} کا رخ تبدیل کیے بغیر،  قوت کے خط عمل پر سمتیہ   کی  گھساٹ ، شکل کو عمود دار کسی بھی محور پر،  \عددی{\vec{F}_{g,\text{\RL{سل}}}} کی قوت مروڑ تبدیل نہیں کرتی۔)

قوتوں کا \عددی{x} جزو موجود نہیں لہٰذا  مساوات \حوالہ{مساوات_توازن_شرائط_الف}  \عددی{(F_{\text{\RL{صافی}},x}=0)} کوئی معلومات فراہم نہیں کرتی۔ مساوات \حوالہ{مساوات_توازن_شرائط_ب} \عددی{(F_{\text{\RL{صافی}},y}=0)}    \عددی{y} اجزاء کے لئے ذیل دیتی ہے۔
%eq12.18
\begin{align}\label{مساوات_توازن_نمونی_الف}
F_l+F_r-Mg-mg=0
\end{align}

اس مساوات میں دو نا معلوم   قوت، \عددی{F_l} اور \عددی{F_r}،   موجود ہیں لہٰذا ہمیں قوت مروڑ کے توازن کی  مساوات \حوالہ{مساوات_توازن_شرائط_پ} بھی  استعمال کرنی ہو گی۔ ہم شکل \حوالہء{12.5} کے مستوی کو عمود دار کسی بھی محور گھماو  پر مساوات کا اطلاق کر سکتے ہیں۔ آئیں شہتیری کے بائیں سر پر محور گھماو رکھ کر حل کریں۔ ہم قوت مروڑ کو علامت مختص کرنے کا عمومی  طریقہ بروئے کار لائیں گے: اگر ساکن جسم کو  محور گھماو پر قوت مروڑ گھڑی وار گھمانے کی کوشش کرے، قوت مروڑ منفی  ہو گی؛ اگر خلاف گھڑی گھمانے کی کوشش کرے، قوت مروڑ مثبت ہو گی۔ آخر میں ہم قوت مروڑ \عددی{r_{\perp}F} روپ میں لکھتے ہیں، جہاں  \عددی{\vec{F}_l} کے لئے \عددی{r_{\perp}} کی قیمت \عددی{0}، \عددی{M\vec{g}} کے لئے \عددی{L\!/\!4}، \عددی{m\vec{g}} کے لئے \عددی{L\!/\!2}، اور \عددی{\vec{F}_r} کے لئے \عددی{L} ہے۔

 اب ہم توازن کی مساوات \عددی{\tau_{\text{\RL{صافی}}}=0}  ذیل لکھ سکتے ہیں
 \begin{align*}
 (0)(F_l)-(L\!/\!4)(Mg)-(L\!/\!2)(mg)+(L)(F_r)=0
 \end{align*}
 جو  ذیل دیگی۔
 \begin{align*}
 F_r&=\frac{1}{4}Mg+\frac{1}{2}mg\\
 &=\frac{1}{4}(\SI{2.7}{\kilo\gram})(\SI{9.8}{\meter\per\second\squared})+\frac{1}{2}(\SI{1.8}{\kilo\gram})(\SI{9.8}{\meter\per\second\squared})\\
 &=\SI{15.44}{\newton}\approx\SI{15}{\newton}\quad\quad\text{\RL{(جواب)}}
 \end{align*}
 اب \عددی{F_l} کے لئے  مساوات \حوالہ{مساوات_توازن_نمونی_الف} حل کر کے درج بالا نتیجہ پر کر کے ذیل حاصل کرتے ہیں۔
 \begin{align*}
 F_l&=(M+m)g-F_r\\
 &=(\SI{2.7}{\kilo\gram}+\SI{1.8}{\kilo\gram})(\SI{9.8}{\meter\per\second\squared})-\SI{15.44}{\newton}\\
 &=\SI{28.66}{\newton}\approx\SI{29}{\newton}\quad\quad\text{\RL{(جواب)}}
 \end{align*}
 
\ترچھا{ لائحہ  عمل پر غور کریں:} قوت  کے توازن کی مساوات لکھ کر،   دو نا معلوم  متغیرات  کی بنا، ہم  پھنس گئے۔ اگر ہم بغیر سوچے سمجھے  کسی  محور پر قوت مروڑ کے توازن کی مساوات لکھتے، ہمیں  وہاں بھی دو نا معلوم متغیرات کا سامنا ہوتا۔ تاہم،  ایک نا معلوم قوت، جو یہاں \عددی{\vec{F}_l}  ہے،  کے نقطہ اطلاق سے گزرتی محور منتخب کر کے، ہم پھنسنے سے بچ گئے۔اس انتخاب کی بدولت، قوت مروڑ کے توازن کی مساوات سے نا معلوم   قدر \عددی{F_l} خارج ہوتی ہے، اور یوں ہم مساوات حل کر کے \عددی{F_r} دریافت کرنے میں کامیاب ہوئے۔ اس کے بعد، قوت کے توازن کی مساوات دوبارہ لیتے ہوئے باقی قوت کی قدر معلوم کرنا ممکن ہوا۔
\انتہا{نمونی سوال}
%-------------------------

%sample problem 12.02 balancing a lean boom p334
\ابتدا{نمونی سوال}\موٹا{چول دار بازو  متوازن بنانا}\\
شکل \حوالہء{12.6a} میں (کمیت \عددی{M=\SI{430}{\kilo\gram}} کی)  تجوری کو معاون  چول دار بازو سے     بلا کمیت رسّی   کے ذریعے لٹکا دکھایا گیا 
ہے، جہاں \عددی{a=\SI{1.9}{\meter}} اور \عددی{b=\SI{2.5}{\meter}} ہے۔ بازو کی کمیت \عددی{m=\SI{85}{\kilo\gram}} ، اور افقی رسا  بلا کمیت ہے۔

(ا) رسا میں تناو \عددی{T_c} کیا ہے؟ دوسرے لفظوں میں بازو پر رسا کی قوت  \عددی{\vec{T}_c} کی قدر  کیا ہے؟

\جزوحصہء{کلیدی تصورات}
یہاں نظام چول دار بازو ہے، جس پر عمل پیرا قوت شکل \حوالہء{12.6b} کے آزاد جسمی خاکے میں پیش ہیں۔ رسا سے  بازو پر قوت \عددی{\vec{T}_c} ہے۔ تجاذبی قوت  جو \عددی{m\vec{g}} کے برابر ہے، بازو کے مرکز کمیت ( بازو کے وسط)  پر عمل کرتی ہے۔ چول سے بازو پر  قوت کا انتصابی جزو  \عددی{\vec{F}_v}،  اور  افقی جزو \عددی{\vec{F}_h} ہے۔ رسّی سے بازو پر قوت \عددی{\vec{T}_r} ہے۔ بازو، رسّی، اور تجوری  ساکن ہیں، لہٰذا \عددی{\vec{T}_r} کی قدر تجوری کے وزن کے برابر:\عددی{T_r=Mg} ہو گی۔ ہم \عددی{xy} محددی نظام کا مبدا \عددی{O} چول پر رکھتے ہیں۔نظام سکونی توازن  میں ہے، لہٰذا  اس پر توازن کی  مساوات کا اطلاق ہو گا۔

\موٹا{حساب:}\quad
مساوات \حوالہ{مساوات_توازن_شرائط_پ}  \عددی{(\tau_{\text{\RL{صافی}},z}=0)} سے آغاز کرتے ہیں۔  یاد رہے، ہم قوت \عددی{\vec{T}_c} کی قدر جاننا چاہتے ہیں، نا کہ  نقطہ \عددی{O} پر موجود چال پر عمل پیرا قوت \عددی{\vec{F}_v} اور \عددی{\vec{F}_h} کی قدریں۔ قوت مروڑ کے حساب سے   \عددی{\vec{F}_v} اور \عددی{\vec{F}_h} خارج کرنے کی غرض سے ہم نقطہ \عددی{O} سے گزرتی، شکل کے مستوی کو عمود دار محور گھماو منتخب کرتے ہیں۔ یوں  \عددی{\vec{F}_v} اور \عددی{\vec{F}_h} کے  معیار اثر کا بازو صفر ہوں گے۔ شکل \حوالہء{12.6b} میں \عددی{\vec{T}_c}، \عددی{\vec{T}_r}، اور \عددی{m\vec{g}} کے خط عمل نقطہ دار ہیں۔ مطابقتی معیار اثر کا بازو \عددی{a}، \عددی{b}، اور \عددی{b\!/\!2} ہیں۔

قوت مروڑ کو \عددی{r_{\perp}F} روپ میں لکھ کر، قوت مروڑ کی علامت کا قاعدہ استعمال کر کے، توازن کی مساوات \عددی{\tau_{\text{\RL{صافی}},z}=0} ذیل لکھی جائے گی۔
%eq12.19
\begin{align}
(a)(T_c)-(b)(T_r)-(\tfrac{1}{2}b)(mg)=0
\end{align}
\عددی{T_r} کی جگہ \عددی{Mg}  ڈال کر \عددی{T_c} کے لئے حل کر کے ذیل حاصل ہو گا۔
\begin{align*}
T_c&=\frac{gb(M+\tfrac{1}{2}m)}{a}\\
&=\frac{(\SI{9.8}{\meter\per\second\squared})(\SI{2.5}{\meter})(\SI{430}{\kilo\gram}+85\!/\!2\,\si{\kilo\gram})}{\SI{1.9}{\meter}}\\
&=\SI{6093}{\newton}\approx\SI{6100}{\newton}\quad\quad\text{\RL{(جواب)}}
\end{align*}

(ب) چول سے بازو پر صافی قوت کی قدر \عددی{F} تلاش کریں۔

\جزوحصہء{کلیدی تصور}
اب ہمیں افقی جزو \عددی{F_h} اور انتصابی جزو \عددی{F_v} درکار ہیں، جن سے صافی قوت کی قدر \عددی{F} حاصل ہو گی۔ ہم \عددی{T_c} جانتے ہیں لہٰذا  بازو  پر قوت کی توازن کی  مساوات کا اطلاق کرتے ہیں۔

\موٹا{حساب:}\quad
افقی توازن کے لئے، ہم \عددی{F_{\text{\RL{صافی}},x}=0} ذیل لکھ سکتے ہیں:
%eq12.20
\begin{align}
F_h-T_c=0
\end{align}
اور یوں ذیل ہو گا۔
\begin{align*}
F_h=T_c=\SI{6093}{\newton}
\end{align*}
انتصابی جزو کے لئے ہم  \عددی{F_{\text{\RL{صافی}},y}=0}  کو درج ذیل لکھتے ہیں۔
\begin{align*}
F_v-mg-T_r=0
\end{align*}
\عددی{T_r} کی جگہ \عددی{Mg} ڈال کر \عددی{F_v} کے لئے حل کر کے ذیل حاصل ہو گا۔
\begin{align*}
F_v&=(m+M)g=(\SI{85}{\kilo\gram}+\SI{430}{\kilo\gram})(\SI{9.8}{\meter\per\second\squared})\\
&=\SI{5047}{\newton}
\end{align*}
مسئلہ فیثاغورث استعمال کر کے ذیل حاصل ہو گا۔
\begin{align*}
F&=\sqrt{F_h^2+F_v^2}\\
&=\sqrt{(\SI{6093}{\newton})^2+(\SI{5047}{\newton})^2}\approx \SI{7900}{\newton}\quad\text{\RL{(جواب)}}
\end{align*}
یاد رہے، \عددی{F} کی قیمت تجوری اور بازو کے مجموعی  وزن: \عددی{\SI{5000}{\newton}} ، یا افقی رسا میں تناو: \عددی{\SI{6100}{\newton}} سے کافی زیادہ ہے۔
\انتہا{نمونی سوال}
%-----------------------------

%sample problem 12.03 Balancing a leaning ladder p335
\ابتدا{نمونی سوال}\موٹا{دیوار کے ساتھ کھڑی سیڑھی}\\
شکل \حوالہء{12.7a} میں سیڑھی، جس کی لمبائی \عددی{L=\SI{12}{\meter}} اور  کمیت \عددی{m=\SI{45}{\kilo\gram}} ہے، چکنی      دیوار کے ساتھ کھڑی ہے (چکنی دیوار اور سیڑھی کے بیچ رگڑ نہیں ہو گی)۔ سیڑھی کا بالا سر فرش سے \عددی{h=\SI{9.3}{\meter}} بلندی پر ہے ، اور  سیڑھی  کا مرکز کمیت نچلے سر سے سیڑھی کے ہمراہ  \عددی{L\!/\!3} فاصلے پر ہے۔ فرش بلا رگڑ نہیں ہے۔ ایک شخص، جس کی کمیت \عددی{M=\SI{72}{\kilo\gram}} ہے،  سیڑھی چڑھتا ہے حتٰی کہ ، سیڑھی کے نچلے سر سے  شخص  کا مرکز کمیت \عددی{L\!/\!2} فاصلے پر ہوتا  ہے۔ سیڑھی پر دیوار اور فرش سے قوتوں کی قدریں کیا ہوں گی؟

\جزوحصہء{کلیدی تصورات}
ہم شخص اور سیڑھی کو  اپنا نظام   مان کر  نظام کا آزاد جسمی خاکہ، جس پر  عمل پیرا قوت دکھائے گئے ہیں،   بناتے ہیں (شکل \حوالہء{12.7b})۔ نظام سکونی توازن میں ہے، لہٰذا  اس پر قوت کی توازن اور قوت مروڑ کی توازن کی مساوات (مساوات \حوالہ{مساوات_توازن_شرائط_الف} تا مساوات \حوالہ{مساوات_توازن_شرائط_پ}) کا اطلاق  ممکن ہے۔

\موٹا{حساب:}\quad
شکل \حوالہء{12.7b} میں  شخص کو سیڑھی پر نقطے سے ظاہر کیا گیا ہے۔ شخص  پر تجاذبی قوت \عددی{M\vec{g}}  کے سمتیہ کو خط عمل (سمتیہ قوت سے گزرتی  اور اس کے ہمراہ لکیر)  پر گھسیٹ کر، سمتیہ کی دم  نقطے پر رکھی گئی ہے۔ (قوت یوں منتقل کرنے سے، شکل کو عمود دار،  کسی بھی محور گھماو کے لحاظ سے قوت مروڑ تبدیل نہیں ہوتی۔ یوں ، قوت مروڑ کی   توازن  کی مساوات، جو ہم استعمال کریں گے،  اثر انداز نہیں ہوتی۔)

دیوار سے سیڑھی پر صرف افقی قوت \عددی{\vec{F}_w} عمل کرتی ہے (بلا رگڑ دیوار پر رگڑی قوت موجود نہیں ہو سکتی، لہٰذا  سیڑھی پر دیوار کے ہمراہ انتصابی قوت صفر ہو گی)۔ فرش سے سیڑھی پر  قوت \عددی{\vec{F}_p} کا  افقی جزو \عددی{\vec{F}_{px}} ہے جو سکونی رگڑی قوت ہے، اور انتصابی جزو \عددی{\vec{F}_{py}}  ہے  جو عمودی قوت ہے۔

توازن کی مساوات  استعمال کرنے کی خاطر،  ہم مساوات \حوالہ{مساوات_توازن_شرائط_پ} \عددی{(\tau_{\text{\RL{صافی}},z}=0)}    سے آغاز کرتے ہیں، جو  قوت مروڑ کی توازن کی مساوات  ہے۔ قوت مروڑ کے حساب کے لئے محور گھماو منتخب کرتے وقت، یاد رہے سیڑھی کے دو سروں
 پر دو  نا معلوم  قوت (\عددی{\vec{F}_w} اور \عددی{\vec{F}_p}) پائے جاتے ہیں۔ ان میں سے ایک، مثلاً \عددی{\vec{F}_p}، خارج کرنے کے لئے ہم محور گھماو، شکل کے مستوی کو عمود دار، نقطہ \عددی{O} پر رکھتے ہیں (شکل \حوالہء{12.7b})۔ ہم محددی نظام کا مبدا بھی \عددی{O} پر رکھتے ہیں۔
  ہم  \عددی{O} پر قوت مروڑ مساوات \حوالہ{مساوات_گھماو_قوت_مروڑ_فائے} تا مساوات \حوالہ{مساوات_گھماو_صافی_قوت_مروڑ_الف}  میں سے کوئی ایک استعمال کر کے معلوم کر سکتے ہیں، تاہم  یہاں مساوات \حوالہ{مساوات_گھماو_صافی_قوت_مروڑ_الف} \عددی{(\tau=r_{\perp}F)}  کا استعمال  سب سے آسان ہے۔ \ترچھا{مبدا کا مقام سوچ سمجھ کر منتخب کرنے سے  قوت مروڑ کا حساب آسان بنایا جا سکتا ہے۔}
  
  دیوار سے افقی قوت \عددی{\vec{F}_w} کا معیار اثر کا بازو \عددی{r_{\perp}} معلوم کرنے کے لئے، ہم اس سمتیہ  کے اندر گزرتا خط عمل کھینچتے ہیں (شکل \حوالہء{12.7c} میں  اس کو نقطہ دار لکیر سے ظاہر کیا گیا ہے)۔ یوں \عددی{O} سے خط عمل تک عمود دار فاصلہ \عددی{r_{\perp}} ہو گا۔  شکل \حوالہء{12.7c} میں    \عددی{r_{\perp}}  محور \عددی{y} کے ہمراہ ، \عددی{h} کے برابر ہے۔ ہم اسی طرح تجاذبی قوت \عددی{M\vec{g}} اور \عددی{m\vec{g}}  کے خط عمل کھینچ کر دیکھتے ہیں کہ ان کا معیار اثر کا بازو محور \عددی{x} کے ہمراہ ہے۔ شکل \حوالہء{12.7a} میں دی گئی \عددی{a} کے لئے، معیار اثر کا بازو بالترتیب  \عددی{a\!/\!2} (شخص نصف سیڑھی چڑھ چکا ہے) اور \عددی{a\!/\!3}  ( سیڑھی کا مرکز کمیت، سیڑھی کے نچلے سر سے،   ایک تہائی فاصلے پر ہے)ہیں۔ چونکہ \عددی{\vec{F}_{px}} اور \عددی{\vec{F}_{py}}  مبدا پر عمل پیرا ہیں لہٰذا ان کے معیار اثر کا بازو صفر ہے۔
  
  قوت مروڑ \عددی{r_{\perp}F} روپ میں لکھ کر، توازن کی مساوات \عددی{\tau_{\text{\RL{صافی}},z}=0} ذیل لکھی جائے گی۔
  %eq12.21
  \begin{align}\label{مساوات_توازن_سیڑھی_نمونی_الف}
  -(h)(F_w)+(a\!/\!2)(Mg)+(a\!/\!3)(mg)+(0)(F_{px})+(0)(F_{py})=0
  \end{align}
  (مثبت قوت مروڑ خلاف گھڑی گھماو کے مترادف اور منفی قوت مروڑ گھڑی وار گھماو کے مترادف ہے۔)
  
سیڑھی، دیوار، اور فرش  قائمہ تکون بناتے ہیں، جس پر   مسئلہ فیثاغورث  کا اطلاق ذیل دیگا۔
\begin{align*}
a=\sqrt{L^2-h^2}=\SI{7.58}{\meter}
\end{align*}
اس کے بعد، مساوات \حوالہ{مساوات_توازن_سیڑھی_نمونی_الف} ذیل دیگی۔
\begin{align*}
F_w&=\frac{ga(M\!/\!2+m\!/\!3)}{h}\\
&=\frac{(\SI{9.8}{\meter\per\second\squared})(\SI{7.58}{\meter})(72\!/\!2\,\si{\kilo\gram}+45\!/\!3\,\si{\kilo\gram})}{\SI{9.3}{\meter}}\\
&=\SI{407}{\newton}\approx\SI{410}{\newton}\quad\quad\text{\RL{(جواب)}}
\end{align*}

اب ہمیں شکل \حوالہء{12.7d} اور  قوت کی توازن کی مساوات استعمال کرنی ہو گی۔ مساوات  \عددی{F_{\text{\RL{صافی}},x}=0}  ذیل دیگی:
\begin{align*}
F_w-F_{px}=0
\end{align*}
لہٰذا ذیل ہو گا۔
\begin{align*}
F_{px}=F_w=\SI{410}{\newton}\quad\quad\text{\RL{(جواب)}}
\end{align*}
مساوات  \عددی{F_{\text{\RL{صافی}},y}=0} ذیل دیگی:
\begin{align*}
F_{py}-Mg-mg=0
\end{align*}
لہٰذا ذیل ہو گا۔
\begin{align*}
F_{py}&=(M+m)g=(\SI{72}{\kilo\gram}+\SI{45}{\kilo\gram})(\SI{9.8}{\meter\per\second\squared})\\
&=\SI{1146.6}{\newton}\approx \SI{1100}{\newton}\quad\quad\text{\RL{(جواب)}}
\end{align*}

\انتہا{نمونی سوال}
%----------------------

%sample problem 12.04 balancing the leaning tower of Pisa   p337
\ابتدا{نمونی سوال}\موٹا{پیسا کے جھکا بُرج  کا توازن}\\
فرض کریں پیسا کا بُرج، رداس \عددی{R=\SI{9.8}{\meter}}  ، کا خالی یکساں  بیلن ہے، جو \عددی{h=\SI{60}{\meter}}  بلند ہے۔ اس کا مرکز کمیت، وسطی محور پر، \عددی{h\!/\!2} بلندی پر پایا جاتا ہے۔ شکل \حوالہء{12.8a} میں بیلن  سیدھا کھڑا ہے۔ شکل \حوالہء{12.8b} میں بیلن دائیں
 طرف (بُرج کے  جنوب جانب)  \عددی{\theta=\SI{5.5}{\degree}}  جھکا  ہے، جو مرکز کمیت کو   \عددی{d} فاصلہ   دور  کرتا  ہے۔ فرض کریں، زمین صرف دو قوت بُرج پر پیدا کرتی ہے۔ عمودی قوت \عددی{\vec{F}_{NL}} بائیں (شمالی) دیوار پر، اور عمودی قوت \عددی{\vec{F}_{NR}} دائیں (جنوبی) دیوار پر۔ جھکاو کی بدولت قدر  \عددی{F_{NR}} میں کتنی  فی صد  تبدیلی رونما ہوتی ہے؟
 
 \جزوحصہء{کلیدی تصور}
 چونکہ بُرج کھڑا ہے، یہ توازن میں ہو گا اور کسی بھی نقطہ کے لحاظ سے اس کر قوت مروڑ کا مجموعہ صفر ہو گا۔
 
 \موٹا{حساب:}\quad
 ہم دائیں دیوار پر \عددی{F_{NR}} جاننا چاہتے ہیں، نا کہ بائیں دیوار پر \عددی{ّF_{NL}} لہٰذا  بائیں  طرف چول رکھ کر قوت مروڑ کا حساب کرتے ہیں۔ سیدھا کھڑے بُرج پر عمل پیرا قوت شکل \حوالہء{12.8c} میں دیے گئے ہیں۔ تجاذبی قوت  \عددی{m\vec{g}}، جو مرکز کمیت پر عمل کرتی ہے،کا خط عمل انتصابی ہے اور اس کا معیار اثر کا بازو  (چول سے خط عمل کا عمود دار فاصلہ)    \عددی{R} ہے۔ منتخب چول کے لحاظ سے  اس قوت کے ساتھ وابستہ قوت مروڑ بُرج کو گھڑی وار گھمانے کی کوشش کرتی ہے لہٰذا  یہ منفی ہو گی۔ جنوبی دیوار پر عمود دار قوت \عددی{\vec{F}_{NR}} کا خط عمل انتصابی ہے، اور اس کا معیار اثر کا بازو \عددی{2R} ہے۔ چول پر  اس قوت سے وابستہ قوت مروڑ خلاف گھڑی گھماو پیدا کرتی ہے لہٰذا یہ مثبت ہو گی۔ آئیں قوت مروڑ  کے توازن کی مساوات \عددی{(\tau_{\text{\RL{صافی}},z}=0)} لکھیں:
 \begin{align*}
 -(R)(mg)+(2R)(F_{NR})=0
 \end{align*}
 جو ذیل دیتی ہے۔
 \begin{align*}
 F_{NR}=\frac{1}{2}mg
 \end{align*}
 یہ  نتیجہ بغیر حل کیے ہم جان سکتے تھے: مرکز کمیت وسطی محور پر پایا جاتا ہے، لہٰذا دایاں طرف بیلن کا  نصف وزن اٹھاتا ہے۔
 
 شکل \حوالہء{12.8b} میں مرکز کمیت دائیں طرف منتقل  ہے، جہاں \عددی{d} ذیل ہے۔
 \begin{align*}
 d=\frac{1}{2}h\tan\theta
 \end{align*}
 قوت مروڑ کے توازن کی مساوات میں اب تجاذبی قوت کا معیار اثر کا بازو \عددی{R+d} ہو گا اور دائیں عمودی  قوت  کی قدر \عددی{F_{NR}'} نئی قیمت ہو گی (شکل \حوالہء{12.8d})۔ یوں ذیل لکھا جاتا ہے:
 \begin{align*}
 -(R+d)(mg)+(2R)(F_{NR}')=0
 \end{align*}
 جو ذیل دیگا۔
 \begin{align*}
 F_{NR}'=\frac{R+d}{2R}mg
 \end{align*}
 اس نئی قیمت کو  پرانی قیمت سے تقسیم کر کے \عددی{d} کی قیمت ڈال کر ذیل ہو گا۔
 \begin{align*}
 \frac{F_{NR}'}{F_{NR}}=\frac{R+d}{R}=1+\frac{d}{R}=1+\frac{0.5h\tan\theta}{R}
 \end{align*}
 اس میں \عددی{h=\SI{60}{\meter}} ، \عددی{R=\SI{9.8}{\meter}}، اور \عددی{\theta=\SI{5.5}{\degree}} ڈال کر ذیل نتیجہ حاصل کرتے ہیں۔
 \begin{align*}
 \frac{F_{NR}'}{F_{NR}}=1.29
 \end{align*}
 یوں ہمارے سادہ نمونہ کے تحت، اگرچہ جھکاو بہت معمولی ہے، جنوبی دیوار پر قوت میں اضافہ تقریباً \عددی{30} فی صد بڑھا ہے، جس کی وجہ سے    جنوبی دیوار   پچکنے کا خطرہ لاحق ہے۔  بارش کے ساتھ بُرج کے نیچے سے   مٹی   نکل  جانا جھکاو کی وجہ بنی ہے۔ بُرج کے نیچے پانی کے انعکاس کا نظام نسب کر کے جھکاو پر قابو پایا گیا ہے۔
\انتہا{نمونی سوال}
%-------------------------------

%12.3 Elasticity p338
\حصہ{لچک}
\موٹا{مقاصد}\\
اس حصہ کو پڑھنے کے بعد آپ ذیل کے قابل ہوں گے۔
\begin{enumerate}[1.]
\item
بلا تعین صورت جان پائیں گے۔
\item
 جبر ،   بگاڑ   ، اور  مقیاس  ینگ کے تعلق کی مساوات تناو اور داب کے لئے  استعمال کر پائیں گے۔
\item
مغلوبی مضبوطی اور اخیر مضبوطی میں فرق جان پائیں گے۔
\item
جبر ،  بگاڑ ، اور   مقیاس    قینچ کی مساوات کا اطلاق  قینچ کرنے (کاٹنے)  کے لئے کر پائیں گے۔
\item
ماقوائی داب، بگاڑ ، اور   مقیاس  حجم کے  تعلق  کی مساوات  کا  اطلاق ماقوائی  جبر  کے لئے  کر پائیں گے۔
\end{enumerate}

\موٹا{کلیدی تصورات}\\
\begin{itemize}
\item
جسم پر قوتوں کی عمل سے  جسم  کے   لچکی  رویہ (مسخ ہونے)  کو تین مقیاس لچک بیان کرتے ہیں۔ بگاڑ (لمبائی میں کسری  تبدیلی)  اور اطلاقی جبر (اکائی رقبہ پر قوت)  کا (درج ذیل)  رشتہ خطی ہے، جہاں  تناسبی مستقل   مقیاس کہلاتا ہے۔
\begin{align*}
\text{\RL{جبر}}=\text{\RL{مقیاس}} \times \text{\RL{بگاڑ}}
\end{align*}
\item
تان یا داب کی صورت میں جبر و بگاڑ کا رشتہ ذیل  لکھا جاتا ہے:
\begin{align*}
\frac{F}{A}=E\frac{\Delta L}{L}
\end{align*}
جہاں \عددی{\Delta L\!/\!L} جسم میں تان یا دب  کا  بگاڑ،  \عددی{F}  بگاڑ پیدا کرنے والی  لاگو قوت کی قدر،  \عددی{A}  عمودی تراش کا رقبہ ہے جس پر \عددی{F} (رقبے کو عمود دار)  لاگو کی گئی ہے،  اور \عددی{E} جسم کا   مقیاس ینگ ہے۔ جبر  \عددی{F\!/\!A} ہو گا۔
\item
قینچ جبر کی صورت میں   جسم  کا جبر و بگاڑ رشتہ ذیل لکھا جاتا ہے:
\begin{align*}
\frac{F}{A}=G\frac{\Delta x}{L}
\end{align*}
جہاں \عددی{\Delta x\!/\!L} جس  کا قینچ بگاڑ، \عددی{\Delta x}  لاگو قوت \عددی{\vec{F}} کے رخ میں جسم کے ایک سر کا ہٹاو ، اور \عددی{G}  جسم کا مقیاس  قینچ ہے۔ جبر  \عددی{F\!/\!A} ہو گا۔
\item
ماقوائی داب کی صورت میں جسم پر اطراف  کا سیال  جبر لاگو کرتا ہے؛  جبر و بگاڑ کا رشتہ ذیل لکھا جائے گا:
\begin{align*}
p=B\frac{\Delta V}{V}
\end{align*}
جہاں \عددی{p}  جسم پر سیال کا  دباو  (ماقوائی  جبر)ہے، \عددی{\Delta V\!/\!V} دباو   کی  پیدا   (بگاڑ)  جسم کے حجم میں مطلق کسری تبدیلی ہے، اور \عددی{B} جسم کا    مقیاس حجم  ہے۔
\end{itemize}

%---------------------------
%indeterminate structures p338
\جزوحصہء{بلا تعیین جسم}
اس باب کے مسائل  میں  ہمارے پاس صرف  تین غیر تابع مساوات  ہوں گے؛  عموماً  توازن قوت  کی  دو اور محور گھماو پر  توازن قوت مروڑ  کی  ایک مساوات ہو گی۔ یوں، اگر کسی مسئلے میں تین سے زیادہ نا معلوم متغیر ہوں، ہم اس کو حل کرنے سے قاصر ہوں گے۔

غیر  متشاکل بوجھ سے لدے ہوئی گاڑی پر غور کریں۔ اس کے چاروں پہیوں پر ایک دوسرے سے مختلف   قوت کیا ہیں؟ چونکہ ہمارے پاس صرف تین مساوات ہیں لہٰذا ان قوتوں کو معلوم کرنا ممکن نہیں۔ اسی طرح، تین پائے کے  میز  کا توازن کا مسئلہ  ہم حل کر سکتے ہیں، تاہم چار پائے کے میز کے لئے  حل ممکن نہیں ہو گا۔ اس طرح کے مسائل جن میں مساوات سے نا معلوم مقادیر  کی تعداد زیادہ ہو،\اصطلاح{ بلا تعیین }\فرہنگ{بلا تعیین}\حاشیہب{indeterminate}\فرہنگ{indeterminate} کہلاتے ہیں۔

اس کے باوجود، حقیقی دنیا میں بلا تعیین مسائل کے حل موجود ہیں۔ اگر آپ گاڑی کے پہیوں کو چار مختلف ترازو پر رکھیں، یقیناً ترازو  غیر مبہم نتائج دیں گی؛ جن کا مجموعہ عین گاڑی کے وزن کے برابر ہو گا۔ ایسی کونسی بات ہے جو ہم نہیں جانتے، اور جس کے نہ جانتے ہوئے ہم مسئلہ حل کرنے سے قاصر ہیں؟

حقیقت یہ ہے، کہے بغیر ، ہم جن اجسام پر سکونی  توازن  کی مساوات کا اطلاق  کرتے ہیں ، ہم انہیں  کامل  استوار  تصور کرتے ہیں۔یعنی ہم فرض کرتے ہیں کہ لاگو قوت ان اجسام کو کسی طرح بھی  مسخ نہیں   کرتی۔ درحقیقت کامل استوار جسم کہیں نہیں پایا جاتا۔ مثلاً، گاڑی کے پہیے آسانی سے بوجھ تلے مسخ ہو کر  سکونی  توازن  کے مقام پر بیٹھتے ہیں۔

آپ کا واسطہ چار پائے کے لڑکھڑاتے  میز  سے ضرور پڑا ہو گا۔ایک پائے کے نیچے  تہہ دار کاغذ  رکھ کر اسے مستحکم کیا جا سکتا ہے۔میز پر  ہاتھی  بٹھانے سے اگر میز ٹوٹ نہ جائے، آپ یقین کر سکتے ہیں اس کے پائے گاڑی کے پہیوں کی طرح مسخ ہوں گے۔  چاروں پائے زمین کو مس کریں گے، ان پر زمین سے عمود دار قوتیں  غیر مبہم   (اور ایک دوسرے سے مختلف) قیمت اختیار کریں گی، اور  میز لڑکھڑائے گا نہیں (شکل \حوالہء{12.9})۔ ایسی  یا اس سے ملتی جلتی  صورتوں میں، جہاں  مسخ ہونا شامل ہو،  ہم  قوت  کی انفرادی قیمت کیسے جان سکتے ہیں؟

 بلا تعیین مسئلہ حل کرنے کے لئے، توازن کی مساوات کے ساتھ ہمیں \ترچھا{ لچک  } کی معلومات  بھی بروئے کار لانی ہو گی۔  طبیعیات کی وہ شاخ جو قوت کے زیر اثر اجسام کے مسخ جانے  کی بات کرتی ہے، لچک کہلاتی ہے۔
 
 %---------------------
 %checkpoint 3 p339
 \ابتدا{آزمائش}
 چھت سے یکساں سلاخ ، جس کا وزن \عددی{\SI{10}{\newton}} ہے، دو  دھاگوں  سے لٹکایا گیا ہے ، جو سلاخ پر اوپر وار \عددی{\vec{F}_1} اور \عددی{\vec{F}_2} قوت  پیدا کرتے  ہیں۔ شکل \حوالہء{؟؟} میں سلاخ  چار مختلف  نقطوں  سے باندھ کر لٹکایا گیا ہے۔ ان میں کونسی صورت  بلا تعیین ہے، اگر ہے بھی۔(بلا تعیین صورت 
 میں ہم \عددی{\vec{F}_1} اور \عددی{\vec{F}_2} معلوم نہیں کر پائیں گے۔)
 \انتہا{آزمائش}
 %----------------------------
 
 %elasticity p339
 
 \جزوحصہء{لچک}
   بہت سارے جوہر، تین بُعدی\ترچھا{  جالی  } میں متوازن  مقامات پر بیٹھ کر  ،    دھاتی جسم، مثلاً ، کیل بناتے ہیں۔ تین بُعدی جالی تکراری  نظم و ضبط رکھتی  ہے اور اس   میں ہر جوہر  قریبی جوہر سے   مقررہ فاصلے  پر  ہو گا۔ بین جوہر قوتیں،    جنہیں شکل \حوالہء{12.10}  میں اسپرنگ سے ظاہر کیا گیا ہے، جوہر کو اپنی جگہ پر رکھتی ہیں۔ یہ جالی  حیرت کن   استواریت  رکھتی ہے؛ دوسرے لفظوں میں بین  جوہر  قوت نہایت طاقتور ہیں، اور شکل \حوالہء{12.10} میں اسپرنگ بہت زیادہ   اکڑ ہوں گے۔ یہی وجہ ہے، ہم عام اجسام، مثلاً، دھاتی سیڑھی، میز، اور چمچ  کو کامل استوار سمجھتے ہیں۔ ہاں  ربڑ اور پلاسٹک   کے اجسام  ہمیں استوار نظر نہیں آتے۔ان اجسام کے جوہر شکل \حوالہء{12.10} کی طرح  استوار جالی  \ترچھا{ نہیں } بناتے؛  بلکہ   یہ سالماتی لچکیلی  زنجیر  بناتے ہیں جو قریبی زنجیر  کے ساتھ   ڈھیلی   جفتگی  رکھتے ہیں۔
   
   حقیقی \قول{استوار} اجسام کسی حد تک لچکی ہوں گے، اور یوں انہیں دبا کر، تان کر، اور مروڑ کر ہم ان کی شکل و صورت معمولی تبدیل کر سکتے ہیں۔ درپیش مقادیر جاننے کے لئے، چھت سے انتصابی   لٹکی فولادی سلاخ پر غور کرتے ہیں، جس کی لمبائی \عددی{\SI{1}{\meter}} اور  قطر  \عددی{\SI{1}{\centi\meter}} ہے۔ سلاخ   کے سر سے چھوٹی گاڑی لٹکانے سے  سلاخ کی لمبائی میں \عددی{\SI{0.5}{\milli\meter}} یعنی \عددی{\SI{0.05}{\percent}} کا اضافہ  ہو گا۔مزید، گاڑی   ہٹانے  پر سلاخ واپس اپنی اصل لمبائی اختیار کرتی ہے۔
   
سلاخ سے دو گاڑیاں لٹکانے پر، سلاخ ہمیشہ کے لئے  کھنچ   جاتی ہے، اور وزن ہٹانے سے اصل لمبائی اختیار نہیں کرتی۔ تین گاڑیاں لٹکانے  پر، سلاخ ٹوٹ جائے گی۔ عین ٹوٹنے سے قبل، لمبائی میں اضافہ \عددی{\SI{0.2}{\percent}} سے کم ہو گا۔ اگرچہ جسامت  کے اضلاع میں تبدیلی زیادہ نہیں، انجینئری میں اس کے دور رس نتائج ہوں گے۔ (آیا جہاز کا پَر جہاز کے ساتھ جڑا رہے گا، یقیناً، یہ  اہمیت کے حامل بات  ہے۔)

%p340
\موٹا{تین طریقے۔}\quad
قوت لاگو کرنے پر،  ٹھوس جسم  کا  طول و عرض تین طرح تبدیل ہو سکتا ہے۔ شکل \حوالہء{12.11a} میں  بیلن  کھینچ کر لمبا کیا گیا ہے۔ شکل \حوالہء{12.11b} میں  بیلن کی لمبی محور  کو عمود دار قوت لاگو کر کے بیلن  مسخ کیا گیا ہے۔ شکل \حوالہء{12.11c} میں  سیال میں ٹھوس جسم رکھ کر بلند داب کے زیر اثر تمام اطراف سے جسم دبایا گیا ہے۔  مسخ ہونے  کی ان تین صورتوں میں  یہ بات مشترک ہے کہ   \اصطلاح{جبر}\فرہنگ{جبر}\حاشیہب{stress}\فرہنگ{stress}، یعنی اکائی رقبہ پر لاگو قوت، جسم میں \اصطلاح{بگاڑ }\فرہنگ{بگاڑ}\حاشیہب{strain}\فرہنگ{strain} (کسری تبدیلی) پیدا کرتا  ہے۔ شکل \حوالہء{12.11a}   میں  \ترچھا{تناوی جبر} ، شکل \حوالہء{12.11b} میں \ترچھا{  قینچ جبر}، اور شکل \حوالہء{12.11c} میں \ترچھا{  ماقوائی جبر  } دکھایا گیا ہے۔

جبر اور بگاڑ  تینوں صورتوں میں مختلف  روپ اختیار کرتے ہیں، تاہم انجینئری کے مقاصد کے لئے  جبر اور بگاڑ  راست متناسب ہیں۔راست  متناسب کا مستقل
\اصطلاح{ مقیاس لچک }\فرہنگ{مقیاس لچک}\حاشیہب{modulus of elasticity}\فرہنگ{modulus of elasticity} کہلاتا ہے۔یوں ذیل ہو گا۔
%eq 12.22
\begin{align}\label{مساوات_لچک_خطی_الف}
\text{\RL{جبر}}=\text{\RL{مقیاس}}\times \text{\RL{بگاڑ}}
\end{align}
تناوی خاصیت کے معیاری پرکھ میں   پرکھی بیلن  (جیسا شکل \حوالہء{12.12} میں دکھایا گیا ہے) پر تناوی جبر  صفر قیمت سے آہستہ آہستہ بڑھایا جاتا ہے حتٰی  کہ بیلن  ٹوٹ جائے، اور ساتھ ساتھ بگاڑ ناپ کر ترسیم کی جاتی  ہے۔ یوں شکل \حوالہء{12.13} کے  طرز کی جبر بالمقابل بگاڑ ترسیم حاصل ہو گی۔لاگو  جبر کی  وسیع  حد تک   جبر  اور بگاڑ کا تعلق خطی  ہے، اور جبر ہٹانے پر  پرکھی جسم واپس اصل   طول و عرض اختیار کرتا ہے؛ اس خطی خطہ میں مساوات \حوالہ{مساوات_لچک_خطی_الف} کا اطلاق ہو گا۔ پرکھی جسم کی \اصطلاح{  مغلوبی مضبوطی }\فرہنگ{مضبوطی!مغلوبی}\حاشیہب{yield strength}\فرہنگ{strength!yield} \عددی{S_y} سے جبر بڑھانے پر جسم ہمیشہ کے لئے مسخ ہو جاتا ہے۔ جبر  مسلسل  بڑھانے پر جب \اصطلاح{اخیر مضبوطی}\فرہنگ{مضبوطی!اخیر}\حاشیہب{ultimate strength}\فرہنگ{strength!ultimate} \عددی{S_u}  کو پہنچتی ہے، جسم  ٹوٹ جاتا ہے۔

%-----------------
%tension and compression  p340

\جزوجزوحصہء{تان اور داب}
سادہ تان یا داب کے لئے، جسم پر جبر کی تعریف \عددی{F\!/\!A} ہے، جہاں  جسم کے رقبہ  \عددی{A} پر عمود دار قوت کی قدر \عددی{F}  ہے۔ بگاڑ سے مراد بے  بُعد مقدار \عددی{\Delta L\!/\!L} ہے، جو جسم کی لمبائی میں کسری  (یا بعض اوقات فی صد)   تبدیلی ہو گی۔اگر جسم  ایک لمبی سلاخ ہو اور جبر مغلوبی مضبوطی سے تجاوز نہ کرے،  نا صرف  پوری  سلاخ کا بگاڑ  بلکہ   اس کے  ہر  ٹکڑے   کا بگاڑ دیے گئے جبر کے لئے ایک جتنا ہو گا۔ چونکہ بگاڑ بے بُعد ہے، مساوات \حوالہ{مساوات_لچک_خطی_الف} میں مقیاس کے بُعد وہی ہو گا جو جبر کا ہے؛ یعنی قوت فی  اکائی  رقبہ۔

%p341
تناوی اور دباو  جبر کے مقیاس کو \اصطلاح{مقیاس ینگ }\فرہنگ{مقیاس!ینگ}\حاشیہب{Young's modulus}\فرہنگ{modulus!Young's} کہتے ہیں، جس کو انجینئری میں \عددی{E} سے ظاہر کیا جاتا ہے۔ یوں مساوات \حوالہ{مساوات_لچک_خطی_الف} ذیل روپ اختیار کرتی ہے۔
%eq 12.23
\begin{align}\label{مساوات_لچک_ینگ_الف}
\frac{F}{A}=E\frac{\Delta L}{L}
\end{align}
جسم میں بگاڑ \عددی{\Delta L\!/\!L}  با آسانی  \اصطلاح{بگاڑ  پیما }\فرہنگ{پیما!بگاڑ}\حاشیہب{strain gauge}\فرہنگ{gauge!strain} سے ناپا جاتا ہے (شکل \حوالہء{12.14})، جس کو جسم کے ساتھ  گوند سے چسپاں کیا جاتا ہے؛ اس کے  برقی خواص  بگاڑ پر منحصر ہیں، جنہیں ناپ کر بگاڑ ناپا جاتا ہے۔

اگرچہ  تان اور داب دونوں میں جسم کا  مقیاس ینگ تقریباً  ایک  ہو سکتا ہے، جسم کی  اخیر مضبوطی   جبر  کے دو قسموں کے لئے بالکل مختلف ہو گی۔ کانکریٹ ( کنکر اور سیمنٹ کے مسالے سے بنائی گئی)  داب میں انتہائی مضبوط لیکن تان میں اتنی کمزور کہ تان کی صورت میں   کہیں استعمال نہیں ہوتی۔ جدول \حوالہ{جدول_لچک_لچکی_خواص} میں انجینئری دلچسپی کے چند اشیاء کے مقیاس ینگ اور دیگر لچکی خواص  پیش ہیں۔

\begin{table}
\caption{انجینئری دلچسپی کے منتخب  اشیاء کے چند لچکی خواص}
\label{جدول_لچک_لچکی_خواص}
\centering
\begin{tabular}{RCCCC}
\toprule
&&\text{\RL{مقیاس}}&\text{\RL{اخیر}}&\text{\RL{مغلوبی}}\\[0.5ex]
&\rho\, \text{\RL{کثافت}}&E\,\text{\RL{ینگ}}&S_u\,\text{\RL{مضبوطی}}&S_y\,\text{\RL{مضبوطی}}\\[0.5ex]
\text{\RL{مادہ}}&(\si{\kilo\gram\per\meter\cubed})&(\SI{e9}{\newton\per\meter\squared})&(\SI{e6}{\newton\per\meter\squared})&(\SI{e6}{\newton\per\meter\squared})\\[0.5ex]
\midrule
\text{\RL{فولاد}}&7860&200&400&250\\
\text{\RL{سلور}}&2710&70&110&95\\
\text{\RL{شیشہ}}&2190&65&50&\text{\textemdash}\\
\text{\RL{کانکریٹ}}&2320&30&40&\text{\textemdash}\\
\text{\RL{لکڑ}}&525&13&50&\text{\textemdash}\\
\text{\RL{ہڈی}}&1900&9&170&\text{\textemdash}\\
\text{\RL{پالی اسٹرن}}&1050&3&48&\text{\textemdash}\\
\bottomrule
\end{tabular}
\end{table}

%-----------------------------
%shearing p341
\جزوجزوحصہء{قینچ}
قینچ کی صورت میں  بھی جبر فی اکائی رقبہ قوت ہو گا، تاہم قوت سمتیہ رقبے کے مستوی میں ہو گا نا کہ رقبے کو عمود دار۔ بگاڑ بے بُعد نسبت \عددی{\Delta x\!/\!L} ہو گی، جہاں  مقادیر کی تعریف شکل \حوالہء{12.11b} میں پیش ہے۔ مطابقتی مقیاس، جس کو انجینئری  شعبہ میں \عددی{G} سے ظاہر کیا جاتا ہے، کو
\اصطلاح{ مقیاس قینچ }\فرہنگ{مقیاس!قینچ}\حاشیہب{shear modulus}\فرہنگ{modulus!shear} کہتے ہیں۔  قینچ کے لئے مساوات \حوالہ{مساوات_لچک_خطی_الف}  ذیل لکھی جائے گی۔
%eq 12.24
\begin{align}\label{مساوات_لچک_ینگ_ب}
\frac{F}{A}=G\frac{\Delta x}{L}
\end{align}
گھماو کے دوران بوجھ پڑنے پر دھرا قینچ  سے ٹوٹتا ہے اور  جھکنے کے دوران ہڈی قینچ سے  ٹوٹتی ہے۔

%-------------------
%hydraulic streee p 341
\جزوجزوحصہء{ماقوائی جبر}
شکل \حوالہء{12.11c} میں   سیال کا دباو  (فشار سیال) \عددی{p} جسم پر جبر پیدا کرتا ہے، اور جیسا آپ باب \حوالہء{14} میں دیکھیں گے،دباو کی تعریف  اکائی رقبہ پر قوت  ہے۔ بگاڑ \عددی{\Delta V\!/\!V} ہو گا، جہاں  جسم کا اصل (ابتدائی) حجم \عددی{V} اور  حجم  میں تبدیلی کی مطلق قیمت  \عددی{\Delta V} ہے۔ مطابقتی مقیاس، جس کی علامت \عددی{B} ہے، مادے کا  \اصطلاح{  مقیاس حجم  }\فرہنگ{مقیاس!حجم}\حاشیہب{bulk modulus}\فرہنگ{modulus!bulk} کہلاتا ہے۔ ہم کہتے ہیں جسم زیر \ترچھا{  ماقوائی دباو  }ہے، اور دباو  کو \ترچھا{ ماقوائی جبر}   پکارا جا سکتا ہے۔ یہاں مساوات \حوالہ{مساوات_لچک_خطی_الف}  ذیل لکھی جائے گی۔
%eq12.25
\begin{align}\label{مساوات_لچک_ینگ_پ}
p=B\frac{\Delta V}{V}
\end{align}

پانی کے لئے مقیاس حجم \عددی{\SI{2.2e9}{\newton\per\meter\squared}} اور فولاد کے لئے عددی{\SI{1.6e11}{\newton\per\meter\squared}} ہے۔ بحر الکاہل  کی تہہ میں، جو اوسطاً \عددی{\SI{4000}{\meter}} ہے، دباو \عددی{\SI{4.0e7}{\newton\per\meter\squared}} ہو گا۔ اس گہرائی پر پانی کے  دباو  سے پانی کے حجم میں کسری تبدیلی \عددی{\Delta V\!/\!V}  کی قیمت \عددی{\SI{1.8}{\percent}} ہو گی؛ فولادی جسم کے لئے صرف   \عددی{\SI{0.025}{\percent}} ہو گی۔ عموماً، ٹھوس اجسام، استوار جوہری جالی کی بدولت،  سیال سے کم   داب پذیر ہوں گے، جس  میں جوہر یا سالمہ  قریبی جوہر یا سالمہ   کے ساتھ  کمزور جفتگی  رکھتا  ہے۔

%-----------------------------
%sample problem 12.05 stress ans strain of elongated rod p342
\ابتدا{نمونی سوال}\موٹا{کھنچی   سلاخ میں جبر اور بگاڑ}\\
فولادی سلاخ کا ایک سر  شکنجہ میں پکڑ کر، دوسرے سر کی سطح  پر \عددی{F=\SI{62}{\kilo\newton}} قدر کی   عمود دار  قوت (سطح پر یکساں)  لاگو کر کے سلاخ کھنچی  جاتی ہے۔ قوت لاگو کرنے سے قبل، سلاخ کا رداس \عددی{R=\SI{9.5}{\milli\meter}} اور لمبائی \عددی{L=\SI{81}{\centi\meter}} ہے۔ سلاخ پر جبر ، سلاخ کی طوالت، اور  بگاڑ کیا ہیں؟

\جزوحصہء{کلیدی تصورات}
(1) سلاخ کے سر پر قوت عمود دار اور یکساں ہے لہٰذا  قوت کی قدر \عددی{F} اور رقبہ \عددی{A} کی نسبت، جبر ہو گا۔ مساوات \حوالہ{مساوات_لچک_ینگ_الف} کا بایاں ہاتھ یہی نسبت ہے۔ (2)  طوالت \عددی{\Delta L}  کا جبر اور مقیاس ینگ \عددی{E} سے تعلق مساوات \حوالہ{مساوات_لچک_ینگ_الف} \عددی{(F\!/\!A=E\Delta L\!/\!L)}  دیتی ہے۔ (3)  طوالت اور اصل لمبائی \عددی{L} کی نسبت بگاڑ ہو گا۔

\موٹا{حساب:}\quad
جبر تلاش کرنے کے لئے ہم درج ذیل لکھتے ہیں۔
\begin{align*}
\text{\RL{جبر}}&=\frac{F}{A}=\frac{F}{\pi R^2}=\frac{\SI{6.2e4}{\newton}}{(\pi)(\SI{9.5e-3}{\meter})^2}\\
&=\SI{2.2e8}{\newton\per\meter\squared}\quad\quad\text{\RL{(جواب)}}
\end{align*}
فولاد کا  مغلوبی جبر \عددی{\SI{2.5e8}{\newton\per\meter\squared}} ہے، لہٰذا  سلاخ ٹوٹنے کے قریب ہو گی۔

فولاد کا مقیاس ینگ جدول \حوالہ{جدول_لچک_لچکی_خواص} سے دیکھ کر معلوم کرتے ہیں۔ یوں مساوات \حوالہ{مساوات_لچک_ینگ_الف} سے ذیل طوالت حاصل ہو گی۔
\begin{align*}
\Delta L&=\frac{(F\!/\!A)L}{E}=\frac{(\SI{2.2e8}{\newton\per\meter\squared})(\SI{0.81}{\meter})}{\SI{2.0e11}{\newton\per\meter\squared}}\\
&=\SI{8.9e-4}{\meter}=\SI{0.89}{\milli\meter}\quad\quad\text{\RL{(جواب)}}
\end{align*}
بگاڑ ذیل ہو گا۔
\begin{align*}
\frac{\Delta L}{L}&=\frac{\SI{8.9e-4}{\meter}}{\SI{0.81}{\meter}}\\
&=\num{1.1e-3}=\SI{0.11}{\percent}\quad\quad\text{\RL{(جواب)}}
\end{align*}
\انتہا{نمونی سوال}
%----------------------------

%sample problem 12.06 balancing a wobbly table p342
\ابتدا{نمونی سوال}\موٹا{لڑکھڑاتے میز کا  توازن}\\
ایک میز کے تین پائے \عددی{\SI{1.00}{\meter}} اور چوتھا \عددی{d=\SI{0.50}{\milli\meter}} زیادہ لمبا ہے، جس کی بدولت میز لڑکھڑاتا ہے۔میز پر   \عددی{M=\SI{290}{\kilo\gram}} کمیت کا فولادی بیلن رکھنے سے میز کے چاروں پائے زیر داب ہوتے ہیں اور میز لڑکھڑانا  بند کرتا ہے۔ (میز کی کمیت \عددی{M} سے بہت کم ہے۔)  پائے لکڑ کے ہیں اور ان کا  عمودی تراش رقبہ \عددی{A=\SI{1.0}{\centi\meter\squared}}  اور مقیاس ینگ \عددی{E=\SI{1.3e10}{\newton\per\meter\squared}} ہے۔ فرش سے میز کے پایوں پر قوت کی قدریں معلوم کریں۔

\جزوحصہء{کلیدی تصورات}
ہم میز اور فولادی بیلن کو  نظام مانتے ہیں۔ صورت حال شکل \حوالہء{12.9} کی طرح ہے؛ فرق صرف اتنا ہے کہ یہاں میز پر فولادی بیلن رکھا گیا ہے۔ اگر میز   کا بالا تختہ استوا سطح   ہو، پائے لازماً  درج ذیل طرح زیر داب ہوں گے: ہر چھوٹا پایا  برابر دبا ہو گا (ہم اس مقدار کو \عددی{\Delta L_3} لکھتے ہیں) لہٰذا  تینوں پر قوت  کی  ایک جتنی  قدر \عددی{F_3} عمل کرے گی۔واحد لمبا پایا  زیادہ دبے گا (جس کو ہم \عددی{\Delta L_4} لکھتے ہیں) لہٰذا اس پر قوت کی قدر \عددی{F_4} بڑی ہو گی۔دوسرے لفظوں میں میز کا تختہ  استوا  ہونے کی صورت میں ذیل ہو گا۔
%eq 12.26
\begin{align}\label{مساوات_لچک_پایا_لمبائی_الف}
\Delta L_4=\Delta L_3+d
\end{align}

مساوات \حوالہ{مساوات_لچک_ینگ_الف}  ،    لمبائی میں تبدیلی اور  اس  تبدیلی کو پیدا کرنے والی قدر کا تعلق \عددی{\Delta L=FL\!/\!AE}   دیتی  ہے، جہاں پائے کی اصل لمبائی \عددی{L} ہے۔ مساوات \حوالہ{مساوات_لچک_پایا_لمبائی_الف} میں \عددی{\Delta L_4} اور \عددی{\Delta L_3}  کے لئے یہ  تعلق استعمال کیا جا سکتا ہے۔

\موٹا{حساب:}\quad
یوں  اس تعلق سے ذیل لکھا جا سکتا ہے۔
%eq 12.27
\begin{align}\label{مساوات_لچک_پایا_لمبائی_ب}
\frac{F_4L}{AE}=\frac{F_3L}{AE}+d
\end{align}
اس مساوات میں دو نا معلوم مقادیر  ، \عددی{F_4} اور \عددی{F_3} ، ہیں لہٰذا اسے حل کرنا ممکن نہیں۔

\عددی{F_4} اور \عددی{F_3}  کی دوسری مساوات لکھنے کی غرض سے ہم انتصابی \عددی{y}  محور  منتخب کر کے  انتصابی  قوتوں کی توازن کی 
مساوات \عددی{(F_{\text{\RL{صافی}},y}=0)} لکھتے ہیں:
%eq12.28
\begin{align}\label{مساوات_لچک_پایا_لمبائی_پ}
3F_3+F_4-Mg=0
\end{align}
جہاں نظام پر تجاذبی قوت \عددی{Mg} کے برابر ہے۔ (\ترچھا{تین} پایوں پر قوت \عددی{\vec{F}_3} ہے۔) ہمزاد  مساوات \حوالہ{مساوات_لچک_پایا_لمبائی_ب} اور مساوات \حوالہ{مساوات_لچک_پایا_لمبائی_پ}  ، مثلاً \عددی{F_3} کے لئے ، حل کرنے کے لئے ہم مساوات \حوالہ{مساوات_لچک_پایا_لمبائی_پ}  سے \عددی{F_4=Mg-3F_3} لکھ کر مساوات \حوالہ{مساوات_لچک_پایا_لمبائی_ب} میں ڈالتے ہیں، جس کو حل کر کے ذیل حاصل ہو گا۔
\begin{align*}
F_3&=\frac{Mg}{4}-\frac{dAE}{4L}\\
&=\frac{(\SI{290}{\kilo\gram})(\SI{9.8}{\meter\per\second\squared})}{4}-\frac{(\SI{5.0e-4}{\meter})(\SI{e-4}{\meter\squared})(\SI{1.3e10}{\newton\per\meter\squared})}{(4)(\SI{1.00}{\meter})}\\
&=\SI{548}{\newton}\approx \SI{5.5e2}{\newton}\quad\quad\text{\RL{(جواب)}}
\end{align*}
مساوات \حوالہ{مساوات_لچک_پایا_لمبائی_پ} سے   ذیل حاصل ہو گا۔
\begin{align*}
F_4&=Mg-3F_3=(\SI{290}{\kilo\gram})(\SI{9.8}{\meter\per\second\squared})-3(\SI{548}{\newton})\\
&\approx \SI{1.2}{\kilo\newton}\quad\quad\text{\RL{(جواب)}}
\end{align*}
آپ دیکھ سکتے ہیں  کہ ہر چھوٹا پایا \عددی{\SI{0.42}{\meter}} اور لمبا پایا \عددی{\SI{0.92}{\meter}} دبا ہے۔
\انتہا{نمونی سوال}
%----------------------------

%review and summary p343
\جزوجزوحصہء{نظر ثانی اور خلاصہ}
\موٹا{سکونی توازن}\quad
جب استوار جسم ساکن ہو ہم کہتے ہیں وہ \اصطلاح{ سکونی توازن }\فرہنگ{توازن!سکونی}\حاشیہب{static equilibrium}\فرہنگ{equilibrium!static} میں ہے۔ایسے جسم پر بیرونی قوتوں کا سمتی مجموعہ صفر کے برابر ہو گا۔
\begin{align*}
\vec{F}_{\text{\RL{صافی}}}=0\quad\quad\text{\RL{(توازن قوت)}}\tag{\setlatin{\حوالہ{مساوات_توازن_شرط_ایک}}}
\end{align*}
اگر تمام قوت \عددی{xy} مستوی میں ہوں، یہ سمتی مساوات ذیل دو جزوی مساوات کی معادل ہے۔
\begin{align*}
F_{\text{\RL{صافی}},y}=0 \quad \text{\RL{اور}}\quad F_{\text{\RL{صافی}},x}=0\quad\text{\RL{(توازن قوت)}}\tag{\setlatin{\حوالہ{مساوات_توازن_شرائط_الف}، \, \حوالہ{مساوات_توازن_شرائط_ب}}}
\end{align*}
سکونی توازن سے یہ بھی مراد ہے کہ  کسی بھی نقطہ کے لحاظ سے جسم پر بیرونی قوت مروڑ کا سمتی مجموعہ صفر کے برابر ہو گا۔
\begin{align*}
\vec{\tau}_{\text{\RL{صافی}}}=0\quad\quad\text{\RL{(توازن قوت مروڑ)}}\tag{\setlatin{\حوالہ{مساوات_توازن_شرط_دو}}}
\end{align*}
اگر قوت \عددی{xy} مستوی میں ہوں، تمام قوت مروڑ سمتیات محور \عددی{z} کو  متوازی ہوں گے، اور مساوات \حوالہ{مساوات_توازن_شرط_دو} ذیل واحد جزوی  مساوات کی معادل ہو گی۔
\begin{align*}
\tau_{\text{\RL{صافی}},z}=0\quad\text{\RL{(توازن قوت مروڑ)}}  \tag{\setlatin{\حوالہ{مساوات_توازن_شرائط_پ}}}
\end{align*}

\موٹا{مرکز کمیت}\quad
جسم کے ہر   حصے پر تجاذبی قوت   انفرادی طور پر عمل کرتی ہے۔ انفرادی اعمال کا صافی اثر  جاننے کے لئے جسم کے \اصطلاح{ مرکز کمیت } پر مساوی کل تجاذبی قوت \عددی{\vec{F}_g} تصور کرنا ہو گا۔ اگر جسم کے تمام حصوں پر ثقلی اسراع \عددی{\vec{g}} ایک ہو، مرکز کمیت اور مرکز ثقل ایک نقطہ پر ہوں گے۔

\موٹا{مقیاس لچک}\quad
جسم پر عمل پیرا قوتوں  کو جسم کا رد عمل ، جو  مسخ ہونے کی صورت میں ہو گا،  تین \اصطلاح{  مقیاس  لچک  } سے بیان کیا جاتا ہے۔ \اصطلاح{ بگاڑ } (لمبائی میں کسری تبدیلی) اور  \اصطلاح{جبر } (اکائی رقبے پر قوت) کا تعلق خطی ہے، جہاں تناسب کا مستقل مقیاس کہلاتا ہے۔ ان کا عمومی رشتہ ذیل ہے۔
\begin{align*}
\text{\RL{جبر}}=\text{\RL{مقیاس}}\times \text{\RL{بگاڑ}}\tag{\setlatin{\حوالہ{مساوات_لچک_خطی_الف}}}
\end{align*}

\موٹا{تان اور داب}\quad
جب جسم تان یا داب   کے زیر اثر ہو، مساوات \حوالہ{مساوات_لچک_خطی_الف} ذیل لکھی جائے گی:
\begin{align*}
\frac{F}{A}=E\frac{\Delta L}{L}\tag{\setlatin{\حوالہ{مساوات_لچک_ینگ_الف}}}
\end{align*}
جہاں \عددی{\Delta L\!/\!L}  جسم کا تناوی  بگاڑ یا دباو بگاڑ ہے،  \عددی{F} لاگو قوت \عددی{\vec{F}}  کی قدر، \عددی{A}  وہ رقبہ عمودی تراش    ہے  جس پر قوت \عددی{\vec{F}}   (شکل \حوالہء{12.11a} میں عمود دار) عمل کرتی ہے، اور \عددی{E} جسم کا \اصطلاح{مقیاس ینگ } ہے۔ جبر \عددی{F\!/\!A} ہو گا۔

\موٹا{قینچ}\quad
جب جسم جبر قینچ کے زیر اثر ہو، مساوات \حوالہ{مساوات_لچک_خطی_الف} ذیل لکھی جائے گی:
\begin{align*}
\frac{F}{A}=G\frac{\Delta x}{L}
\end{align*}
جہاں \عددی{\Delta x\!/\!L} جسم  کا بگاڑ قینچ ہے، \عددی{\Delta x}  لاگو قوت \عددی{\vec{F}} کے رخ  (جیسا شکل \حوالہء{12.11b} میں دکھایا گیا ہے) جسم کے ایک سر کا ہٹاو ہے، اور \عددی{G}  جسم کا\اصطلاح{ مقیاس قینچ } ہے۔ جبر \عددی{F\!/\!A} ہو گا۔

\موٹا{ماقوائی جبر}\quad
جب  تمام اطراف سے سیال کا دباو جسم پر ماقوائی دباو ڈالتا ہو، مساوات \حوالہ{مساوات_لچک_خطی_الف} ذیل لکھی جائے گی:
\begin{align*}
p=B\frac{\Delta V}{V}\tag{\setlatin{\حوالہ{مساوات_لچک_ینگ_پ}}}
\end{align*}
جہاں جسم پر سیال کی وجہ سے دباو (\ترچھا{ماقوائی جبر}) \عددی{p} ہے، بگاڑ   \عددی{\Delta V\!/\!V}  جسم کے حجم میں  \عددی{p} کی بدولت مطلق  کسری تبدیلی  ہے، اور \عددی{B} جسم کا \اصطلاح{ مقیاس حجم }ہے۔

%--------------------------
%questions p343
\حصہء{سوالات}
\setcounter{questioncounter}{0}
%Q1 p343
\ابتدا{سوال}
سلاخ  کا ایک سر دیوار پر چول سے جوڑا گیا ہے، جبکہ سلاخ کا دوسرا سر  رسّی سے باندھا گیا ہے (شکل \حوالہء{12.15})۔ لکھ کر حساب کیے بغیر دی گئی تین صورتوں کی درجہ بندی  (ا)  رسّی سے سلاخ پر قوت ، (ب) سلاخ پر چول کے انتصابی قوت،  اور (ج) چول سے سلاخ پر افقی قوت کی قدر کے لحاظ سے ، اعظم قیمت اول رکھ کر، کریں۔
\انتہا{سوال}
%---------------------------
\ابتدا{سوال}
استوار شہتیر  دو ستون پر  رکھا گیا ہے جو زمین  میں دھنسے  ہوئے ہیں  (شکل \حوالہء{12.16})۔بھاری  تجوری چھ مختلف مقامات پر   باری باری رکھی جاتی ہے۔ شہتیر کی کمیت نظر انداز کریں۔ (ا)  ستون \عددی{A} پر زیادہ سے زیادہ داب اول رکھ کر اور زیادہ سے زیادہ تان آخر میں رکھ کر، ان مقامات  کی درجہ بندی کریں، اور وہ مقام  (اگر موجود ہو)  معلوم کریں جو ستون پر صفر قوت دیگا۔ (ب) ستون \عددی{B} پر قوت کے لحاظ سے ان مقامات کی درجہ بندی کریں۔
 \انتہا{سوال}
%---------------------------
\ابتدا{سوال}
گھومتے ہوئے یکساں قرص، جو بلا رگڑ فرش پر  پھسل کر حرکت کرتا ہے،  کے  فضائی جائزہ شکل \حوالہء{12.17} میں پیش ہیں۔ہر ایک قرص پر \عددی{F}، \عددی{2F}، اور \عددی{3F} قدر کی تین قوتیں  چکر پر، یا وسط پر، اور یا چکا اور وسط کے نصف فاصلے پر، عمل کرتی ہیں۔ قوت سمتیات قرص کے ساتھ ساتھ گھومتے ہیں، اور شکل \حوالہء{12.17} میں  ان کا لمحاتی رخ دائیں یا بائیں ہے۔ کون کون ے قرص توازن میں ہیں؟
\انتہا{سوال}
%-----------------
%Q4 p343
\ابتدا{سوال}
ایک سیڑھی بلا رگڑ دیوار کے ساتھ کھڑی ہے، جبکہ فرش کی رگڑی قوت اس کو پھسل کر گرنے سے روکتی ہے۔ فرض کریں آپ سیڑھی کے پیندا کو دیوار کے قریب لاتے ہیں۔ بتائیں درج ذیل میں کون قدر کے لحاظ سے بڑھتا ہے، گھٹتا ہے،  یا وہی رہتا ہے؟ (ا)  فرش  سے سیڑھی پر  انتصابی قوت،  (ب) دیوار سے سیڑھی پر قوت، (ج)  فرش کی سیڑھی پر سکونی رگڑی قوت، اور (د)  سکونی رگڑی قوت کی زیادہ سے زیادہ قیمت \عددی{f_{s,\text{\RL{بلندتر}}}}۔
\انتہا{سوال}
%-----------------------
%Q5 p344
\ابتدا{سوال}
چار کھلونا  پرندے  شکل \حوالہء{12.18} میں تین بلا کمیت سلاخوں سے لٹکے  دکھائے گئے ہیں۔ تمام سلاخ افقی ہیں۔ہر سلاخ ڈور سے لٹکی ہے  جو سلاخ کے بائیں سر سے ایک چوتھائی فاصلے پر باندھی گئی ہے۔ اگر پرندہ \عددی{1} کی کمیت \عددی{m_1=\SI{48}{\kilo\gram}} ہو (1) پرندہ \عددی{2}، (ب) پرندہ \عددی{3}، اور (ج) پرندہ \عددی{4} کی کمیت کیا ہو  گی؟
\انتہا{سوال}
%-------------------
\ابتدا{سوال}
یکساں ڈنڈا  ، جس پر چار قوت عمل کرتی ہیں ، کا فضائی جائزہ شکل \حوالہء{12.19} میں پیش ہے۔ فرض کریں ہم محور گھماو نقطہ \عددی{O} پر  رکھ کر  قوت مروڑ تلاش کر دیکھیں  قوت مروڑ  توازن میں ہیں۔ کیا قوت مروڑ تب بھی توازن میں ہوں گی اگر ہم محور گھماو (ا) نقطہ \عددی{A}   پر (ڈنڈے پر) ، (ب) نقطہ \عددی{B} پر  (ڈنڈے کی سیدھ میں)  ،  (ج) نقطہ \عددی{C} پر (ڈنڈے کی ایک جانب)  رکھیں؟ (د)  فرض کریں نقطہ \عددی{O} پر قوت مروڑ  توازن میں نہیں ہیں۔ کیا کوئی  نقطہ ایسا ہو گا جس پر قوت مروڑ توازن میں ہوں گی؟
\انتہا{سوال}
%-----------------------
\ابتدا{سوال}
ساکن سلاخ \عددی{AC} ، جس کی کمیت \عددی{\SI{5.0}{\kilo\gram}} ہے، دیوار کے ساتھ  رسّی  ، اور سلاخ اور دیوار کے بیچ رگڑ کی مدد سے   رکھا گیا ہے (شکل \حوالہء{12.20})۔ یکساں سلاخ کی لمبائی \عددی{\SI{1}{\meter}} ہے اور زاویہ \عددی{\theta=\SI{30}{\degree}} ہے۔ (ا)  واحد مساوات استعمال کر کے سلاخ پر رسّی کی قوت \عددی{\vec{T}} کی قدر تلاش کرنے کے لئے محور گھماو کس نقطہ پر رکھنا ہو گا؟ اس محور کے لحاظ سے، اور خلاف گھڑی قوت مروڑ مثبت لیتے ہوئے،  (ب) سلاخ کے وزن کی قوت مروڑ \عددی{\tau_w}  اور (ج)  سلاخ پر رسّی کی کھینچ کی قوت مروڑ \عددی{\tau_r} کی علامت کیا  ہو گی؟ (د)  کیا \عددی{\tau_r} کی قدر \عددی{\tau_w} کی قدر سے کم ہے، زیادہ ہے، یا دونوں برابر ہیں؟
\انتہا{سوال}
%----------------------
%Q8 p344
\ابتدا{سوال}
تین کھلونے بلا کمیت  جرثقیل  اور  ڈور   کے نظام سے لٹکے ہیں (شکل \حوالہء{12.21})۔  ایک ڈور چھت سے   لٹک کر دائیں جرثقیل  سے ہو کر آخر کار  بائیں ہاتھ نچلے جرثقیل پر اختتام پذیر ہوتی ہے۔جرثقیل کے گرد ڈور آدھا چکر لپٹی ہے۔دیگر چھوٹی ڈور سے  جرثقیل چھت سے،  یا کھلونے جرثقیل سے،  لٹکائے گئے ہیں۔ دو کھلونوں کا وزن دیا گیا ہے۔ (ا)  تیسرے کھلونے کا وزن کیا ہے؟ (\ترچھا{اشارہ:}  جب ڈور ایک جرثقیل کے گرد آدھا چکر  کاٹے، جرثقیل  پر  ڈور کی تان کی دگنی قوت عمل  کرتی ہے۔) (ب)بائیں ہاتھ   چھوٹی ڈور  ، جس کی نشاندہی \عددی{T}   سے کی گئی ہے، میں تان کتنی ہے؟
\انتہا{سوال}
%-------------------------
%Q9 p344
\ابتدا{سوال}
انتصابی سلاخ کا نچلا سر چول دار ہے جبکہ اس کا بالا سر رسّی سے بندھا ہوا ہے (شکل \حوالہء{12.22})۔  سلاخ پر افقی قوت \عددی{\vec{F}_a} لاگو کی جاتی ہے۔  قوت  لاگو کرنے کا نقطہ بلند کرنے سے کیا رسّی میں تان بڑھتی ہے، گھٹتی ہے، یا تبدیل نہیں ہوتی؟
\انتہا{سوال}
%--------------------------
\ابتدا{سوال}
افقی سل دو رسّیوں، \عددی{A} اور \عددی{B}،  سے لٹکی ہے (شکل \حوالہء{12.23})۔  ماسوائے ابتدائی لمبائیوں  کے، رسّیاں متماثل  ہیں۔  رسّی \عددی{A} کے لحاظ سے سل کا  مرکز کمیت \عددی{B} کے زیادہ قریب ہے۔ (ا)  سل کے مرکز کمیت پر قوت مروڑ ناپتے ہوئے کیا  \عددی{A} کی  قوت مروڑ کی قدر \عددی{B} کی قوت مروڑ کی قدر سے زیادہ ہو گی، کم ہو گی، یا  اس کے برابر  ہو گی؟ (ب) سل پر کونسی رسّی زیادہ قوت ڈالتی ہے؟  (ج)  اگر اس وقت رسّیوں  کی لمبائیاں برابر ہے، کس رسّی  کی ابتدائی لمبائی (سل لٹکانے سے قبل)  کم تھی؟
\انتہا{سوال}
%-----------------------
%Q11 p344
\ابتدا{سوال}
ذیل جدول میں  تین سلاخوں کی ابتدائی لمبائیاں اور  سلاخوں   کے سر پر قوت لاگو کرنے کے بعد لمبائیوں میں تبدیلی پیش ہے۔ بگاڑ کے لحاظ سے، اعظم قیمت اول رکھ کر، سلاخوں کی درجہ بندی کریں۔
\begin{center}
\begin{tabular}{ccc}
\toprule
&ابتدائی لمبائی & لمبائی میں تبدیلی\\
\midrule
سلاخ \عددی{A}&  \(2L_0\) &  \(\Delta L_0\)\\
سلاخ \عددی{B}&\(4L_0\) &  \(2\Delta L_0\)\\
سلاخ \عددی{C}&\(10L_0\) &  \(4\Delta L_0\)\\
\bottomrule
\end{tabular}
\end{center}
\انتہا{سوال}
%-----------------------
%Q12 p344
\ابتدا{سوال}
سات جرثقیل شکل \حوالہء{12.24} میں دکھائے گئے ہیں۔ ایک لمبی رسّی تمام جرثقیل  کے گرد لپٹی ہے، جبکہ چھوٹی رسّیاں  جرثقیل کو چھت سے یا وزن کو جرثقیل سے لٹکاتی ہیں۔ ایک کے علاوہ تمام وزن (نیوٹن میں)  دیے گئے ہیں۔ (ا)  یہ  وزن کتنا ہے؟ (\ترچھا{اشارہ:} جب رسّی جرثقیل کے گرد نصف دائرہ لپٹی ہو، جرثقیل پر قوت رسّی کے  تناو  کی دگنی ہو گی۔) (ب)  چھوٹی رسّی جس کی نشاندہی \عددی{T} سے کی گئی ہے، میں تان کتنی ہے؟
\انتہا{سوال}
%----------------------

%Problems p345
\حصہء{سوالات}
\setcounter{questioncounter}{0}
%Module 12.1 equilibrium p345
\جزوحصہء{توازن}
%Q1 p345
\ابتدا{سوال}
روز مرہ   اجسام  کی پوری جسامت    پر \عددی{g}  اتنا معمولی تبدیل ہوتا ہے کہ کسی بھی جسم کا مرکز کمیت اور  مرکز ثقل ایک تصور کیا  جا سکتا ہے۔ آئیں ایک فرضی مثال لیتے ہیں جہاں \عددی{g} زیادہ تبدیل ہوتا ہو۔ شکل \حوالہء{12.25} میں چھ ذرے دو قطار میں ، بلا کمیت سلاخوں کے ڈھانچے  پر   جکڑا  گیا ہے۔ ہر ذرے کی کمیت \عددی{m} ہے ۔ کنارے پر قریبی ذروں کے بیچ فاصلہ \عددی{\SI{2.00}{\meter}} ہے۔  ذیل جدول ہر ذرے کے مقام پر \عددی{g} کی قیمت  (\عددی{\si{\meter\per\second\squared}} میں) دیتا ہے۔ دیا گیا محددی نظام استعمال کرکے چھ ذروی نظام کے مرکز کمیت  کا (ا)  \عددی{x}  محدد \عددی{x_{\text{\RL{مرکزکمیت}}}} اور (ب)  \عددی{y} محدد \عددی{y_{\text{\RL{مرکزکمیت}}}} تلاش کریں۔ (ج)  چھ ذروی نظام  کے مرکز ثقل کا (ج) \عددی{x} محدد \عددی{x_{\text{\RL{مرکزثقل}}}}  اور (د) \عددی{y} محدد \عددی{y_{\text{\RL{مرکزثقل}}}} تلاش کریں۔
\begin{center}
\begin{tabular}{CC|CC}
\toprule
\text{\RL{ذرہ}} & g & \text{\RL{ذرہ}} & g\\
\midrule
1&8.00&4&7.40\\
2&7.80&5&7.60\\
3&7.60&6&7.80\\
\bottomrule
\end{tabular}
\end{center}
\انتہا{سوال}
%-----------------------

%Module 12.2 some examples of static equilibrium p345
\جزوحصہء{سکونی توازن کی چند مثالیں}
%Q2 p345
\ابتدا{سوال}
ایک گاڑی، جس کی کمیت \عددی{\SI{1360}{\kilo\gram}} ہے، کے اگلے اور پچھلے  دھروں کے بیچ \عددی{\SI{3.05}{\meter}} فاصلہ ہے۔ مرکز کمیت اگلے دھرے سے \عددی{\SI{1.78}{\meter}} پیچھے ہے۔ استوا زمین پر کھڑی گاڑی کے  (ا)  اگلے پہیے پر   اور (ب) پچھلے پہیے پر زمین سے کتنی قوت  پڑتی ہے؟
\انتہا{سوال}
%------------------------
\ابتدا{سوال}
یکساں کرہ، جس کی کمیت \عددی{m=\SI{0.85}{\kilo\gram}} اور رداس \عددی{r=\SI{4.2}{\centi\meter}} ہے، بلا کمیت رسّی سے دیوار کے ساتھ بندھا ہے (شکل \حوالہء{12.26})۔  کرہ کے مرکز کمیت سے \عددی{L=\SI{8.0}{\centi\meter}} بلندی  پر رسّی دیوار سے بندھی ہے۔ (ا) رسّی  میں تناو اور (ب) دیوار سے کرہ پر قوت  تلاش کریں۔
\انتہا{سوال}
%------------------------
%Q4 p345
\ابتدا{سوال}
کمان  کو وسطی نقطہ سے کھینچا جاتا ہے حتٰی  کہ  تیر انداز  کی قوت اور  ڈور میں تان برابر ہوں۔ ڈور کے دو حصوں میں  زاویہ کیا ہو گا؟
\انتہا{سوال}
%---------------------
\ابتدا{سوال}
بلا کمیت رسّی    دو  نقطوں کے بیچ افقی باندھی جاتی ہے۔ ان نقطوں کے بیچ فاصلہ \عددی{\SI{3.44}{\meter}} ہے۔ رسّی کے وسط سے \عددی{\SI{3160}{\newton}} وزن لٹکانے سے ، رسّی   \عددی{\SI{35.0}{\centi\meter}} جھکتی ہے۔ رسّی میں تان کیا ہے؟
\انتہا{سوال}
%---------------------
%Q6 p345
\ابتدا{سوال}
ایک پاڑ ، جس کی کمیت \عددی{\SI{60}{\kilo\gram}}  اور لمبائی \عددی{\SI{5.0}{\meter}} ہے، کو  افقی حالت میں  پاڑ کے سر پر بندھی انتصابی رسّیاں  رکھتی ہیں۔ پار کے ایک سر سے \عددی{\SI{1.5}{\meter}} فاصلے پر \عددی{\SI{80}{\kilo\gram}} کا شخص کھڑا ہے۔ (ا) قریبی رسّی اور (ب) دور رسّی میں تان معلوم کریں۔
\انتہا{سوال}
%------------------------
%Q7 p345
\ابتدا{سوال}
ایک شخص، جس کی کمیت \عددی{\SI{75}{\kilo\gram}} ہے، \عددی{\SI{10}{\kilo\gram}} کمیت کی \عددی{\SI{5.0}{\meter}} لمبی سیڑھی استعمال کرتا ہے۔سیڑھی کا ایک سر دیوار سے \عددی{\SI{2.5}{\meter}} فاصلے پر اور دوسرا   بلا رگڑ دیوار کے ساتھ جوڑ کر، شخص سیڑھی چڑھتا ہے۔ (ا) سیڑھی کی  دیوار پر قوت اور (ب)   زمین کی سیڑھی پر قوت کی قدر کیا ہے، اور (ج) افق کے ساتھ  سیڑھی پر زمین کی  قوت کا زاویہ کیا ہے؟
\انتہا{سوال}
%---------------------
\ابتدا{سوال}
متوازن ہنڈولے پر  بیٹھے گنتی دار   بچوں کا وزن (نیوٹن میں)  شکل \حوالہء{12.27} میں پیش ہے۔ چول \عددی{f}  پر  (ا) صفحہ سے باہر رخ اور (ب) صفحہ کے اندر رخ محور پر کونسا بچہ سب سے بڑی قوت مروڑ پیدا کرتا ہے؟
\انتہا{سوال}
%-----------------------
\ابتدا{سوال}
افقی \اصطلاح{میٹر  مسطر }\فرہنگ{میٹر!مسطر} تلوار کی دھار پر \حاشیہب{meter stick}\فرہنگ{meter!stick}  \عددی{\SI{50.0}{\centi\meter}} کی نشان پر  توازن میں ہے۔ \عددی{\SI{12.0}{\centi\meter}} نشان پر  دو سکے، جن کی انفرادی کمیت  \عددی{\SI{5.00}{\gram}}    ہے، رکھنے سے   مسطر \عددی{\SI{45.5}{\centi\meter}} نشان پر توازن اختیار کرتا ہے۔ میٹر مسطر کی کمیت تلاش کریں۔
\انتہا{سوال}
%-------------------------
%Q10 p345
\ابتدا{سوال}
شکل \حوالہء{12.28}  کا نظام توازن میں ہے، اور درمیانی ڈور  افقی ہے۔ سل \عددی{A}  کا وزن \عددی{\SI{40}{\newton}}، سل \عددی{B}  کا وزن \عددی{\SI{50}{\newton}}، اور زاویہ \عددی{\phi=\SI{35}{\degree}} ہے۔ (ا)  تناو \عددی{T_1}، (ب) تناو \عددی{T_2}، (ج) تناو \عددی{T_3}، اور (د) زاویہ \عددی{\theta} تلاش کریں۔
\انتہا{سوال}
%-------------------------
\ابتدا{سوال}
ایک غوطہ باز، جس کا وزن \عددی{\SI{580}{\newton}}  ہے، \عددی{L=\SI{4.5}{\meter}} لمبے  بلا کمیت تختہ غوطہ پر کھڑا ہے۔ تختہ دو  تیک   سے جوڑا گیا ہے، جن کے بیچ فاصلہ \عددی{d=\SI{1.5}{\meter}} ہے۔ تختے   پر   بائیں تیک  کی قوت (ا) کی قدر اور (ب) رخ  (اوپر  یا نیچے)کیا ہیں؟ تختے پر دائیں تیک کی قوت (ج) کی قدر اور (د) رخ کیا ہیں؟ (ہ)  کونسا تیک (بایاں یا دائیں)  کھنچا جا رہا ہے، اور (و) کونسا تیک دبا جا رہا ہے؟
\انتہا{سوال}
%--------------------
%Q12 p346
\ابتدا{سوال}
 کیچڑ  میں پھنسی گاڑی  کو نکالنے کے لئے ایک شخص گاڑی کا اگلا حصہ  رسّی کے ذریعہ سامنے \عددی{\SI{18}{\meter}} دور  کھمبے کے ساتھ باندھ کر رسّی  کو وسطی نقطہ
  سے \عددی{\SI{550}{\newton}} قوت کے ساتھ   پہلو کی  طرف کھینچتا ہے (شکل \حوالہء{12.30})۔ رسّی کا وسطی نقطہ اپنی جگہ سے  \عددی{\SI{0.30}{\meter}} دور ہٹتا ہے، تاہم گاڑی ٹس سے مس نہیں ہوتی۔ رسّی سے گاڑی پر قوت کی قدر کیا ہے؟ (رسّی کھنچ کر لمبی ہوتی ہے۔)
\انتہا{سوال}
%-------------------------
\ابتدا{سوال}
شکل \حوالہء{12.31  }  میں  ٹانگ کے نچلے حصہ اور پاوں کی اندرونی ساخت  پیش ہے۔پنجوں پر  کھڑا ہونے کے لئے   ایڑی  زمین سے اٹھائی جاتی ہے اور  زمین کو پاوں   ،عملاً ، نقطہ \عددی{P} پر مس کرتا ہے۔ فرض کریں \عددی{a=\SI{5.0}{\centi\meter}}، \عددی{b=\SI{15}{\centi\meter}}، اور   شخص کا وزن \عددی{W=\SI{900}{\newton}} ہے۔    نقطہ \عددی{A} پر  پنڈلی  کے عضلہ کی قوت  (ا) کی قدر اور (ب) رخ  (اوپر یا نیچے) کیا ہیں؟ اس نقطہ پر پنڈلی کی  ہڈی  کی قوت (ج) کی قدر اور (د) رخ (اوپر یا نیچے) کیا ہیں؟
\انتہا{سوال}
%----------------------
%Q14 p346
\ابتدا{سوال}
افقی یکساں  پاڑ، جس کی لمبائی \عددی{\SI{2.0}{\meter}} اور کمیت \عددی{\SI{50.0}{\kilo\gram}} ہے، عمارت سے دو رسّوں کے ذریعے لٹکا ہے (شکل \حوالہء{12.32})۔ پاڑ پر کئی رنگ و روغن کے ڈبے رکھا گئے ہیں۔ ان ڈبوں کی مجموعی کمیت \عددی{\SI{75.0}{\kilo\gram}} ہے۔ دائیں رسّی میں تناو \عددی{\SI{722}{\newton}} ہے۔ رنگ کے ڈبوں کا مرکز کمیت اس رسّی سے کتنے  افقی فاصلے پر ہے؟
\انتہا{سوال}
%---------------------
\ابتدا{سوال}
قوت \عددی{\vec{F}_1} ، \عددی{\vec{F}_2}، اور \عددی{\vec{F}_3}  ایک  ڈھانچے  پر ، جس کا فضائی جائزہ شکل \حوالہء{12.33} میں پیش ہے، عمل پیرا ہیں۔  ڈھانچہ متوازن بنانے کی غرض سے ہم چوتھی قوت نقطہ \عددی{P} پر لاگو کرتے ہیں۔ چوتھی قوت  کا افقی جزو \عددی{\vec{F}_h} اور انتصابی جزو \عددی{\vec{F}_v} ہے۔ہمیں معلوم ہے کہ \عددی{a=\SI{2.0}{\meter}}،  \عددی{b=\SI{3.0}{\meter}}، \عددی{c=\SI{1.0}{\meter}}، \عددی{F_1=\SI{20}{\newton}}، 
\عددی{F_2=\SI{10}{\newton}}، اور \عددی{F_3=\SI{5.0}{\newton}} ہے۔ (ا) \عددی{F_h} ، (ب) \عددی{F_v} ، اور (ج) \عددی{d} تلاش کریں۔
\انتہا{سوال}
%------------------------------
%Q16 p346
\ابتدا{سوال}
مکعب یکساں پیٹی، جس کا ضلع \عددی{\SI{0.750}{\meter}}  اور وزن \عددی{\SI{500}{\newton}} ہے،   فرش پر   انتہائی چھوٹی   اکڑ رکاوٹ کے ساتھ مس  پڑی ہے۔فرش سے کتنے کم سے کم بلندی پر \عددی{\SI{350}{\newton}}  افقی قوت لاگو کر کے پیٹی پلٹی جا سکتی ہے؟
\انتہا{سوال}
%---------------------
\ابتدا{سوال}
ایک شہتیر، جس کا وزن \عددی{\SI{500}{\newton}} اور لمبائی \عددی{\SI{3.0}{\meter}} ہے، افقی لٹکا ہے  (شکل \حوالہء{12.34})۔  اس جا بایاں سر چول دار ہے اور دایاں سر رسّی سے باندھا گیا ہے۔ رسّی کا دوسرا سر چول سے \عددی{D} بلندی پر دیوار میں لگے قابلے  سے  باندھا گیا ہے۔رسّی ،بغیر ٹوٹے   ، زیادہ سے زیادہ \عددی{\SI{1200}{\newton}} تناو  برداشت کر سکتی  ہے۔ (ا)  \عددی{D} کو کونسی قیمت رسّی میں اتنا تناو پیدا کرے گی؟ (ب)  رسّی کو  ٹوٹنے سے   بچانے کے لئے  کیا \عددی{D}    اس قیمت سے بڑھانی ہو گی یا گھٹانی ہو گی؟
\انتہا{سوال}
%---------------------
\ابتدا{سوال} 
افقی پاڑ  \عددی{2}، جس کی یکساں کمیت \عددی{m_2=\SI{30.0}{\kilo\gram}} اور لمبائی \عددی{L_2=\SI{2.00}{\meter}}  ہے، افقی پاڑ \عددی{1} سے لٹک رہا ہے، جس کی  یکساں کمیت \عددی{m_1=\SI{50.0}{\kilo\gram}}  ہے (شکل \حوالہء{12.35})۔ پاڑ \عددی{2} پر  کیلوں کی ڈبی رکھی گئی ہے ،   جس کی کمیت \عددی{\SI{20.0}{\kilo\gram}} اور جس کا مرکز کمیت پاڑ کے بائیں سر سے \عددی{d=\SI{0.500}{\meter}} فاصلے پر ہے۔بالا دائیں  ہاتھ   رسّی میں تناو \عددی{T} کیا ہو گا؟
\انتہا{سوال}
%-----------------------
%Q19 p346
\ابتدا{سوال}
سروتا کی مدد سے  اخروٹ  توڑنے کے لئے  اخروٹ کے چھلکا پر دونوں جانب سے \عددی{\SI{40}{\newton}} قوت درکار ہے۔ شکل \حوالہء{12.36}  میں پیش سروتا  میں  \عددی{L=\SI{12}{\centi\meter}} اور \عددی{d=\SI{2.6}{\centi\meter}} ہے۔ سروتا کے دستے پر  درکار عمود دار قوت \عددی{F_{\perp}} تلاش کریں۔
\انتہا{سوال}
%------------------------
\ابتدا{سوال}
ایک کھلاڑی \عددی{M=\SI{7.2}{\kilo\gram}} گیند ہاتھ میں پکڑے  ہوئے ہے (شکل \حوالہء{12.37})۔ بازو کا اوپری حصہ انتصابی اور نچلا حصہ  ، جس کی
 کمیت \عددی{\SI{1.8}{\kilo\gram}} ہے، افقی ہے۔ (ا)  بازو کے نچلے حصے پر اوپری  بازو کے عضلہ  کی قوت  کی قدر کیا ہو گی؟ (ب)  کوہنی  پر ہڈیوں کے بیچ قوت کی قدر کیا ہو گی؟
\انتہا{سوال}
%-----------------------
%Q21 p346
\ابتدا{سوال}ُ
شکل \حوالہء{12.38} میں  نظام توازن میں ہے۔کانکریٹ سل، جس کی کمیت \عددی{\SI{225}{\kilo\gram}} ہے، یکساں  بازو  سے لٹک رہی ہے، جس کی کمیت \عددی{\SI{45.0}{\kilo\gram}} ہے۔ ایک رسّی  زمین سے، بازو کے اوپر سے گزر کر، سل کو پہنچ کر، سل کو پکڑ کر رکھتی ہے۔ زاویہ \عددی{\phi=\SI{30.0}{\degree}}  
اور \عددی{\theta=\SI{45.0}{\degree}} کے لئے  (ا)  رسّی میں تناو  \عددی{T}، اور  چول سے بازو پر قوت کا  (ب) افقی اور (ج) انتصابی جزو کیا ہو گا؟
\انتہا{سوال}
%-------------------------
%Q22 p347
