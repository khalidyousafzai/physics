%equilibrium and elasticity p327
\باب{توازن اور لچک}
%12.1 equilibrium p327
\حصہ{توازن}
\موٹا{مقاصد}\\
اس حصہ کو پڑھ کر آپ ذیل کے قابل ہوں گے۔
\begin{enumerate}[1.]
\item
توازن اور سکونی توازن میں فرق کر پائیں گے۔
\item
سکونی توازن کے چار شرائط جان پائیں گے۔
\item
مرکز  ثقل    اور  اس کا مرکز کمیت  سے تعلق   سمجھا پائیں گے۔
\item
ذروں کی  دی گئی تقسیم کے لئے  مرکز ثقل اور مرکز کمیت کے محدد  کا حساب کر پائیں گے۔
\end{enumerate}

\موٹا{کلیدی تصور}\\
\begin{itemize}
\item
استوار جسم جب ساکن ہو، وہ سکونی توازن میں ہو گا۔ ایسے جسم کے لئے، جسم پر بیرونی قوتوں کا مجموعہ صفر ہو گا۔
\begin{align*}
\vec{F}_{\text{\RL{صافی}}}=0 \quad\quad\text{\RL{(قوتوں کا توازن)}}
\end{align*}
اگر تمام قوت \عددی{xy} مستوی میں ہوں، یہ مساوات   ذیل دو جزوی مساوات کی معادل ہو گی.
\begin{align*}
F_{\text{\RL{صافی}},y}&=0 \quad \text{\RL{اور}}\quad F_{\text{\RL{صافی}},x}=0  \quad\quad\text{\RL{(قوتوں کا توازن)}}
\end{align*}
\item
سکونی توازن سے مراد یہ بھی ہے کہ کسی بھی نقطے  کے لحاظ سے جسم پر   بیرونی  قوت مروڑ  کا مجموعہ صفر ہو گا:
\begin{align*}
\vec{\tau}_{\text{\RL{صافی}}}=0\quad\quad\text{\RL{(قوت مروڑ کا توازن)}}
\end{align*}
اور اگر تمام قوت \عددی{xy} مستوی میں ہوں تب تمام قوت مروڑ سمتیات محور \عددی{z} کو متوازی ہوں گے، اور  قوت مروڑ کے توازن کی مساوات ذیل  یک جزوی  مساوات کی معادل ہو گی۔
\begin{align*}
\tau_{\text{\RL{صافی}},z}=0\quad\quad\text{\RL{(قوت مروڑ کا توازن)}}
\end{align*}
\item
تجاذبی قوت جسم کے ہر ذرے پر انفرادی عمل  کرتی ہے۔تمام انفرادی  اعمال کا صافی  اثر  جاننے کے لئے مرکز کمیت پر معادل تجاذبی قوت \عددی{\vec{F}_g} فرض  کرنی  ہو گی۔اگر جسم کے تمام ٹکڑوں پر ثقلی اسراع \عددی{\vec{g}} ایک ہو، ثقلی مرکز جسم کے مرکز کمیت پر ہو گا۔
\end{itemize}

\جزوحصہ{طبیعیات کیا ہے؟}
انسانی بنائی چیزیں، لاگو قوتوں سے قطع نظر، مستحکم   تصور کی جاتی ہیں۔  تجاذبی قوت اور ہوائی قوتوں کے باوجود ہم توقع کرتے ہیں کہ عمارت کھڑی رہے گی، اور پُل سمندر میں  گرے گا نہیں۔

طبیعیات کے مرکز توجہ  وہ حقیقت ہے جو عمل پیرا قوتوں کے باوجود  جسم کو  مستحکم رکھتا ہے۔ اس باب میں استحکام  کے دو نقطہ نظر پر غور کیا جائے گا: استوار جسم پر عمل پیرا قوت اور قوت مروڑ کا \ترچھا{ توازن } اور نا  استوار اجسام کی  \ترچھا{لچک} ، جس پر اجسام کا  مسخ ہونا منحصر ہے۔ اگر  طبیعیات درست کی جائے،اس پر   انجینئری اور طبیعیات کے جریدوں  میں لاتعداد  مضامین  لکھے جائیں گے؛ اگر غلط کی جائے، اخبار   کا سرنامہ بنے گا اور قانونی کارروائی ہو گی۔

\جزوحصہء{توازن}
ذیل اجسام پر غور کریں: (1) میز پر  پڑی ساکن کتاب، (2) بلا رگڑ سطح پر مستقل سمتی رفتار سے حرکت پذیر قرص، (3)  چھت کے پنکھے کے چکر کھاتے پَر، اور (4)  سیدھی راہ پر چلتے سائیکل کا پہیا۔ ان چار اجسام کے لئے
\begin{enumerate}[1.]
\item
مرکز کمیت کا خطی معیار حرکت \عددی{\vec{P}} ایک مستقل ہے۔
\item
مرکز کمیت یا کسی دوسرے نقطہ کے لحاظ سے ان کا زاوی معیار حرکت \عددی{\vec{L}} بھی ایک مستقل ہے۔
\end{enumerate}

ہم کہتے ہیں یہ جسم\اصطلاح{ توازن }\فرہنگ{توازن}\حاشیہب{equilibrium}\فرہنگ{equilibrium} میں ہیں۔ یوں توازن کے  دو  شرائط ذیل ہیں۔
%eq 12.1
\begin{align}\label{مساوات_توازن_تعریف_الف}
\vec{L}=\text{\RL{مستقل}}\quad \text{\RL{اور}}\quad \vec{P}=\text{\RL{مستقل}}
\end{align}

اس باب میں ہم صرف  ان صورتوں پر غور کرتے ہیں جہاں  مساوات  \حوالہ{مساوات_توازن_تعریف_الف} میں مستقل کی قیمت صفر ہو؛ یعنی   ہم ان اجسام میں دلچسپی رکھتے ہیں جو حوالہ چوکھٹ کے لحاظ سے  ساکن ہوں؛  خطی سکون اور گھمیری سکون میں ہم دلچسپی رکھتے ہیں۔ ایسے اجسام\اصطلاح{ سکونی توازن }\فرہنگ{توازن!سکونی}\حاشیہب{static equilibrium}\فرہنگ{equilibrium!static} میں ہوں گے۔ باب کے آغاز میں چار  اجسام میں صرف میز پر پڑی کتاب سکونی توازن میں ہے۔

شکل \حوالہء{12.1} میں   دکھائی گئی چٹان  ، فی الحال ، سکونی توازن میں ہے۔مساجد،  پُل، گھر، وغیرہ بھی سکونی توازن میں ہیں؛ یہ وقت  گزرنے کے باوجود  ساتھ  ساکن رہتے ہیں۔

جیسا ہم حصہ \حوالہء{8.3} میں ذکر کر چکے ، اگر  سکونی توازن سے قوت کے بل بوتے  پر   نکالے جانے  کے بعد جسم واپس  سکونی توازن  کو لوٹے،ہم کہتے ہیں یہ  جسم\ترچھا{ مستحکم } سکونی توازن میں ہے۔ نصف کرہ کے تل میں رکھا گیا  کنچا اس کی ایک مثال ہے۔ اس کے برعکس، اگر  چھوٹی قوت جسم کو ہلا کر  توازن ختم کر پائے، جسم \ترچھا{ غیر مستحکم } سکونی توازن میں ہو گا۔

\موٹا{زنجیری اثر۔}\quad
فرض کریں ہم  ایک اینٹ یوں کھڑی کریں کہ اس کا مرکز کمیت عین ایک  کنارے کے اوپر ہو (شکل \حوالہء{12.2a})۔  تجاذبی قوت \عددی{\vec{F}_g} کا خط عمل  اسی کنارے  سے گزرتا ہے لہٰذا  اس کنارے  پر \عددی{\vec{F}_g} کی قوت مروڑ  صفر ہو گی۔ اینٹ توازن میں ہے۔معمولی   اضطراب  اس توازن کو برباد کر دیگا۔ جیسے ہی \عددی{\vec{F}_g} کا خط عمل  کنارے  سے  ایک  طرف ہو (شکل \حوالہء{12.2b})، \عددی{\vec{F}_g} کی پیدا کردہ قوت مروڑ اینٹ کو اس طرف گھمائے گی۔ یوں شکل \حوالہء{12.2a} میں اینٹ غیر مستحکم توازن میں ہے۔

شکل \حوالہء{12.2c} میں اینٹ اتنی غیر مستحکم نہیں۔ اینٹ   گرانے کے لئے ضروری ہے کہ  قوت   اینٹ  اتنی گھمائے کہ اینٹ کا مرکز کمیت کنارے کو پار کر جائے۔ معمولی قوت اس اینٹ کو نہیں گرا سکتی، تاہم  انگلی سے جھٹکا دے کر اسے گرایا جا سکتا ہے۔(اینٹوں کو قطار میں کھڑا کر کے ، پہلی اینٹ کو جھٹکا دے کر گرانے سے تمام اینٹیں گرائی جا سکتی ہیں۔)

\موٹا{سل۔}\quad
شکل \حوالہء{12.2d} میں دکھایا گیا سل مزید زیادہ مستحکم ہے۔  مرکز کمیت کو سل کے کنارے کی دوسری طرف لی جانے کے لئے مرکز کمیت کو  کافی زیادہ  دور لے جانا ہو گا۔ انگلی کا جھٹکا سل کا   پاسا  نہیں پلٹ سکتا۔ (اسی لئے سل قطار میں رکھ کر زنجیری اثر  پیدا  نہیں کیا جا سکتا۔) شکل \حوالہء{12.3} میں  شہتیر  پر بیٹھا مزدور سل کی مانند جبکہ اس پر کھڑا مزدور اینٹ کی مانند ہو گا (جس کو ہوا کا جھٹکا نیچے لا سکتا ہے)۔

سکونی توازن اطلاقی انجینئری   کے لئے  بہت  ضروری ہے۔ تخلیق کار  تمام بیرونی قوت اور قوت مروڑ کی نشاندہی کر کے، بہتر تراکیب اور  مواد  استعمال کر کے، یقینی بناتا ہے کہ ان کی موجودگی کے باوجود عمارت یا مشین مستحکم رہے۔ یوں پُل  کا نقشہ تیار کرتے وقت  تخلیق کار تفصیلی تجزیہ کر کے   یقینی بناتا ہے  کہ پُل پر  آمد و رفت اور ہوائی قوتوں   کو پُل   سہ سکے۔

\جزوحصہء{توازن کے شرائط}
جسم کی   مستقیم  حرکت ، خطی معیار حرکت کے روپ میں   نیوٹن کے قانون دوم کو ، جو (ذیل)  مساوات \حوالہء{9.27} دیتی ہے،    مطمئن کرتی ہے۔
%eq 12.2
\begin{align}
\vec{F}_{\text{\RL{صافی}}}=\frac{\dif \vec{P}}{\dif t}
\end{align}
اگر جسم مستقیم  توازن  میں ہو؛ یعنی اگر  \عددی{\vec{P}} ایک مستقل ہو،  تب \عددی{\dif\vec{P}\!/\!\dif t=0} ہو گا لہٰذا لازماً درج ذیل ہو گا۔
%eq 12.3
\begin{align}\label{مساوات_توازن_شرط_ایک}
\vec{F}_{\text{\RL{صافی}}}=0\quad\quad\text{\RL{(متوازن قوت}}
\end{align}

جسم کی    گھمیری   حرکت ، زاوی  معیار حرکت کے روپ میں   نیوٹن کے قانون دوم کو ، جو (ذیل)  مساوات \حوالہ{مساوات_لڑھکاو_ذروں_نظام_ت} دیتی ہے،    مطمئن کرتی ہے۔
%eq 12.4
\begin{align}
\vec{\tau}_{\text{\RL{صافی}}}=\frac{\dif \vec{L}}{\dif t}
\end{align}
اگر جسم  گھمیری   توازن میں ہو؛ یعنی اگر  \عددی{\vec{L}} ایک مستقل ہو،  تب \عددی{\dif\vec{L}\!/\!\dif t=0} ہو گا لہٰذا لازماً درج ذیل ہو گا۔
%eq 12.5
\begin{align}\label{مساوات_توازن_شرط_دو}
\vec{\tau}_{\text{\RL{صافی}}}=0\quad\quad\text{\RL{(متوازن قوت مروڑ)}}
\end{align}
یوں  جسم کا توازن میں ہونے کے لئے ذیل دو شرائط   ہیں۔

\ابتدا{قاعدہء}
\begin{enumerate}[1.]
\item
جسم پر تمام بیرونی قوتوں کا سمتی مجموعہ صفر  ہونا لازم ہے۔
\item
ہر ممکنہ  نقطہ  کے لحاظ سے، جسم پر بیرونی قوت مروڑ  کا سمتی مجموعہ صفر ہونا   لازم ہے۔
\end{enumerate}
\انتہا{قاعدہء}
%------------------------------------


ہاں یہ شرائط\ترچھا{ سکونی } توازن کے لئے بھی ہیں۔ یہ شرائط عمومی صورت کے لئے بھی درست ہیں، جہاں \عددی{\vec{P}} اور \عددی{\vec{L}} مستقل ضرور لیکن غیر صفر ہوں۔

مساوات \حوالہ{مساوات_توازن_شرط_ایک} اور مساوات \حوالہ{مساوات_توازن_شرط_دو}، بطور سمتی مساوات، درحقیقت (ذیل)  تین تین جزوی مساوات  کی معادل ہیں۔
%eq 12.6
\begin{gather}
\begin{aligned}\label{مساوات_توازن_چہ}
&\text{\RL{متوازن قوت}}  & \text{\RL{متوازن قوت مروڑ}}\\
&F_{\text{\RL{صافی}},x}=0   &\tau_{\text{\RL{صافی}},x}=0\\
&F_{\text{\RL{صافی}},y}=0  &\tau_{\text{\RL{صافی}},y}=0\\
&F_{\text{\RL{صافی}},z}=0  &\tau_{\text{\RL{صافی}},z}=0
\end{aligned}
\end{gather}

\موٹا{اصل مساوات۔}\quad
ہم صرف  ان صورتوں پر غور کرتے ہیں جس میں جسم پر لاگو قوت \عددی{xy} مستوی میں پائے جاتے ہیں۔ یوں مسئلہ کم  پیچیدہ ہو گا۔ اس طرح جسم پر عمل پیرا قوت صرف محور \عددی{z} کی  متوازی محور   کے گرد  جسم  گھما سکتے ہیں۔ اس مفروضے کے ساتھ  مساوات \حوالہ{مساوات_توازن_چہ} میں سے قوت کی ایک مساوات اور قوت مروڑ کی دو مساوات سے چھٹکارا  حاصل ہو گا۔ یوں ذیل باقی رہتی ہیں۔
%eq 12.7, 12.8, 12.9
\begin{align}
F_{\text{\RL{صافی}},x}&=0 \label{مساوات_توازن_شرائط_الف} \\
F_{\text{\RL{صافی}},y}&=0  \label{مساوات_توازن_شرائط_ب} \\
\tau_{\text{\RL{صافی}},z}&=0 \label{مساوات_توازن_شرائط_پ}  
\end{align}
یہاں، \عددی{\tau_{\text{\RL{صافی}},z}}  وہ صافی قوت مروڑ ہے جو محور \عددی{z} یا اس کے متوازی کسی محور پر  بیرونی  قوت پیدا کرتی ہیں۔

جمی ہوئی   برف  پر مستقل سمتی رفتار سے حرکت کرتا قرص مساوات \حوالہ{مساوات_توازن_شرائط_الف}، مساوات \حوالہ{مساوات_توازن_شرائط_ب}، اور مساوات \حوالہ{مساوات_توازن_شرائط_پ}   مطمئن کرتا ہے، لہٰذا یہ توازن میں ہو گا،\ترچھا{ تاہم یہ سکونی توازن میں  ہرگز نہیں}۔ سکونی توازن کے  لئے قرص کا خطی معیار حرکت \عددی{\vec{P}} ایک مستقل ہونے کے ساتھ ساتھ  صفر ہونا  لازم  ہے؛ قرص کا جمی ہوئی برف پر ساکن ہونا لازم ہے۔ یوں، سکونی توازن کے لئے درج ذیل  شرط  بھی لازم   ہے۔

\ابتدا{قاعدہء}
جسم کے  خطی معیار حرکت \عددی{\vec{P}}  کا صفر ہونا لازم ہے۔
\انتہا{قاعدہء}

%----------------------
%Checkpoint 1 p330
\ابتدا{آزمائش}
یکساں سلاخ ، جس پر سلاخ کو عمود دار دو یا دو سے زیادہ قوت عمل کرتی ہیں،  کے چھ فضائی نظارے شکل \حوالہء{؟؟} میں پیش ہیں۔ قوتوں کی قدریں      (غیر صفر رکھ کر اور)   تبدیل کر کے  کون کونسی سلاخ سکونی توازن میں لائی جا سکتی ہیں؟
\انتہا{آزمائش}
%------------------

%the center of gravity p330
\جزوحصہء{مرکز ثقل}
جسم پر تجاذبی قوت  ، جسم کے انفرادی ٹکڑوں (جوہر) پر تجاذبی قوتوں کا سمتی مجموعہ ہو گا۔ انفرادی ٹکڑوں کی بات کرتے ہوئے ہم ذیل کہتے ہیں۔

\ابتدا{قاعدہء}
جسم پر تجاذبی قوت \عددی{\vec{F}_g} \قول{    عملاً}   جسم کے \اصطلاح{ مرکز ثقل }\فرہنگ{مرکز ثقل}\حاشیہب{center of gravity}\فرہنگ{center of gravity} پر    عمل کرتی ہے۔
\انتہا{قاعدہء}
%---------------------------

یہاں لفظ \قول{عملاً} کا مطلب یہ ہے کہ  اگر کسی طرح انفرادی ٹکڑوں  پر تجاذبی قوت ختم کر دی جائے اور تجاذبی قوت  \عددی{\vec{F}_g} جسم کے مرکز ثقل پر پیدا کر دی جائے، جسم پر صافی قوت اور  (کسی بھی محور کے لحاظ سے)  جسم پر صافی قوت مروڑ تبدیل نہیں ہوں گی۔

اب تک،  ہم فرض کرتے رہے ہیں کہ تجاذبی قوت  \عددی{\vec{F}_g} جسم کے مرکز کمیت پر عمل کرتی ہے، جو   اس  کے  مترادف  ہے  کہ ہم کہیں جسم کا مرکز ثقل  جسم کے مرکز کمیت  پر پایا جاتا ہے۔ یاد  کریں، کمیت \عددی{M}   جسم پر تجاذبی  قوت \عددی{\vec{F}_g=M\vec{g}}     عمل کرتی ہے، جہاں \عددی{\vec{g}}   جسم کا وہ اسراع ہے جو جسم پر \عددی{\vec{F}_g} لاگو کرنے سے پیدا ہو گا۔ ہم  جلد  ذیل  ثابت کرتے ہیں۔

%----------------------
%p331
\ابتدا{قاعدہء}
اگر جسم کے تمام ٹکڑوں کے لئے \عددی{\vec{g}}  ایک ہو، جسم کا مرکز ثقل اور جسم کا مرکز کمیت ایک نقطے پر ہوں گے۔
\انتہا{قاعدہء}
