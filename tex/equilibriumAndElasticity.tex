%equilibrium and elasticity p327
\باب{توازن اور لچک}
%12.1 equilibrium p327
\حصہ{توازن}
\موٹا{مقاصد}\\
اس حصہ کو پڑھ کر آپ ذیل کے قابل ہوں گے۔
\begin{enumerate}[1.]
\item
توازن اور سکونی توازن میں فرق کر پائیں گے۔
\item
سکونی توازن کے چار شرائط جان پائیں گے۔
\item
مرکز  ثقل    اور  اس کا مرکز کمیت  سے تعلق   سمجھا پائیں گے۔
\item
ذروں کی  دی گئی تقسیم کے لئے  مرکز ثقل اور مرکز کمیت کے محدد  کا حساب کر پائیں گے۔
\end{enumerate}

\موٹا{کلیدی تصور}\\
\begin{itemize}
\item
استوار جسم جب ساکن ہو، وہ سکونی توازن میں ہو گا۔ ایسے جسم کے لئے، جسم پر بیرونی قوتوں کا مجموعہ صفر ہو گا۔
\begin{align*}
\vec{F}_{\text{\RL{صافی}}}=0 \quad\quad\text{\RL{(قوتوں کا توازن)}}
\end{align*}
اگر تمام قوت \عددی{xy} مستوی میں ہوں، یہ مساوات   ذیل دو جزوی مساوات کی معادل ہو گی.
\begin{align*}
F_{\text{\RL{صافی}},y}&=0 \quad \text{\RL{اور}}\quad F_{\text{\RL{صافی}},x}=0  \quad\quad\text{\RL{(قوتوں کا توازن)}}
\end{align*}
\item
سکونی توازن سے مراد یہ بھی ہے کہ کسی بھی نقطے  کے لحاظ سے جسم پر   بیرونی  قوت مروڑ  کا مجموعہ صفر ہو گا:
\begin{align*}
\vec{\tau}_{\text{\RL{صافی}}}=0\quad\quad\text{\RL{(قوت مروڑ کا توازن)}}
\end{align*}
اور اگر تمام قوت \عددی{xy} مستوی میں ہوں تب تمام قوت مروڑ سمتیات محور \عددی{z} کو متوازی ہوں گے، اور  قوت مروڑ کے توازن کی مساوات ذیل  یک جزوی  مساوات کی معادل ہو گی۔
\begin{align*}
\tau_{\text{\RL{صافی}},z}=0\quad\quad\text{\RL{(قوت مروڑ کا توازن)}}
\end{align*}
\item
تجاذبی قوت جسم کے ہر ذرے پر انفرادی عمل  کرتی ہے۔تمام انفرادی  اعمال کا صافی  اثر  جاننے کے لئے مرکز کمیت پر معادل تجاذبی قوت \عددی{\vec{F}_g} فرض  کرنی  ہو گی۔اگر جسم کے تمام ٹکڑوں پر ثقلی اسراع \عددی{\vec{g}} ایک ہو، ثقلی مرکز جسم کے مرکز کمیت پر ہو گا۔
\end{itemize}

\جزوحصہ{طبیعیات کیا ہے؟}
انسانی بنائی چیزیں، لاگو قوتوں سے قطع نظر، مستحکم   تصور کی جاتی ہیں۔  تجاذبی قوت اور ہوائی قوتوں کے باوجود ہم توقع کرتے ہیں کہ عمارت کھڑی رہے گی، اور پُل سمندر میں  گرے گا نہیں۔

طبیعیات کے مرکز توجہ  وہ حقیقت ہے جو عمل پیرا قوتوں کے باوجود  جسم کو  مستحکم رکھتا ہے۔ اس باب میں استحکام  کے دو نقطہ نظر پر غور کیا جائے گا: استوار جسم پر عمل پیرا قوت اور قوت مروڑ کا \ترچھا{ توازن } اور نا  استوار اجسام کی  \ترچھا{لچک} ، جس پر اجسام کا  مسخ ہونا منحصر ہے۔ اگر  طبیعیات درست کی جائے،اس پر   انجینئری اور طبیعیات کے جریدوں  میں لاتعداد  مضامین  لکھے جائیں گے؛ اگر غلط کی جائے، اخبار   کا سرنامہ بنے گا اور قانونی کارروائی ہو گی۔

\جزوحصہء{توازن}
ذیل اجسام پر غور کریں: (1) میز پر  پڑی ساکن کتاب، (2) بلا رگڑ سطح پر مستقل سمتی رفتار سے حرکت پذیر قرص، (3)  چھت کے پنکھے کے چکر کھاتے پَر، اور (4)  سیدھی راہ پر چلتے سائیکل کا پہیا۔ ان چار اجسام کے لئے
\begin{enumerate}[1.]
\item
مرکز کمیت کا خطی معیار حرکت \عددی{\vec{P}} ایک مستقل ہے۔
\item
مرکز کمیت یا کسی دوسرے نقطہ کے لحاظ سے ان کا زاوی معیار حرکت \عددی{\vec{L}} بھی ایک مستقل ہے۔
\end{enumerate}

ہم کہتے ہیں یہ جسم\اصطلاح{ توازن }\فرہنگ{توازن}\حاشیہب{equilibrium}\فرہنگ{equilibrium} میں ہیں۔ یوں توازن کے  دو  شرائط ذیل ہیں۔
%eq 12.1
\begin{align}\label{مساوات_توازن_تعریف_الف}
\vec{L}=\text{\RL{مستقل}}\quad \text{\RL{اور}}\quad \vec{P}=\text{\RL{مستقل}}
\end{align}

اس باب میں ہم صرف  ان صورتوں پر غور کرتے ہیں جہاں  مساوات  \حوالہ{مساوات_توازن_تعریف_الف} میں مستقل کی قیمت صفر ہو؛ یعنی   ہم ان اجسام میں دلچسپی رکھتے ہیں جو حوالہ چوکھٹ کے لحاظ سے  ساکن ہوں؛  خطی سکون اور گھمیری سکون میں ہم دلچسپی رکھتے ہیں۔ ایسے اجسام\اصطلاح{ سکونی توازن }\فرہنگ{توازن!سکونی}\حاشیہب{static equilibrium}\فرہنگ{equilibrium!static} میں ہوں گے۔ باب کے آغاز میں چار  اجسام میں صرف میز پر پڑی کتاب سکونی توازن میں ہے۔

شکل \حوالہء{12.1} میں   دکھائی گئی چٹان  ، فی الحال ، سکونی توازن میں ہے۔مساجد،  پُل، گھر، وغیرہ بھی سکونی توازن میں ہیں؛ یہ وقت  گزرنے کے باوجود  ساتھ  ساکن رہتے ہیں۔

جیسا ہم حصہ \حوالہء{8.3} میں ذکر کر چکے ، اگر  سکونی توازن سے قوت کے بل بوتے  پر   نکالے جانے  کے بعد جسم واپس  سکونی توازن  کو لوٹے،ہم کہتے ہیں یہ  جسم\ترچھا{ مستحکم } سکونی توازن میں ہے۔ نصف کرہ کے تل میں رکھا گیا  کنچا اس کی ایک مثال ہے۔ اس کے برعکس، اگر  چھوٹی قوت جسم کو ہلا کر  توازن ختم کر پائے، جسم \ترچھا{ غیر مستحکم } سکونی توازن میں ہو گا۔

\موٹا{زنجیری اثر۔}\quad
فرض کریں ہم  ایک اینٹ یوں کھڑی کریں کہ اس کا مرکز کمیت عین ایک  کنارے کے اوپر ہو (شکل \حوالہء{12.2a})۔  تجاذبی قوت \عددی{\vec{F}_g} کا خط عمل  اسی کنارے  سے گزرتا ہے لہٰذا  اس کنارے  پر \عددی{\vec{F}_g} کی قوت مروڑ  صفر ہو گی۔ اینٹ توازن میں ہے۔معمولی   اضطراب  اس توازن کو برباد کر دیگا۔ جیسے ہی \عددی{\vec{F}_g} کا خط عمل  کنارے  سے  ایک  طرف ہو (شکل \حوالہء{12.2b})، \عددی{\vec{F}_g} کی پیدا کردہ قوت مروڑ اینٹ کو اس طرف گھمائے گی۔ یوں شکل \حوالہء{12.2a} میں اینٹ غیر مستحکم توازن میں ہے۔

شکل \حوالہء{12.2c} میں اینٹ اتنی غیر مستحکم نہیں۔ اینٹ   گرانے کے لئے ضروری ہے کہ  قوت   اینٹ  اتنی گھمائے کہ اینٹ کا مرکز کمیت کنارے کو پار کر جائے۔ معمولی قوت اس اینٹ کو نہیں گرا سکتی، تاہم  انگلی سے جھٹکا دے کر اسے گرایا جا سکتا ہے۔(اینٹوں کو قطار میں کھڑا کر کے ، پہلی اینٹ کو جھٹکا دے کر گرانے سے تمام اینٹیں گرائی جا سکتی ہیں۔)

\موٹا{سل۔}\quad
شکل \حوالہء{12.2d} میں دکھایا گیا سل مزید زیادہ مستحکم ہے۔  مرکز کمیت کو سل کے کنارے کی دوسری طرف لی جانے کے لئے مرکز کمیت کو  کافی زیادہ  دور لے جانا ہو گا۔ انگلی کا جھٹکا سل کا   پاسا  نہیں پلٹ سکتا۔ (اسی لئے سل قطار میں رکھ کر زنجیری اثر  پیدا  نہیں کیا جا سکتا۔) شکل \حوالہء{12.3} میں  شہتیر  پر بیٹھا مزدور سل کی مانند جبکہ اس پر کھڑا مزدور اینٹ کی مانند ہو گا (جس کو ہوا کا جھٹکا نیچے لا سکتا ہے)۔

سکونی توازن اطلاقی انجینئری   کے لئے  بہت  ضروری ہے۔ تخلیق کار  تمام بیرونی قوت اور قوت مروڑ کی نشاندہی کر کے، بہتر تراکیب اور  مواد  استعمال کر کے، یقینی بناتا ہے کہ ان کی موجودگی کے باوجود عمارت یا مشین مستحکم رہے۔ یوں پُل  کا نقشہ تیار کرتے وقت  تخلیق کار تفصیلی تجزیہ کر کے   یقینی بناتا ہے  کہ پُل پر  آمد و رفت اور ہوائی قوتوں   کو پُل   سہ سکے۔

\جزوحصہء{توازن کے شرائط}
جسم کی   مستقیم  حرکت ، خطی معیار حرکت کے روپ میں   نیوٹن کے قانون دوم کو ، جو (ذیل)  مساوات \حوالہء{9.27} دیتی ہے،    مطمئن کرتی ہے۔
%eq 12.2
\begin{align}
\vec{F}_{\text{\RL{صافی}}}=\frac{\dif \vec{P}}{\dif t}
\end{align}
اگر جسم مستقیم  توازن  میں ہو؛ یعنی اگر  \عددی{\vec{P}} ایک مستقل ہو،  تب \عددی{\dif\vec{P}\!/\!\dif t=0} ہو گا لہٰذا لازماً درج ذیل ہو گا۔
%eq 12.3
\begin{align}\label{مساوات_توازن_شرط_ایک}
\vec{F}_{\text{\RL{صافی}}}=0\quad\quad\text{\RL{(متوازن قوت}}
\end{align}

جسم کی    گھمیری   حرکت ، زاوی  معیار حرکت کے روپ میں   نیوٹن کے قانون دوم کو ، جو (ذیل)  مساوات \حوالہ{مساوات_لڑھکاو_ذروں_نظام_ت} دیتی ہے،    مطمئن کرتی ہے۔
%eq 12.4
\begin{align}
\vec{\tau}_{\text{\RL{صافی}}}=\frac{\dif \vec{L}}{\dif t}
\end{align}
اگر جسم  گھمیری   توازن میں ہو؛ یعنی اگر  \عددی{\vec{L}} ایک مستقل ہو،  تب \عددی{\dif\vec{L}\!/\!\dif t=0} ہو گا لہٰذا لازماً درج ذیل ہو گا۔
%eq 12.5
\begin{align}\label{مساوات_توازن_شرط_دو}
\vec{\tau}_{\text{\RL{صافی}}}=0\quad\quad\text{\RL{(متوازن قوت مروڑ)}}
\end{align}
یوں  جسم کا توازن میں ہونے کے لئے ذیل دو شرائط   ہیں۔

\ابتدا{قاعدہء}
\begin{enumerate}[1.]
\item
جسم پر تمام بیرونی قوتوں کا سمتی مجموعہ صفر  ہونا لازم ہے۔
\item
ہر ممکنہ  نقطہ  کے لحاظ سے، جسم پر بیرونی قوت مروڑ  کا سمتی مجموعہ صفر ہونا   لازم ہے۔
\end{enumerate}
\انتہا{قاعدہء}
%------------------------------------


ہاں یہ شرائط\ترچھا{ سکونی } توازن کے لئے بھی ہیں۔ یہ شرائط عمومی صورت کے لئے بھی درست ہیں، جہاں \عددی{\vec{P}} اور \عددی{\vec{L}} مستقل ضرور لیکن غیر صفر ہوں۔

مساوات \حوالہ{مساوات_توازن_شرط_ایک} اور مساوات \حوالہ{مساوات_توازن_شرط_دو}، بطور سمتی مساوات، درحقیقت (ذیل)  تین تین جزوی مساوات  کی معادل ہیں۔
%eq 12.6
\begin{gather}
\begin{aligned}\label{مساوات_توازن_چہ}
&\text{\RL{متوازن قوت}}  & \text{\RL{متوازن قوت مروڑ}}\\
&F_{\text{\RL{صافی}},x}=0   &\tau_{\text{\RL{صافی}},x}=0\\
&F_{\text{\RL{صافی}},y}=0  &\tau_{\text{\RL{صافی}},y}=0\\
&F_{\text{\RL{صافی}},z}=0  &\tau_{\text{\RL{صافی}},z}=0
\end{aligned}
\end{gather}

\موٹا{اصل مساوات۔}\quad
ہم صرف  ان صورتوں پر غور کرتے ہیں جس میں جسم پر لاگو قوت \عددی{xy} مستوی میں پائے جاتے ہیں۔ یوں مسئلہ کم  پیچیدہ ہو گا۔ اس طرح جسم پر عمل پیرا قوت صرف محور \عددی{z} کی  متوازی محور   کے گرد  جسم  گھما سکتے ہیں۔ اس مفروضے کے ساتھ  مساوات \حوالہ{مساوات_توازن_چہ} میں سے قوت کی ایک مساوات اور قوت مروڑ کی دو مساوات سے چھٹکارا  حاصل ہو گا۔ یوں ذیل باقی رہتی ہیں۔
%eq 12.7, 12.8, 12.9
\begin{align}
F_{\text{\RL{صافی}},x}&=0 \label{مساوات_توازن_شرائط_الف} \\
F_{\text{\RL{صافی}},y}&=0  \label{مساوات_توازن_شرائط_ب} \\
\tau_{\text{\RL{صافی}},z}&=0 \label{مساوات_توازن_شرائط_پ}  
\end{align}
یہاں، \عددی{\tau_{\text{\RL{صافی}},z}}  وہ صافی قوت مروڑ ہے جو محور \عددی{z} یا اس کے متوازی کسی محور پر  بیرونی  قوت پیدا کرتی ہیں۔

جمی ہوئی   برف  پر مستقل سمتی رفتار سے حرکت کرتا قرص مساوات \حوالہ{مساوات_توازن_شرائط_الف}، مساوات \حوالہ{مساوات_توازن_شرائط_ب}، اور مساوات \حوالہ{مساوات_توازن_شرائط_پ}   مطمئن کرتا ہے، لہٰذا یہ توازن میں ہو گا،\ترچھا{ تاہم یہ سکونی توازن میں  ہرگز نہیں}۔ سکونی توازن کے  لئے قرص کا خطی معیار حرکت \عددی{\vec{P}} ایک مستقل ہونے کے ساتھ ساتھ  صفر ہونا  لازم  ہے؛ قرص کا جمی ہوئی برف پر ساکن ہونا لازم ہے۔ یوں، سکونی توازن کے لئے درج ذیل  شرط  بھی لازم   ہے۔

\ابتدا{قاعدہء}
جسم کے  خطی معیار حرکت \عددی{\vec{P}}  کا صفر ہونا لازم ہے۔
\انتہا{قاعدہء}

%----------------------
%Checkpoint 1 p330
\ابتدا{آزمائش}
یکساں سلاخ ، جس پر سلاخ کو عمود دار دو یا دو سے زیادہ قوت عمل کرتی ہیں،  کے چھ فضائی نظارے شکل \حوالہء{؟؟} میں پیش ہیں۔ قوتوں کی قدریں      (غیر صفر رکھ کر اور)   تبدیل کر کے  کون کونسی سلاخ سکونی توازن میں لائی جا سکتی ہیں؟
\انتہا{آزمائش}
%------------------

%the center of gravity p330
\جزوحصہء{مرکز ثقل}
جسم پر تجاذبی قوت  ، جسم کے انفرادی ٹکڑوں (جوہر) پر تجاذبی قوتوں کا سمتی مجموعہ ہو گا۔ انفرادی ٹکڑوں کی بات کرتے ہوئے ہم ذیل کہتے ہیں۔

\ابتدا{قاعدہء}
جسم پر تجاذبی قوت \عددی{\vec{F}_g} \قول{    عملاً}   جسم کے \اصطلاح{ مرکز ثقل }\فرہنگ{مرکز ثقل}\حاشیہب{center of gravity}\فرہنگ{center of gravity} پر    عمل کرتی ہے۔
\انتہا{قاعدہء}
%---------------------------

یہاں لفظ \قول{عملاً} کا مطلب یہ ہے کہ  اگر کسی طرح انفرادی ٹکڑوں  پر تجاذبی قوت ختم کر دی جائے اور تجاذبی قوت  \عددی{\vec{F}_g} جسم کے مرکز ثقل پر پیدا کر دی جائے، جسم پر صافی قوت اور  (کسی بھی محور کے لحاظ سے)  جسم پر صافی قوت مروڑ تبدیل نہیں ہوں گی۔

اب تک،  ہم فرض کرتے رہے ہیں کہ تجاذبی قوت  \عددی{\vec{F}_g} جسم کے مرکز کمیت پر عمل کرتی ہے، جو   اس  کے  مترادف  ہے  کہ ہم کہیں جسم کا مرکز ثقل  جسم کے مرکز کمیت  پر پایا جاتا ہے۔ یاد  کریں، کمیت \عددی{M}   جسم پر تجاذبی  قوت \عددی{\vec{F}_g=M\vec{g}}     عمل کرتی ہے، جہاں \عددی{\vec{g}}   جسم کا وہ اسراع ہے جو جسم پر \عددی{\vec{F}_g} لاگو کرنے سے پیدا ہو گا۔نیچے دیے گئے  ثبوت میں ہم ذیل ثابت کریں گے۔

%----------------------
%p331
\ابتدا{قاعدہء}
اگر جسم کے تمام ٹکڑوں کے لئے \عددی{\vec{g}}  ایک ہو، جسم کا مرکز ثقل اور جسم کا مرکز کمیت ایک نقطے پر ہوں گے۔
\انتہا{قاعدہء}
%-------------------------------

سطح زمین پر \عددی{\vec{g}} کی قدر بہت کم تبدیل ہوتی ہے اور  (عام زندگی میں جن بلندیوں سے  ہمیں واسطہ پڑتا ہے ان) بلندی کے ساتھ   \عددی{\vec{g}}کی قدر زیادہ تبدیل نہیں ہوتی  لہٰذا روز مرہ اشیاء  کے لئے  درج بالا تخمیناً درست ہو گا۔ یوں  چوہے یا بھینس  کے لئے تجاذبی قوت کا   ان کے مرکز کمیت پر عمل پیرا   ہونا  فرض کرنا درست ہو گا۔ ذیل ثبوت کے بعد ہم اسی مفروضے پر چلیں گے۔

\جزوجزوحصہء{ثبوت}
ہم  جسم کے انفرادی ٹکڑوں پر  پہلے غور کرتے ہیں۔ شکل \حوالہء{12.4a} میں  وسیع جسم  ، جس کی کمیت \عددی{M} ، اور   جسم کا ایک چھوٹا ٹکڑا جس کی کمیت \عددی{m_i} ہے، پیش ہے۔ ہر ٹکڑے پر تجاذبی قوت \عددی{\vec{F}_{gi}}،   جو \عددی{m_i\vec{g}_i} کے برابر ہے، عمل کرتی ہے۔ \عددی{\vec{g}_i} میں زیرنوشت کہتی ہے \عددی{\vec{g}_i}   ٹکڑا \عددی{i}کے مقام پر ثقلی اسراع ہے  (دیگر ٹکڑوں کے لئے اس کی قیمت مختلف ہو سکتی ہے)۔

شکل \حوالہء{12.4a} میں  ہر ایک    ٹکڑے  پر قوت \عددی{\vec{F}_{gi}} عمل کر کے،  مبدا \عددی{O}  کے لحاظ سے ٹکڑے پر  قوت مروڑ \عددی{\tau_i} ، جس کا معیار اثر  کا بازو \عددی{x_i} ہے، پیدا کرتی ہے۔ مساوات \حوالہ{مساوات_گھماو_صافی_قوت_مروڑ_الف}  \عددی{(\tau=r_{\perp}F)}  کی راہ نمائی  میں ہم ہر ایک قوت مروڑ \عددی{\tau_i}   ذیل لکھ سکتے ہیں۔
%eq12.10
\begin{align}
\tau_i=x_iF_{gi}
\end{align}
یوں، جسم کے تمام ٹکڑوں پر صافی قوت مروڑ ذیل ہو گی۔
%eq12.11
\begin{align}\label{مساوات_توازن_ثبوت_صفر}
\tau_{\text{\RL{صافی}}}=\sum \tau_i=\sum x_i F_{gi}
\end{align}

اب، پورا  جسم   لیتے ہیں۔شکل \حوالہء{12.4b} میں جسم کے مرکز  ثقل  پر تجاذبی  قوت \عددی{\vec{F}_g} عمل کرتا دکھایا گیا ہے۔مبدا     \عددی{O} کے لحاظ سے   اس  قوت  کا معیار اثر کا بازو \عددی{x_{\text{\RL{مرکزکمیت}}}}  اور جسم پر  پیدا قوت  مروڑ \عددی{\tau}  ہے۔ مساوات \حوالہ{مساوات_گھماو_صافی_قوت_مروڑ_الف} دوبارہ استعمال کر کے یہ قوت مروڑ ذیل لکھی جا سکتی ہے۔
%eq1.212
\begin{align}\label{مساوات_توازن_ثبوت_الف}
\tau=x_{\text{\RL{مرکزثقل}}} F_g
\end{align}
جسم پر تجاذبی قوت \عددی{\vec{F}_g} ، جسم کے تمام ٹکڑوں پر تجاذبی  قوت \عددی{\vec{F}_{gi}} کا  مجموعہ ہو گا۔ یوں  مساوات \حوالہ{مساوات_توازن_ثبوت_الف} میں \عددی{F_g} کی جگہ \عددی{\sum F_{gi}} ڈال کر ذیل لکھا جائے گا۔
%eq12.13
\begin{align}\label{مساوات_توازن_ثبوت_ب}
\tau=x_{\text{\RL{مرکزثقل}}}\sum F_{gi}
\end{align}

یاد کریں، مرکز ثقل پر عمل پیرا  قوت \عددی{\vec{F}_g} سے پیدا قوت مروڑ  اس صافی  قوت مروڑ کے برابر ہو گا جو جسم کے تمام ٹکڑوں پر عمل پیرا قوت \عددی{\vec{F}_{g}} پیدا کرتی ہیں۔ (مرکز ثقل کی تعریف یہی ہے۔) یوں مساوات \حوالہ{مساوات_توازن_ثبوت_ب}  کا \عددی{\tau}، مساوات \حوالہ{مساوات_توازن_ثبوت_صفر}  کے  \عددی{\tau_{\text{\RL{صافی}}}}  کے برابر ہے۔ دونوں مساوات کو برابر رکھ کر ذیل لکھا جا سکتا ہے۔
\begin{align*}
x_{\text{\RL{مرکزثقل}}}\sum F_{gi}=\sum x_iF_{gi}
\end{align*}
\عددی{F_{gi}} کی جگہ \عددی{m_ig_i} ڈال کر ذیل حاصل ہو گا۔
%eq12.14
\begin{align}
x_{\text{\RL{مرکزثقل}}}\sum m_ig_i=\sum x_im_ig_i
\end{align}
اب کلیدی تصور پیش کرتے ہیں: اگر  ٹکڑوں کا مقامات   پر اسراع \عددی{g_i} ایک ہو، ہم \عددی{g_i} منسوخ کرکے ذیل لکھ سکتے ہیں۔
%eq12.15
\begin{align}\label{مساوات_توازن_ثبوت_ج}
x_{\text{\RL{مرکزثقل}}}\sum m_i=\sum x_im_i
\end{align}
تمام ٹکڑوں کی کمیتوں کا مجموعہ \عددی{\sum m_i} جسم کی کمیت \عددی{M} دیتا ہے۔ یوں  مساوات \حوالہ{مساوات_توازن_ثبوت_ج} ذیل لکھی جا سکتی ہے۔
%eq12.16
\begin{align}
x_{\text{\RL{مرکزثقل}}}=\frac{1}{M}\sum x_im_i
\end{align}
اس مساوات کا دایاں ہاتھ جسم کے مرکز ثقل (مساوات \حوالہء{9.4}) کا محدد \عددی{x_{\text{\RL{مرکزثقل}}}}  دیتی ہے۔ یوں  ثبوت مکمل ہوتا ہے۔ اگر جسم کے تمام ٹکڑوں کے مقام  پر تجاذبی اسراع ایک ہو، جسم کا مرکز ثقل اور مرکز کمیت  مماثل ہوں گے۔
%eq12.17
\begin{align}
x_{\text{\RL{مرکزثقل}}}=x_{\text{\RL{مرکزکمیت}}}
\end{align}

%-------------------------
%12.2 some examples of static equilibrium p332
\حصہ{سکونی توازن کی چند مثالیں}
\موٹا{مقاصد}\\
اس حصہ کو پڑھنے کے بعد آپ ذیل کے قابل ہوں گے۔
\begin{enumerate}[1.]
\item
سکونی توازن کے لئے قوت اور قوت مروڑ کی شرائط   کا اطلاق کر پائیں گے۔
\item
سمجھ پائیں گے کہ مبدا (جس کے لحاظ سے قوت مروڑ کا حساب کیا جائے گا)  کا مقام سوچ سمجھ کر منتخب کرنے سے ایک یا ایک سے زیادہ نا معلوم قوت کو قوت مروڑ کی مساوات سے خارج کرنا ممکن ہو گا، جس سے قوت مروڑ کا حساب آسان ہو گا۔
\end{enumerate}

\موٹا{کلیدی تصور}\\
\begin{itemize}
\item
جب استوار جسم   ساکن حالت میں ہو ہم کہتے ہیں وہ سکونی توازن  میں ہے۔ایسے جسم کے لئے، جسم پر بیرونی قوتوں کا سمتی مجموعہ صفر کے برابر ہو گا۔
\begin{align*}
\vec{F}_{\text{\RL{صافی}}}=0 \quad\quad\text{\RL{(متوازن قوت)}}
\end{align*}
اگر تما قوت \عددی{xy} مستوی میں ہوں،درج بالا  سمتی مساوات   ذیل دو جزوی مساوات کے  مترادف ہو گی۔
\begin{align*}
F_{\text{\RL{صافی}},y}=0\quad \text{\RL{اور}}\quad F_{\text{\RL{صافی}},x}=0 \quad\quad\text{\RL{(متوازن قوت)}}
\end{align*}
\item
سکونی توازن  سے یہ بھی مراد ہے کہ، \ترچھا{ کسی بھی } نقطہ کے لحاظ سے،   جسم پر بیرونی قوت مروڑ کا سمتی مجموعہ  صفر کے برابر ہو گا۔
\begin{align*}
\vec{\tau}_{\text{\RL{صافی}}}=0\quad\quad\text{\RL{(متوازن قوت مروڑ)}}
\end{align*}
اگر بیرونی قوت \عددی{xy} مستوی میں ہوں، تمام قوت مروڑ محور \عددی{z} کے متوازی ہوں گی، اور درج بالا سمتی مساوات  ذیل  جزوی مساوات کی  مماثل ہو گی۔
\begin{align*}
\tau_{\text{\RL{صافی}},z}=0\quad\quad\text{\RL{(متوازن قوت مروڑ)}}
\end{align*}
\end{itemize}

%--------------------
%Some examples of static equilibrium p332
\جزوحصہء{سکونی توازن کی چند مثالیں}
یہاں ہم سکونی توازن کے کئی نمونی مسائل  پر  غور کریں گے۔ہر مسئلے میں ایک یا ایک سے زیادہ اجسام پر مبنی نظام منتخب کر کے توازن کی مساوات (مساوات \حوالہ{مساوات_توازن_شرائط_الف}، مساوات \حوالہ{مساوات_توازن_شرائط_ب}، اور مساوات \حوالہ{مساوات_توازن_شرائط_پ}) کا اطلاق کریں گے۔  تمام قوت \عددی{xy} مستوی میں ہیں لہٰذا قوت مروڑ \عددی{z} محور کو متوازی ہوں گے۔ یوں، مساوات \حوالہ{مساوات_توازن_شرائط_پ} کا اطلاق  کرتے ہوئے ، ہم  محور \عددی{z} کے متوازی  قوت مروڑ کی محور منتخب کرتے ہیں۔اگرچہ محور \عددی{z}  کے متوازی ہر محور پر مساوات  \حوالہ{مساوات_توازن_شرائط_پ}  کا اطلاق ممکن ہے، جیسا آپ دیکھیں گے، بعض محور کے انتخاب کی صورت میں   ایک یا ایک سے زیادہ نا معلوم قوت   خارج  ہوں گی، جس کی بدولت  مساوات  \حوالہ{مساوات_توازن_شرائط_پ} کا حل نسبتاً آسان ہو گا۔

%--------------
%Checkpoint 2 p332
\ابتدا{آزمائش}
یکساں سلاخ ، جو  سکونی توازن میں ہے، کا فضائی جائزہ شکل \حوالہء{؟؟} میں پیش ہے۔ (ا)  کیا قوتوں کو متوازن کر کے آپ \عددی{\vec{F}_1} اور \عددی{\vec{F}_2} کی قدریں  تلاش کر سکتے ہیں؟ (ب)  \عددی{\vec{F}_2} کی قدر تلاش کرنے کے لئے، محور گھماو کس نقطہ پر رکھ کر \عددی{\vec{F}_1} کو مساوات سے خارج کیا جا سکتا ہے؟ (ج)  \عددی{\vec{F}_2} کی قدر \عددی{\SI{65}{\newton}} حاصل ہو گی۔ \عددی{\vec{F}_1} کی قدر کیا ہے؟
\انتہا{آزمائش}
%-----------------------

%sample problem 12.01 balancing a horizontal beam p333
\ابتدا{نمونی سوال}\موٹا{افقی شہتیری   متوازن بنانا}\\
شکل \حوالہء{12.5a} میں، کمیت \عددی{m=\SI{1.8}{\kilo\gram}} کی یکساں  شہتیری ، جس کی لمبائی \عددی{L} ہے، دو ترازو پر رکھی گئی ہے۔ کمیت \عددی{M=\SI{2.7}{\kilo\gram}}  کی یکساں سل شہتیری پر رکھی گئی ہے۔سل کا مرکز شہتیری کے  بائیں سر سے \عددی{L\!/\!4}  فاصلے پر ہے۔ ترازو کیا وزن دیں گے؟

\جزوحصہء{کلیدی تصورات}
سکونی توازن کا کوئی بھی مسئلہ حل کرنے سے پہلے ذیل کرنا ہو گا: نظام کی نشاندہی  کریں اور اس کا آزاد جسمی  خاکہ بنائیں، جس پر تمام قوتوں  کی نشاندہی   ہو۔ یہاں ہم شہتیری اور سل کو  نظام مانتے ہیں۔ اس کے بعد ، نظام پر قوت دکھائیں، جیسا شکل \حوالہء{12.5b}  کے آزاد جسمی خاکہ میں کیا گیا ہے۔(نظام کے  انتخاب کے لئے تجربہ درکار ہے، اور عموماً ایک سے زیادہ  ممکنات ہوں گے۔) نظام سکونی توازن میں ہے ، لہٰذا قوتوں کے توازن  کی مساوات  (مساوات \حوالہ{مساوات_توازن_شرائط_الف} اور مساوات \حوالہ{مساوات_توازن_شرائط_ب}) اور قوت مروڑ کے توازن کی مساوات (مساوات \حوالہ{مساوات_توازن_شرائط_پ}) کا اطلاق کیا جا سکتا ہے۔

\موٹا{حساب:}\quad
بائیں ترازو سے شہتیری پر عمودی قوت \عددی{\vec{F}_l} اور  دائیں ترازو سے عمودی  قوت  \عددی{\vec{F}_r} ہے۔ہم ان قوت کی قدریں جاننا چاہتے ہیں۔ تجاذبی قوت \عددی{\vec{F}_{g,\text{\RL{شہتیر}}}}،    جو  \عددی{m\vec{g}} کے برابر ہے، شہتیری  کے مرکز کمیت پر عمل کرتی ہے۔ اسی طرح،  سل  پر تجاذبی قوت \عددی{\vec{F}_{g,\text{\RL{سل}}}}، جو \عددی{M\vec{g}} کے برابر ہے، سل کے مرکز کمیت پر عمل کرتی ہے۔تاہم،  شکل \حوالہء{12.5b} سادہ بنانے کی غرض سے، سل کو نقطہ سے ظاہر کیا گیا ہے، اور سمتیہ \عددی{\vec{F}_{g,\text{\RL{سل}}}}  کی دم اس نقطہ پر رکھی گئی ہے۔ (سمتیہ  \عددی{\vec{F}_{g,\text{\RL{سل}}}} کا رخ تبدیل کیے بغیر،  قوت کے خط عمل پر سمتیہ   کی  گھساٹ ، شکل کو عمود دار کسی بھی محور پر،  \عددی{\vec{F}_{g,\text{\RL{سل}}}} کی قوت مروڑ تبدیل نہیں کرتی۔)

قوتوں کا \عددی{x} جزو موجود نہیں لہٰذا  مساوات \حوالہ{مساوات_توازن_شرائط_الف}  \عددی{(F_{\text{\RL{صافی}},x}=0)} کوئی معلومات فراہم نہیں کرتی۔ مساوات \حوالہ{مساوات_توازن_شرائط_ب} \عددی{(F_{\text{\RL{صافی}},y}=0)}    \عددی{y} اجزاء کے لئے ذیل دیتی ہے۔
%eq12.18
\begin{align}\label{مساوات_توازن_نمونی_الف}
F_l+F_r-Mg-mg=0
\end{align}

اس مساوات میں دو نا معلوم   قوت، \عددی{F_l} اور \عددی{F_r}،   موجود ہیں لہٰذا ہمیں قوت مروڑ کے توازن کی  مساوات \حوالہ{مساوات_توازن_شرائط_پ} بھی  استعمال کرنی ہو گی۔ ہم شکل \حوالہء{12.5} کے مستوی کو عمود دار کسی بھی محور گھماو  پر مساوات کا اطلاق کر سکتے ہیں۔ آئیں شہتیری کے بائیں سر پر محور گھماو رکھ کر حل کریں۔ ہم قوت مروڑ کو علامت مختص کرنے کا عمومی  طریقہ بروئے کار لائیں گے: اگر ساکن جسم کو  محور گھماو پر قوت مروڑ گھڑی وار گھمانے کی کوشش کرے، قوت مروڑ منفی  ہو گی؛ اگر خلاف گھڑی گھمانے کی کوشش کرے، قوت مروڑ مثبت ہو گی۔ آخر میں ہم قوت مروڑ \عددی{r_{\perp}F} روپ میں لکھتے ہیں، جہاں  \عددی{\vec{F}_l} کے لئے \عددی{r_{\perp}} کی قیمت \عددی{0}، \عددی{M\vec{g}} کے لئے \عددی{L\!/\!4}، \عددی{m\vec{g}} کے لئے \عددی{L\!/\!2}، اور \عددی{\vec{F}_r} کے لئے \عددی{L} ہے۔

 اب ہم توازن کی مساوات \عددی{\tau_{\text{\RL{صافی}}}=0}  ذیل لکھ سکتے ہیں
 \begin{align*}
 (0)(F_l)-(L\!/\!4)(Mg)-(L\!/\!2)(mg)+(L)(F_r)=0
 \end{align*}
 جو  ذیل دیگی۔
 \begin{align*}
 F_r&=\frac{1}{4}Mg+\frac{1}{2}mg\\
 &=\frac{1}{4}(\SI{2.7}{\kilo\gram})(\SI{9.8}{\meter\per\second\squared})+\frac{1}{2}(\SI{1.8}{\kilo\gram})(\SI{9.8}{\meter\per\second\squared})\\
 &=\SI{15.44}{\newton}\approx\SI{15}{\newton}\quad\quad\text{\RL{(جواب)}}
 \end{align*}
 اب \عددی{F_l} کے لئے  مساوات \حوالہ{مساوات_توازن_نمونی_الف} حل کر کے درج بالا نتیجہ پر کر کے ذیل حاصل کرتے ہیں۔
 \begin{align*}
 F_l&=(M+m)g-F_r\\
 &=(\SI{2.7}{\kilo\gram}+\SI{1.8}{\kilo\gram})(\SI{9.8}{\meter\per\second\squared})-\SI{15.44}{\newton}\\
 &=\SI{28.66}{\newton}\approx\SI{29}{\newton}\quad\quad\text{\RL{(جواب)}}
 \end{align*}
 
\ترچھا{ لائحہ  عمل پر غور کریں:} قوت  کے توازن کی مساوات لکھ کر،   دو نا معلوم  متغیرات  کی بنا، ہم  پھنس گئے۔ اگر ہم بغیر سوچے سمجھے  کسی  محور پر قوت مروڑ کے توازن کی مساوات لکھتے، ہمیں  وہاں بھی دو نا معلوم متغیرات کا سامنا ہوتا۔ تاہم،  ایک نا معلوم قوت، جو یہاں \عددی{\vec{F}_l}  ہے،  کے نقطہ اطلاق سے گزرتی محور منتخب کر کے، ہم پھنسنے سے بچ گئے۔اس انتخاب کی بدولت، قوت مروڑ کے توازن کی مساوات سے نا معلوم   قدر \عددی{F_l} خارج ہوتی ہے، اور یوں ہم مساوات حل کر کے \عددی{F_r} دریافت کرنے میں کامیاب ہوئے۔ اس کے بعد، قوت کے توازن کی مساوات دوبارہ لیتے ہوئے باقی قوت کی قدر معلوم کرنا ممکن ہوا۔
\انتہا{نمونی سوال}
%-------------------------

%sample problem 12.02 balancing a lean boom p334
\ابتدا{نمونی سوال}\موٹا{چول دار بازو  متوازن بنانا}\\
شکل \حوالہء{12.6a} میں (کمیت \عددی{M=\SI{430}{\kilo\gram}} کی)  تجوری کو معاون  چول دار بازو سے     بلا کمیت رسی   کے ذریعے لٹکا دکھایا گیا 
ہے، جہاں \عددی{a=\SI{1.9}{\meter}} اور \عددی{b=\SI{2.5}{\meter}} ہے۔ بازو کی کمیت \عددی{m=\SI{85}{\kilo\gram}} ، اور افقی رسا  بلا کمیت ہے۔

(ا) رسا میں تناو \عددی{T_c} کیا ہے؟ دوسرے لفظوں میں بازو پر رسا کی قوت  \عددی{\vec{T}_c} کی قدر  کیا ہے؟

\جزوحصہء{کلیدی تصورات}
یہاں نظام چول دار بازو ہے، جس پر عمل پیرا قوت شکل \حوالہء{12.6b} کے آزاد جسمی خاکے میں پیش ہیں۔ رسا سے  بازو پر قوت \عددی{\vec{T}_c} ہے۔ تجاذبی قوت  جو \عددی{m\vec{g}} کے برابر ہے، بازو کے مرکز کمیت ( بازو کے وسط)  پر عمل کرتی ہے۔ چول سے بازو پر  قوت کا انتصابی جزو  \عددی{\vec{F}_v}،  اور  افقی جزو \عددی{\vec{F}_h} ہے۔ رسی سے بازو پر قوت \عددی{\vec{T}_r} ہے۔ بازو، رسی، اور تجوری  ساکن ہیں، لہٰذا \عددی{\vec{T}_r} کی قدر تجوری کے وزن کے برابر:\عددی{T_r=Mg} ہو گی۔ ہم \عددی{xy} محددی نظام کا مبدا \عددی{O} چول پر رکھتے ہیں۔نظام سکونی توازن  میں ہے، لہٰذا  اس پر توازن کی  مساوات کا اطلاق ہو گا۔

\موٹا{حساب:}\quad
مساوات \حوالہ{مساوات_توازن_شرائط_پ}  \عددی{(\tau_{\text{\RL{صافی}},z}=0)} سے آغاز کرتے ہیں۔  یاد رہے، ہم قوت \عددی{\vec{T}_c} کی قدر جاننا چاہتے ہیں، نا کہ  نقطہ \عددی{O} پر موجود چال پر عمل پیرا قوت \عددی{\vec{F}_v} اور \عددی{\vec{F}_h} کی قدریں۔ قوت مروڑ کے حساب سے   \عددی{\vec{F}_v} اور \عددی{\vec{F}_h} خارج کرنے کی غرض سے ہم نقطہ \عددی{O} سے گزرتی، شکل کے مستوی کو عمود دار محور گھماو منتخب کرتے ہیں۔ یوں  \عددی{\vec{F}_v} اور \عددی{\vec{F}_h} کے  معیار اثر کا بازو صفر ہوں گے۔ شکل \حوالہء{12.6b} میں \عددی{\vec{T}_c}، \عددی{\vec{T}_r}، اور \عددی{m\vec{g}} کے خط عمل نقطہ دار ہیں۔ مطابقتی معیار اثر کا بازو \عددی{a}، \عددی{b}، اور \عددی{b\!/\!2} ہیں۔

قوت مروڑ کو \عددی{r_{\perp}F} روپ میں لکھ کر، قوت مروڑ کی علامت کا قاعدہ استعمال کر کے، توازن کی مساوات \عددی{\tau_{\text{\RL{صافی}},z}=0} ذیل لکھی جائے گی۔
%eq12.19
\begin{align}
(a)(T_c)-(b)(T_r)-(\tfrac{1}{2}b)(mg)=0
\end{align}
\عددی{T_r} کی جگہ \عددی{Mg}  ڈال کر \عددی{T_c} کے لئے حل کر کے ذیل حاصل ہو گا۔
\begin{align*}
T_c&=\frac{gb(M+\tfrac{1}{2}m)}{a}\\
&=\frac{(\SI{9.8}{\meter\per\second\squared})(\SI{2.5}{\meter})(\SI{430}{\kilo\gram}+85\!/\!2\,\si{\kilo\gram})}{\SI{1.9}{\meter}}\\
&=\SI{6093}{\newton}\approx\SI{6100}{\newton}\quad\quad\text{\RL{(جواب)}}
\end{align*}

(ب) چول سے بازو پر صافی قوت کی قدر \عددی{F} تلاش کریں۔

\جزوحصہء{کلیدی تصور}
اب ہمیں افقی جزو \عددی{F_h} اور انتصابی جزو \عددی{F_v} درکار ہیں، جن سے صافی قوت کی قدر \عددی{F} حاصل ہو گی۔ ہم \عددی{T_c} جانتے ہیں لہٰذا  بازو  پر قوت کی توازن کی  مساوات کا اطلاق کرتے ہیں۔

\موٹا{حساب:}\quad
افقی توازن کے لئے، ہم \عددی{F_{\text{\RL{صافی}},x}=0} ذیل لکھ سکتے ہیں:
%eq12.20
\begin{align}
F_h-T_c=0
\end{align}
اور یوں ذیل ہو گا۔
\begin{align*}
F_h=T_c=\SI{6093}{\newton}
\end{align*}
انتصابی جزو کے لئے ہم  \عددی{F_{\text{\RL{صافی}},y}=0}  کو درج ذیل لکھتے ہیں۔
\begin{align*}
F_v-mg-T_r=0
\end{align*}
\عددی{T_r} کی جگہ \عددی{Mg} ڈال کر \عددی{F_v} کے لئے حل کر کے ذیل حاصل ہو گا۔
\begin{align*}
F_v&=(m+M)g=(\SI{85}{\kilo\gram}+\SI{430}{\kilo\gram})(\SI{9.8}{\meter\per\second\squared})\\
&=\SI{5047}{\newton}
\end{align*}
مسئلہ فیثاغورث استعمال کر کے ذیل حاصل ہو گا۔
\begin{align*}
F&=\sqrt{F_h^2+F_v^2}\\
&=\sqrt{(\SI{6093}{\newton})^2+(\SI{5047}{\newton})^2}\approx \SI{7900}{\newton}\quad\text{\RL{(جواب)}}
\end{align*}
یاد رہے، \عددی{F} کی قیمت تجوری اور بازو کے مجموعی  وزن: \عددی{\SI{5000}{\newton}} ، یا افقی رسا میں تناو: \عددی{\SI{6100}{\newton}} سے کافی زیادہ ہے۔
\انتہا{نمونی سوال}
%-----------------------------

%sample problem 12.03 Balancing a leaning ladder p335
\ابتدا{نمونی سوال}\موٹا{دیوار کے ساتھ کھڑی سیڑھی}\\
شکل \حوالہء{12.7a} میں سیڑھی، جس کی لمبائی \عددی{L=\SI{12}{\meter}} اور  کمیت \عددی{m=\SI{45}{\kilo\gram}} ہے، چکنی      دیوار کے ساتھ کھڑی ہے (چکنی دیوار اور سیڑھی کے بیچ رگڑ نہیں ہو گی)۔ سیڑھی کا بالا سر فرش سے \عددی{h=\SI{9.3}{\meter}} بلندی پر ہے ، اور  سیڑھی  کا مرکز کمیت نچلے سر سے سیڑھی کے ہمراہ  \عددی{L\!/\!3} فاصلے پر ہے۔ فرش بلا رگڑ نہیں ہے۔ ایک شخص، جس کی کمیت \عددی{M=\SI{72}{\kilo\gram}} ہے،  سیڑھی چڑھتا ہے حتٰی کہ ، سیڑھی کے نچلے سر سے  شخص  کا مرکز کمیت \عددی{L\!/\!2} فاصلے پر ہوتا  ہے۔ سیڑھی پر دیوار اور فرش سے قوتوں کی قدریں کیا ہوں گی؟

\جزوحصہء{کلیدی تصورات}
ہم شخص اور سیڑھی کو  اپنا نظام   مان کر  نظام کا آزاد جسمی خاکہ، جس پر  عمل پیرا قوت دکھائے گئے ہیں،   بناتے ہیں (شکل \حوالہء{12.7b})۔ نظام سکونی توازن میں ہے، لہٰذا  اس پر قوت کی توازن اور قوت مروڑ کی توازن کی مساوات (مساوات \حوالہ{مساوات_توازن_شرائط_الف} تا مساوات \حوالہ{مساوات_توازن_شرائط_پ}) کا اطلاق  ممکن ہے۔

\موٹا{حساب:}\quad
شکل \حوالہء{12.7b} میں  شخص کو سیڑھی پر نقطے سے ظاہر کیا گیا ہے۔ شخص  پر تجاذبی قوت \عددی{M\vec{g}}  کے سمتیہ کو خط عمل (سمتیہ قوت سے گزرتی  اور اس کے ہمراہ لکیر)  پر گھسیٹ کر، سمتیہ کی دم  نقطے پر رکھی گئی ہے۔ (قوت یوں منتقل کرنے سے، شکل کو عمود دار،  کسی بھی محور گھماو کے لحاظ سے قوت مروڑ تبدیل نہیں ہوتی۔ یوں ، قوت مروڑ کی   توازن  کی مساوات، جو ہم استعمال کریں گے،  اثر انداز نہیں ہوتی۔)

دیوار سے سیڑھی پر صرف افقی قوت \عددی{\vec{F}_w} عمل کرتی ہے (بلا رگڑ دیوار پر رگڑی قوت موجود نہیں ہو سکتی، لہٰذا  سیڑھی پر دیوار کے ہمراہ انتصابی قوت صفر ہو گی)۔ فرش سے سیڑھی پر  قوت \عددی{\vec{F}_p} کا  افقی جزو \عددی{\vec{F}_{px}} ہے جو سکونی رگڑی قوت ہے، اور انتصابی جزو \عددی{\vec{F}_{py}}  ہے  جو عمودی قوت ہے۔

توازن کی مساوات  استعمال کرنے کی خاطر،  ہم مساوات \حوالہ{مساوات_توازن_شرائط_پ} \عددی{(\tau_{\text{\RL{صافی}},z}=0)}    سے آغاز کرتے ہیں، جو  قوت مروڑ کی توازن کی مساوات  ہے۔ قوت مروڑ کے حساب کے لئے محور گھماو منتخب کرتے وقت، یاد رہے سیڑھی کے دو سروں
 پر دو  نا معلوم  قوت (\عددی{\vec{F}_w} اور \عددی{\vec{F}_p}) پائے جاتے ہیں۔ ان میں سے ایک، مثلاً \عددی{\vec{F}_p}، خارج کرنے کے لئے ہم محور گھماو، شکل کے مستوی کو عمود دار، نقطہ \عددی{O} پر رکھتے ہیں (شکل \حوالہء{12.7b})۔ ہم محددی نظام کا مبدا بھی \عددی{O} پر رکھتے ہیں۔
  ہم  \عددی{O} پر قوت مروڑ مساوات \حوالہ{مساوات_گھماو_قوت_مروڑ_فائے} تا مساوات \حوالہ{مساوات_گھماو_صافی_قوت_مروڑ_الف}  میں سے کوئی ایک استعمال کر کے معلوم کر سکتے ہیں، تاہم  یہاں مساوات \حوالہ{مساوات_گھماو_صافی_قوت_مروڑ_الف} \عددی{(\tau=r_{\perp}F)}  کا استعمال  سب سے آسان ہے۔ \ترچھا{مبدا کا مقام سوچ سمجھ کر منتخب کرنے سے  قوت مروڑ کا حساب آسان بنایا جا سکتا ہے۔}
  
  دیوار سے افقی قوت \عددی{\vec{F}_w} کا معیار اثر کا بازو \عددی{r_{\perp}} معلوم کرنے کے لئے، ہم اس سمتیہ  کے اندر گزرتا خط عمل کھینچتے ہیں (شکل \حوالہء{12.7c} میں  اس کو نقطہ دار لکیر سے ظاہر کیا گیا ہے)۔ یوں \عددی{O} سے خط عمل تک عمود دار فاصلہ \عددی{r_{\perp}} ہو گا۔  شکل \حوالہء{12.7c} میں    \عددی{r_{\perp}}  محور \عددی{y} کے ہمراہ ، \عددی{h} کے برابر ہے۔ ہم اسی طرح تجاذبی قوت \عددی{M\vec{g}} اور \عددی{m\vec{g}}  کے خط عمل کھینچ کر دیکھتے ہیں کہ ان کا معیار اثر کا بازو محور \عددی{x} کے ہمراہ ہے۔ شکل \حوالہء{12.7a} میں دی گئی \عددی{a} کے لئے، معیار اثر کا بازو بالترتیب  \عددی{a\!/\!2} (شخص نصف سیڑھی چڑھ چکا ہے) اور \عددی{a\!/\!3}  ( سیڑھی کا مرکز کمیت، سیڑھی کے نچلے سر سے،   ایک تہائی فاصلے پر ہے)ہیں۔ چونکہ \عددی{\vec{F}_{px}} اور \عددی{\vec{F}_{py}}  مبدا پر عمل پیرا ہیں لہٰذا ان کے معیار اثر کا بازو صفر ہے۔
  
  قوت مروڑ \عددی{r_{\perp}F} روپ میں لکھ کر، توازن کی مساوات \عددی{\tau_{\text{\RL{صافی}},z}=0} ذیل لکھی جائے گی۔
  %eq12.21
  \begin{align}\label{مساوات_توازن_سیڑھی_نمونی_الف}
  -(h)(F_w)+(a\!/\!2)(Mg)+(a\!/\!3)(mg)+(0)(F_{px})+(0)(F_{py})=0
  \end{align}
  (مثبت قوت مروڑ خلاف گھڑی گھماو کے مترادف اور منفی قوت مروڑ گھڑی وار گھماو کے مترادف ہے۔)
  
سیڑھی، دیوار، اور فرش  قائمہ تکون بناتے ہیں، جس پر   مسئلہ فیثاغورث  کا اطلاق ذیل دیگا۔
\begin{align*}
a=\sqrt{L^2-h^2}=\SI{7.58}{\meter}
\end{align*}
اس کے بعد، مساوات \حوالہ{مساوات_توازن_سیڑھی_نمونی_الف} ذیل دیگی۔
\begin{align*}
F_w&=\frac{ga(M\!/\!2+m\!/\!3)}{h}\\
&=\frac{(\SI{9.8}{\meter\per\second\squared})(\SI{7.58}{\meter})(72\!/\!2\,\si{\kilo\gram}+45\!/\!3\,\si{\kilo\gram})}{\SI{9.3}{\meter}}\\
&=\SI{407}{\newton}\approx\SI{410}{\newton}\quad\quad\text{\RL{(جواب)}}
\end{align*}

اب ہمیں شکل \حوالہء{12.7d} اور  قوت کی توازن کی مساوات استعمال کرنی ہو گی۔ مساوات  \عددی{F_{\text{\RL{صافی}},x}=0}  ذیل دیگی:
\begin{align*}
F_w-F_{px}=0
\end{align*}
لہٰذا ذیل ہو گا۔
\begin{align*}
F_{px}=F_w=\SI{410}{\newton}\quad\quad\text{\RL{(جواب)}}
\end{align*}
مساوات  \عددی{F_{\text{\RL{صافی}},y}=0} ذیل دیگی:
\begin{align*}
F_{py}-Mg-mg=0
\end{align*}
لہٰذا ذیل ہو گا۔
\begin{align*}
F_{py}&=(M+m)g=(\SI{72}{\kilo\gram}+\SI{45}{\kilo\gram})(\SI{9.8}{\meter\per\second\squared})\\
&=\SI{1146.6}{\newton}\approx \SI{1100}{\newton}\quad\quad\text{\RL{(جواب)}}
\end{align*}

\انتہا{نمونی سوال}
%----------------------

%sample problem 12.04 balancing the leaning tower of Pisa   p337
