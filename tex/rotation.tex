%p257
\باب{گھماو}
\حصہ{گھماو کے متغیر}
\جزوحصہء{مقاصد}
اس حصہ کو پڑھنے کے بعد آپ درج ذیل کے قابل ہوں گے۔
\begin{enumerate}[1.]
\item
جان پائیں گے اگر جسم کے تمام حصے ایک  محور کے گرد  ہم قدم گھومیں، یہ   استوار  جسم ہو گا۔ (اس باب میں ایسے اجسام پر گفتگو کی جائے گی۔)
\item
جان پائیں گے کہ  اندرونی حوالہ لکیر اور مقررہ  بیرونی حوالہ لکیر  کے بیچ زاویہ،  استوار جسم کا زاویاتی مقام دیگا۔
\item
ابتدائی اور اختتامی زاویاتی مقام  کا زاویاتی ہٹاو کے ساتھ تعلق استعمال کر پائیں گے۔
\item
اوسط زاویائی سمتی رفتار،  زاویائی ہٹاو، اور ہٹاو کو درکار دورانیے کا  تعلق استعمال کر پائیں گے۔
\item
اوسط زاویائی  اسراع ،  زاویائی  سمتی رفتار میں تبدیلی، اور اس تبدیلی کو درکار دورانیے کا  تعلق استعمال کر پائیں گے۔
\item
جان پائیں گے کہ خلاف  گھڑی  حرکت مثبت  رخ اور گھڑی وار حرکت منفی  رخ ہو گا۔
\item
زاویائی مقام   کو\ترچھا{   وقت  کا تفاعل } جانتے ہوئے، کسی بھی لمحے پر لمحاتی زاویائی سمتی رفتار اور دو مختلف وقتوں کے بیچ اوسط زاویائی سمتی رفتار تعین کر  پائیں گے۔
\item
زاویائی مقام   بالمقابل   وقت   کی \ترچھا{ ترسیم }سے  کسی بھی لمحے پر لمحاتی زاویائی سمتی رفتار اور دو مختلف وقتوں کے بیچ اوسط زاویائی سمتی رفتار تعین کر  پائیں گے۔
\item
جان پائیں گے کہ لمحاتی زاویائی  سمتی رفتار  کی قدر لمحاتی زاویائی رفتار ہو گی۔
\item
زاویائی  سمتی رفتار    کو  \ترچھا{وقت  کا تفاعل } جانتے ہوئے، کسی بھی لمحے پر لمحاتی زاویائی  اسراع اور دو مختلف وقتوں کے بیچ اوسط زاویائی  اسراع تعین کر  پائیں گے۔
\item
زاویائی سمتی رفتار    بالمقابل   وقت   کی\ترچھا{ ترسیم }سے  کسی بھی لمحے پر لمحاتی زاویائی اسراع اور دو مختلف وقتوں کے بیچ اوسط زاویائی اسراع تعین کر  پائیں گے۔
\item
وقت کے ساتھ زاویائی اسراع تفاعل کا تکمل  لے کر جسم کی زاویائی سمتی رفتار میں تبدیلی تعین کر پائیں گے۔

وقت کے ساتھ زاویائی  سمتی رفتار  تفاعل کا تکمل  لے کر جسم کے  زاویائی مقام میں تبدیلی تعین کر پائیں گے۔
\end{enumerate}

\جزوحصہء{کلیدی تصور}
\begin{itemize}
\item
مقررہ محور،  جو محور گھماو  کہلاتی ہے،  کے گرد استوار جسم کا  گھماو  بیان کرنے کی خاطر ،    جسم کے اندر محور کو عمودی   حوالہ لکیر فرض کی جاتی ہے جو جسم کے ساتھ ہم قدم محور کے گرد گھومتی ہے۔   ایک مقررہ رخ کے ساتھ اس لکیر کا زاویائی مقام \عددی{\theta} ناپا جاتا ہے۔ جب \عددی{\theta} کی پیمائش ریڈیئن میں ہو، ذیل ہو گا،
\begin{align*}
\theta=\frac{s}{r}\quad\quad \text{\RL{(ریڈیئن ناپ)}}
\end{align*}
جہاں رداس \عددی{r} کے دائری راہ کا قوسی فاصلہ \عددی{s} اور ریڈیئن میں زاویہ \عددی{\theta} ہے۔
\item
زاویہ کی  درجہ میں اور چکر میں پیمائش کا ریڈیئن پیمائش سے تعلق ذیل ہے۔
\begin{align*}
\text{\RL{ریڈیئن}}\,2\pi=\SI{360}{\degree}=\text{\RL{چکر}}\,1
\end{align*}
\item
ایک جسم جو محور گھماو  کے گرد گھوم کر  اپنا زاویائی مقام \عددی{\theta_1} سے تبدیل کر کے \عددی{\theta_2} کرے،  ذیل زاویائی ہٹاو  سے گزرتا ہے،
\begin{align*}
\Delta \theta=\theta_2-\theta_1
\end{align*}
جہاں خلاف گھڑی گھماو کے لئے \عددی{\Delta \theta} مثبت اور گھڑی وار گھماو کے لئے منفی ہو گا۔
\item
اگر  جسم \عددی{\Delta t} دورانیہ میں \عددی{\Delta \theta} زاویائی ہٹاو  گھومے، اس کی اوسط زاویائی سمتی  رفتار  \عددی{\omega_{\text{\RL{اوسط}}}} ذیل ہو گی۔
\begin{align*}
\omega_{\text{\RL{اوسط}}}=\frac{\Delta \theta}{\Delta t}
\end{align*}
جسم کی ( لمحاتی ) زاویائی  سمتی رفتار \عددی{\omega} ذیل ہو گی۔
\begin{align*}
\omega=\frac{\dif \theta}{\dif t}
\end{align*}
اوسط  زاویائی سمتی رفتار \عددی{\omega_{\text{\RL{اوسط}}}}  اور سمتی رفتار  \عددی{\omega} دونوں سمتی مقادیر ہیں، جن کا رخ دایاں ہاتھ قاعدہ  دیگا۔ خلاف گھڑی گھماو کے لئے ان کا رخ مثبت اور گھڑی وار گھماو کے لئے منفی ہو گا۔ زاویائی سمتی رفتار کی قدر جسم  کی زاویائی رفتار ہو گی۔
\item
اگر \عددی{\Delta t=t_2-t_1} دورانیہ میں جسم کی زاویائی سمتی رفتار \عددی{\omega_1} سے تبدیل ہو کر  \عددی{\omega_2} ہو، اس کا  اوسط زاویائی  اسراع \عددی{\alpha_{\text{\RL{اوسط}}}} ذیل ہو گا۔
\begin{align*}
\alpha_{\text{\RL{اوسط}}}=\frac{\omega_2-\omega_1}{t_2-t_1}=\frac{\Delta \omega}{\Delta t}
\end{align*}
جسم کا  ( لمحاتی ) زاویائی اسراع \عددی{\alpha}ذیل ہو گا۔
\begin{align*}
\alpha=\frac{\dif \omega}{\dif t}
\end{align*}
\عددی{\alpha_{\text{\RL{اوسط}}}} اور \عددی{\alpha} دونوں سمتی مقادیر ہیں۔
\end{itemize}

\حصہء{طبیعیات کیا ہے؟}
جیسا ہم پہلے ذکر کر چکے، طبیعیات  کی توجہ کا ایک  مرکز \قول{ حرکیات }ہے۔ تاہم، اب تک ہم صرف\اصطلاح{ مستقیم   حرکت } پر بات کرتے رہے ہیں، جس میں جسم سیدھی یا قوسی  لکیر  پر حرکت کرتا ہے (شکل \حوالہء{10-1a})۔ اب ہم \اصطلاح{ گھماو } پر نظر ڈالتے ہیں، جس میں جسم کسی محور کے گرد گھومتا ہے (شکل \حوالہء{10.1b})۔

گھماو تقریباً ہر مشین میں نظر آتا ہے، اور جب  آپ دروازہ کھولتے ہیں آپ اس کو دیکھتے ہیں۔کھیل میں  گھماو اہم کردار ادا کرتا ہے، جیسا  گیند کو زیادہ دور پھینکنے کے لئے (گھومتے  گیند  کو ہوا زیادہ دیر  اٹھا  کر سکتی ہے)، اور کرکٹ میں گیند  قوسی  راہ پر پھینکنے کے لئے (گھومتے گیند کو ہوا دائیں یا بائیں دھکیلتی ہے)۔ گھماو زیادہ اہم مسائل ، جیسا      عمر رسیدہ  ہوائی جہاز میں دھاتی حصوں   کا ٹوٹ پھوٹ، میں بھی  کلیدی کردار ادا کرتا ہے۔

گھماو پر بحث سے قبل   ، حرکت میں ملوث متغیرات متعارف کرتے ہیں، جیسا ہم نے باب \حوالہء{2} میں مستقیم حرکت پر بحث سے قبل کیا۔ ہم دیکھتے ہیں کہ گھماو کے  متغیرات عین   با ب \حوالہء{2} میں یک بُعدی  حرکت  کے متغیرات کی طرح ہیں؛  ایک اہم خصوصی صورت وہ ہے جہاں اسراع (جو یہاں زاویائی اسراع ہو گا)   مستقل ہو۔ ہم دیکھتے ہیں  نیوٹن کا دوسرا قاعدہ  زاویائی حرکت کے لئے بھی لکھا جا سکتا ہے، تاہم  اب قوت  کی بجائے ایک نئی  مقدار جو \ترچھا{  قوت مروڑ } کہلاتی ہے استعمال  کرنا ہو گا۔  کام اور  کام و حرکی توانائی  مسئلے کا اطلاق   بھی گھماو  حرکت  پر کیا جا سکتا ہے، تاہم  کمیت کی بجائے ایک نئی مقدار جو \ترچھا{زاویائی جمود} کہلاتی ہے استعمال کرنا ہو  گا۔ مختصراً،  ہم جو کچھ پڑھ چکے ہیں، اس کا اطلاق گھماو حرکت میں ہو گا، تاہم کبھی کبھار معمولی تبدیلی  کی ضرورت پیش آئے گی۔

\موٹا{انتباہ:}
اگرچہ اس باب میں زیادہ تر حقائق محض  دوبارہ پیش کیے گئے ہیں، دیکھا یہ گیا ہے کہ طلبہ و طالبات کو اس باب میں دشواری پیش آتی ہے۔ اساتذہ کرام اس کی کئی وجوہات پیش کرتے ہیں جن میں سے دو  پر اتفاق پایا جاتا ہے: \عددی{1} یہاں  علامت    کی تعداد بہت زیادہ ہے (جنہیں  یونانی حروف  میں لکھ کر  مشکل میں  مزید اضافہ پیدا ہوتا ہے)، اور \عددی{2}  آپ خطی حرکت سے زیادہ واقف ہیں (اسی لئے  کمرے کے ایک کونے سے دوسرے کونے تک آپ  با آسانی جا سکتے ہیں)،  لیکن گھماو سے آپ کا واسطہ کم رہا ہے (اسی لئے تفریح  گاہ میں آپ  تفریحی جھولے پر سوار ہونے کے لئے پیسہ خرچنے کے لئے راضی ہوتے ہیں)۔ جہاں آپ کو دشواری ہو، دیکھیں آیا مسئلے کو  باب \حوالہء{2} کا یک بُعدی خطی مسئلہ   تصور کرنے  آسانی پیدا ہوتی ہے۔ مثلاً، اگر آپ سے\ترچھا{ زاویائی } فاصلہ معلوم کرنے کو کہا جائے، وقتی طور پر  لفظ \ترچھا{زاویائی} کو بھول جائیں اور دیکھیں آیا باب \حوالہء{2}  کی ترقیم اور تصورات استعمال کر کے جواب حاصل کرنا آسان ہوتا ہے۔

\جزوحصہء{گھماو کے  متغیر}
ہم مقررہ محور  پر استوار  جسم کے گھماو  پر غور کرنا چاہتے ہیں۔\اصطلاح{ استوار  جسم }\فرہنگ{استوار جسم!تعریف}\حاشیہب{rigid body}\فرہنگ{rigid body!defined} سے مراد  وہ جسم ہے جس  کے تمام  حصے  ، جسم کی شکل و صورت تبدیل کیے بغیر، ہم قدم  گھوم سکتے ہیں۔ \اصطلاح{مقررہ محور }\فرہنگ{مقررہ محور!تعریف}\حاشیہب{fixed axis}\فرہنگ{fixed axis!defined} سے مراد وہ محور ہے جو حرکت نہیں کرتی اور   جس  پر گھوما جا سکتا ہے۔یوں ہم ایسے جسم پر غور نہیں کریں گے جیسا  سورج   (جو گیس  کا کرہ  ہے) جس کے  حصے ایک ساتھ حرکت نہیں کرتے۔ ہم زمین پر  لڑھکتے گیند کی بھی بات نہیں کرتے چونکہ اس کا محور خود حرکت پذیر ہے (ایسی گیند کی حرکت،   گھماو اور  مستقیم حرکت کا ملاپ ہے )۔

شکل \حوالہء{10.2} میں  مقررہ محور پر ، جو\اصطلاح{ محور گھماو}\فرہنگ{ محور گھماو!تعریف}\حاشیہب{rotation axis}\فرہنگ{rotation axis!defined}یا \اصطلاح{گھماو کی محور } کہلاتی ہے، اختیاری شکل کا استوار  جسم  گھوم رہا ہے۔ خالص  گھماو  (\ترچھا{زاویائی حرکت}) میں ،  جسم کا ہر نقطہ ایسے  دائرہ  پر حرکت کرتا ہے، جس کا مرکز  محور  گھماو پر واقع ہے، اور  ہر نقطہ کسی مخصوص وقتی  وقفہ  میں ایک جتنا زاویہ طے کرتا  ہے۔ خالص مستقیم حرکت (خطی حرکت)  میں، جسم کا ہر نقطہ کسی مخصوص وقتی دورانیہ میں  ایک جتنا  \ترچھا{خطی فاصلہ } طے کرتا ہے۔

آئیں باری باری خطی مقادیر  مقام، ہٹاو، سمتی رفتار، اور اسراع کے مماثل زاویائی  مقادیر  پر  غور کرتے ہیں۔

\جزوحصہء{زاویائی مقام}
شکل \حوالہء{10.2} میں گھماو کو عمودی، جسم کے ساتھ  گھومتی، جسم  سے پکی  جڑی   \ترچھا{ حوالہ لکیر } دکھائی گئی ہے  ۔ کسی مقررہ رخ کے ساتھ ، جس کو ہم \اصطلاح{ صفر زاویائی مقام }\فرہنگ{زاویائی مقام!صفر}\حاشیہب{zero angular position}\فرہنگ{angular position!zero} مانتے ہیں، اس لکیر کا زاویہ لکیر کا \اصطلاح{ زاویائی مقام }\فرہنگ{زاویائی مقام!تعریف}\حاشیہب{angular position}\فرہنگ{angular position!defined}  ہو گا۔ شکل \حوالہء{10.3} میں  محور \عددی{x} کے مثبت رخ کے ساتھ زاویائی مقام  \عددی{\theta} ناپا گیا ہے۔ ہندسہ سے ہم جانتے ہیں درج ذیل ہو گا۔
\begin{align}\label{مساوات_گھماو_رداسی_فاصلہ_الف}
\theta=\frac{s}{r}\quad\quad \text{\RL{(ریڈیئن ناپ)}}
\end{align}
یہاں محور \عددی{x}  (جو صفر زاویائی مقام ہے) سے حوالہ  لکیر  تک دائری قوس کی لمبائی \عددی{s}، اور دائرے کا رداس \عددی{r} ہے۔

اس طرح تعین کیا گیا زاویہ  ، درجہ یا چکر کی بجائے ، \اصطلاح{ریڈیئن }\فرہنگ{ریڈیئن}\حاشیہب{radian}\فرہنگ{radian} میں ناپا جاتا ہے۔ ریڈیئن دو لمبائیوں  کی نسبت   (تقابلی تعلق)ہے  لہٰذا یہ  بے بُعد خالص عدد ہو گا۔ دائرے  کا محیط \عددی{2\pi r} ہے لہٰذا ایک مکمل دائرے میں \عددی{2\pi} ریڈیئن ہوں گے۔
\begin{align}
\text{\RL{چکر}}\, 1=\SI{360}{\degree}=\frac{2\pi r}{r}=\text{\RL{ریڈیئن}}\, 2\pi
\end{align}
یا
\begin{align}
\text{\RL{ریڈیئن}}\,1=\SI{57.3}{\degree}=\text{\RL{چکر}}\,0.159
\end{align}
 محور گھماو پر حوالہ لکیر کی  مکمل  چکر کے بعد ہم \عددی{\theta} واپس  صفر\ترچھا{ نہیں } کرتے۔اگر حوالہ لکیر صفر زاویائی مقام سے  ابتدا کر کے دو چکر  مکمل  کرے، لکیر کا زاویائی مقام \عددی{\theta=4\pi} ریڈیئن ہو گا۔
 
محور \عددی{x} پر  خالص مستقیم حرکت کے لئے  \عددی{x(t)} ، یعنی مقام بالمقابل وقت،  جانتے ہوئے ہم حرکت پذیر جسم کے بارے میں وہ سب کچھ معلوم کر سکتے ہیں جنہیں جاننا مقصود ہو۔ اسی طرح، خالص گھماو  کے لئے \عددی{\theta(t)}، یعنی زاویائی مقام بالمقابل وقت، جانتے ہوئے ہم گھومتے  جسم  کے بارے میں  وہ سب کچھ معلوم کر سکتے ہیں جنہیں جاننا مقصود ہو۔

\جزوحصہء{زاویائی ہٹاو}
اگر شکل \حوالہء{10.3}  کا جسم  محور گھماو پر شکل \حوالہء{10.4}  کی طرح  گھوم کر حوالہ لکیر کا زاویائی مقام \عددی{\theta_1} سے  تبدیل کر کے \عددی{\theta_2}  کرے، جسم کا زاویائی ہٹاو  \عددی{\Delta \theta} ذیل ہو گا۔
\begin{align}
\Delta \theta=\theta_2-\theta_1
\end{align}
زاویائی ہٹاو کی یہ تعریف نہ صرف استوار جسم بلکہ جسم کے ہر    اندرونی ذرہ کے لئے درست ہے۔

\موٹا{گھڑیاں منفی ہیں۔}
محور \عددی{x} پر  مستقیم حرکت کی صورت میں جسم کا ہٹاو \عددی{\Delta x}  مثبت یا منفی ہو گا، جو  ،محور پر جسم کی حرکت کے رخ پر منحصر ہے۔ اسی طرح، گھماو کی صورت میں جسم کا  زاویائی ہٹاو \عددی{\Delta \theta} درج ذیل قاعدہ کے تحت  مثبت یا منفی ہو گا۔

\ابتدا{قاعدہ}
خلاف گھڑی زاویائی ہٹاو مثبت اور گھڑی وار ہٹاو منفی ہو گا۔
\انتہا{قاعدہ}

\قول{گھڑیاں  منفی ہیں} کا فقرہ اس قاعدے کو یاد رکھنے  میں مدد دے سکتا ہے۔یاد رہے  گھڑی  کے سیکنڈ   کی سوئی کا ہر قدم آپ کی زندگی کاٹتی ہے۔

\ابتدا{آزمائش}
قرص اپنے وسطی محور کے گرد گھوم سکتا ہے۔ درج ذیل  ابتدائی  اور اختتامی زاویائی مقام کی  مرتب جوڑیوں میں کونسی  منفی زاویائی ہٹاو دیتی ہیں؟ (ا)  ابتدائی \عددی{-3}  ریڈیئن، اختتامی \عددی{+5} ریڈیئن؛ 
(ب)   ابتدائی \عددی{-3}  ریڈیئن، اختتامی \عددی{-7} ریڈیئن؛  (ج)   ابتدائی \عددی{7}  ریڈیئن، اختتامی \عددی{-3} ریڈیئن۔
\انتہا{آزمائش}

\جزوحصہء{زاویائی سمتی رفتار}
فرض کریں ایک جسم وقت \عددی{t_1} پر زاویائی مقام \عددی{\theta_1} پر اور  وقت \عددی{t_2} پر زاویائی مقام \عددی{\theta_2} پر  ہو، جیسا شکل \حوالہء{10.4} میں دکھایا گیا ہے۔  ہم \عددی{t_1} تا \عددی{t_2} وقتی دورانیہ \عددی{\Delta t} میں جسم کی \اصطلاح{ اوسط زاویائی سمتی رفتار }\فرہنگ{زاویائی سمتی رفتار!اوسط، تعریف}\حاشیہب{average angular velocity}\فرہنگ{angular velocity!average, defined}  \عددی{\omega_{\text{\RL{اوسط}}}} کی تعریف ذیل کرتے ہیں،
\begin{align}\label{مساوات_گھماو_اوسط_زاویائی_سمتی_رفتار}
\omega_{\text{\RL{اوسط}}}=\frac{\theta_2-\theta_1}{t_2-t_1}=\frac{\Delta \theta}{\Delta t}
\end{align}
جہاں وقت دورانیہ \عددی{\Delta t} میں زاویائی ہٹاو \عددی{\Delta \omega} ہے۔ (زاویائی سمتی رفتار کے لئے یونانی  حروف  تہجی کا ، چھوٹی لکھائی میں  ،  آخری حرف  \موٹا{اومیگا } \عددی{\omega}  استعمال کیا جائے گا۔)
%----------------------------------------------------------------
%p261
مساوات \حوالہ{مساوات_گھماو_اوسط_زاویائی_سمتی_رفتار}  میں \عددی{\Delta t} صفر کے قریب تر کرنے سے  نسبت کی درج ذیل  تحدیدی  قیمت  حاصل ہو گی  جو \اصطلاح{   لمحاتی زاویائی سمتی رفتار}\فرہنگ{زاویائی سمتی رفتار، لمحاتی،تعریف}\حاشیہب{instantaneous angular velocity}\فرہنگ{angular velocity!instantaneous, defined} \عددی{\omega} (یا      مختصراً \اصطلاح{ زاویائی سمتی رفتار } ) کہلاتی ہے۔
\begin{align}\label{مساوات_گھماو_لمحاتی_زاویائی_سمتی_رفتار}
\omega=\lim_{\Delta t\to 0}\frac{\Delta \theta}{\Delta t}=\frac{\dif \theta}{\dif t}
\end{align}
اگر \عددی{\theta(t)}  معلوم ہو، اس کا تفرق لے کر   زاویائی سمتی رفتار \عددی{\omega} حاصل   ہو گی۔

چونکہ اس جسم کے تمام ذرے ہم قدم ہیں، لہٰذا مساوات \حوالہ{مساوات_گھماو_اوسط_زاویائی_سمتی_رفتار} اور مساوات \حوالہ{مساوات_گھماو_لمحاتی_زاویائی_سمتی_رفتار} نا صرف مکمل  گھومتے  استوار جسم  کے لئے بلکہ  \ترچھا{  جسم کے ہر  ذرے }کے لئے درست ہیں۔ زاویائی سمتی رفتار کی  عمومی مستعمل اکائی ریڈیئن فی سیکنڈ \عددی{(\si{\radian\per\second})}، چکر فی سیکنڈ   ، اور چکر فی منٹ ہے۔

محور \عددی{x} پر مثبت رخ حرکت کرتے ہوئے  ذرے کی سمتی رفتار \عددی{v} مثبت  جبکہ منفی رخ حرکت کی صورت میں منفی ہو گی۔ اسی طرح محور پر مثبت رخ (خلاف گھڑی) گھماو کی صورت میں استوار جسم کی زاویائی سمتی  رفتار مثبت  جبکہ منفی رخ  (گھڑی وار) گھماو کی صورت میں منفی ہو گی۔ (\قول{گھڑیاں منفی ہیں } اب بھی درست ہے۔) زاویائی سمتی رفتار کی قدر\اصطلاح{ زاویائی رفتار }\فرہنگ{زاویائی رفتار!تعریف}\حاشیہب{angular speed}\فرہنگ{angular speed!defined}کہلاتی ہے۔ہم  زاویائی رفتار کے لئے  بھی \عددی{\omega} علامت استعمال کریں گے۔

\جزوحصہء{زاویائی اسراع}
گھومتے ہوئے جسم کی زاویائی سمتی رفتار  مستقل نہ ہونے کی صورت میں جسم زاویائی اسراع سے دو چار ہو گا۔فرض کریں وقت \عددی{t_1} پر جسم کی زاویائی سمتی رفتار \عددی{\omega_1} اور \عددی{t_2} پر \عددی{\omega_2} ہے۔ دورانیہ \عددی{t_1} تا \عددی{t_2}   میں گھومتے ہوئے جسم کی \اصطلاح{ اوسط زاویائی اسراع }\فرہنگ{زاویائی اسراع!اوسط،تعریف}\حاشیہب{average angular acceleration}\فرہنگ{angular acceleration, average, defined}\عددی{\alpha_{\text{\RL{اوسط}}}}   کی تعریف  ذیل ہے،
\begin{align}\label{مساوات_گھماو_زاویائی_اوسط_اسراع}
\alpha_{\text{\RL{اوسط}}}=\frac{\omega_2-\omega_1}{t_2-t_1}=\frac{\Delta \omega}{\Delta t}
\end{align}
جہاں ی \عددی{\Delta \omega}  زاویائی سمتی رفتار  میں  \عددی{\Delta t}   کے دوران  تبدیل ہے۔\اصطلاح{   لمحاتی زاویائی اسراع }\فرہنگ{زاویائی اسراع!تعریف}\حاشیہب{instantaneous angular acceleration}\فرہنگ{angular acceleration!instantaneous, defined}(یا مختصر \اصطلاح{اً زاویائی اسراع})، جس سے ہمیں زیادہ دلچسپی ہے، \عددی{\Delta t} صفر کے قریب تر کرنے سے نسبت کی، درج ذیل،  تحدیدی قیمت کو کہتے ہیں۔
\begin{align}\label{مساوات_گھماو_زاویائی_لمحاتی_اسراع}
\alpha=\lim_{\Delta t\to 0}\frac{\Delta \omega}{\Delta t}=\frac{\dif \omega}{\dif t}
\end{align}
مساوات \حوالہ{مساوات_گھماو_زاویائی_اوسط_اسراع} اور مساوات \حوالہ{مساوات_گھماو_زاویائی_لمحاتی_اسراع}\ترچھا{  جسم کے ہر ذرے} کے لئے درست ہیں۔ زاویائی اسراع کی عمومی مستعمل اکائی ریڈیئن فی مربع  سیکنڈ \عددی{(\si{\radian\per\second\squared})} اور  چکر فی مربع سیکنڈ ہے۔

%---------------------------------------
%Sample Problem 10.01  p262
\ابتدا{نمونی سوال}\موٹا{زاویائی مقام سے زاویائی سمتی رفتار کا حصول}\\
شکل \حوالہء{10.5a} میں قرص اپنے  وسطی محور کے گرد گھوم رہا ہے۔ قرص پر حوالہ لکیر کا زاویائی مقام \عددی{\theta(t)} ذیل ہے، جہاں \عددی{t} اور \عددی{\theta} بالترتیب سیکنڈ اور ریڈیئن میں ہیں، اور صفر زاویائی مقام شکل  میں  دکھایا گیا ہے۔
\begin{align}\label{مساوات_گھماو_نمونی_قرص}
\theta=-1.00-0.600t+0.250t^2
\end{align}
(آپ چاہیں تو وقتی طور پر لفظ \قول{زاویائی مقام}  سے \قول{زاویائی} خارج کر کے اور \عددی{\theta} علامت کی جگہ \عددی{x} استعمال کر کے  مسئلے کو باب \حوالہء{2}   کی ترقیم  میں لے جائیں۔ آپ کو باب \حوالہء{2} کی یک بُعدی  حرکت کے مقام کی مساوات   حاصل ہو گی۔)

(ا)قرص کا زاویائی مقام بالمقابل وقت  \عددی{t=\SI{-3.0}{\second}} تا \عددی{t=\SI{5.4}{\second}}   ترسیم کریں۔ قرص اور اس پر زاویائی   مقام کی حوالہ لکیر   کا خاکہ \عددی{t=\SI{-2.0}{\second}}،،  اور \عددی{t=\SI{4.0}{\second}}   ،  اور اس لمحے پر بنائیں جب ترسیم \عددی{t} محور سے گزرتی ہے۔

\جزوحصہ{کلیدی تصور}
قرص کے زاویائی مقام سے مراد اس پر کھینچی حوالہ لکیر کا مقام \عددی{\theta(t)}  ہے، جو مساوات  \حوالہ{مساوات_گھماو_نمونی_قرص} دیتی ہے؛ لہٰذا ہم مساوات \حوالہ{مساوات_گھماو_نمونی_قرص} ترسیم کرتے ہیں؛ نتیجہ شکل \حوالہء{10.5b} میں پیش ہے۔

\موٹا{حساب:}\quad
قرص اور حوالہ لکیر کا مقام کسی مخصوص لمحے پر  خاکہ بنانے کے لئے ضروری ہے کہ اس لمحے پر ہمیں \عددی{\theta} معلوم ہو، جو مساوات \حوالہ{مساوات_گھماو_نمونی_قرص} میں لمحے کا وقت ڈالنے سے حاصل ہو گا۔ یوں \عددی{t=\SI{-2.0}{\second}} کے لئے ذیل ہو گا۔
\begin{align*}
\theta&=-1.00-(0.600)(-2.0)+(0.250)(-2.0)^2\\
&=\SI{1.2}{\radian}=\SI{1.2}{\radian}\,\frac{\SI{360}{\degree}}{\text{\RL{ریڈیئن}}2\pi}=\SI{69}{\degree}
\end{align*}
یہ نتیجہ کہتا ہے کہ   قرص پر موجود  حوالہ لکیر لمحہ    \عددی{t=\SI{-2.0}{\second}} پر  صفر مقام سے مثبت رخ (خلاف گھڑی) \عددی{1.2} ریڈیئن یعنی \عددی{\SI{69}{\degree}}  گھوم کر ہو گی۔ شکل \حوالہء{10.5b}   کے خاکہ \عددی{1}  میں حوالہ لکیر کا یہ مقام  دکھایا گیا ہے۔

اسی طرح \عددی{t=0} پر \عددی{\theta} کی قیمت \عددی{-1.00} ریڈیئن یا \عددی{\SI{-57}{\degree}} ہو گی، جس کے تحت حوالہ لکیر  صفر زاویائی مقام سے  \عددی{1.0} ریڈیئن یا \عددی{\SI{57}{\degree}} منفی رخ (گھڑی وار)  گھوم کر ہو گی، جیسا  خاکہ \عددی{3} میں دکھایا گیا ہے۔ لمحہ \عددی{t=\SI{4.0}{\second}} پر \عددی{\theta} کی قیمت \عددی{0.60}  ریڈیئن یعنی \عددی{\SI{34}{\degree}}  ہو گی (خاکہ \عددی{5})۔ جس لمحے ترسیم محور \عددی{t} سے گزرتی ہے، \عددی{\theta=0} ہو گا اور حوالہ لکیر لمحاتی عین صفر مقام پر ہو گی (خاکہ \عددی{2} اور \عددی{4})۔

(ب) شکل \حوالہء{10.5b} میں \عددی{\theta(t)} کی کم سے کم قیمت   کس  \عددی{t_{\text{\RL{کمتر}}}}   پر ہو گی؟  \عددی{\theta} کی  کم سے کم قیمت کیا ہے؟

\جزوحصہء{کلیدی تصور}
تفاعل  کی انتہا قیمت (یہاں کم سے کم قیمت)  معلوم کرنے کی خاطر  ہم تفاعل کا ایک گنّا  تفرق  لے کر صفر کے برابر رکھتے ہیں۔

\موٹا{حساب:}\quad
تفاعل \عددی{\theta(t)} کا ایک گنّا تفرق ذیل ہے۔
\begin{align}\label{مساوات_گھماو_نمونی_رفتار}
\frac{\dif \theta}{\dif t}=-0.600+0.500t
\end{align}
اس کو صفر کے برابر رکھ کر \عددی{t} کے لئے حل  کر   کے  لمحہ \عددی{t_{\text{\RL{کمتر}}}} حاصل ہو گا جس پر \عددی{\theta(t)} کی قیمت کم سے کم ہو گی۔
\begin{align*}
t_{\text{\RL{کمتر}}}&=\SI{1.2}{\second}\quad \quad \text{\RL{(جواب)}}
\end{align*}
\عددی{\theta(t)} کی کم سے کم قیمت جاننے کے لئے ہم مساوات \حوالہ{مساوات_گھماو_نمونی_قرص} میں \عددی{t_{\text{\RL{کمتر}}}} ڈالتے ہیں، جو ذیل دیگا۔
\begin{align*}
\theta&=\text{\RL{ریڈیئن}}\, -.136\approx \SI{-77.9}{\degree} \quad \quad \text{\RL{(جواب)}}
\end{align*}
\عددی{\theta(t)} کی\ترچھا{ کم سے کم } قیمت  (شکل \حوالہء{10.5b} میں نشیب)  صفر زاویائی مقام سے قرص کی\ترچھا{ زیادہ سے زیادہ  گھڑی وار} گھماو  ہے، جو خاکہ \عددی{3} سے کچھ زیادہ ہو گا۔

(ج)قرص کی زاویائی سمتی رفتار \عددی{\omega}  وقت \عددی{t=\SI{-3.0}{\second}}  تا \عددی{t=\SI{6.0}{\second}} ترسیم کریں۔قرص کا خاکہ \عددی{t=\SI{-2.0}{\second}}، \عددی{t=\SI{4.0}{\second}}، اور \عددی{t_{\text{\RL{کمتر}}}} پر بنائیں ، اور بتائیں ان لمحات پر گھومنے کا رخ اور  \عددی{\omega}  کی علامت  کیا ہو گی۔

\جزوحصہء{کلیدی تصور}
مساوات \حوالہ{مساوات_گھماو_لمحاتی_زاویائی_سمتی_رفتار} کے تحت زاویائی سمتی رفتار \عددی{\omega} سے مراد \عددی{\dif\theta\!/\!\dif t} ہے جو مساوات \حوالہ{مساوات_گھماو_نمونی_رفتار} دیتی ہے۔ یوں ذیل ہو گا۔
\begin{align}\label{مساوات_گھماو_رفتار_الف}
\omega=-0.600+0.500t
\end{align}
اس تفاعل ، \عددی{\omega(t)}،  کی ترسیم شکل \حوالہء{10.5c} میں پیش ہے۔ یہ تفاعل خطی ہے لہٰذا اس کی ترسیم ایک سیدھی لکیر ہے۔ ترسیم کی  ڈھلوان  \عددی{\SI{0.500}{\radian\per\second\squared}}ہے  اور  انتصابی محور  (جو دکھایا نہیں گیا)  کو  ترسیم \عددی{\SI{-0.600}{\radian\per\second}} پر قطع کرتی ہے۔

\موٹا{حساب:}\quad
قرص کا خاکہ \عددی{t=\SI{-2.0}{\second}} پر بنانے کی خاطر ہم  مساوات \حوالہ{مساوات_گھماو_رفتار_الف} میں یہ قیمت ڈال کر ذیل حاصل کرتے ہیں۔
\begin{align*}
\omega=\SI{-1.6}{\radian\per\second}\quad\quad \text{\RL{(جواب)}}
\end{align*}
منفی کی علامت کہتی ہے کہ \عددی{t=\SI{-2.0}{\second}} پر قرص گھڑی وار (منفی رخ) گھوم رہا ہے (جیسا شکل \حوالہء{10.5c} میں دائیں  ہاتھ خاکے میں  دکھایا گیا ہے)۔

مساوات \حوالہ{مساوات_گھماو_رفتار_الف} میں \عددی{t=\SI{4.0}{\second}} ڈال کر ذیل حاصل ہو گا۔
\begin{align*}
\omega=\SI{1.4}{\radian\per\second}\quad\quad \text{\RL{(جواب)}}
\end{align*}
مضمر مثبت علامت کہتی ہے قرص مثبت رخ (خلاف گھڑی) گھوم رہا ہے (شکل \حوالہء{10.5c} میں دایاں ہاتھ خاکہ)۔

\عددی{t_{\text{\RL{کمتر}}}}  کے لئے ہم جانتے ہیں \عددی{\dif\theta\!/\!\dif t=0} ہو گا۔یوں \عددی{\omega=0} ہو گا۔جب حوالہ لکیر  ، شکل \حوالہء{10.5b}\عددی{\theta} میں \عددی{\theta}  کی کم سے کم قیمت کو پہنچتی ہے   ، قرص لمحاتی رکتا ہے، جیسا شکل \حوالہء{10.5c} میں وسطی خاکہ عندیہ دیتا ہے۔  شکل \حوالہء{10.5c} میں \عددی{\omega} بالمقابل \عددی{t} کی ترسیم  پر صفر نقطہ  ، جہاں  ترسیم منفی ( گھڑی وار ) گھماو سے مثبت ( خلاف گھڑی)  گھماو کا آغاز کرتی ہے، وہ نقطہ ہے جہاں قرص لمحاتی رکتا ہے۔

(د) جزو ا تا جزو ج کے نتائج استعمال کر کے \عددی{t=\SI{-3.0}{\second}} تا \عددی{t=\SI{6.0}{\second}}   قرص کی حرکت  بیان کریں۔

\موٹا{بیان:}\quad
جب ہم،  \عددی{t=\SI{-3.0}{\second}} پر   ، قرص پر پہلی مرتبہ نظر  ڈالتے ہیں، اس کا زاویائی مقام  مثبت  ،  گھماو گھڑی وار  اور رفتار میں کمی دیکھنے کو ملتی ہے۔ یہ \عددی{\theta=-1.36}  ریڈیئن  پر لمحاتی رکنے کے بعد  خلاف گھڑی  گھومنا شروع کرتا ہے اور آخر کار  اس کا زاویائی مقام دوبارہ  مثبت ہوتا ہے۔
\انتہا{نمونی سوال}
%---------------------------
%Sample Problem 10.02 p264
\ابتدا{نمونی سوال}\موٹا{زاویائی اسراع سے زاویائی سمتی رفتار کا حصول}\\
ایک بچہ لٹو  ذیل زاویائی اسراع سے گھماتا ہے، جہاں \عددی{t} اور \عددی{\alpha} بالترتیب سیکنڈ اور ریڈیئن فی مربع  سیکنڈ میں ہے۔
\begin{align*}
\alpha=5t^3-4t
\end{align*}
لمحہ \عددی{t=0} پر لٹو کی زاویائی سمتی رفتار \عددی{\SI{5}{\radian\per\second}} ، اور   حوالہ لکیر کا زاویائی مقام \عددی{\theta=2}  ریڈیئن ہے۔

(ا) لٹو کی زاویائی سمتی رفتار  \عددی{\omega(t)}  کا ریاضی فقرہ  حاصل کریں؛ یعنی ایسا تفاعل معلوم کریں جو وقت پر  زاویائی سمتی رفتار  کا انحصار صریحاً  دے۔ (ہم جانتے ہیں ایسا تفاعل موجود ہے چونکہ لٹو زاویائی اسراع  سے گزر رہا ہے؛ یوں اس کی زاویائی سمتی رفتار تبدیل ہو گی۔ )

\جزوحصہء{کلیدی تصور}
\عددی{\alpha(t)}تعریف کے  رو  سے  \عددی{\omega(t)}  کا وقتی تفرق ہو گا۔یوں، وقت کے لحاظ سے  \عددی{\alpha(t)} کا تکمل \عددی{\omega(t)} دیگا۔

\موٹا{حساب:}\quad
مساوات \حوالہ{مساوات_گھماو_زاویائی_لمحاتی_اسراع}  ذیل کہتی ہے
\begin{align*}
\dif \omega=\alpha \dif t
\end{align*}
لہٰذا 
\begin{align*}
\int \dif \omega=\int \alpha \dif t
\end{align*}
ہو گا جو ذیل کے گی، جہاں \عددی{C} تکمل کا مستقل ہے۔
\begin{align*}
\omega=\int (5t^3-4t)\dif t=\frac{5}{4}t^4-\frac{4}{2}t^2+C
\end{align*}
ہم جانتے ہیں \عددی{t=0} پر \عددی{\omega=\SI{5}{\radian\per\second}} ہے؛ اس معلومات کو در ج بالا میں ڈال کر:
\begin{align*}
\SI{5}{\radian\per\second}=0-0+C
\end{align*}
تکمل کا مستقل   \عددی{C=\SI{5}{\radian\per\second}}    حاصل ہو گا۔یوں  درکار  تفاعل ذیل ہو گا۔
\begin{align*}
\omega=\frac{5}{4}t^4-\frac{4}{2}t^2+5\quad\quad \text{\RL{(جواب)}}
\end{align*}
(ب) لٹو کے زاویائی مقام \عددی{\theta(t)} کا ریاضی فقرہ تلاش کریں۔

\جزوحصہء{کلیدی تصور}
تعریف کے رو سے \عددی{\theta(t)} کا وقتی تفرق \عددی{\omega(t)} دیگا۔ یوں، وقت کے لحاظ سے \عددی{\omega(t)} کا تکمل \عددی{\theta(t)} دیگا۔

\موٹا{حساب:}\quad
مساوات \حوالہ{مساوات_گھماو_لمحاتی_زاویائی_سمتی_رفتار} کے تحت:
\begin{align*}
\dif \theta=\omega \dif t
\end{align*}
ہو گا جس سے ذیل لکھا جا سکتا ہے،
\begin{align*}
\theta&=\int \omega \dif t=\int (\frac{5}{4}t^4-\frac{4}{2}t^2+5)\dif t\\
&=\frac{1}{4}t^5-\frac{2}{3}t^3+5t+C'\\
&=\frac{1}{4}t^5-\frac{2}{3}t^3+5t+2\quad\quad\text{\RL{(جواب)}}
\end{align*}
جہاں \عددی{t=0}  پر \عددی{\theta=\SI{2}{\radian}} جانتے ہوئے \عددی{C'} کی قیمت  حاصل کی گئی۔
\انتہا{نمونی سوال}
%------------------------------
%p264
\حصہء{کیا زاویائی مقادیر سمتیات   ہیں؟}
ہم   اکیلے  ذرے  کا مقام، سمتی رفتار، اور اسراع سمتیات سے بیان کر سکتے ہیں۔ اگر ذرہ  صرف ایک  محور پر حرکت کرتا ہو،  سمتی ترقیم استعمال کرنا ضرورت نہیں۔ایسے ذرے کو صرف دو رخ  دستیاب ہیں جنہیں مثبت اور منفی علامت سے ظاہر کیا جا سکتا ہے۔

اسی طرح استوار جسم  قائمہ محور  پر  ، محور کے ہمراہ  دیکھتے ہوئے، صرف خلاف گھڑی اور گھڑی وار   گھوم سکتا ہے۔ان رخ کو ہم مثبت اور منفی سے ظاہر کر سکتے ہیں۔ یہاں ایک سوال اٹھتا ہے: \قول{کیا ہم گھومتے جسم کے زاویائی ہٹاو، زاویائی سمتی رفتار، اور زاویائی اسراع کو سمتیات  سمجھ  سکتے ہیں؟} اس کا جواب ہے \قول{جی ہاں} (زاویائی ہٹاو کے  لئے   نیچے  پیش انتباہ  ضرور دیکھیں۔)

\موٹا{زاویائی سمتی رفتار۔} زاویائی سمتی رفتار کو دیکھیں۔ شکل \حوالہء{10.6a} میں \عددی{\omega=33\tfrac{1}{3}}   چکر فی سیکنڈ  کی   مستقل زاویائی رفتار   سے گھڑی وار  رخ  گھومتا ہوا قرص دکھایا گیا ہے۔ جیسا شکل \حوالہء{10.6b} میں دکھایا گیا ہے، ہم اس کی سمتی زاویائی  رفتار  گھماو کے محور پر سمتیہ   \عددی{\vec{\omega}} سے ظاہر کر سکتے ہیں۔ اس کا طریقہ کار یوں ہے: سمتیہ کی لمبائی   کسی موزوں پیمانہ کے تحت   رکھی جاتی ہے، مثلاً   \عددی{\SI{1}{\centi\meter}} کو \عددی{10} چکر فی منٹ  کی مطابقت سے رکھ  جا سکتا ہے۔ اس کے بعد \عددی{\vec{\omega}} کا رخ تعین کرنے کے لئے ہم \اصطلاح{ دائیں ہاتھ کا قاعدہ }استعمال کرتے ہیں، جو شکل \حوالہء{10.6c} میں پیش ہے: قرص کو دائیں ہاتھ میں یوں پکڑیں کہ  انگلیاں    \ترچھا{ گھماو کے رخ}  ہوں۔ آپ کا سیدھا کھڑا انگوٹھا  زاویائی سمتی رفتار کے سمتیہ کا رخ دیگا۔ اگر قرص مخالف رخ گھومے، دائیں ہاتھ قاعدہ کے تحت \عددی{\vec{\omega}}  بھی  مخالف رخ ہو گا۔

زاویائی مقادیر       سمتیات سے ظاہر کرنے کی عادت مشکل سے  ڈلتی ہے۔ ہم فوراً سوچتے ہیں  کہ سمتیہ کے  ہمراہ  کوئی چیز  حرکت  کرے گی۔یہاں  ایسا نہیں ہو گا۔  اس کے بجائے کوئی چیز (جیسا استوار جسم) سمتیہ کے رخ کے  \ترچھا{گرد  } گھومتی ہے۔ خالص گھماو کی دنیا میں،  سمتیہ کا رخ کسی چیز کی حرکت کا رخ نہیں بلکہ  گھماو کا محور دیگا۔ بہرحال، سمتیہ حرکت بھی تعین کرتا ہے۔مزید، یہ   سمتیات  سلجھانے کے  ان تمام قواعد کی تعمیل کرتا ہے جو    باب \حوالہء{3} میں  پیش  کیے گئے۔ زاویائی اسراع \عددی{\vec{\alpha}}  بھی ایک  سمتیہ ہے، اور یہ بھی ان قواعد کی تعمیل کرتا ہے۔

اس باب میں صرف   قائمہ محور پر گھماو کی بات کی جائے گی۔ ان میں سمتیات استعمال کرنے کی ضرورت نہیں؛ ہم زاویائی سمتی رفتار کو \عددی{\omega} اور زاویائی اسراع کو \عددی{\alpha} سے ظاہر کر  کے، خلاف گھڑی گھماو کو مثبت اور گھڑی وار گھماو کو منفی  کی علامت سے ظاہر کر سکتے ہیں۔

\موٹا{زاویائی ہٹاو۔} پہلے  انتباہ کی بات کرتے ہیں: زاویائی ہٹاو (ماسوائے  انتہائی چھوٹا  ہٹاو) کو سمتیہ سے ظاہر نہیں کیا جا سکتا۔ کیوں نہیں؟ ہم یقیناً اس کے رخ اور قدر کی بات کر سکتے ہیں، جیسا شکل \حوالہء{10.6} میں زاویائی سمتی رفتار کے لئے کیا گیا۔ تاہم، سمتیہ سے ظاہر کیے جانے کے قابل ہونے کے لئے ضروری ہے کہ مقدار سمتی جمع کے قواعد پر پورا اترتی  ہو۔ ان قواعد میں ایک قاعدہ کہتا ہے کہ  سمتیات جمع کرتے وقت ان کی ترتیب غیر ضروری ہے۔ زاویائی ہٹاو اس قاعدہ پر پورا نہیں اترتا۔

شکل \حوالہء{10.7} میں  دی گئی مثال پر غور کریں۔ ایک کتاب  کو، جو ابتدائی طور پر افقی پڑی ہے، دو مرتبہ \عددی{\SI{90}{\degree}} زاویائی ہٹاو سے گزارا گیا ہے؛ ایک مرتبہ شکل \حوالہء{10.7a}   اور دوسری مرتبہ شکل \حوالہء{10.7b} کی طرح۔  دونوں  میں ہٹاو  برابر، لیکن  ترتیب ایک نہیں، اور  آخر میں کتاب  ایک جیسی سمت بند نہیں۔ دوسری مثال لیتے ہیں۔ دایاں ہاتھ لٹکا کر ہتھیلی  ران پر رکھیں۔ کلائی سخت  کر کے،  (1)  بازو   سامنے اتنا اٹھائیں   کہ افقی ہو، (2)  اس کو پورا  دائیں لے جائیں، اور (3) اس کے بعد ہاتھ واپس نیچے ران تک لے جائیں۔ آپ کی ہتھیلی اب سامنے رخ ہو گی۔ اگر آپ یہی عمل الٹ ترتیب سے دہرائیں، آپ کی ہتھیلی  آخر میں کس رخ ہو گی؟ ان مثال سے ہم دیکھتے ہیں کہ زاویائی  ہٹاو  کا مجموعہ انہیں جمع کرنے کی   ترتیب پر منحصر ہے، لہٰذا  ہٹاو کو سمتیہ تصور نہیں کیا جا سکتا۔

%-------------------------------------
%sec 10-2 Rotation With Constant Angular Acceleration  p266
\حصہ{مستقل اسراع کے ساتھ گھماو}
\موٹا{مقاصد}\\
اس حصہ کو پڑھنے کے بعد آپ ذیل کے قابل ہوں گے۔
\begin{enumerate}[1.]
\item
مستقل زاویائی اسراع کی صورت میں زاویائی مقام، زاویائی ہٹاو، زاویائی سمتی رفتار، زاویائی اسراع، اور  گزرے دارانیے کے   تعلق  (جدول \حوالہ{جدول_گھماو_مستقل_اسراع_مساوات}) استعمال کر پائیں گے۔
\end{enumerate}

\موٹا{کلیدی تصور}
\begin{itemize}
\item
مستقل زاویائی اسراع  (جس میں  \عددی{\alpha}  مستقل ہو گا) گھماو حرکت کی ایک اہم   خصوصی صورت ہے، جس کی مجرد حرکیات  مساوات  ذیل ہیں۔
\begin{align*}
\omega&=\omega_0+\alpha t\\
\theta-\theta_0&=\omega_0 t+\frac{1}{2}\alpha t^2\\
\omega^2&=\omega_0^2+2\alpha(\theta-\theta_0)\\
\theta-\theta_0&=\frac{1}{2}(\omega+\omega_0)t\\
\theta-\theta_0&=\omega t-\frac{1}{2}\alpha t^2
\end{align*}
\end{itemize}

\جزوحصہء{مستقل زاویائی اسراع کا گھماو}
مستقیم حرکت  میں \ترچھا{ مستقل خطی اسراع}  کی حرکت (مثلاً، زمین پر گرتا ہوا جسم) ایک اہم خصوصی صورت ہے۔ جدول \حوالہء{2.1} میں  اس طرح کی حرکت  کو مطمئن کرتی مساوات پیش کی گئیں۔

خالص گھماو میں \ترچھا{ مستقل زاویائی اسراع }  ایک اہم خصوصی صورت ہے؛ اس   کو مطمئن کرنے  والی مطابقتی  مساوات  پائی جاتی ہیں۔ ہم انہیں یہاں اخذ نہیں کریں گے، بلکہ مطابقتی خطی مساوات میں  مساوی زاویائی متغیرات ڈال کر انہیں پیش کرتے ہیں۔ جدول \حوالہ{جدول_گھماو_مستقل_اسراع_مساوات} میں مساوات کی دونوں فہرست (مساوات \حوالہء{2.11} اور مساوات \حوالہء{2.15} تا مساوات \حوالہء{2.18}؛ مساوات \حوالہ{مساوات_گھماو_زاویائی_الف} تا مساوات \حوالہ{مساوات_گھماو_زاویائی_ٹ}) پیش کی گئی ہیں۔

\begin{table}
\caption{مستقل خطی اسراع اور مستقل زاویائی اسراع کی حرکت کی مساوات}
\label{جدول_گھماو_مستقل_اسراع_مساوات}
\centering
\begin{minipage}{0.45\textwidth}
\begin{align}
&\text{\RL{خطی مساوات}}\notag\\
v&=v_0+at\tag{\setlatin{2.11}}\\
x-x_0&=v_0t+\frac{1}{2}at^2\tag{\setlatin{2.15}}\\
v^2&=v_0^2+2a(x-x_0)\tag{\setlatin{2.16}}\\
x-x_0&=\frac{1}{2}(v_0+v)t\tag{\setlatin{2.17}}\\
x-x_0&=vt-\frac{1}{2}at^2\tag{\setlatin{2.18}}
\end{align}
\end{minipage}\hfill
\begin{minipage}{0.45\textwidth}
\begin{align}
&\text{\RL{زاویائی مساوات}}\notag\\
\omega&=\omega_0+\alpha t\label{مساوات_گھماو_زاویائی_الف}\\
\theta-\theta_0&=\omega_0 t+\frac{1}{2}\alpha t^2 \label{مساوات_گھماو_زاویائی_ب}\\
\omega^2&=\omega_0^2+2\alpha(\theta-\theta_0) \label{مساوات_گھماو_زاویائی_پ}\\
\theta-\theta_0&=\frac{1}{2}(\omega_0+\omega)t  \label{مساوات_گھماو_زاویائی_ت}\\
\theta-\theta_0&=\omega t-\frac{1}{2}\alpha t^2  \label{مساوات_گھماو_زاویائی_ٹ}
\end{align}
\end{minipage}
\end{table}


یاد رہے مساوات \حوالہء{2.11} اور مساوات \حوالہء{2.15} مستقل خطی اسراع کی بنیادی مساوات ہیں، جن سے  فہرست کی باقی تمام مساوات اخذ کی جا سکتی ہیں۔ اس طرح، مساوات \حوالہ{مساوات_گھماو_زاویائی_الف} اور مساوات \حوالہ{مساوات_گھماو_زاویائی_ب} مستقل زاویائی اسراع کی بنیادی مساوات ہیں، جن سے زاویائی مساوات کی فہرست کی باقی تمام مساوات اخذ کی جا سکتی ہیں۔ مستقل زاویائی اسراع کا سادہ مسئلہ حل کرنے کے لئے آپ عموماً  زاویائی فہرست سے (اگر یہ فہرست آپ کے پاس موجود ہو)    ایک مساوات استعمال کر پائیں گے۔ آپ وہ مساوات منتخب کریں گے جس میں صرف وہ متغیر غیر معلوم ہو جو آپ کو  درکار ہو۔ بہتر طریقہ یہ ہو گا کہ آپ مساوات \حوالہ{مساوات_گھماو_زاویائی_الف} اور مساوات \حوالہ{مساوات_گھماو_زاویائی_ب} یاد کر لیں اور جب ضرورت پیش آئے، انہیں بطور ہمزاد مساوات حل کریں۔

%--------------------------
\ابتدا{آزمائش}
گھومے جسم کا زاویائی مقام \عددی{\theta(t)} چار مختلف صورتوں میں (ا)  \عددی{\theta=3t-4}، (ب) \عددی{\theta=-5t^3+4t^2+6}، (ج) \عددی{\theta=2\!/\!t^2-4\!/\!t}، اور (د) \عددی{\theta=5t^2-3} ہے۔  جدول \حوالہ{جدول_گھماو_مستقل_اسراع_مساوات} کی  زاویائی مساوات  کا اطلاق کن صورتوں پر ہو گا؟
\انتہا{آزمائش}
%--------------------------------------

%sample problem 10.03 p267
\ابتدا{نمونی سوال}\موٹا{مستقل زاویائی اسراع، چکی کا پاٹ}\\
شکل \حوالہء{10.8} میں  پاٹ  مستقل زاویائی اسراع \عددی{\alpha=\SI{0.34}{\radian\per\second\squared}} سے گھوم رہا ہے۔ وقت \عددی{t=0} پر اس کی  زاویائی سمتی رفتار  \عددی{\omega_0=\SI{-4.6}{\radian\per\second}} ہے، اور اس پر کھینچی گئی حوالہ لکیر کا مقام \عددی{\theta_0=0} ہے۔

(ا)  وقت \عددی{t=0} سے  کتنی دیر بعد حوالہ لکیر  زاویائی مقام \عددی{\theta=5.0}  چکر پر ہو گی؟

\موٹا{کلیدی تصور}\\
چونکہ زاویائی  اسراع   مستقل ہے لہٰذا ہم جدول \حوالہ{جدول_گھماو_مستقل_اسراع_مساوات} سے مساوات چن سکتے ہیں۔ ہم مساوات \حوالہ{مساوات_گھماو_زاویائی_ب}
\begin{align*}
\theta-\theta_0=\omega_0 t+\frac{1}{2}\alpha t^2
\end{align*}
 کا انتخاب اس لئے  کرتے ہیں کہ اس میں  صرف ایک متغیر، \عددی{t}،  نا معلوم ہے اور ہمیں   یہی درکار ہے۔
 
 \موٹا{حساب:}\quad
 دی گئی معلومات ڈال کر اور \عددی{\theta_0=0} اور \عددی{\theta=\text{\RL{چکر}}\,5.0=10\pi\,\si{\radian}}  لیتے ہوئے  ذیل  ہو گا۔
 \begin{align*}
 10\pi\,\si{\radian}=(\SI{-4.6}{\radian\per\second})t+\frac{1}{2}(\SI{0.35}{\radian\per\second\squared})t^2
 \end{align*}
 (اکائیوں  کے ثبات کی خاطر ہم \عددی{5.0} چکر کو \عددی{10\pi} ریڈیئن میں تبدیل کرتے ہیں۔)  اس  دو درجی الجبرائی مساوات کو حل کرنے سے ذیل حاصل ہو گا۔
 \begin{align*}
 t=\SI{32}{\second}\quad\quad \text{\RL{(جواب)}}
 \end{align*}
 ان ایک عجیب بات پر غور کریں۔ جب ہم پہلی مرتبہ پاٹ  پر نظر ڈالتے ہیں یہ منفی رخ گھوم کر \عددی{\theta=0} سمت بند مقام   سے گزرتا  ہے۔ اس کے باوجود \عددی{\SI{32}{\second}} بعد ہم اسے \عددی{\theta=5.0}   چکر     مثبت سمت بند   مقام پر پاتے ہیں۔ اس دورانیے میں ایسا کیا ہوا کہ پاٹ  مثبت سمت بند مقام پر ہو سکتا ہے؟
 
 (ب)وقت  \عددی{t=0} اور \عددی{t=\SI{32}{\second}} کے بیچ پاٹ کے گھماو پر تبصرہ کریں۔
 
 \موٹا{تبصرہ:}\quad
پاٹ ابتدائی طور پر منفی  (گھڑی وار)    رخ \عددی{\omega_0=\SI{-4.6}{\radian\per\second}}   زاویائی  رفتار سے حرکت کرتا ہے، تاہم اس کا زاویائی  اسراع \عددی{\alpha} مثبت ہے۔ ابتدائی زاویائی رفتار اور زاویائی اسراع کی علامتیں الٹ ہونے  کی بدولت  پاٹ  منفی رخ چلتے چلتے  بتدریج آہستہ ہوتے رک کر  مثبت رخ گھومنا شروع کرتا ہے۔ حوالہ لکیر  مثبت رخ چل کر \عددی{\theta=0} مقام سے دوبارہ گزرتی ہے اور \عددی{t=\SI{32}{\second}} گزرنے تک مثبت رخ  مزید  \عددی{5.0} چکر کاٹ چکا ہوتا ہے۔

(ج)پاٹ کس وقت \عددی{t} پر لمحاتی رکتا ہے؟

\موٹا{حساب:}\quad
ہم دوبارہ زاویائی مساوات کی فہرست پر نظر ڈالتے ہیں اور ایسی مساوات لینا چاہتے ہیں جس میں صرف  \عددی{t}   نا معلوم متغیر  ہو۔ تاہم، اب مساوات میں \عددی{\omega} کا ہونا بھی ضروری ہے، تا کہ ہم اس کو \عددی{0} لے کر  مطابقتی \عددی{t} کے لئے حل کریں۔ ہم مساوات \حوالہ{مساوات_گھماو_زاویائی_الف} منتخب کرتے ہیں، جو ذیل دیگی۔
\begin{align*}
t=\frac{\omega-\omega_0}{\alpha}=\frac{0-(\SI{-4.6}{\radian\per\second})}{\SI{0.35}{\radian\per\second\squared}}=\SI{13}{\second}\quad\quad \text{\RL{(جواب)}}
\end{align*}
\انتہا{نمونی سوال}
%----------------------------

%sample problem 10.04 p267
\ابتدا{نمونی سوال}\موٹا{مستقل زاویائی اسراع، پہیے کی  سواری}\\
تفریح گاہ میں  ایک بڑا  پہیا چلاتے ہوئے  آپ کی نظر  پہیے پر سوار ایک شخص  پر پڑتی ہے جو  پریشان نظر آتا ہے۔آپ پہیے کی زاویائی سمتی  رفتار مستقل زاویائی اسراع  کے ساتھ  \عددی{\SI{3.40}{\radian\per\second}} سے  \عددی{20.0} چکروں میں کم کر کے \عددی{\SI{2.00}{\radian\per\second}}  کرتے ہیں۔ (اس شخص کو \قول{گھومتا  شخص} تصور کرنے سے \قول{مستقیم حرکت کرتا  شخص} کہنا زیادہ بہتر ہو گا۔)

(ا)   زاویائی سمتی رفتار کی کمی کے دوران مستقل زاویائی اسراع کیا ہو گی؟

\موٹا{کلیدی تصور}\\
پہیے کی زاویائی اسراع   مستقل ہے، لہٰذا ہم  اس کی زاویائی سمتی رفتار اور زاویائی ہٹاو کا تعلق  مستقل زاویائی اسراع کی مساوات (مساوات \حوالہ{مساوات_گھماو_زاویائی_الف} اور مساوات \حوالہ{مساوات_گھماو_زاویائی_ب}) سے  جان  سکتے ہیں۔

\موٹا{حساب:}\quad
آئیں دیکھیں آیا ہم ان بنیادی مساوات کو حل کر پائیں گے۔ ابتدائی زاویائی سمتی رفتار \عددی{\omega_0=\SI{3.40}{\radian\per\second}}، زاویائی ہٹاو \عددی{\theta-\theta_0=\text{\RL{چکر}}\, 20.0}، اور   ہٹاو کے آخر پر زاویائی سمتی رفتار \عددی{\omega=\SI{2.00}{\radian\per\second}} ہے۔ہم مستقل زاویائی اسراع \عددی{\alpha} جاننا چاہتے ہیں۔ دونوں مساوات میں وقت \عددی{t} پایا جاتا ہے، جس میں ضروری نہیں ہم دلچسپی رکھتے ہوں۔

نا معلوم \عددی{t} خارج کرنے کے لئے ہم مساوات \حوالہ{مساوات_گھماو_زاویائی_الف} سے 
\begin{align*}
t=\frac{\omega-\omega_0}{\alpha}
\end{align*}
لکھ کر مساوات \حوالہ{مساوات_گھماو_زاویائی_ب} میں ڈالتے ہیں۔
\begin{align*}
\theta-\theta_0=\omega_0\big(\frac{\omega-\omega_0}{\alpha}\big)+\frac{1}{2}\alpha\big(\frac{\omega-\omega_0}{\alpha}\big)^2
\end{align*}
\عددی{\alpha} کے لئے حل کر کے، دی گئی معلومات پُر کر کے ، اور \عددی{20.0} چکر کو \عددی{\SI{125.7}{\radian}} میں بدل کر ذیل حاصل ہو گا۔
\begin{align*}
\alpha&=\frac{\omega^2-\omega_0^2}{2(\theta-\theta_0)}=\frac{(\SI{2.00}{\radian\per\second})^2-(\SI{3.40}{\radian\per\second})^2}{2(\SI{125.7}{\radian})}\\
&=\SI{-0.0301}{\radian\per\second\squared}\quad\quad\quad\text{\RL{(جواب)}}
\end{align*}
(ب)  رفتار کتنے وقت میں کم کی گئی؟

\موٹا{حساب:}\quad
چونکہ اب ہم \عددی{\alpha} جانتے ہیں، مساوات \حوالہ{مساوات_گھماو_زاویائی_الف} استعمال کر کے \عددی{t} حاصل کیا جا سکتا ہے۔
\begin{align*}
t&=\frac{\omega-\omega_0}{\alpha}=\frac{\SI{2.00}{\radian\per\second}-\SI{3.40}{\radian\per\second}}{\SI{-0.0301}{\radian\per\second\squared}}\\
&=\SI{46.5}{\second}\quad\quad\quad\text{\RL{(جواب)}}
\end{align*}
\انتہا{نمونی سوال}
%--------------------------------

%section 10-3 Relating The Linear And Angular Variables p268

\حصہ{خطی اور زاویائی متغیرات کا    رشتہ}
\موٹا{مقاصد}\\
اس حصے کو پڑھنے کے بعد آپ ذیل کے قابل ہوں گے۔
\begin{enumerate}[1.]
\item
 قائمہ محور  پر گھومتے ہوئے استوار  جسم کے زاویائی  متغیرات   (زاویائی مقام، زاویائی سمتی رفتار، اور زاویائی اسراع) کا  جسم پر ایک  ذرے  ، جو کسی رداس پر پایا جاتا ہو، کے خطی متغیرات (مقام، سمتی رفتار، اور اسراع)  کے ساتھ تعلق  جان پائیں گے۔
 \item
 مماسی اسراع اور رداسی اسراع میں تمیز کر پائیں گے، اور  کسی محور پر گھومتے ہوئے جسم پر موجود ذرے  کے لئے بڑھتی زاویائی رفتار اور گھٹتی زاویائی رفتار  کی  صورت میں   دونوں کے سمتیہ  بنا پائیں گے۔
\end{enumerate}

\موٹا{کلیدی تصور}\\
\begin{itemize}
\item
گھومتے جسم   پر محور  گھماو  سے عمودی  فاصلہ  \عددی{r} پر  پائے جانے والا  نقطہ، رداس \عددی{r} کے  دائرے پر حرکت کرتا ہے۔ اگر جسم زاویہ \عددی{\theta} گھومے، یہ نقطہ درج ذیل    قوسی فاصلہ \عددی{s} طے کریگا، جہاں \عددی{\theta} ریڈیئن میں ناپا جائے گا۔
\begin{align*}
s=\theta r\quad\quad\quad \text{\RL{(ریڈیئن ناپ)}} 
\end{align*}
\item
اس  نقطے کا خطی سمتی رفتار \عددی{\vec{v}}  دائرے کو مماسی ہو گا؛ نقطے کا  خطی رفتار ذیل ہو گا، جہاں \عددی{\omega}  جسم اور نقطے کا (ریڈیئن فی سیکنڈ)  زاویائی رفتار ہے۔
\begin{align*}
v=\omega r\quad\quad \quad \text{\RL{(ریڈیئن ناپ)}}
\end{align*}
\item
اس نقطے کے  خطی اسراع \عددی{\vec{a}} کے دو حصے ہوں گے؛ ایک مماسی  جزو اور دوسرا رداسی جزو۔ مماسی جزو ذیل ہو گا، جہاں  \عددی{\alpha}  جسم کے  (ریڈیئن فی مربع سیکنڈ میں)  زاویائی اسراع  کی قدر ہے۔
\begin{align*}
a_{t}=\alpha r \quad\quad \text{\RL{(ریڈیئن ناپ)}}
\end{align*}
رداسی جزو ذیل ہو گا۔
\begin{align*}
a_r=\frac{v^2}{r}=\omega^2r\quad\quad \text{\RL{(ریڈیئن ناپ)}}
\end{align*}
\item
اگر یہ نقطہ یکساں دائری  حرکت کرتا ہو،  اس نقطے اور جسم کا دوری عرصہ \عددی{T} ذیل ہو گا۔
\begin{align*}
T=\frac{2\pi r}{v}=\frac{2\pi}{\omega}\quad\quad\text{\RL{(ریڈیئن ناپ)}}
\end{align*}
\end{itemize}

%p268
\جزوحصہء{خطی اور زاویائی متغیرات کا   رشتہ}
محور   گھماو کے گرد دائرے پر مستقل خطی رفتار \عددی{v} کے ساتھ  حرکت کرتے ہوئے  ذرے کی یکساں دائری حرکت  پر حصہ \حوالہء{4.5} میں غور کیا گیا۔ جب استوار جسم  کسی محور پر گھومتا ہے، جسم کا پر ذرہ اپنے ایک دائرے پر  اسی  محور کے گرد گھومتا ہے۔ چونکہ جسم استوار (بلا لچک) ہے، ایسے تمام ذرے  ہم قدم چل کر ایک جتنے وقت میں ایک چکر مکمل کرتے ہیں؛ ان سب کی زاویائی رفتار \عددی{w}  برابر  ہے۔

تاہم، ایک ذرہ جتنا محور سے دور ہو گا، اتنا اس کے دائرے کا محیط بڑا ہو گا، لہٰذا اس کی خطی  رفتار  \عددی{v} اتنی زیادہ ہو گی۔ \اصطلاح{گھومنے والے  جھولے }\فرہنگ{گھومنے والا جھولا}\حاشیہب{merry go round}\فرہنگ{merry go round} پر بیٹھ کر آپ  اسے محسوس کر سکتے ہیں۔ مرکز سے جتنے فاصلے پر بھی  آپ   ہوں، آپ کی زاویائی رفتار  \عددی{\omega} ایک جتنی ہو گی، تاہم     مرکز سے دور  ہونے پر    آپ کی خطی رفتار \عددی{v}   بڑھے گی۔

ہم  جسم پر کسی مخصوص نقطے کے خطی متغیرات \عددی{s}، \عددی{v}، اور \عددی{a}  اور سی  جسم کے زاویائی متغیرات \عددی{\theta}، \عددی{\omega}، اور \عددی{\alpha} کا تعلق جاننا چاہتے ہیں۔ متغیرات کی  ان  فہرست  کا  رشتہ  \ترچھا{ محور گھماو  سے  نقطے کے عمودی فاصلہ } \عددی{r} کے ذریعے ہو گا۔ یہ عمودی فاصلہ ،  نقطے اور محور گھماو کے بیچ  عمودی   لکیر پر ناپا جائے گا۔ یہ فاصلہ اس دائرے کا رداس \عددی{r} ہو گا جس پر محور  گھماو  کے گرد نقطہ  حرکت کرتا ہے۔

%p269
\جزوجزوحصہء{مقام}
اگر استوار جسم پر کھینچی گئی حوالہ لکیر زاویہ \عددی{\theta}  گھومے، محور گھماو سے  \عددی{r}   فاصلے پر موجود جسم کے اندر نقطہ دائری قوس پر فاصلہ \عددی{s} طے کرے گا، جہاں \عددی{s} کی قیمت مساوات \حوالہ{مساوات_گھماو_رداسی_فاصلہ_الف}  دیتی ہے۔
\begin{align}\label{مساوات_گھماو_خطی_زاویائی_تعلق_الف}
s=\theta r \quad\quad\text{\RL{(ریڈیئن ناپ)}}
\end{align}
مساوات \حوالہ{مساوات_گھماو_خطی_زاویائی_تعلق_الف}  ہمارا پہلی  خطی و زاویائی تعلق ہے۔\ترچھا{ انتباہ:} زاویہ \عددی{\theta}کی ناپ  ریڈیئن  میں لازمی ہے چونکہ درج بالا مساوات زاویے کی  ریڈیئن  میں ناپ کی تعریف ہے۔

\جزوجزوحصہء{رفتار}
رداس \عددی{r}  کو مستقل رکھ کر وقت کے ساتھ مساوات   \حوالہ{مساوات_گھماو_خطی_زاویائی_تعلق_الف} کا  تفرق  ذیل دیگا۔
\begin{align*}
\frac{\dif s}{\dif t}=\frac{\dif \theta}{\dif t} r 
\end{align*}
لیکن، \عددی{\dif s\!/\!\dif t}  نقطے کی خطی  رفتار  (خطی سمتی رفتار  کی قدر)، اور \عددی{\dif\theta\!/\!\dif t}  گھومتے جسم کی  زاویائی رفتار \عددی{\omega} ہے۔ یوں ذیل ہو گا۔
\begin{align}\label{مساوات_گھماو_خطی_زاویائی_تعلق_ب}
v=\omega r \quad\quad\text{\RL{(ریڈیئن ناپ)}}
\end{align}
\ترچھا{انتباہ:} زاویائی رفتار \عددی{\omega} لازماً ریڈیئن فی سیکنڈ میں ناپی  جائے  گی۔

استوار جسم  کے تمام اندرونی   نقطے  ایک زاویائی رفتار  \عددی{\omega} سے گھومتے ہیں لہٰذا مساوات \حوالہ{مساوات_گھماو_خطی_زاویائی_تعلق_ب} کہتی ہے زیادہ رداس \عددی{r} پر واقع نقطے کی خطی رفتار \عددی{v} زیادہ ہو گی۔ شکل \حوالہء{10.9a} ہمیں  یاد دلاتی ہے کہ ہر   نقطے  کی خطی سمتی رفتار ہمیشہ   نقطے کی دائری راہ کو مماسی ہو گی۔

اگر جسم کا زاویائی رفتار \عددی{w} مستقل ہو، مساوات \حوالہ{مساوات_گھماو_خطی_زاویائی_تعلق_ب} کہتی ہے جسم کے اندر  نقطے کی خطی رفتار  \عددی{v} بھی مستقل ہو گی۔یوں، جسم کے اندر موجود ہر نقطہ  یکساں دائری حرکت کرتا ہے۔استوار جسم کے  ہر اندرونی نقطے کی حرکت  کا دوری   عرصہ \عددی{T} مساوات \حوالہء{4.35}ذیل دیتی ہے۔
\begin{align}
T=\frac{2\pi r}{v}
\end{align}
اس مساوات کے تحت ، ایک چکر کے  فاصلے  \عددی{2\pi r}    کو اس رفتار سے تقسیم کر کے جس  سے فاصلہ طے کیا جائے ، ایک چکر کا وقت   حاصل ہو گا۔ مساوات \حوالہ{مساوات_گھماو_خطی_زاویائی_تعلق_ب} سے \عددی{v}  ڈال کر \عددی{r} منسوخ کر کے ذیل حاصل ہو گا۔
\begin{align}
T=\frac{2\pi}{\omega}\quad\quad\text{\RL{(ریڈیئن ناپ)}}
\end{align}
یہ معادل مساوات کہتی ہے    ایک چکر کا زاویائی فاصلہ ، \عددی{2\pi}  ریڈیئن  ، اس زاویائی رفتار سے تقسیم کر کے ، جس سے زاویائی  فاصلہ طے کیا جائے، ایک چکر کا وقت حاصل ہو گا۔

\جزوجزوحصہء{اسراع}
رداس \عددی{r} مستقل رکھ کر \عددی{t}  کے لحاظ سے  مساوات \حوالہ{مساوات_گھماو_خطی_زاویائی_تعلق_ب} کا  تفرق  ذیل دیگا۔
\begin{align}\label{مساوات_گھماو_اسراع_زاویائی_الف}
\frac{\dif v}{\dif t}=\frac{\dif \omega}{\dif t}r
\end{align}
یہاں  ہم  ایک پیچیدگی  کا سامنا  کرتے ہیں۔ مساوات \حوالہ{مساوات_گھماو_اسراع_زاویائی_الف} کا بایاں ہاتھ \عددی{\dif v\!/\!\dif t} خطی اسراع کے  صرف اس حصے کو ظاہر کرتا ہے جو خطی سمتی رفتار \عددی{\vec{v}} کی\ترچھا{ قدر } \عددی{v}  سے وابستہ ہے۔ سمتی رفتار \عددی{\vec{v}} کی طرح خطی اسراع کا  یہ حصہ نقطے کی راہ کو مماسی ہو گا۔ہم اسے خطی اسراع کا \ترچھا{  مماسی جزو } \عددی{a_t} کہہ کر ذیل لکھتے ہیں، جہاں \عددی{\alpha=\dif\omega\!/\!\dif t} ہے۔
\begin{align}\label{مساوات_گھماو_اسراع_ب}
a_t=\alpha r\quad\quad \text{\RL{(ریڈیئن ناپ)}}
\end{align}
\ترچھا{انتباہ:} مساوات \حوالہ{مساوات_گھماو_اسراع_ب} میں  زاویائی اسراع \عددی{\alpha}  کا ریڈیئن  ناپ  میں  ہونا لازم ۔
%----------------------------------
%p270

