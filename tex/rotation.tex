%p257
%in table 10.01 the tags of equation 2.11 etc needs proper reference usage
\باب{گھماو}
\حصہ{گھماو کے متغیر}
\جزوحصہء{مقاصد}
اس حصہ کو پڑھنے کے بعد آپ درج ذیل کے قابل ہوں گے۔
\begin{enumerate}[1.]
\item
جان پائیں گے اگر جسم کے تمام حصے ایک  محور کے گرد  ہم قدم گھومیں، یہ   استوار  جسم ہو گا۔ (اس باب میں ایسے اجسام پر گفتگو کی جائے گی۔)
\item
جان پائیں گے کہ  اندرونی حوالہ لکیر اور مقررہ  بیرونی حوالہ لکیر  کے بیچ زاویہ،  استوار جسم کا زاویاتی مقام دیگا۔
\item
ابتدائی اور اختتامی زاویاتی مقام  کا زاویاتی ہٹاو کے ساتھ تعلق استعمال کر پائیں گے۔
\item
اوسط زاوی سمتی رفتار،  زاوی ہٹاو، اور ہٹاو کو درکار دورانیے کا  تعلق استعمال کر پائیں گے۔
\item
اوسط زاوی  اسراع ،  زاوی  سمتی رفتار میں تبدیلی، اور اس تبدیلی کو درکار دورانیے کا  تعلق استعمال کر پائیں گے۔
\item
جان پائیں گے کہ خلاف  گھڑی  حرکت مثبت  رخ اور گھڑی وار حرکت منفی  رخ ہو گا۔
\item
زاوی مقام   کو\ترچھا{   وقت  کا تفاعل } جانتے ہوئے، کسی بھی لمحے پر لمحاتی زاوی سمتی رفتار اور دو مختلف وقتوں کے بیچ اوسط زاوی سمتی رفتار تعین کر  پائیں گے۔
\item
زاوی مقام   بالمقابل   وقت   کی \ترچھا{ ترسیم }سے  کسی بھی لمحے پر لمحاتی زاوی سمتی رفتار اور دو مختلف وقتوں کے بیچ اوسط زاوی سمتی رفتار تعین کر  پائیں گے۔
\item
جان پائیں گے کہ لمحاتی زاوی  سمتی رفتار  کی قدر لمحاتی زاوی رفتار ہو گی۔
\item
زاوی  سمتی رفتار    کو  \ترچھا{وقت  کا تفاعل } جانتے ہوئے، کسی بھی لمحے پر لمحاتی زاوی  اسراع اور دو مختلف وقتوں کے بیچ اوسط زاوی  اسراع تعین کر  پائیں گے۔
\item
زاوی سمتی رفتار    بالمقابل   وقت   کی\ترچھا{ ترسیم }سے  کسی بھی لمحے پر لمحاتی زاوی اسراع اور دو مختلف وقتوں کے بیچ اوسط زاوی اسراع تعین کر  پائیں گے۔
\item
وقت کے ساتھ زاوی اسراع تفاعل کا تکمل  لے کر جسم کی زاوی سمتی رفتار میں تبدیلی تعین کر پائیں گے۔

وقت کے ساتھ زاوی  سمتی رفتار  تفاعل کا تکمل  لے کر جسم کے  زاوی مقام میں تبدیلی تعین کر پائیں گے۔
\end{enumerate}

\جزوحصہء{کلیدی تصور}
\begin{itemize}
\item
مقررہ محور،  جو محور گھماو  کہلاتی ہے،  کے گرد استوار جسم کا  گھماو  بیان کرنے کی خاطر ،    جسم کے اندر محور کو عمودی   حوالہ لکیر فرض کی جاتی ہے جو جسم کے ساتھ ہم قدم محور کے گرد گھومتی ہے۔   ایک مقررہ رخ کے ساتھ اس لکیر کا زاوی مقام \عددی{\theta} ناپا جاتا ہے۔ جب \عددی{\theta} کی پیمائش ریڈیئن میں ہو، ذیل ہو گا،
\begin{align*}
\theta=\frac{s}{r}\quad\quad \text{\RL{(ریڈیئن ناپ)}}
\end{align*}
جہاں رداس \عددی{r} کے دائری راہ کا قوسی فاصلہ \عددی{s} اور ریڈیئن میں زاویہ \عددی{\theta} ہے۔
\item
زاویہ کی  درجہ میں اور چکر میں پیمائش کا ریڈیئن پیمائش سے تعلق ذیل ہے۔
\begin{align*}
\text{\RL{ریڈیئن}}\,2\pi=\SI{360}{\degree}=\text{\RL{چکر}}\,1
\end{align*}
\item
ایک جسم جو محور گھماو  کے گرد گھوم کر  اپنا زاوی مقام \عددی{\theta_1} سے تبدیل کر کے \عددی{\theta_2} کرے،  ذیل زاوی ہٹاو  سے گزرتا ہے،
\begin{align*}
\Delta \theta=\theta_2-\theta_1
\end{align*}
جہاں خلاف گھڑی گھماو کے لئے \عددی{\Delta \theta} مثبت اور گھڑی وار گھماو کے لئے منفی ہو گا۔
\item
اگر  جسم \عددی{\Delta t} دورانیہ میں \عددی{\Delta \theta} زاوی ہٹاو  گھومے، اس کی اوسط زاوی سمتی  رفتار  \عددی{\omega_{\text{\RL{اوسط}}}} ذیل ہو گی۔
\begin{align*}
\omega_{\text{\RL{اوسط}}}=\frac{\Delta \theta}{\Delta t}
\end{align*}
جسم کی ( لمحاتی ) زاوی  سمتی رفتار \عددی{\omega} ذیل ہو گی۔
\begin{align*}
\omega=\frac{\dif \theta}{\dif t}
\end{align*}
اوسط  زاوی سمتی رفتار \عددی{\omega_{\text{\RL{اوسط}}}}  اور سمتی رفتار  \عددی{\omega} دونوں سمتی مقادیر ہیں، جن کا رخ دایاں ہاتھ قاعدہ  دیگا۔ خلاف گھڑی گھماو کے لئے ان کا رخ مثبت اور گھڑی وار گھماو کے لئے منفی ہو گا۔ زاوی سمتی رفتار کی قدر جسم  کی زاوی رفتار ہو گی۔
\item
اگر \عددی{\Delta t=t_2-t_1} دورانیہ میں جسم کی زاوی سمتی رفتار \عددی{\omega_1} سے تبدیل ہو کر  \عددی{\omega_2} ہو، اس کا  اوسط زاوی  اسراع \عددی{\alpha_{\text{\RL{اوسط}}}} ذیل ہو گا۔
\begin{align*}
\alpha_{\text{\RL{اوسط}}}=\frac{\omega_2-\omega_1}{t_2-t_1}=\frac{\Delta \omega}{\Delta t}
\end{align*}
جسم کا  ( لمحاتی ) زاوی اسراع \عددی{\alpha}ذیل ہو گا۔
\begin{align*}
\alpha=\frac{\dif \omega}{\dif t}
\end{align*}
\عددی{\alpha_{\text{\RL{اوسط}}}} اور \عددی{\alpha} دونوں سمتی مقادیر ہیں۔
\end{itemize}

\حصہء{طبیعیات کیا ہے؟}
جیسا ہم پہلے ذکر کر چکے، طبیعیات  کی توجہ کا ایک  مرکز \قول{ حرکیات }ہے۔ تاہم، اب تک ہم صرف\اصطلاح{ مستقیم   حرکت } پر بات کرتے رہے ہیں، جس میں جسم سیدھی یا قوسی  لکیر  پر حرکت کرتا ہے (شکل \حوالہء{10-1a})۔ اب ہم \اصطلاح{ گھماو } پر نظر ڈالتے ہیں، جس میں جسم کسی محور کے گرد گھومتا ہے (شکل \حوالہء{10.1b})۔

گھماو تقریباً ہر مشین میں نظر آتا ہے، اور جب  آپ دروازہ کھولتے ہیں آپ اس کو دیکھتے ہیں۔کھیل میں  گھماو اہم کردار ادا کرتا ہے، جیسا  گیند کو زیادہ دور پھینکنے کے لئے (گھومتے  گیند  کو ہوا زیادہ دیر  اٹھا  کر سکتی ہے)، اور کرکٹ میں گیند  قوسی  راہ پر پھینکنے کے لئے (گھومتے گیند کو ہوا دائیں یا بائیں دھکیلتی ہے)۔ گھماو زیادہ اہم مسائل ، جیسا      عمر رسیدہ  ہوائی جہاز میں دھاتی حصوں   کا ٹوٹ پھوٹ، میں بھی  کلیدی کردار ادا کرتا ہے۔

گھماو پر بحث سے قبل   ، حرکت میں ملوث متغیرات متعارف کرتے ہیں، جیسا ہم نے باب \حوالہء{2} میں مستقیم حرکت پر بحث سے قبل کیا۔ ہم دیکھتے ہیں کہ گھماو کے  متغیرات عین   با ب \حوالہء{2} میں یک بُعدی  حرکت  کے متغیرات کی طرح ہیں؛  ایک اہم خصوصی صورت وہ ہے جہاں اسراع (جو یہاں زاوی اسراع ہو گا)   مستقل ہو۔ ہم دیکھتے ہیں  نیوٹن کا دوسرا قاعدہ  زاوی حرکت کے لئے بھی لکھا جا سکتا ہے، تاہم  اب قوت  کی بجائے ایک نئی  مقدار جو \ترچھا{  قوت مروڑ } کہلاتی ہے استعمال  کرنا ہو گا۔  کام اور  کام و حرکی توانائی  مسئلے کا اطلاق   بھی گھماو  حرکت  پر کیا جا سکتا ہے، تاہم  کمیت کی بجائے ایک نئی مقدار جو \ترچھا{زاوی جمود} کہلاتی ہے استعمال کرنا ہو  گا۔ مختصراً،  ہم جو کچھ پڑھ چکے ہیں، اس کا اطلاق گھماو حرکت میں ہو گا، تاہم کبھی کبھار معمولی تبدیلی  کی ضرورت پیش آئے گی۔

\موٹا{انتباہ:}
اگرچہ اس باب میں زیادہ تر حقائق محض  دوبارہ پیش کیے گئے ہیں، دیکھا یہ گیا ہے کہ طلبہ و طالبات کو اس باب میں دشواری پیش آتی ہے۔ اساتذہ کرام اس کی کئی وجوہات پیش کرتے ہیں جن میں سے دو  پر اتفاق پایا جاتا ہے: \عددی{1} یہاں  علامت    کی تعداد بہت زیادہ ہے (جنہیں  یونانی حروف  میں لکھ کر  مشکل میں  مزید اضافہ پیدا ہوتا ہے)، اور \عددی{2}  آپ خطی حرکت سے زیادہ واقف ہیں (اسی لئے  کمرے کے ایک کونے سے دوسرے کونے تک آپ  با آسانی جا سکتے ہیں)،  لیکن گھماو سے آپ کا واسطہ کم رہا ہے (اسی لئے تفریح  گاہ میں آپ  تفریحی جھولے پر سوار ہونے کے لئے پیسہ خرچنے کے لئے راضی ہوتے ہیں)۔ جہاں آپ کو دشواری ہو، دیکھیں آیا مسئلے کو  باب \حوالہء{2} کا یک بُعدی خطی مسئلہ   تصور کرنے  آسانی پیدا ہوتی ہے۔ مثلاً، اگر آپ سے\ترچھا{ زاوی } فاصلہ معلوم کرنے کو کہا جائے، وقتی طور پر  لفظ \ترچھا{زاوی} کو بھول جائیں اور دیکھیں آیا باب \حوالہء{2}  کی ترقیم اور تصورات استعمال کر کے جواب حاصل کرنا آسان ہوتا ہے۔

\جزوحصہء{گھماو کے  متغیر}
ہم مقررہ محور  پر استوار  جسم کے گھماو  پر غور کرنا چاہتے ہیں۔\اصطلاح{ استوار  جسم }\فرہنگ{استوار جسم!تعریف}\حاشیہب{rigid body}\فرہنگ{rigid body!defined} سے مراد  وہ جسم ہے جس  کے تمام  حصے  ، جسم کی شکل و صورت تبدیل کیے بغیر، ہم قدم  گھوم سکتے ہیں۔ \اصطلاح{مقررہ محور }\فرہنگ{مقررہ محور!تعریف}\حاشیہب{fixed axis}\فرہنگ{fixed axis!defined} سے مراد وہ محور ہے جو حرکت نہیں کرتی اور   جس  پر گھوما جا سکتا ہے۔یوں ہم ایسے جسم پر غور نہیں کریں گے جیسا  سورج   (جو گیس  کا کرہ  ہے) جس کے  حصے ایک ساتھ حرکت نہیں کرتے۔ ہم زمین پر  لڑھکتے گیند کی بھی بات نہیں کرتے چونکہ اس کی  محور خود حرکت پذیر ہے (ایسی گیند کی حرکت،   گھماو اور  مستقیم حرکت کا ملاپ ہے )۔

شکل \حوالہء{10.2} میں  مقررہ محور پر ، جو\اصطلاح{ محور گھماو}\فرہنگ{ محور گھماو!تعریف}\حاشیہب{rotation axis}\فرہنگ{rotation axis!defined}یا \اصطلاح{گھماو کی محور } کہلاتی ہے، اختیاری شکل کا استوار  جسم  گھوم رہا ہے۔ خالص  گھماو  (\ترچھا{زاوی حرکت}) میں ،  جسم کا ہر نقطہ ایسے  دائرہ  پر حرکت کرتا ہے، جس کا مرکز  محور  گھماو پر واقع ہے، اور  ہر نقطہ کسی مخصوص وقتی  وقفہ  میں ایک جتنا زاویہ طے کرتا  ہے۔ خالص مستقیم حرکت (خطی حرکت)  میں، جسم کا ہر نقطہ کسی مخصوص وقتی دورانیہ میں  ایک جتنا  \ترچھا{خطی فاصلہ } طے کرتا ہے۔

آئیں باری باری خطی مقادیر  مقام، ہٹاو، سمتی رفتار، اور اسراع کے مماثل زاوی  مقادیر  پر  غور کرتے ہیں۔

\جزوحصہء{زاوی مقام}
شکل \حوالہء{10.2} میں گھماو کو عمودی، جسم کے ساتھ  گھومتی، جسم  سے پکی  جڑی   \ترچھا{ حوالہ لکیر } دکھائی گئی ہے  ۔ کسی مقررہ رخ کے ساتھ ، جس کو ہم \اصطلاح{ صفر زاوی مقام }\فرہنگ{زاوی مقام!صفر}\حاشیہب{zero angular position}\فرہنگ{angular position!zero} مانتے ہیں، اس لکیر کا زاویہ لکیر کا \اصطلاح{ زاوی مقام }\فرہنگ{زاوی مقام!تعریف}\حاشیہب{angular position}\فرہنگ{angular position!defined}  ہو گا۔ شکل \حوالہء{10.3} میں  محور \عددی{x} کے مثبت رخ کے ساتھ زاوی مقام  \عددی{\theta} ناپا گیا ہے۔ ہندسہ سے ہم جانتے ہیں درج ذیل ہو گا۔
\begin{align}\label{مساوات_گھماو_رداسی_فاصلہ_الف}
\theta=\frac{s}{r}\quad\quad \text{\RL{(ریڈیئن ناپ)}}
\end{align}
یہاں محور \عددی{x}  (جو صفر زاوی مقام ہے) سے حوالہ  لکیر  تک دائری قوس کی لمبائی \عددی{s}، اور دائرے کا رداس \عددی{r} ہے۔

اس طرح تعین کیا گیا زاویہ  ، درجہ یا چکر کی بجائے ، \اصطلاح{ریڈیئن }\فرہنگ{ریڈیئن}\حاشیہب{radian}\فرہنگ{radian} میں ناپا جاتا ہے۔ ریڈیئن دو لمبائیوں  کی نسبت   (تقابلی تعلق)ہے  لہٰذا یہ  بے بُعد خالص عدد ہو گا۔ دائرے  کا محیط \عددی{2\pi r} ہے لہٰذا ایک مکمل دائرے میں \عددی{2\pi} ریڈیئن ہوں گے۔
\begin{align}
\text{\RL{چکر}}\, 1=\SI{360}{\degree}=\frac{2\pi r}{r}=\text{\RL{ریڈیئن}}\, 2\pi
\end{align}
یا
\begin{align}
\text{\RL{ریڈیئن}}\,1=\SI{57.3}{\degree}=\text{\RL{چکر}}\,0.159
\end{align}
 محور گھماو پر حوالہ لکیر کی  مکمل  چکر کے بعد ہم \عددی{\theta} واپس  صفر\ترچھا{ نہیں } کرتے۔اگر حوالہ لکیر صفر زاوی مقام سے  ابتدا کر کے دو چکر  مکمل  کرے، لکیر کا زاوی مقام \عددی{\theta=4\pi} ریڈیئن ہو گا۔
 
محور \عددی{x} پر  خالص مستقیم حرکت کے لئے  \عددی{x(t)} ، یعنی مقام بالمقابل وقت،  جانتے ہوئے ہم حرکت پذیر جسم کے بارے میں وہ سب کچھ معلوم کر سکتے ہیں جنہیں جاننا مقصود ہو۔ اسی طرح، خالص گھماو  کے لئے \عددی{\theta(t)}، یعنی زاوی مقام بالمقابل وقت، جانتے ہوئے ہم گھومتے  جسم  کے بارے میں  وہ سب کچھ معلوم کر سکتے ہیں جنہیں جاننا مقصود ہو۔

\جزوحصہء{زاوی ہٹاو}
اگر شکل \حوالہء{10.3}  کا جسم  محور گھماو پر شکل \حوالہء{10.4}  کی طرح  گھوم کر حوالہ لکیر کا زاوی مقام \عددی{\theta_1} سے  تبدیل کر کے \عددی{\theta_2}  کرے، جسم کا زاوی ہٹاو  \عددی{\Delta \theta} ذیل ہو گا۔
\begin{align}
\Delta \theta=\theta_2-\theta_1
\end{align}
زاوی ہٹاو کی یہ تعریف نہ صرف استوار جسم بلکہ جسم کے ہر    اندرونی ذرہ کے لئے درست ہے۔

\موٹا{گھڑیاں منفی ہیں۔}
محور \عددی{x} پر  مستقیم حرکت کی صورت میں جسم کا ہٹاو \عددی{\Delta x}  مثبت یا منفی ہو گا، جو  ،محور پر جسم کی حرکت کے رخ پر منحصر ہے۔ اسی طرح، گھماو کی صورت میں جسم کا  زاوی ہٹاو \عددی{\Delta \theta} درج ذیل قاعدہ کے تحت  مثبت یا منفی ہو گا۔

\ابتدا{قاعدہ}
خلاف گھڑی زاوی ہٹاو مثبت اور گھڑی وار ہٹاو منفی ہو گا۔
\انتہا{قاعدہ}

\قول{گھڑیاں  منفی ہیں} کا فقرہ اس قاعدے کو یاد رکھنے  میں مدد دے سکتا ہے۔یاد رہے  گھڑی  کے سیکنڈ   کی سوئی کا ہر قدم آپ کی زندگی کاٹتی ہے۔

\ابتدا{آزمائش}
قرص اپنے وسطی محور کے گرد گھوم سکتا ہے۔ درج ذیل  ابتدائی  اور اختتامی زاوی مقام کی  مرتب جوڑیوں میں کونسی  منفی زاوی ہٹاو دیتی ہیں؟ (ا)  ابتدائی \عددی{-3}  ریڈیئن، اختتامی \عددی{+5} ریڈیئن؛ 
(ب)   ابتدائی \عددی{-3}  ریڈیئن، اختتامی \عددی{-7} ریڈیئن؛  (ج)   ابتدائی \عددی{7}  ریڈیئن، اختتامی \عددی{-3} ریڈیئن۔
\انتہا{آزمائش}

\جزوحصہء{زاوی سمتی رفتار}
فرض کریں ایک جسم وقت \عددی{t_1} پر زاوی مقام \عددی{\theta_1} پر اور  وقت \عددی{t_2} پر زاوی مقام \عددی{\theta_2} پر  ہو، جیسا شکل \حوالہء{10.4} میں دکھایا گیا ہے۔  ہم \عددی{t_1} تا \عددی{t_2} وقتی دورانیہ \عددی{\Delta t} میں جسم کی \اصطلاح{ اوسط زاوی سمتی رفتار }\فرہنگ{زاوی سمتی رفتار!اوسط، تعریف}\حاشیہب{average angular velocity}\فرہنگ{angular velocity!average, defined}  \عددی{\omega_{\text{\RL{اوسط}}}} کی تعریف ذیل کرتے ہیں،
\begin{align}\label{مساوات_گھماو_اوسط_زاوی_سمتی_رفتار}
\omega_{\text{\RL{اوسط}}}=\frac{\theta_2-\theta_1}{t_2-t_1}=\frac{\Delta \theta}{\Delta t}
\end{align}
جہاں وقت دورانیہ \عددی{\Delta t} میں زاوی ہٹاو \عددی{\Delta \omega} ہے۔ (زاوی سمتی رفتار کے لئے یونانی  حروف  تہجی کا ، چھوٹی لکھائی میں  ،  آخری حرف  \موٹا{اومیگا } \عددی{\omega}  استعمال کیا جائے گا۔)
%----------------------------------------------------------------
%p261
مساوات \حوالہ{مساوات_گھماو_اوسط_زاوی_سمتی_رفتار}  میں \عددی{\Delta t} صفر کے قریب تر کرنے سے  نسبت کی درج ذیل  تحدیدی  قیمت  حاصل ہو گی  جو \اصطلاح{   لمحاتی زاوی سمتی رفتار}\فرہنگ{زاوی سمتی رفتار، لمحاتی،تعریف}\حاشیہب{instantaneous angular velocity}\فرہنگ{angular velocity!instantaneous, defined} \عددی{\omega} (یا      مختصراً \اصطلاح{ زاوی سمتی رفتار } ) کہلاتی ہے۔
\begin{align}\label{مساوات_گھماو_لمحاتی_زاوی_سمتی_رفتار}
\omega=\lim_{\Delta t\to 0}\frac{\Delta \theta}{\Delta t}=\frac{\dif \theta}{\dif t}
\end{align}
اگر \عددی{\theta(t)}  معلوم ہو، اس کا تفرق لے کر   زاوی سمتی رفتار \عددی{\omega} حاصل   ہو گی۔

چونکہ اس جسم کے تمام ذرے ہم قدم ہیں، لہٰذا مساوات \حوالہ{مساوات_گھماو_اوسط_زاوی_سمتی_رفتار} اور مساوات \حوالہ{مساوات_گھماو_لمحاتی_زاوی_سمتی_رفتار} نا صرف مکمل  گھومتے  استوار جسم  کے لئے بلکہ  \ترچھا{  جسم کے ہر  ذرے }کے لئے درست ہیں۔ زاوی سمتی رفتار کی  عمومی مستعمل اکائی ریڈیئن فی سیکنڈ \عددی{(\si{\radian\per\second})}، چکر فی سیکنڈ   ، اور چکر فی منٹ ہے۔

محور \عددی{x} پر مثبت رخ حرکت کرتے ہوئے  ذرے کی سمتی رفتار \عددی{v} مثبت  جبکہ منفی رخ حرکت کی صورت میں منفی ہو گی۔ اسی طرح محور پر مثبت رخ (خلاف گھڑی) گھماو کی صورت میں استوار جسم کی زاوی سمتی  رفتار مثبت  جبکہ منفی رخ  (گھڑی وار) گھماو کی صورت میں منفی ہو گی۔ (\قول{گھڑیاں منفی ہیں } اب بھی درست ہے۔) زاوی سمتی رفتار کی قدر\اصطلاح{ زاوی رفتار }\فرہنگ{زاوی رفتار!تعریف}\حاشیہب{angular speed}\فرہنگ{angular speed!defined}کہلاتی ہے۔ہم  زاوی رفتار کے لئے  بھی \عددی{\omega} علامت استعمال کریں گے۔

\جزوحصہء{زاوی اسراع}
گھومتے ہوئے جسم کی زاوی سمتی رفتار  مستقل نہ ہونے کی صورت میں جسم زاوی اسراع سے دو چار ہو گا۔فرض کریں وقت \عددی{t_1} پر جسم کی زاوی سمتی رفتار \عددی{\omega_1} اور \عددی{t_2} پر \عددی{\omega_2} ہے۔ دورانیہ \عددی{t_1} تا \عددی{t_2}   میں گھومتے ہوئے جسم کی \اصطلاح{ اوسط زاوی اسراع }\فرہنگ{زاوی اسراع!اوسط،تعریف}\حاشیہب{average angular acceleration}\فرہنگ{angular acceleration, average, defined}\عددی{\alpha_{\text{\RL{اوسط}}}}   کی تعریف  ذیل ہے،
\begin{align}\label{مساوات_گھماو_زاوی_اوسط_اسراع}
\alpha_{\text{\RL{اوسط}}}=\frac{\omega_2-\omega_1}{t_2-t_1}=\frac{\Delta \omega}{\Delta t}
\end{align}
جہاں ی \عددی{\Delta \omega}  زاوی سمتی رفتار  میں  \عددی{\Delta t}   کے دوران  تبدیل ہے۔\اصطلاح{   لمحاتی زاوی اسراع }\فرہنگ{زاوی اسراع!تعریف}\حاشیہب{instantaneous angular acceleration}\فرہنگ{angular acceleration!instantaneous, defined}(یا مختصر \اصطلاح{اً زاوی اسراع})، جس سے ہمیں زیادہ دلچسپی ہے، \عددی{\Delta t} صفر کے قریب تر کرنے سے نسبت کی، درج ذیل،  تحدیدی قیمت کو کہتے ہیں۔
\begin{align}\label{مساوات_گھماو_زاوی_لمحاتی_اسراع}
\alpha=\lim_{\Delta t\to 0}\frac{\Delta \omega}{\Delta t}=\frac{\dif \omega}{\dif t}
\end{align}
مساوات \حوالہ{مساوات_گھماو_زاوی_اوسط_اسراع} اور مساوات \حوالہ{مساوات_گھماو_زاوی_لمحاتی_اسراع}\ترچھا{  جسم کے ہر ذرے} کے لئے درست ہیں۔ زاوی اسراع کی عمومی مستعمل اکائی ریڈیئن فی مربع  سیکنڈ \عددی{(\si{\radian\per\second\squared})} اور  چکر فی مربع سیکنڈ ہے۔

%---------------------------------------
%Sample Problem 10.01  p262
\ابتدا{نمونی سوال}\موٹا{زاوی مقام سے زاوی سمتی رفتار کا حصول}\\
شکل \حوالہء{10.5a} میں قرص اپنے  وسطی محور کے گرد گھوم رہا ہے۔ قرص پر حوالہ لکیر کا زاوی مقام \عددی{\theta(t)} ذیل ہے، جہاں \عددی{t} اور \عددی{\theta} بالترتیب سیکنڈ اور ریڈیئن میں ہیں، اور صفر زاوی مقام شکل  میں  دکھایا گیا ہے۔
\begin{align}\label{مساوات_گھماو_نمونی_قرص}
\theta=-1.00-0.600t+0.250t^2
\end{align}
(آپ چاہیں تو وقتی طور پر لفظ \قول{زاوی مقام}  سے \قول{زاوی} خارج کر کے اور \عددی{\theta} علامت کی جگہ \عددی{x} استعمال کر کے  مسئلے کو باب \حوالہء{2}   کی ترقیم  میں لے جائیں۔ آپ کو باب \حوالہء{2} کی یک بُعدی  حرکت کے مقام کی مساوات   حاصل ہو گی۔)

(ا)قرص کا زاوی مقام بالمقابل وقت  \عددی{t=\SI{-3.0}{\second}} تا \عددی{t=\SI{5.4}{\second}}   ترسیم کریں۔ قرص اور اس پر زاوی   مقام کی حوالہ لکیر   کا خاکہ \عددی{t=\SI{-2.0}{\second}}،،  اور \عددی{t=\SI{4.0}{\second}}   ،  اور اس لمحے پر بنائیں جب ترسیم \عددی{t} محور سے گزرتی ہے۔

\جزوحصہ{کلیدی تصور}
قرص کے زاوی مقام سے مراد اس پر کھینچی حوالہ لکیر کا مقام \عددی{\theta(t)}  ہے، جو مساوات  \حوالہ{مساوات_گھماو_نمونی_قرص} دیتی ہے؛ لہٰذا ہم مساوات \حوالہ{مساوات_گھماو_نمونی_قرص} ترسیم کرتے ہیں؛ نتیجہ شکل \حوالہء{10.5b} میں پیش ہے۔

\موٹا{حساب:}\quad
قرص اور حوالہ لکیر کا مقام کسی مخصوص لمحے پر  خاکہ بنانے کے لئے ضروری ہے کہ اس لمحے پر ہمیں \عددی{\theta} معلوم ہو، جو مساوات \حوالہ{مساوات_گھماو_نمونی_قرص} میں لمحے کا وقت ڈالنے سے حاصل ہو گا۔ یوں \عددی{t=\SI{-2.0}{\second}} کے لئے ذیل ہو گا۔
\begin{align*}
\theta&=-1.00-(0.600)(-2.0)+(0.250)(-2.0)^2\\
&=\SI{1.2}{\radian}=\SI{1.2}{\radian}\,\frac{\SI{360}{\degree}}{\text{\RL{ریڈیئن}}2\pi}=\SI{69}{\degree}
\end{align*}
یہ نتیجہ کہتا ہے کہ   قرص پر موجود  حوالہ لکیر لمحہ    \عددی{t=\SI{-2.0}{\second}} پر  صفر مقام سے مثبت رخ (خلاف گھڑی) \عددی{1.2} ریڈیئن یعنی \عددی{\SI{69}{\degree}}  گھوم کر ہو گی۔ شکل \حوالہء{10.5b}   کے خاکہ \عددی{1}  میں حوالہ لکیر کا یہ مقام  دکھایا گیا ہے۔

اسی طرح \عددی{t=0} پر \عددی{\theta} کی قیمت \عددی{-1.00} ریڈیئن یا \عددی{\SI{-57}{\degree}} ہو گی، جس کے تحت حوالہ لکیر  صفر زاوی مقام سے  \عددی{1.0} ریڈیئن یا \عددی{\SI{57}{\degree}} منفی رخ (گھڑی وار)  گھوم کر ہو گی، جیسا  خاکہ \عددی{3} میں دکھایا گیا ہے۔ لمحہ \عددی{t=\SI{4.0}{\second}} پر \عددی{\theta} کی قیمت \عددی{0.60}  ریڈیئن یعنی \عددی{\SI{34}{\degree}}  ہو گی (خاکہ \عددی{5})۔ جس لمحے ترسیم محور \عددی{t} سے گزرتی ہے، \عددی{\theta=0} ہو گا اور حوالہ لکیر لمحاتی عین صفر مقام پر ہو گی (خاکہ \عددی{2} اور \عددی{4})۔

(ب) شکل \حوالہء{10.5b} میں \عددی{\theta(t)} کی کم سے کم قیمت   کس  \عددی{t_{\text{\RL{کمتر}}}}   پر ہو گی؟  \عددی{\theta} کی  کم سے کم قیمت کیا ہے؟

\جزوحصہء{کلیدی تصور}
تفاعل  کی انتہا قیمت (یہاں کم سے کم قیمت)  معلوم کرنے کی خاطر  ہم تفاعل کا ایک گنّا  تفرق  لے کر صفر کے برابر رکھتے ہیں۔

\موٹا{حساب:}\quad
تفاعل \عددی{\theta(t)} کا ایک گنّا تفرق ذیل ہے۔
\begin{align}\label{مساوات_گھماو_نمونی_رفتار}
\frac{\dif \theta}{\dif t}=-0.600+0.500t
\end{align}
اس کو صفر کے برابر رکھ کر \عددی{t} کے لئے حل  کر   کے  لمحہ \عددی{t_{\text{\RL{کمتر}}}} حاصل ہو گا جس پر \عددی{\theta(t)} کی قیمت کم سے کم ہو گی۔
\begin{align*}
t_{\text{\RL{کمتر}}}&=\SI{1.2}{\second}\quad \quad \text{\RL{(جواب)}}
\end{align*}
\عددی{\theta(t)} کی کم سے کم قیمت جاننے کے لئے ہم مساوات \حوالہ{مساوات_گھماو_نمونی_قرص} میں \عددی{t_{\text{\RL{کمتر}}}} ڈالتے ہیں، جو ذیل دیگا۔
\begin{align*}
\theta&=\text{\RL{ریڈیئن}}\, -.136\approx \SI{-77.9}{\degree} \quad \quad \text{\RL{(جواب)}}
\end{align*}
\عددی{\theta(t)} کی\ترچھا{ کم سے کم } قیمت  (شکل \حوالہء{10.5b} میں نشیب)  صفر زاوی مقام سے قرص کی\ترچھا{ زیادہ سے زیادہ  گھڑی وار} گھماو  ہے، جو خاکہ \عددی{3} سے کچھ زیادہ ہو گا۔

(ج)قرص کی زاوی سمتی رفتار \عددی{\omega}  وقت \عددی{t=\SI{-3.0}{\second}}  تا \عددی{t=\SI{6.0}{\second}} ترسیم کریں۔قرص کا خاکہ \عددی{t=\SI{-2.0}{\second}}، \عددی{t=\SI{4.0}{\second}}، اور \عددی{t_{\text{\RL{کمتر}}}} پر بنائیں ، اور بتائیں ان لمحات پر گھومنے کا رخ اور  \عددی{\omega}  کی علامت  کیا ہو گی۔

\جزوحصہء{کلیدی تصور}
مساوات \حوالہ{مساوات_گھماو_لمحاتی_زاوی_سمتی_رفتار} کے تحت زاوی سمتی رفتار \عددی{\omega} سے مراد \عددی{\dif\theta\!/\!\dif t} ہے جو مساوات \حوالہ{مساوات_گھماو_نمونی_رفتار} دیتی ہے۔ یوں ذیل ہو گا۔
\begin{align}\label{مساوات_گھماو_رفتار_الف}
\omega=-0.600+0.500t
\end{align}
اس تفاعل ، \عددی{\omega(t)}،  کی ترسیم شکل \حوالہء{10.5c} میں پیش ہے۔ یہ تفاعل خطی ہے لہٰذا اس کی ترسیم ایک سیدھی لکیر ہے۔ ترسیم کی  ڈھلوان  \عددی{\SI{0.500}{\radian\per\second\squared}}ہے  اور  انتصابی محور  (جو دکھایا نہیں گیا)  کو  ترسیم \عددی{\SI{-0.600}{\radian\per\second}} پر قطع کرتی ہے۔

\موٹا{حساب:}\quad
قرص کا خاکہ \عددی{t=\SI{-2.0}{\second}} پر بنانے کی خاطر ہم  مساوات \حوالہ{مساوات_گھماو_رفتار_الف} میں یہ قیمت ڈال کر ذیل حاصل کرتے ہیں۔
\begin{align*}
\omega=\SI{-1.6}{\radian\per\second}\quad\quad \text{\RL{(جواب)}}
\end{align*}
منفی کی علامت کہتی ہے کہ \عددی{t=\SI{-2.0}{\second}} پر قرص گھڑی وار (منفی رخ) گھوم رہا ہے (جیسا شکل \حوالہء{10.5c} میں دائیں  ہاتھ خاکے میں  دکھایا گیا ہے)۔

مساوات \حوالہ{مساوات_گھماو_رفتار_الف} میں \عددی{t=\SI{4.0}{\second}} ڈال کر ذیل حاصل ہو گا۔
\begin{align*}
\omega=\SI{1.4}{\radian\per\second}\quad\quad \text{\RL{(جواب)}}
\end{align*}
مضمر مثبت علامت کہتی ہے قرص مثبت رخ (خلاف گھڑی) گھوم رہا ہے (شکل \حوالہء{10.5c} میں دایاں ہاتھ خاکہ)۔

\عددی{t_{\text{\RL{کمتر}}}}  کے لئے ہم جانتے ہیں \عددی{\dif\theta\!/\!\dif t=0} ہو گا۔یوں \عددی{\omega=0} ہو گا۔جب حوالہ لکیر  ، شکل \حوالہء{10.5b}\عددی{\theta} میں \عددی{\theta}  کی کم سے کم قیمت کو پہنچتی ہے   ، قرص لمحاتی رکتا ہے، جیسا شکل \حوالہء{10.5c} میں وسطی خاکہ عندیہ دیتا ہے۔  شکل \حوالہء{10.5c} میں \عددی{\omega} بالمقابل \عددی{t} کی ترسیم  پر صفر نقطہ  ، جہاں  ترسیم منفی ( گھڑی وار ) گھماو سے مثبت ( خلاف گھڑی)  گھماو کا آغاز کرتی ہے، وہ نقطہ ہے جہاں قرص لمحاتی رکتا ہے۔

(د) جزو ا تا جزو ج کے نتائج استعمال کر کے \عددی{t=\SI{-3.0}{\second}} تا \عددی{t=\SI{6.0}{\second}}   قرص کی حرکت  بیان کریں۔

\موٹا{بیان:}\quad
جب ہم،  \عددی{t=\SI{-3.0}{\second}} پر   ، قرص پر پہلی مرتبہ نظر  ڈالتے ہیں، اس کا زاوی مقام  مثبت  ،  گھماو گھڑی وار  اور رفتار میں کمی دیکھنے کو ملتی ہے۔ یہ \عددی{\theta=-1.36}  ریڈیئن  پر لمحاتی رکنے کے بعد  خلاف گھڑی  گھومنا شروع کرتا ہے اور آخر کار  اس کا زاوی مقام دوبارہ  مثبت ہوتا ہے۔
\انتہا{نمونی سوال}
%---------------------------
%Sample Problem 10.02 p264
\ابتدا{نمونی سوال}\موٹا{زاوی اسراع سے زاوی سمتی رفتار کا حصول}\\
ایک بچہ لٹو  ذیل زاوی اسراع سے گھماتا ہے، جہاں \عددی{t} اور \عددی{\alpha} بالترتیب سیکنڈ اور ریڈیئن فی مربع  سیکنڈ میں ہے۔
\begin{align*}
\alpha=5t^3-4t
\end{align*}
لمحہ \عددی{t=0} پر لٹو کی زاوی سمتی رفتار \عددی{\SI{5}{\radian\per\second}} ، اور   حوالہ لکیر کا زاوی مقام \عددی{\theta=2}  ریڈیئن ہے۔

(ا) لٹو کی زاوی سمتی رفتار  \عددی{\omega(t)}  کا ریاضی فقرہ  حاصل کریں؛ یعنی ایسا تفاعل معلوم کریں جو وقت پر  زاوی سمتی رفتار  کا انحصار صریحاً  دے۔ (ہم جانتے ہیں ایسا تفاعل موجود ہے چونکہ لٹو زاوی اسراع  سے گزر رہا ہے؛ یوں اس کی زاوی سمتی رفتار تبدیل ہو گی۔ )

\جزوحصہء{کلیدی تصور}
\عددی{\alpha(t)}تعریف کے  رو  سے  \عددی{\omega(t)}  کا وقتی تفرق ہو گا۔یوں، وقت کے لحاظ سے  \عددی{\alpha(t)} کا تکمل \عددی{\omega(t)} دیگا۔

\موٹا{حساب:}\quad
مساوات \حوالہ{مساوات_گھماو_زاوی_لمحاتی_اسراع}  ذیل کہتی ہے
\begin{align*}
\dif \omega=\alpha \dif t
\end{align*}
لہٰذا 
\begin{align*}
\int \dif \omega=\int \alpha \dif t
\end{align*}
ہو گا جو ذیل کے گی، جہاں \عددی{C} تکمل کا مستقل ہے۔
\begin{align*}
\omega=\int (5t^3-4t)\dif t=\frac{5}{4}t^4-\frac{4}{2}t^2+C
\end{align*}
ہم جانتے ہیں \عددی{t=0} پر \عددی{\omega=\SI{5}{\radian\per\second}} ہے؛ اس معلومات کو در ج بالا میں ڈال کر:
\begin{align*}
\SI{5}{\radian\per\second}=0-0+C
\end{align*}
تکمل کا مستقل   \عددی{C=\SI{5}{\radian\per\second}}    حاصل ہو گا۔یوں  درکار  تفاعل ذیل ہو گا۔
\begin{align*}
\omega=\frac{5}{4}t^4-\frac{4}{2}t^2+5\quad\quad \text{\RL{(جواب)}}
\end{align*}
(ب) لٹو کے زاوی مقام \عددی{\theta(t)} کا ریاضی فقرہ تلاش کریں۔

\جزوحصہء{کلیدی تصور}
تعریف کے رو سے \عددی{\theta(t)} کا وقتی تفرق \عددی{\omega(t)} دیگا۔ یوں، وقت کے لحاظ سے \عددی{\omega(t)} کا تکمل \عددی{\theta(t)} دیگا۔

\موٹا{حساب:}\quad
مساوات \حوالہ{مساوات_گھماو_لمحاتی_زاوی_سمتی_رفتار} کے تحت:
\begin{align*}
\dif \theta=\omega \dif t
\end{align*}
ہو گا جس سے ذیل لکھا جا سکتا ہے،
\begin{align*}
\theta&=\int \omega \dif t=\int (\frac{5}{4}t^4-\frac{4}{2}t^2+5)\dif t\\
&=\frac{1}{4}t^5-\frac{2}{3}t^3+5t+C'\\
&=\frac{1}{4}t^5-\frac{2}{3}t^3+5t+2\quad\quad\text{\RL{(جواب)}}
\end{align*}
جہاں \عددی{t=0}  پر \عددی{\theta=\SI{2}{\radian}} جانتے ہوئے \عددی{C'} کی قیمت  حاصل کی گئی۔
\انتہا{نمونی سوال}
%------------------------------
%p264
\حصہء{کیا زاوی مقادیر سمتیات   ہیں؟}
ہم   اکیلے  ذرے  کا مقام، سمتی رفتار، اور اسراع سمتیات سے بیان کر سکتے ہیں۔ اگر ذرہ  صرف ایک  محور پر حرکت کرتا ہو،  سمتی ترقیم استعمال کرنا ضرورت نہیں۔ایسے ذرے کو صرف دو رخ  دستیاب ہیں جنہیں مثبت اور منفی علامت سے ظاہر کیا جا سکتا ہے۔

اسی طرح استوار جسم  قائمہ محور  پر  ، محور کے ہمراہ  دیکھتے ہوئے، صرف خلاف گھڑی اور گھڑی وار   گھوم سکتا ہے۔ان رخ کو ہم مثبت اور منفی سے ظاہر کر سکتے ہیں۔ یہاں ایک سوال اٹھتا ہے: \قول{کیا ہم گھومتے جسم کے زاوی ہٹاو، زاوی سمتی رفتار، اور زاوی اسراع کو سمتیات  سمجھ  سکتے ہیں؟} اس کا جواب ہے \قول{جی ہاں} (زاوی ہٹاو کے  لئے   نیچے  پیش انتباہ  ضرور دیکھیں۔)

\موٹا{زاوی سمتی رفتار۔} زاوی سمتی رفتار کو دیکھیں۔ شکل \حوالہء{10.6a} میں \عددی{\omega=33\tfrac{1}{3}}   چکر فی سیکنڈ  کی   مستقل زاوی رفتار   سے گھڑی وار  رخ  گھومتا ہوا قرص دکھایا گیا ہے۔ جیسا شکل \حوالہء{10.6b} میں دکھایا گیا ہے، ہم اس کی سمتی زاوی  رفتار  گھماو کے محور پر سمتیہ   \عددی{\vec{\omega}} سے ظاہر کر سکتے ہیں۔ اس کا طریقہ کار یوں ہے: سمتیہ کی لمبائی   کسی موزوں پیمانہ کے تحت   رکھی جاتی ہے، مثلاً   \عددی{\SI{1}{\centi\meter}} کو \عددی{10} چکر فی منٹ  کی مطابقت سے رکھ  جا سکتا ہے۔ اس کے بعد \عددی{\vec{\omega}} کا رخ تعین کرنے کے لئے ہم \اصطلاح{ دائیں ہاتھ کا قاعدہ }استعمال کرتے ہیں، جو شکل \حوالہء{10.6c} میں پیش ہے: قرص کو دائیں ہاتھ میں یوں پکڑیں کہ  انگلیاں    \ترچھا{ گھماو کے رخ}  ہوں۔ آپ کا سیدھا کھڑا انگوٹھا  زاوی سمتی رفتار کے سمتیہ کا رخ دیگا۔ اگر قرص مخالف رخ گھومے، دائیں ہاتھ قاعدہ کے تحت \عددی{\vec{\omega}}  بھی  مخالف رخ ہو گا۔

زاوی مقادیر       سمتیات سے ظاہر کرنے کی عادت مشکل سے  ڈلتی ہے۔ ہم فوراً سوچتے ہیں  کہ سمتیہ کے  ہمراہ  کوئی چیز  حرکت  کرے گی۔یہاں  ایسا نہیں ہو گا۔  اس کے بجائے کوئی چیز (جیسا استوار جسم) سمتیہ کے رخ کے  \ترچھا{گرد  } گھومتی ہے۔ خالص گھماو کی دنیا میں،  سمتیہ کا رخ کسی چیز کی حرکت کا رخ نہیں بلکہ  گھماو کی محور دیگا۔ بہرحال، سمتیہ حرکت بھی تعین کرتا ہے۔مزید، یہ   سمتیات  سلجھانے کے  ان تمام قواعد کی تعمیل کرتا ہے جو    باب \حوالہء{3} میں  پیش  کیے گئے۔ زاوی اسراع \عددی{\vec{\alpha}}  بھی ایک  سمتیہ ہے، اور یہ بھی ان قواعد کی تعمیل کرتا ہے۔

اس باب میں صرف   قائمہ محور پر گھماو کی بات کی جائے گی۔ ان میں سمتیات استعمال کرنے کی ضرورت نہیں؛ ہم زاوی سمتی رفتار کو \عددی{\omega} اور زاوی اسراع کو \عددی{\alpha} سے ظاہر کر  کے، خلاف گھڑی گھماو کو مثبت اور گھڑی وار گھماو کو منفی  کی علامت سے ظاہر کر سکتے ہیں۔

\موٹا{زاوی ہٹاو۔} پہلے  انتباہ کی بات کرتے ہیں: زاوی ہٹاو (ماسوائے  انتہائی چھوٹا  ہٹاو) کو سمتیہ سے ظاہر نہیں کیا جا سکتا۔ کیوں نہیں؟ ہم یقیناً اس کے رخ اور قدر کی بات کر سکتے ہیں، جیسا شکل \حوالہء{10.6} میں زاوی سمتی رفتار کے لئے کیا گیا۔ تاہم، سمتیہ سے ظاہر کیے جانے کے قابل ہونے کے لئے ضروری ہے کہ مقدار سمتی جمع کے قواعد پر پورا اترتی  ہو۔ ان قواعد میں ایک قاعدہ کہتا ہے کہ  سمتیات جمع کرتے وقت ان کی ترتیب غیر ضروری ہے۔ زاوی ہٹاو اس قاعدہ پر پورا نہیں اترتا۔

شکل \حوالہء{10.7} میں  دی گئی مثال پر غور کریں۔ ایک کتاب  کو، جو ابتدائی طور پر افقی پڑی ہے، دو مرتبہ \عددی{\SI{90}{\degree}} زاوی ہٹاو سے گزارا گیا ہے؛ ایک مرتبہ شکل \حوالہء{10.7a}   اور دوسری مرتبہ شکل \حوالہء{10.7b} کی طرح۔  دونوں  میں ہٹاو  برابر، لیکن  ترتیب ایک نہیں، اور  آخر میں کتاب  ایک جیسی سمت بند نہیں۔ دوسری مثال لیتے ہیں۔ دایاں ہاتھ لٹکا کر ہتھیلی  ران پر رکھیں۔ کلائی سخت  کر کے،  (1)  بازو   سامنے اتنا اٹھائیں   کہ افقی ہو، (2)  اس کو پورا  دائیں لے جائیں، اور (3) اس کے بعد ہاتھ واپس نیچے ران تک لے جائیں۔ آپ کی ہتھیلی اب سامنے رخ ہو گی۔ اگر آپ یہی عمل الٹ ترتیب سے دہرائیں، آپ کی ہتھیلی  آخر میں کس رخ ہو گی؟ ان مثال سے ہم دیکھتے ہیں کہ زاوی  ہٹاو  کا مجموعہ انہیں جمع کرنے کی   ترتیب پر منحصر ہے، لہٰذا  ہٹاو کو سمتیہ تصور نہیں کیا جا سکتا۔

%-------------------------------------
%sec 10-2 Rotation With Constant Angular Acceleration  p266
\حصہ{مستقل اسراع کے ساتھ گھماو}
\موٹا{مقاصد}\\
اس حصہ کو پڑھنے کے بعد آپ ذیل کے قابل ہوں گے۔
\begin{enumerate}[1.]
\item
مستقل زاوی اسراع کی صورت میں زاوی مقام، زاوی ہٹاو، زاوی سمتی رفتار، زاوی اسراع، اور  گزرے دارانیے کے   تعلق  (جدول \حوالہ{جدول_گھماو_مستقل_اسراع_مساوات}) استعمال کر پائیں گے۔
\end{enumerate}

\موٹا{کلیدی تصور}
\begin{itemize}
\item
مستقل زاوی اسراع  (جس میں  \عددی{\alpha}  مستقل ہو گا) گھماو حرکت کی ایک اہم   خصوصی صورت ہے، جس کی مجرد حرکیات  مساوات  ذیل ہیں۔
\begin{align*}
\omega&=\omega_0+\alpha t\\
\theta-\theta_0&=\omega_0 t+\frac{1}{2}\alpha t^2\\
\omega^2&=\omega_0^2+2\alpha(\theta-\theta_0)\\
\theta-\theta_0&=\frac{1}{2}(\omega+\omega_0)t\\
\theta-\theta_0&=\omega t-\frac{1}{2}\alpha t^2
\end{align*}
\end{itemize}

\جزوحصہء{مستقل زاوی اسراع کا گھماو}
مستقیم حرکت  میں \ترچھا{ مستقل خطی اسراع}  کی حرکت (مثلاً، زمین پر گرتا ہوا جسم) ایک اہم خصوصی صورت ہے۔ جدول \حوالہء{2.1} میں  اس طرح کی حرکت  کو مطمئن کرتی مساوات پیش کی گئیں۔

خالص گھماو میں \ترچھا{ مستقل زاوی اسراع }  ایک اہم خصوصی صورت ہے؛ اس   کو مطمئن کرنے  والی مطابقتی  مساوات  پائی جاتی ہیں۔ ہم انہیں یہاں اخذ نہیں کریں گے، بلکہ مطابقتی خطی مساوات میں  مساوی زاوی متغیرات ڈال کر انہیں پیش کرتے ہیں۔ جدول \حوالہ{جدول_گھماو_مستقل_اسراع_مساوات} میں مساوات کی دونوں فہرست (مساوات \حوالہء{2.11} اور مساوات \حوالہء{2.15} تا مساوات \حوالہء{2.18}؛ مساوات \حوالہ{مساوات_گھماو_زاوی_الف} تا مساوات \حوالہ{مساوات_گھماو_زاوی_ٹ}) پیش کی گئی ہیں۔

\begin{table}
\caption{مستقل خطی اسراع اور مستقل زاوی اسراع کی حرکت کی مساوات}
\label{جدول_گھماو_مستقل_اسراع_مساوات}
\centering
\begin{minipage}{0.45\textwidth}
\begin{align}
&\text{\RL{خطی مساوات}}\notag\\
v&=v_0+at\tag{\setlatin{\حوالہء{2.11}}}\\
x-x_0&=v_0t+\frac{1}{2}at^2\tag{\setlatin{\حوالہء{2.15}}}\\
v^2&=v_0^2+2a(x-x_0)\tag{\setlatin{\حوالہء{2.16}}}\\
x-x_0&=\frac{1}{2}(v_0+v)t\tag{\setlatin{\حوالہء{2.17}}}\\
x-x_0&=vt-\frac{1}{2}at^2\tag{\setlatin{\حوالہء{2.18}}}
\end{align}
\end{minipage}\hfill
\begin{minipage}{0.45\textwidth}
\begin{align}
&\text{\RL{زاوی مساوات}}\notag\\
\omega&=\omega_0+\alpha t\label{مساوات_گھماو_زاوی_الف}\\
\theta-\theta_0&=\omega_0 t+\frac{1}{2}\alpha t^2 \label{مساوات_گھماو_زاوی_ب}\\
\omega^2&=\omega_0^2+2\alpha(\theta-\theta_0) \label{مساوات_گھماو_زاوی_پ}\\
\theta-\theta_0&=\frac{1}{2}(\omega_0+\omega)t  \label{مساوات_گھماو_زاوی_ت}\\
\theta-\theta_0&=\omega t-\frac{1}{2}\alpha t^2  \label{مساوات_گھماو_زاوی_ٹ}
\end{align}
\end{minipage}
\end{table}


یاد رہے مساوات \حوالہء{2.11} اور مساوات \حوالہء{2.15} مستقل خطی اسراع کی بنیادی مساوات ہیں، جن سے  فہرست کی باقی تمام مساوات اخذ کی جا سکتی ہیں۔ اس طرح، مساوات \حوالہ{مساوات_گھماو_زاوی_الف} اور مساوات \حوالہ{مساوات_گھماو_زاوی_ب} مستقل زاوی اسراع کی بنیادی مساوات ہیں، جن سے زاوی مساوات کی فہرست کی باقی تمام مساوات اخذ کی جا سکتی ہیں۔ مستقل زاوی اسراع کا سادہ مسئلہ حل کرنے کے لئے آپ عموماً  زاوی فہرست سے (اگر یہ فہرست آپ کے پاس موجود ہو)    ایک مساوات استعمال کر پائیں گے۔ آپ وہ مساوات منتخب کریں گے جس میں صرف وہ متغیر غیر معلوم ہو جو آپ کو  درکار ہو۔ بہتر طریقہ یہ ہو گا کہ آپ مساوات \حوالہ{مساوات_گھماو_زاوی_الف} اور مساوات \حوالہ{مساوات_گھماو_زاوی_ب} یاد کر لیں اور جب ضرورت پیش آئے، انہیں بطور ہمزاد مساوات حل کریں۔

%--------------------------
\ابتدا{آزمائش}
گھومے جسم کا زاوی مقام \عددی{\theta(t)} چار مختلف صورتوں میں (ا)  \عددی{\theta=3t-4}، (ب) \عددی{\theta=-5t^3+4t^2+6}، (ج) \عددی{\theta=2\!/\!t^2-4\!/\!t}، اور (د) \عددی{\theta=5t^2-3} ہے۔  جدول \حوالہ{جدول_گھماو_مستقل_اسراع_مساوات} کی  زاوی مساوات  کا اطلاق کن صورتوں پر ہو گا؟
\انتہا{آزمائش}
%--------------------------------------

%sample problem 10.03 p267
\ابتدا{نمونی سوال}\موٹا{مستقل زاوی اسراع، چکی کا پاٹ}\\
شکل \حوالہء{10.8} میں  پاٹ  مستقل زاوی اسراع \عددی{\alpha=\SI{0.34}{\radian\per\second\squared}} سے گھوم رہا ہے۔ وقت \عددی{t=0} پر اس کی  زاوی سمتی رفتار  \عددی{\omega_0=\SI{-4.6}{\radian\per\second}} ہے، اور اس پر کھینچی گئی حوالہ لکیر کا مقام \عددی{\theta_0=0} ہے۔

(ا)  وقت \عددی{t=0} سے  کتنی دیر بعد حوالہ لکیر  زاوی مقام \عددی{\theta=5.0}  چکر پر ہو گی؟

\موٹا{کلیدی تصور}\\
چونکہ زاوی  اسراع   مستقل ہے لہٰذا ہم جدول \حوالہ{جدول_گھماو_مستقل_اسراع_مساوات} سے مساوات چن سکتے ہیں۔ ہم مساوات \حوالہ{مساوات_گھماو_زاوی_ب}
\begin{align*}
\theta-\theta_0=\omega_0 t+\frac{1}{2}\alpha t^2
\end{align*}
 کا انتخاب اس لئے  کرتے ہیں کہ اس میں  صرف ایک متغیر، \عددی{t}،  نا معلوم ہے اور ہمیں   یہی درکار ہے۔
 
 \موٹا{حساب:}\quad
 دی گئی معلومات ڈال کر اور \عددی{\theta_0=0} اور \عددی{\theta=\text{\RL{چکر}}\,5.0=10\pi\,\si{\radian}}  لیتے ہوئے  ذیل  ہو گا۔
 \begin{align*}
 10\pi\,\si{\radian}=(\SI{-4.6}{\radian\per\second})t+\frac{1}{2}(\SI{0.35}{\radian\per\second\squared})t^2
 \end{align*}
 (اکائیوں  کے ثبات کی خاطر ہم \عددی{5.0} چکر کو \عددی{10\pi} ریڈیئن میں تبدیل کرتے ہیں۔)  اس  دو درجی الجبرائی مساوات کو حل کرنے سے ذیل حاصل ہو گا۔
 \begin{align*}
 t=\SI{32}{\second}\quad\quad \text{\RL{(جواب)}}
 \end{align*}
 ان ایک عجیب بات پر غور کریں۔ جب ہم پہلی مرتبہ پاٹ  پر نظر ڈالتے ہیں یہ منفی رخ گھوم کر \عددی{\theta=0} سمت بند مقام   سے گزرتا  ہے۔ اس کے باوجود \عددی{\SI{32}{\second}} بعد ہم اسے \عددی{\theta=5.0}   چکر     مثبت سمت بند   مقام پر پاتے ہیں۔ اس دورانیے میں ایسا کیا ہوا کہ پاٹ  مثبت سمت بند مقام پر ہو سکتا ہے؟
 
 (ب)وقت  \عددی{t=0} اور \عددی{t=\SI{32}{\second}} کے بیچ پاٹ کے گھماو پر تبصرہ کریں۔
 
 \موٹا{تبصرہ:}\quad
پاٹ ابتدائی طور پر منفی  (گھڑی وار)    رخ \عددی{\omega_0=\SI{-4.6}{\radian\per\second}}   زاوی  رفتار سے حرکت کرتا ہے، تاہم اس کا زاوی  اسراع \عددی{\alpha} مثبت ہے۔ ابتدائی زاوی رفتار اور زاوی اسراع کی علامتیں الٹ ہونے  کی بدولت  پاٹ  منفی رخ چلتے چلتے  بتدریج آہستہ ہوتے رک کر  مثبت رخ گھومنا شروع کرتا ہے۔ حوالہ لکیر  مثبت رخ چل کر \عددی{\theta=0} مقام سے دوبارہ گزرتی ہے اور \عددی{t=\SI{32}{\second}} گزرنے تک مثبت رخ  مزید  \عددی{5.0} چکر کاٹ چکا ہوتا ہے۔

(ج)پاٹ کس وقت \عددی{t} پر لمحاتی رکتا ہے؟

\موٹا{حساب:}\quad
ہم دوبارہ زاوی مساوات کی فہرست پر نظر ڈالتے ہیں اور ایسی مساوات لینا چاہتے ہیں جس میں صرف  \عددی{t}   نا معلوم متغیر  ہو۔ تاہم، اب مساوات میں \عددی{\omega} کا ہونا بھی ضروری ہے، تا کہ ہم اس کو \عددی{0} لے کر  مطابقتی \عددی{t} کے لئے حل کریں۔ ہم مساوات \حوالہ{مساوات_گھماو_زاوی_الف} منتخب کرتے ہیں، جو ذیل دیگی۔
\begin{align*}
t=\frac{\omega-\omega_0}{\alpha}=\frac{0-(\SI{-4.6}{\radian\per\second})}{\SI{0.35}{\radian\per\second\squared}}=\SI{13}{\second}\quad\quad \text{\RL{(جواب)}}
\end{align*}
\انتہا{نمونی سوال}
%----------------------------

%sample problem 10.04 p267
\ابتدا{نمونی سوال}\موٹا{مستقل زاوی اسراع، پہیے کی  سواری}\\
تفریح گاہ میں  ایک بڑا  پہیا چلاتے ہوئے  آپ کی نظر  پہیے پر سوار ایک شخص  پر پڑتی ہے جو  پریشان نظر آتا ہے۔آپ پہیے کی زاوی سمتی  رفتار مستقل زاوی اسراع  کے ساتھ  \عددی{\SI{3.40}{\radian\per\second}} سے  \عددی{20.0} چکروں میں کم کر کے \عددی{\SI{2.00}{\radian\per\second}}  کرتے ہیں۔ (اس شخص کو \قول{گھومتا  شخص} تصور کرنے سے \قول{مستقیم حرکت کرتا  شخص} کہنا زیادہ بہتر ہو گا۔)

(ا)   زاوی سمتی رفتار کی کمی کے دوران مستقل زاوی اسراع کیا ہو گی؟

\موٹا{کلیدی تصور}\\
پہیے کی زاوی اسراع   مستقل ہے، لہٰذا ہم  اس کی زاوی سمتی رفتار اور زاوی ہٹاو کا تعلق  مستقل زاوی اسراع کی مساوات (مساوات \حوالہ{مساوات_گھماو_زاوی_الف} اور مساوات \حوالہ{مساوات_گھماو_زاوی_ب}) سے  جان  سکتے ہیں۔

\موٹا{حساب:}\quad
آئیں دیکھیں آیا ہم ان بنیادی مساوات کو حل کر پائیں گے۔ ابتدائی زاوی سمتی رفتار \عددی{\omega_0=\SI{3.40}{\radian\per\second}}، زاوی ہٹاو \عددی{\theta-\theta_0=\text{\RL{چکر}}\, 20.0}، اور   ہٹاو کے آخر پر زاوی سمتی رفتار \عددی{\omega=\SI{2.00}{\radian\per\second}} ہے۔ہم مستقل زاوی اسراع \عددی{\alpha} جاننا چاہتے ہیں۔ دونوں مساوات میں وقت \عددی{t} پایا جاتا ہے، جس میں ضروری نہیں ہم دلچسپی رکھتے ہوں۔

نا معلوم \عددی{t} خارج کرنے کے لئے ہم مساوات \حوالہ{مساوات_گھماو_زاوی_الف} سے 
\begin{align*}
t=\frac{\omega-\omega_0}{\alpha}
\end{align*}
لکھ کر مساوات \حوالہ{مساوات_گھماو_زاوی_ب} میں ڈالتے ہیں۔
\begin{align*}
\theta-\theta_0=\omega_0\big(\frac{\omega-\omega_0}{\alpha}\big)+\frac{1}{2}\alpha\big(\frac{\omega-\omega_0}{\alpha}\big)^2
\end{align*}
\عددی{\alpha} کے لئے حل کر کے، دی گئی معلومات پُر کر کے ، اور \عددی{20.0} چکر کو \عددی{\SI{125.7}{\radian}} میں بدل کر ذیل حاصل ہو گا۔
\begin{align*}
\alpha&=\frac{\omega^2-\omega_0^2}{2(\theta-\theta_0)}=\frac{(\SI{2.00}{\radian\per\second})^2-(\SI{3.40}{\radian\per\second})^2}{2(\SI{125.7}{\radian})}\\
&=\SI{-0.0301}{\radian\per\second\squared}\quad\quad\quad\text{\RL{(جواب)}}
\end{align*}
(ب)  رفتار کتنے وقت میں کم کی گئی؟

\موٹا{حساب:}\quad
چونکہ اب ہم \عددی{\alpha} جانتے ہیں، مساوات \حوالہ{مساوات_گھماو_زاوی_الف} استعمال کر کے \عددی{t} حاصل کیا جا سکتا ہے۔
\begin{align*}
t&=\frac{\omega-\omega_0}{\alpha}=\frac{\SI{2.00}{\radian\per\second}-\SI{3.40}{\radian\per\second}}{\SI{-0.0301}{\radian\per\second\squared}}\\
&=\SI{46.5}{\second}\quad\quad\quad\text{\RL{(جواب)}}
\end{align*}
\انتہا{نمونی سوال}
%--------------------------------

%section 10-3 Relating The Linear And Angular Variables p268

\حصہ{خطی اور زاوی متغیرات کا    رشتہ}
\موٹا{مقاصد}\\
اس حصے کو پڑھنے کے بعد آپ ذیل کے قابل ہوں گے۔
\begin{enumerate}[1.]
\item
 قائمہ محور  پر گھومتے ہوئے استوار  جسم کے زاوی  متغیرات   (زاوی مقام، زاوی سمتی رفتار، اور زاوی اسراع) کا  جسم پر ایک  ذرے  ، جو کسی رداس پر پایا جاتا ہو، کے خطی متغیرات (مقام، سمتی رفتار، اور اسراع)  کے ساتھ تعلق  جان پائیں گے۔
 \item
 مماسی اسراع اور رداسی اسراع میں تمیز کر پائیں گے، اور  کسی محور پر گھومتے ہوئے جسم پر موجود ذرے  کے لئے بڑھتی زاوی رفتار اور گھٹتی زاوی رفتار  کی  صورت میں   دونوں کے سمتیہ  بنا پائیں گے۔
\end{enumerate}

\موٹا{کلیدی تصور}\\
\begin{itemize}
\item
گھومتے جسم   پر محور  گھماو  سے عمودی  فاصلہ  \عددی{r} پر  پائے جانے والا  نقطہ، رداس \عددی{r} کے  دائرے پر حرکت کرتا ہے۔ اگر جسم زاویہ \عددی{\theta} گھومے، یہ نقطہ درج ذیل    قوسی فاصلہ \عددی{s} طے کریگا، جہاں \عددی{\theta} ریڈیئن میں ناپا جائے گا۔
\begin{align*}
s=\theta r\quad\quad\quad \text{\RL{(ریڈیئن ناپ)}} 
\end{align*}
\item
اس  نقطے کا خطی سمتی رفتار \عددی{\vec{v}}  دائرے کو مماسی ہو گا؛ نقطے کا  خطی رفتار ذیل ہو گا، جہاں \عددی{\omega}  جسم اور نقطے کا (ریڈیئن فی سیکنڈ)  زاوی رفتار ہے۔
\begin{align*}
v=\omega r\quad\quad \quad \text{\RL{(ریڈیئن ناپ)}}
\end{align*}
\item
اس نقطے کے  خطی اسراع \عددی{\vec{a}} کے دو حصے ہوں گے؛ ایک مماسی  جزو اور دوسرا رداسی جزو۔ مماسی جزو ذیل ہو گا، جہاں  \عددی{\alpha}  جسم کے  (ریڈیئن فی مربع سیکنڈ میں)  زاوی اسراع  کی قدر ہے۔
\begin{align*}
a_{t}=\alpha r \quad\quad \text{\RL{(ریڈیئن ناپ)}}
\end{align*}
رداسی جزو ذیل ہو گا۔
\begin{align*}
a_r=\frac{v^2}{r}=\omega^2r\quad\quad \text{\RL{(ریڈیئن ناپ)}}
\end{align*}
\item
اگر یہ نقطہ یکساں دائری  حرکت کرتا ہو،  اس نقطے اور جسم کا دوری عرصہ \عددی{T} ذیل ہو گا۔
\begin{align*}
T=\frac{2\pi r}{v}=\frac{2\pi}{\omega}\quad\quad\text{\RL{(ریڈیئن ناپ)}}
\end{align*}
\end{itemize}

%p268
\جزوحصہء{خطی اور زاوی متغیرات کا   رشتہ}
محور   گھماو کے گرد دائرے پر مستقل خطی رفتار \عددی{v} کے ساتھ  حرکت کرتے ہوئے  ذرے کی یکساں دائری حرکت  پر حصہ \حوالہء{4.5} میں غور کیا گیا۔ جب استوار جسم  کسی محور پر گھومتا ہے، جسم کا پر ذرہ اپنے ایک دائرے پر  اسی  محور کے گرد گھومتا ہے۔ چونکہ جسم استوار (بلا لچک) ہے، ایسے تمام ذرے  ہم قدم چل کر ایک جتنے وقت میں ایک چکر مکمل کرتے ہیں؛ ان سب کی زاوی رفتار \عددی{w}  برابر  ہے۔

تاہم، ایک ذرہ جتنا محور سے دور ہو گا، اتنا اس کے دائرے کا محیط بڑا ہو گا، لہٰذا اس کی خطی  رفتار  \عددی{v} اتنی زیادہ ہو گی۔ \اصطلاح{گھومنے والے  جھولے }\فرہنگ{گھومنے والا جھولا}\حاشیہب{merry go round}\فرہنگ{merry go round} پر بیٹھ کر آپ  اسے محسوس کر سکتے ہیں۔ مرکز سے جتنے فاصلے پر بھی  آپ   ہوں، آپ کی زاوی رفتار  \عددی{\omega} ایک جتنی ہو گی، تاہم     مرکز سے دور  ہونے پر    آپ کی خطی رفتار \عددی{v}   بڑھے گی۔

ہم  جسم پر کسی مخصوص نقطے کے خطی متغیرات \عددی{s}، \عددی{v}، اور \عددی{a}  اور سی  جسم کے زاوی متغیرات \عددی{\theta}، \عددی{\omega}، اور \عددی{\alpha} کا تعلق جاننا چاہتے ہیں۔ متغیرات کی  ان  فہرست  کا  رشتہ  \ترچھا{ محور گھماو  سے  نقطے کے عمودی فاصلہ } \عددی{r} کے ذریعے ہو گا۔ یہ عمودی فاصلہ ،  نقطے اور محور گھماو کے بیچ  عمودی   لکیر پر ناپا جائے گا۔ یہ فاصلہ اس دائرے کا رداس \عددی{r} ہو گا جس پر محور  گھماو  کے گرد نقطہ  حرکت کرتا ہے۔

%p269
\جزوجزوحصہء{مقام}
اگر استوار جسم پر کھینچی گئی حوالہ لکیر زاویہ \عددی{\theta}  گھومے، محور گھماو سے  \عددی{r}   فاصلے پر موجود جسم کے اندر نقطہ دائری قوس پر فاصلہ \عددی{s} طے کرے گا، جہاں \عددی{s} کی قیمت مساوات \حوالہ{مساوات_گھماو_رداسی_فاصلہ_الف}  دیتی ہے۔
\begin{align}\label{مساوات_گھماو_خطی_زاوی_تعلق_الف}
s=\theta r \quad\quad\text{\RL{(ریڈیئن ناپ)}}
\end{align}
مساوات \حوالہ{مساوات_گھماو_خطی_زاوی_تعلق_الف}  ہمارا پہلی  خطی و زاوی تعلق ہے۔\ترچھا{ انتباہ:} زاویہ \عددی{\theta}کی ناپ  ریڈیئن  میں لازمی ہے چونکہ درج بالا مساوات زاویے کی  ریڈیئن  میں ناپ کی تعریف ہے۔

\جزوجزوحصہء{رفتار}
رداس \عددی{r}  کو مستقل رکھ کر وقت کے ساتھ مساوات   \حوالہ{مساوات_گھماو_خطی_زاوی_تعلق_الف} کا  تفرق  ذیل دیگا۔
\begin{align*}
\frac{\dif s}{\dif t}=\frac{\dif \theta}{\dif t} r 
\end{align*}
لیکن، \عددی{\dif s\!/\!\dif t}  نقطے کی خطی  رفتار  (خطی سمتی رفتار  کی قدر)، اور \عددی{\dif\theta\!/\!\dif t}  گھومتے جسم کی  زاوی رفتار \عددی{\omega} ہے۔ یوں ذیل ہو گا۔
\begin{align}\label{مساوات_گھماو_خطی_زاوی_تعلق_ب}
v=\omega r \quad\quad\text{\RL{(ریڈیئن ناپ)}}
\end{align}
\ترچھا{انتباہ:} زاوی رفتار \عددی{\omega} لازماً ریڈیئن فی سیکنڈ میں ناپی  جائے  گی۔

استوار جسم  کے تمام اندرونی   نقطے  ایک زاوی رفتار  \عددی{\omega} سے گھومتے ہیں لہٰذا مساوات \حوالہ{مساوات_گھماو_خطی_زاوی_تعلق_ب} کہتی ہے زیادہ رداس \عددی{r} پر واقع نقطے کی خطی رفتار \عددی{v} زیادہ ہو گی۔ شکل \حوالہء{10.9a} ہمیں  یاد دلاتی ہے کہ ہر   نقطے  کی خطی سمتی رفتار ہمیشہ   نقطے کی دائری راہ کو مماسی ہو گی۔

اگر جسم کا زاوی رفتار \عددی{w} مستقل ہو، مساوات \حوالہ{مساوات_گھماو_خطی_زاوی_تعلق_ب} کہتی ہے جسم کے اندر  نقطے کی خطی رفتار  \عددی{v} بھی مستقل ہو گی۔یوں، جسم کے اندر موجود ہر نقطہ  یکساں دائری حرکت کرتا ہے۔استوار جسم کے  ہر اندرونی نقطے کی حرکت  کا دوری   عرصہ \عددی{T} مساوات \حوالہء{4.35}ذیل دیتی ہے۔
\begin{align}
T=\frac{2\pi r}{v}
\end{align}
اس مساوات کے تحت ، ایک چکر کے  فاصلے  \عددی{2\pi r}    کو اس رفتار سے تقسیم کر کے جس  سے فاصلہ طے کیا جائے ، ایک چکر کا وقت   حاصل ہو گا۔ مساوات \حوالہ{مساوات_گھماو_خطی_زاوی_تعلق_ب} سے \عددی{v}  ڈال کر \عددی{r} منسوخ کر کے ذیل حاصل ہو گا۔
\begin{align}
T=\frac{2\pi}{\omega}\quad\quad\text{\RL{(ریڈیئن ناپ)}}
\end{align}
یہ معادل مساوات کہتی ہے    ایک چکر کا زاوی فاصلہ ، \عددی{2\pi}  ریڈیئن  ، اس زاوی رفتار سے تقسیم کر کے ، جس سے زاوی  فاصلہ طے کیا جائے، ایک چکر کا وقت حاصل ہو گا۔

\جزوجزوحصہء{اسراع}
رداس \عددی{r} مستقل رکھ کر \عددی{t}  کے لحاظ سے  مساوات \حوالہ{مساوات_گھماو_خطی_زاوی_تعلق_ب} کا  تفرق  ذیل دیگا۔
\begin{align}\label{مساوات_گھماو_اسراع_زاوی_الف}
\frac{\dif v}{\dif t}=\frac{\dif \omega}{\dif t}r
\end{align}
یہاں  ہم  ایک پیچیدگی  کا سامنا  کرتے ہیں۔ مساوات \حوالہ{مساوات_گھماو_اسراع_زاوی_الف} کا بایاں ہاتھ \عددی{\dif v\!/\!\dif t} خطی اسراع کے  صرف اس حصے کو ظاہر کرتا ہے جو خطی سمتی رفتار \عددی{\vec{v}} کی\ترچھا{ قدر } \عددی{v}   کی تبدیلی کا ذمہ دار ہے۔ سمتی رفتار \عددی{\vec{v}} کی طرح خطی اسراع کا  یہ حصہ نقطے کی راہ کو مماسی ہو گا۔ہم اسے خطی اسراع کا \ترچھا{  مماسی جزو } \عددی{a_t} کہہ کر ذیل لکھتے ہیں، جہاں \عددی{\alpha=\dif\omega\!/\!\dif t} ہے۔
\begin{align}\label{مساوات_گھماو_اسراع_ب}
a_t=\alpha r\quad\quad \text{\RL{(ریڈیئن ناپ)}}
\end{align}
\ترچھا{انتباہ:} مساوات \حوالہ{مساوات_گھماو_اسراع_ب} میں  زاوی اسراع \عددی{\alpha}  کا ریڈیئن  ناپ  میں  ہونا لازم ہے۔
%----------------------------------
%p270
ساتھ ہی، جیسا مساوات \حوالہء{4.34} ہمیں بتاتی ہے، دائری راہ پر گامزن ذرے (یا نقطے) کے خطی اسراع  کا  (رداسی مرکز کے رخ) \ترچھا{ رداسی جزو }\عددی{a_r=\tfrac{v^2}{r}} ہو گا، جو خطی سمتی  رفتار \عددی{\vec{v}} کے \ترچھا{ رخ}  میں تبدیلی کا ذمہ دار ہو گا۔ مساوات \حوالہ{مساوات_گھماو_خطی_زاوی_تعلق_ب} سے \عددی{v} ڈال کر یہ جزو درج ذیل لکھا جا سکتا ہے۔
\begin{align}\label{مساوات_گھماو_رداسی_اندر_اسراع_جزو}
a_r=\frac{v^2}{r}=\omega^2 r\quad\quad\text{\RL{(ریڈیئن ناپ)}}
\end{align}
یوں، جیسا شکل \حوالہء{10.9b} میں دکھایا گیا ہے،  استوار گھومتے جسم پر نقطے کے خطی اسراع  کے عموماً دو جزو ہوں گے۔جب بھی  جسم کی زاوی سمتی رفتار غیر صفر ہو،   رداسی اندر  کی طرف کا   جزو  \عددی{a_r}  موجود ہو گا  (جو مساوات \حوالہ{مساوات_گھماو_رداسی_اندر_اسراع_جزو} دیتی ہے)۔ مماسی جزو \عددی{a_t} (جو مساوات \حوالہ{مساوات_گھماو_اسراع_زاوی_الف} دیتی ہے) اس صورت ہو گا جب  زاوی اسراع غیر صفر ہو۔

%---------------------------------
\ابتدا{آزمائش}
گھومنے والے جھولے     کے حلقہ   پر چیونٹی بیٹھی ہے۔ اگر اس  نظام (گھومنا والا جھولا  و چیونٹی)  کی  زاوی سمتی رفتار  مستقل ہو، کیا چیونٹی کا (ا) رداسی اسراع اور (ب) مماسی اسراع ہو گا؟ اگر \عددی{\omega} گھٹ رہی ہو، کیا چیونٹی کا (ج) رداسی اسراع اور (د) مماسی اسراع ہو گا؟
\انتہا{آزمائش}
%-----------------------------------

%Sample problem 10.05 p270
\ابتدا{نمونی سوال}\موٹا{تفریح گاہ میں ایک بڑے  حلقہ کی بناوٹ}\\
ہمیں ایک بڑا افقی   حلقہ ، جس کا رداس \عددی{r=\SI{33.1}{\meter}} ہو گا،  بنانے کو کہا گیا ہے جو انتصابی دھرے پر چلے گا۔(یہ چین میں  موجود دنیا کے سب سے بڑے پہیے جتنا ہو گا۔) سوار حلقے کے  بیرونی  دیوار میں موجود دروازے سے داخل ہو کر اس  دیوار کے ساتھ کھڑے ہوں گے (شکل \حوالہء{10.10a})۔ حلقے پر حوالہ لکیر کا زاوی مقام \عددی{\theta(t)}  لمحہ \عددی{t=0} سے لمحہ   \عددی{t=\SI{2.30}{\second}}  تک ذیل دیتی ہے، جہاں \عددی{c=\SI{6.39e-2}{\radian\per\second\cubed}} ہے۔
\begin{align}
\theta=ct^3
\end{align}
لمحہ \عددی{t=\SI{2.30}{\second}} کے بعد جھولنے  کے پھیرا  مکمل ہونے تک  زاوی رفتار مستقل  رکھی جائے گی۔ گھومنا شروع ہونے کے بعد، سوار کے پاوں تلے فرش ہٹا دی جائے گی، لیکن وہ گرے گا نہیں؛ بلکہ وہ دیوار کے ساتھ مضبوطی سے جکڑا  محسوس کرے گا۔لمحہ \عددی{t=\SI{2.20}{\second}} پر شخص  کی زاوی رفتار \عددی{\omega}، خطی رفتار \عددی{v}، زاوی اسراع \عددی{\alpha}، مماسی اسراع \عددی{a_t}،  رداسی اسراع \عددی{a_r}، اور   اسراع \عددی{\vec{a}} تلاش کرتے ہیں۔

\جزوحصہء{کلیدی تصور}
(1)مساوات \حوالہ{مساوات_گھماو_لمحاتی_زاوی_سمتی_رفتار}    زاوی رفتار \عددی{\omega}  دیتی ہے۔ (2) مساوات \حوالہ{مساوات_گھماو_خطی_زاوی_تعلق_ب} (  دائری راہ پر) خطی رفتار \عددی{v}  اور (محور گھماو کے گرد)   زاوی رفتار \عددی{\omega}    کا تعلق \عددی{v=\omega r} دیتی ہے۔ (3)مساوات \حوالہ{مساوات_گھماو_زاوی_لمحاتی_اسراع}    زاوی اسراع \عددی{\alpha}  دیتی ہے \عددی{(\alpha=\dif \omega\!/\!\dif t)}۔ (4)  مساوات \حوالہ{مساوات_گھماو_اسراع_ب} (دائری راہ کے ہمراہ) مماسی اسراع \عددی{\alpha_t} اور  (محور گھماو کے گرد) زاوی اسراع \عددی{\alpha} کا تعلق \عددی{(a_t=\alpha r)}  دیتی ہے۔ (5)  مساوات \حوالہ{مساوات_گھماو_رداسی_اندر_اسراع_جزو} رداسی اسراع \عددی{(a_r=\omega^2 r)} دیتی ہے۔  (6)  مماسی اور رداسی اسراع   پورے اسراع \عددی{\vec{a}} کے دو آپس میں عمودی جزو ہیں۔

\موٹا{حساب:}\quad
آئیں  ان اقدام سے گزریں۔دیے گئے  زاوی مقام  تفاعل  کا وقتی تفرق لے کر \عددی{t=\SI{2.20}{\second}}   پُر کر کے زاوی سمتی رفتار معلوم کرتے ہیں۔
\begin{gather}
\begin{aligned}\label{مساوات_گھماو_نمونی_زاوی_الف}
\omega&=\frac{\dif \theta}{\dif t}=\frac{\dif}{\dif t}(ct^3)=3ct^2\\
&=3(\SI{6.39e-2}{\radian\per\second\cubed})(\SI{2.20}{\second})^2\\
&=\SI{0.928}{\radian\per\second}\quad\quad\text{\RL{(جواب)}}
\end{aligned}
\end{gather} 
مساوات \حوالہ{مساوات_گھماو_خطی_زاوی_تعلق_ب}  اس لمحے کی ذیل خطی رفتار دیگی۔
\begin{gather}
\begin{aligned}\label{مساوات_گھماو_نمونی_مثال_رفتار_الف}
v&=\omega r=3ct^2 r\\
&=3(\SI{6.39e-2}{\radian\per\second\cubed})(\SI{2.20}{\second})^2(\SI{33.1}{\meter})\\
&=\SI{30.7}{\meter\per\second}\quad\quad\text{\RL{(جواب)}}
\end{aligned}
\end{gather}
اگرچہ یہ  رفتار    \عددی{(\SI{111}{\kilo\meter\per\hour})}  تیز ہے، ایسی رفتار تفریح گاہوں میں عام ہیں، اور خطرے کا باعث نہیں ؛ (جیسا باب \حوالہء{2} میں ذکر کیا گیا) ہمارا  جسم اسراع کو ردعمل کرتا ہے، خطی رفتار کو نہیں (ہم   رفتار پیما نہیں سرعت پیما ہیں)۔ مساوات \حوالہ{مساوات_گھماو_نمونی_مثال_رفتار_الف} کہتی ہے  خطی رفتار  ، وقت کے مربع  کے ساتھ بڑھے گی( تاہم یہ اضافہ \عددی{t=\SI{2.20}{\second}} پر رک جائے گا)۔

اس کے بعد، مساوات \حوالہ{مساوات_گھماو_نمونی_زاوی_الف} کا وقت تفرق لے کر زاوی اسراع معلوم کرتے ہیں۔
\begin{align*}
\alpha&=\frac{\dif \omega}{\dif r}=\frac{\dif}{\dif t}(3ct^2)=6ct\\
&=6(\SI{6.39e-2}{\radian\per\second\cubed})(\SI{2.20}{\second})=\SI{0.843}{\radian\per\second\squared}\quad\text{\RL{(جواب)}}
\end{align*}
اب مساوات \حوالہ{مساوات_گھماو_اسراع_ب} مماسی اسراع \عددی{a_t} دیگی:
\begin{gather}
\begin{aligned}\label{مساوات_گھماو_نمونی_بڑا_پہیا_مماسی}
a_t&=\alpha r=6ctr\\
&=6(\SI{6.39e-2}{\radian\per\second\cubed})(\SI{2.20}{\second})(\SI{33.1}{\meter})\\
&=\SI{27.91}{\meter\per\second\squared}\approx\SI{27.9}{\meter\per\second\squared}\quad\quad\text{\RL{(جواب)}}
\end{aligned}
\end{gather}
جو \عددی{2.8g}، جہاں \عددی{g=\SI{9.8}{\meter\per\second\squared}} ہے، کے برابر ہے (جو  مناسب  ہے اور   پُر لطف ہو گا)۔ مساوات \حوالہ{مساوات_گھماو_نمونی_بڑا_پہیا_مماسی} کہتی ہے مماسی اسراع وقت کے ساتھ بڑھ رہا ہے (تاہم یہ اضافہ \عددی{t=\SI{2.30}{\second}} پر رک جائے گا)۔ مساوات \حوالہ{مساوات_گھماو_رداسی_اندر_اسراع_جزو} سے رداسی اسراع لکھتے  کر:
\begin{align*}
a_r=\omega^2 r
\end{align*}
مساوات \حوالہ{مساوات_گھماو_نمونی_زاوی_الف} سے   \عددی{\omega=3ct^2} ڈالتے ہیں:
\begin{gather}
\begin{aligned}\label{مساوات_گھماو_نمونی_بڑا_پہیا_رداسی}
a_r&=(3ct^2)^2r=9c^2t^4r\\
&=9(\SI{6.39e-2}{\radian\per\second\cubed})^2(\SI{2.20}{\second})^4(\SI{33.1}{\meter})\\
&=\SI{28.49}{\meter\per\second\squared}\approx\SI{28.5}{\meter\per\second\squared}\quad\quad\text{\RL{(جواب)}}
\end{aligned}
\end{gather}
جو \عددی{2.9 g} دیتا ہے (یہ بھی مناسب  ہے اور پُر لطف ہو گا)۔

رداسی اور مماسی اسراع ایک دوسرے کو عمودی ہیں اور سوار کے اسراع \عددی{\vec{a}} کے  جزو  ہیں (شکل \حوالہء{10.10b})۔ اسراع \عددی{\vec{a}} کی قدر ذیل ہو گی:
\begin{gather}
\begin{aligned}
a&=\sqrt{a_r^2+a_t^2}\\
&=\sqrt{(\SI{28.49}{\meter\per\second\squared})^2+(\SI{27.91}{\meter\per\second\squared})^2}\\
&\approx\SI{39.9}{\meter\per\second\squared}\quad\quad\text{\RL{(جواب)}}
\end{aligned}
\end{gather}
جو \عددی{4.1g} کے برابر ہے (یہ یقیناً پُر لطف ہو گ!)۔ یہ تمام مقادیر مناسب  ہیں۔

اسراع \عددی{\vec{a}} کی سمت بندی  جاننے کے لئے ہم زاویہ \عددی{\theta} معلوم کرتے ہیں (شکل \حوالہء{10.10b})۔
\begin{align*}
\tan\theta=\frac{a_t}{a_r}
\end{align*}
آئیں اعدادی نتائج پُر کرنے کی  بجائے  ہم مساوات \حوالہ{مساوات_گھماو_نمونی_بڑا_پہیا_مماسی} اور مساوات \حوالہ{مساوات_گھماو_نمونی_بڑا_پہیا_رداسی}  کے الجبرائی نتائج استعمال کرتے ہیں۔
\begin{align}
\theta=\tan^{-1}\big(\frac{6ctr}{9c^2t^4r}\big)=\tan^{-1}\big(\frac{2}{3ct^3}\big)
\end{align}
ریاضی نتیجے کا بڑا فائدہ یہ ہے کہ  ہم اب دیکھ سکتے ہیں کہ   (1) زاویے پر رداس کا کوئی اثر نہیں ہو گا اور  (2)  اس کی قیمت \عددی{t} کی قیمت \عددی{0} تا \عددی{\SI{2.20}{\second}}  بڑھانے سے  گھٹتی ہے۔ رداسی اسراع  (جو \عددی{t^4} پر منحصر ہے ) بہت جلد مماسی اسراع( جو صرف  \عددی{t} پر منحصر ہے) پر غالب ہو کر  سمتیہ اسراع \عددی{\vec{a}} کو رداسی رخ موڑتا ہے۔ وقت \عددی{t=\SI{2.20}{\second}} پر ذیل ہو گا۔
\begin{align*}
\theta=\tan^{-1}\frac{2}{3(\SI{6.39e-2}{\radian\per\second\cubed})(\SI{2.20}{\second})^3}=\SI{44.4}{\degree}\quad\quad\text{\RL{(جواب)}}
\end{align*}
\انتہا{نمونی سوال}
%---------------------------

\حصہ{گھماو کی حرکی توانائی}
\موٹا{مقاصد}\\
اس حصہ کو پڑھنے کے بعد آپ درج ذیل کے قابل ہوں گے۔
\begin{enumerate}[1.]
\item
ذرے کا گھمیری جمود  نقطہ  پر   تلاش  کر پائیں گے۔
\item
 قائمہ  محور کے گرد گھومتے ہوئے متعدد ذروں کا کل  گھمیری جمود تلاش کر پائیں گے۔
 \item
 گھمیری جمود اور زاوی رفتار کی صورت میں جسم کی  گھمیری حرکی توانائی  تعین کر پائیں گے۔
\end{enumerate}

\موٹا{کلیدی تصور}\\
\begin{itemize}
\item
قائمہ محور پر گھومتے  استوار جسم کی حرکی توانائی \عددی{K} ذیل ہو گی، 
\begin{align*}
K=\frac{1}{2}I\omega^2\quad\quad\text{\RL{(ریڈیئن ناپ)}}
\end{align*}
جہاں \عددی{I} جسم کا  گھمیری جمود  کہلاتا ہے، جس کی تعریف   انفرادی ذروں کے نظام کے لئے درج ذیل ہے۔
\begin{align*}
I=\sum m_ir_i^2
\end{align*}
\end{itemize}

\جزوحصہء{گھماو کی حرکی توانائی}
میز آرا  کا تیزی سے گھومتا دھار دار   پھل یقیناً  گھومنے کی بنا حرکی توانائی رکھتا ہے۔ ہم اس  توانائی کو کس طرح  بیان کر سکتے ہیں؟  ہم توانائی کے عمومی کلیہ \عددی{K=\tfrac{1}{2}mv^2}  سے پورے آرا کی حرکی توانائی حاصل نہیں کر سکتے چونکہ یہ آرے کے مرکز کمیت کی حرکی توانائی دیگا، جو صفر ہے۔

اس کے بجائے، میز آرا (اور  کسی بھی دوسرے گھومتے استوار جسم) کو  ہم مختلف رفتار سے حرکت کرتے ذروں کا مجموعہ تصور کرتے ہیں۔ ان ذروں کی انفرادی حرکی توانائیاں جمع کر کے پورے جسم کی حرکی توانائی حاصل کی جا سکتی ہے۔ یوں گھومتے جسم کی حرکی توانائی ذیل ہوگی،
%eq 10.31
\begin{align}
K&=\frac{1}{2}m_1v_1^2+\frac{1}{2}m_2v_2^2+\frac{1}{2}m_3v_3^2+\cdots\notag\\
&=\sum\frac{1}{2}m_iv_i^2 \label{مساوات_گھماو_حرکی_توانائی_الف}
\end{align}
جہاں \عددی{i} ویں ذرے کی کمیت \عددی{m_i} اور رفتار \عددی{v_i} ہے۔ مجموعہ جسم کے تمام ذروں پر لیا جائے گا۔

مساوات \حوالہ{مساوات_گھماو_حرکی_توانائی_الف} میں مشکل یہ ہے کہ ہر ذرے کی رفتار دوسرے سے مختلف ہو سکتی ہے۔ اس مشکل سے بچنے کی خاطر ہم مساوات \حوالہ{مساوات_گھماو_خطی_زاوی_تعلق_ب} سے  \عددی{v=\omega r} ڈال کر ذیل لکھتے ہیں، جس میں \عددی{\omega} تمام ذروں کے لئے  برابر ہے۔
%eq 10.32
\begin{align}\label{مساوات_گھماو_جمود_الف}
K=\sum\frac{1}{2}m_i(\omega r_i)^2=\frac{1}{2}\big(\sum m_ir_i^2\big)\omega^2
\end{align}

مساوات \حوالہ{مساوات_گھماو_جمود_الف} میں  دائیں ہاتھ قوسین میں بند مقدار  ، محور گھماو کے لحاظ سے گھومتے جسم  کی کمیت کی تقسیم پیش کرتی ہے۔ یہ مقدار، محور گھماو کے لحاظ سے گھومتے جسم کا \اصطلاح{ گھمیری جمود }\فرہنگ{جمود!گھمیری}\حاشیہب{rotational inertia}\فرہنگ{inertia!rotational}(یا   \اصطلاح{جمودی معیار اثر}\فرہنگ{معیار اثر!جمودی}\حاشیہب{moment of inertia}\فرہنگ{inertia!moment of})  کہلاتا ہے ، جس کو ہم \عددی{I} سے ظاہر کرتے ہیں۔ محور گھماو کے لحاظ سے جسم کے \عددی{I} کی قیمت   اٹل ہو گی ۔ (\ترچھا{انتباہ:}  \عددی{I} کی قیمت صرف اس صورت  با معنی ہو گی جب  اس محور کا ذکر کیا جائے۔)  کسی دوسری  محور گھماو پر اسی جسم کا \عددی{I} عموماً  مختلف  ہو گا، تاہم اب بھی اس کی قیمت مستقل ہو گی۔

ہم ذیل لکھ  کر،
%eq 10.33
\begin{align}\label{مساوات_گھماو_آئے_تعریف}
I=\sum m_ir_i^2\quad\quad\text{\RL{(گھمیری جمود)}}
\end{align}
مساوات \حوالہ{مساوات_گھماو_جمود_الف} میں ڈال کر  مطلوبہ تعلق:
%eq 10.34
\begin{align}\label{مساوات_گھماو_حرکی_گھمیری_تعریف}
K=\frac{1}{2}I\omega^2\quad\quad\text{\RL{(ریڈیئن ناپ)}}
\end{align}
حاصل کرتے ہیں۔چونکہ \عددی{v=\omega r} استعمال کر کے درج بالا تعلق حاصل کیا گیا لہٰذا  \عددی{\omega} کی قیمت ریڈیئن ناپ میں لکھنی ضروری ہے۔ جمودی معیار اثر  \عددی{I} کی اکائی کلوگرام مربع میٹر  \عددی{(\si{\kilo\gram\meter\squared})} ہے۔

\موٹا{طریقہ کار۔}اگر جسم چند ذروں پر مشتمل  ہو، ہم ہر ذرے کی انفرادی حرکی توانائی \عددی{mr^2}   تلاش کر کے تمام کا مجموعہ، مساوات \حوالہ{مساوات_گھماو_آئے_تعریف}  کی طرح، لے کر جسم کا  کل گھمیری جمود \عددی{I} معلوم کر سکتے ہیں۔ جسم کی کل گھمیری حرکی توانائی جاننے  کے لئے    معلوم شدہ \عددی{I} کو   مساوات \حوالہ{مساوات_گھماو_حرکی_گھمیری_تعریف} میں ڈالنا ہو گا۔ چند ذروں کے لئے یہ  طریقہ کار استعمال کیا جائے گا؛   اگر جسم  میں ذروں کی تعداد بہت زیادہ  ہو (جیسا ایک سلاخ میں ہو گا) تب  کیا ہو گا؟ اگلے حصے میں ہم  اس  قسم کے \ترچھا{ استمراری اجسام  } کو نپٹنا سیکھیں گے؛ فکر  مت کریں، نتائج منٹوں میں حاصل ہوں گے۔

مساوات \حوالہ{مساوات_گھماو_حرکی_گھمیری_تعریف} جو  خالص گھماو کی صورت میں استوار جسم کی حرکی توانائی  \عددی{K=\tfrac{1}{2}I\omega^2}دیتی ہے،  خالص مستقیم حرکت کی صورت میں حرکی توانائی  کلیہ \عددی{K=\tfrac{1}{2}Mv_{\text{\RL{مرکزکمیت}}}^2} کی  زاوی  معادل  ہے۔ دونوں کلیوں میں \عددی{\tfrac{1}{2}} جزو ضربی پایا جاتا ہے۔ ایک کلیہ میں کمیت \عددی{M}  جبکہ دوسرے میں \عددی{I} (جس میں کمیت اور  کمیت  کی تقسیم  دونوں شامل ہیں)  پایا جاتا ہے۔ساتھ ہی دونوں کلیوں میں رفتار کا مربع پایا جاتا ہے (ایک میں مستقیم اور دوسرے میں زاوی )۔ مستقیم اور زاوی حرکت کی حرکی توانائی دو مختلف توانائیاں نہیں۔ دونوں حرکی توانائی ہے، تاہم مسئلہ دیکھ کر موزوں صورت اپنائی   گئی ہے۔

ہم پہلے کہہ چکے ہیں  کہ گھومتے  جسم کا گھمیری جمود نا صرف کمیت بلکہ کمیت کی تقسیم پر بھی منحصر ہو گا۔ آئیں ایک ایسی مثال دیکھیں جس کو آپ حقیقتاً محسوس کر  سکتے ہیں۔ ایک   لمبی   اور بھاری  سلاخ ، پہلے   طولی  محور پر (شکل \حوالہء{10.11a})    اور اس کے  بعد  وسطی نقطہ سے گزرتی    اور سلاخ کو عمودی   محور  پر  (شکل \حوالہء{10.11b}) گھمائیں۔  دونوں صورتوں  میں کمیت  ایک جتنی ہے، تاہم  پہلی صورت میں گھمانا زیادہ  آسان ہو گا۔پہلی صورت میں  کمیت  کی تقسیم محور گھماو کے  زیادہ قریب ہے؛ یوں شکل \حوالہء{10.11a} میں سلاخ کا گھمیری جمود شکل \حوالہء{10.11b} سے کم ہو گا جس کی بدولت شکل \حوالہء{10.11a} میں گھمانا زیادہ آسان ہو گا۔ کم گھمیری جمود  کی صورت میں گھمانا زیادہ آسان ہو گا۔

\ابتدا{آزمائش}
تین کرہ انتصابی محور کے گرد گھومتے شکل میں دکھائے گئے ہیں۔ہر کمیت کے مرکز سے محور تک عمودی  فاصلہ بھی دیا گیا ہے۔ اس محور پر گھمیری جمود کے لحاظ سے کمیتوں کی درجہ بندی کریں۔ زیادہ قیمت اول رکھیں۔
\begin{center}
\begin{tikzpicture}
\pgfmathsetmacro{\ksepX}{1}
\pgfmathsetmacro{\ksepY}{0.75}
\draw[thick](0,0)--++(0,0.5+2*\ksepY)node[pos=0.35,pin={135:{\text{\RL{محور گھماو}}}}]{};
\draw[](3*\ksepX,0.25)node[circle,draw,inner sep=0pt,minimum width=4pt,fill](aa){};
\draw[](2*\ksepX,0.25+\ksepY)node[circle,draw,inner sep=0pt,minimum width=6pt,fill](bb){};
\draw[](\ksepX,0.25+2*\ksepY)node[circle,draw,inner sep=0pt,minimum width=8pt,fill](cc){};
\draw(aa.east)node[right]{\(\SI{4}{\kilo\gram}\)};
\draw(bb.east)node[right]{\(\SI{9}{\kilo\gram}\)};
\draw(cc.east)node[right]{\(\SI{36}{\kilo\gram}\)};
\draw(aa)--++(-3*\ksepX,0)node[pos=0.5,above]{\(\SI{3}{\meter}\)};
\draw(bb)--++(-2*\ksepX,0)node[pos=0.5,above]{\(\SI{2}{\meter}\)};
\draw(cc)--++(-1*\ksepX,0)node[pos=0.5,above]{\(\SI{1}{\meter}\)};
\end{tikzpicture}
\end{center}
\انتہا{آزمائش}
%-------------------------
% section 10.5 p 273
