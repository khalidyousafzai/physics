%p257
%figures and external references pending. chapter is complete otherwise
%figures in Checkpoints too missing but will not show if searched using \\حوالہء{} strategy
\باب{گھماو}
\حصہ{گھماو کے متغیر}
\جزوحصہء{مقاصد}
اس حصہ کو پڑھنے کے بعد آپ درج ذیل کے قابل ہوں گے۔
\begin{enumerate}[1.]
\item
جان پائیں گے اگر جسم کے تمام حصے ایک  محور کے گرد  ہم قدم گھومیں، یہ   استوار  جسم ہو گا۔ (اس باب میں ایسے اجسام پر گفتگو کی جائے گی۔)
\item
جان پائیں گے کہ  اندرونی حوالہ لکیر اور مقررہ  بیرونی حوالہ لکیر  کے بیچ زاویہ،  استوار جسم کا زاویاتی مقام دیگا۔
\item
ابتدائی اور اختتامی زاویاتی مقام  کا زاویاتی ہٹاو کے ساتھ تعلق استعمال کر پائیں گے۔
\item
اوسط زاوی سمتی رفتار،  زاوی ہٹاو، اور ہٹاو کو درکار دورانیے کا  تعلق استعمال کر پائیں گے۔
\item
اوسط زاوی  اسراع ،  زاوی  سمتی رفتار میں تبدیلی، اور اس تبدیلی کو درکار دورانیے کا  تعلق استعمال کر پائیں گے۔
\item
جان پائیں گے کہ خلاف  گھڑی  حرکت مثبت  رخ اور گھڑی وار حرکت منفی  رخ ہو گا۔
\item
زاوی مقام   کو\ترچھا{   وقت  کا تفاعل } جانتے ہوئے، کسی بھی لمحے پر لمحاتی زاوی سمتی رفتار اور دو مختلف وقتوں کے بیچ اوسط زاوی سمتی رفتار تعین کر  پائیں گے۔
\item
زاوی مقام   بالمقابل   وقت   کی \ترچھا{ ترسیم }سے  کسی بھی لمحے پر لمحاتی زاوی سمتی رفتار اور دو مختلف وقتوں کے بیچ اوسط زاوی سمتی رفتار تعین کر  پائیں گے۔
\item
جان پائیں گے کہ لمحاتی زاوی  سمتی رفتار  کی قدر لمحاتی زاوی رفتار ہو گی۔
\item
زاوی  سمتی رفتار    کو  \ترچھا{وقت  کا تفاعل } جانتے ہوئے، کسی بھی لمحے پر لمحاتی زاوی  اسراع اور دو مختلف وقتوں کے بیچ اوسط زاوی  اسراع تعین کر  پائیں گے۔
\item
زاوی سمتی رفتار    بالمقابل   وقت   کی\ترچھا{ ترسیم }سے  کسی بھی لمحے پر لمحاتی زاوی اسراع اور دو مختلف وقتوں کے بیچ اوسط زاوی اسراع تعین کر  پائیں گے۔
\item
وقت کے ساتھ زاوی اسراع تفاعل کا تکمل  لے کر جسم کی زاوی سمتی رفتار میں تبدیلی تعین کر پائیں گے۔

وقت کے ساتھ زاوی  سمتی رفتار  تفاعل کا تکمل  لے کر جسم کے  زاوی مقام میں تبدیلی تعین کر پائیں گے۔
\end{enumerate}

\جزوحصہء{کلیدی تصور}
\begin{itemize}
\item
مقررہ محور،  جو محور گھماو  کہلاتی ہے،  کے گرد استوار جسم کا  گھماو  بیان کرنے کی خاطر ،    جسم کے اندر محور کو عمودی   حوالہ لکیر فرض کی جاتی ہے جو جسم کے ساتھ ہم قدم محور کے گرد گھومتی ہے۔   ایک مقررہ رخ کے ساتھ اس لکیر کا زاوی مقام \عددی{\theta} ناپا جاتا ہے۔ جب \عددی{\theta} کی پیمائش ریڈیئن میں ہو، ذیل ہو گا،
\begin{align*}
\theta=\frac{s}{r}\quad\quad \text{\RL{(ریڈیئن ناپ)}}
\end{align*}
جہاں رداس \عددی{r} کے دائری راہ کا قوسی فاصلہ \عددی{s} اور ریڈیئن میں زاویہ \عددی{\theta} ہے۔
\item
زاویہ کی  درجہ میں اور چکر میں پیمائش کا ریڈیئن پیمائش سے تعلق ذیل ہے۔
\begin{align*}
\text{\RL{ریڈیئن}}\,2\pi=\SI{360}{\degree}=\text{\RL{چکر}}\,1
\end{align*}
\item
ایک جسم جو محور گھماو  کے گرد گھوم کر  اپنا زاوی مقام \عددی{\theta_1} سے تبدیل کر کے \عددی{\theta_2} کرے،  ذیل زاوی ہٹاو  سے گزرتا ہے،
\begin{align*}
\Delta \theta=\theta_2-\theta_1
\end{align*}
جہاں خلاف گھڑی گھماو کے لئے \عددی{\Delta \theta} مثبت اور گھڑی وار گھماو کے لئے منفی ہو گا۔
\item
اگر  جسم \عددی{\Delta t} دورانیہ میں \عددی{\Delta \theta} زاوی ہٹاو  گھومے، اس کی اوسط زاوی سمتی  رفتار  \عددی{\omega_{\text{\RL{اوسط}}}} ذیل ہو گی۔
\begin{align*}
\omega_{\text{\RL{اوسط}}}=\frac{\Delta \theta}{\Delta t}
\end{align*}
جسم کی ( لمحاتی ) زاوی  سمتی رفتار \عددی{\omega} ذیل ہو گی۔
\begin{align*}
\omega=\frac{\dif \theta}{\dif t}
\end{align*}
اوسط  زاوی سمتی رفتار \عددی{\omega_{\text{\RL{اوسط}}}}  اور سمتی رفتار  \عددی{\omega} دونوں سمتی مقادیر ہیں، جن کا رخ دایاں ہاتھ قاعدہ  دیگا۔ خلاف گھڑی گھماو کے لئے ان کا رخ مثبت اور گھڑی وار گھماو کے لئے منفی ہو گا۔ زاوی سمتی رفتار کی قدر جسم  کی زاوی رفتار ہو گی۔
\item
اگر \عددی{\Delta t=t_2-t_1} دورانیہ میں جسم کی زاوی سمتی رفتار \عددی{\omega_1} سے تبدیل ہو کر  \عددی{\omega_2} ہو، اس کا  اوسط زاوی  اسراع \عددی{\alpha_{\text{\RL{اوسط}}}} ذیل ہو گا۔
\begin{align*}
\alpha_{\text{\RL{اوسط}}}=\frac{\omega_2-\omega_1}{t_2-t_1}=\frac{\Delta \omega}{\Delta t}
\end{align*}
جسم کا  ( لمحاتی ) زاوی اسراع \عددی{\alpha}ذیل ہو گا۔
\begin{align*}
\alpha=\frac{\dif \omega}{\dif t}
\end{align*}
\عددی{\alpha_{\text{\RL{اوسط}}}} اور \عددی{\alpha} دونوں سمتی مقادیر ہیں۔
\end{itemize}

\حصہء{طبیعیات کیا ہے؟}
جیسا ہم پہلے ذکر کر چکے، طبیعیات  کی توجہ کا ایک  مرکز \قول{ حرکیات }ہے۔ تاہم، اب تک ہم صرف\اصطلاح{ مستقیم   حرکت } پر بات کرتے رہے ہیں، جس میں جسم سیدھی یا قوسی  لکیر  پر حرکت کرتا ہے (شکل \حوالہء{10-1a})۔ اب ہم \اصطلاح{ گھماو } پر نظر ڈالتے ہیں، جس میں جسم کسی محور کے گرد گھومتا ہے (شکل \حوالہء{10.1b})۔

گھماو تقریباً ہر مشین میں نظر آتا ہے، اور جب  آپ دروازہ کھولتے ہیں آپ اس کو دیکھتے ہیں۔کھیل میں  گھماو اہم کردار ادا کرتا ہے، جیسا  گیند کو زیادہ دور پھینکنے کے لئے (گھومتے  گیند  کو ہوا زیادہ دیر  اٹھا  کر سکتی ہے)، اور کرکٹ میں گیند  قوسی  راہ پر پھینکنے کے لئے (گھومتے گیند کو ہوا دائیں یا بائیں دھکیلتی ہے)۔ گھماو زیادہ اہم مسائل ، جیسا      عمر رسیدہ  ہوائی جہاز میں دھاتی حصوں   کا ٹوٹ پھوٹ، میں بھی  کلیدی کردار ادا کرتا ہے۔

گھماو پر بحث سے قبل   ، حرکت میں ملوث متغیرات متعارف کرتے ہیں، جیسا ہم نے باب \حوالہء{2} میں مستقیم حرکت پر بحث سے قبل کیا۔ ہم دیکھتے ہیں کہ گھماو کے  متغیرات عین   با ب \حوالہء{2} میں یک بُعدی  حرکت  کے متغیرات کی طرح ہیں؛  ایک اہم خصوصی صورت وہ ہے جہاں اسراع (جو یہاں زاوی اسراع ہو گا)   مستقل ہو۔ ہم دیکھتے ہیں  نیوٹن کا دوسرا قاعدہ  زاوی حرکت کے لئے بھی لکھا جا سکتا ہے، تاہم  اب قوت  کی بجائے ایک نئی  مقدار جو \ترچھا{  قوت مروڑ } کہلاتی ہے استعمال  کرنا ہو گا۔  کام اور  کام و حرکی توانائی  مسئلے کا اطلاق   بھی گھماو  حرکت  پر کیا جا سکتا ہے، تاہم  کمیت کی بجائے ایک نئی مقدار جو \ترچھا{زاوی جمود} کہلاتی ہے استعمال کرنا ہو  گا۔ مختصراً،  ہم جو کچھ پڑھ چکے ہیں، اس کا اطلاق گھماو حرکت میں ہو گا، تاہم کبھی کبھار معمولی تبدیلی  کی ضرورت پیش آئے گی۔

\موٹا{انتباہ:}
اگرچہ اس باب میں زیادہ تر حقائق محض  دوبارہ پیش کیے گئے ہیں، دیکھا یہ گیا ہے کہ طلبہ و طالبات کو اس باب میں دشواری پیش آتی ہے۔ اساتذہ کرام اس کی کئی وجوہات پیش کرتے ہیں جن میں سے دو  پر اتفاق پایا جاتا ہے: \عددی{1} یہاں  علامت    کی تعداد بہت زیادہ ہے (جنہیں  یونانی حروف  میں لکھ کر  مشکل میں  مزید اضافہ پیدا ہوتا ہے)، اور \عددی{2}  آپ خطی حرکت سے زیادہ واقف ہیں (اسی لئے  کمرے کے ایک کونے سے دوسرے کونے تک آپ  با آسانی جا سکتے ہیں)،  لیکن گھماو سے آپ کا واسطہ کم رہا ہے (اسی لئے تفریح  گاہ میں آپ  تفریحی جھولے پر سوار ہونے کے لئے پیسہ خرچنے کے لئے راضی ہوتے ہیں)۔ جہاں آپ کو دشواری ہو، دیکھیں آیا مسئلے کو  باب \حوالہء{2} کا یک بُعدی خطی مسئلہ   تصور کرنے  آسانی پیدا ہوتی ہے۔ مثلاً، اگر آپ سے\ترچھا{ زاوی } فاصلہ معلوم کرنے کو کہا جائے، وقتی طور پر  لفظ \ترچھا{زاوی} کو بھول جائیں اور دیکھیں آیا باب \حوالہء{2}  کی ترقیم اور تصورات استعمال کر کے جواب حاصل کرنا آسان ہوتا ہے۔

\جزوحصہء{گھماو کے  متغیر}
ہم مقررہ محور  پر استوار  جسم کے گھماو  پر غور کرنا چاہتے ہیں۔\اصطلاح{ استوار  جسم }\فرہنگ{استوار جسم!تعریف}\حاشیہب{rigid body}\فرہنگ{rigid body!defined} سے مراد  وہ جسم ہے جس  کے تمام  حصے  ، جسم کی شکل و صورت تبدیل کیے بغیر، ہم قدم  گھوم سکتے ہیں۔ \اصطلاح{مقررہ محور }\فرہنگ{مقررہ محور!تعریف}\حاشیہب{fixed axis}\فرہنگ{fixed axis!defined} سے مراد وہ محور ہے جو حرکت نہیں کرتی اور   جس  پر گھوما جا سکتا ہے۔یوں ہم ایسے جسم پر غور نہیں کریں گے جیسا  سورج   (جو گیس  کا کرہ  ہے) جس کے  حصے ایک ساتھ حرکت نہیں کرتے۔ ہم زمین پر  لڑھکتے گیند کی بھی بات نہیں کرتے چونکہ اس کی  محور خود حرکت پذیر ہے (ایسی گیند کی حرکت،   گھماو اور  مستقیم حرکت کا ملاپ ہے )۔

شکل \حوالہء{10.2} میں  مقررہ محور پر ، جو\اصطلاح{ محور گھماو}\فرہنگ{ محور گھماو!تعریف}\حاشیہب{rotation axis}\فرہنگ{rotation axis!defined}یا \اصطلاح{گھماو کی محور } کہلاتی ہے، اختیاری شکل کا استوار  جسم  گھوم رہا ہے۔ خالص  گھماو  (\ترچھا{زاوی حرکت}) میں ،  جسم کا ہر نقطہ ایسے  دائرہ  پر حرکت کرتا ہے، جس کا مرکز  محور  گھماو پر واقع ہے، اور  ہر نقطہ کسی مخصوص وقتی  وقفہ  میں ایک جتنا زاویہ طے کرتا  ہے۔ خالص مستقیم حرکت (خطی حرکت)  میں، جسم کا ہر نقطہ کسی مخصوص وقتی دورانیہ میں  ایک جتنا  \ترچھا{خطی فاصلہ } طے کرتا ہے۔

آئیں باری باری خطی مقادیر  مقام، ہٹاو، سمتی رفتار، اور اسراع کے مماثل زاوی  مقادیر  پر  غور کرتے ہیں۔

\جزوحصہء{زاوی مقام}
شکل \حوالہء{10.2} میں گھماو کو عمودی، جسم کے ساتھ  گھومتی، جسم  سے پکی  جڑی   \ترچھا{ حوالہ لکیر } دکھائی گئی ہے  ۔ کسی مقررہ رخ کے ساتھ ، جس کو ہم \اصطلاح{ صفر زاوی مقام }\فرہنگ{زاوی مقام!صفر}\حاشیہب{zero angular position}\فرہنگ{angular position!zero} مانتے ہیں، اس لکیر کا زاویہ لکیر کا \اصطلاح{ زاوی مقام }\فرہنگ{زاوی مقام!تعریف}\حاشیہب{angular position}\فرہنگ{angular position!defined}  ہو گا۔ شکل \حوالہء{10.3} میں  محور \عددی{x} کے مثبت رخ کے ساتھ زاوی مقام  \عددی{\theta} ناپا گیا ہے۔ ہندسہ سے ہم جانتے ہیں درج ذیل ہو گا۔
%eq 10.1
\begin{align}\label{مساوات_گھماو_رداسی_فاصلہ_الف}
\theta=\frac{s}{r}\quad\quad \text{\RL{(ریڈیئن ناپ)}}
\end{align}
یہاں محور \عددی{x}  (جو صفر زاوی مقام ہے) سے حوالہ  لکیر  تک دائری قوس کی لمبائی \عددی{s}، اور دائرے کا رداس \عددی{r} ہے۔

اس طرح تعین کیا گیا زاویہ  ، درجہ یا چکر کی بجائے ، \اصطلاح{ریڈیئن }\فرہنگ{ریڈیئن}\حاشیہب{radian}\فرہنگ{radian} میں ناپا جاتا ہے۔ ریڈیئن دو لمبائیوں  کی نسبت   (تقابلی تعلق)ہے  لہٰذا یہ  بے بُعد خالص عدد ہو گا۔ دائرے  کا محیط \عددی{2\pi r} ہے لہٰذا ایک مکمل دائرے میں \عددی{2\pi} ریڈیئن ہوں گے۔
%eq 10.2
\begin{align}\label{مساوات_گھماو_ریڈیئن_اور_درجے}
\text{\RL{چکر}}\, 1=\SI{360}{\degree}=\frac{2\pi r}{r}=\text{\RL{ریڈیئن}}\, 2\pi
\end{align}
یا
%eq 10.3
\begin{align}
\text{\RL{ریڈیئن}}\,1=\SI{57.3}{\degree}=\text{\RL{چکر}}\,0.159
\end{align}
 محور گھماو پر حوالہ لکیر کی  مکمل  چکر کے بعد ہم \عددی{\theta} واپس  صفر\ترچھا{ نہیں } کرتے۔اگر حوالہ لکیر صفر زاوی مقام سے  ابتدا کر کے دو چکر  مکمل  کرے، لکیر کا زاوی مقام \عددی{\theta=4\pi} ریڈیئن ہو گا۔
 
محور \عددی{x} پر  خالص مستقیم حرکت کے لئے  \عددی{x(t)} ، یعنی مقام بالمقابل وقت،  جانتے ہوئے ہم حرکت پذیر جسم کے بارے میں وہ سب کچھ معلوم کر سکتے ہیں جنہیں جاننا مقصود ہو۔ اسی طرح، خالص گھماو  کے لئے \عددی{\theta(t)}، یعنی زاوی مقام بالمقابل وقت، جانتے ہوئے ہم گھومتے  جسم  کے بارے میں  وہ سب کچھ معلوم کر سکتے ہیں جنہیں جاننا مقصود ہو۔

\جزوحصہء{زاوی ہٹاو}
اگر شکل \حوالہء{10.3}  کا جسم  محور گھماو پر شکل \حوالہء{10.4}  کی طرح  گھوم کر حوالہ لکیر کا زاوی مقام \عددی{\theta_1} سے  تبدیل کر کے \عددی{\theta_2}  کرے، جسم کا زاوی ہٹاو  \عددی{\Delta \theta} ذیل ہو گا۔
%eq 10.4
\begin{align}\label{مساوات_گھماو_زاوی_ہٹاو_تعریف}
\Delta \theta=\theta_2-\theta_1
\end{align}
زاوی ہٹاو کی یہ تعریف نہ صرف استوار جسم بلکہ جسم کے ہر    اندرونی ذرہ کے لئے درست ہے۔

\موٹا{گھڑیاں منفی ہیں۔}
محور \عددی{x} پر  مستقیم حرکت کی صورت میں جسم کا ہٹاو \عددی{\Delta x}  مثبت یا منفی ہو گا، جو  ،محور پر جسم کی حرکت کے رخ پر منحصر ہے۔ اسی طرح، گھماو کی صورت میں جسم کا  زاوی ہٹاو \عددی{\Delta \theta} درج ذیل قاعدہ کے تحت  مثبت یا منفی ہو گا۔

\ابتدا{قاعدہ}
خلاف گھڑی زاوی ہٹاو مثبت اور گھڑی وار ہٹاو منفی ہو گا۔
\انتہا{قاعدہ}

\قول{گھڑیاں  منفی ہیں} کا فقرہ اس قاعدے کو یاد رکھنے  میں مدد دے سکتا ہے۔یاد رہے  گھڑی  کے سیکنڈ   کی سوئی کا ہر قدم آپ کی زندگی کاٹتی ہے۔

\ابتدا{آزمائش}
قرص اپنے وسطی محور کے گرد گھوم سکتا ہے۔ درج ذیل  ابتدائی  اور اختتامی زاوی مقام کی  مرتب جوڑیوں میں کونسی  منفی زاوی ہٹاو دیتی ہیں؟ (ا)  ابتدائی \عددی{-3}  ریڈیئن، اختتامی \عددی{+5} ریڈیئن؛ 
(ب)   ابتدائی \عددی{-3}  ریڈیئن، اختتامی \عددی{-7} ریڈیئن؛  (ج)   ابتدائی \عددی{7}  ریڈیئن، اختتامی \عددی{-3} ریڈیئن۔
\انتہا{آزمائش}

\جزوحصہء{زاوی سمتی رفتار}
فرض کریں ایک جسم وقت \عددی{t_1} پر زاوی مقام \عددی{\theta_1} پر اور  وقت \عددی{t_2} پر زاوی مقام \عددی{\theta_2} پر  ہو، جیسا شکل \حوالہء{10.4} میں دکھایا گیا ہے۔  ہم \عددی{t_1} تا \عددی{t_2} وقتی دورانیہ \عددی{\Delta t} میں جسم کی \اصطلاح{ اوسط زاوی سمتی رفتار }\فرہنگ{زاوی سمتی رفتار!اوسط، تعریف}\حاشیہب{average angular velocity}\فرہنگ{angular velocity!average, defined}  \عددی{\omega_{\text{\RL{اوسط}}}} کی تعریف ذیل کرتے ہیں،
%eq 10.5
\begin{align}\label{مساوات_گھماو_اوسط_زاوی_سمتی_رفتار}
\omega_{\text{\RL{اوسط}}}=\frac{\theta_2-\theta_1}{t_2-t_1}=\frac{\Delta \theta}{\Delta t}
\end{align}
جہاں وقت دورانیہ \عددی{\Delta t} میں زاوی ہٹاو \عددی{\Delta \omega} ہے۔ (زاوی سمتی رفتار کے لئے یونانی  حروف  تہجی کا ، چھوٹی لکھائی میں  ،  آخری حرف  \موٹا{اومیگا } \عددی{\omega}  استعمال کیا جائے گا۔)
%----------------------------------------------------------------
%p261
مساوات \حوالہ{مساوات_گھماو_اوسط_زاوی_سمتی_رفتار}  میں \عددی{\Delta t} صفر کے قریب تر کرنے سے  نسبت کی درج ذیل  تحدیدی  قیمت  حاصل ہو گی  جو \اصطلاح{   لمحاتی زاوی سمتی رفتار}\فرہنگ{زاوی سمتی رفتار، لمحاتی،تعریف}\حاشیہب{instantaneous angular velocity}\فرہنگ{angular velocity!instantaneous, defined} \عددی{\omega} (یا      مختصراً \اصطلاح{ زاوی سمتی رفتار } ) کہلاتی ہے۔
%eq 10.6
\begin{align}\label{مساوات_گھماو_لمحاتی_زاوی_سمتی_رفتار}
\omega=\lim_{\Delta t\to 0}\frac{\Delta \theta}{\Delta t}=\frac{\dif \theta}{\dif t}
\end{align}
اگر \عددی{\theta(t)}  معلوم ہو، اس کا تفرق لے کر   زاوی سمتی رفتار \عددی{\omega} حاصل   ہو گی۔

چونکہ اس جسم کے تمام ذرے ہم قدم ہیں، لہٰذا مساوات \حوالہ{مساوات_گھماو_اوسط_زاوی_سمتی_رفتار} اور مساوات \حوالہ{مساوات_گھماو_لمحاتی_زاوی_سمتی_رفتار} نا صرف مکمل  گھومتے  استوار جسم  کے لئے بلکہ  \ترچھا{  جسم کے ہر  ذرے }کے لئے درست ہیں۔ زاوی سمتی رفتار کی  عمومی مستعمل اکائی ریڈیئن فی سیکنڈ \عددی{(\si{\radian\per\second})}، چکر فی سیکنڈ   ، اور چکر فی منٹ ہے۔

محور \عددی{x} پر مثبت رخ حرکت کرتے ہوئے  ذرے کی سمتی رفتار \عددی{v} مثبت  جبکہ منفی رخ حرکت کی صورت میں منفی ہو گی۔ اسی طرح محور پر مثبت رخ (خلاف گھڑی) گھماو کی صورت میں استوار جسم کی زاوی سمتی  رفتار مثبت  جبکہ منفی رخ  (گھڑی وار) گھماو کی صورت میں منفی ہو گی۔ (\قول{گھڑیاں منفی ہیں } اب بھی درست ہے۔) زاوی سمتی رفتار کی قدر\اصطلاح{ زاوی رفتار }\فرہنگ{زاوی رفتار!تعریف}\حاشیہب{angular speed}\فرہنگ{angular speed!defined}کہلاتی ہے۔ہم  زاوی رفتار کے لئے  بھی \عددی{\omega} علامت استعمال کریں گے۔

\جزوحصہء{زاوی اسراع}
گھومتے ہوئے جسم کی زاوی سمتی رفتار  مستقل نہ ہونے کی صورت میں جسم زاوی اسراع سے دو چار ہو گا۔فرض کریں وقت \عددی{t_1} پر جسم کی زاوی سمتی رفتار \عددی{\omega_1} اور \عددی{t_2} پر \عددی{\omega_2} ہے۔ دورانیہ \عددی{t_1} تا \عددی{t_2}   میں گھومتے ہوئے جسم کی \اصطلاح{ اوسط زاوی اسراع }\فرہنگ{زاوی اسراع!اوسط،تعریف}\حاشیہب{average angular acceleration}\فرہنگ{angular acceleration, average, defined}\عددی{\alpha_{\text{\RL{اوسط}}}}   کی تعریف  ذیل ہے،
%eq 10.7
\begin{align}\label{مساوات_گھماو_زاوی_اوسط_اسراع}
\alpha_{\text{\RL{اوسط}}}=\frac{\omega_2-\omega_1}{t_2-t_1}=\frac{\Delta \omega}{\Delta t}
\end{align}
جہاں ی \عددی{\Delta \omega}  زاوی سمتی رفتار  میں  \عددی{\Delta t}   کے دوران  تبدیل ہے۔\اصطلاح{   لمحاتی زاوی اسراع }\فرہنگ{زاوی اسراع!تعریف}\حاشیہب{instantaneous angular acceleration}\فرہنگ{angular acceleration!instantaneous, defined}(یا مختصر \اصطلاح{اً زاوی اسراع})، جس سے ہمیں زیادہ دلچسپی ہے، \عددی{\Delta t} صفر کے قریب تر کرنے سے نسبت کی، درج ذیل،  تحدیدی قیمت کو کہتے ہیں۔
%eq 10.8
\begin{align}\label{مساوات_گھماو_زاوی_لمحاتی_اسراع}
\alpha=\lim_{\Delta t\to 0}\frac{\Delta \omega}{\Delta t}=\frac{\dif \omega}{\dif t}
\end{align}
مساوات \حوالہ{مساوات_گھماو_زاوی_اوسط_اسراع} اور مساوات \حوالہ{مساوات_گھماو_زاوی_لمحاتی_اسراع}\ترچھا{  جسم کے ہر ذرے} کے لئے درست ہیں۔ زاوی اسراع کی عمومی مستعمل اکائی ریڈیئن فی مربع  سیکنڈ \عددی{(\si{\radian\per\second\squared})} اور  چکر فی مربع سیکنڈ ہے۔

%---------------------------------------
%Sample Problem 10.01  p262
\ابتدا{نمونی سوال}\موٹا{زاوی مقام سے زاوی سمتی رفتار کا حصول}\\
شکل \حوالہء{10.5a} میں قرص اپنے  وسطی محور کے گرد گھوم رہا ہے۔ قرص پر حوالہ لکیر کا زاوی مقام \عددی{\theta(t)} ذیل ہے، جہاں \عددی{t} اور \عددی{\theta} بالترتیب سیکنڈ اور ریڈیئن میں ہیں، اور صفر زاوی مقام شکل  میں  دکھایا گیا ہے۔
%eq 10.9
\begin{align}\label{مساوات_گھماو_نمونی_قرص}
\theta=-1.00-0.600t+0.250t^2
\end{align}
(آپ چاہیں تو وقتی طور پر لفظ \قول{زاوی مقام}  سے \قول{زاوی} خارج کر کے اور \عددی{\theta} علامت کی جگہ \عددی{x} استعمال کر کے  مسئلے کو باب \حوالہء{2}   کی ترقیم  میں لے جائیں۔ آپ کو باب \حوالہء{2} کی یک بُعدی  حرکت کے مقام کی مساوات   حاصل ہو گی۔)

(ا)قرص کا زاوی مقام بالمقابل وقت  \عددی{t=\SI{-3.0}{\second}} تا \عددی{t=\SI{5.4}{\second}}   ترسیم کریں۔ قرص اور اس پر زاوی   مقام کی حوالہ لکیر   کا خاکہ \عددی{t=\SI{-2.0}{\second}}،،  اور \عددی{t=\SI{4.0}{\second}}   ،  اور اس لمحے پر بنائیں جب ترسیم \عددی{t} محور سے گزرتی ہے۔

\جزوحصہ{کلیدی تصور}
قرص کے زاوی مقام سے مراد اس پر کھینچی حوالہ لکیر کا مقام \عددی{\theta(t)}  ہے، جو مساوات  \حوالہ{مساوات_گھماو_نمونی_قرص} دیتی ہے؛ لہٰذا ہم مساوات \حوالہ{مساوات_گھماو_نمونی_قرص} ترسیم کرتے ہیں؛ نتیجہ شکل \حوالہء{10.5b} میں پیش ہے۔

\موٹا{حساب:}\quad
قرص اور حوالہ لکیر کا مقام کسی مخصوص لمحے پر  خاکہ بنانے کے لئے ضروری ہے کہ اس لمحے پر ہمیں \عددی{\theta} معلوم ہو، جو مساوات \حوالہ{مساوات_گھماو_نمونی_قرص} میں لمحے کا وقت ڈالنے سے حاصل ہو گا۔ یوں \عددی{t=\SI{-2.0}{\second}} کے لئے ذیل ہو گا۔
\begin{align*}
\theta&=-1.00-(0.600)(-2.0)+(0.250)(-2.0)^2\\
&=\SI{1.2}{\radian}=\SI{1.2}{\radian}\,\frac{\SI{360}{\degree}}{\text{\RL{ریڈیئن}}2\pi}=\SI{69}{\degree}
\end{align*}
یہ نتیجہ کہتا ہے کہ   قرص پر موجود  حوالہ لکیر لمحہ    \عددی{t=\SI{-2.0}{\second}} پر  صفر مقام سے مثبت رخ (خلاف گھڑی) \عددی{1.2} ریڈیئن یعنی \عددی{\SI{69}{\degree}}  گھوم کر ہو گی۔ شکل \حوالہء{10.5b}   کے خاکہ \عددی{1}  میں حوالہ لکیر کا یہ مقام  دکھایا گیا ہے۔

اسی طرح \عددی{t=0} پر \عددی{\theta} کی قیمت \عددی{-1.00} ریڈیئن یا \عددی{\SI{-57}{\degree}} ہو گی، جس کے تحت حوالہ لکیر  صفر زاوی مقام سے  \عددی{1.0} ریڈیئن یا \عددی{\SI{57}{\degree}} منفی رخ (گھڑی وار)  گھوم کر ہو گی، جیسا  خاکہ \عددی{3} میں دکھایا گیا ہے۔ لمحہ \عددی{t=\SI{4.0}{\second}} پر \عددی{\theta} کی قیمت \عددی{0.60}  ریڈیئن یعنی \عددی{\SI{34}{\degree}}  ہو گی (خاکہ \عددی{5})۔ جس لمحے ترسیم محور \عددی{t} سے گزرتی ہے، \عددی{\theta=0} ہو گا اور حوالہ لکیر لمحاتی عین صفر مقام پر ہو گی (خاکہ \عددی{2} اور \عددی{4})۔

(ب) شکل \حوالہء{10.5b} میں \عددی{\theta(t)} کی کم سے کم قیمت   کس  \عددی{t_{\text{\RL{کمتر}}}}   پر ہو گی؟  \عددی{\theta} کی  کم سے کم قیمت کیا ہے؟

\جزوحصہء{کلیدی تصور}
تفاعل  کی انتہا قیمت (یہاں کم سے کم قیمت)  معلوم کرنے کی خاطر  ہم تفاعل کا ایک گنّا  تفرق  لے کر صفر کے برابر رکھتے ہیں۔

\موٹا{حساب:}\quad
تفاعل \عددی{\theta(t)} کا ایک گنّا تفرق ذیل ہے۔
%eq 10.10
\begin{align}\label{مساوات_گھماو_نمونی_رفتار}
\frac{\dif \theta}{\dif t}=-0.600+0.500t
\end{align}
اس کو صفر کے برابر رکھ کر \عددی{t} کے لئے حل  کر   کے  لمحہ \عددی{t_{\text{\RL{کمتر}}}} حاصل ہو گا جس پر \عددی{\theta(t)} کی قیمت کم سے کم ہو گی۔
\begin{align*}
t_{\text{\RL{کمتر}}}&=\SI{1.2}{\second}\quad \quad \text{\RL{(جواب)}}
\end{align*}
\عددی{\theta(t)} کی کم سے کم قیمت جاننے کے لئے ہم مساوات \حوالہ{مساوات_گھماو_نمونی_قرص} میں \عددی{t_{\text{\RL{کمتر}}}} ڈالتے ہیں، جو ذیل دیگا۔
\begin{align*}
\theta&=\text{\RL{ریڈیئن}}\, -.136\approx \SI{-77.9}{\degree} \quad \quad \text{\RL{(جواب)}}
\end{align*}
\عددی{\theta(t)} کی\ترچھا{ کم سے کم } قیمت  (شکل \حوالہء{10.5b} میں نشیب)  صفر زاوی مقام سے قرص کی\ترچھا{ زیادہ سے زیادہ  گھڑی وار} گھماو  ہے، جو خاکہ \عددی{3} سے کچھ زیادہ ہو گا۔

(ج)قرص کی زاوی سمتی رفتار \عددی{\omega}  وقت \عددی{t=\SI{-3.0}{\second}}  تا \عددی{t=\SI{6.0}{\second}} ترسیم کریں۔قرص کا خاکہ \عددی{t=\SI{-2.0}{\second}}، \عددی{t=\SI{4.0}{\second}}، اور \عددی{t_{\text{\RL{کمتر}}}} پر بنائیں ، اور بتائیں ان لمحات پر گھومنے کا رخ اور  \عددی{\omega}  کی علامت  کیا ہو گی۔

\جزوحصہء{کلیدی تصور}
مساوات \حوالہ{مساوات_گھماو_لمحاتی_زاوی_سمتی_رفتار} کے تحت زاوی سمتی رفتار \عددی{\omega} سے مراد \عددی{\dif\theta\!/\!\dif t} ہے جو مساوات

 \حوالہ{مساوات_گھماو_نمونی_رفتار} دیتی ہے۔ یوں ذیل ہو گا۔
 %eq 10.11
\begin{align}\label{مساوات_گھماو_رفتار_الف}
\omega=-0.600+0.500t
\end{align}
اس تفاعل ، \عددی{\omega(t)}،  کی ترسیم شکل \حوالہء{10.5c} میں پیش ہے۔ یہ تفاعل خطی ہے لہٰذا اس کی ترسیم ایک سیدھی لکیر ہے۔ ترسیم کی  ڈھلوان  \عددی{\SI{0.500}{\radian\per\second\squared}}ہے  اور  انتصابی محور  (جو دکھایا نہیں گیا)  کو  ترسیم \عددی{\SI{-0.600}{\radian\per\second}} پر قطع کرتی ہے۔

\موٹا{حساب:}\quad
قرص کا خاکہ \عددی{t=\SI{-2.0}{\second}} پر بنانے کی خاطر ہم  مساوات \حوالہ{مساوات_گھماو_رفتار_الف} میں یہ قیمت ڈال کر ذیل حاصل کرتے ہیں۔
\begin{align*}
\omega=\SI{-1.6}{\radian\per\second}\quad\quad \text{\RL{(جواب)}}
\end{align*}
منفی کی علامت کہتی ہے کہ \عددی{t=\SI{-2.0}{\second}} پر قرص گھڑی وار (منفی رخ) گھوم رہا ہے (جیسا شکل \حوالہء{10.5c} میں دائیں  ہاتھ خاکے میں  دکھایا گیا ہے)۔

مساوات \حوالہ{مساوات_گھماو_رفتار_الف} میں \عددی{t=\SI{4.0}{\second}} ڈال کر ذیل حاصل ہو گا۔
\begin{align*}
\omega=\SI{1.4}{\radian\per\second}\quad\quad \text{\RL{(جواب)}}
\end{align*}
مضمر مثبت علامت کہتی ہے قرص مثبت رخ (خلاف گھڑی) گھوم رہا ہے (شکل \حوالہء{10.5c} میں دایاں ہاتھ خاکہ)۔

\عددی{t_{\text{\RL{کمتر}}}}  کے لئے ہم جانتے ہیں \عددی{\dif\theta\!/\!\dif t=0} ہو گا۔یوں \عددی{\omega=0} ہو گا۔جب حوالہ لکیر  ، شکل \حوالہء{10.5b}\عددی{\theta} میں \عددی{\theta}  کی کم سے کم قیمت کو پہنچتی ہے   ، قرص لمحاتی رکتا ہے، جیسا شکل \حوالہء{10.5c} میں وسطی خاکہ عندیہ دیتا ہے۔  شکل \حوالہء{10.5c} میں \عددی{\omega} بالمقابل \عددی{t} کی ترسیم  پر صفر نقطہ  ، جہاں  ترسیم منفی ( گھڑی وار ) گھماو سے مثبت ( خلاف گھڑی)  گھماو کا آغاز کرتی ہے، وہ نقطہ ہے جہاں قرص لمحاتی رکتا ہے۔

(د) جزو ا تا جزو ج کے نتائج استعمال کر کے \عددی{t=\SI{-3.0}{\second}} تا \عددی{t=\SI{6.0}{\second}}   قرص کی حرکت  بیان کریں۔

\موٹا{بیان:}\quad
جب ہم،  \عددی{t=\SI{-3.0}{\second}} پر   ، قرص پر پہلی مرتبہ نظر  ڈالتے ہیں، اس کا زاوی مقام  مثبت  ،  گھماو گھڑی وار  اور رفتار میں کمی دیکھنے کو ملتی ہے۔ یہ \عددی{\theta=-1.36}  ریڈیئن  پر لمحاتی رکنے کے بعد  خلاف گھڑی  گھومنا شروع کرتا ہے اور آخر کار  اس کا زاوی مقام دوبارہ  مثبت ہوتا ہے۔
\انتہا{نمونی سوال}
%---------------------------
%Sample Problem 10.02 p264
\ابتدا{نمونی سوال}\موٹا{زاوی اسراع سے زاوی سمتی رفتار کا حصول}\\
ایک بچہ لٹو  ذیل زاوی اسراع سے گھماتا ہے، جہاں \عددی{t} اور \عددی{\alpha} بالترتیب سیکنڈ اور ریڈیئن فی مربع  سیکنڈ میں ہے۔
\begin{align*}
\alpha=5t^3-4t
\end{align*}
لمحہ \عددی{t=0} پر لٹو کی زاوی سمتی رفتار \عددی{\SI{5}{\radian\per\second}} ، اور   حوالہ لکیر کا زاوی مقام \عددی{\theta=2}  ریڈیئن ہے۔

(ا) لٹو کی زاوی سمتی رفتار  \عددی{\omega(t)}  کا ریاضی فقرہ  حاصل کریں؛ یعنی ایسا تفاعل معلوم کریں جو وقت پر  زاوی سمتی رفتار  کا انحصار صریحاً  دے۔ (ہم جانتے ہیں ایسا تفاعل موجود ہے چونکہ لٹو زاوی اسراع  سے گزر رہا ہے؛ یوں اس کی زاوی سمتی رفتار تبدیل ہو گی۔ )

\جزوحصہء{کلیدی تصور}
\عددی{\alpha(t)}تعریف کے  رو  سے  \عددی{\omega(t)}  کا وقتی تفرق ہو گا۔یوں، وقت کے لحاظ سے  \عددی{\alpha(t)} کا تکمل \عددی{\omega(t)} دیگا۔

\موٹا{حساب:}\quad
مساوات \حوالہ{مساوات_گھماو_زاوی_لمحاتی_اسراع}  ذیل کہتی ہے
\begin{align*}
\dif \omega=\alpha \dif t
\end{align*}
لہٰذا 
\begin{align*}
\int \dif \omega=\int \alpha \dif t
\end{align*}
ہو گا جو ذیل کے گی، جہاں \عددی{C} تکمل کا مستقل ہے۔
\begin{align*}
\omega=\int (5t^3-4t)\dif t=\frac{5}{4}t^4-\frac{4}{2}t^2+C
\end{align*}
ہم جانتے ہیں \عددی{t=0} پر \عددی{\omega=\SI{5}{\radian\per\second}} ہے؛ اس معلومات کو در ج بالا میں ڈال کر:
\begin{align*}
\SI{5}{\radian\per\second}=0-0+C
\end{align*}
تکمل کا مستقل   \عددی{C=\SI{5}{\radian\per\second}}    حاصل ہو گا۔یوں  درکار  تفاعل ذیل ہو گا۔
\begin{align*}
\omega=\frac{5}{4}t^4-\frac{4}{2}t^2+5\quad\quad \text{\RL{(جواب)}}
\end{align*}
(ب) لٹو کے زاوی مقام \عددی{\theta(t)} کا ریاضی فقرہ تلاش کریں۔

\جزوحصہء{کلیدی تصور}
تعریف کے رو سے \عددی{\theta(t)} کا وقتی تفرق \عددی{\omega(t)} دیگا۔ یوں، وقت کے لحاظ سے \عددی{\omega(t)} کا تکمل \عددی{\theta(t)} دیگا۔

\موٹا{حساب:}\quad
مساوات \حوالہ{مساوات_گھماو_لمحاتی_زاوی_سمتی_رفتار} کے تحت:
\begin{align*}
\dif \theta=\omega \dif t
\end{align*}
ہو گا جس سے ذیل لکھا جا سکتا ہے،
\begin{align*}
\theta&=\int \omega \dif t=\int (\frac{5}{4}t^4-\frac{4}{2}t^2+5)\dif t\\
&=\frac{1}{4}t^5-\frac{2}{3}t^3+5t+C'\\
&=\frac{1}{4}t^5-\frac{2}{3}t^3+5t+2\quad\quad\text{\RL{(جواب)}}
\end{align*}
جہاں \عددی{t=0}  پر \عددی{\theta=\SI{2}{\radian}} جانتے ہوئے \عددی{C'} کی قیمت  حاصل کی گئی۔
\انتہا{نمونی سوال}
%------------------------------
%p264
\حصہء{کیا زاوی مقادیر سمتیات   ہیں؟}
ہم   اکیلے  ذرے  کا مقام، سمتی رفتار، اور اسراع سمتیات سے بیان کر سکتے ہیں۔ اگر ذرہ  صرف ایک  محور پر حرکت کرتا ہو،  سمتی ترقیم استعمال کرنا ضرورت نہیں۔ایسے ذرے کو صرف دو رخ  دستیاب ہیں جنہیں مثبت اور منفی علامت سے ظاہر کیا جا سکتا ہے۔

اسی طرح استوار جسم  قائمہ محور  پر  ، محور کے ہمراہ  دیکھتے ہوئے، صرف خلاف گھڑی اور گھڑی وار   گھوم سکتا ہے۔ان رخ کو ہم مثبت اور منفی سے ظاہر کر سکتے ہیں۔ یہاں ایک سوال اٹھتا ہے: \قول{کیا ہم گھومتے جسم کے زاوی ہٹاو، زاوی سمتی رفتار، اور زاوی اسراع کو سمتیات  سمجھ  سکتے ہیں؟} اس کا جواب ہے \قول{جی ہاں} (زاوی ہٹاو کے  لئے   نیچے  پیش انتباہ  ضرور دیکھیں۔)

\موٹا{زاوی سمتی رفتار۔} زاوی سمتی رفتار کو دیکھیں۔ شکل \حوالہء{10.6a} میں \عددی{\omega=33\tfrac{1}{3}}   چکر فی سیکنڈ  کی   مستقل زاوی رفتار   سے گھڑی وار  رخ  گھومتا ہوا قرص دکھایا گیا ہے۔ جیسا شکل \حوالہء{10.6b} میں دکھایا گیا ہے، ہم اس کی سمتی زاوی  رفتار  گھماو کے محور پر سمتیہ   \عددی{\vec{\omega}} سے ظاہر کر سکتے ہیں۔ اس کا طریقہ کار یوں ہے: سمتیہ کی لمبائی   کسی موزوں پیمانہ کے تحت   رکھی جاتی ہے، مثلاً   \عددی{\SI{1}{\centi\meter}} کو \عددی{10} چکر فی منٹ  کی مطابقت سے رکھ  جا سکتا ہے۔ اس کے بعد \عددی{\vec{\omega}} کا رخ تعین کرنے کے لئے ہم \اصطلاح{ دائیں ہاتھ کا قاعدہ }استعمال کرتے ہیں، جو شکل \حوالہء{10.6c} میں پیش ہے: قرص کو دائیں ہاتھ میں یوں پکڑیں کہ  انگلیاں    \ترچھا{ گھماو کے رخ}  ہوں۔ آپ کا سیدھا کھڑا انگوٹھا  زاوی سمتی رفتار کے سمتیہ کا رخ دیگا۔ اگر قرص مخالف رخ گھومے، دائیں ہاتھ قاعدہ کے تحت \عددی{\vec{\omega}}  بھی  مخالف رخ ہو گا۔

زاوی مقادیر       سمتیات سے ظاہر کرنے کی عادت مشکل سے  ڈلتی ہے۔ ہم فوراً سوچتے ہیں  کہ سمتیہ کے  ہمراہ  کوئی چیز  حرکت  کرے گی۔یہاں  ایسا نہیں ہو گا۔  اس کے بجائے کوئی چیز (جیسا استوار جسم) سمتیہ کے رخ کے  \ترچھا{گرد  } گھومتی ہے۔ خالص گھماو کی دنیا میں،  سمتیہ کا رخ کسی چیز کی حرکت کا رخ نہیں بلکہ  گھماو کی محور دیگا۔ بہرحال، سمتیہ حرکت بھی تعین کرتا ہے۔مزید، یہ   سمتیات  سلجھانے کے  ان تمام قواعد کی تعمیل کرتا ہے جو    باب \حوالہء{3} میں  پیش  کیے گئے۔ زاوی اسراع \عددی{\vec{\alpha}}  بھی ایک  سمتیہ ہے، اور یہ بھی ان قواعد کی تعمیل کرتا ہے۔

اس باب میں صرف   قائمہ محور پر گھماو کی بات کی جائے گی۔ ان میں سمتیات استعمال کرنے کی ضرورت نہیں؛ ہم زاوی سمتی رفتار کو \عددی{\omega} اور زاوی اسراع کو \عددی{\alpha} سے ظاہر کر  کے، خلاف گھڑی گھماو کو مثبت اور گھڑی وار گھماو کو منفی  کی علامت سے ظاہر کر سکتے ہیں۔

\موٹا{زاوی ہٹاو۔} پہلے  انتباہ کی بات کرتے ہیں: زاوی ہٹاو (ماسوائے  انتہائی چھوٹا  ہٹاو) کو سمتیہ سے ظاہر نہیں کیا جا سکتا۔ کیوں نہیں؟ ہم یقیناً اس کے رخ اور قدر کی بات کر سکتے ہیں، جیسا شکل \حوالہء{10.6} میں زاوی سمتی رفتار کے لئے کیا گیا۔ تاہم، سمتیہ سے ظاہر کیے جانے کے قابل ہونے کے لئے ضروری ہے کہ مقدار سمتی جمع کے قواعد پر پورا اترتی  ہو۔ ان قواعد میں ایک قاعدہ کہتا ہے کہ  سمتیات جمع کرتے وقت ان کی ترتیب غیر ضروری ہے۔ زاوی ہٹاو اس قاعدہ پر پورا نہیں اترتا۔

شکل \حوالہء{10.7} میں  دی گئی مثال پر غور کریں۔ ایک کتاب  کو، جو ابتدائی طور پر افقی پڑی ہے، دو مرتبہ \عددی{\SI{90}{\degree}} زاوی ہٹاو سے گزارا گیا ہے؛ ایک مرتبہ شکل \حوالہء{10.7a}   اور دوسری مرتبہ شکل \حوالہء{10.7b} کی طرح۔  دونوں  میں ہٹاو  برابر، لیکن  ترتیب ایک نہیں، اور  آخر میں کتاب  ایک جیسی سمت بند نہیں۔ دوسری مثال لیتے ہیں۔ دایاں ہاتھ لٹکا کر ہتھیلی  ران پر رکھیں۔ کلائی سخت  کر کے،  (1)  بازو   سامنے اتنا اٹھائیں   کہ افقی ہو، (2)  اس کو پورا  دائیں لے جائیں، اور (3) اس کے بعد ہاتھ واپس نیچے ران تک لے جائیں۔ آپ کی ہتھیلی اب سامنے رخ ہو گی۔ اگر آپ یہی عمل الٹ ترتیب سے دہرائیں، آپ کی ہتھیلی  آخر میں کس رخ ہو گی؟ ان مثال سے ہم دیکھتے ہیں کہ زاوی  ہٹاو  کا مجموعہ انہیں جمع کرنے کی   ترتیب پر منحصر ہے، لہٰذا  ہٹاو کو سمتیہ تصور نہیں کیا جا سکتا۔

%-------------------------------------
%sec 10-2 Rotation With Constant Angular Acceleration  p266
\حصہ{مستقل اسراع کے ساتھ گھماو}
\موٹا{مقاصد}\\
اس حصہ کو پڑھنے کے بعد آپ ذیل کے قابل ہوں گے۔
\begin{enumerate}[1.]
\item
مستقل زاوی اسراع کی صورت میں زاوی مقام، زاوی ہٹاو، زاوی سمتی رفتار، زاوی اسراع، اور  گزرے دارانیے کے   تعلق  (جدول \حوالہ{جدول_گھماو_مستقل_اسراع_مساوات}) استعمال کر پائیں گے۔
\end{enumerate}

\موٹا{کلیدی تصور}
\begin{itemize}
\item
مستقل زاوی اسراع  (جس میں  \عددی{\alpha}  مستقل ہو گا) گھماو حرکت کی ایک اہم   خصوصی صورت ہے، جس کی مجرد حرکیات  مساوات  ذیل ہیں۔
\begin{align*}
\omega&=\omega_0+\alpha t\\
\theta-\theta_0&=\omega_0 t+\frac{1}{2}\alpha t^2\\
\omega^2&=\omega_0^2+2\alpha(\theta-\theta_0)\\
\theta-\theta_0&=\frac{1}{2}(\omega+\omega_0)t\\
\theta-\theta_0&=\omega t-\frac{1}{2}\alpha t^2
\end{align*}
\end{itemize}

\جزوحصہء{مستقل زاوی اسراع کا گھماو}
مستقیم حرکت  میں \ترچھا{ مستقل خطی اسراع}  کی حرکت (مثلاً، زمین پر گرتا ہوا جسم) ایک اہم خصوصی صورت ہے۔ جدول \حوالہء{2.1} میں  اس طرح کی حرکت  کو مطمئن کرتی مساوات پیش کی گئیں۔

خالص گھماو میں \ترچھا{ مستقل زاوی اسراع }  ایک اہم خصوصی صورت ہے؛ اس   کو مطمئن کرنے  والی مطابقتی  مساوات  پائی جاتی ہیں۔ ہم انہیں یہاں اخذ نہیں کریں گے، بلکہ مطابقتی خطی مساوات میں  مساوی زاوی متغیرات ڈال کر انہیں پیش کرتے ہیں۔ جدول \حوالہ{جدول_گھماو_مستقل_اسراع_مساوات} میں مساوات کی دونوں فہرست (مساوات \حوالہء{2.11} اور مساوات \حوالہء{2.15} تا مساوات \حوالہء{2.18}؛ مساوات \حوالہ{مساوات_گھماو_زاوی_الف} تا مساوات \حوالہ{مساوات_گھماو_زاوی_ٹ}) پیش کی گئی ہیں۔
%eq 10.12, eq 10.13, eq 10.14, eq 10.15, eq 10.16
\begin{table}
\caption{مستقل خطی اسراع اور مستقل زاوی اسراع کی حرکت کی مساوات}
\label{جدول_گھماو_مستقل_اسراع_مساوات}
\centering
\begin{minipage}{0.45\textwidth}
\begin{align}
&\text{\RL{خطی مساوات}}\notag\\
v&=v_0+at\tag{\setlatin{\حوالہء{2.11}}}\\
x-x_0&=v_0t+\frac{1}{2}at^2\tag{\setlatin{\حوالہء{2.15}}}\\
v^2&=v_0^2+2a(x-x_0)\tag{\setlatin{\حوالہء{2.16}}}\\
x-x_0&=\frac{1}{2}(v_0+v)t\tag{\setlatin{\حوالہء{2.17}}}\\
x-x_0&=vt-\frac{1}{2}at^2\tag{\setlatin{\حوالہء{2.18}}}
\end{align}
\end{minipage}\hfill
\begin{minipage}{0.45\textwidth}
\begin{align}
&\text{\RL{زاوی مساوات}}\notag\\
\omega&=\omega_0+\alpha t\label{مساوات_گھماو_زاوی_الف}\\                                          %eq 10.12
\theta-\theta_0&=\omega_0 t+\frac{1}{2}\alpha t^2 \label{مساوات_گھماو_زاوی_ب}\\
\omega^2&=\omega_0^2+2\alpha(\theta-\theta_0) \label{مساوات_گھماو_زاوی_پ}\\
\theta-\theta_0&=\frac{1}{2}(\omega_0+\omega)t  \label{مساوات_گھماو_زاوی_ت}\\
\theta-\theta_0&=\omega t-\frac{1}{2}\alpha t^2  \label{مساوات_گھماو_زاوی_ٹ}		%eq 10.16
\end{align}
\end{minipage}
\end{table}


یاد رہے مساوات \حوالہء{2.11} اور مساوات \حوالہء{2.15} مستقل خطی اسراع کی بنیادی مساوات ہیں، جن سے  فہرست کی باقی تمام مساوات اخذ کی جا سکتی ہیں۔ اس طرح، مساوات \حوالہ{مساوات_گھماو_زاوی_الف} اور مساوات \حوالہ{مساوات_گھماو_زاوی_ب} مستقل زاوی اسراع کی بنیادی مساوات ہیں، جن سے زاوی مساوات کی فہرست کی باقی تمام مساوات اخذ کی جا سکتی ہیں۔ مستقل زاوی اسراع کا سادہ مسئلہ حل کرنے کے لئے آپ عموماً  زاوی فہرست سے (اگر یہ فہرست آپ کے پاس موجود ہو)    ایک مساوات استعمال کر پائیں گے۔ آپ وہ مساوات منتخب کریں گے جس میں صرف وہ متغیر غیر معلوم ہو جو آپ کو  درکار ہو۔ بہتر طریقہ یہ ہو گا کہ آپ مساوات \حوالہ{مساوات_گھماو_زاوی_الف} اور مساوات \حوالہ{مساوات_گھماو_زاوی_ب} یاد کر لیں اور جب ضرورت پیش آئے، انہیں بطور ہمزاد مساوات حل کریں۔

%--------------------------
\ابتدا{آزمائش}
گھومے جسم کا زاوی مقام \عددی{\theta(t)} چار مختلف صورتوں میں (ا)  \عددی{\theta=3t-4}، (ب) \عددی{\theta=-5t^3+4t^2+6}، (ج) \عددی{\theta=2\!/\!t^2-4\!/\!t}، اور (د) \عددی{\theta=5t^2-3} ہے۔  جدول \حوالہ{جدول_گھماو_مستقل_اسراع_مساوات} کی  زاوی مساوات  کا اطلاق کن صورتوں پر ہو گا؟
\انتہا{آزمائش}
%--------------------------------------

%sample problem 10.03 p267
\ابتدا{نمونی سوال}\موٹا{مستقل زاوی اسراع، چکی کا پاٹ}\\
شکل \حوالہء{10.8} میں  پاٹ  مستقل زاوی اسراع \عددی{\alpha=\SI{0.34}{\radian\per\second\squared}} سے گھوم رہا ہے۔ وقت \عددی{t=0} پر اس کی  زاوی سمتی رفتار  \عددی{\omega_0=\SI{-4.6}{\radian\per\second}} ہے، اور اس پر کھینچی گئی حوالہ لکیر کا مقام \عددی{\theta_0=0} ہے۔

(ا)  وقت \عددی{t=0} سے  کتنی دیر بعد حوالہ لکیر  زاوی مقام \عددی{\theta=5.0}  چکر پر ہو گی؟

\موٹا{کلیدی تصور}\\
چونکہ زاوی  اسراع   مستقل ہے لہٰذا ہم جدول \حوالہ{جدول_گھماو_مستقل_اسراع_مساوات} سے مساوات چن سکتے ہیں۔ ہم مساوات \حوالہ{مساوات_گھماو_زاوی_ب}
\begin{align*}
\theta-\theta_0=\omega_0 t+\frac{1}{2}\alpha t^2
\end{align*}
 کا انتخاب اس لئے  کرتے ہیں کہ اس میں  صرف ایک متغیر، \عددی{t}،  نا معلوم ہے اور ہمیں   یہی درکار ہے۔
 
 \موٹا{حساب:}\quad
 دی گئی معلومات ڈال کر اور \عددی{\theta_0=0} اور \عددی{\theta=\text{\RL{چکر}}\,5.0=10\pi\,\si{\radian}}  لیتے ہوئے  ذیل  ہو گا۔
 \begin{align*}
 10\pi\,\si{\radian}=(\SI{-4.6}{\radian\per\second})t+\frac{1}{2}(\SI{0.35}{\radian\per\second\squared})t^2
 \end{align*}
 (اکائیوں  کے ثبات کی خاطر ہم \عددی{5.0} چکر کو \عددی{10\pi} ریڈیئن میں تبدیل کرتے ہیں۔)  اس  دو درجی الجبرائی مساوات کو حل کرنے سے ذیل حاصل ہو گا۔
 \begin{align*}
 t=\SI{32}{\second}\quad\quad \text{\RL{(جواب)}}
 \end{align*}
 ان ایک عجیب بات پر غور کریں۔ جب ہم پہلی مرتبہ پاٹ  پر نظر ڈالتے ہیں یہ منفی رخ گھوم کر \عددی{\theta=0} سمت بند مقام   سے گزرتا  ہے۔ اس کے باوجود \عددی{\SI{32}{\second}} بعد ہم اسے \عددی{\theta=5.0}   چکر     مثبت سمت بند   مقام پر پاتے ہیں۔ اس دورانیے میں ایسا کیا ہوا کہ پاٹ  مثبت سمت بند مقام پر ہو سکتا ہے؟
 
 (ب)وقت  \عددی{t=0} اور \عددی{t=\SI{32}{\second}} کے بیچ پاٹ کے گھماو پر تبصرہ کریں۔
 
 \موٹا{تبصرہ:}\quad
پاٹ ابتدائی طور پر منفی  (گھڑی وار)    رخ \عددی{\omega_0=\SI{-4.6}{\radian\per\second}}   زاوی  رفتار سے حرکت کرتا ہے، تاہم اس کا زاوی  اسراع \عددی{\alpha} مثبت ہے۔ ابتدائی زاوی رفتار اور زاوی اسراع کی علامتیں الٹ ہونے  کی بدولت  پاٹ  منفی رخ چلتے چلتے  بتدریج آہستہ ہوتے رک کر  مثبت رخ گھومنا شروع کرتا ہے۔ حوالہ لکیر  مثبت رخ چل کر \عددی{\theta=0} مقام سے دوبارہ گزرتی ہے اور \عددی{t=\SI{32}{\second}} گزرنے تک مثبت رخ  مزید  \عددی{5.0} چکر کاٹ چکا ہوتا ہے۔

(ج)پاٹ کس وقت \عددی{t} پر لمحاتی رکتا ہے؟

\موٹا{حساب:}\quad
ہم دوبارہ زاوی مساوات کی فہرست پر نظر ڈالتے ہیں اور ایسی مساوات لینا چاہتے ہیں جس میں صرف  \عددی{t}   نا معلوم متغیر  ہو۔ تاہم، اب مساوات میں \عددی{\omega} کا ہونا بھی ضروری ہے، تا کہ ہم اس کو \عددی{0} لے کر  مطابقتی \عددی{t} کے لئے حل کریں۔ ہم مساوات \حوالہ{مساوات_گھماو_زاوی_الف} منتخب کرتے ہیں، جو ذیل دیگی۔
\begin{align*}
t=\frac{\omega-\omega_0}{\alpha}=\frac{0-(\SI{-4.6}{\radian\per\second})}{\SI{0.35}{\radian\per\second\squared}}=\SI{13}{\second}\quad\quad \text{\RL{(جواب)}}
\end{align*}
\انتہا{نمونی سوال}
%----------------------------

%sample problem 10.04 p267
\ابتدا{نمونی سوال}\موٹا{مستقل زاوی اسراع، پہیے کی  سواری}\\
تفریح گاہ میں  ایک بڑا  پہیا چلاتے ہوئے  آپ کی نظر  پہیے پر سوار ایک شخص  پر پڑتی ہے جو  پریشان نظر آتا ہے۔آپ پہیے کی زاوی سمتی  رفتار مستقل زاوی اسراع  کے ساتھ  \عددی{\SI{3.40}{\radian\per\second}} سے  \عددی{20.0} چکروں میں کم کر کے \عددی{\SI{2.00}{\radian\per\second}}  کرتے ہیں۔ (اس شخص کو \قول{گھومتا  شخص} تصور کرنے سے \قول{مستقیم حرکت کرتا  شخص} کہنا زیادہ بہتر ہو گا۔)

(ا)   زاوی سمتی رفتار کی کمی کے دوران مستقل زاوی اسراع کیا ہو گی؟

\موٹا{کلیدی تصور}\\
پہیے کی زاوی اسراع   مستقل ہے، لہٰذا ہم  اس کی زاوی سمتی رفتار اور زاوی ہٹاو کا تعلق  مستقل زاوی اسراع کی مساوات (مساوات \حوالہ{مساوات_گھماو_زاوی_الف} اور مساوات \حوالہ{مساوات_گھماو_زاوی_ب}) سے  جان  سکتے ہیں۔

\موٹا{حساب:}\quad
آئیں دیکھیں آیا ہم ان بنیادی مساوات کو حل کر پائیں گے۔ ابتدائی زاوی سمتی رفتار \عددی{\omega_0=\SI{3.40}{\radian\per\second}}، زاوی ہٹاو \عددی{\theta-\theta_0=\text{\RL{چکر}}\, 20.0}، اور   ہٹاو کے آخر پر زاوی سمتی رفتار \عددی{\omega=\SI{2.00}{\radian\per\second}} ہے۔ہم مستقل زاوی اسراع \عددی{\alpha} جاننا چاہتے ہیں۔ دونوں مساوات میں وقت \عددی{t} پایا جاتا ہے، جس میں ضروری نہیں ہم دلچسپی رکھتے ہوں۔

نا معلوم \عددی{t} خارج کرنے کے لئے ہم مساوات \حوالہ{مساوات_گھماو_زاوی_الف} سے 
\begin{align*}
t=\frac{\omega-\omega_0}{\alpha}
\end{align*}
لکھ کر مساوات \حوالہ{مساوات_گھماو_زاوی_ب} میں ڈالتے ہیں۔
\begin{align*}
\theta-\theta_0=\omega_0\big(\frac{\omega-\omega_0}{\alpha}\big)+\frac{1}{2}\alpha\big(\frac{\omega-\omega_0}{\alpha}\big)^2
\end{align*}
\عددی{\alpha} کے لئے حل کر کے، دی گئی معلومات پُر کر کے ، اور \عددی{20.0} چکر کو \عددی{\SI{125.7}{\radian}} میں بدل کر ذیل حاصل ہو گا۔
\begin{align*}
\alpha&=\frac{\omega^2-\omega_0^2}{2(\theta-\theta_0)}=\frac{(\SI{2.00}{\radian\per\second})^2-(\SI{3.40}{\radian\per\second})^2}{2(\SI{125.7}{\radian})}\\
&=\SI{-0.0301}{\radian\per\second\squared}\quad\quad\quad\text{\RL{(جواب)}}
\end{align*}
(ب)  رفتار کتنے وقت میں کم کی گئی؟

\موٹا{حساب:}\quad
چونکہ اب ہم \عددی{\alpha} جانتے ہیں، مساوات \حوالہ{مساوات_گھماو_زاوی_الف} استعمال کر کے \عددی{t} حاصل کیا جا سکتا ہے۔
\begin{align*}
t&=\frac{\omega-\omega_0}{\alpha}=\frac{\SI{2.00}{\radian\per\second}-\SI{3.40}{\radian\per\second}}{\SI{-0.0301}{\radian\per\second\squared}}\\
&=\SI{46.5}{\second}\quad\quad\quad\text{\RL{(جواب)}}
\end{align*}
\انتہا{نمونی سوال}
%--------------------------------

%section 10-3 Relating The Linear And Angular Variables p268

\حصہ{خطی اور زاوی متغیرات کا    رشتہ}
\موٹا{مقاصد}\\
اس حصے کو پڑھنے کے بعد آپ ذیل کے قابل ہوں گے۔
\begin{enumerate}[1.]
\item
 قائمہ محور  پر گھومتے ہوئے استوار  جسم کے زاوی  متغیرات   (زاوی مقام، زاوی سمتی رفتار، اور زاوی اسراع) کا  جسم پر ایک  ذرے  ، جو کسی رداس پر پایا جاتا ہو، کے خطی متغیرات (مقام، سمتی رفتار، اور اسراع)  کے ساتھ تعلق  جان پائیں گے۔
 \item
 مماسی اسراع اور رداسی اسراع میں تمیز کر پائیں گے، اور  کسی محور پر گھومتے ہوئے جسم پر موجود ذرے  کے لئے بڑھتی زاوی رفتار اور گھٹتی زاوی رفتار  کی  صورت میں   دونوں کے سمتیہ  بنا پائیں گے۔
\end{enumerate}

\موٹا{کلیدی تصور}\\
\begin{itemize}
\item
گھومتے جسم   پر محور  گھماو  سے عمودی  فاصلہ  \عددی{r} پر  پائے جانے والا  نقطہ، رداس \عددی{r} کے  دائرے پر حرکت کرتا ہے۔ اگر جسم زاویہ \عددی{\theta} گھومے، یہ نقطہ درج ذیل    قوسی فاصلہ \عددی{s} طے کریگا، جہاں \عددی{\theta} ریڈیئن میں ناپا جائے گا۔
\begin{align*}
s=\theta r\quad\quad\quad \text{\RL{(ریڈیئن ناپ)}} 
\end{align*}
\item
اس  نقطے کا خطی سمتی رفتار \عددی{\vec{v}}  دائرے کو مماسی ہو گا؛ نقطے کا  خطی رفتار ذیل ہو گا، جہاں \عددی{\omega}  جسم اور نقطے کا (ریڈیئن فی سیکنڈ)  زاوی رفتار ہے۔
\begin{align*}
v=\omega r\quad\quad \quad \text{\RL{(ریڈیئن ناپ)}}
\end{align*}
\item
اس نقطے کے  خطی اسراع \عددی{\vec{a}} کے دو حصے ہوں گے؛ ایک مماسی  جزو اور دوسرا رداسی جزو۔ مماسی جزو ذیل ہو گا، جہاں  \عددی{\alpha}  جسم کے  (ریڈیئن فی مربع سیکنڈ میں)  زاوی اسراع  کی قدر ہے۔
\begin{align*}
a_{t}=\alpha r \quad\quad \text{\RL{(ریڈیئن ناپ)}}
\end{align*}
رداسی جزو ذیل ہو گا۔
\begin{align*}
a_r=\frac{v^2}{r}=\omega^2r\quad\quad \text{\RL{(ریڈیئن ناپ)}}
\end{align*}
\item
اگر یہ نقطہ یکساں دائری  حرکت کرتا ہو،  اس نقطے اور جسم کا دوری عرصہ \عددی{T} ذیل ہو گا۔
\begin{align*}
T=\frac{2\pi r}{v}=\frac{2\pi}{\omega}\quad\quad\text{\RL{(ریڈیئن ناپ)}}
\end{align*}
\end{itemize}

%p268
\جزوحصہء{خطی اور زاوی متغیرات کا   رشتہ}
محور   گھماو کے گرد دائرے پر مستقل خطی رفتار \عددی{v} کے ساتھ  حرکت کرتے ہوئے  ذرے کی یکساں دائری حرکت  پر حصہ \حوالہء{4.5} میں غور کیا گیا۔ جب استوار جسم  کسی محور پر گھومتا ہے، جسم کا پر ذرہ اپنے ایک دائرے پر  اسی  محور کے گرد گھومتا ہے۔ چونکہ جسم استوار (بلا لچک) ہے، ایسے تمام ذرے  ہم قدم چل کر ایک جتنے وقت میں ایک چکر مکمل کرتے ہیں؛ ان سب کی زاوی رفتار \عددی{w}  برابر  ہے۔

تاہم، ایک ذرہ جتنا محور سے دور ہو گا، اتنا اس کے دائرے کا محیط بڑا ہو گا، لہٰذا اس کی خطی  رفتار  \عددی{v} اتنی زیادہ ہو گی۔ \اصطلاح{گھومنے والے  جھولے }\فرہنگ{گھومنے والا جھولا}\حاشیہب{merry go round}\فرہنگ{merry go round} پر بیٹھ کر آپ  اسے محسوس کر سکتے ہیں۔ مرکز سے جتنے فاصلے پر بھی  آپ   ہوں، آپ کی زاوی رفتار  \عددی{\omega} ایک جتنی ہو گی، تاہم     مرکز سے دور  ہونے پر    آپ کی خطی رفتار \عددی{v}   بڑھے گی۔

ہم  جسم پر کسی مخصوص نقطے کے خطی متغیرات \عددی{s}، \عددی{v}، اور \عددی{a}  اور سی  جسم کے زاوی متغیرات \عددی{\theta}، \عددی{\omega}، اور \عددی{\alpha} کا تعلق جاننا چاہتے ہیں۔ متغیرات کی  ان  فہرست  کا  رشتہ  \ترچھا{ محور گھماو  سے  نقطے کے عمودی فاصلہ } \عددی{r} کے ذریعے ہو گا۔ یہ عمودی فاصلہ ،  نقطے اور محور گھماو کے بیچ  عمودی   لکیر پر ناپا جائے گا۔ یہ فاصلہ اس دائرے کا رداس \عددی{r} ہو گا جس پر محور  گھماو  کے گرد نقطہ  حرکت کرتا ہے۔

%p269
\جزوجزوحصہء{مقام}
اگر استوار جسم پر کھینچی گئی حوالہ لکیر زاویہ \عددی{\theta}  گھومے، محور گھماو سے  \عددی{r}   فاصلے پر موجود جسم کے اندر نقطہ دائری قوس پر فاصلہ \عددی{s} طے کرے گا، جہاں \عددی{s} کی قیمت مساوات \حوالہ{مساوات_گھماو_رداسی_فاصلہ_الف}  دیتی ہے۔
%eq 10.17
\begin{align}\label{مساوات_گھماو_خطی_زاوی_تعلق_الف}
s=\theta r \quad\quad\text{\RL{(ریڈیئن ناپ)}}
\end{align}
مساوات \حوالہ{مساوات_گھماو_خطی_زاوی_تعلق_الف}  ہمارا پہلی  خطی و زاوی تعلق ہے۔\ترچھا{ انتباہ:} زاویہ \عددی{\theta}کا ناپ  ریڈیئن  میں لازمی ہے چونکہ درج بالا مساوات زاویے کے   ریڈیئن  میں ناپ کی تعریف ہے۔

\جزوجزوحصہء{رفتار}
رداس \عددی{r}  کو مستقل رکھ کر وقت کے ساتھ مساوات   \حوالہ{مساوات_گھماو_خطی_زاوی_تعلق_الف} کا  تفرق  ذیل دیگا۔
\begin{align*}
\frac{\dif s}{\dif t}=\frac{\dif \theta}{\dif t} r 
\end{align*}
لیکن، \عددی{\dif s\!/\!\dif t}  نقطے کی خطی  رفتار  (خطی سمتی رفتار  کی قدر)، اور \عددی{\dif\theta\!/\!\dif t}  گھومتے جسم کی  زاوی رفتار \عددی{\omega} ہے۔ یوں ذیل ہو گا۔
%eq 10.18
\begin{align}\label{مساوات_گھماو_خطی_زاوی_تعلق_ب}
v=\omega r \quad\quad\text{\RL{(ریڈیئن ناپ)}}
\end{align}
\ترچھا{انتباہ:} زاوی رفتار \عددی{\omega} لازماً ریڈیئن فی سیکنڈ میں ناپی    جائے  گی۔

استوار جسم  کے تمام اندرونی   نقطے  ایک زاوی رفتار  \عددی{\omega} سے گھومتے ہیں لہٰذا مساوات \حوالہ{مساوات_گھماو_خطی_زاوی_تعلق_ب} کہتی ہے زیادہ رداس \عددی{r} پر واقع نقطے کی خطی رفتار \عددی{v} زیادہ ہو گی۔ شکل \حوالہء{10.9a} ہمیں  یاد دلاتی ہے کہ ہر   نقطے  کی خطی سمتی رفتار ہمیشہ   نقطے کی دائری راہ کو مماسی ہو گی۔

اگر جسم کا زاوی رفتار \عددی{w} مستقل ہو، مساوات \حوالہ{مساوات_گھماو_خطی_زاوی_تعلق_ب} کہتی ہے جسم کے اندر  نقطے کی خطی رفتار  \عددی{v} بھی مستقل ہو گی۔یوں، جسم کے اندر موجود ہر نقطہ  یکساں دائری حرکت کرتا ہے۔استوار جسم کے  ہر اندرونی نقطے کی حرکت  کا دوری   عرصہ \عددی{T} مساوات \حوالہء{4.35}ذیل دیتی ہے۔
%eq 10.19
\begin{align}\label{مساوات_گھماو_دوری_عرصہ_الف}
T=\frac{2\pi r}{v}
\end{align}
اس مساوات کے تحت ، ایک چکر کے  فاصلے  \عددی{2\pi r}    کو اس رفتار سے تقسیم کر کے جس  سے فاصلہ طے کیا جائے ، ایک چکر کا وقت   حاصل ہو گا۔ مساوات \حوالہ{مساوات_گھماو_خطی_زاوی_تعلق_ب} سے \عددی{v}  ڈال کر \عددی{r} منسوخ کر کے ذیل حاصل ہو گا۔
%eq 10.20
\begin{align}\label{مساوات_گھماو_دوری_عرصہ_ب}
T=\frac{2\pi}{\omega}\quad\quad\text{\RL{(ریڈیئن ناپ)}}
\end{align}
یہ معادل مساوات کہتی ہے    ایک چکر کا زاوی فاصلہ ، \عددی{2\pi}  ریڈیئن  ، اس زاوی رفتار سے تقسیم کر کے ، جس سے زاوی  فاصلہ طے کیا جائے، ایک چکر کا وقت حاصل ہو گا۔

\جزوجزوحصہء{اسراع}
رداس \عددی{r} مستقل رکھ کر \عددی{t}  کے لحاظ سے  مساوات \حوالہ{مساوات_گھماو_خطی_زاوی_تعلق_ب} کا  تفرق  ذیل دیگا۔
%eq 10.21
\begin{align}\label{مساوات_گھماو_اسراع_زاوی_الف}
\frac{\dif v}{\dif t}=\frac{\dif \omega}{\dif t}r
\end{align}
یہاں  ہم  ایک پیچیدگی  کا سامنا  کرتے ہیں۔ مساوات \حوالہ{مساوات_گھماو_اسراع_زاوی_الف} کا بایاں ہاتھ \عددی{\dif v\!/\!\dif t} خطی اسراع کے  صرف اس حصے کو ظاہر کرتا ہے جو خطی سمتی رفتار \عددی{\vec{v}} کی\ترچھا{ قدر } \عددی{v}   کی تبدیلی کا ذمہ دار ہے۔ سمتی رفتار \عددی{\vec{v}} کی طرح خطی اسراع کا  یہ حصہ نقطے کی راہ کو مماسی ہو گا۔ہم اسے خطی اسراع کا \ترچھا{  مماسی جزو } \عددی{a_t} کہہ کر ذیل لکھتے ہیں، جہاں \عددی{\alpha=\dif\omega\!/\!\dif t} ہے۔
%eq 10.22
\begin{align}\label{مساوات_گھماو_اسراع_ب}
a_t=\alpha r\quad\quad \text{\RL{(ریڈیئن ناپ)}}
\end{align}
\ترچھا{انتباہ:} مساوات \حوالہ{مساوات_گھماو_اسراع_ب} میں  زاوی اسراع \عددی{\alpha}  کا ریڈیئن  ناپ  میں  ہونا لازم ہے۔
%----------------------------------
%p270
ساتھ ہی، جیسا مساوات \حوالہء{4.34} ہمیں بتاتی ہے، دائری راہ پر گامزن ذرے (یا نقطے) کے خطی اسراع  کا  (رداسی مرکز کے رخ) \ترچھا{ رداسی جزو }\عددی{a_r=\tfrac{v^2}{r}} ہو گا، جو خطی سمتی  رفتار \عددی{\vec{v}} کے \ترچھا{ رخ}  میں تبدیلی کا ذمہ دار ہو گا۔ مساوات \حوالہ{مساوات_گھماو_خطی_زاوی_تعلق_ب} سے \عددی{v} ڈال کر یہ جزو درج ذیل لکھا جا سکتا ہے۔
%eq 10.23
\begin{align}\label{مساوات_گھماو_رداسی_اندر_اسراع_جزو}
a_r=\frac{v^2}{r}=\omega^2 r\quad\quad\text{\RL{(ریڈیئن ناپ)}}
\end{align}
یوں، جیسا شکل \حوالہء{10.9b} میں دکھایا گیا ہے،  استوار گھومتے جسم پر نقطے کے خطی اسراع  کے عموماً دو جزو ہوں گے۔جب بھی  جسم کی زاوی سمتی رفتار غیر صفر ہو،   رداسی اندر  کی طرف کا   جزو  \عددی{a_r}  موجود ہو گا  (جو مساوات \حوالہ{مساوات_گھماو_رداسی_اندر_اسراع_جزو} دیتی ہے)۔ مماسی جزو \عددی{a_t} (جو مساوات \حوالہ{مساوات_گھماو_اسراع_زاوی_الف} دیتی ہے) اس صورت ہو گا جب  زاوی اسراع غیر صفر ہو۔

%---------------------------------
\ابتدا{آزمائش}
گھومنے والے جھولے     کے حلقہ   پر چیونٹی بیٹھی ہے۔ اگر اس  نظام (گھومنا والا جھولا  و چیونٹی)  کی  زاوی سمتی رفتار  مستقل ہو، کیا چیونٹی کا (ا) رداسی اسراع اور (ب) مماسی اسراع ہو گا؟ اگر \عددی{\omega} گھٹ رہی ہو، کیا چیونٹی کا (ج) رداسی اسراع اور (د) مماسی اسراع ہو گا؟
\انتہا{آزمائش}
%-----------------------------------

%Sample problem 10.05 p270
\ابتدا{نمونی سوال}\موٹا{تفریح گاہ میں ایک بڑے  حلقہ کی بناوٹ}\\
ہمیں ایک بڑا افقی   حلقہ ، جس کا رداس \عددی{r=\SI{33.1}{\meter}} ہو گا،  بنانے کو کہا گیا ہے جو انتصابی دھرے پر چلے گا۔(یہ چین میں  موجود دنیا کے سب سے بڑے پہیے جتنا ہو گا۔) سوار حلقے کے  بیرونی  دیوار میں موجود دروازے سے داخل ہو کر اس  دیوار کے ساتھ کھڑے ہوں گے (شکل \حوالہء{10.10a})۔ حلقے پر حوالہ لکیر کا زاوی مقام \عددی{\theta(t)}  لمحہ \عددی{t=0} سے لمحہ   \عددی{t=\SI{2.30}{\second}}  تک ذیل دیتی ہے، جہاں \عددی{c=\SI{6.39e-2}{\radian\per\second\cubed}} ہے۔
\begin{align}
\theta=ct^3
\end{align}
لمحہ \عددی{t=\SI{2.30}{\second}} کے بعد جھولنے  کے پھیرا  مکمل ہونے تک  زاوی رفتار مستقل  رکھی جائے گی۔ گھومنا شروع ہونے کے بعد، سوار کے پاوں تلے فرش ہٹا دی جائے گی، لیکن وہ گرے گا نہیں؛ بلکہ وہ دیوار کے ساتھ مضبوطی سے جکڑا  محسوس کرے گا۔لمحہ \عددی{t=\SI{2.20}{\second}} پر شخص  کی زاوی رفتار \عددی{\omega}، خطی رفتار \عددی{v}، زاوی اسراع \عددی{\alpha}، مماسی اسراع \عددی{a_t}،  رداسی اسراع \عددی{a_r}، اور   اسراع \عددی{\vec{a}} تلاش کرتے ہیں۔

\جزوحصہء{کلیدی تصور}
(1)مساوات \حوالہ{مساوات_گھماو_لمحاتی_زاوی_سمتی_رفتار}    زاوی رفتار \عددی{\omega}  دیتی ہے۔ (2) مساوات \حوالہ{مساوات_گھماو_خطی_زاوی_تعلق_ب} (  دائری راہ پر) خطی رفتار \عددی{v}  اور (محور گھماو کے گرد)   زاوی رفتار \عددی{\omega}    کا تعلق \عددی{v=\omega r} دیتی ہے۔ (3)مساوات \حوالہ{مساوات_گھماو_زاوی_لمحاتی_اسراع}    زاوی اسراع \عددی{\alpha}  دیتی ہے \عددی{(\alpha=\dif \omega\!/\!\dif t)}۔ (4)  مساوات \حوالہ{مساوات_گھماو_اسراع_ب} (دائری راہ کے ہمراہ) مماسی اسراع \عددی{\alpha_t} اور  (محور گھماو کے گرد) زاوی اسراع \عددی{\alpha} کا تعلق \عددی{(a_t=\alpha r)}  دیتی ہے۔ (5)  مساوات \حوالہ{مساوات_گھماو_رداسی_اندر_اسراع_جزو} رداسی اسراع \عددی{(a_r=\omega^2 r)} دیتی ہے۔  (6)  مماسی اور رداسی اسراع   پورے اسراع \عددی{\vec{a}} کے دو آپس میں عمودی جزو ہیں۔

\موٹا{حساب:}\quad
آئیں  ان اقدام سے گزریں۔دیے گئے  زاوی مقام  تفاعل  کا وقتی تفرق لے کر \عددی{t=\SI{2.20}{\second}}   پُر کر کے زاوی سمتی رفتار معلوم کرتے ہیں۔
\begin{gather}
\begin{aligned}\label{مساوات_گھماو_نمونی_زاوی_الف}
\omega&=\frac{\dif \theta}{\dif t}=\frac{\dif}{\dif t}(ct^3)=3ct^2\\
&=3(\SI{6.39e-2}{\radian\per\second\cubed})(\SI{2.20}{\second})^2\\
&=\SI{0.928}{\radian\per\second}\quad\quad\text{\RL{(جواب)}}
\end{aligned}
\end{gather} 
مساوات \حوالہ{مساوات_گھماو_خطی_زاوی_تعلق_ب}  اس لمحے کی ذیل خطی رفتار دیگی۔
\begin{gather}
\begin{aligned}\label{مساوات_گھماو_نمونی_مثال_رفتار_الف}
v&=\omega r=3ct^2 r\\
&=3(\SI{6.39e-2}{\radian\per\second\cubed})(\SI{2.20}{\second})^2(\SI{33.1}{\meter})\\
&=\SI{30.7}{\meter\per\second}\quad\quad\text{\RL{(جواب)}}
\end{aligned}
\end{gather}
اگرچہ یہ  رفتار    \عددی{(\SI{111}{\kilo\meter\per\hour})}  تیز ہے، ایسی رفتار تفریح گاہوں میں عام ہیں، اور خطرے کا باعث نہیں ؛ (جیسا باب \حوالہء{2} میں ذکر کیا گیا) ہمارا  جسم اسراع کو ردعمل کرتا ہے، خطی رفتار کو نہیں (ہم   رفتار پیما نہیں سرعت پیما ہیں)۔ مساوات \حوالہ{مساوات_گھماو_نمونی_مثال_رفتار_الف} کہتی ہے  خطی رفتار  ، وقت کے مربع  کے ساتھ بڑھے گی( تاہم یہ اضافہ \عددی{t=\SI{2.20}{\second}} پر رک جائے گا)۔

اس کے بعد، مساوات \حوالہ{مساوات_گھماو_نمونی_زاوی_الف} کا وقت تفرق لے کر زاوی اسراع معلوم کرتے ہیں۔
\begin{align*}
\alpha&=\frac{\dif \omega}{\dif r}=\frac{\dif}{\dif t}(3ct^2)=6ct\\
&=6(\SI{6.39e-2}{\radian\per\second\cubed})(\SI{2.20}{\second})=\SI{0.843}{\radian\per\second\squared}\quad\text{\RL{(جواب)}}
\end{align*}
اب مساوات \حوالہ{مساوات_گھماو_اسراع_ب} مماسی اسراع \عددی{a_t} دیگی:
\begin{gather}
\begin{aligned}\label{مساوات_گھماو_نمونی_بڑا_پہیا_مماسی}
a_t&=\alpha r=6ctr\\
&=6(\SI{6.39e-2}{\radian\per\second\cubed})(\SI{2.20}{\second})(\SI{33.1}{\meter})\\
&=\SI{27.91}{\meter\per\second\squared}\approx\SI{27.9}{\meter\per\second\squared}\quad\quad\text{\RL{(جواب)}}
\end{aligned}
\end{gather}
جو \عددی{2.8g}، جہاں \عددی{g=\SI{9.8}{\meter\per\second\squared}} ہے، کے برابر ہے (جو  مناسب  ہے اور   پُر لطف ہو گا)۔ مساوات \حوالہ{مساوات_گھماو_نمونی_بڑا_پہیا_مماسی} کہتی ہے مماسی اسراع وقت کے ساتھ بڑھ رہا ہے (تاہم یہ اضافہ \عددی{t=\SI{2.30}{\second}} پر رک جائے گا)۔ مساوات \حوالہ{مساوات_گھماو_رداسی_اندر_اسراع_جزو} سے رداسی اسراع لکھتے  کر:
\begin{align*}
a_r=\omega^2 r
\end{align*}
مساوات \حوالہ{مساوات_گھماو_نمونی_زاوی_الف} سے   \عددی{\omega=3ct^2} ڈالتے ہیں:
\begin{gather}
\begin{aligned}\label{مساوات_گھماو_نمونی_بڑا_پہیا_رداسی}
a_r&=(3ct^2)^2r=9c^2t^4r\\
&=9(\SI{6.39e-2}{\radian\per\second\cubed})^2(\SI{2.20}{\second})^4(\SI{33.1}{\meter})\\
&=\SI{28.49}{\meter\per\second\squared}\approx\SI{28.5}{\meter\per\second\squared}\quad\quad\text{\RL{(جواب)}}
\end{aligned}
\end{gather}
جو \عددی{2.9 g} دیتا ہے (یہ بھی مناسب  ہے اور پُر لطف ہو گا)۔

رداسی اور مماسی اسراع ایک دوسرے کو عمودی ہیں اور سوار کے اسراع \عددی{\vec{a}} کے  جزو  ہیں (شکل \حوالہء{10.10b})۔ اسراع \عددی{\vec{a}} کی قدر ذیل ہو گی:
\begin{gather}
\begin{aligned}
a&=\sqrt{a_r^2+a_t^2}\\
&=\sqrt{(\SI{28.49}{\meter\per\second\squared})^2+(\SI{27.91}{\meter\per\second\squared})^2}\\
&\approx\SI{39.9}{\meter\per\second\squared}\quad\quad\text{\RL{(جواب)}}
\end{aligned}
\end{gather}
جو \عددی{4.1g} کے برابر ہے (یہ یقیناً پُر لطف ہو گ!)۔ یہ تمام مقادیر مناسب  ہیں۔

اسراع \عددی{\vec{a}} کی سمت بندی  جاننے کے لئے ہم زاویہ \عددی{\theta} معلوم کرتے ہیں (شکل \حوالہء{10.10b})۔
\begin{align*}
\tan\theta=\frac{a_t}{a_r}
\end{align*}
آئیں اعدادی نتائج پُر کرنے کی  بجائے  ہم مساوات \حوالہ{مساوات_گھماو_نمونی_بڑا_پہیا_مماسی} اور مساوات \حوالہ{مساوات_گھماو_نمونی_بڑا_پہیا_رداسی}  کے الجبرائی نتائج استعمال کرتے ہیں۔
\begin{align}
\theta=\tan^{-1}\big(\frac{6ctr}{9c^2t^4r}\big)=\tan^{-1}\big(\frac{2}{3ct^3}\big)
\end{align}
ریاضی نتیجے کا بڑا فائدہ یہ ہے کہ  ہم اب دیکھ سکتے ہیں کہ   (1) زاویے پر رداس کا کوئی اثر نہیں ہو گا اور  (2)  اس کی قیمت \عددی{t} کی قیمت \عددی{0} تا \عددی{\SI{2.20}{\second}}  بڑھانے سے  گھٹتی ہے۔ رداسی اسراع  (جو \عددی{t^4} پر منحصر ہے ) بہت جلد مماسی اسراع( جو صرف  \عددی{t} پر منحصر ہے) پر غالب ہو کر  سمتیہ اسراع \عددی{\vec{a}} کو رداسی رخ موڑتا ہے۔ وقت \عددی{t=\SI{2.20}{\second}} پر ذیل ہو گا۔
\begin{align*}
\theta=\tan^{-1}\frac{2}{3(\SI{6.39e-2}{\radian\per\second\cubed})(\SI{2.20}{\second})^3}=\SI{44.4}{\degree}\quad\quad\text{\RL{(جواب)}}
\end{align*}
\انتہا{نمونی سوال}
%---------------------------

\حصہ{گھماو کی حرکی توانائی}
\موٹا{مقاصد}\\
اس حصہ کو پڑھنے کے بعد آپ درج ذیل کے قابل ہوں گے۔
\begin{enumerate}[1.]
\item
ذرے کا گھمیری جمود  نقطہ  پر   تلاش  کر پائیں گے۔
\item
 قائمہ  محور کے گرد گھومتے ہوئے متعدد ذروں کا کل  گھمیری جمود تلاش کر پائیں گے۔
 \item
 گھمیری جمود اور زاوی رفتار کی صورت میں جسم کی  گھمیری حرکی توانائی  تعین کر پائیں گے۔
\end{enumerate}

\موٹا{کلیدی تصور}\\
\begin{itemize}
\item
قائمہ محور پر گھومتے  استوار جسم کی حرکی توانائی \عددی{K} ذیل ہو گی، 
\begin{align*}
K=\frac{1}{2}I\omega^2\quad\quad\text{\RL{(ریڈیئن ناپ)}}
\end{align*}
جہاں \عددی{I} جسم کا  گھمیری جمود  کہلاتا ہے، جس کی تعریف   انفرادی ذروں کے نظام کے لئے درج ذیل ہے۔
\begin{align*}
I=\sum m_ir_i^2
\end{align*}
\end{itemize}

\جزوحصہء{گھماو کی حرکی توانائی}
میز آرا  کا تیزی سے گھومتا دھار دار   پھل یقیناً  گھومنے کی بنا حرکی توانائی رکھتا ہے۔ ہم اس  توانائی کو کس طرح  بیان کر سکتے ہیں؟  ہم توانائی کے عمومی کلیہ \عددی{K=\tfrac{1}{2}mv^2}  سے پورے آرا کی حرکی توانائی حاصل نہیں کر سکتے چونکہ یہ آرے کے مرکز کمیت کی حرکی توانائی دیگا، جو صفر ہے۔

اس کے بجائے، میز آرا (اور  کسی بھی دوسرے گھومتے استوار جسم) کو  ہم مختلف رفتار سے حرکت کرتے ذروں کا مجموعہ تصور کرتے ہیں۔ ان ذروں کی انفرادی حرکی توانائیاں جمع کر کے پورے جسم کی حرکی توانائی حاصل کی جا سکتی ہے۔ یوں گھومتے جسم کی حرکی توانائی ذیل ہوگی،
%eq 10.31
\begin{align}
K&=\frac{1}{2}m_1v_1^2+\frac{1}{2}m_2v_2^2+\frac{1}{2}m_3v_3^2+\cdots\notag\\
&=\sum\frac{1}{2}m_iv_i^2 \label{مساوات_گھماو_حرکی_توانائی_الف}
\end{align}
جہاں \عددی{i} ویں ذرے کی کمیت \عددی{m_i} اور رفتار \عددی{v_i} ہے۔ مجموعہ جسم کے تمام ذروں پر لیا جائے گا۔

مساوات \حوالہ{مساوات_گھماو_حرکی_توانائی_الف} میں مشکل یہ ہے کہ ہر ذرے کی رفتار دوسرے سے مختلف ہو سکتی ہے۔ اس مشکل سے بچنے کی خاطر ہم مساوات \حوالہ{مساوات_گھماو_خطی_زاوی_تعلق_ب} سے  \عددی{v=\omega r} ڈال کر ذیل لکھتے ہیں، جس میں \عددی{\omega} تمام ذروں کے لئے  برابر ہے۔
%eq 10.32
\begin{align}\label{مساوات_گھماو_جمود_الف}
K=\sum\frac{1}{2}m_i(\omega r_i)^2=\frac{1}{2}\big(\sum m_ir_i^2\big)\omega^2
\end{align}

مساوات \حوالہ{مساوات_گھماو_جمود_الف} میں  دائیں ہاتھ قوسین میں بند مقدار  ، محور گھماو کے لحاظ سے گھومتے جسم  کی کمیت کی تقسیم پیش کرتی ہے۔ یہ مقدار، محور گھماو کے لحاظ سے گھومتے جسم کا \اصطلاح{ گھمیری جمود }\فرہنگ{جمود!گھمیری}\حاشیہب{rotational inertia}\فرہنگ{inertia!rotational}(یا   \اصطلاح{جمودی معیار اثر}\فرہنگ{معیار اثر!جمودی}\حاشیہب{moment of inertia}\فرہنگ{inertia!moment of})  کہلاتا ہے ، جس کو ہم \عددی{I} سے ظاہر کرتے ہیں۔ محور گھماو کے لحاظ سے جسم کے \عددی{I} کی قیمت   اٹل ہو گی ۔ (\ترچھا{انتباہ:}  \عددی{I} کی قیمت صرف اس صورت  با معنی ہو گی جب  اس محور کا ذکر کیا جائے۔)  کسی دوسری  محور گھماو پر اسی جسم کا \عددی{I} عموماً  مختلف  ہو گا، تاہم اب بھی اس کی قیمت مستقل ہو گی۔

ہم ذیل لکھ  کر،
%eq 10.33
\begin{align}\label{مساوات_گھماو_آئے_تعریف}
I=\sum m_ir_i^2\quad\quad\text{\RL{(گھمیری جمود)}}
\end{align}
مساوات \حوالہ{مساوات_گھماو_جمود_الف} میں ڈال کر  مطلوبہ تعلق:
%eq 10.34
\begin{align}\label{مساوات_گھماو_حرکی_گھمیری_تعریف}
K=\frac{1}{2}I\omega^2\quad\quad\text{\RL{(ریڈیئن ناپ)}}
\end{align}
حاصل کرتے ہیں۔چونکہ \عددی{v=\omega r} استعمال کر کے درج بالا تعلق حاصل کیا گیا لہٰذا  \عددی{\omega} کی قیمت ریڈیئن ناپ میں لکھنی ضروری ہے۔ جمودی معیار اثر  \عددی{I} کی اکائی کلوگرام مربع میٹر  \عددی{(\si{\kilo\gram\meter\squared})} ہے۔

\موٹا{طریقہ کار۔}اگر جسم چند ذروں پر مشتمل  ہو، ہم ہر ذرے کی انفرادی حرکی توانائی \عددی{mr^2}   تلاش کر کے تمام کا مجموعہ، مساوات \حوالہ{مساوات_گھماو_آئے_تعریف}  کی طرح، لے کر جسم کا  کل گھمیری جمود \عددی{I} معلوم کر سکتے ہیں۔ جسم کی کل گھمیری حرکی توانائی جاننے  کے لئے    معلوم شدہ \عددی{I} کو   مساوات \حوالہ{مساوات_گھماو_حرکی_گھمیری_تعریف} میں ڈالنا ہو گا۔ چند ذروں کے لئے یہ  طریقہ کار استعمال کیا جائے گا؛   اگر جسم  میں ذروں کی تعداد بہت زیادہ  ہو (جیسا ایک سلاخ میں ہو گا) تب  کیا ہو گا؟ اگلے حصے میں ہم  اس  قسم کے \ترچھا{ استمراری اجسام  } کو نپٹنا سیکھیں گے؛ فکر  مت کریں، نتائج منٹوں میں حاصل ہوں گے۔

مساوات \حوالہ{مساوات_گھماو_حرکی_گھمیری_تعریف} جو  خالص گھماو کی صورت میں استوار جسم کی حرکی توانائی  \عددی{K=\tfrac{1}{2}I\omega^2}دیتی ہے،  خالص مستقیم حرکت کی صورت میں حرکی توانائی  کلیہ \عددی{K=\tfrac{1}{2}Mv_{\text{\RL{مرکزکمیت}}}^2} کی  زاوی  معادل  ہے۔ دونوں کلیوں میں \عددی{\tfrac{1}{2}} جزو ضربی پایا جاتا ہے۔ ایک کلیہ میں کمیت \عددی{M}  جبکہ دوسرے میں \عددی{I} (جس میں کمیت اور  کمیت  کی تقسیم  دونوں شامل ہیں)  پایا جاتا ہے۔ساتھ ہی دونوں کلیوں میں رفتار کا مربع پایا جاتا ہے (ایک میں مستقیم اور دوسرے میں زاوی )۔ مستقیم اور زاوی حرکت کی حرکی توانائی دو مختلف توانائیاں نہیں۔ دونوں حرکی توانائی ہے، تاہم مسئلہ دیکھ کر موزوں صورت اپنائی   گئی ہے۔

ہم پہلے کہہ چکے ہیں  کہ گھومتے  جسم کا گھمیری جمود نا صرف کمیت بلکہ کمیت کی تقسیم پر بھی منحصر ہو گا۔ آئیں ایک ایسی مثال دیکھیں جس کو آپ حقیقتاً محسوس کر  سکتے ہیں۔ ایک   لمبی   اور بھاری  سلاخ ، پہلے   طولی  محور پر (شکل \حوالہء{10.11a})    اور اس کے  بعد  وسطی نقطہ سے گزرتی    اور سلاخ کو عمودی   محور  پر  (شکل \حوالہء{10.11b}) گھمائیں۔  دونوں صورتوں  میں کمیت  ایک جتنی ہے، تاہم  پہلی صورت میں گھمانا زیادہ  آسان ہو گا۔پہلی صورت میں  کمیت  کی تقسیم محور گھماو کے  زیادہ قریب ہے؛ یوں شکل \حوالہء{10.11a} میں سلاخ کا گھمیری جمود شکل \حوالہء{10.11b} سے کم ہو گا جس کی بدولت شکل \حوالہء{10.11a} میں گھمانا زیادہ آسان ہو گا۔ کم گھمیری جمود  کی صورت میں گھمانا زیادہ آسان ہو گا۔

\ابتدا{آزمائش}
تین کرہ انتصابی محور کے گرد گھومتے شکل میں دکھائے گئے ہیں۔ہر کمیت کے مرکز سے محور تک عمودی  فاصلہ بھی دیا گیا ہے۔ اس محور پر گھمیری جمود کے لحاظ سے کمیتوں کی درجہ بندی کریں۔ زیادہ قیمت اول رکھیں۔
\begin{center}
\begin{tikzpicture}
\pgfmathsetmacro{\ksepX}{1}
\pgfmathsetmacro{\ksepY}{0.75}
\pgfmathsetmacro{\kradA}{4*(4)^(1/3)}
\pgfmathsetmacro{\kradB}{4*(9)^(1/3)}
\pgfmathsetmacro{\kradC}{4*(36)^(1/3)}
\draw[thick](0,0)--++(0,0.5+2*\ksepY)node[pos=0.35,pin={135:{\text{\RL{محور گھماو}}}}]{};
\draw[](3*\ksepX,0.25)node[circle,draw,inner sep=0pt,minimum width=\kradA,fill](aa){};
\draw[](2*\ksepX,0.25+\ksepY)node[circle,draw,inner sep=0pt,minimum width=\kradB,fill](bb){};
\draw[](\ksepX,0.25+2*\ksepY)node[circle,draw,inner sep=0pt,minimum width=\kradC,fill](cc){};
\draw(aa.east)node[right]{\(\SI{4}{\kilo\gram}\)};
\draw(bb.east)node[right]{\(\SI{9}{\kilo\gram}\)};
\draw(cc.east)node[right]{\(\SI{36}{\kilo\gram}\)};
\draw(aa)--++(-3*\ksepX,0)node[pos=0.5,above]{\(\SI{3}{\meter}\)};
\draw(bb)--++(-2*\ksepX,0)node[pos=0.5,above]{\(\SI{2}{\meter}\)};
\draw(cc)--++(-1*\ksepX,0)node[pos=0.5,above]{\(\SI{1}{\meter}\)};
\end{tikzpicture}
\end{center}
\انتہا{آزمائش}
%-------------------------

% section 10.5 p 273
\حصہ{گھمیری جمود کا حساب}
\موٹا{مقاصد}\\
اس حصے کو پڑھنے کے بعد آپ ذیل کے قابل ہوں گے۔
\begin{enumerate}[1.]
\item
ان اجسام کا گھمیری جمود معلوم کر پائیں گے جو جدول \حوالہء{10.1} میں دیے گئے ہیں۔
\item
جسم کے کمیتی ٹکڑوں پر تکمل لے کر جسم کا گھمیری جمود تلاش کر پائیں گے۔
\item
جسم کے  مرکز  کمیت سے گزرتی محور گھماو  سے ہٹ کر  متوازی محور  کے لئے متوازی محور مسئلے کا اطلاق کر پائیں گے۔
\end{enumerate}

\موٹا{کلیدی تصورات}\\
\begin{itemize}
\item
انفرادی ذروں پر مشتمل جسم کے  گھمیری جمود  کی تعریف  :
\begin{align*}
I=\sum m_ir_i^2
\end{align*}
اور جس  جسم میں کمیت کی تقسیم استمراری ہو  ذیل ہے۔
\begin{align*}
I=\int r^2\dif m
\end{align*}
انفرادی ذرے  کا محور گھماو سے عمودی فاصلہ \عددی{r_i}  ہے۔  اسی طرح تکمل میں  کمیت کے ٹکڑے کا محور گھماو سے عمودی فاصلہ  \عددی{r} ہے اور تکمل پورے جسم پر لیا جاتا ہے تا کہ کمیت  کے  تمام ٹکڑے شامل کیے جائیں۔
\item
کسی بھی محور پر   جسم کے گھمیری جمود  \عددی{I}  اور    مرکز کمیت سے گزرتی متوازی محور پر  اسی جسم کے گھمیری جمود   کا تعلق:
\begin{align*}
I=I_{\text{\RL{مرکزکمیت}}}+Mh^2
\end{align*}
 مسئلہ متوازی محور  دیتا ہے۔ دو محوروں کے بیچ عمودی فاصلہ \عددی{h} ہے، اور  مرکز کمیت سے گزرتی محور گھماو پر  جسم کا گھمیری جمود \عددی{I_{\text{\RL{مرکزکمیت}}}} ہے۔    مرکز کمیت سے گزرتی محور گھماو سے جتنا  دور  اصل محور گھماو ہٹائی گئی، ، ہم \عددی{h} کو وہ فاصلہ تصور کر سکتے ہیں۔
\end{itemize}

\جزوحصہء{گھمیری جمود کا حساب}
چند ذروں پر مشتمل استوار جسم کا گھمیری جمود ، محور گھماو پر، مساوات \حوالہ{مساوات_گھماو_آئے_تعریف}  \عددی{(I=\sum m_ir_i^2)}   دیتی ہے؛ یوں ہم ہر ذرے کا \عددی{mr^2} تلاش کر کے تمام کا مجموعہ لیتے ہیں۔(یاد رکھیں کہ محور گھماو سے ذرے کا  عمودی فاصلہ  \عددی{r} ہو گا۔)

اگر جسم  قریب قریب  انتہائی زیادہ ذروں پر مشتمل ہو (جسم\ترچھا{ استمراری } ہو گا)، مساوات \حوالہ{مساوات_گھماو_آئے_تعریف} کا استعمال بہت لمبا کام ہو  گا  جس کے لئے  کمپیوٹر  درکار ہو گا۔ بہتر یہ ہو گا،  ہم  مساوات \حوالہ{مساوات_گھماو_آئے_تعریف} کے مجموعہ کی جگہ تکمل لے کر گھمیری جمود کی تعریف ذیل کریں۔
%eq 10.35
\begin{align}\label{مساوات_گھماو_گھمیری_جمود_استمراری_الف}
I=\int r^2\dif m\quad\quad\text{\RL{(گھمیری جمود، استمراری جسم)}}
\end{align}
جدول \حوالہء{10.2} میں  عام شکل و صورت  کے نو  اجسام  کے لئے ، تکمل کے نتائج پیش کیے گئے ہیں اور مستمل محور گھماو کی نشاندہی کی گئی ہے۔

\جزوجزوحصہء{مسئلہ متوازی محور}
فرض کریں ہم  دی گئی محور گھماو پر ایک جسم کا، جس کی کمیت \عددی{M} ہو،   گھمیری جمود \عددی{I}  جاننا چاہتے ہیں۔ یقیناً، ہم مساوات \حوالہ{مساوات_گھماو_گھمیری_جمود_استمراری_الف}  کے تکمل  سے   \عددی{I} حاصل کر سکتے ہیں۔ تاہم، جسم کے مرکز کمیت سے گزرتی ایسی محور گھماو ، جو  دی گئی محور کے\ترچھا{ متوازی  } ہو، پر   گھمیری جمود \عددی{I_{\text{\RL{مرکزکمیت}}}}  جانتے ہوئے  ، ایک آسان راستہ اختیار کیا جا سکتا ہے۔ مرکز کمیت سے گزرتی محور گھماو  اور دی گئی محور کے بیچ عمودی فاصل \عددی{h} ہونے کی صورت میں (یاد رہے، دونوں محور آپس میں متوازی ہیں) دی گئی محور پر گھمیری جمود \عددی{I} ذیل ہو گا۔
%eq 10.36
\begin{align}\label{مساوات_گھماو_مسئلہ_متوازی_محور}
I=I_{\text{\RL{مرکزکمیت}}}+Mh^2\quad\quad\text{\RL{(مسئلہ متوازی محور)}}
\end{align}
یوں  تصور کریں  جیسا مرکز کمیت سے گزرتی محور گھماو کو دور ہٹا کر \عددی{h} فاصلے پر  رکھا گیا ہے۔ یہ مساوات \اصطلاح{مسئلہ متوازی محور} \فرہنگ{مسئلہ!متوازی محور}\حاشیہب{parallel axis theorem}\فرہنگ{theorem!parallel axis} کہلاتی ہے۔

\جزوجزوحصہء{مسئلہ متوازی محور کا ثبوت}
شکل \حوالہء{10.12} میں اختیاری شکل و صورت   جسم کا، جس  کا مرکز کمیت \عددی{O} ہے،  عمودی تراش دکھایا گیا ہے۔ محددی نظام کا مبدا \عددی{O} پر رکھیں۔شکل کے   مستوی  کو عمودی، \عددی{O} سے گزرتی   ، ایک  محور  لیں؛  اس محور کو متوازی، نقطہ \عددی{P} سے گزرتی ، دوسری محور لیں۔ نقطہ \عددی{P} کے  محدد \عددی{a} اور \عددی{b} ہیں۔

فرض کریں   کسی عمومی  محدد \عددی{x} اور \عددی{y} پر  \عددی{\dif m} کمیت کا چھوٹا ٹکڑا ہے۔ نقطہ \عددی{P} پر محور کے لحاظ سے جسم  کا گھمیری جمود  مساوات \حوالہ{مساوات_گھماو_گھمیری_جمود_استمراری_الف} کے تحت ذیل ہو گا،
\begin{align*}
I=\int r^2\dif m=\int [(x-a)^2+(y-b)^2]\dif m
\end{align*}
جس کو ترتیب نو کے بعد ذیل لکھا جا سکتا  ہے۔
%eq 10.37
\begin{align}\label{مساوات_گھماو_جمود_ثبوت_الف}
I=\int (x^2+y^2)\dif m-2a\int x\dif m-2b\int y\dif m+\int(a^2+b^2)\dif m
\end{align}
مرکز کمیت کی تعریف (مساوات \حوالہء{9.9}) کہتی ہے، مساوات \حوالہ{مساوات_گھماو_جمود_ثبوت_الف}  کے درمیانے دو تکمل  مرکز کمیت  (ایک مستقل سے ضرب کر کے) دیتے ہیں، لہٰذا یہ تکمل (انفرادی طور پر)  صفر کے برابر ہوں گے۔چونکہ \عددی{O} سے \عددی{\dif m} تک فاصلہ  \عددی{R} ہے جو \عددی{x^2+y^2}  کے برابر ہے لہٰذا پہلا تکمل \عددی{I_{\text{\RL{مرکزکمیت}}}} دیگا۔ شکل \حوالہء{10.12} کو دیکھ کر ہم جانتے ہیں  مساوات  \حوالہ{مساوات_گھماو_جمود_ثبوت_الف}  کا آخری تکمل \عددی{Mh^2} کے برابر ہے، جہاں جسم کی کل کمیت \عددی{M} ہے۔ یوں مساوات \حوالہ{مساوات_گھماو_جمود_ثبوت_الف} تخفیف کے بعد مساوات \حوالہ{مساوات_گھماو_مسئلہ_متوازی_محور} دیتی ہے، جسے ہم ثابت کرنا چاہتے تھے۔

%-----------------------------
%Checkpoint 5 p275
\ابتدا{آزمائش}
شکل  \حوالہء{??} میں کتاب  کی طرح جسم   (جس کا ایک ضلع دوسرے سے لمبا ہے) اور جسم   کے رخ کو عمودی  چار  ممکنہ محور گھماو  دکھائے گئے ہیں۔ جسم کے گھمیری جمود کے لحاظ سے، اعظم  قیمت اول رکھ کر،  ان محور کی درجہ بندی کریں۔
\انتہا{آزمائش}
%-----------------------------

%Sample Problem 10.06 p275
\ابتدا{نمونی سوال} \موٹا{دو ذروی جسم کا گھمیری جمود}\\
شکل \حوالہء{10.13a} میں  کمیت \عددی{m} کے دو ذروں پر مشتمل استوار جسم دکھایا گیا ہے۔قابل نظر انداز کمیت  کا سلاخ ، جس کی لمبائی \عددی{L} ہے  کمیتوں کے بیچ  لگا ہے۔

(ا) سلاخ کو عمودی، جسم کے مرکز کمیت سے گزرتی محور گھماو (جیسا شکل میں دکھایا گیا ہے)  پر جسم کا گھمیری جمود کیا ہو گا؟

\جزوحصہء{کلیدی تصور}
جسم صرف دو ذروں پر                                                      (جن کی کمیت ہے)  مشتمل ہے، لہٰذا ہم  تکمل کے  بجائے  مساوات \حوالہ{مساوات_گھماو_آئے_تعریف} استعمال کر کے گھمیری جمود \عددی{I_{\text{\RL{مرکزکمیت}}}} تلاش کر سکتے ہیں۔ہم انفرادی کمیت کا گھمیری جمود تلاش کر کے دونوں کا مجموعہ لیں گے۔

\موٹا{حساب:}\quad
محور گھماو سے \عددی{\tfrac{1}{2}L}  عمودی فاصلے پر کمیت \عددی{m} کے   دو ذروں کا  (مجموعی) گھمیری جمود ذیل ہو گا۔
\begin{align*}
I&=\sum m_ir_i^2=(m)(\tfrac{1}{2}L)^2+(m)(\tfrac{1}{2}L)^2\\
&=\tfrac{1}{2}mL^2\quad\quad\quad\text{\RL{(جواب)}}
\end{align*}
(ب)پہلی محور کو متوازی،  سلاخ کے بائیں سر   سے گزرتی، محور گھماو (شکل \حوالہء{10.13b})  پر جسم کا گھمیری جمود کیا ہو گا؟

\جزوحصہء{کلیدی تصورات}
اتنی آسان صورت میں \عددی{I} باآسانی دونوں طریقوں سے معلوم کیا جا سکتا ہے۔ پہلا طریقہ جزو ا کی طرح ہے۔ دوسرا، زیادہ طاقتور طریقہ مسئلہ متوازی محور استعمال کرتا ہے۔

\موٹا{پہلا طریقہ:}\quad
ہم جزو ا کی طرح \عددی{I} معلوم کرتے ہیں، تاہم اب سلاخ کے بائیں سر پر موجود ذرے کا\عددی{r_i} صفر اور دائیں سر پر ذرے کا \عددی{L} ہو گا۔ مساوات \حوالہ{مساوات_گھماو_آئے_تعریف} اب ذیل دیگی۔
\begin{align*}
I=m(0)^2+m(L)^2=mL^2\quad\quad\text{\RL{(جواب)}}
\end{align*}
\موٹا{دوسرا طریقہ:}\quad
ہم مرکز کمیت سے گزرتی محور گھماو پر جسم کا گھمیری جمود جانتے ہیں اور دوسرا محور   مرکز کمیت سے گزرتی  محور کو  متوازی ہے لہٰذا مسئلہ متوازی محور (مساوات \حوالہ{مساوات_گھماو_مسئلہ_متوازی_محور}) بروئے کار لایا جا سکتا ہے۔ یوں ذیل ہو گا۔
\begin{align*}
I&=I_{\text{\RL{مرکاکمیت}}}+Mh^2=\tfrac{1}{2}mL^2+(2m)(\tfrac{1}{2}L)^2\\
&=mL^2\quad\quad\quad\text{\RL{(جواب)}}
\end{align*}
\انتہا{نمونی سوال}
%------------------------------

%Sample Problem 10.07 p276
\ابتدا{نمونی سوال}\موٹا{یکساں سلاخ کا گھمیری جمود با تکمل}\\
کمیت \عددی{M} اور لمبائی \عددی{L}  کی  یکساں سلاخ  محور \عددی{x} پر یوں رکھا گیا ہے کہ سلاخ کا وسط مبدا پر ہو (شکل \حوالہء{10.14})۔

(ا)  سلاخ کے وسط پر، سلاخ کو عمودی محور گھماو  پر سلاخ کا گھمیری جمود کیا ہو گا؟

\جزوحصہء{کلیدی تصورات}
(1) سلاخ انتہائی زیادہ ذروں پر، جو محور گھماو سے  انتہائی زیادہ  تعداد کے     مختلف  فاصلوں پر موجود ہیں،   مشتمل ہے ۔ ہم ہر ذرے کا انفرادی گھمیری جمود ہرگز معلوم نہیں کرنا چاہتے۔(ہم اپنی باقی تمام زندگی اس کام میں گزار سکتے ہیں۔) لہٰذا، ہم محور گھماو سے \عددی{r} فاصلے پر  کمیت  کے ایک چھوٹے ٹکڑے \عددی{\dif m} کے لئے گھمیری جمود کا عمومی  الجبرائی فقرہ: \عددی{r^2\dif m}  لکھتے ہیں۔ (2)   ایک ایک کر کے تمام چھوٹے حصوں کے گھمیری جمود جمع کرنے کے  بجائے ، ہم  اس فقرے کا تکمل لے کر  مجموعہ معلوم  کرتے ہیں۔ مساوات \حوالہ{مساوات_گھماو_گھمیری_جمود_استمراری_الف} سے ذیل لکھا جاتا ہے۔
%eq 10.38
\begin{align}\label{مساوات_گھماو_گھمیری_جمود_ٹکڑا_کمیت}
I=\int r^2\dif m
\end{align}
(3) سلاخ یکساں ہے اور محور گھماو عین مرکز کمیت سے گزرتا ہے، لہٰذا ہم گھمیری جمود \عددی{I_{\text{\RL{مرکزکمیت}}}}  معلوم کر رہے ہیں۔

\موٹا{حساب:}\quad
ہم محدد \عددی{x} کے لحاظ سے تکمل حاصل کرنا چاہتے ہیں (نا کہ کمیت \عددی{m} کے لحاظ سے جیسا تکمل کہتا  ہے)،  لہٰذا   کمیت کے ٹکڑے \عددی{\dif m} کا سلاخ  کے ہمراہ لمبائی \عددی{\dif x} کے ساتھ رشتہ درکار ہو گا۔ (شکل \حوالہء{10.14} میں ایک ایسا ٹکڑا دکھایا گیا ہے۔) سلاخ یکساں ہے، لہٰذا  تمام ٹکڑوں   کی   کمیت اور لمبائی  کی نسبت  برابر  ہو گی۔ یوں ذیل  ہو گا۔
\begin{align*}
\frac{\dif m\,\text{\RL{ٹکڑے کی کمیت}}}{\dif x\,\text{\RL{ٹکڑے کی لمبائی}}}&=\frac{M\,\text{\RL{سلاخ کی کمیت}}}{L\,\text{\RL{سلاخ کی لمبائی }}}
\end{align*}
یا
\begin{align*}
\dif m&=\frac{M}{L}\dif x
\end{align*}
مساوات \حوالہ{مساوات_گھماو_گھمیری_جمود_ٹکڑا_کمیت}  میں \عددی{r} کی جگہ  \عددی{x} اور \عددی{\dif m} کی جگہ درج بالا نتیجہ ڈال کر ، سلاخ کے ایک سر سے دوسرے سر تک (یعنی \عددی{x=-\tfrac{L}{2}} تا \عددی{x=\tfrac{L}{2}})  تکمل لیتے  ہوئے  کمیت کے تمام ٹکڑے شامل کرتے ہیں۔یوں ذیل ملتا ہے۔
\begin{align*}
I&=\int_{x=-L\!/\!2}^{x=+L\!/\!2}x^2\left(\frac{M}{L}\right)\dif x\\
&=\frac{M}{3L}\Big[x^3\Big]_{-L\!/\!2}^{L\!/\!2}=\frac{M}{3L}\left[\left(\frac{L}{2}\right)^3-\left(-\frac{L}{2}\right)^3\right]\\
&=\frac{1}{12}ML^2\quad\quad\quad\text{\RL{(جواب)}}
\end{align*}
(ب) ایک نئی محور گھماو   پر، جو سلاخ کے بائیں سر سے گزرتی اور سلاخ کو عمودی ہے،  سلاخ کا گھمیری جمود کیا ہو گا؟

\جزوحصہء{کلیدی تصورات}
ہم  محور \عددی{x} کا مبدا سلاخ کے بائیں سر پر منتقل کر کے  تکمل \عددی{x=0} تا \عددی{x=L} لے کر \عددی{I} معلوم کر  سکتے ہیں۔ تاہم، ہم زیادہ آسان اور طاقتور مسئلہ متوازی محور (مساوات \حوالہ{مساوات_گھماو_مسئلہ_متوازی_محور})  استعمال کرتے ہیں، جس میں  محور گھماو  کی سمت بندی تبدیل کیے بغیر اسے دوسری جگہ منتقل کرتے ہیں۔

\موٹا{حساب:}\quad
مرکز کمیت سے گزرتی  محور  کے متوازی  ، سلاخ کے بائیں سر  پر، نئی محور رکھ کر ہم مسئلہ متوازی محور (مساوات \حوالہ{مساوات_گھماو_مسئلہ_متوازی_محور})  استعمال کر سکتے ہیں۔ ہم جزو ا سے جانتے ہیں کہ
 \عددی{I_{\text{\RL{مرکزکمیت}}}=\tfrac{1}{12}ML^2} ہے۔ شکل \حوالہء{10.14} میں سلاخ کے وسط سے  نئی محور گھماو تک فاصلہ \عددی{\tfrac{1}{2}L} ہے۔ یوں مساوات \حوالہ{مساوات_گھماو_مسئلہ_متوازی_محور} ذیل دیگی۔
 \begin{align*}
 I&=I_{\text{\RL{مرکزکمیت}}}+Mh^2=\tfrac{1}{2}ML^2+(M)(\tfrac{1}{2}L)^2\\
 &=\tfrac{1}{3}ML^2\quad\quad\quad\text{\RL{(جواب)}}
 \end{align*}
 درحقیقت، یہ نتیجہ سلاخ کے  بائیں   یا دائیں سر پر ہر  ، سلاخ کو عمودی،  محور گھماو کے لئے درست ہے۔
\انتہا{نمونی سوال}
%-----------------------------

%Sample Problem 10.08 p277
\ابتدا{نمونی سوال}\موٹا{گھمیری جمودی توانائی؛چکری پرکھ}\\
مشین کے بڑے حصوں کا، جو  لمبے عرصہ   تیز رفتار سے  چکر کاٹتے ہوں،معائنہ\ترچھا{   چکری پرکھ  کے نظام } میں کرنا ضروری ہے۔ اس نظام میں،   فولادی بیلن  کے اندر، جس  کی اندرونی  جانب   سیسہ کی اینٹیں نصب ہوں، مشین کے حصے  کو مخصوص   چکری رفتار تک   (جس پر حصے کو پرکھنا مقصود ہو) لایا جاتا ہے۔اس دوران بیلن کا منہ فولادی ڈھکن  سے بند رکھا جاتا ہے۔ اگر مشین کا حصہ مطلوبہ چکری رفتار برداشت نہ کرتے ہوئے ٹوٹ جائے، اس کے ٹکڑے سیسہ کی ملائم اینٹوں میں دھنس کر محفوظ ہوں گے، جن کا معائنہ بعد میں کرنا ممکن ہو گا۔

\سن{1985} میں  ایک  ادارہ نے ، جو مشین پرکھنے کا کام کرتا ہے،  \عددی{\SI{272}{\kilo\gram}} ٹھوس فولادی (قرص شکل  کا)   مدور  ،جس کا رداس \عددی{R=\SI{38.0}{\centi\meter}} تھا ، پرکھنے کا کام لیا۔عین \عددی{\omega=14000} چکر فی منٹ کی زاویائی رفتار کو پہنچ کر\اصطلاح{آزمائش کار   معمار }\فرہنگ{آزمائش کار معمار}\حاشیہب{test engineer}\فرہنگ{test!engineer} ایک آواز سنتا ہے۔ تفتیش  کرنے پر معلوم ہوا  سیسہ کی  اینٹیں کمرے سے   باہر  بھکری  پڑی ہیں، کمرے کا دروازہ  گاڑیاں کھڑی کرنے  کی  جگہ   میں پڑا ملا، ایک سیسہ کی اینٹ پڑوسی کے باورچی خانے  کی دیوار توڑ کر اندر  پہنچی تھی،ادارے کی   عمارت  کے ستون ناکارہ ہو چکے تھے،  چکر خانہ کا پکا  فرش  \عددی{\SI{0.5}{\centi\meter}} زمین میں دھنس چکا تھا، اور  چکری نظام کا \عددی{\SI{900}{\kilo\gram}} ڈھکن اڑ کر چھت سے گزرتے ہوئے بالائی منزل میں داخل ہونے بعد واپس چکری نظام پر گر کر پڑا تھا۔ خوش قسمتی سے کوئی بھی ٹکڑا آزمائش کار معمار کے  کمرے کی طرف نہیں گیا۔

اس دھماکے میں کتنی توانائی خارج کی گئی؟

\جزوحصہء{کلیدی تصور}
خارج  توانائی \عددی{14000} چکر فی منٹ پر  مدور کی گھمیری حرکی توانائی \عددی{K} کے برابر ہو گی۔

\موٹا{حساب:}\quad
ہم مساوات \حوالہ{مساوات_گھماو_حرکی_گھمیری_تعریف} سے \عددی{K} کی قیمت \عددی{K=\tfrac{1}{2}I\omega^2}  تلاش کرتے ہیں، لیکن  اس سے پہلے مدور کا  گھمیری جمود \عددی{I}  جاننا ضروری ہے۔ قرص  کا گھمیری جمود جدول \حوالہء{10.2c}   کے تحت  \عددی{(I=\tfrac{1}{2}MR^2)} ہے۔ یوں ذیل ہو گا۔
\begin{align*}
I=\tfrac{1}{2}MR^2=\tfrac{1}{2}(\SI{272}{\kilo\gram})(\SI{0.38}{\meter})^2=\SI{19.64}{\kilo\gram\meter\squared}
\end{align*}
مدور کی زاوی رفتار   ، ریڈیئن ناپ میں حاصل کرتے ہیں۔
\begin{align*}
\omega&=(\text{\RL{چکر فی منٹ}}\, 14000)(\text{\RL{ریڈیئن فی چکر}}\, 2\pi)\big(\frac{\SI{1}{\minute}}{\SI{60}{\second}}\big)\\
&=\SI{1.466e3}{\radian\per\second}
\end{align*}
یوں مساوات \حوالہ{مساوات_گھماو_حرکی_گھمیری_تعریف} کے تحت  خارج توانائی  ذیل ہے (جو  بہت بڑی مقدار ہے)۔
\begin{align*}
K&=\tfrac{1}{2}I\omega^2=\tfrac{1}{2}(\SI{19.64}{\kilo\gram\meter\squared})(\SI{1.466e3}{\radian\per\second})^2\\
&=\SI{2.1e7}{\joule}\quad\quad\quad\text{\RL{(جواب)}}
\end{align*}
\انتہا{نمونی سوال}
%------------------------------------

%section 10.6 p277
\حصہ{قوت مروڑ}
اس حصے کو پڑھنے کے بعد آپ ذیل کے قابل ہوں گے۔
\begin{enumerate}[1.]
\item
جان پائیں گے کہ جسم پر قوت مروڑ میں قوت اور،   محور گھماو سے قوت کے نقطہ اطلاق تک  کا ، تعین گر سمتیہ شامل ہیں۔
\item
(ا)  تعین گر سمتیہ اور سمتیہ  قوت کے بیچ زاویے کی مدد سے ، (ب)   خط عمل اور قوت  کے معیار اثر کے  بازو کی مدد سے، اور (ج)  تعین گر سمتیہ کو قوت کے عمودی جزو  کی مدد سے قوت مروڑ تلاش کر پائیں گے۔
\item
جان پائیں گے کہ قوت مروڑ جاننے   کے لئے  محور گھماو جاننا لازم ہے۔
\item
جان پائیں گے کہ قوت مروڑ کو مثبت یا منفی علامت مختص کی جاتی ہے،  جس کا دارومدار اس رخ پر ہو گا جس رخ قوت مروڑ جسم کو  محور گھماو پر گھمانے کی کوشش کرتی ہے (یاد رہے، \قول{گھڑیاں منفی ہیں})۔
\item
جہاں ایک سے زیادہ قوت  مروڑ جسم پر عمل کرتی ہوں، صافی قوت مروڑ حاصل کر پائیں گے۔
\end{enumerate}

\جزوحصہء{کلیدی تصورات}
\begin{itemize}
\item
 قوت \عددی{\vec{F}} کی  ، محور گھماو پر  جسم کو گھمانے کی ، کوشش کو قوت مروڑ کہتے ہیں۔ اگر محور گھماو کے لحاظ سے،  \عددی{\vec{F}}  جس نقطہ پر عمل کرتی ہو ، اس نقطے کا تعین گر سمتیہ \عددی{\vec{r}} ہو،  تب قوت مروڑ کی قدر ذیل ہو گی،
 \begin{align*}
 \tau=rF_t=r_{\perp}R=rF\sin\phi
 \end{align*}
 جہاں \عددی{\vec{r}} کو \عددی{\vec{F}} کا عمودی جزو \عددی{F_t} ہے اور \عددی{\phi} قوت \عددی{\vec{F}} اور سمتیہ \عددی{\vec{r}} کے بیچ زاویہ ہے۔  محور گھماو اور  \عددی{\vec{F}} سے گزرتی مبسوط  لکیر   کے بیچ عمودی فاصلہ \عددی{r_{\perp}} ہے۔ مبسوط  لکیر  کو \عددی{\vec{F}} کا \قول{    خط عمل} ، اور \عددی{r_{\perp}}  کو \عددی{\vec{F}} کا \قول{ معیار اثر } کہتے ہیں۔ اسی طرح \عددی{r} کو \عددی{F_t} کا معیار اثر کہیں گے۔
 \item
 قوت مروڑ کی اکائی نیوٹن میٹر \عددی{(\si{\newton\meter})} ہے۔ ساکن جسم کو محور گھماو پر  خلاف گھڑی گھمانے کی کوشش کرنے والی قوت مروڑ  \عددی{\tau} مثبت ہو گی، گھڑی وار گھمانے کی کوشش کرنے والی منفی ہو گی۔
 \end{itemize}
 
 \جزوحصہء{قوت مروڑ}
 دروازے کا دستہ    چول  سے دور  ، کسی مقصد سے ، رکھا جاتا ہے۔ دروازہ کھولنے کے لئے   قوت لگانی ضروری ہے، تاہم قوت کا رخ اور لگانے کا مقام بھی اہمیت رکھتے ہیں۔ اگر آپ ، دستے کے  بجائے ، چول کے قریب قوت کا اطلاق کریں یا دروازے کی سطح   کو قوت  \عددی{\SI{90}{\degree}} پر  لاگو نہ کریں،  دروازہ کھولنے کے لئے  آپ کو اس قوت سے زیادہ قوت  درکار ہو گی ، جو دستے پر دروازے  کی سطح کو عمودی درکار چاہیے۔
 
 شکل \حوالہء{10.16a} میں   جسم کا عمودی تراش دکھایا گیا ہے۔ یہ جسم ، \عددی{O} سے گزرتی،    تراش کو عمودی محور گھماو پر  ،    آزادی سے گھوم سکتا ہے۔ نقطہ \عددی{P} پر، جس کا \عددی{O}   کے لحاظ سے تعین گر سمتیہ \عددی{\vec{r}} ہے، قوت \عددی{\vec{F}} کا اطلاق کیا گیا ہے۔  \عددی{\vec{F}} اور \عددی{\vec{r}} کر رخ آپس میں زاویہ \عددی{\phi} پر ہیں۔ (ہم اپنی آسانی کے لئے صرف ان  قوت کی بات کرتے ہیں  ، جن کا  محور گھماو کو متوازی جزو نہیں پایا جاتا؛ یوں \عددی{\vec{F}} صفحے کی سطح میں ہو گی۔)
 
 یہ جاننے کے لئے کہ محور گھماو پر \عددی{\vec{F}}  جسم کو کیسے گھماتی ہے، ہم \عددی{\vec{F}} کو دو اجزاء میں تقسیم کرتے ہیں (شکل \حوالہء{10.16b})۔ ایک جزو، جو  \ترچھا{رداسی جزو } \عددی{F_r}  کہلاتا ہے، \عددی{\vec{r}} کے ہمراہ ہو گا۔چونکہ  یہ جزو  \عددی{O} سے گزرتی لکیر  کے ہمراہ ہے، لہٰذا  یہ گھماو  میں کردار ادا نہیں کرتا۔ (اگر آپ دروازے کو دروازے کے سطح کے ہمراہ  کھینچیں، دروازہ کبھی بھی نہیں کھلے گا۔)  \عددی{\vec{F}} کا دوسرا جزو، جو\ترچھا{  مماسی جزو } \عددی{F_t} کہلاتا ہے، \عددی{\vec{r}} کو عمودی ہے اور اس کی قدر \عددی{F_t=F\sin\phi} ہے۔ یہ جزو گھماو کا سبب بنتا ہے۔
 
 \موٹا{قوت مروڑ کا حساب۔}
\عددی{\vec{F}} کی جسم گھمانے کی صلاحیت،  قوت \عددی{\vec{F}}  کے مماسی جزو \عددی{F_t}    کی قدر کے علاوہ \عددی{O} سے   (قوت کے ) اطلاقی نقطے کے  فاصلے پر منحصر ہے۔ان دونوں وجوہات کو شامل کرنے کی خاطر ہم (درج ذیل)  ایک نئی مقدار متعارف کرتے ہیں جو\اصطلاح{ قوت مروڑ}\فرہنگ{قوت مروڑ!تعریف}\حاشیہب{torque}\فرہنگ{torque!defined}  \عددی{\tau} کہلاتی ہے، جو دو جزو ضربیوں کا حاصل ضرب ہو گا۔
%eq 10.39
\begin{align}\label{مساوات_گھماو_قوت_مروڑ_فائے}
\tau=(r)(F\sin\phi)
\end{align}
قوت مروڑ کا حساب (درج ذیل)  دو معادل طریقوں:
%eq 10.40
\begin{align}\label{مساوات_گھماو_قوت_مروڑ_ذرے_پر}
\tau=(r)(F\sin\phi)=rF_t
\end{align}
اور
%eq 10.41
\begin{align}\label{مساوات_گھماو_صافی_قوت_مروڑ_الف}
\tau=(r)(F\sin\phi)=r_{\perp}F
\end{align}
سے ممکن ہے،  جہاں \عددی{O}    پر محور گھماو  اور  \عددی{\vec{F}} سمتیہ  سے گزرتی مبسوط لکیر  کے بیچ عمودی فاصلہ \عددی{r_{\perp}} ہے (شکل \حوالہء{10.16c})۔ اس مبسوط لکیر کو  \عددی{\vec{F}} کا \اصطلاح{   خط عمل}\فرہنگ{خط عمل}\حاشیہب{line of action}\فرہنگ{line of action} ، اور \عددی{r_{\perp}} کو    \عددی{\vec{F}}   کا\اصطلاح{  معیار اثر  کا بازو}\فرہنگ{معیار اثر! کا بازو}\حاشیہب{moment arm}\فرہنگ{moment!arm} کہتے ہیں۔ شکل \حوالہء{10.16b} میں دکھایا گیا ہے کہ  ہم \عددی{\vec{r}} کی قدر \عددی{r} کو  جزو قوت \عددی{F_t} کا  معیار اثر کا بازو کہہ سکتے ہیں۔

جب آپ کسی جسم،  مثلاً  پیچ کس ، پر اس نیت سے قوت لگاتے ہیں کہ یہ گھومے، آپ قوت مروڑ لاگو کرتے ہیں۔ قوت مروڑ کی بین الاقوامی اکائی نیوٹن میٹر \عددی{(\si{\newton\meter})} ہے۔ \ترچھا{انتباہ:} نیوٹن میٹر کی اکائی کام کے لئے بھی مستعمل ہے۔ تاہم ، قوت مروڑ اور کام دو مختلف مقادیر ہیں ۔ کام کے لئے عام طور جاول  اکائی  \عددی{(\SI{1}{\joule}=\SI{1}{\newton\meter})} استعمال کی جاتی ہے جبکہ قوت مروڑ کے لئے صرف نیوٹن میٹر اکائی  استعمال ہو گی۔

\موٹا{گھڑیاں منفی ہیں۔}
باب \حوالہء{11} میں قوت مروڑ کے لئے سمتیہ ترقیم استعمال کی جائے گی؛ یہاں   واحد محور پر گھماو کی بات کی جائے گی لہٰذا  الجبرائی علامت  استعمال کی جائے گی۔ اگر  قوت مروڑ خلاف گھڑی گھماو پیدا کرنے کی کوشش کرے،  یہ مثبت ہو گی اور اگر گھڑی وار کوشش کرے تب منفی ہو گی۔ (حصہ \حوالہء{10.1} میں ہم نے کہا \قول{گھڑیاں منفی ہیں}۔ یہ فقرہ یہاں بھی  کارآمد ہے۔)

اصول انطباق (جس کا ذکر باب \حوالہء{5} میں کیا گیا)   کو قوت مروڑ مطمئن کرتے ہیں: جب جسم پر کئی قوت مروڑ عمل کرتی ہوں، جسم پر \اصطلاح{ صافی قوت مروڑ }\فرہنگ{قوت مروڑ!صافی}\حاشیہب{net torque}\فرہنگ{torque!net}(یا \اصطلاح{  ماحصل قوت مروڑ}\فرہنگ{قوت مروڑ!ماحصل}\حاشیہب{resultant torque}\فرہنگ{torque!resultant})  انفرادی قوت مروڑ کا مجموعہ ہو گا۔  صافی قوت مروڑ کی علامت \عددی{\tau_{\text{\RL{صافی}}}} ہے۔

%Checkpoint 6 p278
%-----------------------
\ابتدا{آزمائش}
میٹر سلاخ کا فضائی جائزہ شکل \حوالہء{؟؟} میں پیش ہے؛ سلاخ کا چول \عددی{\SI{20}{\centi\meter}}  پر پایا جاتا ہے۔ سلاخ پر پانچوں قوت افقی  اور ان کی قدریں برابر ہیں۔ اعظم قیمت اول رکھ کر،  قوتوں  کی درجہ بندی ان کی پیدا قوت مروڑ کے لحاظ سے کریں۔
\انتہا{آزمائش}
%-----------------------------

%Section 10.7 Newton's Second Law For Rotation  p279
\حصہ{ نیوٹن کا دوسرا قانون برائے گھماو}
\موٹا{مقاصد}\\
اس حصے کو پڑھنے کے بعد آپ ذیل کے قابل ہوں گے۔
\begin{enumerate}[1.]
\item
گھماو کی صورت میں جسم پر صافی قوت مروڑ  کا،  جسم کے گھمیری جمود   اور گھمیری اسراع کے ساتھ،  رشتہ نیوٹن کے دوسرے قانون سے جان پائیں گے۔تمام مقادیر   مختص محور گھماو کے لحاظ سے ہیں۔
\end{enumerate}
\موٹا{کلیدی تصور}\\
\begin{itemize}
\item
نیوٹن کے دوسرے قانون  کا گھمیری مماثل ذیل ہے، 
\begin{align*}
\tau_{\text{\RL{صافی}}}=I\alpha
\end{align*}
جہاں ذرے یا استوار جسم پر صافی قوت مروڑ \عددی{\tau_{\text{\RL{صافی}}}}  ہے، محور گھماو  پر ذرے یا جسم کا گھمیری جمود \عددی{I} ہے، اور \عددی{\alpha} اس محور پر ماحصل زاوی اسراع ہے۔
\end{itemize}

%-------------------------------------
%p279
\جزوحصہء{نیوٹن کا دوسرا قانون برائے گھماو}
قوت مروڑ استوار جسم کو گھما سکتی ہے، جیسا آپ دروازہ قوت مروڑ سے کھولتے اور بند کرتے ہیں۔ہم استوار جسم پر  صافی  قوت مروڑ \عددی{\tau_{\text{\RL{صافی}}}}  اور محور گھماو پر  جسم کی اس زاوی اسراع \عددی{\alpha}  کا تعلق جاننا چاہتے ہیں جو یہ قوت مروڑ پیدا کرتی ہے۔ ہم نیوٹن کے دوسرے قانون \عددی{(F_{\text{\RL{صافی}}}=ma)} کو دیکھ کر،  جو محددی محور پر  کمیت \عددی{m}  کے  جسم پر صافی قوت \عددی{F_{\text{\RL{صافی}}}}  سے پیدا جسم کی خطی اسراع \عددی{a} کا تعلق دیتا ہے،  ایسا کریں گے۔ ہم \عددی{F_{\text{\RL{صافی}}}} کی جگہ \عددی{\tau_{\text{\RL{صافی}}}}،  \عددی{m} کی جگہ \عددی{I}،  اور \عددی{a} کی جگہ \عددی{\alpha} رکھ کر ذیل لکھتے ہیں۔
%eq 10.42
\begin{align}\label{مساوات_گھماو_نیوٹن_کا_دوسرا_قانون_برائے_گھماو}
\tau_{\text{\RL{صافی}}}=I\alpha\quad\quad\text{\RL{(نیوٹن کا دوسرا قانون برائے گھماو)}}
\end{align}

\جزوجزوحصہء{مساوات \حوالہ{مساوات_گھماو_نیوٹن_کا_دوسرا_قانون_برائے_گھماو} کا ثبوت}
پہلے شکل \حوالہء{10.17} میں پیش سادہ صورت کے لئے  مساوات \حوالہ{مساوات_گھماو_نیوٹن_کا_دوسرا_قانون_برائے_گھماو}  ثابت کرتے ہیں۔بلا کمیت     سلاخ  اور اس کے ایک سر پر  کمیت \عددی{m}  کا ذرہ مل کر استوار جسم دیتے ہیں۔سلاخ کی لمبائی \عددی{r} ہے اور یہ اپنے دوسرے سر  پر،  سطح  صفحہ کو عمودی  محور گھماو (دھرے) پر ، گھوم سکتی ہے۔ یوں،  ذرہ صرف دائری راہ پر ، جس کے وسط پر محور گھماو ہے،  حرکت  کا مجاز ہے۔

ذرے پر قوت \عددی{\vec{F}} عمل کرتی ہے۔ تاہم، ذرہ صرف دائری راہ پر حرکت کر سکتا ہے، لہٰذا قوت کا صرف    مماسی جزو \عددی{F_t}  (جو  دائری راہ کو مماس سے)   ذرے کو اس راہ پر  مسرع کر سکتا ہے۔ ہم \عددی{F_t} اور  اس راہ پر ذرے کے مماسی اسراع \عددی{a_t} کا تعلق نیوٹن کے دوسرے قانون سے لکھتے ہیں۔
\begin{align*}
F_t=ma_t
\end{align*}
ذرے پر قوت مروڑ ، مساوات \حوالہ{مساوات_گھماو_قوت_مروڑ_ذرے_پر} کے تحت ذیل ہو گا۔
\begin{align*}
\tau=F_t r=ma_tr
\end{align*}
مساوات \حوالہ{مساوات_گھماو_اسراع_ب}  \عددی{(a_t=\alpha r)} سے  اس کو ذیل لکھ سکتے ہیں۔
%eq 10.43
\begin{align}\label{مساوات_گھماو_ثبوت_الف}
\tau=m(\alpha r)r =(mr^2)\alpha
\end{align}
دائیں ہاتھ قوسین میں بند مقدار، محور گھماو پر ذرے کا گھمیری جمود ہے (مساوات \حوالہ{مساوات_گھماو_آئے_تعریف} دیکھیں، تاہم یہاں صرف ایک ذرے کی بات کی جا رہی ہے)۔ یوں گھمیری جمود کے لئے \عددی{I} لکھ کر مساوات \حوالہ{مساوات_گھماو_ثبوت_الف}   ذیل  لکھی جا سکتی ہے۔
%eq 10.44
\begin{align}\label{مساوات_گھماو_ثبوت_ب}
\tau=I\alpha\quad\quad\text{\RL{(ریڈیئن ناپ)}}
\end{align}
جہاں ایک سے زیادہ قوت ذرے پر عمل کرتی ہوں مساوات \حوالہ{مساوات_گھماو_ثبوت_ب} ذیل صورت اختیار کرے گی، جسے ہم ثابت کرنا چاہتے تھے۔
%eq 10.45
\begin{align}\label{مساوات_گھماو_ثبوت_پ}
\tau_{\text{\RL{صافی}}}=I\alpha\quad\quad\text{\RL{(ریڈیئن ناپ)}}
\end{align}
چونکہ  ہر جسم انفرادی  ذروں کا مجموعہ ہو گا لہٰذا اس مساوات کو  کسی بھی استوار جسم تک ، جو    مقررہ محور گھماو پر گھومتا ہو،  وسعت  دی جا سکتی ہے۔

%-------------------------
%Checkpoint 7 p280
\ابتدا{آزمائش}
شکل \حوالہء{؟؟} میں  میٹر سلاخ   کا فضائی جائزہ پیش ہے۔سلاخ کے وسط سے بائیں جانب نقطہ چول ہے جس پر سلاخ چکر کاٹ سکتی ہے۔ سلاخ پر دو افقی قوت \عددی{\vec{F}_1} اور \عددی{\vec{F}_2} لاگو کی جاتی ہیں۔ صرف \عددی{\vec{F}_1} دکھائی گئی ہے۔ قوت \عددی{\vec{F}_2} سلاخ کو عمودی ہے اور سلاخ کے دائیں سر پر لاگو کی جاتی ہے۔ سلاخ نہ گھومنے کی صورت میں (ا)  \عددی{\vec{F}_2} کا رخ کیا ہو گا اور (ب) کیا  \عددی{F_2}  کی  قیمت \عددی{F_1} سے کم ہو گی، زیادہ ہو گا، یا اس کے برابر ہو گی؟
\انتہا{آزمائش}
%--------------------

%Sample Problem 10.09 p280
\ابتدا{نمونی سوال}\موٹا{نیوٹن کے قانون   دوم برائے   گھماو  کا کولا میں  استعمال}\\
کولا کشتی کا وہ داو ہے جس میں پہلوان دوسرے کو  کولہے کی زد پر لا کر گراتا ہے۔ آئیں پہلوانوں کی کشتی  کو طبیعی دان کے نقطہ نظر سے دیکھیں۔ کولہے پر \عددی{\SI{80}{\kilo\gram}} حریف  کو   چڑھا کر  \عددی{\vec{F}} قوت کے ساتھ  دائیں کولہے پر نقطہ گھماو (محور گھماو)  رکھ کر \عددی{d_1=\SI{0.30}{\meter}}  معیار اثر کا بازو  استعمال کرتے ہوئے ، آپ حریف کو زمین پر مارتے ہیں (شکل \حوالہء{10.18})۔ آپ نقطہ گھماو  پر اس کو \عددی{\alpha=\SI{6.0}{\radian\per\second\squared}} زاوی اسراع   سے (جو شکل میں \ترچھا{گھڑی وار }ہے)   گھمانا چاہتے ہیں۔ فرض کریں  نقطہ گھماو کے لحاظ سے اس کا گھمیری جمود \عددی{I=\SI{15}{\kilo\gram\meter\squared}} ہے۔

(ا) زمین پر گرانے  سے قبل   اگر آپ حریف   کو آگے  جھکا  کر  اس کا مرکز کمیت اپنے کولہے پر رکھیں تو \عددی{\vec{F}} کی قدر کیا ہو گی (شکل \حوالہء{10.18a})؟

\جزوحصہء{کلیدی تصور}
ہم \عددی{\kvec{F}} کا  زاوی اسراع  سے رشتہ نیوٹن کے قانون دوم برائے گھماو \عددی{\tau_{\text{\RL{صافی}}}=I\alpha}  سے جانتے ہیں۔

\موٹا{حساب:}
زمین سے حریف کے پاوں  اٹھنے کے بعد، ہم کہہ سکتے ہیں اس پر تین قوت عمل پیرا ہوں گی: آپ  کی کھینچ \عددی{\vec{F}}، نقطہ گھماو پر آپ کی حریف پر  عمودی قوت \عددی{\vec{N}} (شکل \حوالہء{10.18} میں اسے نہیں دکھایا گیا)، اور  تجاذبی قوت \عددی{\vec{F}_g}۔نقطہ گھماو پر تینوں قوتوں کی قوت مروڑ جانتے ہوئے ہم   \عددی{\tau_{\text{\RL{صافی}}}=I\alpha}استعمال کر پائیں گے۔

مساوات \حوالہ{مساوات_گھماو_صافی_قوت_مروڑ_الف} \عددی{(\tau=r_{\perp}F) }  کے تحت آپ کی کھینچ  \عددی{\vec{F}} سے پیدا قوت مروڑ \عددی{-d_1F} ہو گی، جہاں  \عددی{d_1} معیار اثر کا بازو \عددی{r_{\perp}} ہے ، اور منفی علامت کہتی ہے کہ یہ مروڑ گھڑی وار گھماو کی کوشش کرتی ہے۔ قوت \عددی{\vec{N}} نقطہ گھماو سے گزرتی ہے لہٰذا اس کا معیار اثر کا  بازو  \عددی{r_{\perp}=0} ہو گا  اور یوں    اس کی قوت مروڑ   بھی صفر ہو گی۔
تجاذبی قوت \عددی{\vec{F}_g} حریف کے مرکز کمیت  پر عمل کرتی ہے۔مرکز کمیت عین نقطہ گھماو پر ہے لہٰذا \عددی{\vec{F}_g} کا معیار اثر کا بازو  \عددی{r_{\perp}=0} ہو گا اور یوں اس کی قوت مروڑ بھی صفر ہو گی۔ یوں حریف پر صرف  آپ کی کھینچ \عددی{\vec{F}} کی   قوت مروڑ عمل کرتی ہے اور ہم \عددی{\tau_{\text{\RL{صافی}}}=I\alpha} ذیل  لکھ سکتے ہیں۔
\begin{align*}
-d_1F=I\alpha
\end{align*}
یوں ذیل حاصل ہو گا۔
\begin{align*}
F&=\frac{-I\alpha}{d_1}=\frac{-(\SI{15}{\kilo\gram\meter\squared})(\SI{-6.0}{\radian\per\second\squared})}{\SI{0.30}{\meter}}\\
&=\SI{300}{\newton}\quad\quad\quad\text{\RL{(جواب)}}
\end{align*}
(ب)  اگر  گرنے سے پہلے آپ کا حریف سیدھا کھڑا رہے تا کہ \عددی{\vec{F}_g} کا معیار اثر کا بازو \عددی{d_2=\SI{0.12}{\meter}} ہو تب \عددی{\vec{F}} کی قدر کیا ہو گا (شکل \حوالہء{10.18b})؟

\جزوحصہء{کلیدی تصور}
چونکہ \عددی{\vec{F}_g=mg} کا معیار اثر کا بازو اب صفر نہیں، اس کی قوت مروڑ اب \عددی{d_2mg} ہو گی جو خلاف گھڑی ہونے کی بنا مثبت ہے۔

\موٹا{حساب:}
ہم \عددی{\tau_{\text{\RL{صافی}}}=I\alpha} اب ذیل لکھتے ہیں
\begin{align*}
-d_1F+d_2mg=I\alpha
\end{align*}
جو ذیل دیگا۔
\begin{align*}
F&=\SI{300}{\newton}+\frac{(\SI{0.12}{\meter})(\SI{80}{\kilo\gram})(\SI{9.8}{\meter\per\second\squared})}{\SI{0.30}{\meter}}\\
&=\SI{613.5}{\newton}\approx\SI{610}{\newton}\quad\quad\text{\RL{(جواب)}}
\end{align*}
اس نتیجے  کے تحت  اگر آپ حریف کو جھکا  کر اس کا مرکز کمیت اپنے کولہے پر رکھ نہ سکیں، آپ کو کافی زیادہ قوت لگانی ہو گی۔ ایک اچھا پہلوان یہ حقیقت جانتا ہے۔
\انتہا{نمونی سوال}
%---------------------------------

%Sample problem 10.10  p281
\ابتدا{نمونی سوال}\موٹا{نیوٹن کا دوسرا قانون، قوت مروڑ، قرص}\\
کمیت  \عددی{M=\SI{2.5}{\kilo\gram}} اور رداس \عددی{R=\SI{20}{\centi\meter}} کا یکساں قرص مقررہ افقی دھرے پر نصب  شکل \حوالہء{10.19a} میں دکھایا گیا ہے۔ قرص کے\اصطلاح{  چکا }\فرہنگ{چکا}\حاشیہب{rim}\فرہنگ{rim} پر بلا کمیت  دھاگا لپیٹ کر اس سے  \عددی{m=\SI{1.2}{\kilo\gram}} کمیت  کی اینٹ آویزاں کی  گئی ہے۔ ساکن اینٹ رہا کی جاتی ہے۔ اینٹ کا  اسراع، قرص کا  زاوی  اسراع ، اور دھاگے میں تناو تلاش کریں۔دھاگہ پھسلتا  نہیں  اور دھرا   بے رگڑ ہے۔

\جزوحصہء{کلیدی تصورات}
(1)  اینٹ کو ایک نظام تصور کر کے اس کی اسراع \عددی{a} اور اس پر عمل پیرا قوت کا تعلق ہم نیوٹن کے قانون دوم \عددی{(\vec{F}_{\text{\RL{صافی}}}=m\vec{a})} سے لکھ سکتے ہیں۔
(2) قرص کو ایک نظام تصور کرتے ہوئے ہم اس کے  زاوی اسراع \عددی{\alpha}   اور  اس پر قوت مروڑ   کا تعلق  نیوٹن کے قانون دوم برائے گھماو   \عددی{(\tau_{\text{\RL{صافی}}}=I\alpha)}سے لکھ سکتے ہیں۔ (3)  اینٹ اور قرص  کی حرکات  کو ملانے کے لئے ہم اس حقیقت کو بروئے کار لاتے ہیں کہ اینٹ کا  خطی اسراع \عددی{a} اور قرص کے    چکا   کا (  مماسی ) خطی  اسراع \عددی{a_t}   برابر ہیں۔ (الجھنے سے بچنے کی خاطر ہم اسراع  کی قدروں اور الجبرائی  علامتوں  پر علیحدہ علیحدہ غور کرتے ہیں۔)

\موٹا{اینٹ پر قوتیں:}\quad
شکل \حوالہء{10.19b} کے  آزاد جسمی  خاکے  میں اینٹ پر لاگو قوتیں  دکھائی گئی ہیں:دھاگے سے قوت \عددی{\vec{T}}، اور تجاذبی قوت \عددی{\vec{F}_g} کی قدر \عددی{mg} ہے۔    انتصابی  \عددی{y} محور کے ہمراہ  اجزاء کے لئے نیوٹن کا قانون دوم  \عددی{F_{\text{\RL{صافی}},y}=ma_y} لکھتے ہیں:
%eq 10.46
\begin{align}\label{مساوات_گھماو_نمونی_قرص_مروڑ}
T-mg=m(-a)
\end{align}
جہاں (  محور \عددی{y} کے ہمراہ نیچے رخ)  اسراع  کی قدر \عددی{a} ہے۔ تاہم،   ہم اس  مساوات کو \عددی{a} کے لئے حل نہیں کر سکتے، چونکہ اس میں دوسرا نامعلوم متغیر \عددی{T} بھی پایا جاتا ہے۔

\موٹا{قرص پر قوت مروڑ:}\quad
گزشتہ مرتبہ جب ہم محور \عددی{y}  سے آگے بڑھ نہیں سکے، ہم  نے محور \عددی{x} کا سہارا لیا۔ اس مرتبہ ہم قرص کے گھماو کا سہارا اٹھاتے  ہوئے نیوٹن کا قانون دوم  زاوی روپ میں لکھتے ہیں۔ قوت مروڑ اور گھمیری جمود \عددی{I}  تلاش کر نے کے لئے ، ہم نقطہ \عددی{O} پر ، قرص کو عمودی اور اس  کے وسط سے گزرتی لکیر  ،محور گھماو  لیتے ہیں (شکل \حوالہء{10.19c})۔

مساوات \حوالہ{مساوات_گھماو_قوت_مروڑ_ذرے_پر} \عددی{(\tau=rF_t)}  قوت مروڑ دیتی  ہے۔ قرص پر تجاذبی قوت اور دھرے  کی قوت دونوں قرص کے وسط پر عمل کرتے ہیں لہٰذا ان کا   فاصلہ  \عددی{r=0}  اور یوں قوت  مروڑ صفر ہوں گی۔ قرص پر دھاگے کی قوت \عددی{\vec{T}}  \عددی{r=R} فاصلے پر   قرص کے  چکا  پر مماسی عمل کرتی ہے۔اس طرح، اس کی قوت مروڑ \عددی{-RT} ہو گی؛ چونکہ قوت مروڑ قرص کو ساکن حالت سے گھڑی وار گھمانے کی کوشش کرتی ہے لہٰذا اس کی علامت منفی ہے۔ منفی (گھڑی وار)  زاوی اسراع  کی قدر  \عددی{\alpha} لیتے ہیں۔ جدول \حوالہء{10.2c} کے تحت قرص کا   گھمیری جمود  \عددی{\tfrac{1}{2}MR^2} ہے۔ یوں عمومی مساوات \عددی{\tau_{\text{\RL{صافی}}}=I\alpha} ذیل لکھی جائے گی،
%eq 10.47
\begin{align}\label{مساوات_گھماو_نمونی_قرص_دوسرا}
-RT=\tfrac{1}{2}MR^2(-\alpha)
\end{align}
بظاہر، یہ مساوات کسی کام کی نہیں ہے؛ اس میں دو نامعلوم متغیرات \عددی{\alpha} اور \عددی{T} پائے جاتے ہیں جبکہ ہمیں \عددی{a} چاہیے۔ طبیعی علم بروئے کار لاتے ہوئے ہم اس کو فائدہ مند ٹہرا سکتے ہیں: چونکہ دھاگہ پھسلتا نہیں، اینٹ کے خطی اسراع کی قدر \عددی{a} اور  قرص کے  چکا  کے ( مماسی ) اسراع  کی قدر \عددی{a_t} برابر ہوں گی۔ یوں،  مساوات \حوالہ{مساوات_گھماو_اسراع_ب} \عددی{(a_t=\alpha r)} کے تحت  \عددی{\alpha=a\!/\!R} ہو گا۔ مساوات \حوالہ{مساوات_گھماو_نمونی_قرص_دوسرا} میں  یہ معلومات ڈال کر ذیل حاصل ہو گا۔
%eq 10.48
\begin{align}\label{مساوات_گھماو_نمونی_قرص_ب}
T=\tfrac{1}{2}Ma
\end{align}
\موٹا{نتائج کی  یکجائی:}\quad
مساوات \حوالہ{مساوات_گھماو_نمونی_قرص_مروڑ} اور مساوات \حوالہ{مساوات_گھماو_نمونی_قرص_ب} ملا کر ذیل حاصل ہو گا۔
\begin{align*}
a&=g\frac{2m}{M+2m}=(\SI{9.8}{\meter\per\second\squared})\frac{(2)(\SI{1.2}{\kilo\gram})}{\SI{2.5}{\kilo\gram}+(2)(\SI{1.2}{\kilo\gram})}\\
&=\SI{4.8}{\meter\per\second\squared}\quad\quad\quad\text{\RL{(جواب)}}
\end{align*}
اب مساوات \حوالہ{مساوات_گھماو_نمونی_قرص_ب} سے \عددی{T} حاصل ہو گا۔
\begin{align*}
T&=\tfrac{1}{2}Ma=\tfrac{1}{2}(\SI{2.5}{\kilo\gram})(\SI{4.8}{\meter\per\second\squared})\\
&=\SI{6.0}{\newton}\quad\quad\text{\RL{(جواب)}}
\end{align*}
جیسا ہمیں توقع کرنی چاہیے، گرتی اینٹ کا اسراع \عددی{a} آزادانہ گرنے کے اسراع \عددی{g} سے کم ، اور دھاگے میں تناو \عددی{T=\SI{6.0}{\newton}} لٹکی اینٹ پر تجاذبی قوت \عددی{mg=\SI{11.8}{\newton}} سے کم ہے۔ ساتھ ہی ہم دیکھتے ہیں کہ \عددی{a} اور \عددی{T} دونوں پر قرص کی کمیت  پر منحصر ہیں جبکہ ان پر رداس  کا کوئی اثر نہیں۔

تصدیق کے طور پر، ہم دیکھتے ہیں کہ بلا کمیت قرص \عددی{(M=0)} کی صورت میں  \عددی{a=g} اور \عددی{T=0} ہو گا۔ہم یہی توقع رکھتے ہیں؛ اینٹ ایک آزاد جسم کی طرح زمین پر گرتی ہے۔ مساوات \حوالہ{مساوات_گھماو_اسراع_ب} سے قرص کے زاوی اسراع کی قدر تلاش کرتے ہیں۔
\begin{align*}
\alpha=\frac{a}{R}=\frac{\SI{4.8}{\meter\per\second\squared}}{\SI{0.2}{\meter}}=\SI{24}{\radian\per\second}\quad\text{\RL{(جواب)}}
\end{align*}
\انتہا{نمونی سوال}
%-----------------------------------

%section 10.8 Work And Rotational Kinetic Energy p282
\حصہ{کام اور  گھمیری  حرکی توانائی}
\موٹا{مقاصد}\\
اس حصے کر پڑھنے کے بعد آپ ذیل کے قابل ہوں گے۔
\begin{enumerate}[1.]
\item
گھومتے جسم پر لاگو قوت مروڑ کا زاویہ گھماو کے   لحاظ سے  تکمل   لے کر   ، گھومتے جسم پر لاگو قوت مروڑ کا   سرانجام  کام معلوم کر پائیں گے۔
\item
مسئلہ کام و حرکی توانائی  استعمال کر کے  جسم کے گھمیری  حرکی  توانائی میں تبدیلی  اور  سرانجام   کام  کا رشتہ جان پائیں گے۔
\item
 کام اور اس زاویے کے  تعلق   سے، جس پر جسم گھومتا ہے،\ترچھا{ مستقل } قوت مروڑ  کا سرانجام کام تلاش کر پائیں گے۔
 \item
 کام کی شرح معلوم کر کے قوت مروڑ کی طاقت جان پائیں گے۔
 \item
 کسی لمحے پر  قوت مروڑ اور اس لمحے پر زاوی سمتی رفتار  کے رشتہ سے قوت مروڑ کی طاقت جان پائیں گے۔

\end{enumerate}
\موٹا{کلیدی تصورات}\\
\begin{itemize}
\item
زاوی حرکت میں کام اور طاقت کی ذیل  مساوات مستقیم حرکت کی مساوات سے مطابقت رکھتی ہیں۔
\begin{align*}
W&=\int_{\theta_i}^{\theta_f}\tau\dif \theta\\
P&=\frac{\dif W}{\dif t}\tau\omega
\end{align*}
\item
جب \عددی{\tau} مستقل ہو، تکمل  گھٹ کر  ذیل دیگا۔
\begin{align*}
W=\tau(\theta_f-\theta_i)
\end{align*}
\item
گھومتے اجسام کے  لئے مسئلہ کام و حرکی توانائی ذیل روپ اختیار کرتا ہے۔
\begin{align*}
\Delta K=K_f-K_i=\tfrac{1}{2}I\omega_f^2-\tfrac{1}{2}I\omega_i^2=W
\end{align*}
\end{itemize}

%---------------------------------
%p282
\جزوحصہء{کام اور گھمیری حرکی توانائی}
جیسا ہم باب \حوالہء{7} میں ذکر کر چکے،  جب  قوت \عددی{F} استوار جسم پر، جس کی کمیت \عددی{m} ہو، عمل کر کے اس کو محددی محور   پر مسرع کرے، قوت اس جسم پر کام سرانجام دیتی  ہے۔ یوں، جسم کی حرکی توانائی \عددی{(K=\tfrac{1}{2}mv^2)} تبدیل  ہو سکتی ہے۔ فرض کریں جسم کی صرف یہی توانائی تبدیل ہوتی ہے۔ ایسی صورت میں حرکی توانائی کی تبدیلی \عددی{\Delta K} اور کام \عددی{W} کا  تعلق درجہ    ذیل مسئلہ کام و حرکی توانائی (مساوات \حوالہء{7.10}) دیگا۔
%eq 10.49
\begin{align}\label{مساوات_گھمیری_مسئلہ_کام_حرکی_توانائی}
\Delta K=K_f-K_i=\tfrac{1}{2}mv_f^2-\tfrac{1}{2}mv_i^2=W\quad\quad\text{\RL{(مسئلہ کام و حرکی توانائی)}}
\end{align}
محور \عددی{x}  پر رہنے کی پابند حرکت  کے لئے کام  کی درج ذیل مساوات \حوالہء{7.32} دیگی۔
%eq 10.50
\begin{align}
W=\int_{x_i}^{x_f}F\dif x\quad\quad\text{\RL{(کام، یک بُعدی حرکت)}}
\end{align}
جب \عددی{F} مستقل  اور جسم کا ہٹاو \عددی{d} ہو، یہ گھٹ کر \عددی{W=Fd} دیتی ہے۔ کام کرنے کی شرح  طاقت کہلاتی ہے، جو ہم مساوات \حوالہء{7.43} اور مساوات \حوالہء{7.48} سے معلوم کر سکتے ہیں۔
%eq 10.51
\begin{align}
P=\frac{\dif W}{\dif t}=Fv\quad\quad\text{\RL{(طاقت، یک بُعدی حرکت)}}
\end{align}

آئیں اس سے ملتی جلتی  گھمیری صورت پر غور کرتے ہیں۔جب  قوت مروڑ   ، مقررہ  محور گھماو پر   ، استوار جسم   کو مسرع  کرے، قوت مروڑ جسم پر کام \عددی{W} سرانجام دیتی  ہے۔ یوں، جسم کی گھمیری حرکی توانائی \عددی{(K=\tfrac{1}{2}I\omega^2)} تبدیل ہو سکتی ہے۔ فرض کریں جسم کی صرف یہی توانائی تبدیل ہوتی ہے۔ ایسی صورت میں حرکی توانائی میں تبدیلی \عددی{\Delta K} اور  کام \عددی{W} کا رشتہ مسئلہ کام و حرکی توانائی دیگا، تاہم اب حرکی توانائی  کے بجائے گھمیری حرکی توانائی  کی بات کی جائے گی۔
%eq 10.52
\begin{align}\label{مساوات_گھماو_زاوی_معادل_الف}
\Delta K=K_f-K_i=\tfrac{1}{2}I\omega_f^2-\tfrac{1}{2}I\omega_i^2\quad\quad\text{\RL{(مسئلہ کام و حرکی توانائی)}}
\end{align}
یہاں، \عددی{I} مقررہ محور پر جسم کا  گھمیری جمود اور \عددی{\omega_i} اور \عددی{\omega_f}  کام سے قبل اور اس کے بعد جسم کی زاوی رفتار ہیں۔

%p283
ساتھ ہی، ہم مساوات \حوالہء{10.52} کی  معادل  گھمیری مساوات سے کام تلاش کر  سکتے ہیں:
%eq 10.53
\begin{align}\label{مساوات_گھماو_کام_زاوی_الف}
W=\int_{\theta_i}^{\theta_f}\tau\dif \theta\quad\quad\text{\RL{(کام، مقررہ محور پر گھماو)}}
\end{align}
جہاں \عددی{\tau} وہ قوت مروڑ ہے جو کام \عددی{W} سرانجام دیتی ہے، اور \عددی{\theta_i} اور \عددی{\theta_f}،   کام سے قبل اور اس کے بعد، جسم کے  زاوی مقام ہیں۔ جب \عددی{\tau} مستقل ہو، مساوات \حوالہ{مساوات_گھماو_کام_زاوی_الف} گھٹ کر ذیل صورت اختیار کرتی ہے۔
%eq 10.54
\begin{align}\label{مساوات_گھماو_زاوی_معادل_درمیانی}
W=\tau(\theta_f-\theta_i)\quad\quad\text{\RL{(کام، مستقل قوت مروڑ)}}
\end{align}
کام کرنے کی شرح طاقت کہلاتی ہے، جو ہم  مساوات \حوالہء{10.51} کی  معادل   گھمیری  ذیل مساوات سے تلاش کر سکتے ہیں۔
%eq 10.55
\begin{align}\label{مساوات_گھماو_زاوی_معادل_اخیر}
P=\frac{\dif W}{\dif t}=\tau\omega\quad\quad\text{\RL{(طاقت، مقررہ محور پر گھماو)}}
\end{align}
جدول \حوالہ{جدول_گھماو_گھمیری_مستقیم_مساوات} میں  مقررہ محور پر استوار جسم کے گھماو کی  چند مساوات  اور مطابقتی مستقیم حرکت کی مساوات پیش ہیں۔
\begin{table}
\caption{مستقیم  اور مطابقتی   گھمیری  حرکت کی چند مساوات}
\label{جدول_گھماو_گھمیری_مستقیم_مساوات}
\renewcommand{\arraystretch}{1.5} 
\begin{tabular}{rl|rl}
\toprule
\multicolumn{2}{c}{خالص مستقیم حرکت (مقررہ رخ)}& \multicolumn{2}{c}{خالص گھماو (مقررہ محور)}\\
\midrule
مقام&\(x\) & زاوی مقام &  \(\theta\)\\
سمتی رفتار& \(v=\dif x\!/\!\dif t\) &  زاوی سمتی رفتار  &  \(\omega=\dif \theta\!/\!\dif t\)\\
اسراع &  \(a=\dif v\!/\!\dif t\) & زاوی اسراع &  \(\alpha=\dif \omega \!/\!\dif t\)\\
کمیت&  \(m\) & گھمیری جمود &  \(I\)\\
نیوٹن کا قانون دوم &  \(F_{\text{\RL{صافی}}}=ma\) & نیوٹن کا  قانون دوم &  \(\tau_{\text{\RL{صافی}}}=I\alpha\)\\
کام &  \(W=\int F\dif x\) & کام &  \(W=\int \tau\theta\) \\
حرکی توانائی &  \(K=\tfrac{1}{2}mv^2\) & حرکی توانائی &  \(K=\tfrac{1}{2}I\omega^2\) \\
طاقت (مستقل قوت) &  \(P=Fv\) & طاقت (مستقل قوت مروڑ) &  \(P=\tau\omega\) \\
مسئلہ کام و حرکی توانائی &  \(W=\Delta K\) & مسئلہ کام و حرکی توانائی &  \(W=\Delta K\)\\
\bottomrule
\end{tabular}
\end{table}

%p283
\جزوجزوحصہء{مساوات \حوالہ{مساوات_گھماو_زاوی_معادل_الف} تا مساوات \حوالہ{مساوات_گھماو_زاوی_معادل_اخیر} کا   ثبوت}
آئیں دوبارہ شکل \حوالہء{10.17}   کو دیکھتے ہیں۔ بلا کمیت     سلاخ  اور اس کے ایک سر پر  کمیت \عددی{m}  کا ذرہ مل کر استوار جسم دیتے ہیں۔ گھماو کے دوران، قوت \عددی{\vec{F}} جسم پر کام سرانجام دیتی ہے۔ ہم فرض کرتے ہیں کہ \عددی{\vec{F}} جسم کی صرف حرکی توانائی تبدیل کرتی ہے۔ ایسی صورت میں مساوات \حوالہ{مساوات_گھمیری_مسئلہ_کام_حرکی_توانائی} کا  مسئلہ کام و حرکی توانائی  استعمال کیا جا سکتا ہے لہٰذا ذیل ہو گا۔
%eq 10.56
\begin{align}\label{مساوات_گھماو_مساوات_کے_ثبوت_الف}
\Delta K=K_f-K_i=W
\end{align}
مساوات \حوالہ{مساوات_گھماو_مساوات_کے_ثبوت_الف} میں  \عددی{K=\tfrac{1}{2}mv^2} اور  مساوات \حوالہ{مساوات_گھماو_خطی_زاوی_تعلق_ب} \عددی{(v=\omega r)} استعمال  کر  کے اسے ذیل لکھا جا سکتا ہے۔
%eq 10.57
\begin{align}\label{مساوات_گھماو_مساوات_کے_ثبوت_ب}
\Delta K=\tfrac{1}{2}mr^2\omega_f^2-\tfrac{1}{2}mr^2\omega_i^2=W
\end{align}
مساوات \حوالہ{مساوات_گھماو_آئے_تعریف} کے تحت  واحد  ذروی جسم کا گھمیری جمود \عددی{I=mr^2} ہے، جو  مساوات \حوالہ{مساوات_گھماو_مساوات_کے_ثبوت_ب} میں ڈال کر ذیل حاصل ہو گا، جو مساوات \حوالہ{مساوات_گھماو_زاوی_معادل_الف} ہے۔
\begin{align*}
\Delta K=\tfrac{1}{2}I\omega_f^2-\tfrac{1}{2}I\omega_i^2=W
\end{align*}
ہم نے مساوات یک   ذروی جسم کے لئے ثابت کی، تاہم ہر جسم متعدد ذروں پر مشتمل ہو گا لہٰذا یہ مقررہ محور پر  گھمائے گئے ہر استوار جسم  کے لئے درست ہے۔

%p284
آئیں اب شکل \حوالہء{10.17}  میں جسم  پر سرانجام کام \عددی{W} اور جسم پر  \عددی{\vec{F}} کی بنا قوت مروڑ \عددی{\tau} کا  تعلق  جانیں۔ جب ذرہ دائری راہ پر چلتے ہوئے \عددی{\dif s} فاصلہ طے کرتا ہے، قوت کا صرف مماسی جزو \عددی{F_t} اس راہ پر ذرے کو اسراع پذیر کرتا ہے۔ یوں صرف \عددی{F_t} ذرے پر کام سرانجام  دیگی۔ ہم اس کام \عددی{\dif W} کو \عددی{F_t\dif s} لکھتے ہیں۔ ہم \عددی{\dif s} کی جگہ \عددی{r\dif \theta} لکھ سکتے ہیں، جہاں  ذرہ زاویہ \عددی{\dif \theta} طے کرتا ہے۔  یوں ذیل ہو گا۔
%eq 10.58
\begin{align}\label{مساوات_گھماو_کام_تفرقی_الف}
\dif W=F_t r\dif\theta
\end{align}
مساوات \حوالہ{مساوات_گھماو_قوت_مروڑ_ذرے_پر} سے  ہم دیکھتے ہیں کہ ماحصل \عددی{F_t r} اور قوت مروڑ \عددی{\tau} برابر ہوں گے لہٰذا  مساوات \حوالہ{مساوات_گھماو_کام_تفرقی_الف} ذیل روپ اختیار کرتی ہے۔
%eq 10.59
\begin{align}\label{مساوات_گھماو_کام_تفرقی_ج}
\dif W=\tau\dif\theta
\end{align}
یوں \عددی{\theta_i} تا \عددی{\theta_f} کے  متناہی زاوی ہٹاو کے دوران سرانجام کام ذیل ہو گا، 
\begin{align*}
W=\int_{\theta_i}^{\theta_f}\tau\dif\theta
\end{align*}
جو مساوات \حوالہ{مساوات_گھماو_کام_زاوی_الف} ہے۔ یہ مساوات مقررہ محور پر  گھومتے ہر استوار جسم کے لئے درست ہے۔ مساوات \حوالہ{مساوات_گھماو_کام_تفرقی_ج}  سے بلا واسطہ مساوات \حوالہ{مساوات_گھماو_زاوی_معادل_درمیانی} حاصل ہوتی ہے۔

گھمیری حرکت کے لئے مساوات \حوالہ{مساوات_گھماو_کام_تفرقی_ج} سے طاقت \عددی{P}   لکھتے ہیں:
\begin{align*}
P=\frac{\dif W}{\dif t}=\tau\frac{\dif\theta}{\dif t}=\tau\omega
\end{align*}
جو مساوات \حوالہ{مساوات_گھماو_زاوی_معادل_اخیر} ہے۔

%------------------------
%Sample Problem 10.11 p284
\ابتدا{نمونی سوال}\موٹا{کام، گھمیری حرکی توانائی، قوت مروڑ، قرص}\\
شکل \حوالہء{10.19} میں وقت \عددی{t=0}  پر قرص  ساکن حالت  سے آغاز کرتا ہے؛ بلا کمیت دھاگے میں تناو \عددی{\SI{6.0}{\newton}} اور قرص کا زاوی اسراع \عددی{\SI{-24}{\radian\per\second\squared}} ہے۔ اس کی گھمیری حرکی توانائی \عددی{t=\SI{2.5}{\second}} پر کیا ہو گی؟

\جزوحصہء{کلیدی تصور}
ہم مساوات \حوالہ{مساوات_گھماو_حرکی_گھمیری_تعریف} \عددی{(K=\tfrac{1}{2}I\omega^2)} سے \عددی{K} تلاش کر سکتے ہیں۔ہم جانتے ہیں کہ \عددی{I=\tfrac{1}{2}MR^2} ہے، تاہم تا حال  \عددی{t=\SI{2.5}{\second}} پر \عددی{\omega} نہیں جانتے۔ زاوی اسراع \عددی{\alpha} کی مستقل قیمت \عددی{\SI{-24}{\radian\per\second\squared}} لہٰذا ہم جدول \حوالہ{جدول_گھماو_مستقل_اسراع_مساوات}  میں پیش مستقل زاوی اسراع کی  مساوات  استعمال کر سکتے ہیں۔

\موٹا{حساب:}\quad
ہم \عددی{\alpha} اور \عددی{\omega_0=0} جانتے ہیں اور \عددی{\omega} جاننا چاہتے ہیں، لہٰذا مساوات \حوالہ{مساوات_گھماو_زاوی_الف} استعمال کرتے ہیں۔
\begin{align*}
\omega=\omega_0+\alpha t =0+\alpha t=\alpha t
\end{align*}
مساوات \حوالہ{مساوات_گھماو_حرکی_گھمیری_تعریف} میں \عددی{\omega=\alpha t} اور \عددی{I=\tfrac{1}{2}MR^2} ڈال کر ذیل حاصل ہو گا۔
\begin{align*}
K&=\tfrac{1}{2}I\omega^2=\tfrac{1}{2}(\tfrac{1}{2}MR^2)(\alpha t)^2=\tfrac{1}{4}M(R\alpha t)^2\\
&=\tfrac{1}{4}{\SI{2.5}{\kilo\gram}}[(\SI{0.20}{\meter})(\SI{-24}{\radian\per\second\squared})(\SI{2.5}{\second})]^2\\
&=\SI{90}{\joule}\quad\quad\text{\RL{(جواب)}}
\end{align*}
\جزوحصہء{کلیدی تصور}
ہم یہی جواب  سرانجام کام سے  قرص کی حرکی توانائی معلوم کر کے حاصل کر سکتے ہیں۔

\موٹا{حساب:}\quad
پہلے ہم  قرص پر صافی سرانجام کام \عددی{W} اور قرص کی حرکی توانائی میں \ترچھا{تبدیلی} کا رشتہ،  مساوات \حوالہ{مساوات_گھماو_زاوی_معادل_الف} \عددی{(K_f-K_i=W)}  میں پیش  ، مسئلہ کام و حرکی توانائی  سے  لکھتے ہیں۔ \عددی{K_f} کی جگہ \عددی{K} اور \عددی{K_i} کی جگہ \عددی{0} ڈال کر ذیل ہو گا۔
%eq 10.60
\begin{align}\label{مساوات_گھماو_نمونی_کام_حرکی_الف}
K=K_i+W=0+W=W
\end{align}

اس کے بعد، ہم کام \عددی{W} جاننا چاہیں گے۔ مساوات \حوالہ{مساوات_گھماو_کام_زاوی_الف} یا مساوات \حوالہ{مساوات_گھماو_زاوی_معادل_درمیانی}  سے  \عددی{W} اور قرص پر عمل پیرا قوت مروڑ کا تعلق لکھا جا سکتا ہے۔ دھاگے کی قوت \عددی{\vec{T}}  واحد قوت   ہے جس کی قوت مروڑ \عددی{(-TR)}  زاوی اسراع پیدا کر کے قرص پر کام سرانجام دیتی ہے۔ چونکہ \عددی{\alpha} مستقل ہے، لہٰذا یہ قوت مروڑ بھی مستقل ہو گی۔ یوں مساوات \حوالہ{مساوات_گھماو_زاوی_معادل_درمیانی} استعمال کی جا سکتی ہے، جس سے ذیل لکھا جاتا ہے۔
%eq 10.61
\begin{align}\label{مساوات_گھماو_نمونی_کام_حرکی}
W=\tau(\theta_f-\theta_i)=-TR(\theta_f-\theta_i)
\end{align}
چونکہ \عددی{\alpha} مستقل ہے،  مساوات \حوالہ{مساوات_گھماو_زاوی_ب} استعمال کر کے \عددی{\theta_f-\theta_i} معلوم کیا جا سکتا ہے۔ یوں \عددی{\omega_i=0} کے لئے ذیل ہو گا۔
\begin{align*}
\theta_f-\theta_i=\omega_i t+\tfrac{1}{2}\alpha t^2=0+\tfrac{1}{2}\alpha t^2=\tfrac{1}{2}\alpha t^2
\end{align*}
اس کو مساوات \حوالہ{مساوات_گھماو_نمونی_کام_حرکی} میں ڈال کر حاصل نتیجہ مساوات \حوالہ{مساوات_گھماو_نمونی_کام_حرکی_الف} میں پُر کرتے ہیں۔  دی گئی معلومات  \عددی{T=\SI{6.0}{\newton}} اور \عددی{\alpha=\SI{-24}{\radian\per\second\squared}} ڈال کر ذیل ہو گا۔
\begin{align*}
K&=W=-TR(\theta_f-\theta_i)=-TR(\tfrac{1}{2}\alpha t^2)=-\tfrac{1}{2}TR\alpha t^2\\
&=-\tfrac{1}{2}(\SI{6.0}{\newton})(\SI{0.20}{\meter})(\SI{-24}{\radian\per\second\squared})(\SI{2.5}{\second})^2\\
&=\SI{90}{\joule}\quad\quad\text{\RL{(جواب)}}
\end{align*}
\انتہا{نمونی سوال}
%-----------------------------

%Review & Summary p285
\جزوجزوحصہء{نظر ثانی اور خلاصہ}
\موٹا{زاوی مقام}\quad
مقررہ محور پر، جو \اصطلاح{ محور گھماو  } کہلاتی ہے،  استوار جسم کے  گھماو  کی بات کرتے ہوئے ، ہم فرض کرتے ہیں   کہ جسم کے ساتھ  ، محور گھماو کو عمودی\اصطلاح{ حوالہ  لکیر    } پکی جڑی ہے، جو جسم کے ساتھ ساتھ گھومتی ہے۔ کسی مخصوص مقررہ رخ کے لحاظ سے ہم  اس لکیر کا \اصطلاح{ زاوی مقام } \عددی{\theta}  ناپتے ہیں۔ جب \عددی{\theta} کی پیمائش \اصطلاح{ ریڈیئن } میں ہو، ذیل ہو گا، جہاں دائری راہ کی  قوسی لمبائی \عددی{s}، رداس \عددی{r}، اور زاویہ \عددی{\theta} ہے۔
\begin{align*}
\theta=\frac{s}{r}\quad\quad\text{\RL{(ریڈیئن   ناپ)}} \tag{\setlatin{\حوالہ{مساوات_گھماو_رداسی_فاصلہ_الف}}}
\end{align*}
ریڈیئن،   چکر ، اور درجات میں  ناپ کا تعلق ذیل ہے۔
\begin{align*}
\text{\RL{چکر}}\, 1=\SI{360}{\degree}=2\pi\,\si{\radian}\tag{\setlatin{\حوالہ{مساوات_گھماو_ریڈیئن_اور_درجے}}}
\end{align*}
\موٹا{زاوی ہٹاو}\quad
جب ایک  جسم محور گھماو پر گھوم کر اپنا زاوی مقام \عددی{\theta_1} سے تبدیل کر کے \عددی{\theta_2} کرے، جسم کا \اصطلاح{زاوی ہٹاو }ذیل ہو گا،
\begin{align*}
\Delta \theta=\theta_2-\theta_1 \tag{\setlatin{\حوالہ{مساوات_گھماو_زاوی_ہٹاو_تعریف}}}
\end{align*}
جہاں خلاف گھڑی گھماو کے لئے \عددی{\Delta \theta} مثبت اور گھڑی وار کے لئے منفی ہو گا۔

\موٹا{زاوی سمتی رفتار اور رفتار}\quad
اگر وقتی دورانیہ \عددی{\Delta t} میں  جسم  \عددی{\Delta \theta} زاوی ہٹاو   گھومے ، اس کی \اصطلاح{ اوسط زاوی سمتی رفتار } \عددی{\theta_{\text{\RL{اوسط}}}} ذیل ہو گی۔
\begin{align*}
\omega_{\text{\RL{اوسط}}}=\frac{\Delta \theta}{\Delta t}\tag{\setlatin{\حوالہ{مساوات_گھماو_اوسط_زاوی_سمتی_رفتار}}}
\end{align*}
جسم کی \اصطلاح{( لمحاتی ) زاوی سمتی رفتار } ذیل ہو گی۔
\begin{align*}
\omega=\frac{\dif \theta}{\dif t}\tag{\setlatin{\حوالہ{مساوات_گھماو_لمحاتی_زاوی_سمتی_رفتار}}}
\end{align*}
\عددی{\omega_{\text{\RL{اوسط}}}} اور \عددی{\omega}   سمتیات  ہیں، جن کا رخ\اصطلاح{ دائیں ہاتھ کا قانون } دیگا (شکل \حوالہء{10.6})۔  خلاف گھڑی گھماو کے لئے دونوں مثبت اور گھڑی وار گھماو کے لئے منفی ہوں گے۔ جسم کے زاوی  سمتی رفتار کی قدر اس کی\اصطلاح{ زاوی رفتار }کہلاتی ہے۔

\موٹا{زاوی اسراع}\quad
اگر \عددی{t_1} تا \عددی{t_2} کے  وقتی وقفہ \عددی{\Delta t} میں جسم کی زاوی سمتی رفتار \عددی{\omega_1} سے تبدیل ہو کر \عددی{\omega_2} ہو، جسم کا\اصطلاح{ اوسط زاوی اسراع }ذیل ہو گا۔
\begin{align*}
\alpha_{\text{\RL{اوسط}}}=\frac{\omega_2-\omega_1}{t_2-t_1}=\frac{\Delta \omega}{\Delta t} \tag{\setlatin{\حوالہ{مساوات_گھماو_زاوی_اوسط_اسراع}}}
\end{align*}
جسم کی \اصطلاح{(لمحاتی) زاوی اسراع ذیل ہو گا۔}
\begin{align*}
\alpha=\frac{\dif \omega}{\dif t} \tag{\setlatin{\حوالہ{مساوات_گھماو_زاوی_لمحاتی_اسراع}}}
\end{align*}
\عددی{\alpha_{\text{\RL{اوسط}}}} اور \عددی{\alpha} دونوں سمتیات ہیں۔

\موٹا{مستقل زاوی اسراع کی  مجرد حرکیات مساوات}\quad
\ترچھا{مستقل زاوی اسراع} \عددی{(\alpha=\text{\RL{مستقل}})}  گھمیری حرکت کی ایک خاص قسم  ہے۔ اس کی  مجرد حرکیات  مساوات،  جو جدول \حوالہ{جدول_گھماو_مستقل_اسراع_مساوات} میں دی گئی ہیں ، ذیل ہیں۔
\begin{align*}
\omega&=\omega_0+\alpha t \tag{\setlatin{\حوالہ{مساوات_گھماو_زاوی_الف}}}\\
\theta-\theta_0&=\omega_0 t+\tfrac{1}{2}\alpha t^2 \tag{\setlatin{\حوالہ{مساوات_گھماو_زاوی_ب}}}\\
\omega^2&=\omega_0^2+2\alpha(\theta-\theta_0) \tag{\setlatin{\حوالہ{مساوات_گھماو_زاوی_پ}}}\\
\theta-\theta_0&=\tfrac{1}{2}(\omega_0+\omega)t \tag{\setlatin{\حوالہ{مساوات_گھماو_زاوی_ت}}}\\
\theta-\theta_0&=\omega t-\tfrac{1}{2}\alpha t^2 \tag{\setlatin{\حوالہ{مساوات_گھماو_زاوی_ٹ}}}
\end{align*}

\موٹا{خطی اور زاوی متغیرات کا تعلق}\quad
گھومتے استوار  جسم کا اندرونی نقطہ، جو محور گھماو سے \عددی{r} \ترچھا{ عمودی فاصلہ }   پر ہو، رداس \عددی{r} کے دائرے پر حرکت کرتا ہے۔ اگر جسم زاویہ \عددی{\theta}  سے گھومے، یہ نقطہ   ذیل قوسی فاصلہ \عددی{s} طے کرتا ہے، جہاں \عددی{\theta}  کا  ناپ ریڈیئن میں ہے۔
\begin{align*}
s=\theta r\quad\quad\text{\RL{(ریڈیئن ناپ)}} \tag{\setlatin{\حوالہ{مساوات_گھماو_خطی_زاوی_تعلق_الف}}}
\end{align*}

نقطے کا خطی سمتی رفتار \عددی{\vec{v}} دائرے کو مماسی ہو گا؛ نقطہ کی خطی رفتار  \عددی{v} ذیل ہو گی،
\begin{align*}
v=\omega r\quad\quad\text{\RL{(ریڈیئن ناپ)}} \tag{\setlatin{\حوالہ{مساوات_گھماو_خطی_زاوی_تعلق_ب}}}
\end{align*}
جہاں \عددی{\omega} جسم کی (ریڈیئن فی سیکنڈ میں)  زاوی رفتار ہے۔

نقطے  کے خطی اسراع \عددی{\vec{a}}  کا \ترچھا{ مماسی } اور  \ترچھا{رداسی  } جزو ہو گا۔ مماسی جزو ذیل ہو گا،
\begin{align*}
a_t=\alpha r \quad\quad\text{\RL{(ریڈیئن ناپ)}} \tag{\setlatin{\حوالہ{مساوات_گھماو_اسراع_ب}}}
\end{align*}
جہاں (ریڈیئن فی مربع سیکنڈ میں)  جسم کے زاوی اسراع کی قدر  \عددی{\alpha} ہے۔ اسراع \عددی{\vec{a}} کا رداسی جزو ذیل ہو گا۔
\begin{align*}
a_r=\frac{v^2}{r}=\omega^2 r \quad\quad\text{\RL{(ریڈیئن ناپ)}} \tag{\setlatin{\حوالہ{مساوات_گھماو_رداسی_اندر_اسراع_جزو}}}
\end{align*}

اگر نقطہ  یکساں دائری حرکت کرتا ہو، جسم اور نقطے کی حرکت کا دوری عرصہ \عددی{T} ذیل ہو گا۔
\begin{align*}
T=\frac{2\pi r}{v}=\frac{2\pi}{\omega} \quad\quad\text{\RL{(ریڈیئن ناپ)}}
 \tag{\setlatin{\حوالہ{مساوات_گھماو_دوری_عرصہ_الف}،\, \حوالہ{مساوات_گھماو_دوری_عرصہ_ب}}}
\end{align*}

\موٹا{گھمیری حرکی توانائی اور گھمیری جمود}\quad
مقررہ محور پر گھومتے ہوئے استوار جسم کی حرکی توانائی \عددی{K} ذیل ہو گی،
\begin{align*}
K=\tfrac{1}{2}I\omega^2\quad\quad\text{\RL{(ریڈیئن ناپ)}} \tag{\setlatin{\حوالہ{مساوات_گھماو_حرکی_گھمیری_تعریف}}}
\end{align*}
جہاں \عددی{I} جسم کا \اصطلاح{ گھمیری جمود } ہے ، جس کی تعریف انفرادی ذروں کے نظام کے لئے:
\begin{align*}
I=\sum m_i r_i^2  \tag{\setlatin{\حوالہ{مساوات_گھماو_آئے_تعریف}}}
\end{align*}
اور استمراری  کمیتی تقسیم  کے جسم کے لئے ذیل ہے۔
\begin{align*}
I=\int r^2\dif m  \tag{\setlatin{\حوالہ{مساوات_گھماو_گھمیری_جمود_استمراری_الف}}}
\end{align*}
ان مساوات میں، محور گھماو سے مطلوبہ کمیتی ٹکڑے تک عمودی  فاصلہ \عددی{r_i} اور \عددی{r} ہے، اور تکمل پورے جسم پر لیا  جائے گا  تا کہ اس میں تمام کمیتی ٹکڑے شامل ہوں۔

\موٹا{مسئلہ متوازی محور}\quad
کسی بھی محور پر جسم کے گھمیری جمود \عددی{I}  کا تعلق،  اسی جسم کے مرکز کمیت پر متوازی محور کے لحاظ سے گھمیری جمود  کے ساتھ\ترچھا{ مسئلہ متوازی محور } دیتا ہے۔
\begin{align*}
I=I_{\text{\RL{مرکزکمیت}}}+Mh^2  \tag{\setlatin{\حوالہ{مساوات_گھماو_مسئلہ_متوازی_محور}}}
\end{align*}
یہاں دونوں محور کے بیچ فاصلہ \عددی{h} ہے، اور   مرکز کمیت پر محور کے  لحاظ سے جسم کا گھمیری جمود \عددی{I_{\text{\RL{مرکزکمیت}}}}  ہے۔ ہم \عددی{h} کو  مرکز کمیت پر واقع محور  سے اصل محور گھماو   کا ہٹاو تصور کر سکتے ہیں۔

\موٹا{قوت مروڑ}\quad
گھمیری محور پر قوت \عددی{\vec{F}} کی بنا  جسم  پر گھومنے کے اثر کو\ترچھا{ قوت مروڑ } کہتے ہیں۔ اگر محور گھماو کے لحاظ سے  جس نقطے پر \عددی{\vec{F}} عمل پیرا ہو   اس کا تعین گر سمتیہ \عددی{\vec{r}} ہو، تب  قوت مروڑ کی قدر ذیل ہو گی،
\begin{align*}
\tau=rF_t=r_{\perp}F=rF\sin\phi 
\tag{\setlatin{\حوالہ{مساوات_گھماو_صافی_قوت_مروڑ_الف}،\, \حوالہ{مساوات_گھماو_قوت_مروڑ_ذرے_پر}،\, \حوالہ{مساوات_گھماو_قوت_مروڑ_فائے}}}
\end{align*}
جہاں \عددی{\vec{r}} کو \عددی{\vec{F}} کا عمودی جزو \عددی{F_t} ،   اور   \عددی{\vec{r}} اور \عددی{\vec{F}} کے بیچ زاویہ \عددی{\phi} ہے۔محور گھماو اور  \عددی{\vec{F}} سمتیہ سے       گزرتی مبسوط لکیر   کے بیچ عمودی فاصلہ \عددی{r_{\perp}} ہے اس لکیر کو \عددی{\vec{F}} کا \اصطلاح{ خط عمل } کہتے ہیں، اور \عددی{r_{\perp}}  کو \عددی{\vec{F}} کے \اصطلاح{ معیار اثر کا بازو } کہتے ہیں۔ اسی طرح \عددی{F_t} کے معیار اثر کا بازو \عددی{r} ہے۔

قوت مروڑ کی بین الاقوامی اکائی نیوٹن میٹر \عددی{(\si{\newton\meter})} ہے۔ اگر ساکن جسم کو قوت مروڑ  \عددی{\tau} خلاف گھڑی گھمانے کی کوشش کرے، \عددی{\tau} مثبت ہو گی اور اگر گھڑی وار گھمانے کی کوشش کرے تب منفی ہو گی۔

%p286
\موٹا{نیوٹن کے  قانون دوم کا زاوی روپ}\quad
نیوٹن کے قانون دوم کا زاوی مماثل ذیل ہے،
\begin{align*}
\tau_{\text{\RL{صافی}}}=I\alpha \tag{\setlatin{\حوالہ{مساوات_گھماو_ثبوت_پ}}}
\end{align*}
جہاں ذرے  یا استوار جسم پر قوت مروڑ \عددی{\tau_{\text{\RL{صافی}}}}، محور گھماو پر ذرے یا جسم کا گھمیری جمود \عددی{I}، اور \عددی{\alpha}   اس محور پر ماحصل زاوی اسراع ہے۔

\موٹا{کام اور گھمیری حرکی توانائی}\quad
گھمیری  حرکت  میں کام اور طاقت کے حساب  کی (درج ذیل)  مساوات  مستقیم حرکت کی مساوات  سے مطابقت رکھتی ہیں۔
\begin{align*}
W&=\int_{\theta_i}^{\theta_f}\tau\dif\theta \tag{\setlatin{\حوالہ{مساوات_گھماو_کام_زاوی_الف}}}\\
P&=\frac{\dif W}{\dif t}=\tau\omega \tag{\setlatin{\حوالہ{مساوات_گھماو_زاوی_معادل_اخیر}}}
\end{align*}
جب \عددی{\tau} مستقل ہو مساوات  \حوالہ{مساوات_گھماو_کام_زاوی_الف} گھٹ کر ذیل روپ اختیار کرتی ہے۔
\begin{align*}
W=\tau(\theta_f-\theta_i)\tag{\setlatin{\حوالہ{مساوات_گھماو_زاوی_معادل_درمیانی}}}
\end{align*}
گھومتے اجسام کے لئے مسئلہ کام و حرکی توانائی ذیل روپ اختیار کرتا ہے۔
\begin{align*}
\Delta K=K_f-K_i=\tfrac{1}{2}I\omega_f^2-\tfrac{1}{2}I\omega_i^2=W\tag{\setlatin{\حوالہ{مساوات_گھماو_زاوی_معادل_الف}}}
\end{align*}

%------------------------------------------
%Questions p286
\حصہء{سوالات}
\setcounter{questioncounter}{0}
%Q1 p286
\ابتدا{سوال}
انتصابی  دھرے  پر قرص  کی زاوی سمتی رفتار بالمقابل وقت  ترسیم شکل \حوالہء{10.20} میں پیش ہے۔ قرص کے چکا پر ایک   نقطہ  کے لئے  لمحات  \عددی{a}، \عددی{b}، \عددی{c}، اور \عددی{d} کی درجہ بندی، اعظم اول رکھ کر، (ا) مماسی اور (ب) رداسی اسراع کی قدر کے لحاظ سے کریں۔
\انتہا{سوال}
%-----------------------------
\ابتدا{سوال}
انتصابی  دھرے  پر  قرص  کے گھماو کی تین صورتوں کے لئے زاوی مقام \عددی{\theta} بالمقابل قوت \عددی{t} شکل \حوالہء{10.21} میں پیش ہے۔ ہر ایک صورت میں گھماو کا رخ کسی زاوی مقام \عددی{\theta_{\text{\RL{واپس}}}}  ہو گا۔ (ا) ہر صورت کے لئے کیا   \عددی{\theta=0} کے لحاظ سے  \عددی{\theta_{\text{\RL{واپس}}}}   گھڑی وار ہے، خلاف گھڑی ہے، یا عین \عددی{\theta=0} پر ہے؟  ہر ایک صورت میں (ب) کیا  \عددی{t=0}   سے قبل، اس کے بعد، یا اسی لمحے \عددی{\omega} صفر ہو گا اور (ج) کیا \عددی{\alpha}  مثبت، منفی، یا صفر ہو گا؟
\انتہا{سوال}
%----------------------------
\ابتدا{سوال}
 قرص کے وسط سے گزرتا  انتصابی  دھرے  پر گھومتے  قرص کے چکا  پر قوت لاگو کر کے اس کی زاوی سمتی رفتار  تبدیل کی جاتی ہے۔ اس کی بالترتیب  ابتدائی اور اختتامی سمتی رفتار چار مختلف صورتوں میں ذیل ہیں:
 (ا) ابتدائی \عددی{\SI{-2}{\radian\per\second}}، اختتامی \عددی{\SI{5}{\radian\per\second}}؛ 
 (ب)  \عددی{\SI{2}{\radian\per\second}}، \عددی{\SI{5}{\radian\per\second}}؛ 
 (ج)   \عددی{\SI{-2}{\radian\per\second}}، \عددی{\SI{-5}{\radian\per\second}}؛ 
 (د)   \عددی{\SI{2}{\radian\per\second}}، \عددی{\SI{-5}{\radian\per\second}}۔
 اعظم قیمت اول رکھ کر ان صورتوں کی درجہ بندی  قوت مروڑ کے سرانجام کام کے لحاظ سے کریں۔
\انتہا{سوال}
%-----------------------------
%Q4 p286
\ابتدا{سوال}
شکل \حوالہء{10.22a} کے قرص کا زاوی مقام  شکل \حوالہء{10.22b}  دیتی ہے۔ کیا  (ا) \عددی{t=\SI{1}{\second}} پر، (ب)    \عددی{t=\SI{2}{\second}} پر، اور (ج)  \عددی{t=\SI{3}{\second}} پر اس کی زاوی سمتی رفتار مثبت، منفی، یا صفر ہے؟ (د) کیا زاوی اسراع مثبت یا منفی ہے؟
\انتہا{سوال}
%-------------------------------------
\ابتدا{سوال}
 قرص کے وسط سے گزرتا انتصابی  دھرے  پر گھومتے قرص  پر قوت \عددی{\vec{F}_1} اور \عددی{\vec{F}_2} عمل کرتی ہیں (شکل \حوالہء{10.23})۔گھماو کے دوران، جو خلاف گھڑی  اور مستقل ہے،  قوت دکھائے گئے زاویے  برقرار رکھتی ہیں۔ تاہم، ہم چاہتے ہیں کہ \عددی{\vec{F}_1} کی قدر تبدیل کیے بغیر \عددی{\vec{F}_1} کا زاویہ \عددی{\theta}  گھٹائیں۔ (ا)  سمتی زاوی رفتار تبدیل نہ ہونے کے لئے کیا  \عددی{\vec{F}_2} کی قدر بڑھانی ہو گی، گھٹانی ہو گی، یا  برقرار رکھنی ہو گی؟ کیا (ب) \عددی{\vec{F}_1} اور (ج) \عددی{\vec{F}_2} قرص کو گھڑی وار یا خلاف گھڑی گھمانے کی کوشش کرتی ہیں؟
\انتہا{سوال}
%-----------------------------
%Q6 p286
\ابتدا{سوال}
ایک چوکور جو نقطہ \عددی{P}  پر د انتصابی دھرے  کے گرد  گھوم سکتا ہے ، کا فضائی جائزہ شکل \حوالہء{10.24} میں لیا گیا ہے۔چوکور پر برابر قدر کی پانچ قوت عمل کرتی ہیں، اور \عددی{P} ضلع کا وسطی نقطہ  ہے۔ نقطہ \عددی{P} پر قوت مروڑ   کے لحاظ سے، اعظم اول رکھ کر، قوتوں کی درجہ بندی کریں۔
\انتہا{سوال}
%-------------------------
\ابتدا{سوال}
افقی چول دار  سلاخ کا فضائی جائزہ   شکل \حوالہء{10.25a} میں پیش ہے۔ سلاخ پر دو قوت عمل کرتی ہیں، تاہم سلاخ ساکن رہتا ہے۔اب  اگر سلاخ اور \عددی{\vec{F}_2} کے بیچ زاویہ \عددی{\SI{90}{\degree}} سے گھٹائیں اور سلاخ اب بھی ساکن رہے، کیا \عددی{\vec{F}_2} بڑھانی  ہو گی، گھٹانی ہو گی، یا برقرار رکھنی ہو گی؟
\انتہا{سوال}
%-------------------------
%Q8 p286
\ابتدا{سوال}
افقی چول دار  سلاخ کا فضائی جائزہ   شکل \حوالہء{10.25b} میں پیش ہے۔ سلاخ   کو قوت \عددی{\vec{F}_1} اور \عددی{\vec{F}_2} چول پر    گھماتی ہیں؛ \عددی{\vec{F}_2} اور سلاخ کے بیچ زاویہ \عددی{\phi} ہے۔سلاخ کے زاوی اسراع کی قدر کے لحاظ سے ، اعظم اول رکھ کر، زاویہ \عددی{\phi} کی درج ذیل قیمتوں کی درجہ بندی کریں: \عددی{\SI{90}{\degree}}،  \عددی{\SI{70}{\degree}}، اور  \عددی{\SI{110}{\degree}}۔
\انتہا{سوال}
%---------------------------
\ابتدا{سوال}
یکساں موٹائی   کے دھاتی  چادر کا چوکور جس سے \عددی{\SI{25}{\percent}} حصہ کاٹا گیا ہے، شکل \حوالہء{10.26} میں دکھایا گیا ہے۔ شکل پر تین حرفی نقطے دیے گئے ہیں۔ان نقطوں پر انتصابی محور   کے گرد چادر کے گھمیری جمود کے لحاظ سے ، اعظم اول رکھ کر،  نقطوں کی درجہ بندی   کریں۔
\انتہا{سوال}
%-------------------------------
%Q10 p287
\ابتدا{سوال}
تین چپٹے (ایک جتنے رداس کے)  قرص ، جو قرص کے وسط پر  انتصابی دھرے کے گرد گھوم سکتے ہیں، شکل \حوالہء{10.27} میں پیش ہیں۔ تینوں قرص  وہی   دو مادہ سے بنے ہیں۔ ایک مادہ  دوسرے سے زیادہ کثیف ہے (فی اکائی حجم کمیت کو کثافت کہتے ہیں)۔ قرص \عددی{1} اور \عددی{3} کا بیرونی نصف حصہ کثیر مادے کا ہے۔ قرص \عددی{2}  کا اندرونی نصف حصہ کثیف مادے کا  ہے۔ ایک جتنی  قدر کی دو قوتیں قرص کے بیرونی کنارے پر یا دو مادہ   کے جوڑ پر،  مماسی عمل کرتی ہیں۔  (ا) قرص کے وسط  پر قوت مروڑ، (ب)  قرص کے وسط پر گھمیری جمود، اور (ج) قرص کے اسراع کے لحاظ سے ، اعظم اول رکھ کر، قرص کی درجہ بندی کریں۔
\انتہا{سوال}
%--------------------------------
\ابتدا{سوال}
میٹر سلاخ کا  آدھا حصہ لکڑی کا اور آدھا  فولاد کا بنا ہو ہے (شکل \حوالہء{10.28a})۔ لکڑی  والے سر \عددی{O}  پر  چول ہے۔ فولادی سر \عددی{a} پر قوت \عددی{\vec{F}} عمل کرتی ہے۔ شکل \حوالہء{10.28b} میں سلاخ الٹی رکھی جاتی ہے اور  فولادی سر \عددی{O'} پر چول  جبکہ لکڑی والے سر \عددی{a'} پر قوت لاگو کی جاتی ہے۔ کیا شکل \حوالہء{10.28a}  میں پیدا زاوی اسراع  شکل \حوالہء{10.28b} میں پیدا زاوی  اسراع سے زیادہ، کم، یا اس کے برابر ہے؟
\انتہا{سوال}
%------------------------
%Q12 p287
\ابتدا{سوال}
یکساں کمیتی تقسیم کے تین قرص شکل \حوالہء{10.29} میں پیش ہیں۔ قرص کا  رداس \عددی{R} اور کمیت \عددی{M} دیے گئے ہیں۔ قرص  کے وسط پر  قرص کو عمودی محور گھماو کے گرد قرص گھوم سکتے ہیں۔اپنے اپنے  محور گھماو  پر گھمیری جمود کے لحاظ سے ، اعظم اول رکھ کر، قرص کی درجہ بندی کریں۔
\انتہا{سوال}
%--------------------------

%Problems p287
\جزوحصہء{سوالات}
\setcounter{questioncounter}{0}
%Module 10.1 Rotational Variables p287
\جزوحصہء{گھمیری متغیرات}
%--------------------------
%Q1 p287
\ابتدا{سوال}
ایک اچھا کھلاڑی \عددی{60} فٹ دور  کھلاڑی تک \عددی{85}  میل فی گھنٹہ کی رفتار   اور \عددی{1800} چکر فی منٹ  کے گھماو سے  گیند  پھینک سکتا ہے۔ دوسرے کھلاڑی تک پہنچ کر گیند نے کتنے چکر مکمل کیے ہوں گے؟
\انتہا{سوال}
%---------------------------------------------
\ابتدا{سوال}
گھڑی کی (ا) سیکنڈوں کی سوئی، (ب) منٹوں کی سوئی، اور (ج) گھنٹوں کی سوئی  کی زاوی رفتار  ریڈیئن فی سیکنڈ میں تلاش کریں۔
\انتہا{سوال}
%----------------------------------------
\ابتدا{سوال}
  ڈبل روٹی  کا  مکھن لگا ٹکڑا میز  سے پھسل کر زمین پر چکر کھاتا  گرتا ہے۔ میز سے زمین تک فاصلہ \عددی{\SI{76}{\centi\meter}} اور   \عددی{1} سے  کم چکر کی صورت میں (ا) کم سے کم اور (ب) زیادہ سے زیادہ   زاوی رفتار  کیا ہو گی کہ زمین پر  لگنے کے بعد   مکھن    لگا  طرف زمین پر  ہو؟
\انتہا{سوال}
%--------------------------
%Q4 p287
\ابتدا{سوال}
گھومتے پہیے پر ایک نقطے کا زاوی مقام \عددی{\theta=2.0+4.0t^2+2.0t^3} ہے، جہاں \عددی{\theta} کا  ناپ ریڈیئن اور  \عددی{t} کا  سیکنڈ میں ہے۔ لمحہ \عددی{t=0} پر  نقطے کا (ا)  زاوی مقام   اور  (ب) زاوی سمتی رفتار  کیا ہو گا؟ لمحہ \عددی{t=\SI{4.0}{\second}} پر  اس کا زاوی سمتی رفتار کیا ہو گا؟ (ج) لمحہ \عددی{t=\SI{2.0}{\second}} پر اس کا زاوی اسراع تلاش کریں۔ (د) کیا اس کا زاوی اسراع مستقل ہے؟
\انتہا{سوال}
%--------------------------
\ابتدا{سوال}
 پانی تک \عددی{\SI{10}{\meter}} بلند چبوترہ سے  تیراک \عددی{2.5} چکر کھا  کر  پہنچتا ہے۔صفر ابتدائی انتصابی سمتی رفتار فرض کر کے،   پرواز کے دوران  تیراک کی اوسط زاوی سمتی رفتار تلاش کریں۔
\انتہا{سوال}
%-------------------------
\ابتدا{سوال}
گھومتے پہیے کے چکا  پر  ایک نقطے کا زاوی مقام \عددی{\theta=4.0t-3.0t^2+t^3} ہے، جہاں \عددی{\theta} کا ناپ ریڈیئن اور  \عددی{t} کا سیکنڈ میں ہے۔  لمحہ  (ا) \عددی{t=\SI{2.0}{\second}} اور (ب)  \عددی{t=\SI{4.0}{\second}}   پر  زاوی سمتی رفتار کیا ہوں گی؟  (ج)  وقت \عددی{t=\SI{2.0}{\second}}  سے \عددی{t=\SI{4.0}{\second}}  تک دورانیے میں  اوسط زاوی اسراع کیا ہو گا؟ اس دورانیے کے (ج) آغاز میں اور (د) اختتام پر لمحاتی زاوی اسراع کیا ہو گا؟
\انتہا{سوال}
%----------------------
%Q7 p287
\ابتدا{سوال}
ایک پہیا میں،  جس کا  رداس \عددی{\SI{30}{\centi\meter}} ہے،   آٹھ  تیلیاں  برابر فاصلوں پر نصب ہیں۔ پہیا مقررہ دھرے پر \عددی{2.5} چکر فی سیکنڈ گھوم رہا ہے۔ آپ \عددی{\SI{20}{\centi\meter}} لمبا تیر   مار کر، دھرے  کے متوازی  ، تیلیوں کو چھوئے بغیر ،  پہیے کے اندر  سے  گزارنا چاہتے ہیں۔ تیر اور تیلیوں کو انتہائی پتلا تصور کریں۔ (ا) تیر کی کم سے کم رفتار کیا ہو سکتی ہے؟ (ب) کیا دھرے اور چکا کے بیچ  مارنے کا نقطہ اہمیت رکھتا ہے؟ اگر اہمیت رکھتا ہو، بہترین مقام کیا ہو گا؟
\انتہا{سوال}
%--------------------------
\ابتدا{سوال}
پہیے کا زاوی اسراع \عددی{\alpha=6.0t^4-4.0t^2} ہے ، جہاں \عددی{\alpha} کا ناپ  ریڈیئن فی مربع سیکنڈ اور \عددی{t} کا سیکنڈ میں ہے۔ وقت \عددی{t=0} پر پہیے کی زاوی سمتی
 رفتار \عددی{+\SI{2.0}{\radian\per\second}} اور زاوی مقام \عددی{+\SI{1.0}{\radian}} ہے۔ (ا) زاوی سمتی رفتار \عددی{ (\si{\radian\per\second})}   اور (ب) زاوی مقام (ریڈیئن)  کے  تفاعل وقت (سیکنڈ)  کے لحاظ سے لکھیں۔
\انتہا{سوال}
%--------------------------
%Module 10.2 Rotation with constant angular acceleration p287
\جزوحصہء{مستقل زاوی اسراع کا گھماو}
%Q9 p287
\ابتدا{سوال}
اپنے وسطی محور پر  ڈرم \عددی{\SI{12.60}{\radian\per\second}} زاوی سمتی رفتار سے گھوم رہا ہے۔اگر  اب ڈرم \عددی{\SI{4.20}{\radian\per\second\squared}} کی  مستقل شرح سے آہستہ ہو، اس کو رکنے تک  (ا) کتنا وقت چاہیے ہو گا اور (ب) رکنے تک یہ کتنا زاویہ گھومے گا؟
\انتہا{سوال}
%----------------------------
\ابتدا{سوال}
ساکن حالت سے آغاز کر کے ایک قرص اپنے وسطی محور  پر مستقل زاوی اسراع سے گھومتا ہے۔ ابتدائی \عددی{\SI{5.0}{\second}} میں قرص \عددی{\SI{25}{\radian}} گھومتا ہے۔ اس دورانیہ میں (ا) زاوی اسراع اور (ب) اوسط زاوی سمتی رفتار کی قدر کیا ہیں؟ (ج)  اس \عددی{\SI{5.0}{\second}}  دورانیے کے اختتام پر لمحاتی سمتی رفتار  کیا ہو گی؟ (د)   زاوی اسراع برقرار رہنے کی صورت میں اگلے \عددی{\SI{5.0}{\second}} میں قرص مزید  کتنا  زاویہ طے کرتا ہے؟
\انتہا{سوال}
%------------------------------
\ابتدا{سوال}
ایک قرص جو ابتدائی طور \عددی{\SI{120}{\radian\per\second}} سے گھوم رہا ہے، \عددی{\SI{4.0}{\radian\per\second\squared}} قدر کے مستقل اسراع سے آہستہ ہوتا ہے۔ (ا)قرص کے  رکھنے  تک کتنا وقت  درکار ہو گا؟  (ب)   رکھنے تک قرص کتنا زاویہ طے کریگا؟
\انتہا{سوال}
%-------------------------------
%Q12 p287
\ابتدا{سوال}
ایک گاڑی کی\اصطلاح{ کل }\فرہنگ{کل}\حاشیہب{engine}\فرہنگ{engine} ( انجن )کی زاوی رفتار \عددی{\SI{12}{\second}} میں  \عددی{1200} چکر فی منٹ سے بڑھا کر \عددی{3000} چکر فی منٹ کی جاتی ہے۔ (ا)  اس کا اسراع چکر فی مربع  منٹ میں کیا ہو گا؟ (ب) ان \عددی{\SI{12}{\second}} میں کل (انجن) کتنے چکر کاٹتی ہے؟
\انتہا{سوال}
%-----------------------------------
%Q13 P288
\ابتدا{سوال}
 \اصطلاح{ اڑن پہیا }\فرہنگ{اڑن پہیا}\حاشیہب{flywheel}\فرہنگ{flywheel} \عددی{40}     چکروں میں \عددی{\SI{1.5}{\radian\per\second}}   زاوی رفتار  سے  ساکن حالت کو پہنچتا ہے۔ (ا)  مستقل زاوی اسراع فرض کرتے ہوئے، رکنے کے لئے درکار وقت معلوم کریں۔ (ب)  اس کا زاوی اسراع کیا ہو گا؟ (ج)  \عددی{40} چکر میں سے ابتدائی \عددی{20} چکر  اڑن پہیا کتنے  وقت میں کاٹتا ہے؟
\انتہا{سوال}
%--------------------------------------
\ابتدا{سوال}
ساکن حالت سے آغاز کر کے،   مستقل اسراع کے ساتھ، اپنی وسطی محور پر  قرص گھومتا ہے۔ کسی ایک لمحے قرص \عددی{10}  چکر فی سیکنڈ سے گھومتا ہے؛ \عددی{60} چکر بعد اس کی زاوی رفتار \عددی{15} چکر فی سیکنڈ ہے۔ (ا) قرص کا زاوی اسراع، (ب)یہ   \عددی{60}   چکر کو درکار دورانیہ، (ج)  \عددی{10} چکر فی سیکنڈ رفتار تک پہنچنے  کے لئے درکار دورانیہ، اور (د)  ساکن حالت سے \عددی{10} چکر فی سیکنڈ رفتار تک پہنچنے تک کل چکر تلاش کریں۔
\انتہا{سوال}
%--------------------------------
\ابتدا{سوال}
قرص کے وسطی نقطہ سے گزرتی انتصابی دھرے  پر  ساکن حالت سے قرص آغاز کر کے \عددی{\alpha=\SI{3.0}{\radian\per\second\squared}} سے چل پڑتا ہے۔ کسی
 مخصوص \عددی{\SI{4.0}{\second}} دورانیے میں قرص \عددی{120\pi}  ریڈیئن گھومتا ہے۔قرص کتنے  وقت  میں \عددی{\SI{4.0}{\second}}دورانیے کو پہنچتا ہے؟
\انتہا{سوال}
%----------------------------------
%Q16 p288
\ابتدا{سوال}
ساکن حالت سے آغاز کر کے محور گھماو پر قرص   زاوی اسراع \عددی{\SI{1.50}{\radian\per\second\squared}} سے چلتا ہے۔(ا)  ابتدائی \عددی{2.00} چکر  اور (ب) اگلے \عددی{2.00} چکر کتنے وقت میں طے ہوں گے؟
\انتہا{سوال}
%------------------
\ابتدا{سوال}
لمحہ \عددی{t=0} پر اڑن پہیے کی زاوی سمتی رفتار \عددی{\SI{4.7}{\radian\per\second}}،مستقل  زاوی اسراع \عددی{\SI{-0.25}{\radian\per\second\squared}}، اور    حوالہ لکیر  \عددی{\theta_0=0}  پر ہے۔ (ا) حوالہ لکیر مثبت رخ  زیادہ سے زیادہ کتنا زاویہ \عددی{\theta_{\text{\RL{بلندتر}}}}طے کرے گی؟ کس وقت حوالہ لکیر  (ب) پہلی مرتبہ اور (ج) دوسری مرتبہ   \عددی{\theta=\tfrac{1}{2}\theta_{\text{\RL{بلندتر}}}} پر ہو گی؟ کس (د) منفی وقت   اور (ہ) مثبت وقت پر  حوالہ لکیر \عددی{\theta=\SI{10.5}{\radian}}  پر ہو گی؟ (و) \عددی{\theta} بالمقابل \عددی{t} ترسیم کر کے اس پر اپنے جوابات  کی نشاندہی کریں۔
\انتہا{سوال}
%----------------------
%Q18 p288
\ابتدا{سوال}
\اصطلاح{ نابض }\فرہنگ{نابض}\حاشیہب{pulsar}\فرہنگ{pulsar}  تیزی سے گھومتے   نیوٹران  تارہ  کو کہتے ہیں جو منارہ  نور  کی طرح شعاع  خارج کرتا ہے۔ نابض  ہر چکر   کے دوران زمین پر ایک  مرتبہ شعاع مارتا ہے۔ دو  متواتر شعاعوں کے بیچ دورانیہ   ناپ کر گھومنے کا دوری عرصہ \عددی{T}   معلوم کیا جاتا ہے۔ \اصطلاح{ سدیم السرطان }\فرہنگ{سدیم!السرطان}\حاشیہب{Crab nebula}\فرہنگ{nebula!Crab}  میں موجود نابض کا دوری عرصہ \عددی{T=\SI{0.033}{\second}} ہے، جو ایک سال میں  \عددی{\num{1.26e-5}} سیکنڈ  شرح سے  بڑھ رہا ہے۔ (ا) نابض کا    زاوی اسراع  \عددی{\alpha} کیا ہے؟ (ب)  اگر \عددی{\alpha} مستقل ہو،نابض  آج سے کتنے سال  بعد  رک جائے گا؟ (ج)  یہ نابض \سن{1054}  میں دیکھے گئے   \اصطلاح{   مستعر اعظم   }\فرہنگ{مستعر اعظم}\حاشیہب{supernova}\فرہنگ{supernova}  دھماکے میں پیدا ہوا۔ مستقل \عددی{\alpha} تصور کر  کے ابتدائی \عددی{T} تلاش کریں۔
\انتہا{سوال}
%-------------------------
%module 10.3 relating the linear and angular variables p288
\جزوحصہء{خطی اور زاوی متغیرات کا تعلق}
%Q19 p288
\ابتدا{سوال}
خلائی طیارہ \عددی{\SI{29000}{\kilo\meter\per\hour}} رفتار سے چلتے ہوئے \عددی{\SI{3220}{\kilo\meter}} رداس کا  دائری  موڑ کاٹتا  ہے۔طیارے   (ا) کی زاوی سمتی رفتار، (ب) رداسی اسراع، اور (ج) مماسی اسراع کی قدریں   کیا  ہیں؟
\انتہا{سوال}
%---------------------
\ابتدا{سوال}
ایک جسم مقررہ محور  پر گھومتا ہے، اور  جسم پر حوالہ لکیر ما زاوی مقام \عددی{\theta=0.40e^{2t}} ہے، جہاں  \عددی{\theta} ریڈیئن میں اور \عددی{t} سیکنڈوں  میں ہے۔ محور گھماو سے
 \عددی{\SI{4.0}{\centi\meter}} فاصلے پر  نقطہ ہے۔ لمحہ \عددی{t=0} پر  نقطے  (ا) کے اسراع کے مماسی جزو  اور (ب) اسراع کے رداسی جزو کی قدر کیا ہو گی؟
\انتہا{سوال}
%---------------------
%Q21 p288
\ابتدا{سوال}
\سن{1911} اور \سن{1990} کے بیچ  اطالیہ   کے شہر  پیسا  میں واقع  \اصطلاح{جھکا بُرج }\فرہنگ{پیسا کا جھکا برج}\حاشیہب{leaning tower of Pisa}\فرہنگ{tower!leaning, Pisa} کی چوٹی    جنوب کے رخ سالانہ  اوسطاً \عددی{\SI{1.2}{\milli\meter}}   حرکت کرتی رہی۔ بُرج \عددی{\SI{55}{\meter}} بلند ہے۔ بُرج کے پیندا پر  بُرج کی زاوی رفتار ریڈیئن فی سیکنڈ میں  کتنی  ہے؟ 
\انتہا{سوال}
%----------------------
\ابتدا{سوال}
خلا باز  کو  \عددی{\SI{10}{\meter}} رداس کے \اصطلاح{مرکز گریزہ }\فرہنگ{مرکز گریزہ}\حاشیہب{centrifuge}\فرہنگ{centrifuge}  میں \عددی{\theta=0.30t^2} کے لحاظ سے گھما کر  جانچا جاتا ہے۔ وقت \عددی{t=\SI{5.0}{\second}} پر (ا) زاوی سمتی رفتار، (ب) خطی سمتی رفتار، (ج) مماسی اسراع، اور (د) رداسی اسراع کی قدریں کیا ہوں گی؟
\انتہا{سوال}
%-------------------------
%Q23 p288
\ابتدا{سوال}
ایک اڑن پہیا جس کا قطر \عددی{\SI{1.20}{\meter}} ہے \عددی{200} چکر فی منٹ کی زاوی رفتار سے گھوم رہا ہے۔ (ا)  اڑن پہیے کی زاوی رفتار ریڈیئن فی سیکنڈ میں کتنی ہے؟ (ب) اڑن پہیے کے چکا پر نقطے کی خطی رفتار کیا ہو گی؟ (ج)  پہیے کی زاوی رفتار \عددی{\SI{60}{\second}} میں بڑھا کر   \عددی{1000} چکر فی منٹ   کرنے کے لئے مستقل زاوی اسراع (چکر فی مربع  منٹ میں) کیا ہو گا؟
\انتہا{سوال}
%----------------------------
\ابتدا{سوال}
\اصطلاح{گراموفون  }\فرہنگ{گراموفون}\حاشیہب{gramophone}\فرہنگ{gramophone} کی سوئی    (پلاسٹک کی بنی ہوئی)  \اصطلاح{تھالی  }\فرہنگ{تھالی}\حاشیہب{vinyl record}\فرہنگ{record!vinyl}کی چوڑیوں  پر چل کر آواز پیدا کرتی ہے۔چوڑی میں  پیچ و خم  پر چل کر سوئی ارتعاش پذیر ہو گی۔گراموفون  میکانی  ارتعاش کو   پہلے     برقی ارتعاش میں اور اس کے بعد  آواز میں تبدیل کرتا ہے۔ فرض کریں تھالی \عددی{33\tfrac{1}{2}} چکر فی منٹ  شرح سے گھومتی ہے، جس چوڑی کو بجایا جا رہا ہے، اس کا رداس \عددی{\SI{10.0}{\centi\meter}} ہے، اور  چوڑی میں خم یکساں \عددی{\SI{1.75}{\milli\meter}} فاصلوں پر پائے جاتے ہیں۔ خم کس شرح  (ٹکر فی سیکنڈ) سے سوئی کو ٹکراتے ہیں؟
\انتہا{سوال}
%-----------------------------
\ابتدا{سوال}
(ا) سطح زمین پر   \عددی{\SI{40}{\degree}}  شمال کے \اصطلاح{خط عرض بلد  }\فرہنگ{خط عرض بلد}\حاشیہب{latitude}\فرہنگ{latitude} پر  واقع نقطے  کی  قطبی محور پر زاوی رفتار \عددی{\omega} کیا ہو گی؟ (زمین قطبی محور پر گھومتی ہے۔) (ب)   اس نقطے کی خطی رفتار \عددی{v} کیا ہو گی؟\اصطلاح{  خط استوا }\فرہنگ{خط استوا}\حاشیہب{equator}\فرہنگ{equator} پر واقع  نقطہ کی (ج) \عددی{\omega} اور (د) \عددی{v} کیا ہوں گی؟
\انتہا{سوال}
%----------------------------
%Q26 p288
\ابتدا{سوال}
\اصطلاح{دخانی  کل }\فرہنگ{دخانی کل}\حاشیہب{steam engine}\فرہنگ{steam engine} (دخانی انجن  ) کا اڑن پہیا  \عددی{150} چکر فی منٹ کی مستقل زاوی سمتی رفتار سے حرکت میں ہے۔ بھاپ  روکنے  پر \اصطلاح{ بیرم }\فرہنگ{بیرم}\حاشیہب{bearing}\فرہنگ{bearing} کی   رگڑ اور ہوائی رکاوٹ پہیے کو \عددی{2.2}  گھنٹوں میں روکتی ہیں۔ (ا)  رکنے کے دوران پہیے کا مستقل زاوی اسراع ، چکر فی مربع منٹ میں، کیا ہو گا؟ (ب)  رکنے تک پہیا کتنے چکر کاٹتا ہے؟ (ج)  جس لمحہ اڑن پہیے کی زاوی رفتار \عددی{75} چکر فی منٹ ہے، پہیے پر  محور گھماو سے \عددی{\SI{50}{\centi\meter}} فاصلے پر نقطے  کے خطی  اسراع کا مماسی جزو  کیا ہو گا؟ (د)  ذرے کے صافی اسراع کی قدر کیا ہو گی؟
\انتہا{سوال}
%------------------
\ابتدا{سوال}
   \اصطلاح{ تختہ  گھوم  }\فرہنگ{تختی گھوم}\حاشیہب{turntable}\فرہنگ{turntable} پر ، جو \عددی{33\tfrac{1}{3}} چکر فی منٹ سے گھوم رہا ہے،   بیج  کا دانہ  محور گھماو سے \عددی{\SI{6.0}{\centi\meter}} فاصلے پر پڑا ہے۔ (ا) بیج کا اسراع کیا ہے اور (ب) پھسلنے سے بچنے کے لئے   کم سے کم سکونی رگڑ کا مستقل  کیا ہو گا؟ (ج)  اگر ساکن حالت سے تختہ  اس رفتار تک \عددی{\SI{0.25}{\second}} میں مستقل زاوی  اسراع سے پہنچا ہو،  پھسلنے سے  بچنے کے لئے کم سے کم  سکونی رگڑ کا مستقل کیا ہو گا؟
\انتہا{سوال}
%---------------------------
%Q28 p288
\ابتدا{سوال}
رداس \عددی{r_A=\SI{10}{\centi\meter}} اور \عددی{r_C=\SI{25}{\centi\meter}}  کے پہیوں  کو  پٹہ \عددی{B} ملاتا  ہے (شکل \حوالہء{10.31})۔ ساکن حالت سے پہیا \عددی{A} کی زاوی  رفتار  \عددی{\SI{1.6}{\radian\per\second\squared}} مستقل شرح سے بڑھائی جاتی ہے۔ پہیا \عددی{C} کو \عددی{1000} چکر فی منٹ تک پہنچنے کے لئے کتنا وقت درکار ہو گا (پٹہ پھسلتا نہیں ہے)؟  (اشارہ: اگر پٹہ پھسلے نہیں، دونوں پہیوں کے چکے برابر خطی  رفتار سے حرکت کریں گے۔ )
\انتہا{سوال}
%-------------------------------
\ابتدا{سوال}
روشنی کی رفتار ناپنے کی ایک پرانی ترکیب شکل \حوالہء{10.32} میں دکھائی گئی ہے، جس میں  شگاف دار  گھومتا  پہیا استعمال کیا جاتا ہے۔ پہیے کے بیرونی کنارے پر   شگاف  سے کرن  گزر کر دور آئینہ   سے ٹکرا کر،  اسی راہ پر واپس  چلتے ہوئے ،   پہیے  پر پہنچتی ہے؛ اس دورانیہ میں پہیا ایک  شگاف  آگے بڑھتا ہے؛ یوں کرن اگلے شگاف سے گزر پاتی ہے۔ پہیے کا رداس \عددی{\SI{5.0}{\centi\meter}}، شگافوں کی تعداد \عددی{500}، اور آئینے تک فاصلہ \عددی{L=\SI{500}{\meter}} ہے۔ پیمائش سے معلوم ہوتا ہے کہ روشنی کی رفتار \عددی{\SI{3.0e5}{\kilo\meter\per\second}} ہے۔ (ا) پہیے کی مستقل ( زاوی ) رفتار کیا ہے؟ (ب)  پہیے کے کنارے پر  نقطے کی خطی رفتار کیا ہے؟
\انتہا{سوال}
%-------------------------
%Q30 p289
\ابتدا{سوال}
\اصطلاح{مسکن چرخی }\فرہنگ{مسکن چرخی}\حاشیہب{gyroscope}\فرہنگ{gyroscope} کے اڑن پہیے  کو، جس کا رداس \عددی{\SI{2.83}{\centi\meter}}  ہے ، ساکن حالت سے \عددی{\SI{14.2}{\radian\per\second\squared}}  سے مسرع کر کے \عددی{2760} چکر فی منٹ کی زاوی رفتار تک لایا جاتا ہے۔ (ا)  اس دوران    پہیے کے چکا پر  واقع نقطے کے  مماسی اسراع کیا ہو گا؟ (ب)  پوری رفتار پر گھومنے کے دوران نقطے کا رداسی اسراع کیا ہو گا؟ (ج)   اختتامی رفتار تک پہنچنے تک  چکا پر واقع نقطہ کتنا فاصلہ طے کرتا ہے؟
\انتہا{سوال}
%-------------------------
\ابتدا{سوال}
ایک قرص،  جس کا رداس \عددی{\SI{0.25}{\meter}} ہے،قرص کی  وسطی  انتصابی محور  پر \عددی{\SI{800}{\radian}} گھمانا مقصود ہے۔ساکن حالت سے آغاز کر کے، ابتدائی \عددی{\SI{400}{\radian}}  کے دوران قرص کو  مستقل \عددی{\alpha_1} شرح سے مسرع  کیا جاتا ہے جس کے بعد   مستقل \عددی{-\alpha_1} شرح سے اس کی زاوی رفتار گھٹائی جاتی ہے، حتٰی کہ قرص رک جاتا ہے۔   ضروری ہے کہ قرص کے کسی  حصہ کے مرکز مائل اسراع  کی قدر  \عددی{\SI{400}{\meter\per\second\squared}} سے تجاوز نہ کرے۔ (ا) گھماو کا کم سے کم دورانیہ کتنا ہو سکتا ہے؟ (ب) مطابقتی \عددی{\alpha_1} کی قیمت کیا ہو گی؟
\انتہا{سوال}
%----------------------
%Q32 p289
\ابتدا{سوال}
ساکن حالت سے آغاز کر کے گاڑی \عددی{\SI{30.0}{\meter}} رداس  کی دائری  راہ پر چلتی ہے۔ اس کی رفتار \عددی{\SI{0.500}{\meter\per\second\squared}}  مستقل شرح سے بڑھتی ہے۔ (ا)  \ترچھا{صافی } خطی اسراع کی قدر \عددی{\SI{15.0}{\second}} بعد کیا ہو گی؟ (ب)  اس لمحے پر گاڑی کا صافی سمتیہ  اسراع اور      گاڑی  کی سمتی رفتار   آپس میں کس زاویے پر ہیں؟
\انتہا{سوال}
%----------------------

%Module 10.4 kinetic energy of rotation p289
\جزوحصہء{گھماو کی حرکی توانائی}
%Q33 p289
\ابتدا{سوال}
ایک پہیا  \عددی{602} چکر فی منٹ سے گھوم رہا ہے  اور اس  کی  حرکی توانائی  \عددی{\SI{24400}{\joule}} ہے۔ پہیے کا گھمیری جمود تلاش کریں۔
\انتہا{سوال}
%------------------------
\ابتدا{سوال}
ایک پتلی سلاخ  ایک سر پر گھمائی جاتی ہے۔شکل \حوالہء{10.33} میں سلاخ کی   زاوی رفتار بالمقابل وقت پیش ہے۔ محور \عددی{\omega} کا پیمانہ \عددی{\omega_s=\SI{6.0}{\radian\per\second}} تعین کرتا ہے۔ (ا)  سلاخ کے  زاوی  اسراع کی قدر کیا ہے؟ (ب)  لمحہ \عددی{t=\SI{4.0}{\second}} پر سلاخ کی گھمیری  حرکی  توانائی \عددی{\SI{1.60}{\joule}} ہے۔ لمحہ \عددی{t=0} پر سلاخ کی حرکی توانائی کیا ہو گی؟
\انتہا{سوال}
%---------------------

%Module 10.5 calculating the rotational inertia p289
\جزوحصہء{گھمیری جمود  کا حساب}
%Q35 p289
\ابتدا{سوال}
دو یکساں ٹھوس  بیلن  اپنے اپنے وسطی (طولی)  محور پر \عددی{\SI{235}{\radian\per\second}}    سے گھوم رہے  ہیں۔ ان کی انفرادی  کمیت  \عددی{\SI{1.25}{\kilo\gram}}  تاہم رداس مختلف ہیں۔ (ا) چھوٹے بیلن کی ، جس کا رداس \عددی{\SI{0.25}{\meter}} ہے، اور (ب) بڑے بیلن کی، جس کا رداس \عددی{\SI{0.75}{\meter}} ہے، گھمیری حرکی توانائی کیا ہو گی؟
\انتہا{سوال}
%--------------------
\ابتدا{سوال}
 قرص  کے وسط سے  رداسی \عددی{h} فاصلے پر  محور کے گرد قرص گھوم سکتا ہے (شکل \حوالہء{10.34a})۔  قرص کے وسط سے کنارے تک \عددی{h} کی قیمتوں کے لئے قرص کا گھمیری جمود  \عددی{I} شکل \حوالہء{10.34b} میں ترسیم کیا گیا ہے۔ محور \عددی{I} کا پیمانہ \عددی{I_A=\SI{0.050}{\kilo\gram\meter\squared}} اور \عددی{I_B=\SI{0.150}{\kilo\gram\meter\squared}} تعین کرتے ہیں۔ قرص کی کمیت تلاش کریں۔
\انتہا{سوال}
%---------------------
\ابتدا{سوال}
میٹر سلاخ ،   جس کی کمیت \عددی{\SI{0.56}{\kilo\gram}} ہے، کا  گھمیری جمود  \عددی{\SI{20}{\centi\meter}} نشان پر واقع  سلاخ کو عمودی محور پر تلاش کریں۔ (میٹر سلاخ کو پتلی  سلاخ تصور کریں۔)
\انتہا{سوال}
%------------------------
%Q38  p289
\ابتدا{سوال}
بلا کمیت سلاخ کے ساتھ تین ذرے چسپاں کیے گئے ہیں (شکل \حوالہء{10.35})۔ سلاخ کی لمبائی \عددی{L=\SI{6.00}{\centi\meter}} اور ذروں کی انفرادی کمیت \عددی{\SI{0.0100}{\kilo\gram}} ہے۔ یہ نظام سلاخ کے  بائیں  سر پر واقع نقطہ \عددی{O} سے گزرتی عمودی  محور پر گھوم سکتا ہے۔ ہم ایک ذرہ ہٹاتے ہیں (جو \عددی{\SI{33}{\percent}} کمیت بنتا ہے)۔   محور سے (ا) قریب ترین ذرہ     اور (ب) دور ترین  ذرہ ہٹانے پر، نظام کا گھمیری جمود کتنے فی صد کم ہو گا؟
\انتہا{سوال}
%---------------------------
\ابتدا{سوال}
اڑن پہیے کو برقی موٹر  سے  \عددی{200\pi} ریڈیئن فی سیکنڈ  رفتار تک پہنچا کر  گھومتے اڑن پہیے میں ذخیرہ توانائی سے ٹرک چلایا جا سکتا ہے۔ فرض کریں اڑن پہیا ٹھوس  اور  یکساں بیلن ہے، جس کی کمیت \عددی{\SI{500}{\kilo\gram}}  اور رداس \عددی{\SI{1.0}{\meter}} ہے۔ (ا)   بھرائی کے بعد اڑن پہیے کی  حرکی توانائی    کتنی ہو گی؟ (ب)  اگر ٹرک اوسطاً \عددی{\SI{8.0}{\kilo\watt}} طاقت استعمال کرتا ہو،  بھرائی کتنی دیر میں دوبارہ کرنی ہو گی؟
\انتہا{سوال}
%-----------------------------
\ابتدا{سوال}
بالکل ایک جیسے \عددی{15} قرص سیدھ میں رکھ کر سلاخ کی شکل میں ، جس کی لمبائی \عددی{L=\SI{1.0000}{\meter}} اور (کل) کمیت \عددی{M=\SI{100.0}{\milli\gram}}  ہے، جوڑے گئے ہیں (شکل \حوالہء{10.36})۔ قرص یکساں ہیں اور پورا نظام  درمیانے قرص  کے وسطی نقطہ \عددی{O} پر گھوم سکتا ہے۔ (ا)  اس محور پر نظام کا گھمیری جمود تلاش کریں۔ (ب)  نظام کو   \عددی{M}  کمیت اور   \عددی{L}  لمبائی کی سلاخ  تصور کرنے سے  جدول \حوالہء{10.2e} کا کلیہ استعمال کرنے سے  گھمیری جمود کے حساب میں کتنے فی صد سہو پیدا ہو گا۔
\انتہا{سوال}
%-------------------------
%Q41 p289
\ابتدا{سوال}
دو ذروں کو،  جن کی انفرادی کمیت \عددی{m=\SI{0.85}{\kilo\gram}} ہے، ایک دوسرے کے ساتھ اور  \عددی{O} پر واقع محور گھماو   کے ساتھ سے  دو سلاخ جوڑتی ہیں۔ ان سلاخوں کی انفرادی کمیت اور لمبائی \عددی{M=\SI{1.2}{\kilo\gram}} اور \عددی{d=\SI{5.6}{\centi\meter}} ہے (شکل \حوالہء{10.37})۔ نظام محور گھماو پر \عددی{\omega=\SI{0.30}{\radian\per\second\squared}} زاوی رفتار سے گھومتا ہے۔  محور \عددی{O} پر نظام (ا) کا گھمیری جمود اور (ب)  حرکی توانائی کیا ہیں؟
\انتہا{سوال}
%---------------------
\ابتدا{سوال}
چار ذروں کی کمیتیں اور محدد ذیل ہیں: \عددی{\SI{50}{\gram}}، \عددی{x=\SI{2.0}{\centi\meter}}، \عددی{y=\SI{2.0}{\centi\meter}}؛
  \عددی{\SI{25}{\gram}}، \عددی{x=\SI{0}{\centi\meter}}، \عددی{y=\SI{4.0}{\centi\meter}}؛
   \عددی{\SI{25}{\gram}}، \عددی{x=\SI{-3.0}{\centi\meter}}، \عددی{y=\SI{-3.0}{\centi\meter}}؛
    \عددی{\SI{30}{\gram}}، \عددی{x=\SI{-2.0}{\centi\meter}}، \عددی{y=\SI{4.0}{\centi\meter}}۔ محور  (ا) \عددی{x}، (ب) \عددی{y}، (ج) \عددی{z} پر نظام کا گھمیری جمود تلاش کریں۔ (د)  ہم جزو ا اور جزو ب کے جوابات کو بالترتیب \عددی{A} اور \عددی{B} سے ظاہر کرتے ہیں۔ جزو ج کا  جواب \عددی{A} اور \عددی{B} کے روپ  میں لکھیں۔
\انتہا{سوال}
%-----------------------------
%Q43 p290
\ابتدا{سوال}
ٹھوس سل کی کمیت \عددی{\SI{0.172}{\kilo\gram}} اور اضلاع \عددی{a=\SI{3.5}{\centi\meter}}، \عددی{b=\SI{8.4}{\centi\meter}}، 
اور \عددی{c=\SI{1.4}{\centi\meter}} ہیں (شکل \حوالہء{10.38})۔ بڑی سطح کو عمودی، ایک کونے سے گزرتی  ، محور گھماو  پر  سل کا گھمیری جمود تلاش کریں۔
\انتہا{سوال}
%----------------------------
\ابتدا{سوال}
چار  ایک جیسے ذروں کو ، جن کی انفرادی کمیت \عددی{\SI{0.50}{\kilo\gram}} ہے،  \عددی{\SI{2.0}{\meter}\times \SI{2.0}{\meter}} چوکور  کی  چار  راس پر رکھے جاتے ہیں، اور انہیں بلا کمیت سلاخوں سے، جو چوکور کے  اضلاع  بناتے ہیں،  جوڑا جاتا ہے۔ (ا) مخالف اضلاع کے وسطی نقطوں  سے گزرتی محور گھماو پر،  جو چوکور کی سطح میں پایا جاتا ہے، (ب)  ایک ضلع کے وسطی نقطہ سے گزرتی محور گھماو پر، جو چوکور کی سطح کو عمودی ہے، اور (ج)  وتری مخالف  ذروں  سے گزرتی محور گھماو پر، جو  چوکور کی سطح میں پایا جاتا ہے، اس استوار جسم کا گھمیری جمود تلاش کریں۔
\انتہا{سوال}
%---------------------------

%Module 10.6 torque p290
\جزوحصہء{قوت مروڑ}
%Q45 p290
\ابتدا{سوال}
ایک جسم پر ، جس  کا چول نقطہ \عددی{O} پر  ہے،   دو قوت عمل کرتی ہیں (شکل \حوالہء{10.39})۔ اگر \عددی{r_1=\SI{1.30}{\meter}}، \عددی{r_2=\SI{2.15}{\meter}}، \عددی{F_1=\SI{4.20}{\newton}}،  \عددی{\theta_1=\SI{75.0}{\degree}} اور \عددی{\theta_2=\SI{60.0}{\degree}} ہو، چول پر صافی قوت مروڑ کیا ہو گی؟
\انتہا{سوال}
%-------------------------
\ابتدا{سوال}
ایک جسم پر، جس کا چول \عددی{O} پر ہے، تین قوت عمل کرتی ہیں (شکل \حوالہء{10.40})  جو ذیل ہیں:
نقطہ \عددی{A} پر جو   \عددی{O} سے \عددی{\SI{8.0}{\meter}}  فاصلے پر ہے \عددی{F_A=\SI{10}{\newton}}؛ 
نقطہ \عددی{B} پر جو   \عددی{O} سے \عددی{\SI{4.0}{\meter}}  فاصلے پر ہے \عددی{F_B=\SI{16}{\newton}}؛   اور 
نقطہ \عددی{C} پر جو   \عددی{O} سے \عددی{\SI{3.0}{\meter}}  فاصلے پر ہے \عددی{F_C=\SI{19}{\newton}}۔ \عددی{O} پر صافی قوت مروڑ تلاش کریں۔
\انتہا{سوال}
%------------------------
%Q47 p290
\ابتدا{سوال}
بلا کمیت،  \عددی{\SI{1.25}{\meter}} لمبی ، سلاخ کے ایک سر  پر  \عددی{\SI{0.75}{\kilo\gram}} گیند باندھ کر ، سلاخ کا دوسرا  سر چول   سے لٹکایا جاتا ہے۔ جب   حاصل  رقاص  انتصاب کے ساتھ \عددی{\SI{30}{\degree}} پر ہو، چول پر تجاذبی قوت مروڑ کی قدر کیا ہو گی؟
\انتہا{سوال}
%----------------------------
\ابتدا{سوال}
سائیکل  کے  پائیدان  کا بازو \عددی{\SI{0.152}{\meter}} ہے اور سائیکل سوار پائیدان پر  \عددی{\SI{111}{\newton}} نشیبی  قوت لاگو کرتا ہے۔ پائیدان بازو  کے چول پر  اس وقت قوت مروڑ کی قدر کیا ہو گی جب  انتصاب کے ساتھ پائیدان کا زاویہ (ا) \عددی{\SI{30}{\degree}}، (ب)  \عددی{\SI{90}{\degree}}، اور (ج)  \عددی{\SI{180}{\degree}} ہو؟
\انتہا{سوال}
%--------------------------

%module 10.7 newton's second law for rotation p290
\جزوحصہء{نیوٹن کا قانون دوم برائے گھماو}
%Q49 p290
\ابتدا{سوال}
\اصطلاح{تختہ غوطہ }\فرہنگ{تختہ غوطہ}\حاشیہب{diving board}\فرہنگ{diving board}   سے  تالاب میں کود کر   غوطہ خور   کی زاوی رفتار     ، اس کے مرکز کمیت پر، \عددی{\SI{220}{\milli\second}} میں صفر سے \عددی{\SI{6.20}{\radian\per\second}} ہوتی ہے۔ مرکز کمیت  پر اس کا گھمیری جمود \عددی{\SI{12.0}{\kilo\gram\meter\squared}} ہے۔ پرواز کے دوران  (ا) غوطہ خور کے  اوسط زاوی اسراع اور (ب) تختہ سے غوطہ خور پر بیرونی اوسط قوت مروڑ کی قدریں کیا ہیں؟
\انتہا{سوال}
%-----------------------
\ابتدا{سوال}
پہیے پر \عددی{\SI{32.0}{\newton\meter}} قوت مروڑ \عددی{\SI{25.0}{\radian\per\second\squared}} زاوی اسراع پیدا کرتی ہے۔ پہیے کا گھمیری جمود کیا ہے؟
\انتہا{سوال}
%-----------------------
%Q51 p290
\ابتدا{سوال}
بلا رگڑ افقی  دھرے  پر رداس \عددی{R=\SI{5.00}{\centi\meter}} کا \اصطلاح{ جرثقیل }\فرہنگ{جرثقیل}\حاشیہب{pulley}\فرہنگ{pulley}    نصب ہے، جس سے کمیت \عددی{m_1=\SI{460}{\gram}} کی  سل \عددی{1} اور کمیت \عددی{m_2=\SI{500}{\gram}} کی سل \عددی{2} لٹکی ہے (شکل \حوالہء{10.41})۔ ساکن حالت سے رہائی پر   \عددی{\SI{5.0}{\second}} میں سل \عددی{2}   \عددی{\SI{75.0}{\centi\meter}}   گرتی ہے۔  دھاگہ  ہرگز نہیں پھسلتا۔ (ا) سلوں کے زاوی اسراع کی قدر کیا ہے؟ (ب) تناو \عددی{T_2} اور (ج) \عددی{T_1} کتنا ہے؟ (د)   جرثقیل  کے زاوی اسراع کی قدر کیا ہے؟ (ہ) جرثقیل  کا گھمیری جمود کیا ہے؟
\انتہا{سوال}
%------------------
\ابتدا{سوال}
ایک بیلن ، جس کی کمیت \عددی{\SI{2.0}{\kilo\gram}} ہے،اپنی  وسطی طولی  محور   پر، جو \عددی{O} پر واقع ہے،  گھوم سکتا ہے (شکل \حوالہء{10.42})۔  لاگو قوت ذیل ہیں:
 \عددی{F_1=\SI{6.0}{\newton}}، \عددی{F_2=\SI{4.0}{\newton}}، \عددی{F_3=\SI{2.0}{\newton}}، اور \عددی{F_4=\SI{5.0}{\newton}}۔ ساتھ ہی \عددی{r=\SI{5.0}{\centi\meter}} اور \عددی{R=\SI{12}{\centi\meter}} ہیں۔ (ا)  بیلن کے زاوی اسراع (ا)   کی قدر  اور (ب) رخ تلاش کریں۔ (گھماو کے دوران بیلن کے لحاظ سے  قوت انہیں زاویوں پر رہتی ہیں۔)
\انتہا{سوال}
%----------------------
%Q53 p290
\ابتدا{سوال}
یکساں قرص اپنی وسطی  انتصابی  محور پر گھوم سکتا ہے (شکل \حوالہء{10.43})۔ قرص ، جو ابتدائی طور  پر ساکن ہے،  کی کمیت \عددی{\SI{20.0}{\gram}} اور رداس \عددی{\SI{2.0}{\centi\meter}} ہے۔ لمحہ \عددی{t=0} پر دکھائی گئی دو قوت قرص کے چکا پر  مماسی لاگو کی جاتی ہیں، جن کی بدولت \عددی{t=\SI{1.25}{\second}} پر قرص کی خلاف گھڑی  زاوی سمتی رفتار
 \عددی{\SI{250}{\radian\per\second}} ہو گی۔ قوت \عددی{\vec{F}_1} کی قدر \عددی{\SI{0.100}{\newton}} ہے۔ قوت \عددی{\vec{F}_2} کی قدر تلاش کریں۔
\انتہا{سوال}
%------------------------
\ابتدا{سوال}
جاپانی کُشتی   جوڈو  کہلاتی ہے۔  ایک داو میں  آپ حریف کا بایاں پاوں  مار کر اٹھاتے ہیں اور ساتھ ہی اس کو وردی سے پکڑ کر بائیں کھینچتے ہیں۔ نتیجتاً،  حریف اپنے دائیں پاوں پر گھوم کر زمین  پر گرتا ہے۔ شکل \حوالہء{10.44}  میں آپ کا حریف دکھایا گیا ہے، جس میں اس کا بایاں پاوں زمین سے اٹھا  دکھایا گیا ہے۔ محور گھماو نقطہ \عددی{O} پر ہے۔ تجاذبی قوت \عددی{\vec{F}_g} اس کے مرکز کمیت پر عمل کرتی ہے، جو \عددی{O} سے \عددی{d=\SI{28}{\centi\meter}} افقی فاصلے پر ہے۔ اس کی کمیت \عددی{\SI{70}{\kilo\gram}}  اور \عددی{O} پر گھمیری جمود \عددی{\SI{65}{\kilo\gram\meter\squared}} ہے۔ (ا)  آپ کی قوت \عددی{\vec{F}_a} قابل نظر انداز ہونے کی صورت میں اور (ب)  آپ  کی قوت افقی، اس کی قدر \عددی{\SI{300}{\newton}} ، اور نقطہ اطلاق کی بلندی \عددی{h=\SI{1.4}{\meter}} ہونے کی صورت میں \عددی{O} پر  حریف کا ابتدائی زاوی اسراع کیا ہو گا؟
\انتہا{سوال}
%----------------------------
%Q55 p290
\ابتدا{سوال}
یکساں موٹائی  اور کثافت (کمیت فی اکائی حجم) کے   پلاسٹک  کی بے قاعدہ  چادر نقطہ \عددی{O} پر واقع، سطح  چادر  کو عمودی،  محور  پر گھمائی جاتی ہے (شکل \حوالہء{10.45a})۔ اس محور پر چادر کا گھمیری جمود درج ذیل ترکیب سے ناپا جاتا ہے۔ رداس \عددی{\SI{2.00}{\centi\meter}}  اور کمیت \عددی{\SI{0.500}{\kilo\gram}} کا دائری قرص  چادر کے ساتھ  یوں چسپاں کیا جاتا ہے  کہ قرص کا وسط  \عددی{O} پر ہو (شکل \حوالہء{10.45b})۔ لٹو  پر  دھاگہ لپیٹنے کی طرح  قرص  پر  دھاگہ   لپیٹ کر دھاگہ \عددی{\SI{5.0}{\second}} کے لئے کھینچا جاتا ہے۔نتیجتاً،  قرص کے چکا پر مماسی لاگو   \عددی{\SI{0.400}{\newton}} مستقل قوت  قرص اور چادر دونوں کو گھماتی ہے۔ماحصل زاوی رفتار \عددی{\SI{114}{\radian\per\second}} ہے۔ محور گھماو پر چادر کا گھمیری جمود کیا ہو گا؟
\انتہا{سوال}
%------------------------------
%Q56 p291
\ابتدا{سوال}
دو ذرے \عددی{1} اور \عددی{2} جن کی انفرادی کمیت \عددی{m} ہے بلا کمیت سلاخ کے سروں پر جڑے ہیں (شکل \حوالہء{10.46})۔ سلاخ کی لمبائی \عددی{L_1+L_2} ہے، جہاں \عددی{L_1=\SI{20}{\centi\meter}} اور \عددی{L_2=\SI{80}{\centi\meter}} ہے۔ چول پر سلاخ افقی رکھ کر رہا کی جاتی ہے۔ (ا) ذرہ \عددی{1} اور (ب) ذرہ \عددی{2} کے ابتدائی اسراع کی قدر کیا ہو گی؟
\انتہا{سوال}
%----------------------------
\ابتدا{سوال}
ایک جرثقیل   کا رداس \عددی{\SI{10}{\centi\meter}} اور دھرے پر گھمیری جمود \عددی{\SI{1.0e-3}{\kilo\gram\meter\squared}} ہے۔ جرثقیل کے چکا پر مماسی تغیر پذیر   قوت \عددی{F=0.50t+0.30t^2} لاگو کی جاتی ہے ، جہاں \عددی{F} نیوٹن میں اور \عددی{t} سیکنڈ میں ہے۔ جرثقیل ابتدائی طور ساکن ہے۔وقت \عددی{t=\SI{3.0}{\second}} پر اس (ا) کا زاوی اسراع اور (ب) زاوی رفتار کیا ہوں گے؟
\انتہا{سوال}
%--------------------------------

%Module 10.8 work and rotational kinetic energy p291
\جزوحصہء{کام اور گھمیری حرکی توانائی}
%Q58 p291
\ابتدا{سوال}
(ا)  اگر شکل \حوالہء{10.19} میں \عددی{R=\SI{12}{\centi\meter}}، \عددی{M=\SI{400}{\gram}}، اور \عددی{m=\SI{50}{\gram}} ہو،  ساکن حالت سے رہائی کے بعد   \عددی{\SI{50}{\centi\meter}}  اتر کر  سل کی رفتار  کیا ہو گی؟ توانائی کی بقا کا اصول استعمال کر کے جواب معلوم کریں۔ (ب) رداس \عددی{R=\SI{5.0}{\centi\meter}} لے کر جواب دوبارہ حاصل کریں۔
\انتہا{سوال}
%-------------------------------
\ابتدا{سوال}
گاڑی کا \اصطلاح{ خمدار دھرا }\فرہنگ{دھرا!خمدار}\حاشیہب{crankshaft}\فرہنگ{crankshaft} (کرینک شافٹ) \عددی{1800}  چکر فی منٹ رفتار سے گھومتے ہوئے  انجن سے دھرے  (اکسل) تک \عددی{\SI{74.6}{\kilo\watt}} شرح سے   توانائی  پہنچاتا ہے۔ خمدار دھرا کتنی قوت مروڑ فراہم کرتا ہے؟
\انتہا{سوال}
%------------------------
%Q60 p291
\ابتدا{سوال}
ایک پتلی سلاخ ، جس کی لمبائی \عددی{\SI{0.75}{\meter}} اور کمیت \عددی{\SI{0.42}{\kilo\gram}} ہے ، ایک سر سے لٹکی ہے۔ سلاخ کو ایک جانب کھینچ کر  رہا کر کے رقاص کی طرح  جھولنے  دیا جاتا  ہے؛ نشیب سے سلاخ \عددی{\SI{4.0}{\radian\per\second}} زاوی رفتار سے گزرتی ہے۔ رگڑ اور ہوائی رکاوٹ نظر انداز کریں۔ (ا)  نشیبی مقام پر سلاخ کی حرکی توانائی کیا ہو گی  اور (ب)  سلاخ کا مرکز کمیت  نشیبی نقطہ سے کتنی بلندی تک پہنچ پاتا ہے؟
\انتہا{سوال}
%--------------------------
\ابتدا{سوال}
ایک پہیا ، جس کو   دائری پتلی  سلاخ  تصور کیا جا سکتا ہے ، کی کمیت \عددی{\SI{32.0}{\kilo\gram}}  اور  رداس \عددی{\SI{1.20}{\meter}} ہے  اور  اس کی زاوی رفتار \عددی{280}  چکر فی منٹ ہے۔ پہیے کو \عددی{\SI{15.0}{\second}} میں  روکنا مقصود ہے۔ (ا)  پہیا روکنے میں کتنا کام سرانجام ہو گا؟ (ب) درکار اوسط طاقت کیا ہو گی؟
\انتہا{سوال}
%----------------------------
\ابتدا{سوال}
تین ذروں کو، جن کی انفرادی کمیت \عددی{\SI{0.0100}{\kilo\gram}} ہے، بلا کمیت   \عددی{L=\SI{6.00}{\centi\meter}} لمبی سلاخ کے ساتھ جوڑا گیا ہے جو، سلاخ کے  سر پر نقطہ \عددی{O}   پر واقع  سلاخ کو عمودی،  محور پر گھوم سکتی ہے (شکل \حوالہء{10.35})۔ گھمیری  شرح  کو (ا)  \عددی{0}  سے \عددی{\SI{20.0}{\radian\per\second}} کرنے، (ب)  \عددی{\SI{20.0}{\radian\per\second}}  سے \عددی{\SI{40.0}{\radian\per\second}} کرنے، اور (ج) \عددی{\SI{40.0}{\radian\per\second}}  سے \عددی{\SI{60.0}{\radian\per\second}} کرنے میں کتنا کام درکار ہو گا؟ (د)   نظام کی حرکی (جاول میں)  توانائی   بالمقابل   (مربع ریڈیئن فی مربع سیکنڈ میں)  مربع   شرح گھماو کی ترسیم  کی  ڈھلوان  کیا ہو گی؟
\انتہا{سوال}
%----------------------------
%Q63 p291
\ابتدا{سوال}
میٹر سلاخ زمین پر کھڑی کر کے گرنے دی جاتی ہے۔ عین زمین پر پہنچ کر سلاخ کے دوسرے سر کی رفتار کیا ہو گی؟زمین پر رکھا گیا سر پھسلتا نہیں۔ (اشارہ: میٹر سلاخ کو پتلی سلاخ تصور کر کے توانائی کی بقا کا اصول بروئے کا ر لائیں۔)
\انتہا{سوال}
%----------------------------
\ابتدا{سوال}
یکساں بیلن  کو، جس کا رداس \عددی{\SI{10}{\centi\meter}} اور کمیت \عددی{\SI{20}{\kilo\gram}} ہے،  یوں رکھا جاتا ہے کہ  بیلن کی   وسطی طولی محور کے متوازی، \عددی{\SI{5.0}{\centi\meter}} فاصلے پر ، افقی محور کے گرد گھوم سکے۔ (ا)  محور گھماو پر بیلن کا گھمیری جمود تلاش کریں۔  (ب)  بیلن کی وسطی طولی محور کو محور گھماو کی بلندی پر رکھ کر  ساکن بیلن رہا کیا جاتا ہے۔ نشیب سے گزرتے وقت   بیلن کی زاوی رفتار کیا ہو گی؟
\انتہا{سوال}
%-----------------------------
%Q65 p291
\ابتدا{سوال}
ایک بلند بیلنی آتش دان  جس کی بنیاد کمزور  پڑ گئی تھی گرتا ہے۔آتش دان کو پتلی سلاخ تصور کریں جس کی لمبائی \عددی{\SI{55.0}{\meter}} ہے۔ گرنے کے دوران جس لمحے انتصاب کے ساتھ آتش دان \عددی{\SI{35.0}{\degree}} زاویہ بناتا ہے ، اس کے بالا سر کا (ا) رداسی اسراع، اور (ب) مماسی اسراع کیا ہوں گے؟ (اشارہ: توانائی کی بقا بروئے کار لائیں نا کہ قوت مروڑ۔) (ج) مماسی اسراع کس زاویے \عددی{\theta} پر \عددی{g=\SI{9.8}{\meter\per\second\squared}} کے برابر ہو گا؟
\انتہا{سوال}
%-----------------------
\ابتدا{سوال}
یکساں کروی خول، جس کی کمیت \عددی{M=\SI{4.5}{\kilo\gram}} اور رداس \عددی{R=\SI{8.5}{\centi\meter}} ہے،  انتصابی وسطی محور  پر  بلا رگڑ  گھوم سکتا ہے (شکل \حوالہء{10.47})۔بلا کمیت دھاگہ، جس سے \عددی{m=\SI{0.60}{\kilo\gram}} کمیت کا جسم لٹکا ہے،   جرثقیل پر گزار کر کرہ کے خط استوا پر لپیٹا جاتا ہے۔ جرثقیل کا گھمیری جمود   \عددی{I=\SI{3.0e-3}{\kilo\gram\meter\squared}}  اور  رداس \عددی{r=\SI{5.0}{\centi\meter}} ہے۔ جرثقیل کا دھرا بلا رگڑ ہے؛ دھاگہ جرثقیل پر پھسلتا نہیں ہے۔ ساکن حالت سے   \عددی{\SI{82}{\centi\meter}}  گرنے کے بعد جسم کی رفتار کیا ہو گی؟ توانائی کی بقا استعمال کریں۔
\انتہا{سوال}
%-----------------------
\ابتدا{سوال}
پتلا گھیر  (کمیت \عددی{m} اور رداس \عددی{R=\SI{0.150}{\meter}}) اور پتلی سلاخ (کمیت \عددی{m} اور لمبائی \عددی{L=\SI{2.00}{R}}) جوڑ کر استوار نظام بنایا گیا ہے (شکل \حوالہء{10.48})۔ نظام سیدھا کھڑا ہے، تاہم معمولی ہلانے پر نظام،  سلاخ کے نچلے سر پر واقع، سلاخ اور \اصطلاح{ گھیرا }\فرہنگ{گھیرا}\حاشیہب{hoop}\فرہنگ{hoop} کے  مستوی   میں موجود ، افقی محور کے گرد گھومتا ہے۔ فرض کریں معمولی ہلانے سے منتقل قابل نظر انداز ہے۔ نشیبی نقطہ سے گزرتے وقت نظام کی زاوی رفتار کیا ہو گی؟
\انتہا{سوال}
%-------------------------

%Additional problems p291
\جزوحصہء{اضافی سوال}
%Q68 p291
\ابتدا{سوال}
دو  ٹھوس یکساں کرہ   کی انفرادی کمیت \عددی{\SI{1.65}{\kilo\gram}}،  اور رداس \عددی{\SI{0.226}{\meter}} اور \عددی{\SI{0.854}{\meter}} ہیں ۔دونوں اپنی اپنی محور پر، جو کرہ کے مرکز سے گزرتی ہے، گھوم سکتے ہیں۔ (ا)  چھوٹے  کرہ کو ساکن حالت سے  \عددی{\SI{15.5}{\second}} میں \عددی{\SI{317}{\radian\per\second}} زاوی رفتار تک لانے کے لئے درکار \عددی{\tau} کی قدر  کیا ہے؟ (ب)  کرہ کے خط استوا  پر مماسی قوت  کی قدر \عددی{F} کیا ہو گی جو اتنی قوت مروڑ دے؟ (ج) \عددی{\tau} اور (د) \عددی{F} بڑے کرہ کے لئے کیا ہیں؟
\انتہا{سوال}
%--------------------------
%Q69 p291
\ابتدا{سوال}
رداس \عددی{r=\SI{2.00}{\centi\meter}} کا چھوٹا قرص،   رداس \عددی{R=\SI{4.00}{\centi\meter}} کے بڑے قرص کے کنارے    یوں جوڑا گیا ہے  کہ دونوں ایک مستوی میں ہوں (شکل \حوالہء{10.49})۔ بڑے قرص کے مرکز \عددی{O}  پر واقع عمودی محور  کے گرد   نظام گھوم  سکتا ہے۔ دونوں قرص کی  یکساں کثافت (کمیت فی اکائی حجم)  \عددی{\SI{1.40e3}{\kilo\gram\per\meter\cubed}}  اور یکساں موٹائی \عددی{\SI{5.00}{\milli\meter}} ہے۔ محور گھماو پر نظام کا گھمیری جمود تلاش کریں۔
\انتہا{سوال}
%-----------------------
%Q70 p292
\ابتدا{سوال}
ایک پہیا ساکن حالت سے آغاز کر کے \عددی{\SI{2.00}{\radian\per\second\squared}} مستقل اسراع  کے ساتھ گھومتا ہے۔ کسی مخصوص \عددی{\SI{3.00}{\second}} دورانیے میں پہیا \عددی{\SI{90.0}{\radian}} گھومتا ہے۔ (ا)  اس \عددی{\SI{3.00}{\second}} دورانیہ کے آغاز میں  پہیے کی زاوی سمتی رفتار کیا ہے؟  (ب)  \عددی{\SI{3.00}{\second}} دورانیے سے قبل پہیا کتنی دیر حرکت میں رہا؟
\انتہا{سوال}
%--------------------
\ابتدا{سوال}
دو جسم ، جن کی انفرادی کمیت \عددی{\SI{6.20}{\kilo\gram}} ہے،  بلا کمیت دھاگے سے آپس میں باندھے گئے ہیں (شکل \حوالہء{10.50})۔ دھاگہ  \عددی{\SI{2.40}{\centi\meter}} رداس اور \عددی{\SI{7.40e-4}{\kilo\gram\meter\squared}} گھمیری جمود کے جرثقیل  سے گزرتا ہے۔ جرثقیل پر دھاگہ پھسلتا نہیں؛ ہم نہیں جانتے آیا میز اور جسم کے بیچ رگڑ ہے یا نہیں؛ جرثقیل کا دھرا بلا رگڑ ہے۔ ساکن حالت سے رہائی  پر  \عددی{\SI{91.0}{\milli\second}} میں جرثقیل \عددی{\SI{0.130}{\radian}} گھومتا ہے، اور اجسام کا اسراع مستقل ہے۔ (ا) جرثقیل کے زاوی اسراع کی قدر، (ب) اجسام کے اسراع کی قدر، (ج) دھاگے کا  تناو \عددی{T_1} اور (د) دھاگے کا تناو  \عددی{T_2} کیا ہیں؟
\انتہا{سوال}
%-----------------
\ابتدا{سوال}
پتلی سلاخ، جس کی کمیت \عددی{\SI{6.40}{\kilo\gram}} اور لمبائی \عددی{\SI{1.20}{\meter}} ہے، کے دونوں سر  پر \عددی{\SI{1.06}{\kilo\gram}} کا گیند نصب کیا جاتا ہے۔ سلاخ کے مرکز پر واقع انتصابی محور   پر سلاخ افقی مستوی میں  گھوم سکتی ہے۔ کسی مخصوص لمحے پر سلاخ \عددی{39.0}  چکر فی منٹ  سے گھومتی  ہے۔ رگڑ کی وجہ سے \عددی{\SI{32.0}{\second}} میں یہ رک جاتی ہے۔ رگڑ کی   \اصطلاح{آہستہ کن  }\فرہنگ{آہستہ کن}\حاشیہب{retarding}\فرہنگ{retarding} قوت مروڑ مستقل تصور کریں۔ (ا) زاوی اسراع، (ب) آہستہ کن قوت مروڑ، (ج) میکانی توانائی سے حری توانائی میں رگڑ کی  بنا منتقل توانائی کی قدر، اور (د) ان \عددی{\SI{32.0}{\second}} میں چکر کی تعداد تلاش کریں۔ (ہ)  ا فرض کریں آہستہ کن قوت مروڑ مستقل نہیں۔ کیا  جزو ا، ب، ج، اور د مزید معلومات  دیے  بغیر معلوم کیے جا سکتے ہیں؟ جو معلوم کی جا سکتی ہیں  ان کی قیمتیں  کیا ہوں گی؟
\انتہا{سوال}
%-----------------------
%Q73 p292
\ابتدا{سوال}
ہیلی کاپٹر  کے   یکساں پر  کی لمبائی \عددی{\SI{7.8}{\meter}}  اور کمیت \عددی{\SI{110}{\kilo\gram}} ہے، اور ایک  قابلہ    اس کو    مدور دھرے کے ساتھ  جوڑتا ہے۔ (ا) جب مدور \عددی{320} چکر فی منٹ سے گھومتا ہے  (جو اس کی پوری رفتار ہے)، قابلے پر دھرے کی قوت کی قدر کیا ہو گی؟ (اشارہ:اس حساب کے لئے پر کو کمیتی  نقطہ تصور کیا جا سکتا ہے جو پر کے مرکز کمیت پر واقع ہو۔ کیوں؟)  (ب)   ساکن حالت سے \عددی{\SI{6.70}{\second}} میں پوری رفتار تک  پہنچانے کے لئے  مدور پر درکار قوت مروڑ کیا ہو گی؟ ہوا کی رگڑ نظر انداز کریں (اس حساب میں پر کو کمیتی نقطہ تصور نہیں کیا جا سکتا۔ کیوں نہیں؟ پتلی سلاخ پر یکساں کمیتی تقسیم تصور کیا جا سکتا ہے۔)  (ج)  \عددی{320} چکر فی منٹ تک پہنچانے کے لئے قوت مروڑ  پر پر کتنا کام سرانجام  دیگی؟

\انتہا{سوال}
%---------------
%Q74 p292
\ابتدا{سوال}
دو قرص  اپنی اپنی    وسطی انتصابی محور پر گھوم سکتے ہیں (شکل \حوالہء{10.51})۔ لمحہ \عددی{t=0} پر  دونوں قرص کے حوالہ لکیر ایک جیسی سمت بند ہیں۔ قرص \عددی{A} پہلے سے \عددی{\SI{9.5}{\radian\per\second}} زاوی رفتار سے حرکت میں ہے۔قرص \عددی{B} جو ساکن ہے  اب \عددی{\SI{2.2}{\radian\per\second\squared}} مستقل زاوی  اسراع سے چل پڑتا ہے۔ (ا)   حوالہ لکیروں کا زاویہ \عددی{\theta}کس وقت  \عددی{t} پر لمحاتی    ایک  جتنا ہو گا؟ (ب)  کیا \عددی{t=0} کے بعد   \عددی{t}  وہ پہلا لمحہ ہے جب دونوں حوالہ لکیر  سیدھ میں  ہیں؟
\انتہا{سوال}
%-----------------------------
%Q75 p292
\ابتدا{سوال}
رسی پر چلنے والا  شخص اپنا مرکز کمیت رسی پر رکھتا ہے۔لمبا اور بھارا    ڈنڈا  ہاتھ میں ہونا   مدد گار ثابت ہوتا ہے: اگر   مرکز کمیت رسی سے دائیں  منتقل  ہو اور رسی پر گھوم کر گرنے کا خطرہ ہو،  شخص  ڈنڈے کو بائیں حرکت دے کر  گھماو  آہستہ کر کے سنبھلتا ہے۔ فرض کریں شخص کی کمیت \عددی{\SI{70.0}{\kilo\gram}} اور رسی  پر گھمیری جمود \عددی{\SI{15.0}{\kilo\gram\meter\squared}} ہے۔رسی پر اس کے  زاوی اسراع  کی قدر کیا ہو گی  اگر اس کا مرکز کمیت رسی سے \عددی{\SI{5.0}{\centi\meter}} دائیں  ہو ، اور (ا) اس کے پاس ڈنڈا نہ ہو اور (ب) اگر اس کے پاس \عددی{\SI{14.0}{\kilo\gram}}  ڈنڈا ہو جس کا مرکز کمیت رسی سے \عددی{\SI{10}{\centi\meter}} بائیں ہو؟
\انتہا{سوال}
%------------------
%Q76 p292
\ابتدا{سوال}
اس پہیا  \عددی{t=0} پر ساکن حالت سے آغاز کر کے مستقل زاوی اسراع  سے گزرتا ہے۔ لمحہ \عددی{t=\SI{2.0}{\second}} پر پہیے کی زاوی سمتی رفتار \عددی{\SI{5.0}{\radian\per\second}} ہے۔اسراع \عددی{t=\SI{20}{\second}} تک برقرار رہنے کے بعد یک دم ختم ہوتا ہے۔ وقت \عددی{t=0} تا \عددی{t=\SI{40}{\second}} میں پہیا کتنا زاویہ طے کرتا ہے؟
\انتہا{سوال}
%------------------
\ابتدا{سوال}
تختہ گھوم  \عددی{33\tfrac{1}{3}} چکر فی منٹ کی رفتار سے \عددی{\SI{30}{\second}} میں بتدریج آہستہ ہو کر رکتا ہے۔ (ا)  اس کا (مستقل)   زاوی اسراع ، چکر فی مربع منٹ میں ، تلاش کریں۔ (ب) اس دورانیے میں تختہ کتنے چکر کاٹتا ہے؟
\انتہا{سوال}
%--------------------
\ابتدا{سوال}
تین \عددی{L=\SI{0.600}{\meter}} لمبی  یکساں پتلی سلاخوں سے استوار جسم بنایا گیا ہے ، جو لاطینی حرف \تحریر{H} کی شکل میں ہے (شکل \حوالہء{10.52})۔ جسم   افقی محور پر، جو ایک ٹانگ     کی  ہمراہ ہے،  گھوم سکتا ہے۔ جسم کا مستوی افقی رکھ کر جسم  گرنے دیا جاتا ہے۔ جب یہ مستوی انتصابی مقام کو پہنچتا ہے، جسم کی زاوی رفتار کیا ہو گی؟
\انتہا{سوال}
%-----------------------
%Q79 p292
\ابتدا{سوال}
(ا) دکھائیں کہ   کمیت \عددی{M} اور رداس \عددی{R}   کے ٹھوس بیلن کا وسطی  محور پر گھمیری جمود ، اور   کمیت \عددی{M} اور رداس \عددی{R\!/\!\sqrt{2}}  کے  گھیرا  کا وسطی محور پر گھمیری جمود برابر ہیں۔ (ب)  دکھائیں کہ  کسی بھی جسم کا، جس کی کمیت \عددی{M} ہو، کسی بھی محور پر   گھمیری جمود \عددی{I}  \ترچھا{ معادل } گھیرا  کا اسی محور پر  گھمیری جمود  کے برابر ہو گا۔ معادل گھیرا کی کمیت \عددی{M} اور رداس \عددی{k} ذیل ہو گا۔
\begin{align*}
k=\sqrt{\frac{I}{M}}
\end{align*}
معادل گھیرا  کا رداس \عددی{k} اس جسم کا\اصطلاح{  رداس دوار }\فرہنگ{رداس!دوار}\حاشیہب{radius of gyration}\فرہنگ{radius!of gyration} کہلاتا ہے۔
\انتہا{سوال}
%---------------------------
\ابتدا{سوال}
دائری قرص \عددی{\SI{6.00}{\second}} میں مستقل زاوی اسراع کے ساتھ زاوی مقام \عددی{\theta_1=\SI{10.0}{\radian}} سے زاوی مقام \عددی{\theta_2=\SI{70.0}{\radian}} پہنچتا ہے۔ مقام \عددی{\theta_2} پر جسم کی زاوی سمتی رفتار \عددی{\SI{15.0}{\radian\per\second\squared}} ہے۔ (ا)  جسم کی زاوی سمتی  رفتار \عددی{\theta_1} پر کیا تھی؟ (ب)  زاوی اسراع کیا ہے؟ (ج) ابتدائی ساکن حالت میں قرص کا زاوی مقام کیا تھا؟ (د)   ابتدا  سے (جس کو ہم  \عددی{t=0} کہتے ہیں)   \عددی{\theta} بالمقابل \عددی{t}، اور زاوی رفتار \عددی{\omega} بالمقابل \عددی{t} ترسیم کریں۔
\انتہا{سوال}
%------------------
%Q81 p292
\ابتدا{سوال}
ایک پتلی یکساں سلاخ جس کی لمبائی \عددی{\SI{2.0}{\meter}} ہے، ایک سر پر واقع بلا رگڑ  افقی کیل پر گھوم سکتی ہے (شکل \حوالہء{10.53})۔ افق سے \عددی{\theta=\SI{40}{\degree}} اوپر رکھ کر   ساکن حالت سے  سلاخ  رہا کی جاتی ہے۔ افقی مقام سے گزرتے وقت سلاخ کی زاوی رفتار  توانائی کی بقا کا اصول  استعمال کر کے تلاش کریں۔
\انتہا{سوال}
%-----------------------
%Q82 p292
\ابتدا{سوال}
ایک\اصطلاح{ چرخ ہنڈولا }\فرہنگ{چرخ ہنڈولا}\حاشیہب{Ferris wheel}\فرہنگ{Ferris wheel}   جس کا قطر \عددی{\SI{76}{\meter}} ہے \عددی{36} لکڑ گاڑیوں پر مشتمل ہے۔ ہر گاڑی میں \عددی{60} سوار بیٹھ سکتے ہیں۔ تمام  سواریاں  بٹھا کر  چرخ ہنڈولا  کو \عددی{1} چکر فی  \عددی{2} منٹ  کی مستقل زاوی رفتار سے چلایا جاتا ہے۔ صرف سواریوں  کو گھمانے کے لئے  درکار کام کی تخمیناً قیمت تلاش کریں۔
\انتہا{سوال}
%-----------------------
%Q83 p293
\ابتدا{سوال}
کمیت \عددی{M=\SI{500}{\gram}} اور رداس \عددی{R=\SI{12.0}{\centi\meter}} کے یکساں قرص کے گرد بلا کمیت  دھاگہ  لپیٹ کر دھاگے کے سروں سے   \عددی{m_1=\SI{400}{\gram}} کمیت اور \عددی{m_2=\SI{600}{\gram}} کمیت  کی  سل  آویزاں کی   جاتی ہیں (شکل \حوالہء{10.41})۔ قرص کے وسطی افقی محور پر قرص بلا رگڑ گھوم سکتا ہے؛ دھاگہ پھسلتا نہیں ہے۔ نظام ساکن حالت سے رہا کیا جاتا ہے۔ (ا)  سل کے اسراع کی قدر ، (ب)  بائیں جانب دھاگے کا تناو \عددی{T_1}، اور (ج) دائیں جانب دھاگے کا تناو \عددی{T_2} تلاش کریں۔
\انتہا{سوال}
%-------------------------
\ابتدا{سوال}
وسطی سائبیریا  میں، جون \عددی{30} \سن{1908} کی صبح کے   سات بج کر چودہ منٹ  پر ،  \عددی{\SI{61}{\degree}} شمال خط عرض بلد اور \عددی{\SI{102}{\degree}} مشرق  خط طول بلد پر،    کچھ بلندی پر ایک خوف ناک دھماکہ ہوا۔ ؛ جو آگ کا شعلہ   اٹھا وہ   جوہری دھماکے سے پہلے انسان نے کبھی نہیں دیکھا۔ یہ واقع دریا \ترچھا{  تنگسکا  } کے قریب پیش آیا  جس کی  بنا یہ \اصطلاح{تنگسکا وقوعہ  }\فرہنگ{وقوعہ!تنگسکا}\حاشیہب{Tunguska event}\فرہنگ{event!Tunguska} کہلاتا ہے۔ا یک  اتفاقی شاہد  کے مطابق     \قول{آسمان کا بہت بڑا حصہ وقوعہ کی لپیٹ میں آیا}۔ یہ غالباً \عددی{\SI{140}{\meter}}  چوڑے  \ترچھا{پتھری  } سیارچہ  کے پھٹنے  سے پیدا ہوا۔ (ا) صرف زمین کا گھماو  مد نظر رکھتے ہوئے، معلوم کریں کہ سیارچہ کتنی دیر بعد   پہنچنے پر دھماکہ \عددی{\SI{25}{\degree}} مشرق کے خط طول بلد  پر واقع شہر  ہلسنکی   کے اوپر ہوتا۔ ایسی  صورت میں شہر مکمل طور پر تباہ ہو جاتا۔ (ب)  اس کے برعکس اگر سیارچہ\ترچھا{ دھاتی } ہوتا ، یہ سطح زمین پر پہنچ پاتا۔    کتنی دیر بعد پہنچنے پر دھماکہ بحر القیانوس میں \عددی{\SI{20}{\degree}} مغرب کے  خط طول بلد  پر ہوتا؟ (دھماکے سے پیدا   سونامی  بحر القیانوس کے دونوں اطراف  ساحلی آبادی تباہ کرتا۔)
\انتہا{سوال}
%----------------------------
%Q85 p293
\ابتدا{سوال}
گالف کا گیند  افق سے \عددی{\SI{20}{\degree}} زاویے پر \عددی{\SI{60}{\meter\per\second}} رفتار اور \عددی{\SI{90}{\radian\per\second}} شرح گھماو سے پھینکا جاتا ہے۔ ہوا کی گھساٹ  نظر انداز کریں۔ بلند ترین نقطے تک پہنچنے تک گیند کتنے چکر کاٹتا ہے؟
\انتہا{سوال}
%-------------------------
\ابتدا{سوال}
دو دائری چھلوں کا مرکز ایک نقطے پر رکھ کر انہیں تین  بلا کمیت  سلاخوں سے ہم سطحی  جوڑا جاتا ہے (شکل \حوالہء{10.54})۔ نظام کے  مرکز پر واقع انتصابی محور کے گرد نظام، جو فی الحال ساکن ہے،  گھوم سکتا ہے۔ چھلوں کی کمیت، اندرونی رداس، اور بیرونی رداس درج ذیل  جدول میں پیش ہیں۔ بیرونی چھلا کے بیرونی کنارے پر \عددی{\SI{0.300}{\second}} کے لئے \عددی{\SI{12.0}{\newton}} قدر کی  مماسی قوت لاگو کی جاتی ہے۔ اس دورانیے میں نظام کی زاوی رفتار میں تبدیلی کیا ہو گی؟
\begin{center}
\begin{tabular}{CCCC}
\toprule
\text{\RL{چھلا}} & (\si{\kilo\gram})\,\text{\RL{کمیت}} &(\si{\meter})\, \text{\RL{اندرونی رداس}} & (\si{\meter})\,\text{\RL{بیرونی رداس}}\\
\midrule
1& 0.120& 0.0160 & 0.0450\\
2& 0.240 & 0.0900 & 0.1400\\
\bottomrule
\end{tabular}
\end{center}
\انتہا{سوال}
%--------------------------
%Q87 p293
\ابتدا{سوال}
بلا رگڑ افقی دھرے پر \عددی{\SI{0.20}{\meter}} رداس کا پہیا نصب کیا جاتا ہے۔ بلا کمیت دھاگا پہیے کے گرد لپیٹ کر دھاگے کے آزاد سر کے ساتھ ، افق سے \عددی{\theta=\SI{20}{\degree}}   بلا رگڑ  میلان  پر رکھی، \عددی{\SI{2.0}{\kilo\gram}} اینٹ باندھی  جاتی ہے (شکل \حوالہء{10.55})۔میلان سطح  کے ہمراہ     اینٹ  \عددی{\SI{2.0}{\meter\per\second\squared}} اسراع سے نشیبی  حرکت کرتی   ہے۔  دھرے  پر پہیے کا گھمیری جمود کیا ہے؟
\انتہا{سوال}
%------------------------------
\ابتدا{سوال}
ایک پتلے کروی خول کا رداس \عددی{\SI{1.90}{\meter}} ہے۔ خول کو \عددی{\SI{960}{\newton\meter}} قوت مروڑ ، کرہ کے مرکز پر واقع محور کے لحاظ سے، \عددی{\SI{6.20}{\radian\per\second\squared}}زاوی  اسراع دیتی ہے۔ (ا) اس محور پر کرہ کا گھمیری جمود  اور (ب) خول کی کمیت کیا ہے؟
\انتہا{سوال}
%----------------------------
%Q89 p293
\ابتدا{سوال}
سائیکل  سوار ،جس کی کمیت \عددی{\SI{70}{\kilo\gram}} ہے،  چڑھائی  پر  چڑھتے ہوئے  باری باری سائیکل کے نشیب وار حرکت کرتے  پائدان  پر اپنی   پوری کمیت  ڈالتا ہے۔ پائدان   \عددی{\SI{0.40}{\meter}}   قطر   دائرے پر چلتا ہے۔ پائدان کے محور گھماو پر سائیکل سوار زیادہ سے زیادہ کتنی قوت مروڑ ڈالتا ہے۔
\انتہا{سوال}
%---------------------------
%Q90 p293
\ابتدا{سوال}
انجن کا اڑن پہیا \عددی{\SI{25.0}{\radian\per\second}} زاوی رفتار سے گھومتا ہے۔ انجن بند کرنے پر اڑن پہیا مستقل شرح سے بتدریج  آہستہ ہو کر \عددی{\SI{20.0}{\second}} میں رکتا ہے۔ (ا)  اڑن پہیے کا زاوی اسراع ، (ب)  رکنے تک طے شدہ زاویہ، اور (ج) رکنے تک کاٹے گئے چکر تلاش کریں۔
\انتہا{سوال}
%-----------------------
\ابتدا{سوال}
رداس \عددی{\SI{0.20}{\meter}} کا پہیا بلا رگڑ افقی محور پر نصب ہے (شکل \حوالہء{10.19a})۔  محور پر پہیے  کا گھمیری جمود \عددی{\SI{0.400}{\kilo\gram\meter\squared}} ہے۔ بلا کمیت دھاگہ  پہیے کے محیط پر لپیٹ کر  دھاگے کا دوسرا سر \عددی{\SI{6.0}{\kilo\gram}} اینٹ سے باندھا جاتا ہے۔ ساکن حالت سے نظام رہا کیا جاتا ہے۔ جب اینٹ کی حرکی توانائی \عددی{\SI{6.0}{\joule}}  ہوتی ہے، (ا) پہیے کی گھمیری حرکی توانائی کیا ہو گی  اور (ب)  اینٹ کتنا  نشیب وار فاصلہ طے گر چکی ہو گی؟
\انتہا{سوال}
%-------------------------
\ابتدا{سوال}
\اصطلاح{دودھیا کہکشاں }\فرہنگ{کہکشاں!دودھیا}\حاشیہب{Milky Way galaxy}\فرہنگ{galaxy!Milky Way} کے مرکز سے  سورج  کا فاصلہ \عددی{\num{2.3e4}} نوری سل ہے۔  کہکشاں کے مرکز کے گرد سورج \عددی{\SI{250}{\kilo\meter\per\second}} رفتار سے  گھومتا ہے۔ (ا)  کہکشاں کے مرکز کے گرد سورج ایک چکر کتنے عرصہ میں مکمل کرتا ہے؟ (ب)  سورج کی پیدائش سے اب تک، سورج کتنے چکر کاٹ چکا ہے۔ سورج کی پیدائش کو  \عددی{\num{4.5e9}} سال  ہو چکے ہیں۔
\انتہا{سوال}
%-------------------------
%Q93 p293
\ابتدا{سوال}
بلا رگڑ افقی محور پر رداس  \عددی{\SI{0.20}{\meter}} کا پہیا نصب ہے۔ محور پر پہیے کا گھمیری جمود \عددی{\SI{0.050}{\kilo\gram\meter\squared}} ہے۔ پہیے کے  گرد لپٹے  دھاگے کے سر سے \عددی{\SI{2.0}{\kilo\gram}} اینٹ بندھی ہے جو بلا رگڑ افقی سطح پر حرکت کر سکتی ہے۔اگر اینٹ پر \عددی{P=\SI{3.0}{\newton}} قدر کی قوت لاگو کی جائے، جیسا شکل \حوالہء{10.56} میں دکھایا گیا ہے، پہیے کے زاوی اسراع کی قدر کیا ہو گی؟ دھاگہ پہیے پر پھسلتا نہیں ہے۔
\انتہا{سوال}
%--------------------------
\ابتدا{سوال}
ایک ہوائی جہاز کا  ، جو زمین کے لحاظ سے \عددی{\SI{480}{\kilo\meter\per\hour}} سے پرواز کر رہا ہے، پنکھا \عددی{2000} چکر فی منٹ  سے گھوم رہا ہے۔  (ا)  ہوا باز  اور (ب)  زمین پر کھڑے شخص کے نقطہ نظر سے رداس \عددی{\SI{1.50}{\meter}}  پنکھے کے   پر کا سر کس خطی رفتار سے حرکت کرتا ہے۔ جہاز کی سمتی رفتار اور   پنکھے  کا دھرا متوازی ہیں۔
\انتہا{سوال}
%-----------------
\ابتدا{سوال}
تین کمیتوں  کو بلا کمیت سلاخوں سے جوڑ کر استوار جسم بنایا گیا ہے (شکل \حوالہء{10.57})۔ جسم کو نقطہ \عددی{P} پر واقع، جسم کی سطح کو عمودی،  محور پر  گھمانا مقصود ہے۔ اگر \عددی{M=\SI{0.40}{\kilo\gram}}، \عددی{a=\SI{30}{\centi\meter}}، اور \عددی{b=\SI{50}{\centi\meter}} ہو، ساکن حالت سے جسم کو \عددی{\SI{5.0}{\radian\per\second}} زاوی رفتار تک پہنچانے کے لئے کتنا کام درکار ہو گا؟
\انتہا{سوال}
%---------------------
%Q96 p293
\ابتدا{سوال}
مشروب کے ڈبے میں   کنجی   کا شمول  مشروبات    کی صنعت میں    ایک بڑا قدم تھا۔ ڈبے کے بالا  سر میں  وسطی  قابلے پر کنجی    حرکت کر سکتی ہے۔ کنجی  کا ایک سر اوپر کھینچنے سے  کنجی  کا دوسرا سر ڈبے کے بالا سر کے کمزور کردہ  حصے کو نیچے  دباتی  ہے۔ اگر آپ \عددی{\SI{10}{\newton}} قوت سے کنجی  اوپر کھینچیں، کمزور  کردہ حصے پر کتنی قوت عمل کرتی ہے؟ (مشروب  کا ڈبہ لے کر اس عمل پر غور کرنا ہو گا۔)
\انتہا{سوال}
%--------------------
%Q97 p293
\ابتدا{سوال}
جہاز کا پر شکل \حوالہء{10.58} میں پیش ہے، جو نقطہ \عددی{B} پر واقع انتصابی محور کے گرد  \عددی{2000} چکر فی منٹ  سے گھومتا ہے۔ نقطہ \عددی{A} کا، جو   محور سے پر کا  دور ترین نقطہ ہے، رداس \عددی{\SI{1.50}{\meter}}   ہے۔ (ا)  نقطہ \عددی{B} اور محور سے \عددی{\SI{0.150}{\meter}} رداسی فاصلے پر موجود نقطے  کے  مرکز مائل  اسراع کی قدر میں فرق  \عددی{a} کتنا ہو گا؟ (ب)  \عددی{a} بالمقابل  رداسی فاصلے کی ترسیم کھینچیں۔
\انتہا{سوال}
%-------------------
%Q98 p294
\ابتدا{سوال}
بلا رگڑ افقی  محور پر،   شکل \حوالہء{10.59} میں پیش نظام استعمال کر کے ، \عددی{\SI{30}{\kilo\gram}} کا ڈبہ اٹھایا جاتا ہے۔ بیرونی رداس \عددی{R=\SI{0.50}{\meter}} جبکہ \اصطلاح{ نابھ }\فرہنگ{نابھ}\حاشیہب{hub}\فرہنگ{hub} کا رداس   \عددی{r=\SI{0.20}{\meter}} ہے۔افقی قوت \عددی{\vec{F}} ، جس کی قدر \عددی{\SI{140}{\newton}} ہے  ، لاگو کرنے سے   ڈبہ \عددی{\SI{0.8}{\meter\per\second\squared}}  قدر کے اسراع  سے اوپر  اٹھتا  ہے۔ محور پر نظام کا گھمیری جمود کیا ہے؟
\انتہا{سوال}
%---------------------------
\ابتدا{سوال}
بلا کمیت سلاخ ، جس کی لمبائی \عددی{\SI{0.780}{\meter}} ہے، کے ایک سر پر \عددی{\SI{1.30}{\kilo\gram}} گیند نصب ہے۔  سلاخ کے دوسرے سر پر نظام افقی دائرے میں \عددی{5010} چکر فی منٹ رفتار سے گھومتا ہے۔ (ا)  محور گھماو پر نظام کا گھمیری جمود تلاش کریں۔ (ب) گھماو کے مخالف رخ،   گیند پر ہوائی گھساٹ \عددی{\SI{2.30e-2}{\newton}} ہے۔ نظام کو مستقل رفتار سے گھومتے رکھنے کے لئے کتنی قوت مروڑ درکار ہو گی؟
\انتہا{سوال}
%-----------------------
%Q100 p294
\ابتدا{سوال}
دو پتلی سلاخیں  (جن کی انفرادی کمیت \عددی{\SI{0.20}{\kilo\gram}} ہے) آپس میں جوڑ کر ، شکل \حوالہء{10.60} میں پیش،  استوار جسم بنایا جاتا ہے۔ ایک سلاخ کی لمبائی \عددی{L_1=\SI{0.40}{\meter}} اور دوسرے کی \عددی{L_2=\SI{0.50}{\meter}} ہے۔ (ا)  چھوٹی سلاخ کے وسطی نقطے پر واقع ، سطح  صفحہ کو عمودی ، محور پر  استوار جسم کا گھمیری جمود تلاش کریں۔ (ب)    لمبی  سلاخ کے وسطی نقطے پر واقع ، سطح  صفحہ کو عمودی ، محور پر  استوار جسم کا گھمیری جمود تلاش کریں۔
\انتہا{سوال}
%-----------------------------
\ابتدا{سوال}
چار جرثقیل  کو دو  پٹوں  سے ملایا جاتا ہے (شکل \حوالہء{10.61})۔ جرثقیل \عددی{A} (رداس \عددی{\SI{15}{\centi\meter}})  محرک  جرثقیل ہے، اور \عددی{\SI{10}{\radian\per\second}} سے گھومتا ہے۔ جرثقیل \عددی{B} (رداس \عددی{\SI{10}{\centi\meter}}) اور جرثقیل \عددی{A} کو پٹہ \عددی{1 } ملاتا ہے۔ جرثقیل \عددی{B'} (رداس \عددی{\SI{5}{\centi\meter}})  اور جرثقیل \عددی{B} ہم محور ہیں   اور آپس میں پکے جڑے ہیں۔ جرثقیل \عددی{C} (رداس \عددی{\SI{25}{\centi\meter}})   اور جرثقیل \عددی{B'} کو پٹہ \عددی{2} ملاتا ہے۔ (ا)  پٹہ \عددی{1} پر نقطے کی خطی  رفتار، (ب) جرثقیل \عددی{B} کی زاوی رفتار، (ج) جرثقیل \عددی{B'} کی زاوی رفتار، (د)  پٹہ \عددی{2} پر نقطے کی خطی رفتار، اور (ہ) جرثقیل \عددی{C} کی زاوی رفتار تلاش کریں۔ (اشارہ:اگر دو   جرثقیل  ملانے والا  پٹہ نہ پھسلتا ہو، ان جرثقیل کے چکا کی مماسی رفتاریں  برابر ہوں  گی۔)
\انتہا{سوال}
%---------------------
%Q102 p294
\ابتدا{سوال}
تین گیند  کو تین سلاخ ملا کر استوار جسم  دیتے  ہیں (شکل \حوالہء{10.62})، جہاں \عددی{M=\SI{1.6}{\kilo\gram}}، \عددی{L=\SI{0.60}{\meter}}، اور \عددی{\theta=\SI{30}{\degree}} ہے۔ گیند کو ذرہ تصور کیا جا سکتا ہے اور سلاخ بلا کمیت ہیں۔ (ا) نقطہ \عددی{P} پر واقع  جسم کی سطح کو عمودی محور پر اور (ب) نقطہ \عددی{P} پر واقع اور \عددی{2L} لمبی سلاخ  کو عمودی، اور جسم کے مستوی میں پائی جانے والی محور پر جسم کی گھمیری حرکی توانائی اس صورت میں تلاش کریں جب جسم  کی زاوی رفتار \عددی{\SI{1.2}{\radian\per\second}} ہو۔
\انتہا{سوال}
%-------------------------
\ابتدا{سوال}
نقطہ \عددی{A} پر موجود افقی محور کے گرد  (\عددی{\SI{3.0}{\kilo\gram}} کمیت اور \عددی{\SI{4.0}{\meter}} لمبی)  پتلی یکساں سلاخ آزادانہ  گھومتی ہے (شکل \حوالہء{10.63})۔ نقطہ \عددی{A} سلاخ کے  سر سے \عددی{d=\SI{1.0}{\meter}} فاصلے پر ہے۔ انتصابی مقام سے گزرتے وقت سلاخ کی حرکی توانائی \عددی{\SI{20}{\joule}} ہے۔ (ا)  محور \عددی{A} پر سلاخ کا گھمیری جمود کیا ہے؟ (ب)  سلاخ کے سر \عددی{B} کی (خطی) رفتار  اس وقت کیا ہو گی جب سلاخ انتصابی مقام سے گزر رہی ہو؟ (ج)  اوپر جاتے ہوئے سلاخ کس زاویہ \عددی{\theta} پر لمحاتی رکتی ہے؟
\انتہا{سوال}
%-------------------------
\ابتدا{سوال}
چار ذروں کو، جن کی انفرادی کمیت \عددی{\SI{0.20}{\kilo\gram}} ہے، چوکور کے کونوں پر رکھا جاتا ہے۔ چوکور کا اضلاع کی انفرادی لمبائی \عددی{\SI{0.50}{\meter}} ہے۔ ذروں کو بلا کمیت سلاخوں سے جوڑا جاتا ہے۔ استوار جسم انتصابی مستوی میں افقی   محور  \عددی{A} کے گرد گھوم سکتا ہے۔ \عددی{A} ایک ذرے کے مرکز سے گزرتی ہے۔ سلاخ \عددی{AB} افقی رکھ کر جسم کو ساکن حالت سے رہا کیا جاتا ہے (شکل \حوالہء{10.64})۔ (ا)  محور \عددی{A} پر جسم کا گھمیری جمود کیا ہے؟ (ب)  جب سلاخ \عددی{AB} انتصابی مقام سے جھول کر گزرتی ہے،  \عددی{A} پر جسم کی زاوی رفتار کیا ہو گی؟
\انتہا{سوال}
%-------------------
%Q105 p294
\ابتدا{سوال}
چیتا  کو \عددی{\SI{114}{\kilo\meter\per\hour}} کی  حیرت کن  رفتار پر دوڑتا  دیکھا گیا ہے۔ فرض کریں آپ چیتا کے ہمراہ گاڑی میں چلتے ہوئے چیتا کی رفتار جاننے کے لئے رفتار پیما پر نظر ڈالتے ہیں
 جو \عددی{\SI{114}{\kilo\meter\per\hour}} دیتا ہے۔  آپ گاڑی کو چیتا سے مستقل طور پر  \عددی{\SI{8.0}{\meter}} دور رکھتے ہیں، تاہم چیتا گاڑی  کے  ڈر سے  مسلسل دور ہٹتے ہوئے \عددی{\SI{92}{\meter}} رداسی  راہ پر دوڑتا ہے۔ یوں آپ \عددی{\SI{100}{\meter}} رداس کے دائرے پر گاڑی چلاتے ہیں۔ (ا) دائرے راہ پر  چلتے ہوئے آپ کی اور چیتا کی زاوی رفتار کیا ہے؟ (ب)  اس راہ پر چیتا کی خطی رفتار کیا ہو گی؟ (اگر آپ دائری راہ کی لمبائیوں میں فرق  حساب میں شامل نہ کرتے، آپ کہتے چیتا کی رفتار \عددی{\SI{114}{\kilo\meter\per\hour}} ہے؛ جو سراسر غلط ہو گا۔ بظاہر اسی غلطی کی بنا چیتا کی رفتار اتنی زیادہ بتائی گئی۔)
\انتہا{سوال}
%----------------------
\ابتدا{سوال}
ایک پہیے کے چکا پر نقطے کی رفتار \عددی{\SI{6.2}{\second}} میں \عددی{\SI{12}{\meter\per\second}} سے مستقل شرح سے بڑھتے ہوئے \عددی{\SI{25}{\meter\per\second}} ہوتی ہے۔ پہیے کا  قطر  \عددی{\SI{0.75}{\meter}} ہے۔ پہیے کی اوسط زاوی اسراع کیا ہو گی؟
\انتہا{سوال}
%---------------------
\ابتدا{سوال}
ایک جرثقیل  ، جس کا  قطر \عددی{\SI{8.0}{\centi\meter}} ہے،کے گرد \عددی{\SI{5.6}{\meter}} ڈور لپیٹی جاتی ہے۔ ساکن حالت سے آغاز کر کے اس کو \عددی{\SI{1.5}{\radian\per\second\squared}}  مستقل اسراع  دیا جاتا ہے۔ (ا) ڈور مکمل  اترنے تک جرثقیل کتنا زاویہ طے کرتا ہے، اور (ب) ایسا کتنی دیر میں ہو گا؟
\انتہا{سوال}
%------------------------
%Q108 p294
\ابتدا{سوال}
گراموفون کی تھالی \عددی{33\tfrac{1}{3}} چکر فی منٹ سے گھمائی جاتی ہے۔ (ا) اس کی زاوی رفتار ریڈیئن فی سیکنڈ میں کیا ہو گی؟ تھالی  کے محور گھماو سے   (ب) \عددی{\SI{15}{\centi\meter}} اور (ج) \عددی{\SI{7.4}{\centi\meter}} رداسی فاصلے پر نقطے کی خطی رفتار کیا ہو گی؟
\انتہا{سوال}
%-------------------------------

