%p257
\باب{گھماو}
\حصہ{گھماو کے متغیر}
\جزوحصہء{مقاصد}
اس حصہ کو پڑھنے کے بعد آپ درج ذیل کے قابل ہوں گے۔
\begin{enumerate}[1.]
\item
جان پائیں گے اگر جسم کے تمام حصے ایک  محور کے گرد  ہم قدم گھومیں، یہ   استوار  جسم ہو گا۔ (اس باب میں ایسے اجسام پر گفتگو کی جائے گی۔)
\item
جان پائیں گے کہ  اندرونی حوالہ لکیر اور مقررہ  بیرونی حوالہ لکیر  کے بیچ زاویہ،  استوار جسم کا زاویاتی مقام دیگا۔
\item
ابتدائی اور اختتامی زاویاتی مقام  کا زاویاتی ہٹاو کے ساتھ تعلق استعمال کر پائیں گے۔
\item
اوسط زاویائی سمتی رفتار،  زاویائی ہٹاو، اور ہٹاو کو درکار دورانیے کا  تعلق استعمال کر پائیں گے۔
\item
اوسط زاویائی  اسراع ،  زاویائی  سمتی رفتار میں تبدیلی، اور اس تبدیلی کو درکار دورانیے کا  تعلق استعمال کر پائیں گے۔
\item
جان پائیں گے کہ خلاف  گھڑی  حرکت مثبت  رخ اور گھڑی وار حرکت منفی  رخ ہو گا۔
\item
زاویائی مقام   کو\ترچھا{   وقت  کا تفاعل } جانتے ہوئے، کسی بھی لمحے پر لمحاتی زاویائی سمتی رفتار اور دو مختلف وقتوں کے بیچ اوسط زاویائی سمتی رفتار تعین کر  پائیں گے۔
\item
زاویائی مقام   بالمقابل   وقت   کی \ترچھا{ ترسیم }سے  کسی بھی لمحے پر لمحاتی زاویائی سمتی رفتار اور دو مختلف وقتوں کے بیچ اوسط زاویائی سمتی رفتار تعین کر  پائیں گے۔
\item
جان پائیں گے کہ لمحاتی زاویائی  سمتی رفتار  کی قدر لمحاتی زاویائی رفتار ہو گی۔
\item
زاویائی  سمتی رفتار    کو  \ترچھا{وقت  کا تفاعل } جانتے ہوئے، کسی بھی لمحے پر لمحاتی زاویائی  اسراع اور دو مختلف وقتوں کے بیچ اوسط زاویائی  اسراع تعین کر  پائیں گے۔
\item
زاویائی سمتی رفتار    بالمقابل   وقت   کی\ترچھا{ ترسیم }سے  کسی بھی لمحے پر لمحاتی زاویائی اسراع اور دو مختلف وقتوں کے بیچ اوسط زاویائی اسراع تعین کر  پائیں گے۔
\item
وقت کے ساتھ زاویائی اسراع تفاعل کا تکمل  لے کر جسم کی زاویائی سمتی رفتار میں تبدیلی تعین کر پائیں گے۔

وقت کے ساتھ زاویائی  سمتی رفتار  تفاعل کا تکمل  لے کر جسم کے  زاویائی مقام میں تبدیلی تعین کر پائیں گے۔
\end{enumerate}

\جزوحصہء{کلیدی تصور}
\begin{itemize}
\item
مقررہ محور،  جو محور گھماو  کہلاتی ہے،  کے گرد استوار جسم کا  گھماو  بیان کرنے کی خاطر ،    جسم کے اندر محور کو عمودی   حوالہ لکیر فرض کی جاتی ہے جو جسم کے ساتھ ہم قدم محور کے گرد گھومتی ہے۔   ایک مقررہ رخ کے ساتھ اس لکیر کا زاویائی مقام \عددی{\theta} ناپا جاتا ہے۔ جب \عددی{\theta} کی پیمائش ریڈیئن میں ہو، ذیل ہو گا،
\begin{align*}
\theta=\frac{s}{r}\quad\quad \text{\RL{(ریڈیئن ناپ)}}
\end{align*}
جہاں رداس \عددی{r} کے دائری راہ کا قوسی فاصلہ \عددی{s} اور ریڈیئن میں زاویہ \عددی{\theta} ہے۔
\item
زاویہ کی  درجہ میں اور چکر میں پیمائش کا ریڈیئن پیمائش سے تعلق ذیل ہے۔
\begin{align*}
\text{\RL{ریڈیئن}}\,2\pi=\SI{360}{\degree}=\text{\RL{چکر}}\,1
\end{align*}
\item
ایک جسم جو محور گھماو  کے گرد گھوم کر  اپنا زاویائی مقام \عددی{\theta_1} سے تبدیل کر کے \عددی{\theta_2} کرے،  ذیل زاویائی ہٹاو  سے گزرتا ہے،
\begin{align*}
\Delta \theta=\theta_2-\theta_1
\end{align*}
جہاں خلاف گھڑی گھماو کے لئے \عددی{\Delta \theta} مثبت اور گھڑی وار گھماو کے لئے منفی ہو گا۔
\item
اگر  جسم \عددی{\Delta t} دورانیہ میں \عددی{\Delta \theta} زاویائی ہٹاو  گھومے، اس کی اوسط زاویائی سمتی  رفتار  \عددی{\omega_{\text{\RL{اوسط}}}} ذیل ہو گی۔
\begin{align*}
\omega_{\text{\RL{اوسط}}}=\frac{\Delta \theta}{\Delta t}
\end{align*}
جسم کی ( لمحاتی ) زاویائی  سمتی رفتار \عددی{\omega} ذیل ہو گی۔
\begin{align*}
\omega=\frac{\dif \theta}{\dif t}
\end{align*}
اوسط  زاویائی سمتی رفتار \عددی{\omega_{\text{\RL{اوسط}}}}  اور سمتی رفتار  \عددی{\omega} دونوں سمتی مقادیر ہیں، جن کا رخ دایاں ہاتھ قاعدہ  دیگا۔ خلاف گھڑی گھماو کے لئے ان کا رخ مثبت اور گھڑی وار گھماو کے لئے منفی ہو گا۔ زاویائی سمتی رفتار کی قدر جسم  کی زاویائی رفتار ہو گی۔
\item
اگر \عددی{\Delta t=t_2-t_1} دورانیہ میں جسم کی زاویائی سمتی رفتار \عددی{\omega_1} سے تبدیل ہو کر  \عددی{\omega_2} ہو، اس کا  اوسط زاویائی  اسراع \عددی{\alpha_{\text{\RL{اوسط}}}} ذیل ہو گا۔
\begin{align*}
\alpha_{\text{\RL{اوسط}}}=\frac{\omega_2-\omega_1}{t_2-t_1}=\frac{\Delta \omega}{\Delta t}
\end{align*}
جسم کا  ( لمحاتی ) زاویائی اسراع \عددی{\alpha}ذیل ہو گا۔
\begin{align*}
\alpha=\frac{\dif \omega}{\dif t}
\end{align*}
\عددی{\alpha_{\text{\RL{اوسط}}}} اور \عددی{\alpha} دونوں سمتی مقادیر ہیں۔
\end{itemize}

\حصہء{طبیعیات کیا ہے؟}
جیسا ہم پہلے ذکر کر چکے، طبیعیات  کی توجہ کا ایک  مرکز \قول{ حرکیات }ہے۔ تاہم، اب تک ہم صرف\اصطلاح{ مستقیم   حرکت } پر بات کرتے رہے ہیں، جس میں جسم سیدھی یا قوسی  لکیر  پر حرکت کرتا ہے (شکل \حوالہء{10-1a})۔ اب ہم \اصطلاح{ گھماو } پر نظر ڈالتے ہیں، جس میں جسم کسی محور کے گرد گھومتا ہے (شکل \حوالہء{10.1b})۔

گھماو تقریباً ہر مشین میں نظر آتا ہے، اور جب  آپ دروازہ کھولتے ہیں آپ اس کو دیکھتے ہیں۔کھیل میں  گھماو اہم کردار ادا کرتا ہے، جیسا  گیند کو زیادہ دور پھینکنے کے لئے (گھومتے  گیند  کو ہوا زیادہ دیر  اٹھا  کر سکتی ہے)، اور کرکٹ میں گیند  قوسی  راہ پر پھینکنے کے لئے (گھومتے گیند کو ہوا دائیں یا بائیں دھکیلتی ہے)۔ گھماو زیادہ اہم مسائل ، جیسا      عمر رسیدہ  ہوائی جہاز میں دھاتی حصوں   کا ٹوٹ پھوٹ، میں بھی  کلیدی کردار ادا کرتا ہے۔

گھماو پر بحث سے قبل   ، حرکت میں ملوث متغیرات متعارف کرتے ہیں، جیسا ہم نے باب \حوالہء{2} میں مستقیم حرکت پر بحث سے قبل کیا۔ ہم دیکھتے ہیں کہ گھماو کے  متغیرات عین   با ب \حوالہء{2} میں یک بُعدی  حرکت  کے متغیرات کی طرح ہیں؛  ایک اہم خصوصی صورت وہ ہے جہاں اسراع (جو یہاں زاویائی اسراع ہو گا)   مستقل ہو۔ ہم دیکھتے ہیں  نیوٹن کا دوسرا قاعدہ  زاویائی حرکت کے لئے بھی لکھا جا سکتا ہے، تاہم  اب قوت  کی بجائے ایک نئی  مقدار جو \ترچھا{  قوت مروڑ } کہلاتی ہے استعمال  کرنا ہو گا۔  کام اور  کام و حرکی توانائی  مسئلے کا اطلاق   بھی گھماو  حرکت  پر کیا جا سکتا ہے، تاہم  کمیت کی بجائے ایک نئی مقدار جو \ترچھا{زاویائی جمود} کہلاتی ہے استعمال کرنا ہو  گا۔ مختصراً،  ہم جو کچھ پڑھ چکے ہیں، اس کا اطلاق گھماو حرکت میں ہو گا، تاہم کبھی کبھار معمولی تبدیلی  کی ضرورت پیش آئے گی۔

\موٹا{انتباہ:}
اگرچہ اس باب میں زیادہ تر حقائق محض  دوبارہ پیش کیے گئے ہیں، دیکھا یہ گیا ہے کہ طلبہ و طالبات کو اس باب میں دشواری پیش آتی ہے۔ اساتذہ کرام اس کی کئی وجوہات پیش کرتے ہیں جن میں سے دو  پر اتفاق پایا جاتا ہے: \عددی{1} یہاں  علامت    کی تعداد بہت زیادہ ہے (جنہیں  یونانی حروف  میں لکھ کر  مشکل میں  مزید اضافہ پیدا ہوتا ہے)، اور \عددی{2}  آپ خطی حرکت سے زیادہ واقف ہیں (اسی لئے  کمرے کے ایک کونے سے دوسرے کونے تک آپ  با آسانی جا سکتے ہیں)،  لیکن گھماو سے آپ کا واسطہ کم رہا ہے (اسی لئے تفریح  گاہ میں آپ  تفریحی جھولے پر سوار ہونے کے لئے پیسہ خرچنے کے لئے راضی ہوتے ہیں)۔ جہاں آپ کو دشواری ہو، دیکھیں آیا مسئلے کو  باب \حوالہء{2} کا یک بُعدی خطی مسئلہ   تصور کرنے  آسانی پیدا ہوتی ہے۔ مثلاً، اگر آپ سے\ترچھا{ زاویائی } فاصلہ معلوم کرنے کو کہا جائے، وقتی طور پر  لفظ \ترچھا{زاویائی} کو بھول جائیں اور دیکھیں آیا باب \حوالہء{2}  کی ترقیم اور تصورات استعمال کر کے جواب حاصل کرنا آسان ہوتا ہے۔

\جزوحصہء{گھماو کے  متغیر}
ہم مقررہ محور  پر استوار  جسم کے گھماو  پر غور کرنا چاہتے ہیں۔\اصطلاح{ استوار  جسم }\فرہنگ{استوار جسم!تعریف}\حاشیہب{rigid body}\فرہنگ{rigid body!defined} سے مراد  وہ جسم ہے جس  کے تمام  حصے  ، جسم کی شکل و صورت تبدیل کیے بغیر، ہم قدم  گھوم سکتے ہیں۔ \اصطلاح{مقررہ محور }\فرہنگ{مقررہ محور!تعریف}\حاشیہب{fixed axis}\فرہنگ{fixed axis!defined} سے مراد وہ محور ہے جو حرکت نہیں کرتی اور   جس  پر گھوما جا سکتا ہے۔یوں ہم ایسے جسم پر غور نہیں کریں گے جیسا  سورج   (جو گیس  کا کرہ  ہے) جس کے  حصے ایک ساتھ حرکت نہیں کرتے۔ ہم زمین پر  لڑھکتے گیند کی بھی بات نہیں کرتے چونکہ اس کا محور خود حرکت پذیر ہے (ایسی گیند کی حرکت،   گھماو اور  مستقیم حرکت کا ملاپ ہے )۔

شکل \حوالہء{10.2} میں  مقررہ محور پر ، جو\اصطلاح{ محور گھماو}\فرہنگ{ محور گھماو!تعریف}\حاشیہب{rotation axis}\فرہنگ{rotation axis!defined}یا \اصطلاح{گھماو کی محور } کہلاتی ہے، اختیاری شکل کا استوار  جسم  گھوم رہا ہے۔ خالص  گھماو  (\ترچھا{زاویائی حرکت}) میں ،  جسم کا ہر نقطہ ایسے  دائرہ  پر حرکت کرتا ہے، جس کا مرکز  محور  گھماو پر واقع ہے، اور  ہر نقطہ کسی مخصوص وقتی  وقفہ  میں ایک جتنا زاویہ طے کرتا  ہے۔ خالص مستقیم حرکت (خطی حرکت)  میں، جسم کا ہر نقطہ کسی مخصوص وقتی دورانیہ میں  ایک جتنا  \ترچھا{خطی فاصلہ } طے کرتا ہے۔

آئیں باری باری خطی مقادیر  مقام، ہٹاو، سمتی رفتار، اور اسراع کے مماثل زاویائی  مقادیر  پر  غور کرتے ہیں۔

\جزوحصہء{زاویائی مقام}
شکل \حوالہء{10.2} میں گھماو کو عمودی، جسم کے ساتھ  گھومتی، جسم  سے پکی  جڑی   \ترچھا{ حوالہ لکیر } دکھائی گئی ہے  ۔ کسی مقررہ رخ کے ساتھ ، جس کو ہم \اصطلاح{ صفر زاویائی مقام }\فرہنگ{زاویائی مقام!صفر}\حاشیہب{zero angular position}\فرہنگ{angular position!zero} مانتے ہیں، اس لکیر کا زاویہ لکیر کا \اصطلاح{ زاویائی مقام }\فرہنگ{زاویائی مقام!تعریف}\حاشیہب{angular position}\فرہنگ{angular position!defined}  ہو گا۔ شکل \حوالہء{10.3} میں  محور \عددی{x} کے مثبت رخ کے ساتھ زاویائی مقام  \عددی{\theta} ناپا گیا ہے۔ ہندسہ سے ہم جانتے ہیں درج ذیل ہو گا۔
\begin{align}
\theta=\frac{s}{r}\quad\quad \text{\RL{(ریڈیئن ناپ)}}
\end{align}
یہاں محور \عددی{x}  (جو صفر زاویائی مقام ہے) سے حوالہ  لکیر  تک دائری قوس کی لمبائی \عددی{s}، اور دائرے کا رداس \عددی{r} ہے۔

اس طرح تعین کیا گیا زاویہ  ، درجہ یا چکر کی بجائے ، \اصطلاح{ریڈیئن }\فرہنگ{ریڈیئن}\حاشیہب{radian}\فرہنگ{radian} میں ناپا جاتا ہے۔ ریڈیئن دو لمبائیوں  کی نسبت   (تقابلی تعلق)ہے  لہٰذا یہ  بے بُعد خالص عدد ہو گا۔ دائرے  کا محیط \عددی{2\pi r} ہے لہٰذا ایک مکمل دائرے میں \عددی{2\pi} ریڈیئن ہوں گے۔
\begin{align}
\text{\RL{چکر}}\, 1=\SI{360}{\degree}=\frac{2\pi r}{r}=\text{\RL{ریڈیئن}}\, 2\pi
\end{align}
یا
\begin{align}
\text{\RL{ریڈیئن}}\,1=\SI{57.3}{\degree}=\text{\RL{چکر}}\,0.159
\end{align}
 محور گھماو پر حوالہ لکیر کی  مکمل  چکر کے بعد ہم \عددی{\theta} واپس  صفر\ترچھا{ نہیں } کرتے۔اگر حوالہ لکیر صفر زاویائی مقام سے  ابتدا کر کے دو چکر  مکمل  کرے، لکیر کا زاویائی مقام \عددی{\theta=4\pi} ریڈیئن ہو گا۔
 
محور \عددی{x} پر  خالص مستقیم حرکت کے لئے  \عددی{x(t)} ، یعنی مقام بالمقابل وقت،  جانتے ہوئے ہم حرکت پذیر جسم کے بارے میں وہ سب کچھ معلوم کر سکتے ہیں جنہیں جاننا مقصود ہو۔ اسی طرح، خالص گھماو  کے لئے \عددی{\theta(t)}، یعنی زاویائی مقام بالمقابل وقت، جانتے ہوئے ہم گھومتے  جسم  کے بارے میں  وہ سب کچھ معلوم کر سکتے ہیں جنہیں جاننا مقصود ہو۔

\جزوحصہء{زاویائی ہٹاو}
اگر شکل \حوالہء{10.3}  کا جسم  محور گھماو پر شکل \حوالہء{10.4}  کی طرح  گھوم کر حوالہ لکیر کا زاویائی مقام \عددی{\theta_1} سے  تبدیل کر کے \عددی{\theta_2}  کرے، جسم کا زاویائی ہٹاو  \عددی{\Delta \theta} ذیل ہو گا۔
\begin{align}
\Delta \theta=\theta_2-\theta_1
\end{align}
زاویائی ہٹاو کی یہ تعریف نہ صرف استوار جسم بلکہ جسم کے  اندر  ہر ذرہ کے لئے درست ہے۔

\موٹا{گھڑیاں منفی ہیں۔}
محور \عددی{x} پر  مستقیم حرکت کی صورت میں جسم کا ہٹاو \عددی{\Delta x}  مثبت یا منفی ہو گا، جو  ،محور پر جسم کی حرکت کے رخ پر منحصر ہے۔ اسی طرح، گھماو کی صورت میں جسم کا  زاویائی ہٹاو \عددی{\Delta \theta} درج ذیل قاعدہ کے تحت  مثبت یا منفی ہو گا۔

\ابتدا{قاعدہ}
خلاف گھڑی زاویائی ہٹاو مثبت اور گھڑی وار ہٹاو منفی ہو گا۔
\انتہا{قاعدہ}

\قول{گھڑیاں  منفی ہیں} کا فقرہ اس قاعدے کو یاد رکھنے  میں مدد دے سکتا ہے۔یاد رہے  گھڑی  کے سیکنڈ   کی سوئی کا ہر قدم آپ کی زندگی کاٹتی ہے۔

\ابتدا{آزمائش}
قرص اپنے وسطی محور کے گرد گھوم سکتا ہے۔ درج ذیل  ابتدائی  اور اختتامی زاویائی مقام کی  مرتب جوڑیوں میں کونسی  منفی زاویائی ہٹاو دیتی ہیں؟ (ا)  ابتدائی \عددی{-3}  ریڈیئن، اختتامی \عددی{+5} ریڈیئن؛ 
(ب)   ابتدائی \عددی{-3}  ریڈیئن، اختتامی \عددی{-7} ریڈیئن؛  (ج)   ابتدائی \عددی{7}  ریڈیئن، اختتامی \عددی{-3} ریڈیئن۔
\انتہا{آزمائش}

\جزوحصہء{زاویائی سمتی رفتار}
فرض کریں ایک جسم وقت \عددی{t_1} پر زاویائی مقام \عددی{\theta_1} پر اور  وقت \عددی{t_2} پر زاویائی مقام \عددی{\theta_2} پر  ہو، جیسا شکل \حوالہء{10.4} میں دکھایا گیا ہے۔  ہم \عددی{t_1} تا \عددی{t_2} وقتی دورانیہ \عددی{\Delta t} میں جسم کی \اصطلاح{ اوسط زاویائی سمتی رفتار }\فرہنگ{زاویائی سمتی رفتار!اوسط، تعریف}\حاشیہب{average angular velocity}\فرہنگ{angular velocity!average, defined}  \عددی{\omega_{\text{\RL{اوسط}}}} کی تعریف ذیل کرتے ہیں،
\begin{align}
\omega_{\text{\RL{}}}=\frac{\theta_2-\theta_1}{t_2-t_1}=\frac{\Delta \theta}{\Delta t}
\end{align}
جہاں وقت دورانیہ \عددی{\Delta t} میں زاویائی ہٹاو \عددی{\Delta \omega} ہے۔ (زاویائی سمتی رفتار کے لئے یونانی  حروف  تہجی کا ، چھوٹی لکھائی میں  ،  آخری حرف  \موٹا{اومیگا } \عددی{\omega}  استعمال کیا جائے گا۔)
%----------------------------------------------------------------
%p261
