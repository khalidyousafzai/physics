%???KKK not edited
%module3.1
%HRp57
\جزوحصہ{سمتیات اور ان کے اجزاء}
%Q3.1-1
\ابتدا{سوال}
ایک سمتیہ جس کا قدر
 \عددی{\SI{7.3}{\meter}} 
 ہے مثبت x محور کے رخ سے گھڑی کی سوئی کے مخالف رخ
 \عددی{\SI{250}{\degree}} 
 پر x y مستوی میں پایا جاتا ہے،
 الف اس کا x جز اور
ب  وائی جز تلاش کریں۔ 
\انتہا{سوال}
%--------------------------------------
%Q3.1-2
\ابتدا{سوال}
سمتیہ ہٹاؤ r کا قدر\عددی{\SI{15}{\meter}}   ہے اور یہ x y  مستوی میں زاویہ  \عددی{\theta=\SI{30}{\degree}}  کہ رخ ہے، شکل 3.26 دیکھیں اس سمتیہ کے الف x جز اور
 ب y جز تلاش کریں۔
 \انتہا{سوال}
 %--------------------------
 %Q3.1-3
\ابتدا{سوال} 
 سمتیہ a کا x جز -  \عددی{\SI{25}{\meter}}  اور y جزو  \عددی{\SI{40}{\meter}}  ہے۔  الف سمتیہ a کا قدر کتنا ہے؟  ب سمتیہ a کے رخ اور محور x کے مثبت رخ کے بیچ زاویہ کتنا ہے؟ 
\انتہا{سوال}
%--------------------------
%Q3.1-4
\ابتدا{سوال}
 درج ذیل زاویوں کو ریڈین میں بیان کریں: الف\عددی{\SI{20}{\degree}} ، ب \عددی{\SI{50}{\degree}} ، ج \عددی{\SI{100}{\degree}} ۔ درج ذیل زاویوں کو درجوں کی صورت میں پیش کریں: د 0.330 ریڈین،  ح 2.10 ریڈین، 
\انتہا{سوال}
%===========================
%Q3.1-5
\ابتدا{سوال} 
 ایک بحری جہاز شمال کے رخ\عددی{\SI{120}{\kilo\meter}}  دور نقطہ کی جانب پہنچنا چاہتا ہے۔ سفر کے اغاز سے پہلے ہی ایک غیر متوقع اندھی اس کو نقطہ اغاز سے مشرق جانب  \عددی{\SI{100}{\kilo\meter}}   دور دکھیلتا ہے۔ اس جہاز کو اختتامی نقطہ پر پہنچنے کے لیے الف کتنا فاصلہ طے کرنا ہوگا  ب اسے کس رخ سفر کرنا ہوگا؟
\انتہا{سوال}
%-------------------------
%Q3.1-6
\ابتدا{سوال}
شکل 3.27 میں ایک بھاری مشین کو اف کی رخ سے زاویہ  \عددی{\theta=\SI{20}{\degree}} پر رکھے گئے تختے پر  \عددی{d=\SI{12.5}{\meter}} فاصلے تک گھسیٹا جاتا ہے۔ اس مشین کو (الف) انتصابی روح اور  (ب) اف کی رخ کتنا دور منتقل کیا گیا؟ 
\انتہا{سوال}
%--------------------------
%Q3.1-7
\ابتدا{سوال}
ایک ہٹاؤ جس کا قدر  \عددی{\SI{3}{\meter}}  ہے اور دوسرا ہٹاؤ جس کا قدر  \عددی{\SI{4}{\meter}}  ہے پر غور کریں۔ دکھائیں کہ ان ہٹاؤ سمیات کو استعمال کرتے ہوئے  (الف) 
\عددی{\SI{7}{\meter}}  ، (ب)  \عددی{\SI{1}{\meter}} ، اور (ج) \عددی{\SI{5}{\meter}} قدر کے ہٹاؤ حاصل کیے جا سکتے ہیں۔ 
\انتہا{سوال}
%Module 3.2
\موٹا{ اکائی سمتیات، سمتیات کی جمع بذریعہ اجزاء }
%Q3.2-8
\ابتدا{سوال}
ایک شخص  \عددی{\SI{3.1}{\kilo\meter}} شمال کی طرف چلنے کے بعد  \عددی{\SI{2.4}{\kilo\meter}} مغرب اور اخر میں  \عددی{\SI{5.2}{\kilo\meter}}  جنوب کے رخ چلتا ہے۔ (الف) اس کے حرکت کو ظاہر کرنے کے لیے سمتی …
\انتہا{سوال}
%----------------------------
%Q3.2-9
\ابتدا{سوال}
درج ذیل دو سمتیات دیے گئے ہیں 
\begin{align*}
\vec{a}=(\SI{4}{\meter})\hat{i} - (\SI{3}{\meter})\hat{j} + (\SI{1}{\meter})\hat{k}
\end{align*}
اور
\begin{align*}
\vec{b}=(\SI{-1}{\meter})\hat{i} + (\SI{1}{\meter})\hat{j} + (\SI{4}{\meter})\hat{k}
\end{align*}
اکائی سمتیہ علامتیت میں 
(الف)
\(\vec{a}+\vec{b}\)
، (ب)
 \(\vec{a}-\vec{b}\)
اور (ج)
ایک تیسرا سمتیہ 
\(c\)
تلاش کریں جہاں 
\(\vec{a}-\vec{b}+vec{c}=0\)
\انتہا{سوال}

\ابتدا{سوال}
سوال 10 
ہٹاؤ 
\(c\)
 اور
\(d\)
کے میٹروں میں اجزاء
\(c_x=7.4 \)
، 
\(c_y=-3.8 \)
 ،
\(c_z=-6.1\)
؛ 
\(d_x=4.4\)
،
\(d_y=-2.0\)
،
\(d_z=3.3\)
ہیں۔ ان ہٹاؤ کہ مجموعہ
 \(\vec{r}\) 
کے 
(الف) \(x\)،
(ب) \(y\)،
اور 
(ج) \(z\)
اجزا تلاش کریں
\انتہا{سوال}

\ابتدا{سوال} 
سوال 11 
(الف) اگر 
\(\vec{a}=(\SI{4}{\meter})\hat{i}+(\SI{3}{\meter})\hat{j}\)
اور
\(\vec{b}=(\SI{-13}{\meter})\hat{i}+(\SI{7}{\meter})\hat{j}\)
ہوں تب اکائی سمتیا علامتیت میں مجموعہ 
\(a+b\)
کیا ہوگا؟ اس مجموعے کا 
(ب) قدر اور 
(ج) رخ کیا ہوگا؟
\انتہا{سوال}
%------------------------
\ابتدا{سوال}
سوال 12 
ایک گاڑی کو مشرک کی طرف 
\(\SI{50}{\kilo\meter}\) 
، اس کے بعد شمال کی طرف 
 \(\SI{30}{\kilo\meter}\)
اور اخر میں شمال سے مشرک جانب 
 \(\SI{30}{\degree}\)
کے رخ
\(\SI{25}{\kilo\meter}\)
چلایا جاتا ہے۔ اس کا سمتی نقشہ بنائیں۔ ابتدائی نقطہ سے گاڑی کی کل ہٹاؤ کا (الف) قدر اور 
(ب) زاویہ تلاش۔ 
\انتہا{سوال}

\ابتدا{سوال}
سوال 13 
ایک شخص اپنے موجودہ مقام سے 
 \(\SI{3.4}{\kilo\meter}\)
دور شمال سے مشرک جانب 
\(\SI{35}{\degree}\)
کے رخ مقام پر پہنچنا چاہتا ہے۔ تاہم اس کو مجبوراً ایسی گلیوں سے گزرنا ہوگا جو مشرق سے مغرب یا شمال سے جنوب ہیں۔ یہ شخص کتنا کم سے کم فاصلہ طے کر کے اس مقام تک پہنچ سکتا ہے؟ 
\انتہا{سوال}

\ابتدا{سوال}
سوال 14 
اپ ہموار صحرا میں
\(xy\)
محدتی نظام کے ممتاز سے اغاز کرتے ہوئے 
\(xy\)
محدد (مائنس 14 میٹر کم 30 میٹر)
(\(\SI{-140}{\meter}\), \(\SI{30}{\meter}\))
کہ مقام کو پہنچنا چاہتے ہیں۔ اپ کو صرف چار مرتبہ سید میں چلنے کی اجازت ہے۔ اپ کی حرکت کے 
\(x\)
اور 
\(y\) 
اجزاء میٹروں میں بالترتیب درج ذیل ہیں: 
 (\(60\) اور \(20\))
، اس کے بعد 
 (\(-70\) اور \(bx\))
، اس کے بعد 
 (\(c\) اور \(20-\))
، اور اخر میں 
(\(60-\) اور \(70-\))۔ 
بتائیں 
(الف) جوز 
\(bx\)
اور 
(ب) جوز 
\(cy\) 
کیا ہوں گے؟ مجموعی ہٹاؤ کا 
(ج) قدر 
اور 
(ح) مثبت 
\(x\) 
محور کے لحاظ سے زاویہ کیا ہوگا؟ 
\انتہا{سوال}

\ابتدا{سوال}
سوال 15 
شکل 
\(3.28\)
میں دکھائے گئے سمتیات 
\(a\) 
اور 
\(b\)
دونوں کے قدر
\انتہا{سوال}
