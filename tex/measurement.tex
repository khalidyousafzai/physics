%p7 to p11 editing finished. figures and external references pending
%Q6, Q7, Q8, Q10   below is different from that of the book. its answer must be changed accordingly.
%question 33, 34, 35, 36,38,41   is different from the book. answer must be handled correctly
%must work on Q6 and Q7  to get a better table, say havine سیر، کلو، دڑے

\باب{پیمائش}
\حصہء{طبیعیات کیا ہے؟}
سائنس و  انجینئری پیمائش اور موازنے  پر مبنی ہے۔ چیزوں کی پیمائش اور موازنے کے  لئے قواعد کی ضرورت پیش آتی ہے؛  پیمائش اور موازنہ کے  بُعد تعین کرنے کے لئے تجربات کا سہارا لینا  ہو گا۔ طبیعیات اور انجینئری کا ایک مقصد ان تجربات کی بناوٹ اور تجربہ کرنا ہے۔

\حصہء{چیزوں کی پیمائش}
طبیعیات میں ملوث مقداروں کی پیمائش کی طریقے جان کر ہم طبیعیات دریافت کرتے ہیں۔ ان مقداروں میں لمبائی، وقت، کمیت، درجہ حرارت، دباو، اور برقی رو شامل ہیں۔

ہم ہر طبیعی  مقدار کا موازنہ ایک  \اصطلاح{معیار }\فرہنگ{معیار}\حاشیہب{standard}\فرہنگ{standard} کے ساتھ کرکے  طبیعی مقدار  کو اس کی اکائیوں  میں ناپتے ہیں۔ اس مقدار کی ناپ کو ایک منفرد نام دیا جاتا ہے جسے  \اصطلاح{ اکائی }\فرہنگ{اکائی}\حاشیہب{unit}\فرہنگ{unit} کہتے ہیں۔ مثلاً ، لمبائی کی پیمائش   میٹر  \عددی{(\si{\meter\relax})}میں  کی جاتی ہے۔ معیار سے مراد،  مقدار کی ٹھیک ایک اکائی ہے۔ جیسا آپ دیکھیں گے لمبائی کا معیار،  جو ٹھیک ایک میٹر کے برابر ہے، اُس فاصلہ کو کہتے ہیں جو خلاء میں  ،  ایک مخصوص دورانیہ میں  ، شعاع طے کرتی  ہے۔ ہم اکائی اور اس کے معیار کی تعریف جیسا چاہیں کر سکتے ہیں۔ تاہم، ضروری ہے کہ دنیا کے باقی سائنسدان بھی اس تعریف کو معنی خیز اور قابل استعمال  مانیں۔

ایک معیار ، مثلاً لمبائی کا معیار ،  طے کرنے کے بعد ہمیں وہ طریقہ کار وضع  کرنا  ہو گی جس سے   ہر  لمبائی ، چاہے وہ ہائیڈروجن جوہر کا رداس ہو یا دور  کسی ستارے تک فاصلہ ، اس معیار کی صورت میں ظاہر کی جا سکے۔ ایسی ایک ترکیب فیتے کا استعمال ہے ؛  لمبائی کے معیار کو  فیتہ تخمیناً ظاہر کرتا ہے۔ بہرحال، بہت سے  موازنوں میں بلا واسطہ طریقے استعمال کیے  جائیں گے۔ مثلاً ،  جوہر کا رداس یا قریبی ستارے تک فاصلہ فیتہ استعمال کرکے نہیں ناپا جا سکتا۔

\موٹا{اساسی مقادیر}
 طبیعی مقادیر کی تعداد اتنی زیادہ ہے  کہ انہیں منظم کرنا  ایک  مسئلہ ہے۔  خوش قسمتی سے تمام  مقادیر غیر تابع نہیں ہیں؛ مثلاً ، رفتار درحقیقت لمبائی اور وقت کی  تناسب کو کہتے ہیں۔  بین الاقوامی متفقہ  معاہدے کے تحت چند طبیعی مقادیر،  مثلاً ،  لمبائی ،  کمیت، اور وقت  کو\اصطلاح{ اساسی مقادیر }\فرہنگ{اساسی مقادیر}\حاشیہب{base quantities}\فرہنگ{base quantities} منتخب کرکے صرف انہی کو معیار مختص کیے گئے۔   باقی  طبیعی مقادیر ان \قول{ اساسی مقادیر } اور  انہیں کے  معیار (جنہیں \اصطلاح{اساسی معیار }\فرہنگ{اساسی معیار}\حاشیہب{base standards}\فرہنگ{base standards}  کہتے ہیں) کی صورت میں ناپے جاتے ہیں۔ مثلاً،  اساسی مقادیر  لمبائی اور وقت اور انکے اساسی معیار کی  شکل میں \قول{رفتار } تعین کیا جاتا ہے۔ 

اساسی معیار  کا قابل رسائی اور    غیر متغیر  ہونا  لازمی ہے۔ اگر ہم بازو کی لمبائی کو معیار لمبائی تسلیم  کریں تب یہ قابل رسائی ضرور ہو گی، البتہ ہر شخص کے لئے یہ لمبائی مختلف ہوگی لہٰذا یہ غیر  متغیر نہیں ہے۔ سائنس و  انجینئری میں زیادہ سے زیادہ درستگی مطلوب ہونے کی  پیش نظر ہم اساسی  معیار کی   غیر متغیریت  پر  خصوصی توجہ  دیتے  ہیں۔  اس کے بعد اساسی معیار کی بہتر سے بہتر نقل بنا کر    ان لوگوں کو  فراہم  کرتے ہیں جنہیں ضرورت ہو۔


\حصہء{اکائیوں کا بین الاقوامی نظام} 
\سن {1971 } میں ناپ و  تول  کے عمومی اجلاس میں سات مقادیر  کو بطور اساسی مقدار منتخب کرکے بین الاقوامی نظام اکائی کے اساس چنے گئے۔ بین الاقوامی نظام اکائی  کو مختصراً  \قول{\عددی{\textup{SI}} نظام}  کہتے ہیں۔جدول  \حوالہ{جدول_پیمائش_اساسی_اکائیاں} میں تین اساسی مقدار لمبائی، کمیت ، اور وقت دکھائے گئے ہیں۔
\begin{table}[h!]
\caption{بین الاقوامی نظام اکائی کی تین اساسی مقادیر کی اکائیاں}
\label{جدول_پیمائش_اساسی_اکائیاں}
\centering
\begin{tabular}{r r c} 
\toprule
مقدار & اکائی کا نام & اکائی کی علامت\\ 
\midrule
لمبائی & میٹر & \si{\meter} \\
کمیت & کلوگرام &\si{\kilogram} \\
وقت & سیکنڈ & \si{\second} \\
\bottomrule
\end{tabular}
\end{table}
ان اکائیوں کی تعریف انسانی جسامت مد نظر رکھتے ہوئے کی گئی۔

کئی \اصطلاح{ مشتق اکائیوں }\فرہنگ{اکائی!مشتق}\حاشیہب{derived units}\فرہنگ{unit!derived}کی تعریف ان اساسی اکائیوں کی صورت میں کی جاتی ہے۔ مثلاً ، طاقت کی  \عددی{\textup{SI}}  اکائی ، جو \اصطلاح{واٹ }\فرہنگ{واٹ}\حاشیہب{watt}\فرہنگ{watt}   \عددی{(\si{\watt})} کہلاتی ہے، کی تعریف   کمیت، لمبائی ، اور وقت کی اساسی اکائیوں کی صورت میں کی جاتی ہے۔ یوں ، جیسا  آپ باب\حوالہء{ 7 } میں دیکھیں گے ، درج ذیل ہوگا:
\begin{align}
\text{\RL{1 واٹ}}=\SI{1}{\watt}=\SI{1}{\kilogram}\cdot \si{\meter\squared\per\second\cubed}
\end{align}
جہاں آخر میں  کلوگرام مربع میٹر فی مکعب سیکنڈ پڑھا جائے گا۔

بہت بڑی یا بہت چھوٹی مقادیر، جن سے ہمیں طبیعیات میں عموماً واسطہ ہو گا،  \اصطلاح{سائنسی ترقیم }\فرہنگ{ترقیم!سائنسی}\حاشیہب{scientific notation}\فرہنگ{notation!scientific} میں لکھی جاتی ہیں، جو دس کی طاقت استعمال کرتی ہے۔ یوں درج ذیل  ہو گا۔
\begin{align}
\SI{3560000000}{\meter} = \SI{3.56e9}{\meter} 
\end{align}
\begin{align}
\SI{0.000 000 492}{\second} = \SI{4.92e-7}{\second} 
\end{align}
کمپیوٹر  میں  سائنسی ترقیم  مزید مختصر لکھی جاتی ہے؛ مثلاً،   \عددی{3.56E9}   اور   \عددی{{4.92E-7}}  ، جہاں   \عددی{E}\قول{ دس کی طاقت  } ظاہر کرتا  ہے۔ کئی\اصطلاح{  حساب کار }\فرہنگ{حساب کار}\حاشیہب{calculator}\فرہنگ{calculator}  (کلکولیٹر)  مزید مختصر انداز استعمال کرتے ہوئے  \عددی{E} کو خالی جگہ سے ظاہر کیا جاتا ہے۔

ہم اپنی آسانی کے لئے بہت بڑی یا بہت چھوٹی پیمائش  جدول   \حوالہ{جدول_پیمائش_سابقے} میں  پیش سابقے استعمال کر کے  لکھتے ہیں۔
\begin{table}[h!]
\caption{بین الاقوامی نظام اکائی کے سابقے}
\label{جدول_پیمائش_سابقے}
\centering
\renewcommand{\arraystretch}{1.25}
\newcommand*{\isEmpty}{\relax}
\begin{tabular}{crl} 
\toprule
علامت & سابقہ &جزو ضربی\\
\midrule
$\si{\yotta\isEmpty}$ & یوٹا & $10^{24}$\\
$\si{\zetta\isEmpty}$ & زیٹا & $10^{21}$\\
$\si{\exa\isEmpty}$ & اکسا & $10^{18}$\\
$\si{\peta\isEmpty}$ & پٹا & $10^{15}$\\
$\si{\tera\isEmpty}$ & ٹیرا & $10^{12}$\\
$\si{\giga\isEmpty}$ &\موٹا{گیگا} & $10^{9}$\\
$\si{\mega\isEmpty}$ & \موٹا{میگا} & $10^{6}$\\
$\si{\kilo\isEmpty}$ & \موٹا{کلو} & $10^{3}$\\
$\si{\hecto\isEmpty}$ & ہکٹو & $10^{2}$\\
$\si{\deka\isEmpty}$ & ڈیکا & $10^{1}$\\
$\si{\deci\isEmpty}$ & ڈسی & $10^{-1}$\\
$\si{\centi\isEmpty}$ &\موٹا{سنٹی} & $10^{-2}$\\
$\si{\milli\isEmpty}$ & \موٹا{ملی} & $10^{-3}$\\
$\si{\micro\isEmpty}$ & \موٹا{مائیکرو} & $10^{-6}$\\
$\si{\nano\isEmpty}$ & \موٹا{نینو} & $10^{-9}$\\
$\si{\pico\isEmpty}$ & \موٹا{پکو}& $10^{-12}$\\
$\si{\femto\isEmpty}$ &فیمٹو & $10^{-15}$\\
$\si{\atto\isEmpty}$ & اٹو & $10^{-18}$\\
$\si{\zepto\isEmpty}$ & زپٹو & $10^{-21}$\\
$\si{\yocto\isEmpty}$ & یکٹو & $10^{-24}$\\
\bottomrule
\end{tabular}
\end{table}

جیسا آپ دیکھ سکتے ہیں ہر  سابقہ دس کی  ایک مخصوص طاقت ظاہر کرتا ہے، جو بطور جزو ضربی استعمال کیا جاتا ہے۔ بین الاقوامی نظام اکائی کے ساتھ   سابقہ منسلک کرنے سے مراد اس اکائی کو مطابقتی  جزو ضربی سے ضرب دینا ہے۔ یوں ہم کسی ایک مخصوص برقی طاقت کو
\begin{align}
 \text{\RL{واٹ}}\, 1.27\times 10^9 = \text{\RL{گیگا واٹ}} \, 1.27  = \SI{1.27}{\giga\watt}
\end{align}
یا کسی مخصوص وقتی دورانیہ کو  درج ذیل لکھ سکتے ہیں۔
\begin{align}
\text{\RL{سیکنڈ}} \, 2.35\times 10^{-9}  = \text{\RL{نینو سیکنڈ}} \, 2.35 
= \SI{2.35}{\nano\second}.
\end{align}
چند سابقے ، جو ملی  لٹر ، سنٹی میٹر، کلوگرام یا میگا بائٹ میں استعمال ہوتے ہیں ، سے آپ ضرور واقف ہوں گے۔

\حصہء{اکائی کی تبدیلی}
بعض اوقات طبیعی  مقداروں کی اکائی تبدیل کرنے کی ضرورت پیش آتی ہے۔ ہم اصل پیمائش کو\قول{  تبادلی جزو }، جو  ایک \عددی{(1)} کے برابر اکائیوں کی نسبت ہو گی، سے ضرب دیتے ہیں۔ مثلاً ، ایک منٹ اور ساٹھ سیکنڈ مماثل دورانیہ ظاہر کرتے ہیں ، لہٰذا درج ذیل ہو گا۔
\begin{align*}
\frac{\SI{1}{\minute}}{\SI{60}{\second}} = 1
\end{align*}
یا
\begin{align*}
\frac{\SI{60}{\second}}{\SI{1}{\minute}} = 1
\end{align*}

یوں  \عددی{(\SI{1}{\minute})\!/\!(\SI{60}{\second})}   یا \ \عددی{(\SI{60}{\second})\!/\!(\SI{1}{\minute})}  تناسب  بطور\اصطلاح{  تبادلی جزو }\فرہنگ{تبادلی جزو}\حاشیہب{conversion factor}\فرہنگ{conversion factor}استعمال کیا جا سکتا ہے۔ ہم ہرگز \(\frac{1}{60}=1\) یا \(60 = 1\) نہیں لکھ سکتے؛  ہر عدد اور اسکی اکائی کو اکٹھا رکھنا ہو گا۔

ایک \عددی{(1)}  سے ضرب دینے سے مقدار کی قیمت تبدیل نہیں ہوتی لہٰذا ہم جب  چاہیں تبادلی جزو  استعمال کر سکتے ہیں۔ ایسا کرتے ہوئے ہم غیر ضروری اکائیوں کو منسوخ کر سکتے ہیں۔ مثال کے طور پر دو منٹ  کو سیکنڈ  میں تبدیل کرتے ہوئے درج ذیل لکھا جائے گا۔
\begin{align}
\SI{2}{\minute} = (\SI{2}{\minute})(1) = (\SI{2}{\bcancel{\mathrm{\minute}}})(\frac{\SI{60}{\second}}{\SI{1}{\bcancel{\mathrm{\minute}}}}) = \SI{120}{\second}
\end{align}
اگر تبادلہ جزو ضرب متعارف  کرنے سے غیر  ضروری  اکائیاں ایک دوسرے کے ساتھ منسوخ نہ ہوتی ہوں تب جزو ضربی کو اُلٹا کر دوبارہ کوشش کریں۔ اکائیوں کی تبادلہ میں اکائیوں پر متغیرات اور اعداد کے الجبرائی قواعد لاگو ہوں گے۔

\حصہء{لمبائی}
\سن {1792 } میں فرانس کی نوزائیدہ جمہوریہ نے ناپ اور تول کا ایک نیا نظام قائم کیا۔ میٹر اس کا سنگ بنیادی تھا،  جو قطب شمال سے خط استوا تک  فاصلے کا کڑوڑواں حصہ لیا گیا ۔ بعد میں عملی وجوہات کی  بنا    پر ا س زمینی معیار کو ترک کرتے ہوئے ،  \اصطلاح{پلاٹینم و    اریڈیم }\فرہنگ{پلاٹینم و اریڈیم}\حاشیہب{platinum-iridium}\فرہنگ{platinum-iridium} کی  ایک سلاخ  پر لگائے گئے دو باریک لکیروں کے بیچ فاصلہ  \اصطلاح{ میٹر }\فرہنگ{میٹر}\حاشیہب{meter}  قرار  پایا؛ یہ \اصطلاح{معیاری  میٹر  سلاخ }\فرہنگ{میٹر!معیاری سلاخ}\حاشیہب{standard meter bar}\فرہنگ{meter!standard bar} پیرس شہر کے قریب ناپ  و  تول کے   بین الاقوامی محکمہ میں رکھا گیا ہے۔  اس سلاخ کی  بہترین نقل،  دنیا کی معیار ساز تجربہ گاہوں کو   (بطور ثانوی معیار)  فراہم کی گئی۔ \اصطلاح{ثانوی معیار }\فرہنگ{معیار!ثانوی}\حاشیہب{secondary standards}\فرہنگ{standard!secondary} سے ، مزید  قابل رسائی معیار تیار کیے گئے    ، حتٰی کہ آخر کار ہر پیمائشی آلہ  معیاری میٹر سلاخ  پر مبنی تھا۔ 

کچھ عرصہ   بعد ، سلاخ  پر دو باریک لکیروں کے بیچ فاصلہ  کے معیاری میٹر سے بہتر معیار کی ضرورت پیش آئی۔ \سن {1960 } میں شعاع کے  طول موج پر مبنی میٹر کے   معیار پر اتفاق   کیا گیا۔ یہ معیار کرپٹن  \عددی{86}(  جو \اصطلاح{ کرپٹن }\فرہنگ{کرپٹن}\حاشیہب{krypton}\فرہنگ{krypton}کا ایک مخصوص ہم جا ہے )کے جوہروں سے خارج ایک مخصوص نارنجی  سرخ  شعاع کے \عددی{ 1650763.73}  طول موج کا فاصلہ  ٹہرایا گیا۔ یہ شعاع دنیا میں کہیں بھی  \اصطلاح{گیس خروج  نلی }\فرہنگ{گیس خروج نلی}\حاشیہب{gas discharge tube}\فرہنگ{gas discharge tube} سے حاصل کی جا سکتی ہے۔ نئے معیار  کو پرانے  معیار ( میٹر سلاخ  ) کے قریب سے قریب  رکھنے کی غرض سے تول موج کی  (مذکورہ بالا عجیب) تعداد منتخب کی گئی۔

کچھ عرصہ   تک یہ معیار سائنسی دنیا کی ضروریات  پوری  کر  پایا، تاہم سائنس کی دنیا    بہت جلد اتنی آگے بڑھ چکی کہ کرپٹن \عددی{86} کے طول موج پر مبنی معیار سائنسی ضروریات پوری کرنے   کے قابل نہیں رہا۔  آخر کار  \سن{1983} میں  ایک نڈر فیصلہ کیا گیا، اور میٹر وہ فاصلہ قرار پایا جو شعاع ایک مخصوص دورانیہ میں طے کرتی ہے۔ ناپ  و  تول کے سترھویں \عددی{(17)}  عمومی اجلاس میں درج ذیل  طے پایا۔

\ابتدا{تعریف}
خلاء میں ایک سیکنڈ کے \(\tfrac{1}{299792458}\) حصے میں شعاع کا طے کردہ فاصلہ  \اصطلاح{میٹر   }\فرہنگ{میٹر!تعریف}\حاشیہب{meter}\فرہنگ{meter!definition}کہلائے گا۔
\انتہا{تعریف}

وقت کا  (مذکورہ بالا)  دورانیہ یوں منتخب کیا گیا کہ شعاع  کی رفتار  \عددی{c} ٹھیک درج ذیل ہو۔
\begin{align*}
c = \SI{299792458}{\meter\per\second}
\end{align*}


شعاع کی رفتار اٹل ہے۔ یوں   شعاع کی رفتار  سے  میٹر اخذ کرنا ایک بہتر قدم تھا۔

جدول \حوالہ{جدول_پیمائش_چند_تخمینی_فاصلے} میں  فاصلوں کی  وسیع سعت  پیش ہے،  جو   کہکشانی فاصلوں  سے لے کر انتہائی چھوٹی چیزوں کی لمبائیاں دیتا ہے۔
\begin{table}[h!]
\caption{چند تخمینی فاصلے}
\label{جدول_پیمائش_چند_تخمینی_فاصلے}
\centering
\begin{tabular}{r l} 
\toprule
  پیمائش& میٹر  میں لمبائی\\
\midrule
 اول ترین  پیدا کہکشاں تک فاصلہ& $2\times 10^{26}$ \\
  اندرومدا کہکشاں تک فاصلہ & $2\times 10^{22}$\\
  قریب ترین تارے  تک فاصلہ & $4\times 10^{16}$\\
  پلوٹو تک فاصلہ & $6\times 10^{12}$\\
 زمین کا رداس & $6\times 10^{6}$ \\
  بلند ترین پہاڑی کی  اونچائی & $9\times 10^{3}$\\
 صفحے کی موٹائی &  $1\times 10^{-4}$\\
  علامتی وائرس کی لمبائی & $1\times 10^{-8}$\\
  ہائیڈروجن  جوہر کا رداس & $5\times 10^{-11}$\\
  پروٹان کا رداس & $1\times 10^{-15}$\\
\bottomrule
\end{tabular}
\end{table}


\حصہء{با معنی اعداد اور اشاریہ کے مقام}
فرض کریں  آپ ایک مسئلے پر کام کر رہے ہیں جس میں ہر قیمت دو ہندسوں پر مشتمل  ہے۔ ان ہندسوں کو\اصطلاح{ با معنی ہندسے }\فرہنگ{با معنی ہندسے}\حاشیہب{significant figures}\فرہنگ{significant figures} کہتے ہیں۔ اپنا جواب پیش کرتے ہوئے آپ اتنے ہندسے  ہی استعمال کریں گے ۔ اگر مواد دو ہندسوں میں دیا گیا ہو تب  جواب بھی دو ہندسوں پر مشتمل  ہو گا۔  اگرچہ  آپ کا   حساب کار   نتائج  زیادہ ہندسوں میں پیش کرتا ہے،  یہ (اضافی)  ہندسے بے معنی ہیں۔

اس کتاب میں ، دیے گئے مواد میں کم سے کم با معنی ہندسوں کے برابر ، حساب کے اختتامی نتائج   پورمپور کر کے پیش کیے جائیں گے۔ (ہاں ، بعض اوقات ایک اضافی با معنی  ہندسہ بھی رکھا جائے گا۔ ) اگر ضائع کیے جانے والے ہندسوں میں بایاں ترین ہندسہ  \عددی{5} کے برابر یا اس سے بڑا ہو تب آخری رہنے دیا گیا ہندسے کو  \قول{اوپر پورمپور } کیا جاتا ہے؛ دیگر صورت اس کو جوں کا توں  رکھا جاتا ہے۔ مثال کے طور پر \عددی{11.3516}  کو تین با معنی ہندسوں  میں  پورمپور کرکے  \عددی{11.4} جبکہ  \عددی{11.3279}  کو تین با معنی ہندسوں میں  پورمپور کرتے ہوئے  \عددی{11.3}  لکھا جائے گا۔( اس کتاب میں نتائج پیش کرتے ہوئے پورمپور کیے جانے کے باوجود \عددی{\approx} کی بجائے   عموماً \عددی{=} علامت استعمال کی جائے گی۔)

عدد  \عددی{3.15}  یا \عددی{3.15\times 10^3}  میں با معنی ہندسوں کی تعداد صاف ظاہر ہے؛ عدد  \عددی{3000}  میں با معنی ہندسے کتنے  ہیں؟ کیا یہ صرف ایک با معنی ہندسہ \عددی{3\times 10^3}   تک  معلوم ہے ،  یا یہ چار با معنی ہندسوں \عددی{3.000\times 10^3} تک معلوم ہے؟  اس کتاب میں \عددی{ 3000} کی  طرز  پر اعداد میں تمام صفروں کو با معنی  تصور کیا جائے گا۔

با معنی ہندسوں  اور اشاریہ کے  مقام دو  مختلف  باتیں ہیں۔ درج ذیل فاصلوں  \عددی{\SI{35.6}{\milli\meter}}، \عددی{\SI{3.56}{\meter}}، اور \عددی{\SI{0.00356}{\meter}} پر غور کریں۔  تمام میں تین با معنی ہندسے  ہیں، تاہم  ان میں اشاریہ کے مقام  بالترتیب ایک، دو ، اور پانچ  ہیں۔

%-----------------------------
\ابتدا{مثال}
 \موٹا{ دھاگے کا گیند؛  قدر  کے رتبہ  کی تخمین۔}
 
دنیا میں دھاگے کے سب سے بڑے  گیند کا رداس  \عددی{\SI{2}{\meter}} ہے۔  اس گیند میں  دھاگے کی کل لمبائی  \عددی{L} کتنی ہوگی؟اگرچہ ہم گیند سے دھاگہ کھول کر لمبائی  \عددی{L}  ناپ سکتے ہیں ، تاہم ہم ایسا نہیں کرنا چاہتے۔ ہم حساب کے ذریعہ اس کی لمبائی کا تخمینہ لگانا چاہتے ہیں۔ ہمیں فقط   قدر  کا قریبی  رتبہ درکار ہے۔

\موٹا{حساب}

ہم فرض کرتے ہیں گیند کروی ہے ؛ اس کا رداس  \عددی{R=\SI{2}{\meter}} ہے۔ دھاگہ لپیٹتے  ہوئے   دھاگے کے مختلف حصوں کے بیچ خالی جگہ ضرور ہو گی جس کے بارے میں جاننا نا ممکن بات ہے۔ ان خالی جگہوں کو مد نظر رکھتے ہوئے ہم دھاگے کا عمودی تراش ذرا زیادہ تصور کرتے ہیں۔ ہم کہتے ہیں کہ دھاگے کا عمودی تراش  (گول کی بجائے) چوکور ہے جس کا  ضلع \عددی{d=\SI{4}{\milli\meter}} ہے۔ یوں اس کا رقبہ عمودی تراش \عددی{d^2} ،  لمبائی  \عددی{L}  ، اور کل حجم درج ذیل ہوگا:
\begin{align*}
V=(\text{\RL{رقبہ عمودی تراش}})(\text{\RL{لمبائی}})=d^{2}L
\end{align*}
 
جو گیند کے حجم  \عددی{\tfrac{4}{3}\pi R^3} کے برابر ہوگا ؛ \عددی{\pi}  کو تخمیناً \عددی{3} لیتے ہوئے  یہ حجم \عددی{4R^{3}} لکھا جا سکتا ہے۔ یوں درج ذیل ہوگا
\begin{align*}
d^{2}L=4R^{3}
\end{align*}

جس سے درج ذیل حاصل ہو گا۔
\begin{align*}
L&=\frac{4R^{3}}{d^{2}}\\
&=\frac{4(\SI{2}{\meter})^3}{(\SI{4e-3}{\meter})^2}\\
&=\SI{2e6}{\meter}\approx \SI{e6}{\meter}\approx \SI{e3}{\kilo\meter}
\end{align*}

(اتنے  سادہ حساب کے لئے حساب کار ر کی ضرورت پیش نہیں ہونی  چاہئے۔)    قدر  کے  قریبی رتبہ تک  گیند میں تقریباً \عددی{\SI{1000}{\kilo\meter}} دھاگہ ہے۔
\انتہا{مثال}


\حصہ{وقت}
وقت کے دو پہلو ہیں۔ روز مرہ زندگی میں ہم  کام   کاج ترتیب سے رکھنے کی غرض سے وقت جاننا چاہتے ہیں۔ سائنس کی دنیا میں ہم عموماً جاننا چاہتے ہیں کہ ایک واقعہ کتنی  دیر وقوع پذیر ہوا۔ یوں وقت کے کسی بھی  معیار کو دو سوالات کا جواب دینا ہوگا: کب ہوا؟ اس کا دورانیہ کتنا تھا؟ جدول \حوالہ{جدول_پیمائش_تخمینی_دورانیے}  میں چند وقتی وقفے پیش ہیں، جہاں \اصطلاح{ پلانک وقت }\فرہنگ{پلانک!وقت}\حاشیہب{plank time}\فرہنگ{plank!time} سے مراد   \اصطلاح{ابتدائی دھماکے }\فرہنگ{ابتدائی دھماکہ}\حاشیہب{big bang}\فرہنگ{big bang} کے  بعد  وہ   اول ترین وقت ہے جب طبیعیات کے قواعد (جس طرح انہیں ہم اس وقت  جانتے ہیں)  قابل اطلاق ہوں گے۔

\begin{table}[h!]
\caption{چند تخمینی دورانیے}
\label{جدول_پیمائش_تخمینی_دورانیے}
\centering
\begin{tabular}{r l}
\toprule
پیمائش & سیکنڈ میں دورانیہ \\
\midrule
پروٹان کا عرصہ  حیات (محض  اندازہ)  & $3\times 10^{40}$\\
کائنات کی عمر & $5\times 10^{17}$\\
ہرم  خوفو      کی عمر & $1\times 10^{11}$\\
انسانی زندگی   (متوقع) & $2\times 10^{9}$ \\
ایک دن & $9\times 10^{4}$ \\
انسانی دل کی دھڑکنوں کے بیچ وقفہ & $8\times 10^{-1}$\\
میون کا عرصہ حیات & $2\times 10^{-6}$\\
تجربہ گاہ میں  مختصر ترین   شعاع کا دورانیہ & $1\times 10^{-16}$\\
  غیر مستحکم ترین  ذرے کا عرصہ حیات & $1\times 10^{-23}$\\
پلانک وقفہ  & $1\times 10^{-43}$\\
\bottomrule
\end{tabular}
\end{table}

 وہ  مظہر جو اپنے آپ کو دہراتا ہو وقت کا  معیار مقرر کیا جا سکتا ہے۔  محور کے گرد زمین کا ایک چکر ، جو دن کی لمبائی تعین کرتا ہے ، صدیوں تک بطور وقت کا  معیار  استعمال کیا گیا۔  \اصطلاح{سنگ مردہ }\فرہنگ{سنگ مردہ}\حاشیہب{quartz}\فرہنگ{quartz}  (کوارٹز) گھڑی،  جس میں  ایک سنگ مردہ  چھلے  کو مسلسل ارتعاش پذیر رکھا جاتا ہے ، کی پیمانہ بندی فلکیاتی مشاہدات کے ذریعہ، زمین کے گھومنے کے ساتھ  کر کے،  تجربہ گاہ میں وقتی وقفوں کی پیمائش کے لیے  استعمال کیا جا سکتا ہے۔ تاہم جدید سائنس و انجینئری کو  درکار درستگی ایسی پیمانہ بندی  سے ممکن نہیں۔

بہتر معیار وقت کی ضرورت کے درپیش \اصطلاح{ جوہری گھڑیاں  }\فرہنگ{گھڑی!جوہری}\حاشیہب{atomic clocks}\فرہنگ{clock!atomic} تیار کی گئیں۔\سن{1967 } میں ناپ و تول کے تیرھویں     عمومی اجلاس میں \اصطلاح{سیزیم گھڑی }\فرہنگ{گھڑی!سیزیم}\حاشیہب{cesium clock}\فرہنگ{clock!cesium} پر مبنی معیاری سیکنڈ پر اتفاق کیا گیا۔


\ابتدا{تعریف}
سیزیم \عددی{133}  جوہر سے خارج ایک مخصوص طول موج کی شعاع کے  \عددی{\num{9192631770}} ارتعاش کو درکار وقت ایک\اصطلاح{ سیکنڈ }\فرہنگ{سیکنڈ!تعریف}\حاشیہب{second}\فرہنگ{second!definition}  ٹہرایا گیا۔
\انتہا{تعریف}

جوہری گھڑیاں  انتہائی درست وقت بتاتی ہیں۔دو سیزیم گھڑیوں  میں ایک سیکنڈ فرق چھ ہزار سال چلنے کے بعد  پیدا ہو گا۔ اس وقت تیار کی جانے والی گھڑیوں کی درستگی  \عددی{10^{18}}  میں ایک حصے کے برابر ہے ، یعنی \عددی{10^{18}}  سیکنڈ (جو تقریباً \عددی{3\times 10^{10}} سال  کے برابر ہے)  میں صرف ایک سیکنڈ کا فرق ہو سکتا ہے۔

\حصہ{کمیت}  
\جزوحصہء{معیاری کلوگرام} 
فرانس کے شہر پیرس کے قریب ناپ و تول  کے بین الاقوامی محکمہ  میں رکھے گئے پلاٹینم و  اریڈیم کا ایک سلنڈر ، بین الاقوامی معاہدہ کے تحت ، ایک کلوگرام کمیت  ٹہرایا گیا۔ اس کی بہتر سے بہتر  نقل دنیا کے  بیشتر معیار ساز تجربہ گاہوں کو فراہم کی گئی  جن کو استعمال کرتے ہوئے ترازو کی مدد سے کسی بھی جسم کی کمیت ناپی جا سکتی ہے۔ جدول  \حوالہ{جدول_پیمائش_کمیت} میں   قدر کے \عددی{83}  رتبوں پر پھیلی کمیتوں کو  کلوگرام کی صورت میں  پیش کیا گیا ہے۔

\begin{table}[h!]
\caption{چند تخمینی کمیت}
\label{جدول_پیمائش_کمیت}
\centering
\begin{tabular}{rl}
\toprule
چیز & کلوگرام میں کمیت\\
\midrule
معروف کائنات & $1\times 10^{53}$\\
ہماری کہکشاں & $2\times 10^{41}$\\
سورج & $2\times 10^{30}$\\
چاند & $7\times 10^{22}$\\
سیارچہ ایراس& $5\times 10^{15}$\\
چھوٹا  پہاڑ & $1\times 10^{12}$\\
سمندری جہاز& $7\times10^{7}$\\
ہاتھی & $5\times10^{3}$\\
انگور & $3\times10^{-3}$\\
دھول کی ذرہ & $7\times10^{-10}$\\
پینسلین سالمہ & $5\times10^{-17}$\\
یورینیم جوہر & $4\times10^{-25}$\\
 پروٹان & $2\times10^{-27}$\\
 الیکٹران & $9\times10^{-31}$\\
 \bottomrule
\end{tabular}
\end{table}

\جزوحصہء{دوسرا  معیار کمیت} 
جوہروں کی کمیت کا موازنہ معیاری کلوگرام کی بجائے ، زیادہ درستگی کے ساتھ ،  دیگر جوہروں کے ساتھ کیا جا سکتا ہے۔ اسی کی بنا ، ہم دوم معیار کمیت بھی رکھتے ہیں۔  کاربن \عددی{12} جوہر کو بین الاقوامی معاہدہ کے تحت  \عددی{12} \اصطلاح{جوہری کمیتی اکائیوں}\فرہنگ{جوہری کمیتی اکائی}\حاشیہب{atomic mass unit}\فرہنگ{atomic mass unit} کی کمیت مختص کی گئی۔ ان دو اکائیوں کے بیچ رشتہ درج ذیل ہے
\begin{align}
\SI{1}{\atomicmassunit} = \SI{1.66053886e-27}{\kilogram}
\end{align}
جہاں آخری دو ہندسوں میں عدم یقینیت \عددی{\pm10}  ہے۔ سائنس دان کافی درستگی کے ساتھ تجربہ کے ذریعہ کسی بھی جوہر کی کمیت کاربن  \عددی{12} کی کمیت کے لحاظ سے تعین کر سکتے ہیں۔ اس وقت،  کمیت کی روز مرہ زندگی میں مستعمل   اکائیاں، مثلاً کلوگرام ، استعمال کرتے ہوئے ہم اتنی درستگی حاصل کرنے سے قاصر ہیں۔

\جزوحصہ{کثافت} 
\اصطلاح{کثافت  }\فرہنگ{کثافت}\حاشیہب{density}\فرہنگ{density} \عددی{\rho} سے مراد  اکائی حجم میں کمیت ہے۔
\begin{align}
\rho = \frac{m}{V}
\end{align}
اس پر باب  \حوالہء{14} میں مزید تبصرہ کیا جائے گا۔ کثافت کو عام طور پر کلوگرام فی مربع میٹر یا گرام فی مربع سنٹی میٹر میں ناپا جاتا ہے۔ پانی کی کثافت ایک گرام فی مربع سنٹی میٹر یا ایک ہزار کلوگرام فی مربع میٹر ہے جس کو عموماً موازنہ کے لئے  استعمال کیا جاتا ہے۔ پانی کی کثافت کے لحاظ سے  تازہ برف  کی کثافت \عددی{\SI{10}{\percent}} اور پلاٹینم کی کثافت تقریباً  \عددی{21} گنّا  جبکہ لکڑی کی کثافت صرف  \عددی{\SI{64}{\percent}} ہے۔
%========================
%HR_p07_p11
%---------------------
%example 1.02
\ابتدا{مثال}\ترچھا{کثافت اور رقیق کاری }\\
ایسے زلزلہ کے دوران جس میں زمین کی \اصطلاح{رقیق کاری }\فرہنگ{رقیق کاری}\حاشیہب{liquefaction}\فرہنگ{liquefaction} ہو ،  بھاری جسم زمین میں دھنس سکتا ہے۔    رقت  کے دوران مٹی کے ذرے  نہایت  کم رگڑ محسوس کرتے ہوئے  ریلنا  شروع کرتے ہیں اور زمین  دلدل کی کیفیت اختیار کرتی   ہے ۔ ریتیلی زمین کی رقیق کاری کے ممکنات کی پیشنگوئی زمین کے نمونہ کی تناسب   خلا\عددی{e} کے روپ  میں کی جا سکتی ہے۔
\begin{align}\label{مساوات_پیمائش_خلا_تناسب}
e=\frac{V_{\text{\RL{خلا}}}}{V_{\text{\RL{دانے}}}} 
\end{align} 
 یہاں \عددی{V_{\text{\RL{دانے}}}}   نمونے میں ریت کے ذرات کا کل حجم جبکہ \عددی{V_{\text{\RL{خلا}}}}  ذروں  کے بیچ  خلا کا کل حجم ہے ۔اگر \عددی{e} فاصل قیمت \عددی{0.80} سے تجاوز کرتا ہو،  زلزلہ کے دوران رقیق کاری کا امکان ہوگا۔  مطابقتی  ریت  کی کثافت \عددی{\rho_{\text{\RL{ریت}}}}  کیا ہوگی؟  ٹھوس سلیکان ڈائی  اکسائیڈ  ، \عددی{(\ce{SiO2})} (  جو ریت کا بنیادی جزو ہے)   کی کثافت \عددی{\rho_{\ce{SiO2}}=\SI{ 2.6e3}{\kilo\gram\per\meter\cubed}} ہے ۔
 
\جزوحصہء{ کلیدی تصور}
 نمونے میں ریت کی کثافت\عددی{\rho_{\text{\RL{ریت}}}}  سے مراد اکائی حجم میں کمیت ہے ؛ جو ریت کے تمام ذروں کی کل کمیت \عددی{m_{\text{\RL{ریت}}}} اور نمونے کے کل حجم \عددی{V_{\text{\RL{کل}}}}  کا تناسب  :
\begin{align}\label{مساوات_پیمائش_کثافت_ریت}
\rho_{\text{\RL{ریت}}}=\frac{m_{\text{\RL{ریت}}}}{V_{\text{\RL{کل}}}}
\end{align}
ہے۔ 

\موٹا{ حساب: }\quad
نمونے کا کل حجم \عددی{V_{\text{\RL{کل}}}}  درج ذیل ہے
\begin{align*}
V_{\text{\RL{کل}}}=V_{\text{\RL{دانے}}}+V_{\text{\RL{خلا}}}
\end{align*}
 مساوات \حوالہ{مساوات_پیمائش_خلا_تناسب}  میں \عددی{V_{\text{\RL{خلا}}}} ڈال کر  \عددی{V_{\text{\RL{ریت}}}} کے لیے حل کر کے ذیل حاصل ہو گا ۔
\begin{align}\label{مساوات_پیمائش_حجم_دانے}
V_{\text{\RL{دانے}}}=\frac{V_{\text{\RL{کل}}}}{1+e}
\end{align}
مساوات  \حوالہء{1.8} کے تحت ریت کے ذرات کی کل کمیت \عددی{m_{\text{\RL{ریت}}}} سلیکان ڈائی اکسائیڈ کی کثافت ضرب ریت کے ذرات کا کل حجم :
\begin{align}
m_{\text{\RL{ریت}}}=\rho_{\ce{SiO2}}V_{\text{\RL{دانے}}}
\end{align}
 ہوگا ۔ اس کو مساوات  \حوالہ{مساوات_پیمائش_کثافت_ریت}  میں  ڈال کر کے مساوات \حوالہ{مساوات_پیمائش_حجم_دانے}  سے \عددی{V_{\text{\RL{ریت}}}} ڈال کر   ذیل حاصل ہوگا ۔
\begin{align}
\rho_{\text{\RL{ریت}}}=\frac{\rho_{\ce{SiO2}}}{V_{\text{\RL{کل}}}}\,\frac{V_{\text{\RL{کل}}}}{1+e}
\end{align}
فاصل قیمت \عددی{e = 0.80} اور    \عددی{\rho_{\ce{SiO2}}=\SI{2.600e3}{\kilo\gram\per\meter\cubed}} پر کر کے ہم دیکھتے ہیں کہ رقیق کاری اس صورت ہوگی جب ریت کی کثافت درج ذیل سے کم ہو۔
\begin{align*}
\rho_{\text{\RL{}}}=\frac{\SI{2.600e3}{\kilo\gram\per\meter\cubed}}{1.80}=\SI{1.4e3}{\kilo\gram\per\meter\cubed}
\end{align*} 
  رقیق   کاری میں عمارت کئی میٹر زمین میں دھنس سکتی  ہے۔
\انتہا{مثال}
%=================================
%HR_p08b
%edited. figures and external reference pending
\حصہء{نظر ثانی اور خلاصہ }
\جزوحصہء{طبیعیات میں پیمائش }
طبیعی  مقادیر  کی پیمائش پر طبیعیات مبنی ہے۔ کچھ طبیعی مقادیر (مثلاً لمبائی، وقت، اور کمیت) \اصطلاح{   اساسی  مقدار  }  منتخب کیے گئے؛ ہر ایک کی تعریف  \اصطلاح{ معیار } کے مطابق کی گئی  اور اس کو پیمائش کی  \اصطلاح{اکائی }(مثلاً 
\عددی{\si{\meter}} ،  \عددی{\si{\second}}، اور \عددی{\si{\kilo\gram}}) مختص کی گئی۔  دیگر طبیعی  مقادیر  کی تعریف ان اساسی  مقدار اور ان کے معیار اور اکائیوں کی صورت میں کی جاتی ہے ۔ 

\جزوحصہء{بین الاقوامی اکائی }
اس کتاب میں بین الاقوامی اکائی  \عددی{(SI)}  استعمال کی گئی ۔ جدول  \حوالہء{1.1} میں دکھائی گئی تین طبیعی  مقادیر ابتدائی بابوں میں استعمال کی جائیں گی۔ بین الاقوامی معاہدوں کے تحت  اساسی مقداروں کے معیار طے کیے گئے ، جو ہر ایک کے لیے قابل رسائی اور غیر تغیر ہیں۔ اساسی مقدار اور ان سے اخذ دیگر مقادیر  کی تمام طبیعی پیمائشیں انہی معیار کے تحت کی جاتی ہے۔ جدول  \حوالہء{1.2} میں پیش علامتیں اور سابقے استعمال کر کے پیمائشی ترقیم  کی سادہ صورت حاصل ہوتی  ہے۔ 

\جزوحصہء{اکائیوں کی باہم  تبدیلی }
اکائیوں کی تبدیلی  \ترچھا{زنجیری  طریقے } سے  جا سکتی ہے، جس میں اصل مواد   کو یک بعد دیگرے تبادلی ضربیوں  سے، جنہیں  ایک \عددی{(1)} کے روپ میں لکھا گیا ہو، ضرب دے  کر ، اکائیوں سے الجبرائی مقادیر  کی طرح  نپٹا جاتا ہے حتٰی کہ درکار اکائیاں رہ جائیں۔

\جزوحصہء{لمبائی }
وہ فاصلہ ہے جو انتہائی معین وقتی وقفے  کے دوران  بصری شعاع    طے کرتی ہے، میٹر کی تعریف  ہے۔

\جزوحصہ{وقت }
سیکنڈ کی تعریف  سیزیم \عددی{133}  جوہر سے خارج شعاع  کی صورت  میں کی جاتی ہے۔ معیار برقرار رکھتی  تجربہ  گاہوں میں موجود جوہری گھڑیوں کے  صحیح   وقتی اشارے پوری دنیا   میں نشر کیے جاتے ہیں۔

\جزوحصہء{ کمیت }
پیرس شہر کے قریب رکھے گئے  پلاٹینم  و  اریڈیم کمیتی معیار ،  کلوگرام  کی  تعریف  ہے۔ جوہری پیمانہ پر پیمائش کے لیے جوہری کمیتی  اکائی استعمال کی جاتی ہے جس کی تعریف کاربن  \عددی{12}  جوہر کی صورت میں کی جاتی ہے۔

\جزوحصہء{کثافت }
کسی بھی چیز کی کثافت  \عددی{\rho} سے مراد اکائی حجم میں اس کی کمیت ہے۔ 
\begin{align*}
\rho=\frac{m}{V} \tag{\arabicdigits{\setlatin1.8}}
\end{align*}
%======================================
\حصہء{سوالات }
\جزوحصہء{لمبائی اور دیگر اشیاء کی پیمائش}
%----------------------
%Q1 p8
\ابتدا{سوال}
زمین تخمیناً ایک کرہ  ہے جس کا رداس   \عددی{\SI{6.37e6}{\meter}} ہے۔اس کا   (ا)  محیط کلومیٹر میں،  (ب)  سطحی رقبہ مربع   مربع کلو میٹر میں،  اور  (ج) حجم  کعبی کلومیٹر  میں کتنا ہے؟ 
\انتہا{سوال}
%------------------------------
\ابتدا{سوال}
اشاعت کاری میں لمبائی کی مستمل اکائی\ترچھا{   نقطہ } کہلاتی  ہے  ،  جو  انچ کے   \عددی{\tfrac{1}{72}} حصے کے برابر  ہے۔\ترچھا{       مربع  نقطہ } کی صورت میں \عددی{0.1} مربع انچ  لکھیں۔ 
\انتہا{سوال}
%-------------------------
\ابتدا{سوال}
ایک مائیکرو میٹر  \عددی{(\SI{1}{\micro\meter})}  کو عموماً \ترچھا{  مائیکران } کہتے ہیں۔  (ا) کتنے مائیکران   \عددی{\SI{1}{\kilo\meter}} کے برابر ہیں؟ (ب) سنٹی میٹر کا کتنا حصہ \عددی{\SI{1}{\micro\meter}}  ہو گا؟  (ج) کتنے مائیکران ایک گز کے برابر ہوں گے؟
\انتہا{سوال}
%-----------------------------
%Q4
\ابتدا{سوال}
اس کتاب میں فاصلے \ترچھا{نقطہ} اور \ترچھا{  پیکا  } اکائی میں رکھے گئے ہیں: \عددی{12} نقطے \عددی{1} پیکا کے برابر ہے، اور  \عددی{6} پیکا    \عددی{1} انچ کے برابر۔  اگر کتاب میں ایک
 شکل   \عددی{\SI{0.80}{\centi\meter}} غلط رکھی گئی ہو، تب یہ  (ا) پیکا اکائیوں میں اور  (ب) نقطہ اکائیوں میں کتنی غلط رکھی گئی ہے؟ 
\انتہا{سوال}
%-------------------------------
%Q5 p08
\ابتدا{سوال}
ایک مقابلے میں گھوڑے \عددی{4.0}  فرلانگ  دوڑ لگا کر طے کرتے ہیں۔ اس فاصلے کو  (ا) عصا  اور  (ب) زنجیر  کی صورت میں لکھیں۔ (ایک فرلانگ \عددی{\SI{201.168}{\meter}} کے برابر ہے۔ ایک عصا \عددی{\SI{5.0292}{\meter}} اور ایک زنجیر \عددی{\SI{20.117}{\meter}} کے برابر ہے)
\انتہا{سوال}
%====================================
%HR-p09
%edited. figures and external references pending
%---------------------------------


\ابتدا{سوال}
 جدول   \حوالہ{جدول_پیمائش_کلو_تا_ملی_گرام}مکمل کریں۔
\begin{table}
\caption{ملی گرام، گرام، اور کلو گرام کی چند قیمتیں۔}
\label{جدول_پیمائش_کلو_تا_ملی_گرام}
\centering
\begin{tabular}{CCCC}
\toprule
&\si{\milli\gram}&\si{\gram}&\si{\kilo\gram}\\
\midrule
\SI{300}{\milli\gram}&&&\\
\SI{0.50}{\gram}&&&\\
\SI{0.02}{\kilo\gram}&&&\\
\bottomrule
\end{tabular}
\end{table}
(ا) جدول مکمل کریں۔  
(ب)  \عددی{\SI{55}{\milli\gram}}  کتنے   \عددی{\si{\kilo\gram}} کے برابر  ہو گا؟
(ج)   \عددی{\SI{12}{\centi\meter\cubed}}   حجم کتنے     لٹر کے برابر ہو گا؟
\انتہا{سوال} 
%--------------------------------
%Q7 p09
\ابتدا{سوال} \شناخت{سوال_پیمائش_ایکڑ_فٹ}
ماقوائی معمار  پانی کا  حجم  عموماً    ایکڑ فٹ  میں ناپتے ہیں، جس سے مراد ایک  ایکڑ   رقبے پر ایک فٹ  گہرا پانی ہے۔ ایک شہر جس کا رقبہ   \عددی{\SI{26}{\kilo\meter\squared}} ہے میں   \عددی{30} منٹ  کی بارش \عددی{2} انچ پانی برساتی ہے۔  شہر پر کتنا ایکڑ فٹ پانی برستا ہے؟ 
\انتہا{سوال} 
%------------------------------
\ابتدا{سوال} 
ایک سڑک   \عددی{32} میل  اور   \عددی{5} فرلانگ لمبی ہے۔ اس کی لمبائی   \عددی{\si{\kilo\meter}} میں کتنی ہوگی؟ 
\انتہا{سوال} 
%----------------------------------
%Q9
\ابتدا{سوال}
\اصطلاح{بہر  منجمد جنوبی   }\فرہنگ{بحر منجمد جنوبی}\حاشیہب{antarctica}\فرہنگ{antarctica}    تقریباً نیم دائری ہے (شکل   \عددی{1.5})  جس کا رداس   \عددی{\SI{2000}{\kilo\meter}} ہے۔ اس میں برف کی اوسط موٹائی   \عددی{\SI{3000}{\meter}}  ہے ۔ بحر منجمد جنوبی میں کتنے   \عددی{\si{\cubic\centi\meter}} برف پایا جاتا ہے؟ (زمین کی سطح  مستوی تصور کریں۔)
\انتہا{سوال}
%----------------------
%Module 1.2 p09
 \حصہء{وقت}
 %---------------------
 %Q10
\ابتدا{سوال} 
بہت وسیع ممالک مثلاً روس میں مختلف مقامات پر گڑیوں کا وقت ایک دوسرے سے مختلف ہوتا ہے۔ (ا)   خط تول بلد کے   کتنے درجے چلنے کے بعد ایک گھنٹے کا فرق پایا جائے گا؟ (اشارہ: زمین   \عددی{24}
گھنٹے میں   \عددی{\SI{360}{\degree}} گھومتی ہے۔)  ایک      خط تول بلد کتنے منٹ کے برابر ہوگا؟
\انتہا{سوال}
%-------------------------------
\ابتدا{سوال} 
  فرانسیسی  انقلاب کے بعد تقریباً   \عددی{10}  سال تک حکومت کوشش کرتی رہی کہ وقت کی پیمائش مضرب   \عددی{10} رکھی جائے؟ ایک ہفتہ میں   \عددی{10}  دن، ایک  دن میں   \عددی{10} گھنٹے، ایک گھنٹہ میں   \عددی{100} منٹ، اور ایک منٹ میں   \عددی{\SI{100}{\second}} رکھے گئے ۔

(ا) فرانسیسی اعشاری  ہفتہ اور معیاری ہفتہ کی نسبت ، اور   (ب) فرانسیسی اعشاری  سیکنڈ اور معیاری سیکنڈ کی نسبت کیا ہے؟ 
\انتہا{سوال} 
%----------------------------------------
%Q12
\ابتدا{سوال} 
دنیا کا تیز ترین بڑھتا پودا  \قول{ہسپرویوکا } کہلاتا  ہے جو   \عددی{14} دن میں   \عددی{\SI{3.7}{\meter}} بڑا۔  پودے کے بڑھنے کی شرح   \عددی{\si{\micro\meter\per\second}}
کتنی ہے؟ 
\انتہا{سوال} 
%-----------------------------------
\ابتدا{سوال} 
تین گھڑیاں الف، ب، اور پ مختلف رفتار سے چلتی ہیں اور بیک وقت صفر نہیں دیتی۔ شکل   \حوالہء{1.6} میں چار موقوں  پر ان کی  بیک  وقت پیمائش  دکھائی  گئی ہے۔ (مثال کے طور پر جس لمحہ  گھڑی ب   \عددی{\SI{25}{\second}}  دیتی ہے  ، گھڑی پ   \عددی{\SI{92}{\second}} دیتی ہے۔) اگر دو واقعات گھڑی الف  پر   \عددی{\SI{600}{\second}}  فاصلے  پر  واقع ہوں، یہ (الف) گھڑی    ب  پر اور 
(ب) گھڑی    ج پر کتنے  فاصلے پر واقع ہوں گے؟   (ج) جس لمحہ گھڑی  الف \عددی{\SI{400}{\second}} دیتی ہے اس لمحہ گھڑی ب کیا دے گی؟  (د) جس وقت گھڑی     الف   \عددی{\SI{15}{\second}}
دیتی ہے، اس وقت گھڑی  ب کیا دے گی؟ ( قبل از صفر   اوقات منفی  علامت  کے تصور کریں۔)
\انتہا{سوال}
%----------------------------
%Q14
\ابتدا{سوال}
ایک درس (جو  \عددی{50} منٹ کا ہے)  تخمیناً  ایک خورد صدی  کا ہوگا۔

(ا)ایک خورد صدی    دورانیہ   کتنے منٹ ہوگا؟   (ب) درج ذیل کلیہ   استعمال کرتے ہوئے تخمین میں فی صد فرق تلاش کریں۔ 
\begin{align*}
\text{\RL{}}=\big(\frac{\text{\RL{اصل}}-\text{\RL{تخمیناً}}}{\text{\RL{اصل}}}\big)\, 100
\end{align*}
\انتہا{سوال} 
%----------------------------
\ابتدا{سوال} 
دو ہفتوں کا وقت کتنے   \عددی{\si{\micro\second}} ہوگا؟ 
\انتہا{سوال} 
%-------------------------
%Q16, p09
\ابتدا{سوال} 
معیاری وقت کا دارومدار  جوہری گھڑیوں پر ہے۔ اس سے بہتر معیار سیکنڈ  \اصطلاح{نابض }\فرہنگ{نابض}\حاشیہب{pulsars}\فرہنگ{pulsar} پر مبنی ہو سکتا ہے، جو  گھومتے  \اصطلاح{ نیوٹران تارے }\فرہنگ{تارہ!نیوٹران}\حاشیہب{neutron star}\فرہنگ{star!neutron} (انتہائی ٹھوس تارے جن میں صرف نیوٹران پائے جاتے ہیں)  ہیں۔ ان میں سے کئی انتہائی زیادہ مستحکم شرح سے گھومتے ہیں، اور ہر چکر کے دوران ایک مرتبہ زمین پر  شعاع ڈالتے ہیں  (سمندر کے کنارے \اصطلاح{  منارہ   نور }\فرہنگ{منارہ  نور}\حاشیہب{light house}\فرہنگ{light house} کی طرح جو سمندری جہاز کو خطرے سے آگاہ کرتا ہے)۔ نابض   \عددی{PSR\,1937\,\,21}  ایک ایسا تارہ ہے جو ایک  چکر   \عددی{\num{1.55780644887275}\pm 3 \, \si{\milli\second}}     میں پورا کرتا ہے، جہاں آخر  میں   \عددی{\pm 3}  آخری ہندسہ  میں عدم یقینیت دیتا ہے (اس کا    ہرگز \عددی{\SI{\pm 3}{\milli\second}}  مطلب نہیں)۔ 

(ا) یہ نابض    \عددی{7.00}  دنوں میں کتنے چکر کاٹتا ہے؟  (ب) یہ نابض    \عددی{10} لاکھ مرتبہ ٹھیک کتنے وقت میں چکر کاٹتا  ہے ، اور  (ج) اس سے وابستہ عدم یقینیت کیا ہوگا؟ 
\انتہا{سوال}
%============================
%Q17 p10
\ابتدا{سوال}  تجربہ گاہ میں پانچ  گھڑیوں کی  جانچ پڑتال کی جا رہی ہے۔  ہفتے کے سات دن ٹھیک دوپہر  \عددی{12}  بجے  گھڑیوں کی ساعت   ذیل جدول میں  پیش ہے۔  بہترین  ساعت  رکھنے والی  گھڑی پہلے  رکھ کر گڑیوں کی درجہ بندی کریں۔ اپنے انتخاب کی وجہ پیش کریں۔

\begin{center}
\begin{tabular}{rCCCCCC}
\toprule
دن&A&B&C&D&E\\
\midrule
اتوار& 12:36:40 &11:59:59&15:50:45&12:03:59&12:03:59\\
پیر &12:36:56 &12:00:02 &15:51:43&12:02:52&12:02:49\\ 
منگل&12:37:12 &11:59:57 &15:52:41&12:01:45&12:01:54\\
بدھ&12:37:27&12:00:07 &15:53:39&12:00:38&12:01:52\\
جمعرات&12:37:44&12:00:02&15:54:37&11:59:31&12:01:32\\
جمع&12:37:59 &11:59:56 &15:55:35 &11:58:24&12:01:22\\
ہفتہ&12:38:14&12:00:03&15:56:33  &11:57:17 &12:01:12\\
\bottomrule
\end{tabular}
\end{center}

 \انتہا{سوال} 
%-------------------------------
%Q18 p10
\ابتدا{سوال} 
زمین کی گردش دن بدن آہستہ ہو رہی ہے، اور دن لمبا ہوتا جا رہا ہے۔ پہلی عیسوی صدی کا آخری دن   صدی کے پہلے دن سے  \عددی{\SI{1}{\milli\second}}لمبا  تھا۔  \عددی{20}صدیوں میں ایک دن کا دورانیہ کل  کتنا بڑا؟
 \انتہا{سوال} 
%---------------------------------
%Q19 p10
\ابتدا{سوال}  خط استوا پر  پرسکون سمندر کے کنارے ریت پر لیٹ کر آپ  غروب ہوتے سورج کا نظارہ کر رہے ہیں۔ جیسے ہی سورج کا بالا  سر سمندر کے پیچھے غروب ہوتا ہے،  آپ گھڑی میں وقت دیکھ کر قلم بند کرتے ہیں ۔ اس کے بعد  اتنی بلندی پر کھڑے ہو کر کہ آپ کی آنکھ \عددی{H=\SI{1.70}{\meter}}زیادہ اونچائی پر ہو،  آپ دوبارہ سورج کے بالا سر کو غروب ہوتے دیکھ کر وقت قلم بند کرتے ہیں۔ کل دورانیہ\عددی{t=\SI{11.1}{\second}} ملتا ہے۔ زمین کا رداس\عددی{r}کتنا ہے؟ 
\انتہا{سوال} %------------------
%module 1.3 mass p10
\جزوحصہء{کمیت} 
%Q20 p10
\ابتدا{سوال} 
\سن{1992} میں شیشے کی سب سے بڑی بوتل بنائی گئی جس کا حجم\عددی{193} امریکی  گیلن تھا۔ 
(ا) یہ   بوتل \عددی{\SI{1.0}{\micro\meter\cubed}} سے کتنا کم ہے؟  (ب)  اگر  بوتل \عددی{\SI{1.8}{\gram\per\minute}}  کی شرح سے پانی سے بھری  جائے،  کتنا  وقت درکار ہو  گا؟  پانی کی کثافت\عددی{\SI{1000}{\kilo\gram\per\meter\cubed}} ہے۔ 
\انتہا{سوال} 
%-----------------
%Q21 p10
\ابتدا{سوال} 
زمین کی کمیت\عددی{\SI{5.98e24}{\kilo\gram\per\meter\cubed}} ہے۔ زمین کے  جوہر   (ایٹم) کی اوسط کمیت تقریباً \عددی{\SI{40}{\atomicmassunit}}  ہے۔ زمین میں کل کتنے  جوہر  ہیں؟ 
\انتہا{سوال} 
%------------------
%Q22 p10
\ابتدا{سوال} 
سونے کی کثافت\عددی{\SI{19.32}{\gram\per\centi\meter\cubed}} ہے، اور یہ سب سے زیادہ \اصطلاح{ تار پذیر }\فرہنگ{تار پذیر}\حاشیہب{ductile}\فرہنگ{ductile} دھات ہے ، جس کو دبا کر باریک پتہ  یا کھینچ کر باریک تار بنایا  جا سکتا ہے۔ (ا) اگر\عددی{\SI{27.63}{\gram}} سونے 
سے\عددی{\SI{1.000}{\micro\meter}}موٹی چادر بنائی جائے، اس چادر کا رقبہ کتنا ہوگا؟   (ب) اس کے برعکس ، اگر اس
 سے\عددی{\SI{2.500}{\micro\meter}}رداس کا  تار بنایا  جائے، اس تار کی لمبائی کتنی ہوگی؟ 
\انتہا{سوال} 
%---------------------
%Q23 p10
\ابتدا{سوال} 
(ا)  پانی کی کثافت ٹھیک\عددی{\SI{1}{\gram\per\centi\meter\cubed}} فرض کرتے ہوئے،\عددی{\SI{1}{\cubic\meter}}پانی کی کمیت تلاش کریں۔ 
(ب) اگر ایک برتن سے\عددی{\SI{5700}{\meter\cubed}} پانی کی نکاسی\عددی{10.0} گھنٹوں میں  ہو، نکاسی کمیت کی شرح\عددی{\si{\kilo\gram\per\second}} میں  کتنی  ہوگی؟ 
\انتہا{سوال} 
%------------------
%Q24 p10
\ابتدا{سوال} 
 ساحل سمندر پر ریت زیادہ تر  کروی سلیکان ڈائی اکسائیڈ کے دانوں پر مشتمل ہے ، جن کا اوسط رداس\عددی{\SI{50}{\micro\meter}} اور 
 کثافت\عددی{\SI{2600}{\kilo\gram\per\meter\cubed}}ہے۔ کتنی کمیت کے ریتیلی دانوں کا کل سطحی رقبہ (تمام انفرادی کروں کا مجموعی رقبہ)
 \عددی{\SI{1.00}{\meter}}  ضلع   کے  مکعب کے  سطحی رقبہ کے برابر ہوگا؟ 
\انتہا{سوال} 
%--------------------
%Q25 p10
\ابتدا{سوال} 
تیز بارش کے دوران پہاڑی کا ایک حصہ ، جس کی افقی  لمبائی\عددی{\SI{2.5}{\kilo\meter}} ، ڈھلوان کے ہمراہ لمبائی\عددی{\SI{0.8}{\kilo\meter}} ، اور موٹائی \عددی{\SI{2}{\meter}} ہے،  نیچے گرتا ہے۔  مٹی وادی میں\عددی{\SI{0.4}{\kilo\meter \times \SI{0.4}{\kilo\meter}}}  رقبے پر یکساں تقسیم ہوتی ہے۔ مٹی کی کثافت\عددی{\SI{1900}{\kilo\gram\per\meter\cubed}} لیں۔ وادی کے\عددی{\SI{4}{\meter\squared}} رقبے پر مٹی کی کمیت کیا ہوگی؟
 \انتہا{سوال} 
 %------------------------------------
 %Q26 p10
\ابتدا{سوال} 
\اصطلاح{ تودہ ابر بادل  }\فرہنگ{بادل!تودہ ابر}\حاشیہب{cumulus clouds}\فرہنگ{clouds!cumulus} کے  \عددی{\SI{1}{\centi\meter\cubed}} میں تقریباً \عددی{50} تا\عددی{500} پانی کے قطرے پائے جاتے ہیں، جن کا عمومی   رداس\عددی{\SI{10}{\micro\meter}} ہو گا۔ دیے گئے  سعت  کے لیے، درج ذیل کی کمتر اور بلندتر قیمتیں کیا ہوں گی؟  (ا) نلکی شکل  و صورت کے تودہ ابر بادل ، جس کا  رداس\عددی{\SI{1}{\kilo\meter}}اور قد \عددی{\SI{3}{\kilo\meter}} ہو ، میں کتنا \عددی{\si{\meter\cubed}} پانی  ہو  گا؟  (ب) یہ پانی ایک لیٹر کی  کتنی بوتلیں بھر سکتا ہے؟  (ج) پانی کی کثافت\عددی{\SI{1000}{\kilo\gram\per\meter\cubed}}ہے۔   بادل میں  پانی کی کمیت کیا ہو گی؟ 
\انتہا{سوال} 
%-----------------------------
%Q27 p10
\ابتدا{سوال} 
لوہے کی کثافت\عددی{\SI{7.87}{\gram\per\centi\meter\cubed}} ہے ، جبکہ لوہے کے جوہر  (ایٹم)کی کمیت\عددی{\SI{9.27e-26}{\kilo\gram}} ہے۔ فرض کریں  جوہر کروی ہے اور ان کے بیچ  فاصلہ نہیں پایا جاتا۔  (ا)   لوہے کے جوہر کا حجم اور  (ب) قریبی جوہر کے مراکز کے بیچ فاصلہ کیا  ہو گا؟ 
\انتہا{سوال} 
%-------------------------------
%Q28 p10
\ابتدا{سوال} 
جوہر کے  ایک  \اصطلاح{ مول }\فرہنگ{مول}\حاشیہب{mole}\فرہنگ{mole} سے مراد عدد \عددی{\num{6.02e23}} ہے۔  موٹی گھریلو بلی میں، مقدار کے قریبی رتبہ تک،  کتنے  مول  جوہر  ہوں گے؟ ہائیڈروجن جوہر، آکسیجن جوہر، اور کاربن جوہر کی کمیتیں بالترتیب\عددی{\SI{1}{\atomicmassunit}}،
\عددی{\SI{16}{\atomicmassunit}} ، اور\عددی{\SI{12}{\atomicmassunit}}ہیں۔ 
\انتہا{سوال} 
%--------------------------------
%Q29 p10
\ابتدا{سوال} 
آپ ملائیشیا کے مویشی منڈی میں  بیل خریدتے ہیں ، جس کا  وزن مقامی اکائیوں میں \عددی{28.9} پکول  ہے:  ایک پکول  \عددی{100} جن کے برابر ہے،   ایک جن   \عددی{16}  تاہل، ایک تاہل   \عددی{10}  چی ، اور ایک چی \عددی{10} ہون  کے برابر ہے۔ ایک ہون کی کمیت\عددی{\SI{0.3779}{\gram}} ہے۔ بیل کی کمیت\عددی{\si{\kilo\gram}}میں کتنی ہے؟ 
\انتہا{سوال} 
%----------------------------------
%Q30 p10
\ابتدا{سوال} 
رستا ہوئے ظرف میں پانی انڈیلا جاتا ہے۔ پانی کی کمیت وقت \عددی{t} کا تفاعل \عددی{m=5.00t^{0.8} - 3.00t + 20.00} ہے، جہاں\عددی{t\ge 0} ،\عددی{m} کی اکائی گرام  ، اور\عددی{t}کی اکائی سیکنڈ  ہے۔  (ا) کس لمحے پر  پانی کی کمیت  اعظم ہے، اور   (ب)   اعظم کمیت کتنی ہے؟  کمیت میں تبدیلی کی شرح،\عددی{\si{\kilo\gram\per\minute}}  اکائیوں میں ،  (ج)\عددی{t=\SI{2.00}{\second}} اور  (د) \عددی{t=\SI{5.00}{\second}} پر  کیا ہے؟ 
\انتہا{سوال}
%----------------------------
%Q31 p10
 \ابتدا{سوال} 
 سیدھا کھڑا برتن ، جس کی تہہ  کا رقبہ\عددی                            {\SI{14}{\centi\meter}} با \عددی{\SI{17}{\centi\meter}}ہے ،   مٹھائی  سے بھرا جاتا ہے۔ انفرادی  مٹھائی کی کمیت\عددی{\SI{0.0200}{\gram}} اور حجم\عددی{\SI{50.0}{\milli\meter\cubed}} ہے۔ مٹھائیوں  کے بیچ  خلا نظر انداز کریں۔  برتن میں مٹھائیوں کی بلندی کی  شرح   \عددی{\SI{0.250}{\centi\meter\per\second}} ہے۔برتن میں  مٹھائی کی  کمیت میں اضافہ  کی شرح   ( کلوگرام فی منٹ) کیا ہے؟ 
\انتہا{سوال} 
%--------------------------------

%additional problems p10
\موٹا{اضافی سوالات }\\
%Q32 p10
\ابتدا{سوال} 
حقیقی گھر کے لحاظ سے\عددی{1:12}پیمانہ   سے گڑیا  گھر بنایا جاتا ہے (یعنی گڑیا کے گھر کا ہر ضلع حقیقی  گھر کے مطابقتی ضلع کا\عددی{\tfrac{1}{12}}ہوگا)  ۔ ساتھ ہی حقیقی گھر کے\عددی{1:144} پیمانہ سے  مزید چھوٹا گھر تعمیر کیا جاتا ہے، جو گڑیا  گھر کے اندر رکھا جائے گا۔ فرض کریں، حقیقی گھر (شکل\حوالہء{1.7} )  کی لمبائی (سامنے سے) \عددی{\SI{20}{\meter}} ، گہرائی \عددی{\SI{12}{\meter}} ، اور بلندی \عددی{\SI{6}{\meter}} ہے،  اور اس کا چھت ڈھلوانی ہے،  جس کی اونچائی \عددی{\SI{3.0}{\meter}}  ہے۔  (ا) گڑیا گھر  اور  (ب)  گڑیا گھر کے اندر رکھے جانے والے مزید چھوٹے گھر کا حجم  ، مربع میٹر میں  ، کیا ہو گا؟ 
\انتہا{سوال}
%-----------------------
%========================

%Q33 p11
\ابتدا{سوال} 
برصغیر میں لمبائی کی قدیم اکائی کوس ہے جو آئین  اکبری میں پانچ ہزار گز کے برابر رکھا گیا۔ برطانوی سامراج نے  اکبری گز \عددی{33} انچ مقرر کیا۔ یوں  کوس تقریباً\عددی{\SI{419}{\meter}}کے برابر ہے۔ مغلیہ دور کی شاہراہوں (جی ٹی روڈ) پر جگہ جگہ اب بھی کوس مینار کھڑے نظر آتے ہیں۔ دریائے سندھ پر اٹک کے قریب پرانے پُل کے نزدیک ایسا ایک مینار اب بھی کھڑا ہے۔ مغلیہ دور میں پرسنگ (یا فرسخ)   بھی استعمال ہوتا رہا۔ پرسنگ  وہ فاصلہ ہے جو  گھوڑا چل کر ایک گھنٹے میں طے کرتا ہے۔ یوں ایک پرسنگ تقریباً \عددی{3} میل کے برابر ہے۔ موٹروے پر لاہور سے ملتان تک کا فاصلہ\عددی{\SI{350}{\kilo\meter}}اور\عددی{\SI{306}{\meter}}ہے۔ یہ فاصلہ کتنے کوس اور کتنے پرسنگ ہے؟  

جواب: \عددی{83.6}کوس،\عددی{73} پرسنگ  
\انتہا{سوال} 
%------------------------------------
%Q34 p11
\ابتدا{سوال} 
موجودہ دور میں ایک گز\عددی{36}  \اصطلاح{انچ }\فرہنگ{انچ}\حاشیہب{inch}\فرہنگ{inch} یعنی تین فٹ کے برابر مانا جاتا ہے۔ پاکستان میں  اراضی  کی پیمائش ایکڑ، کنال، مرلہ میں کی جاتی ہے۔ ایک ایکڑ میں آٹھ کنال اور  ایک کنال\عددی{20}مرلہ کے برابر ہے۔ ایک کنال ٹھیک\عددی{605}مربع گز یعنی\عددی{\SI{505.857}{\meter\squared}}  کے برابر ہے۔  (ا)\عددی{48.5}کنال کا رقبہ کتنے مرلہ ہوگا،   (ب) یہی رقبہ کتنے مربع گز ہوگا؟  (ج)\عددی{1135}ایکڑ کا رقبہ چوکور  ہے۔ اس کا ایک کنارہ کتنے گز ہوگا؟ 

جواب: (ا)\عددی{970}مرلہ ، (ب)\عددی{29343}مربع گز، (ج)\عددی{2344}گز
\انتہا{سوال} 
%----------------------------------------------------------------
%Q35 p11
\ابتدا{سوال} 
ایک  \اصطلاح{من }\فرہنگ{من}\حاشیہب{maund}\فرہنگ{maund} وزن دس \اصطلاح{ دھڑی }\فرہنگ{دھڑی} کے برابر، ایک دھڑی  چار \اصطلاح{ سیر }\فرہنگ{سیر} کے برابر، ایک سیر چار \اصطلاح{   پاو }\فرہنگ{پاو} کے برابر،  اور ایک پاو  چار \اصطلاح{ چھٹانک }\فرہنگ{چھٹانک} کے برابر ہے۔ ایک من ٹھیک\عددی{\SI{37.3242}{\kilo\gram}}کے برابر ہے۔ ایک شخص تین من دو دھڑی پانچ سیر تین پاو  اور دو چھٹانک  گندم   خریدتا ہے۔ گندم کی کمیت\عددی{\si{\kilo\gram}}میں کتنی  ہے؟ 

جواب: \عددی{\SI{124.9194319}{\kilo\gram}}
\انتہا{سوال} 
%---------------------
%Q36 p11
\ابتدا{سوال} 
سونے کے وزن کی اکائی  \اصطلاح{تولہ  }\فرہنگ{تولہ}\حاشیہب{tola}\فرہنگ{tola}ہے۔ ایک تولہ\عددی{12} \اصطلاح{ماشوں }\فرہنگ{ماشہ} کے برابر ہے۔ ایک تولہ ٹھیک\عددی{\SI{11.6638038}{\gram}}کے برابر ہے۔ آپ سنار سے پانچ تولہ اور سات ماشے کا سونا خریدتے ہیں۔ یہ کتنے گرام کے برابر ہوگا؟ 

جواب:  \عددی{\SI{65.12288}{\gram}}
\انتہا{سوال}
%---------------------------------------------
%Q37 p11
\ابتدا{سوال} 
چینی کے  کعبی  دانے کا ضلع \عددی{\SI{1}{\centi\meter}}ہے۔ اگر   کعبی ڈبے میں ایک \ترچھا{مول } چینی کے کعبی دانے  ہوں،  ڈبے   کا ضلع کیا ہو گا؟ (ایک مول  \عددی{6.02e23} کو کہتے ہیں۔) 
\انتہا{سوال}
%---------------------------------------
%Q38 p11
\ابتدا{سوال} 
ایک یوسف زئی خان کے پاس\عددی{12}ہزار جریب کی اراضی ہے۔ ایک جریب چار کنال کے برابر ہے، اور ایک کنال\عددی{\SI{101.1716}{\meter\squared}}کے برابر ہے۔ یہ اراضی  مربع ہے۔ اس مربع کا ضلع کتنے\عددی{\si{\kilo\meter}}ہوگا؟ 

جواب: \عددی{\SI{1.102}{\kilo\meter}}
\انتہا{سوال} 
%----------------------------------------
%Q39 p11
\ابتدا{سوال} 
برطانوی  گیلن امریکی  گیلن سے مختلف ہے: ایک برطانوی  گیلن \عددی{\num{4.546090}} لیٹر ، جبکہ ایک امریکی   
گیلن \عددی{\num{3.7854118}} لیٹر کے برابر ہے۔ برطانیہ میں خریدی گئی  گاڑی  کے بنانے والے دعویٰ کرتے ہیں کہ  ان کی گاڑی  ایک گیلن
 تیل  میں\عددی{\SI{40}{\kilo\meter}}فاصلہ طے کرتی ہے۔ ایک شخص  گاڑی خرید کر امریکہ لے جاتا ہے۔ امریکہ میں\عددی{750}  میل فاصلہ  (ا) کتنے گیلن تیل میں طے ہونا متوقع ہے اور  (ب) گاڑی حقیقتاً کتنا تیل استعمال کرے گی؟ 
\انتہا{سوال} 
%---------------------------------------
%Q40 p11
\ابتدا{سوال} 
اس باب میں پیش کیے گئے مواد کو استعمال کرتے ہوئے دریافت کریں کہ\عددی{\SI{1.0}{\kilo\gram}}ہائیڈروجن حاصل کرنے کے لیے ہائیڈروجن جوہروں کی تعداد کتنی ہوگی۔  ہائیڈروجن جوہر کی کمیت\عددی{\SI{1.0}{\atomicmassunit}}ہے۔ 
\انتہا{سوال}
%----------------------
%Q41 p11
\ابتدا{سوال} 
ایک ڈرمی جس کی لمبائی دو   فٹ ہے کا حجم\عددی{100} لیٹر ہے۔ اس کا رداس\عددی{\si{\centi\meter}}میں کیا  ہوگا؟ 

جواب: \عددی{\SI{22.85}{\centi\meter}}
\انتہا{سوال} 
%----------------------------------------------------
%Q42 p11
\ابتدا{سوال} 
پانی (\ce{H2O})    کے سالمہ میں دو ہائیڈروجن اور ایک آکسیجن   جوہر پایا جاتا ہے۔ ہائیڈروجن جوہر کی کمیت\عددی{\SI{1.0}{\atomicmassunit}}اور آکسیجن جوہر کی کمیت\عددی{\SI{16}{\atomicmassunit}}ہے۔  (ا) پانی کے سالمہ کی کمیت کتنے\عددی{\si{\kilo\gram}} ہو گی؟  (ب) دنیا کے  تمام بحر میں ( تخمیناً)  کل\عددی{\SI{1.4e21}{\kilo\gram}}پانی پایا جاتا ہے۔ اس پانی میں کتنے سالمے  ہوں گے؟ 
\انتہا{سوال} 
%---------------------------------
%Q43 p11
\ابتدا{سوال} 
ایک شخص خوراک کی مقدار کم کر کے ایک ہفتہ میں\عددی{\SI{2.3}{\kilo\gram}}کمیت گھٹا سکتا ہے۔ کمیت گھٹنے کی شرح \عددی{\si{\milli\gram\per\second}}میں لکھیں۔ 
\انتہا{سوال}
%-----------------------------
%Q44 p11
\ابتدا{سوال}
پانی کی کثافت\عددی{\SI{1.0e3}{\kilo\gram\per\meter\cubed}}ہے۔ سوال  \حوالہ{سوال_پیمائش_ایکڑ_فٹ} میں دیے گئے شہر پر کتنی  کمیت کا پانی برسا؟ 
\انتہا{سوال}
%----------------------------------------
%Q45 p11
\ابتدا{سوال} 
(ا)  بعض اوقات خوردبینی  طبیعیات میں وقت کی اکائی \اصطلاح{ شیک }\فرہنگ{شیک}\حاشیہب{shake}\فرہنگ{shake} استعمال کی جاتی ہے ، جو  تقریباً\عددی{\SI{e-8}{\second}}کے برابر ہے۔ کیا  سال میں سیکنڈوں سے زیادہ سیکنڈ میں شیک پائے جاتے ہیں؟  (ب)  زمین پر بنی آدم \عددی{e6}سال گزار چکا ہے، جبکہ کائنات\عددی{e10}سال پرانی ہے۔ اگر کائنات کی پوری عمر  ایک \قول{ کائناتی دن } تصور کی جائے ، جس میں \قول{ کائناتی سیکنڈ } کی تعداد اتنی ہی ہو جتنی  ایک سادہ دن میں سادہ سیکنڈوں کی تعداد ہوتی ہے، تو بنی آدم  کتنے سیکنڈ سے زمین پر  ہے؟ 
\انتہا{سوال} 
%------------------------------------------
%Q46 p11
\ابتدا{سوال} 
کوئلے کی   کھان سے سالانہ\عددی{\SI{26}{\meter}}گہرائی تک\عددی{200}ایکڑ رقبے کا کوئلہ حاصل کیا جاتا ہے۔ یہ کوئلہ 
کتنے \عددی{\si{\kilo\meter\cubed}}کے برابر ہے؟ 
\انتہا{سوال} 
%-----------------------------------
%Q47 p11
\ابتدا{سوال} 
سورج اور زمین کے درمیان اوسط فاصلے  کو ایک \اصطلاح{ فلکیاتی اکائی }\فرہنگ{فلکیاتی اکائی}\حاشیہب{astronomical unit}\فرہنگ{astronomical unit} کہتے ہیں۔ روشنی کی رفتار تقریباً\عددی{\SI{3.0e8}{\meter\per\second}}ہے۔ روشنی کی رفتار  فلکیاتی اکائی فی منٹ  میں لکھیں۔ 
\انتہا{سوال} 
%------------------------
%Q48 p11
\ابتدا{سوال} 
  مشرقی نیولا کی کمیت تقریباً\عددی{\SI{75}{\gram}}ہوتی ہے، جو تخمیناً \عددی{7.5}مول جوہر کے برابر ہے۔ (جوہروں کا ایک مول \عددی{6.02e23}جوہر  کے برابر ہے۔)  نیولا کے  جسم میں  جوہر کی اوسط کمیت کو جوہری کمیتی اکائی (\عددی{\si{\atomicmassunit}})  میں لکھیں۔ 
\انتہا{سوال} 
%---------------------------------
%Q49 p11
\ابتدا{سوال} 
جاپان میں لمبائی کی روایتی اکائی  \اصطلاح{کین }\فرہنگ{کین}\حاشیہب{ken}\فرہنگ{ken} ہے۔ (ایک کین \عددی{\SI{1.97}{\meter}} کے برابر ہے۔)(ا)  مربع کین اور مربع میٹر  کی نسبت اور  (ب) کعبی کین اور کعبی میٹر کی  نسبت کیا ہے؟ ایک بیلن ، جس کی بلندی\عددی{5.05} کین  اور رداس\عددی{3.00} کین ہے ، کا حجم  (ج)  کعبی کین اور   (د)   کعبی  میٹر میں  کیا  ہوگا؟ 
\انتہا{سوال} 
%--------------------------------
%Q50 p11
\ابتدا{سوال} 
آپ مشرق کی طرف\عددی{24.5} میل کشتی  چلاتے ہیں  جبکہ آپ کو\عددی{24.5}سمندری میل سفر کرنا تھا۔ ایک سمندری  میل \عددی{1.1508}  زمینی میل کے برابر ہے۔ آپ اصل مقام سے کتنا دور ہیں؟
\انتہا{سوال}
%----------------------------------------
