%???KKK not edited
%module3.1
%HRp57
\جزوحصہ{سمتیات اور ان کے اجزاء}
%Q3.1-1
\ابتدا{سوال}
ایک سمتیہ جس کا قدر
 \عددی{\SI{7.3}{\meter}} 
 ہے مثبت x محور کے رخ سے گھڑی کی سوئی کے مخالف رخ
 \عددی{\SI{250}{\degree}} 
 پر x y مستوی میں پایا جاتا ہے،
 الف اس کا x جز اور
ب  وائی جز تلاش کریں۔ 
\انتہا{سوال}
%--------------------------------------
%Q3.1-2
\ابتدا{سوال}
سمتیہ ہٹاؤ r کا قدر\عددی{\SI{15}{\meter}}   ہے اور یہ x y  مستوی میں زاویہ  \عددی{\theta=\SI{30}{\degree}}  کہ رخ ہے، شکل 3.26 دیکھیں اس سمتیہ کے الف x جز اور
 ب y جز تلاش کریں۔
 \انتہا{سوال}
 %--------------------------
 %Q3.1-3
\ابتدا{سوال} 
 سمتیہ a کا x جز -  \عددی{\SI{25}{\meter}}  اور y جزو  \عددی{\SI{40}{\meter}}  ہے۔  الف سمتیہ a کا قدر کتنا ہے؟  ب سمتیہ a کے رخ اور محور x کے مثبت رخ کے بیچ زاویہ کتنا ہے؟ 
\انتہا{سوال}
%--------------------------
%Q3.1-4
\ابتدا{سوال}
 درج ذیل زاویوں کو ریڈین میں بیان کریں: الف\عددی{\SI{20}{\degree}} ، ب \عددی{\SI{50}{\degree}} ، ج \عددی{\SI{100}{\degree}} ۔ درج ذیل زاویوں کو درجوں کی صورت میں پیش کریں: د 0.330 ریڈین،  ح 2.10 ریڈین، 
\انتہا{سوال}
%===========================
%Q3.1-5
\ابتدا{سوال} 
 ایک بحری جہاز شمال کے رخ\عددی{\SI{120}{\kilo\meter}}  دور نقطہ کی جانب پہنچنا چاہتا ہے۔ سفر کے اغاز سے پہلے ہی ایک غیر متوقع اندھی اس کو نقطہ اغاز سے مشرق جانب  \عددی{\SI{100}{\kilo\meter}}   دور دکھیلتا ہے۔ اس جہاز کو اختتامی نقطہ پر پہنچنے کے لیے الف کتنا فاصلہ طے کرنا ہوگا  ب اسے کس رخ سفر کرنا ہوگا؟
\انتہا{سوال}
%-------------------------
%Q3.1-6
\ابتدا{سوال}
شکل 3.27 میں ایک بھاری مشین کو اف کی رخ سے زاویہ  \عددی{\theta=\SI{20}{\degree}} پر رکھے گئے تختے پر  \عددی{d=\SI{12.5}{\meter}} فاصلے تک گھسیٹا جاتا ہے۔ اس مشین کو (الف) انتصابی روح اور  (ب) اف کی رخ کتنا دور منتقل کیا گیا؟ 
\انتہا{سوال}
%--------------------------
%Q3.1-7
\ابتدا{سوال}
ایک ہٹاؤ جس کا قدر  \عددی{\SI{3}{\meter}}  ہے اور دوسرا ہٹاؤ جس کا قدر  \عددی{\SI{4}{\meter}}  ہے پر غور کریں۔ دکھائیں کہ ان ہٹاؤ سمیات کو استعمال کرتے ہوئے  (الف) 
\عددی{\SI{7}{\meter}}  ، (ب)  \عددی{\SI{1}{\meter}} ، اور (ج) \عددی{\SI{5}{\meter}} قدر کے ہٹاؤ حاصل کیے جا سکتے ہیں۔ 
\انتہا{سوال}
%Module 3.2
\موٹا{ اکائی سمتیات، سمتیات کی جمع بذریعہ اجزاء }
%Q3.2-8
\ابتدا{سوال}
ایک شخص  \عددی{\SI{3.1}{\kilo\meter}} شمال کی طرف چلنے کے بعد  \عددی{\SI{2.4}{\kilo\meter}} مغرب اور اخر میں  \عددی{\SI{5.2}{\kilo\meter}}  جنوب کے رخ چلتا ہے۔ (الف) اس کے حرکت کو ظاہر کرنے کے لیے سمتی …
\انتہا{سوال}
%----------------------------
%Q3.2-9
\ابتدا{سوال}
درج ذیل دو سمتیات دیے گئے ہیں 
\begin{align*}
\vec{a}=(\SI{4}{\meter})\hat{i} - (\SI{3}{\meter})\hat{j} + (\SI{1}{\meter})\hat{k}
\end{align*}
اور
\begin{align*}
\vec{b}=(\SI{-1}{\meter})\hat{i} + (\SI{1}{\meter})\hat{j} + (\SI{4}{\meter})\hat{k}
\end{align*}
اکائی سمتیہ علامتیت میں 
(الف)
\انتہا{سوال}

