\documentclass{book}
\usepackage{amsmath,amssymb}
\usepackage{polyglossia}
\setmainlanguage[numerals=maghrib]{arabic}
\setotherlanguages{english}
\newfontfamily\arabicfont[Scale=1.0,Script=Arabic]{Jameel Noori Nastaleeq}
\newfontfamily\urdufont[Scale=1.25,Script=Arabic]{Jameel Noori Nastaleeq}
\begin{document}
\section*{جز حصہ 4.4.1 چکر $\frac{1}{2}$ }
سادہ مادہ پروٹان، نیوٹران، الیکٹران کے ساتھ ساتھ کوارکس اور تمام الیکٹران کیلۓ $ S=1/2 $ ہوگا۔ لہذا یہ اہم ترین صورت ہے۔ مزید $ 1/2 $ چکر سمجھنے کے بعد زیادہ چکر کے ضوابت دریافت کرنا نصبتا آسان ہوگا۔ یہاں صرف دو امتیازی تفاعلات پاۓ جاتے ہیں۔ پہلا $ |1/2 1/2 ) $ ہے جس کو ہما میدان چکر یا غیر رسمی طور پر up arrow اور دوسرا $ |1/2(-1/2)) $ ہے جس کو ہن مخالف میدان چکر یا رسمی طرو پر ڈاؤن ایرو کہتے ہیں۔ انہیں اساس سمتیات لیتے ہوۓ 1/2 چکر ذرے کو دو اجزائی قالب سے ظاہر کر سکتے ہیں۔
$$ \chi=( \left(\begin{matrix} a \\ b \end{matrix}\right))= a\chi_{+} + b\chi_{-} \quad [4.139] $$
جہاں
$$ \chi_{+}=\left(\begin{matrix}1\\0 \end{matrix}\right)\quad [4.140] $$
ہما میدان چکر کو 
$$ \chi_{-}=\left(\begin{matrix}0 \\1 \end{matrix}\right)\quad [4.141] $$
مخالف میدان چکر کو ظاہر کرتے ہیں۔
ساتھ ہی عاملین چکر 2X2 قالب ہوں گے۔ جن کی کارکردگی دیکھنے کی خاطر ہم ان کا اطلاق $ +\chi $ اور $ -\chi $ پر کرتے ہیں۔ مساوات 4.135 ذیل کہتی ہے۔
$$ S^2\chi_{+}=\frac{3}{4}(\hbar)^2\chi_{+} \quad S^2\chi_{-}= \frac{3}{4}(\hbar)^2 \chi_{-} \quad [4.142] $$
ہم $ S^2 $ کو اب تک نامعلوم ارکان کا قالب 
$$ S^2= \left(\begin{matrix}a & b\\c & d\end{matrix}\right) $$
لکھ کر پہلی مساوات کو درج ذیل لکھ سکتے ہیں۔
$$ \left(\begin{matrix}c & d \\ e & f \end{matrix}\right) \left(\begin{matrix}1\\0 \end{matrix}\right) = \frac{3}{4}(\hbar)^2 \left(\begin{matrix}\hbar \\0 \end{matrix}\right) , \quad or \quad \left(\begin{matrix}c\\e \end{matrix}\right)= \left(\begin{matrix}\frac{3}{4}(\hbar)^2 \\ 0 \end{matrix}\right) $$
لہذا $ c=(3/4)\hbar^2 $ اور $ e=0 $ ہوگا۔ دوسری مساوات کے تخط درج ذیل ہوگا۔
$$ (\begin{matrix} c & d \\ e & f end{matrix}) (\begin{matrix} 0 \\ 1 \end{matrix}) = \frac{3}{4}\hbar^2
 (\begin{matrix} 0 \\ 1 \end{matrix}), \quad or \quad (begin{matrix} d \\ f \end{matrix})= (\begin{matrix}0 \\ \frac{3}{4}\hbar^2 \end{matrix}) $$ 
 لہذا $ d=0 $ اور $ f=(\frac{3}{4})(\hbar)^2 $ ہوگا۔ ماخوذ
 $$ S^2= \frac{3}{4}(\hbar)^2 \left(\begin{matrix} 1&0 \\ 0&1 \end{matrix}\right) \quad [4.143$$ 
%page 174 
اسی طرح 
$$ S_{z}\chi_{+}=\frac{\hbar}{2}\chi_{+}, \quad S_{z} \chi_{-}=-\frac{\hbar}{2}\chi_{-}, $$
سے درج ذیل حصل ہوگا۔
$$ S_{z}= \frac{\hbar}{2}\left(\begin{matrix} 1&0 \\ 0&-1 \end{matrix}\right). $$
ساتھ ہی مساوات 4.136 ذیل کہتی ہے۔
$$ S_{+} \chi_{-}=\hbar_{\chi+}, \quad S_{-}\chi_{+}=\hbar_{\chi-} , S_{+} \chi_{+}= S_{-} \chi_{-}=0, $$
لہذا درج ذیل ہوگا
$$ S_{+}=\hbar\left(\begin{matrix} 0&1 \\ 0&0 \end{matrix}\right), \quad S_{-}=\hbar\left(\begin{matrix}0&0 \\ 1&0 \end{matrix}\right) $$
چونکہ 
$ S_{\pm}=S_{x} {\pm} \iota S_{y}, so \quad S_{x}=\frac{1}{2}(S_{+}+S_{-}) and \quad S_{y}= \frac{1}{2\iota} (S_{+}-S_{-}), $ 
لہذا درج ذیل ہو گا
$$ S_{x}=\frac{\hbar}{2}\left(\begin{matrix} 0&1 \\1&0 \end{matrix}\right), \quad S_{y}=\frac{\hbar}{2}\left(\begin{matrix} 0&-\iota \\ \iota& 0 \end{matrix}\right)\hbar\quad [4.147] $$
چونکہ $ S_{x} $ , $ S_{y} $ , $ S_z $ تینوں میں $  \hbar /2 $ کا جز ضربی پایا جاتا ہے لہذا انہیں زیادہ صاف روپ $ S=\frac{(\hbar)}{2} )\sigma \sigma $ لکھا جا سکتا ہے۔ جہاں درج ذیل ہوں گے
$$ \sigma_{x}=\left(\begin{matrix} 0&1 \\ 1&0 \end{matrix}\right), \quad \sigma_{y}= \left(\begin{matrix} 0&-\iota \\ (\iota)&0 \end{matrix}\right), \quad \sigma_{z}=\left(\begin{matrix} 1&0 \\ 0&-1 \end{matrix}\right). \quad [4.148] $$
یہ پالی قالب چکر ہیں۔ دھیان رکھیں کہ $ S_{x} $, $ S_{y} $ , $ S_{z} $  اور $ S^2 $ تمام ہر میشی ہیں۔ جیسا انہیں ہونا چاہیے۔ کیونکہ یہ مشہوداذ کو ظاہر کرتے ہیں۔ اس کے برعکس $ S_{+} $ اور  $ S_{-} $ غیر ہرمیشی ہیں۔ چونکہ یہ غیر مشہود ہیں۔ $ S_{z} $  کی امتیازی سپائنرز درج ذیل ہوں گے
$$ \chi_{+}= \left(\begin{matrix} 1 \\ 0 \end{matrix}\right),(eigenvalue+\frac{\hbar}{2}); \quad \chi_{-}=\left(\begin{matrix} 0 \\ 1 \end{matrix}\right), \quad (eigenvalue-\frac{\hbar}{2}). \quad [4.149] $$
عمومی حال چال مساوات 4.139 میں ایک ذرے کی $ S_{z} $ کی پیمائش  $ a^2 $ احتمال کے ساتھ $ +\hbar/2 $ یا  $ b^2 $ احتمال کے ساتھ $ -\hbar/2  $ دے سکتی ہے۔ چونکہ صرف یہی ممکنات ہیں لہذا درج ذیل ہوگا
$$ |a|^2+|b|^2=1 \quad[4.150] $$
یعنی سپائنر لازما معمول شدہ ہوگا۔
%page 175 
لیکن اس کی بجاۓ آپ $ S_{x} $ کی پیمائش بھی کر سکتے ہیں۔ اس کے کیا نتائج اور ان کے انفرادی احتمالات کیا ہونگے۔ عمومی شماریاتی مفہوم کے تحت ہمیں 
$ S_{x} $ 
کی امتیازی اقدار اور امتیازی سپائنر جاننا ہوگا۔ امتیازی مساوات درج ذیل ہوگا
$$ \begin{vmatrix} -\lambda & \frac{\hbar}{2} & -\lambda \end{vmatrix}=0 \rightarrow \lambda^2=(\frac{\hbar}{2})^2\rightarrow\lambda={\pm}\frac{\hbar}{2} $$
یہ حیرت کی بات نہیں ہے کہ
 $ S_{x} $ 
 کی ممکنہ حقدار وہی ہیں جو
  $ S_{z} $ 
 کی ہیں۔ امتیاذی سپائنر کو ہمیشہ کی طرح حاصل کر تے ہیں
$$ \frac{\hbar}{2}\begin{pmatrix}0&1 \\ 1&0 \end{pmatrix} \begin{pmatrix} \alpha \\ \beta \end{pmatrix}= {\pm}\frac{\hbar}{2}\begin{pmatrix}\alpha \\ \beta \end{pmatrix} \rightarrow \begin{pmatrix}\beta \\ \alpha \end{pmatrix} ={\pm} \begin{pmatrix}\alpha \\ \beta \end{pmatrix} $$ 
لہذا $ \beta={pm}\alpha $ ہوں گے۔ آپ دیکھ سکتے ہیں کہ $ S_{x} $ کے معمول شدہ امتیازی سپائنرز درج ذیل ہوں گے
$$ (\chi_{+})^(x)=\left(\begin{matrix} \frac{1}{\sqrt{2}} \\ \frac{1}{\sqrt{2}} \end{matrix}\right), (eigenvalue+\frac{\hbar}{2}; (\chi_{-})^(x)= \left(\begin{matrix}\frac{1}{\sqrt{2}} \\ \frac{-1}{\sqrt{2}} \end{matrix}\right), (eigenvalue-\frac{\hbar}{2}). \quad[4.151] $$
بطور ہرمیشی قالب کے امتیازی سمتیات یہ فضا کا احاطہ کرتے ہیں۔ قلی سپائنر $ \chi $ مساوات 4.139 کو ان کا خطی جوڑ لکھا جا سکتا ہے۔
$$  \chi=(\frac{a+b}{\sqrt{2}})(\chi_{+})^(x) +( \frac{a-b}{\sqrt{2}}(\chi_{-})^(x). \quad [4.152] $$ 
اگر آپ $ S_{x} $ کی پیمائش کریں تو آپ $ \frac{1}{2}(|a+b|)^2 $ احتمال کے ساتھ $ +\hbar/2 $ جبکہ $ \frac{1}{2}(|a-b|)^2 $ احتمال کے ساتھ $ - \hbar/2 $ حاصل کر سکتے ہیں۔ ( تصدیق کیجیۓ کہ ان احتمالات کا مجموعہ 1 کے برابر ہے۔)
\paragraph{textbf{مثال 4.2 }} \quad فرض کریں $ \frac{1}{2} $ چکر کا ایک ذرہ درج ذیل حال میں ہے
$$ \chi=\frac{1}{\sqrt{6}}\left(\begin{matrix} 1+\iota \\ 2 \end{matrix}\right). $$
بتائیے گا کہ $ S_{z} $ اور $ S_{x} $ کی پیمائش کرتے ہوۓ $ +\hbar/2 $ اور $ -\hbar/2 $ حاصل کرنے کے احتمالات کیا ہونگے۔
\paragraph{:حل}
یہاں $ a=(1+\iota)\sqrt{6} $ اور $ b=\frac{2}{\sqrt{6}} $ ہے۔ لہذا $ S_{z} $ کیلۓ $ \frac{\hbar}{2} $ 
کے حصول کا احتمال $ |(1+\iota)/\sqrt{6}|^2=1/3 $ ہوگا۔ جبکہ $ -\frac{\hbar}{2} $ حاصل کرنے کا احتمال $ |\frac{2}{\sqrt{6}}|^2 =2/3 $ ہوگا۔ $ S_{x} $  کیلۓ $ +\frac{\hbar}{2} $ کے حصول کا احتمال $ (\frac{1}{2}) (3+\iota)/\sqrt{6}|^2=5/6 $ ہوگا۔ جبکہ $ \frac{-\hbar}{2} $ کے حصول کا احتمال
%page 176 
جبکہ $ -\frac{\hbar}{2} $ کے حصول کا احتمال $ \frac{1}{2} (-1+\iota)/\sqrt{6}|^2 =1/6 $  ہوگا۔ اتفاقا $ S_{x} $ کی توقعاتی قیمت درج ذیل ہوگی
$$ \frac{5}{6}(+\frac{\hbar}{2})+\frac{1}{6}(-\frac{\hbar}{2})=\frac{\hbar}{3} $$
جس کو ہم بلاواسطہ درج ذیل طریقے سے بھی حاصل کر سکتے ہیں۔
$$ (S_{x})=(\chi)^(+)S_{x}\chi=\left(\begin{matrix}\frac{1-\iota}{\sqrt{6}} \\ \frac{2}{\sqrt{6}} \end{matrix}\right)\left(\begin{matrix} 0&\frac{\hbar}{2} \\ \frac{\hbar}{2}&0 \end{matrix}\right)\left(\begin{matrix}\frac{1+\iota}{\sqrt{6}} \\ \frac{2}{\sqrt{6}}\end{matrix}\right)=\frac{\hbar}{3}. $$ 
\end{document}
