
ہم \عددی{ \frac{1}{\sqrt{2 \pi }} } کو اپنی آسانی کیلئے تکمل کے باہر نکالتے ہیں. مساوات \حوالہ{ 2.17} میں عددی سر \عددی{ c_{n} } کی جگہ یہاں \عددی{ ( 1/2\pi) \phi(k) \dif k  } کردار ادا کرتا ہے. اب یہ تفاعل موج موزوں \عددی{ \phi(k) } کیلئے معمول پر لایا جا سکتا ہے. لیکن یہ لازم ہے کہ اس میں \عددی{ k } کی قیمتوں کا ایک پلندہ ہو. اور یوں اس میں توانائی اور رفتاروں کا بھی ایک پلندہ ہو گا. اسی لیے اس کو ہم موجی پلندہ کہتے ہیں. 
عمومی کوانٹم مسئلہ میں ہمیں \عددی{ \Psi (x,0) } دیا جاتا ہے اور ہمیں \عددی{ \Psi(x,t) } تلاش کرنے کو کیا جاتا ہے. آزاد ذرے کیلئے اسکا حل مساوات \حوالہ{ 2.100} کی صورت اختیار کرتا ہے. اب سوال یہ پیدا ہوتا ہے کہ ابتدائی طفاعل موج 
\begin{align}
\Psi (x,0) = \frac{1}{\sqrt{2\pi}} \int_{- \infty}^{+ \infty} \phi (k) e^{ikx} \dif k. 
\end{align}
پر پورا اترا ہوا \عددی{ \psi(k) } کیسے تلاش کیا جائے؟ 
یہ فوریئر تجزیے کا کلاسیکی مسئلہ ہے جسکا جواب مسئلہ پلاشرال پیش کرتا ہے. 
\begin{align}
f(x) = \frac{1}{\sqrt{2\pi}} \int_{- \infty}^{+ \infty} F(k) e^{ikx} \dif k \Leftrightarrow F(k) = \frac{1}{\sqrt{2\pi}} \int_{- \infty}^{+ \infty} f(x) e^{-ikx} \dif x
\end{align}
\عددی{    F(k) } کو \عددی{  f(x)   } کا فوریئر بدل کہا جاتا ہے. جبکہ \عددی{   f(x)  } کو \عددی{   F(k)  } کا الٹ فوریئر بدل کہتے ہیں. ان دونوں میں صرف قوت نمائی کئ علامت کا فرق پایا جاتا ہے. ظاہر ہے کہ اجازتی تفاعل پر کچھ شرائط لاگو ہوتی ہیں. تکمل کا موجود ہونا لازم ہے. \عددی{   \Psi(x,0)  } پر لاگو طبی شرط کہ وہ معمول پر لانے کے قابل ہو, تکمل کی موجودگی کی ضمانت دیتا ہے. یوں آزاد ذرے کے عمومی کوانٹم مسئلے کا حل مساوات \حوالہ{ 2.100} ہے جہاں درج ذیل ہو 
\begin{align}
\phi(k) = \frac{1}{\sqrt{2\pi}} \int_{- \infty}^{+\infty} \Psi(x,0)e^{-ikx} \dif x
\end{align}
مثال 2.6
ایک آزاد ذرہ جسکا مکانی خطہ \عددی{  -a \leq x \leq a    } میں ابتدائی طور پر رہنے کا پابند بنایا گیا ہو کو وقت \عددی{   =0t  } چھوڑ دیا جاتا ہے. 
\begin{align*}
\Psi (x,0) = \left\{ \begin{array}{lc}
A, & if -a < x < a, \\ 0, & Otherwise,
\end{array} \right.
\end{align*}
جہاں \عددی{   A  } اور \عددی{  a   } مثبت حقیقی مستقل ہیں. \عددی{   \Psi(x,t)  } تلاش کریں. 
حل 
ہمیں \عددی{  \Psi(x,0)  } کو معمول پہ لانا ہو گا. 
\begin{align*}
1 = \int_{-\infty}^{\infty} \left| \Psi (x,0) \right| ^{2} \dif x = \left| A \right|^{2} \int_{-a}^{a} \dif x = 2a \left| A \right|^{2} \Rightarrow A = \frac{1}{\sqrt{2a}}.
\end{align*}
اس کے بود مساوات \حوالہ{ 2.103} استعمال کرتے ہوئے \عددی{  \psi(k)   }  تلاش کرنا ہو گا.

\begin{align*}
\phi(k) =& \frac{1}{\sqrt{2\pi}} \frac{1}{\sqrt{2a}} \int_{-a}^{a} e^{-ikx} \dif x = \frac{1}{2\sqrt{\pi a}} \left. \frac{e^{-ikx}}{-ik} \right|_{-a}^{a} \\
=& \frac{1}{k\sqrt{\pi a}} \left( \frac{e^{ikx} - e^{-ikx}}{2i} \right) = \frac{1}{\sqrt{\pi a}} \frac{\sin(ka)}{k}
\end{align*}

آخر میں ہم اس کو دوبارہ مساوات \حوالہ{ 2.100} میں پر کرتے ہیں. 
\begin{align}
\Psi (x,t) = \frac{1}{\pi \sqrt{2a}} \int_{-\infty}^{\infty} \frac{\sin (ka)}{k} e^{i(kx-\frac{\hbar k^{2}}{2m}t)} \dif k.
\end{align}
بد قسمتی سے اس تکمل کو بنیادی تفاعل کی صورت میں حل کرنا ممکن نہیں ہے. البتہ اس کی قیمت اعدادی ترکیب سے حاصل کی جا سکتے ہے. شکل 2.8. حقیقتاً ایسی بہت کم صورتیں ہیں جہاں پہ مساوات \حوالہ{ 2.100} کا تکمل حل کرتے ہوئے  \عددی{   \Psi(x,t)  } کی قیمت صریحاً حاصل کی جائے. 
سوال 2.22 میں ایسی ایک خصوصاً خوبصورت مثال پیش کی گئی ہے.
آئیں ایک تحدیدی صورت پر غور کریں. اگر \عددی{     a  } کی قیمت بہت کم ہو تو ابتدائی تفاعل موج ایک مقام پر خوبصورت نوکیلی صورت اختیار کرتا ہے. شکل 2.9 ایسی صورت میں ہم چھوٹے زاویوں کی تخمیم استعمال کرتے ہوئے \عددی{  \sin ka \approx ka    } لکھ سکتے ہیں. لحضہ 
\begin{align*}
\phi (k)  \approx \sqrt{\frac{a}{\pi}};
\end{align*}
جو چپٹا ہے. چونکہ تمام \عددی{       k    } آپس میں کٹ جاتے ہیں. شکل 2.9. یہ عدم یقینیت کے اصول کی ایک مثال ہے. اگر ذرے کی مقام میں پھیلاؤ بہت کم ہے,  تب اس کی معیار حرکت (لحضہ \عددی{       k    } مساوات \حوالہ{ 2.96} کو دیکھیں) کی پھیلاؤ بہت زیادہ ہو گی. اس کی دوسری انتہائی مثال جہاں  \عددی{       a    } بہت بڑا ہو اور مقام کی پھیلاؤ بہت وسیع  ہو (شکل 2.10) 
\begin{align*}
\phi (k)  = \sqrt{\frac{a}{\pi}} \frac{\sin ka }{ka }
\end{align*}
اب \عددی{ \sin z/z  } کی چوٹی \عددی{ z=a  }  پر پائی جاتی ہے اور اس کی قیمت \عددی{   z=\pm\pi   }  ) جو یہاں پر \عددی{  k=\pm \pi/a } کو ظاہر کرتے ہیں ( پر صفر تک پہنچتا ہے, لحضہ زیادہ قیمت کے \عددی{  a } کیلئے \عددی{   k=0 } پر \عددی{   \phi (K)  } نوکیلی صورت اختیار کرے گا. (شکل 2.10)   اس بار ذرے کی معیار حرکت اچھی طرح معین ہے لحضہ اسکا مقام صحیح طور پر معلوم نہیں ہے.

آئیں اب اس تضاد پر دوبارہ بات کریں جسکا ہم ذکر کر چکے ہیں. جہاں مساوات \حوالہ{ 2.94} میں دیا گیا علیحدگی متغیرات سے حاصل حل \عددی{  \Psi_{k} (X, t)   } کو ذرے کی نصف رفتار سے حرکت کرتا ہے جسکو یہ تفاعل موج ظاہر کرتا ہے. حقیقت میں یہ مسئلہ وہیں پر ختم ہو گیا تھا جب ہم جان چکے کہ \عددی{    \Psi_{k}   } طبعی طور پر قابل حصول حل کو ظاہر نہیں کرتا. بحر حال آزاد ذرے کی تفاعل موج مساوات \حوالہ{ 2.100} میں سمعتی رفتار کی معلومات پر غور کرنا ایک دلچسپ کام ہے. بنیادی تصور کچھ یوں ہے. سائن نما تفاعلوں کا خطی میل جسکا حیطہ \عددی{    \phi   } ترمیم کرتا ہو,  کو موجی پلندہ کہتے ہیں. یہ غلاف میں گھرے ہوئے لہروں پر مشتمل ہوتا ہے. انفرادی لہر کی رفتار, جسکور دوری رفتار کہتے ہیں, ہر گز ذرے کئ سمتی رفتار کو ظاہر نہیں کرتا ہے. بلکہ غلاف کی رفتار, جسکو مجموعی رفتار کہتے ہیں, ذرے کئ رفتار کو ظاہر کرتی ہے. غلاف کی رفتار لہروں کی فطرت پر منحصر ہو گی. لحضہ یہ لہروں کی رفتار سے زیادہ, کم یا برابر سمتی رفتار پر حرکت کر سکتا ہے. ایک دھاگے پر امواج کی دوری رفتار اور مجموعی رفتار ایک دوسرے کے برابر ہوتی ہے. پانی کی امواج کیلئے یہ دوری رفتار کا نصف ہو گا. جیسا آپ نے جھیل میں پتھر پھینک کر دیکھا ہو گا. اگر آپ ایک مخصوس لہر پر نظر جائیں رکھیں تو آپ دیکھیں گے کہ پیچھے سے آگے برھتے ہوئے اس کی جسامت بڑھتی ہے اور سب سے آگے پہنچتے ہوئے اسکا حیطہ ختم ہو جاتا ہے. جبکہ انکا مجموعہ نصف رفتار سے حرکت کرتا ہے. یہاں میں نے دکھانا ہو گا کہ کوانٹم میکانیات میں آزاد ذرے کے تفاعل موج کی مجموعی رفتار اسکی دوری رفتار سے دگنی ہے. جو کلاسیکی ذرے کی رفتار ہو گی. یوں ہمیں درج ذیل عمومی صورت کی موجی پلندہ کی مجموعی رفتار تلاش کرنی ہوگی. 
\begin{align*}
\Psi (x,t) = \frac{1}{\sqrt{2\pi}} \int_{- \infty}^{+ \infty} \phi (k) e^{i(kx - \omega t)} \dif k.
\end{align*}
 یہاں \عددی{    \omega = (\hbar k^{2} /2m)     } ہے, لیکن جو کچھ میں کہنے جا رہا ہوں وہ کسی بھی موجی پلندہ کیلئے درست ہے, قطع نظر کئ اسکا انتشاری تعلق کیا ہے. انتشاری تعلق \عددی{  k     } کی صورت میں \عددی{     \omega } کے تفاعل کو کہتے ہیں. ہم فرض کرتے ہیں کہ کسی مخصوص قیمت \عددی{   k_{0}    } پر  \عددی{   \phi (k)    } نوکیلی صورت اختیار کرتا ہے. ہم زیادہ وسعت کا \عددی{   k_{0}    } بھی لے سکتے ہیں لیکن ایسی موجی پلندہ بہت تیزی سے اپنی شکل و صورت تبدیل کرتا ہے. چونکہ اس کے مختلف اجزاء مختلف رفتار سے حرکت کرتے ہیں. اور ایک اجتماع جو مخصوص رفتار سے حرکت کرتا ہو, بے معنی ہو جاتی ہے. چونکہ متکمل \عددی{   k_{0}    } سے دور قابل نظر انداز ہے, لحضہ ہم تفاعل \عددی{     \omega (k)   } کو اس نقطع کے گرد ٹیلر تسلسل سے پھیلا کر صرف ابتدائی اجزاء لیتے ہیں. 
\begin{align*}
\omega (k) \cong \omega_{0} + \omega_{0}^{'} (k-k_{0})
\end{align*}
 جہاں نقطہ \عددی{   k_{0}    } پر \عددی{  k     } کے لحاظ سے \عددی{     \omega } کا تفرق \عددی{   \omega_{0}^{'}   } ہے. 
 تکمل کا وسط \عددی{   k_{0}    } پر رکھنے کی خاطر ہم متغیر \عددی{  k     } کی جگہ متغیر \عددی{     s=k-k_{0} } استعمال کرتے ہیں. یوں درج ذیل ہو گا. 
\begin{align*}
\Psi (x,t) \cong \frac{1}{\sqrt{2 \pi }} \int_{- \infty}^{+ \infty} \phi (k_{0} + s) e^{i[(k_{0} +s)x-(\omega_{0} + \omega_{0}^{'}s )t]} \dif s.
\end{align*}
 وقت \عددی{     t=0  }  پر 
\begin{align*}
\Psi (x,t) = \frac{1}{\sqrt{2 \pi }} \int_{- \infty}^{+ \infty} \phi (k_{0} + s) e^{i(k_{0} +s)x} \dif s,
\end{align*}
 جبکہ بعد کے وقت پر درج ذیل ہو گا. 
\begin{align*}
\Psi (x,t) \cong \frac{1}{\sqrt{2 \pi}} e^{i(-\omega_{0}t+k_{0}\omega_{0}^{'}t)} \int_{-\infty}^{+\infty} \phi (k_{0} + s) e^{i(k_{0} + s ) ( x - \omega_{0}^{'}t)} \dif s.
\end{align*}
 ما سوائے \عددی{     x  }  کو \عددی{      (x - \omega_{0}^{'}  )} پر منتقل کرنے کے یہ \عددی{     \Psi(x,0) } کا تکمل ہے. یوں درج ذیل ہو گا. 
\begin{align}
\Psi(x,t) \cong e^{-i(\omega{0} - k_{0} \omega_{0}^{'})t} \Psi(x-\omega_{0}^{t},0).
\end{align}
 اس میں زاویائی جزروی کے علاوہ جو کسی بھی صورت \عددی{   \left| \Psi \right|^{2}   }  کی قیمت پر اثر انداز نہیں ہوتی, موجی پلندہ رفتار \عددی{   \omega_{0}^{'}   } سے حرکت کرے گا. 
\begin{align}
v_{group} = \frac{\dif \omega}{\dif k}
\end{align}
 جسکی قیمت \عددی{   k = k_{0}   } پر لی گئی ہے. آپ دیکھ سکتے ہیں کہ یہ دوری رفتار سے مختلف ہے جسے درج ذیل مساوات پیش کرتی ہے. 
\begin{align}
v_{phase} = \frac{\omega}{k}.
\end{align}
 یہاں  \عددی{     \omega = (\hbar k^{2} /2m)  } یعنی \عددی{   \omega/k = (\hbar k / 2m)   } ہے جبکہ \عددی{    \dif \omega / \dif k  = (\hbar k /m)   } ہے. جو اسکا دوگنا ہے. یہ اس بات کی تصدیق کرتا ہے کہ موجی پلندہ کی مجموعی رفتار نا کہ ساکن حالات کی دوری رفتار کلاسیکی ذرے کی رفتار دے گی. 
\begin{align}
v_{classical} = v_{group} = 2v_{phase}
\end{align}
سوال 2.18
دکھائیں کہ \عددی{    x    } کسی بھی تفاعل کو \عددی{   [ Ae^{ikx}+Be^{-ikx}]    } یا  \عددی{    [C\cos kx + D\sin kx ]   } لکھا جا سکتا ہے. مستقل \عددی{    C    } اور  \عددی{    D    } کو مستقل  \عددی{    A    } اور \عددی{      B  } کی صورت میں حاصل کریں. اسی طرح مستقل \عددی{   A   } اور  \عددی{    B    } کو مستقل  \عددی{    C    } اور \عددی{      D  } کی صورت میں حاصل کریں. کوانٹم میکانیات میں جب  \عددی{     V=0  } ہو قوت نمائی تفاعل حرکت کرتے امواج کو ظاہر کرتی ہے اور انہیں استعمال کرتے ہوئے آزاد ذرے پر تبصرہ کرنا زیادہ آسان ہوتا ہے. جبکہ  \عددی{     \sin  } اور  \عددی{    \cos   } ساکن امواج کو ظاہر کرتی ہے جو لامتناہی چکور کنواں میں قدرتی طور پر پیدا ہوتی ہے. 
سوال 2.19
مساوات 2.94 میں دی گئی آزاد ذرے کی تفاعل موج کا احتمال رو \عددی{  J   } تلاش کریں. سوال 1.14 دیکھیں. احتمال رو کے بہاؤ کا رخ کیا ہو گا. 
سوال2.20
اس سوال میں آپ کو مسئلہ پلاشرل کا ثبوت حاصل کرنے میں مدد دی جائے گی جہاں آپ متناہی وقفہ کی فوریئر تسلسل سے شروع کر کے اس وقفہ کو لامتناہی وسعت تک پھیلائیں گے. 

\begin{enumerate}[a. ]
\item 
مسئلہ درشلی کہتا ہے کہ وقفہ \عددی{  [-a, +a]   } پر کسی بھی طفاعل  \عددی{   f(x)  } کو فوریئر تسلسل کے پھیلاؤ سے ظاہر کیا جا سکتا ہے:
\begin{align*}
f(x) = \sum_{n=0}^{\infty} [ a_{n}\sin(  n\pi x/a )  + b_{n}\cos(  n\pi x/a )].
\end{align*}
کھائیں کہ اس کو درج ذیل معدل طریقہ سے لکھا جا سکتا ہے.
\begin{align*}
f(x) = \sum_{n=-\infty}^{\infty} c_{n}e^{i n \pi x /a }
\end{align*}
\عددی{  a_{n}  } اور  \عددی{   b_{n} } کی شکل میں \عددی{  c_{n}   } کیا ہو گا ؟
 \item
 فوریئر تسلسل کے مستقل کی حل سے درج ذیل حاصل کریں
\begin{align*}
c_{n} = \frac{1}{2a} \int_{-a}^{+a} f(x)e^{-in\pi x/a} \dif x
\end{align*}
\item
\عددی{  n   }  اور  \عددی{  c_{ n }   } کی جگہ نئے متغیرات \عددی{    k=( n \pi / a) }  اور  \عددی{   F(k) = \sqrt{2/\pi}ac_{n } } استعمال کریں. دکھائیں کہ اس طرح جزو الف اور ب درج ذیل روپ اختیار کرتے ہیں. 
\begin{align*}
f(x) = \frac{1}{\sqrt{2 \pi }} \sum_{n=-\infty}^{\infty} F(k)e^{ikx} \Delta k  \quad : \quad F(k) = \frac{1}{\sqrt{2\pi}} \int_{-a}^{+a} f(x) e^{-ikx} \dif x,
\end{align*}
جہاں ایک \عددی{ n    } سے اگلے \عددی{ n    } تک \عددی{  k   } میں تبدیلی  \عددی{  \Delta k   } ہو گی. 
\item
حد  \عددی{  a \rightarrow \infty   } لیتے ہوئے مسئلہ پلاشرل حاصل کریں.  \عددی{   F(k) } کی صورت میں \عددی{   f(x) } اور \عددی{   f(x) }  کی صورت میں  \عددی{   F(k) }  کے کلیات دو بالکل مختلف جگہوں سے ہوئی.  اس کے باوجود  \عددی{  a \rightarrow \infty   } کے حد میں ان دونوں کی ساخت مشابہت رکھتی ہے. 
\end{enumerate}
سوال 2.21
ایک آزاد ذرے کی ابتدائی طفاعل موج درج ذیل ہے. 
\begin{align*}
\Psi (x,0) = Ae^{ -a \left| x \right| } ,
\end{align*}
جہاں \عددی{  A   } اور  \عددی{  a   } مثبت حقیقی مستقل ہیں. 
\begin{enumerate}[a. ]
\item 
\عددی{  \Psi(x,0) } کو معمول پر لائیں. 
\item
\عددی{  \phi(k)} تلاش کریں. 
\item
\عددی{ \Psi(x,t) } کو تکمل کی صورت میں تیار کریں. 
\item
تحدیدی صورت جہاں \عددی{a   } بہت بڑا ہو اور جہاں \عددی{a   } بہت چھوٹا ہو پر تبصرہ کریں. 
\end{enumerate}
سوال2.22
قوثی موجی پلندہ. ایک آزاد ذرے کی ابتدائی تفاعل موج درج ذیل ہے. 
\begin{align*}
\Psi(x,0) = A e^{-ax^{2}}
\end{align*}
جہاں \عددی{  A } اور \عددی{  a } مستقل حقیقی اور مثبت ہیں. 
\begin{enumerate}[a.]
\item
\عددی{  \Psi(x,0) } کو معمول پر لائیں. 
\item
\عددی{  \Psi(x,t) } تلاش کریں. اشارہ: مربع مکمل کرنے سے درج ذیل صورت کے تکمل با آسانی حل ہوتے ہیں. 
\begin{align*}
\int_{-\infty}^{+\infty} e^{-(ax^{2} +bx)} \dif x
\end{align*}
\عددی{  y \equiv \sqrt{a} [ x + (b/2a)] } لے کر دیکھئے گا کہ \عددی{  ( ax^{2} + bx ) = y^{2} - (b^{2}/4a) } ہو گا. جواب : 
\begin{align*}
\Psi(x,t) = \left( \frac{2a}{\pi} \right) ^{1/4} \frac{e^{-ax^{2}/[1+(2i\hbar at/m)]}}{\sqrt{1+(2i\hbar at/m)}}.
\end{align*}
\عددی{  \left| \Psi(x,t) \right|^{2} } تلاش کریں. اپنا خواب درج ذیل قیمت کی صورت میں لکھیں. 
\begin{align*}
\omega = \sqrt{\frac{a}{1+(2\hbar at/m)^{2}}}.
\end{align*}
نقطہ \عددی{ t = 0 } پر \عددی{ \left| \Psi \right|^{2} } کا خاکہ  \عددی{ x } کے لحاظ سے کھینچیں. کسی بڑے \عددی{ t } پر دوبارہ خاکہ کھینچیں. وقت \عددی{ t = 0 } گزرنے کے ساتھ  \عددی{ \left| \Psi \right|^{2} } کو کیا ہو گا ؟
\item 
\عددی{\langle x \rangle , \langle p \rangle , \langle x^{2} \rangle , \langle p^{2} \rangle , \sigma_{x}  } اور \عددی{ \sigma_{p}} تلاش کریں. جزوی جواب: \عددی{ \langle p^{2} = a\hbar^{2}}, ہاں جواب کو  اس سادہ صورت میں لانے کیلئے آپ کو کافی الجبرا کرنی ہو گی. 
\item
کیا عدم یقینیت کا اصول یہاں کار آمد ہے ؟ کس لمحہ \عددی{ t } پر یہ نظام عدم یقینیت کے اصول کے قریب تر ہو گا ؟
\end{enumerate}
