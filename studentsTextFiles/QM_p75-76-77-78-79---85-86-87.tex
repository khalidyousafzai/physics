چونکہ کسی مخصوص مقام پر ذرے کی موجودگی کا احتمال \عددی{\abs{\psi^2}} ہوتا ہے لہٰذہ آمدی ذرے کے انعکاس کا نسبتی احتمال درج ذیل ہوگا۔ 
\begin{align}
R=\frac{\abs{B}^{2}}{\abs{A}^{2}}=\frac{\beta^{2}}{1+\beta^{2}}
\end{align}
\عددی{R} کو شرح انعکاس کہتے ہیں۔اگر آپ کے پاس ذرات کی ایک شعاع ہو یہ تفاعل آپ کو بتاۓ کا کہ ان میں سے کتنے واپس لوٹ کر آئیں گے۔ ترسیل کا احتمال درج ذیل ہوگا جسے شرح ترسیل کہتے ہیں۔ 
\begin{align}
T=\frac{\abs{F}^{2}}{\abs{A}^{2}}=\frac{\beta^{2}}{1+\beta^{2}}
\end{align}
ظاہر ہے ان احتمال کا مجموعہ ایک ہونا چاہیئے جیسا کہ اس مساوات میں ہے۔
\begin{align}
R+T=1
\end{align}
دھیان رہے کہ \عددی{R} اور \عددی{T}،  \عددی{\beta} کے تفاعل ہیں۔ لہٰذہ  \عددی{E} بھی اسی کا تفاعل ہوگا۔
\حوالہء{مساوات 2.130}، \حوالہء{مساوات2.135} 
\begin{align}
R=\frac{1}{1+(2h^{2}E/m\alpha^{2})}\quad T=\frac{1}{1+(m\alpha^{2}/2h^{2}E)}
\end{align}
ذیادہ توانائ ترسیل کا ااحتمال بڑھاتی ہے ، جیسا کہ ظاہری طور پر ہونا چاہیے۔ یہاں تک سب کچھ ٹھیک ہے لیکن ایک اصولی مسئلہ ہے جسے ہم نظر انداز نہیں کر سکتے۔چونکہ بکھراؤ موج کے تفاعل معمول پر لانے کے قابل نہیں ہیں ۔ لہٰذہ یہ کسی بھی صورت ایک حقیقی ذرے کے حال کو بیان نہیں کر پاتے۔ لیکن ہم اس مسئلے کا حل جانتے ہیں ہمیں ساکن حالات کے ایسے خطی جوڑ تیار کرنے ہوں گے جو معمول پر لاۓ جانے کے قابل ہوں جیسا ہم نے آزاد ذرے کے لیے کیا تھا، حقیقی طبعی ذرات کو یوں تیار کردہ موجی پرندا 

%3.04


%75  76  77  78  
چونکہ کسی مخصوص مقام پر زدے کی موجودگی کا احتمال \عددی{\psi^2} ہوتا ہے لہٰذا عمودی زرے کے انعکاس  کا نسبتی احتمال درج ذیل ہوگا\[R=\frac{\abs B^{2}}{\abs A^{2}}=\frac{\beta^{2}}{1+\beta^{2}}\]
\عددی{ R  }کو شرح انعکاس کہتے ہیں۔ اگر آپ کے پاس زرات کی ایک شعاع ہو یہ تفال آپ کو بتائے گا کہ ان میں سے کتنے واپس لوٹ کر آئیں گے۔ ترسیل کا  احتمال درج ذیل ہوگا جسے شرع ترسیل کہتے ہیں\[T=\frac{\abs F^{2}}{\abs A^{2}}=\frac{1}{1+\beta^{2}}\]
 ظاہر ہے ان احتمال کا مجموعہ ایک \عددی{  1 } ہونا چاہیے جیسا کہ ہے \[R+T=1\]
 دھیان رہے کہ \عددی{ R  } اور  \عددی{  T } متغیر \عددی{ \beta  } کے تفاعل  ہیں لہذا \عددی{ E  } بھی اسی کا تفال ہوگا.
  \[R=\frac{1}{1+(\frac{2\hbar^{2}E}{m\alpha^{2}})}, T =\frac{1}{1+(\frac{m\alpha^{2}}{2\hbar^{2}E})}\]  زیادہ  توانائی ترسیل کا احتمال بڑھاتی ہے جیسا کہ ظاہری طور پر ہونا چاہیے۔  یہاں تک باقی سب ٹھیک ہے لیکن ایک اصولی مسئلہ ہے جسے ہم نظرانداز نہیں کر سکتے ہیں.  چونکہ بکھراؤ  موج کے  تفاعل معمول پر لانے کے قابل نہیں ہیں لہذا یہ کسی صورت بھی حقیقی  ذرے کے حال کو ظاہر نہیں کر سکتے ہیں لیکن ہم اس مسئلے کا حل جانتے ہیں ہمیں ساکن حالات کے ایسے خطی جوڑ تیار کرنے ہونگے جو معمول پر لائے جانے کے قابل ہوں جیسا ہم نے آزاد زرے  کے لیے کیا تھا.  حقیقی طبی ذرات کو یوں تیار کردہ موجی پلندہ ظاہر کرے گا.  یہ ظاہری طور پر سیدھا سادہ اصول عملی استعمال میں پیچیدہ ثابت ہوتا ہے لہذا یہاں سے آگے مسئلے کو کمپیوٹر کی مدد سے حل کرنا بہتر ہوگا.  چونکہ  توانائی کی قیمتوں کا پورا سلسلہ استعمال کیے بغیر آزاد زرے کی تفال موج کو معمول پر نہیں لایا جا سکتا ہے لہذا  \عددی{ R  } اور  \عددی{  T }
  کو \عددی{ E  } کے قریب ذرات کی تخمینی شرع انعکاس   اور شرح ترسیل سمجھنا چاہیے.  یہ ایک عجیب بات ہے کہ ہم لب لباب وقت کے تابع مسئلہ کو جہاں ایک عمدی زرہ مخفی گو سے  بکھر کر لامتناہی کی طرف رواں ہوتا ہے پر غور ثاقل  حالات استعمال کرتے ہوئے کر پائے ہیں. آخر کار \حوالہء{   مساوات2.13 } اور \حوالہء{   مساوات2.132} میں\عددی{  \psi }  وقت کے تابع مخلوط \عددی{  \psi }  نماتفاعل ہے۔ جو مستقل ہیتا کے ساتھ دونوں اطراف لا متناہی تک  پھیلا ہوا ہے۔ اس کے باوجود اس تفاعل پر موزوں سرحدی شرائط مسلط کرتے ہوۓ ہم  ایک زرہ جسے مقامی موجی پلندہ سے ظاہر کیا گیا کی مخفی کو سے انعکاس یا ترسیل کا احتمال تعین کر پاۓ ہیں۔ اس ریاضیاتی کرامت کی وجہ میرے خیال میں یہ حقیقت یے کہ  ہم فضا میں پھیلے ہوۓ تفاعل موج جن کی وقت کی تابعیت نہ ہونے کے برابر ہے کے خطی جوڑ لے کر ایک حرکت پزیر نقطہ کے گرد ایسا تفال موج تیار کر سکتے ہیں جس پر  وقت کے لحاظ سے تفصیلًا غور کیا جا سکتا ہے(\حوالہء{   سوال2.43 })
متعلقہ مساوات جانتے ہوۓ آئیں ڈیلٹا تفاعل  رکاوٹ کے مسئلہ پر غور کریں \حوالہء{   شکل2.16 }
ہمیں صرف \عددی{  \alpha  }  کی علامت تبدیل کرنی ہوگی۔ ظاہر ہے یہ تحدیدی حال کو ختم کرتا ہے (\حوالہء{   سوال 2.2 })
اس کے برعکس شرع انعکاس اور شرع ترسیل جو \عددی{  \alpha^2  } پر منحصر ہوتے ہیں تبدیل نہیں ہوتے کتنی عجیب بات ہے کہ ایک زدہ ایک رکاوٹ کے اوپر سے یا ایک کنواں  کے اوپر سے ایک ہی جیسے آسانی کے ساتھ گزرتا ہے.  کلاسیکی طور پر جیسا کہ آپ جانتے ہیں ایک زدہ کبھی بھی لا متناہی  قد کے رکاوٹ کو عبور نہیں کرسکتا چاہے اس کی توانائی جتنی ہی کیوں نہ ہو.   در حقیقت کلاسیکی مسئلہ بکھراؤ غیر دلچسپ ہوتے ہیں۔ اگر \عددی{  E\textgreater V_{max}  } بلند تر ہو تب \عددی{ R =0 } اور \عددی{  T=1 } ہوگا۔ زرہ ہر صورت رکاوٹ عبور کر پاۓ گا۔ اگر \عددی{ E\textless V_{max} }   ہو تب \عددی{  T=0 } اور \عددی{ R =1 } ہوگا۔ یہ پہاڑی پر وہاں تک چڑھے گا جہاں تک اس میں دم ہو اور اس کے بعد واپس اتر کر واپسی راستےچلتا جاۓ گا. کوانٹائی بکھراؤ زیادہ دلچسپ ہوتے ہیں.  اگر \عددی{ E\textless V_{max} }  بلند تر ہو تب بھی ایک زرے کا مخفی کو کے اندر سے گزرنے کا احتمال خیر صفر ہوگا۔ اس مظہر کو ہم سرنگ زنی کہتے ہیں جس پر جدید برقیات کا بیشتر حصہ منحصر ہے اور جو خوربین میں حیرت کن ترقی کے پشت پر ہے اس کے برعکس اگر {  E\textgreater V_{max}  }  بلند تر ہو تب بھی زرے کے انعکاس کا غیر صفر احتمال ہوگا. اگرچہ میں آپ کو کبھی بھی مشورہ نہیں دوں گا کہ چھت سے نیچے کودیں اور توقع رکھیں کہ کوانٹم میکانیات آپ کی جان بچا پاۓ گا. یہاں سوال \حوالہء{   سوال 2.35 } دیکھیے گا 
\ابتدا{سوال}
درج ذیل تکملات کی قیمتیں تلاش کریں.
\begin{enumerate}
\item\(\int_{-3}^{+1}(x^{3}-3x^{2}+2x-1) \delta(x+2) \dif{x}\)
\item\( \int_{0}^{\infty}[\cos(3x)+2] \delta(x-\pi) \dif{x}\)
\item\(\int_{-1}^{+1}\exp(\abs x+3) \delta (x-2) \dif{x}\)
\end{enumerate}
\انتہا{سوال}
\ابتدا{سوال}
 ڈیلٹا تفال زیر علامت تکمل رہتے ہیں اور دو فکرے \عددی{ D_{1}(x) } اور \عددی{ D_{2}(x) }جو ڈیلٹا تفال پر مبنی ہیں صرف درج صورت میں ایک دوسرے کے برابر ہوں گے  \[\int_{-\infty}^{+\infty}f(x) D_1(x) \dif(x)=\int_{-\infty}^{+\infty}f(x D_2 \dif(x)\] جہاں  \عددی{f(x)} کوئی بھی سادہ تفال ہو سکتا ہے  
\begin{enumerate}
\item درج ذیل دکھائیں
 \[\delta(cx)=\frac{1}{\abs c}\delta(x)\]
جہاں \عددی{c } ایک حقیقی مستقل ہے. \عددی{c-}  کی صورت کو بھی دیکھئے گا 
\item    سیڑھی تفال \عددی{\theta(x)}  درج ذیل ہے 
\[\theta(x)=\{^{1, if x>0.}_{0,if x<0.}\]
 اس نایاب صورت میں جب یہ جاننا ضروری ہو ہم \عددی{\theta(0)} کی تعریف \عددی{\frac{1}{2}} کرتے ہیں دکھائے گا  \(\dif{\theta}/\dif{x}=\delta(x)\)
\end{enumerate}
\انتہا{سوال}
\ابتدا{سوال}
 عدم یقینیت کے اصول کو \حوالہء{   مساوات 2.129}کی تفال موج کے لئے پرکھیں اشارہ چونکہ \عددی{  \psi } کی تفرق کا\عددی{x=0} عدم استرار پایا جاتا ہے لہذا\عددی{p^2}  کا حساب پیچیدہ ہوگا. 
 سوال\حوالہء{   سوال2.24 بی } کا نتیجہ استعمال کریں\عددی{p^{2}=(m\alpha/\hbar)^{2}} 
\انتہا{سوال}
\ابتدا{سوال}
 تفال ڈیلٹا ایکس \عددی{\delta(x)} کا فوریر تبادل کیا ہوگا؟ مسئلہ پلینچرل استعمال کرتے درج ذیل دکھائیں 
\[\delta(x)=\frac{1}{2\pi}\int_{-\infty}^{+\infty} e^{ikx} \dif{k}\]
 یہ کلیہ دیکھ کر کسی بھی عزت مند ریاضی دان پر حملہ ضرور ہوگا اگرچہ صاف ظاہر ہے کہ\عددی{x=0} کی صورت میں یہ متکمل  لامتناہی ہے یہ\عددی{x\neq 0} کی صورت میں صفر کو یا پھر کسی اور چیز کو مرکوز نہیں ہوتا ہے بلکہ متکمل ہمیشہ کے لئے ارتیاش پزیر رہتا ہے۔ 
 ہم \عددی{-l} سے \عددی{+l} تک تکمل لے کر کہہ سکتے ہیں کہ مساوات \حوالہء{   مساوات 2.144} کی صورت میں اس متناہی تکمل کی اوسط قیمت ہے. مسئلہ لے شرل کی بنیادی شرط مربع تکملیت ہے جس پر ڈیلٹا تفاعل پورا نہیں اترتا.  یہ اس مسئلہ کی جڑ ہے اس کے باوجود اگر مساوات \حوالہء{   مساوات 2.144}  کو سنبھل کے استعمال کیا جائے یہ بہت کارآمد ثابت ہو سکتا ہے
\انتہا{سوال}
\ابتدا{سوال}
درج ذیل جڑواں ڈیلٹا تفال مخفی کو پر غور کریں جہاں\عددی{  \alpha }  اور \عددی{  a }  مثبت کو مستقبل ہیں
\[V(x)=-\alpha[\delta(x+\alpha)+\delta(x-\alpha)]\]
\begin{enumerate}
\item اس مخفی کو کا خاکہ کھینچیں
\item  یہ کتنی مقید حالات پیدا کرتا ہے\عددی{\alpha=\hslash/ma } اور \عددی{\alpha=\hslash^2/4ma }کیلئے ایجادتی توانائیاں تلاش کرے اور تفال موج کا خاکہ کھینچیں
\end{enumerate}
 \انتہا{سوال}
\ابتدا{سوال}
جوڈی ڈیلٹا تفاعل کے مخفی کو سوال\حوالہء{   سوال 2.27} کے لیے شرع ترسیل تلاش کریں. 
 \انتہا{سوال}
%#############-------DONE TILL HERE----------###########
%   78   79   70  81  82  83  84  
%FINITE SQUARE WELL
 ہم آخری مثال کے طور پر متناہی چوکور کنواں کی مخفی کو لیتے ہیں 
\[\alpha=\hbar^{2}/4m\alpha\]\\
\[ V(x)= \begin{cases} 
     -V_{0} & -a\le x\le a \\
      0 & \abs{x}\textgreater a
   \end{cases}
\]
جہاں\عددی{ V_{0} } ایک مثبت مستقل ہے. \حوالہء{ شکل 2.17 }
 ڈیلٹا تفاعل کنواں کی طرح یہ مخفی کو بھی مقید حالات کے ساتھ ساتھ جہاں\عددی{  E\textless 0 }  ہوگا بکھراؤ حالات بھی پیدا  کرتا ہے جہاں\عددی{  E\textgreater 0 } ہوگا۔
ہم پہلے مقید حالات پر غور کرتے 
 خطہ\عددی{  x\textless -a }  جہاں مخفی کو صفر ہے شروڈنگر کی مساوات درج زیل روپ اختیار کرتی یے
\[-\frac{\hbar^{2}}{2m} \frac{d^{2} \Psi}{\dif{x^{2}}}=E\Psi,\] 
یا
 \[ \frac{d^{2}\Psi}{\dif{x^{2}}}=K^{2}\Psi\]
جہاں \[K=\frac{\sqrt{-2mE}}{\hbar}\] حقیقی اور مثبت ہے۔  اس کا عمومی حل\عددی{\Psi(x)=A \exp(-kx)+ B exp(kx) }  ہے لیکن          \عددی{ x\apto -\infty } کے صورت میں اس کا پہلا جز لامتناہی تک بڑھتا ہے۔ 
 یوں پہلی کی طرح مساوات\حوالہء{ مساوات 2.119   } دیکھیے گا۔
 طبی طور پر قابل قبول حل درج ذیل ہوگا
\[ Psi(x)=Be^{kx}, for x<-a\]

 خطہ\عددی{  -a\textless x \textless a }  میں\عددی{ V(x)=-V_{0} } ہوگا جہاں شروڈنگر کی مساوات درج زیل روپ اختیار کرے گی\عددی{ V(x)=-V_{0} }  یا \[\frac{\dif ^{2}\psi}{\dif{x^2}}=-l^2\psi \]
 جہاں\[l=\frac{\sqrt{2m(E+V_{0})}}{h}  \]
 اگرچہ مقید حالات کے لئے \عددی{ E } منفی ہوتا ہے اسے \عددی{ -V_{0} } سے بڑا ہونا ہوگا تاکہ \عددی{  V_{min} } کمتر ہو۔  سوال\حوالہء{ سوال 2.2   }  دیکھیے گا. 
 لہذا \عددی{  l } بھی حقیقی اور مثبت ہوگا۔ اس کا عمومی حل\[\psi(x)=C\sin(lx)+D\cos(lx) \]  ہوگا. \عددی{-a\textless x \textless a } کے لئے * جہاں\عددی{  C } اور\عددی{ D } اختیاری مستقل ہیں۔  آخر میں خطہ \عددی{x\greater a}  جہاں ایک بار پھر مخفی کو صفر ہے عمومی حل\عددی{ \psi(x)=F exp (-kx)+G exp(kx) } ہوگا لیکن یہاں\عددی{ x\apto \infty} کی صورت میں دوسرا جز  لامتناہی تک بڑھتا ہے لہذا قابل قبول حل درج ذیل ہوگا \عددی{ \psi(x)=F exp (-kx)}۔
%#############-------DONE TILL HERE----------###########
% AUDI 4;07

 اگلے قدم میں ہمیں سرحدی شرائط مسلط کرنے ہوں گے۔ عددی{ \psi }اور عددی{ \frac{\dif{\psi}}{\dif{x}}  } عددی{ -a  } اور عددی{  +a }  پر استمراری ہے. یہ جانتے ہوئے کہ یہ مخفی کو جفت تفاعل ہے ہم کچھ وقت بچا سکتے ہیں اور ہم فرض کر سکتے ہیں کہ حل مثبت یا تاک ہوںگے سوالحوالہء{ سوال 2.1ج   }
اس کی افادیت یہ  ہے کہ ہمیں سررحدی شرائط صرف ایک طرف مثلًا عددی{ +a  } پر مسلط کرنے ہونگے۔
 چونکہ عددی{ \psi(-x)=\pm \psi(x)  } ہے لہٰزا دوسرا  طرف ہمیں خودبخود حاصل ہو جائے گا. میں جفت حل حاصل کرتا ہوں جبکہ آپ کو سوال حوالہء{  سوال 2.29  }میں تاک  حل تلاش کرنے ہونگے. عددی{ \sin}جفت ہے جبکہ عددی{ \cos} تاک ہے. لہٰزا میں درج زیل حروف کے حل کی تلاش میں ہوں. 
\[\psi(x)={
\begin{case}
Fe^{-kx} & x\textgreater a\\
D\cos(lx) & 0\textless x \textless a\\
\psi(-x) & x\textless 0
\end{cases}\]
نقطہ عددی{ x=a  } پر سای کی استمرار کہتی ہے کہ 
\[ Fe^{-ka}=D\cos(la) \]
 اورعددی{\frac{\dif{\psi}}{\dif{x}}   } کی استمرار کہتی ہے کہ
 \[-kFe^{-ka}=-lD\sin(la) \] 
 مساوات حوالہء{ مساوات 2.153   } کو مساوات حوالہء{  مساوات 2.152  } سے تقسیم کرتے ہوئے درج زیل حاصل ہوگا

\[k=l\tan(la)  \]
 چونکہ \عددی{ k  }  اور \عددی{  l } دونوں \عددی{  E } کے تفال ہیں لہٰزا یہ کلیہ اجازتی توانائیاں دے گا. \عددی{ E  }  کے لئے حل کرنے سے پہلے ہم کچھ بہتر علامتیں متعارف کرتے ہیں\عددی{ z=la  } اور \عددی{ z_{0}=\frac{a}{\hbar}\sqrt{2mV_{0}}  }
مساوات \حوالہء{  مساوات 2.146 } اور \حوالہء{   مساوات 2.148 }کے تحت \عددی{ (k^{2}+l^{2})=2mV_{0}/\hbar^{2}  } اور ہوگا لہٰزا \عددی{ ka=\sqrt{z_{0}^{2}-z^{2}}  } ہوگا اور مساوات 
\حوالہء{  مساوات 2.154  } درج زیل روپ اختیار کرے گی.
\[\tan z=\sqrt{(z_{0}/z)^{2}-1} \]
* یہ \عددی{ z  } لہٰزا \عددی{ E  } کی ماورائی مساوات ہے جس کا متغیر \عددی{ z_{0}  } ہے. جو کنواں کی جسامت کا ناپ ہے. اس کو اعدادی طریقہ سے کمپیوٹر کے زریعے حل کیا جا سکتا یا \عددی{ \tan z  } اور \عددی{ \sqrt{(z_{0}/z)^{2}-1}  } کو ایک ہی جال پر ترسیم کرتے ہوئے ان کے نکتہ تکاتے لیتے ہوئے  حل کیا جا سکتا ہے۔
 شکل \حوالہء{  شکل 2.18  }   دو تحدیدی صورتیں زیادہ دلچسپی کے حامل ہیں ۔ 
\begin{enumerate}
\item ایک چوڑا گہرا کنواں
 اگر \عددی{ z_{0}  } بڑا ہو تب\عددی{ z_{n}=n\pi/2  } کے لئے\عددی{ n  } نکتہ تکاتے سے ذرا سا نیچے ہوگا یوں درج زیل ہوگا
\[E_{n}+V_{0}=\frac{n^{2}\pi^{2}\hbar^{2}}{2m(2a)^{2}} \]
اب\عددی{E+V_{0}   } کنواں کی تہہ سے اوپر توانائی کو ظاہر کرتی ہے اور مساوات کا دایاں ہاتھ ہمیں ایسے لامتناہی چوکور کنواں کی توانائیاں دیتا ہے جس کی چوڑائی \عددی{ 2a  } ہو۔
 مساوات\حوالہء{مساوات 2.27} دیکھیں بلکہ \عددی{ n  } تاک ہونے کی بنا یہ کل توانائیوں کا آدھا ہمیں دیتی ہے جیسا آپ سوال \حوالہء{سوال 2.29} میں دیکھیں گے کل توانائیوں کی باقی نصف تعداد تاک تفاعل موج سے حاصل ہوگی۔ 
 یوں\عددی{ V_{0}\apto \infty } کرتے ہوئے متناہی چوکور کنواں سے لامتناہی چوکور کنواں حاصل ہوگا البتہ کسی بھی متناہی \عددی{ V_{0}  } کی صورت میں مقید حالات کی تعداد متناہی ہوگی۔ 
\item  کم گہرائی کا کنواں 
 جیسے جیسے\عددی{ z_{0}  } کم کیا جاتا ہےمقید حالات کی تعداد کم سے کم ہوتی جاتی ہے  حتی کہ آخر کار \عددی{ (z_{0}\textless \pi/2)  } کیلئےجہاں کمتر تاک حال ختم ہوتا ہےصرف ایک  مقید حال رہ جائے گا۔   ایک دلچسپ بات یہ ہے کہ چاہے کنواں کتنا ہی کمزور کیوں نہ ہو جائے یہ ہر صورت ایک عدد مقید حال دے گا۔ 

\item
\end{enumerate}





 اگر آپ 
\عددی{ \psi  }\حوالہء{  مساوات 2.151  } کو معمول پر لانے میں دلچسپی رکھتے ہیں سوال \حوالہء{  سوال 2.30  } تو کیجئے گا جب کہ میں اب بکھراؤ حالات \عددی{ E\textgreater 0  } کی طرف پڑھتا ہوں بائیں ہاتھ جہاں\عددی{ V(x)=0  } ہے درج ذیل ہوگا 
\[\psi(x)=Ae^{ikx}+Be^{ikx} \]
\عددی{ (x\textless -a)  } جہاں عموما کی طرح درج ذیل ہوگا\[k=\frac{\sqrt{2mE}}{\hbar} \]  کنواں کے اندر جہاں\عددی{ V(x)=-V_{0}  }  ہے
\[\psi{x}=C\sin(lx)+Dcos(lx) \]
 \عددی{(-a\textless x\textless a)   } جہاں پہلے کی طرح درج ذیل ہوگا \[l\equiv \frac{\sqrt{2m(E+V_{0})}}{\hbar} \]  دائیں جانب جہاں ہم فرض کرتے ہیں کہ کوئی عمودی موج نہیں پائی جاتی  درج ذیل ہوگا \[\psi(x)=Fe^{ikx} \] یہاں عمودی  ہیتا\عددی{ A  } ہے اور انعکاسی ہیتا\عددی{ B  }  ہے اور ترسیلی ہیتا \عددی{  F }  ہے چار سرحدی شرائط پائے جاتے ہیں نقطہ *پر* کے تحت استمرار درج ذیل ہوگا *  
نقطہ\عددی{ -a  }  پر\عددی{ \psi(x)  } استمرار درج ذیل دیتی ہے
\[Ae^{-ika}+Be^{ika} = -C\sin(la)+D\cos(la) \]

 مثبت\عددی{ +a  } پر \عددی{ \psi(x)  }کی استمرار درج ذیل دیتی ہے 
\[[C\cos(la)-D\sin(la)]=ikFe^{ika} \]
اور \عددی{ +a  } پر\عددی{\farc{\dif{\psi}}{\dif{x}}   } کے تحت درج ذیل ہوگا\[C\sin(la)+D\cos(la)=Fe^{ika} \]
 ہم ان میں سے \عددی{ C  }  اور \عددی{ D  } کو خارج کر کے باقی دو کو حل کرکے \عددی{ B  }اور \عددی{ F  } تلاش کر سکتے ہیں۔  سوال \حوالہء{ سوال 2.32   }دیکھیے گا
\[B=i\frac{\sin(2la)}{2kl}(l^{2}-k^{2})F \]
\[F=\frac{e^{-2ika}A}{\cos(2la)-i\frac{(k^{2}+l^{2})}{2kl}\sin(2la)} \]
  شرح ترسیل\عددی{ (T=\abs{F^{2}}/\abs{A^{2}})  }  اصل متغیرات کی صورت میں درج ذیل ہوگا

\[T^{-1}=1+\frac{V_{0}^{2}}{4E(E+V_{0})}\sin^{2}\big(\frac{2a}{\hbar}\sqrt{2m(E+V_{0})} \big)  \]
 دیکھیے گا جہاں بھی سائن کی قیمت صفر ہو وہاں\عددی{T=1   } ہوگا یعنی کنواں شفاف ہو گا یعنی جب\[\frac{2a}{\hbar}\sqrt{2m(E_{n}+V_{0})}=n\pi \] جہاں \عددی{ n  }  عدد صحیح ہے۔ یوں مکمل ترسیل کے لیے درکار توانائی درج ذیل ہوگی\[E_{n}+V_{0}=\frac{n^{2}\pi^{2}\hbar^{2}}{2m(2a)^{2}} \]۔
 جو لامتناہی چوکور کنواں کی ایجادتی توانائیاں ہیں۔  شکل\حوالہء{  شکل2.19  }میں توانائی کے لحاظ سے \عددی{T   } کو ترسیم کیا گیا ہے۔




\ ابتدا{سوال} 
متناہی چوکور کنواں کے تاک مقید حال کی تفال  موج کا تجزیہ کریں۔ اجازتی  توانائیوں کی ماروائی مساوات اخذ کر کے اسے ترسیمی طور پر حل کریں اس کی دونوں تحدیدی صورتوں پر غور کریں۔ 
 کیا ہر صورت ایک تاک مقید حال پایا جائے گا ؟ 

\انتہا{سوال}
\ابتدا{سوال}
 مساوات\حوالہء{  مساوات 2.151  } میں دیا گیا \عددی{ \psi(x)  } کو معمول پر لاکر مستقل\عددی{  D }  اور\عددی{  F } کا تعین  کریں۔
\انتہا{سوال}
\ابتدا{سوال}
ڈیراک ڈیلٹا تفاعل کوایک ایسے مسطتیل کی  تحدیدی صورت تصور کیا جاسکتا ہے جس کا  رقبہ ایک \عددی{ 1  } رکھتے ہوئے اس کی چوڑائی صفر کے قریب سے قریب اورقد لامتناہی کے قریب سے قریب کرتے ہوئے حاصل ہو۔ 
دکھائیں کہ اگرچہ ڈیلٹا تفال کنواں مساوات\حوالہء{  مساوات 2.114  } لامتناہی گہرا ہے۔  یہ \عددی{z_{0}\apto 0   }کی بنا ایک کمزور مخفی کو ہے۔ 
 ڈیلٹا تفاعل مخفی کو کو  متناہی چکور کنواں کی تحدیدی  صورت لیتے ہوئے اس کی مقید حال کی توانائی تعین کریں۔ تصدیق کریں کہ آپ کا جواب مساوات\حوالہء{  2.129  } کے مطابق ہے۔ 
 دکھائیں کہ موزوں حد کی صورت میں مساوات\حوالہء{    مساوات 2.169} کی تخفیف مساوات\حوالہء{   مساوات 2.141 } دے گی 
\انتہا{سوال}
\ابتدا{سوال}
مساوات \حوالہء{ مساوات 2.167} اور \حوالہء{ مساوات 2.168   } آخز کریں۔  اشارہ مساوات \حوالہء{  مساوات 2.165 }اور\حوالہء{  مساوات 2.166   }اور \عددی{ C  } اور \عددی{  D }  کے لئے \عددی{ F  } کی صورت میں حل کریں
\[C=\Big[\sin(la)+i\frac{k}{l}\cos(la)   \Big]e^{ika}F ; D=\Big[\cos(la)-i\frac{k}{l}\sin(la)  \Big]e^{ika}F \]
 انہیں واپس مساوات \حوالہء{   مساوات 2.163 } اور \حوالہء{  مساوات 2.164  } میں پر کریں شرح ترسیل حاصل کرکے مساوات\حوالہء{  مساوات 2.169  } کی تصدیق کریں۔
\انتہا{سوال}
\ابتدا{سوال}
مستطیلی رکاوٹ جو خطہ\عددی{ -a\textless x\textless a  }  میں\عددی{v_{x}=+V_{0}\textgreater 0   } لینے سے مساوات \حوالہء{مساوات 2.145    } دیتا ہے کی شرح ترسیل  تعین کریں۔ 
تین صورتیں \عددی{ E\textless V_{0}  } ،
\عددی{ E=V_{0}  }اور \عددی{ E\textgreater v_{0}  } کو علیحدہ علیحدہ حل کریں آپ دیکھیں گے کہ رکاوٹ کے اندر تینوں صورتوں میں تفال موج ایک دوسرے سے مختلف ہو گا۔ 
 جزوی جواب\عددی{ E\textless V_{0}  }  کے لئے
\[T^{-1}=1+\frac{V_{0}^{0}}{4E(V_{0}-E)}\sinh^{2}\Big(\frac{2a}{\hbar}\sqrt{2m(V_{0}-E)}   \Big) \]

\انتہا{سوال}
\ابتدا{سوال}
 درج ذیل سیڑھی مخفی کو پر غور کریں
\[V(x)={
\begin{cases}
0 & x\le 0\\
V_{0}&x\textgreater 0
\end{case} \]

\begin{enumerate}
\item  صورت \عددی{ E\textless V_{0}  } کیلئے شرح انعکاس کا حساب لگائیں اپنے جواب پر تبصرہ کریں 
\item
\عددی{E\textgreater V_{0}   } کی شرح انعکاس کا حساب لگائیں  
\item
ایسا مخفی کو جو رکاوٹ کے دائیں طرف واپس صفر کی طرف نہیں جاتا، میں ترسیلی موج کی رفتار مختلف ہوگی۔  لہذا اس صورت شرح ترسیل\عددی{ \abs{F}^{2}/\abs{A}^{2}  }  نہیں ہوگا جہاں\عددی{A   }   عمودی ہیتا  اور \عددی{  F }ترسیلی ہیتا ہے۔ دکھائیں  کہ\[ T=\sqrt{\frac{E-V_{0}}{E}  }\frac{\abs{F}^{2}}{\abs{A}^{2}} \]
\عددی{ e\textgreater V_{0}  }کیلئے درج ذیل ہوگا۔
 اشارہ آپ اسے مساوات\حوالہء{ مساوات 2.98   } سے حاصل کر سکتے ہیں یا زیادہ خوبصورتی کے ساتھ اور کم معلومات حاصل کیے بغیر احتمال رو سے حاصل کر سکتے ہیں سوال\حوالہء{  سوال 2.19  } صورت\عددی{ E\textless v_{0}  } کی صورت میں \عددی{ T  }  کیا ہوگا؟
\item

صورت \عددی{  E\textgreater v_{0} } کے لیے سی ڈی مخفی کو کے لئے شرح ترسیل تلاش کرکے \عددی{T+R=1   } کی تصدیق کریں۔
\end{enumerate}
\انتہا{سوال}
\ابتدا{سوال}
ایک ذرہ جس کی کمیت \عددی{ E\textgreater 0  } ہو اور حرکی توانائی \عددی{ V_{0}  } ہو ایک اچانک گہرائی کے مخفی توانائی شکل\حوالہء{  شکل 2.34  } کی طرف بڑھتا ہے۔ 

\begin{enumerate}
\item  صورت\عددی{ E=V_{0}/3  }   کی صورت میں اس کی انعکاس کا احتمال کیا ہوگا؟
 اشارہ یہ بالکل سوال  \حوالہء{  سوال 2.34  }کی طرح ہے بس یہاں سیڑھی  اوپر کے بجائے نیچے کی طرف ہے ۔
\item 
 میں نے شکل کو یوں بنایا ہے جیسے ایک گاڑی ایک چٹان سے نیچے گرنے والی ہے البتہ گاڑی کی انعکاس کا احتمال جزالف کے نتیجہ سے بہت کم ہو گا ۔یہ مخفی کو کیوں ایک چٹان کی صحیح ترجمانی نہیں کرتا ہے؟
 اشارہ شکل\حوالہء{  شکل 2.20  } میں جیسے ہی گاڑی\عددی{ x=0  }  پر سے گزرتی ہے اس کی توانائی عدم استمرار کے ساتھ گر کر\عددی{-V_{0}   }  ہوجاتی ہے۔ کیا ایک گرتے ہوئے گاڑی کے لیے ایسا ہی ہوگا ؟
\item
\end{enumerate}
\انتہا{سوال}
%   85 86 87 
\ابتدا{سوال}
عین امکدہ پر\عددی{-a\textless x\textless +a   }  کے بیچ لامتناہی چوکور کنواں پایا جاتا ہے۔ جس کے اندر\عددی{ V(x)=\infty  } اور جس کے باہر\عددی{ V(x)=0  }  ہے ۔ وقت کا  غیرتابع شروڈنگر مساوات پر موزوں سرحدی شرائط مسلط کر کے اسے حل کریں۔ تصدیق کریں کہ آپ کی توانائیاں عین  میری حاصل کردہ توانائیوں کے مطابق ہے مساوات\حوالہء{مساوات 2.27   } اور تصدیق کریں کہ میری \عددی{ \psi  } مساوات میں \حوالہء{  مساوات 2.28  } کی جگہ \عددی{ x\apto (x+a)/2  } پر کر کے اور دوبارہ معمول پر لانے سے آپ کی سائی حاصل ہوگی۔ اپنے اولین تین حل ترسیم کریں اور ان کا موازنہ شکل \حوالہء{   شکل 2.2 } سے کریں۔  دھیان رہے کہ یہاں کنواں کی چوڑائی \عددی{ 2a  }  ہے ۔ 
\انتہا{سوال}
\ابتدا{سوال}
 لامتناہی چوکور کنواں مساوات \حوالہء{    مساوات 2.19 } میں ایک ذرے کے ابتدائی تفاعل موج درج ذیل ہے\[ \psi(x,0)=A\sin^{3}(\pi x/a)\]۔ \عددی{ (0\le x\le a)  }
 مستقل\عددی{ A  }   تعین کر کے\عددی{ \psi(x,t)  } تلاش کریں  اور \عددی{ (0\le x\le a)  } کے لحاظ سے وقت  کا حساب لگائیں۔ توانائی کی توقعاتی قیمت کیا ہے ؟
 اشارہ ۔\عددی{ \sin(m\theta)  } اور 
\عددی{ \cos(m\theta)  } کو تخفیف کے بعد\عددی{ \sin^{n}\theta  }  اور \عددی{ \cos^{n}\theta  }کے خطی جوڑ  لکھا جا سکتا ہے جہاں \عددی{ m=0,1,2,\cdots ,n  }  ہوگا۔
\انتہا{سوال}
\ابتدا{سوال}
 کمیت\عددی{  m} کا ایک ذرہ لامتناہی چوکور کنواں میں مساوات\حوالہء{    مساوات 2.19}میں زمینی حال میں ہے ۔ اچانک کنواں کا دایاں دیوار \عددی{ a  } سے حرکت کرتے ہوئے \عددی{  2a }  منتقل ہوتا ہے جس سے کنواں کی چوڑائی دگنی ہو جاتی ہے۔ اس عمل سے لمحاتی طور پر تفاعل موج اثر انداز نہیں ہوتا۔ اس زرہ کی توانائی کی پیمائش اب کی جاتی ہے۔
\begin{enumerate}
\item کس نتیجے کا امکان سب سے زیادہ ہے ؟اس نتیجے کے حصول کا احتمال کیا ہوگا؟ 
\item کونسا نتیجہ اس کے بعد زیادہ امکان رکھتا ہے اور اس کا احتمال کیا ہوگا؟
\item 
توانائی کی توقعاتی قیمت کیا ہوگی؟ اشارہ۔ اگر آپ دیکھیں کہ آپ کو لامتناہی تسلسل کا سامنا پڑ گیا ہے تب کوئی دوسری ترکیب استعمال کریں۔
\end{enumerate} 
 \انتہا{سوال}
\ابتدا{سوال}
\begin{enumerate}
\item   
 دکھائیں کہ لامتناہی چوکور کنواں میں ایک زرے کی تفال موج کوانٹائی تجدیدی ٹائم \عددی{T=4ma^{2}/\pi \hbar }  کے بعد دوبارہ اپنے اصل روپ میں واپس آتا ہے۔  یعنی کسی بھی حال کے لئے نہ صرف ساکن حال کیلئے\عددی{ \psi(x,T)=\psi(x,0)  } 
\item   
 دیواروں سے ٹکرا کر دائیں سے بائیں اور بائیں سے دائیں حرکت کرتے ہوۓ ایک ذرہ جس کی توانائی \عددی{ E  }  ہو کا کلاسیکی ،تجدیدی وقت کیا ہوگا ؟ 
\item   
 کس توانائی کیلئے یہ دو تجدیدی اوقات ایک دوسرے کے برابر ہیں؟
\end{enumerate} 
\انتہا{سوال}
%86
\ابتدا{سوال}
  ایک ذرہ جس کی کمیت \عددی{ m } ہو درج ذیل مخفی کو میں پایا جاتا ہے 
\[V(x)={
\begin{cases}
\infty & (x\textless 0)\\
-32\hbar^{2}/ma^{2} & (0\le x \le a)\\
0 & (x\textgreater a)
\end{cases} \]

\begin{enumerate}
\item اس کے مقید حلوں کی تعداد کیا ہوگی ؟
\item 
 مقید حال میں سب سے زیادہ توانائی کی صورت میں کنواں کے باہر یعنی \عددی{ x\textgreater a  } پر ذرہ پائے جانے کا احتمال کیا ہوگا ؟
جواب \عددی{ 0.542  } اگرچہ یہ کنواں میں مقید ہے لیکن کنواں سے باہر اور کنواں کے اندر اس کی موجودگی کا امکان ایک جیسا ہے۔ 
\end{enumerate}
\انتہا{سوال}
\ابتدا{سوال}
ایک ذرہ جس کی کمیت\عددی{m   }  ہے ہارمونی مرتعیش کی مخفی کو \حوالہء{ مساوات 2.43   }میں درج ذیل حال سے ابتدا کرتا ہے
\[\psi(x,0)=A\Big(1-2\sqrt{\frac{m\omega}{\hbar}x}  \Big)^{2}e^{\frac{-m\omega}{2\hbar}x^{2}}} \]
 جہاں\عددی{A   } کوئی مستقل ہے۔
\begin{enumerate}
\item
 توانائی کی توقعاتی قیمت کیا ہے ؟
\item
مستقبل کے لمحہ ٹی پرتفال موج درج ذیل ہوگا
\[\psi(x,T)=B\Big(1+2\sqrt{\frac{m\omega}{\hbar}x}  \Big)^{2}e^{\frac{-m\omega}{2\hbar}x^{2}}}   \]
 جہاں \عددی{ B  } کوئی مستقل ہے لمحہ ٹی \عددی{  T } کی کم سے کم ممکنہ قیمت کیا ہو گی؟ 
\end{enumerate}
\انتہا{سوال}
\ابتدا{سوال}
درج ذیل نصف ہارمونی مرتعیش کی اجازتی توانائیاں تلاش کریں
\[V(x)={
\begin{cases}
(1/2)m\omega^{2}x^{2}&x\textgreater 0\\
\infty & x\textless 0
\end{cases} \]
مثلا ایک ایسا سپرنگ  جس کو کھینچا تو جا سکتا ہے لیکن اسے دبایا نہیں جاسکتا ہے۔ 
اشارہ ۔ اس کو حل کرنے کے لئے آپ کو ایک بار اچھی طرح سوچنا پڑے گا جبکہ حقیقی حساب بہت کم  درکار ہوگی۔ 
\انتہا{سوال}
\ابتدا{سوال}
آپ نے سوال \حوالہء{   سوال 2.22 }میں ساکن غوثی ذرہ پلندہ موج کا تجزیہ کیا۔ اب ابتدائی تفال موج 
\[\psi(x,0)=Ae^{-ax^{2}e^{ilx}} \] 
 جہاں \عددی{ l  }ایک حقیقی مستقل ہے سے شروع کرتے ہوئے  متحرک غوسی پلندہ موج کے لیےوہی مسئلہ دوبارہ حل کریں۔
\انتہا{سوال}
\ابتدا{سوال}
مکدہ پر درج زیل لامتناہی چوکور کنواں جس کے وسط پر ڈیلٹا تفال  رکاوٹ ہو کے لیے وقت کا غیرتابع  شرونگر مساوات حل کریں۔
\[v(x)={
\begin{cases}
\alpha\delta (x)& -a\textless x\textless +a\\
\infty \abs{x}\ge a
\end{cases} \] 
 جفت اور طاق تفال امواج کو علیحدہ علیحدہ حل کریں ۔انہیں معمول پر لانے کی ضرورت نہیں ہے ۔ اجازتی توانائیوں کو اگر ضرورت ہو ترسیمی طور پر تلاش کریں۔ ان کا موازنہ ڈیلٹا تفال کی غیر موجودگی میں مطابقتی  توانائیوں کے ساتھ کریں.  تاک حلوں پر   ڈیلٹا تفال کا کوئی اثر نہ ہونے پر تبصرہ کریں ۔ تحدیدی صورت \عددی{ a\apto 0  } اور \عددی{ a\apto \infty  } تبصرہ کریں۔ 
\انتہا{سوال}
\ابتدا{سوال}
 ایسے دو یا دو سے زیادہ وقت کے غیرتابع شروڈ نگر مساوات کے حل جن کی توانائی \عددی{ E  } ایک دوسرے جیسی ہو کو انحطاطی حال کہتے ہیں۔ مثال کے طور پر آزاد زرہ حالات دوہری انحطاطی ہیں۔  ان میں سے ایک حل دائیں رخ اور دوسرا بائیں رخ حرکت کو ظاہر کرتا ہے ۔  لیکن ہم نے ایسے کوئی انحطاطی حل نہیں دیکھے جو معمول پر لانے کے قابل ہوں اور یہ محض ایک اتفاق نہیں ہے۔ 
 درج ذیل مسئلہ ثابت کریں ۔
ایک دور میں کوئی مقید  انحطاطی حال بھی پائے جاتے ہیں ۔ 
اشارہ۔ فرض کریں
\عددی{  \psi_{1} }  اور \عددی{ \psi_{2}  } ایسے دو حل ہوں جن کی توانائی \عددی{ E  } ایک دوسرے جیسی ہو ۔ 
حل \عددی{  \psi_{1} } کی شروڈنگر  مساوات کو \عددی{ \psi_{2}  } سے  ضرب دیں اور اس سے \عددی{  \psi_{2} }کے شروڈنگر مساوات کو \عددی{ \psi_{1}  } سے ضرب دے کر منفی کریں۔  یوں دکھائیں کہ \عددی{\psi_{2}\dif{\psi_{1}}/\dif{x}- \psi_{1}\dif{\psi_{2}}/\dif{x}} ایک مستقل ہو گا۔
 اب \عددی{\psi\apto 0   }پر معمول پر لانے کے قابل ہر حل\عددی{\pm \infty   } ہوگا اس حقیقت کو استعمال کرتے ہوۓ دکھائیں کہ یہ مستقل در حقیقت صفر ہوگا اس سے آپ یہ نتیجہ آخذ کر سکتے ہیں کہ \عددی{\psi_{2}   } دراصل \عددی{\psi_{1}   } کا مضرب ہو گا لہذا یہ دو حل الگ الگ نہیں ہو سکتے ہیں ۔
\انتہا{سوال}
\ابتدا{سوال}
فرض کریں کمیت\حوالہء{ m   }   کا ایک معطی ایک دایری چھلا پر بےرگڑ حرکت کرتا ہے۔ چھلے کا محیط \عددی{ L  } ہے ۔ یہ ایک آزاد ذرہ کی طرح ہے لیکن یہاں\عددی{ \psi(x+L)=\psi(x)  } ہوگا۔  اس کے ساکن حال تلاش کریں اور انہیں معمول پر لائیں اور ان کے مطابقتی ایجاد تی توانائیاں تلاش کریں ۔
 آپ دیکھیں گے کہ ہر توانائی\عددی{E_{n}   } کے لئے دو آپس میں غیرتابع حل پائے جائیں گے جن میں سے ایک گھڑی وار اور دوسرا خلاف گھڑی حرکت کے لیے ہوگا۔ جنہیں آپ\عددی{ \psi_{n}^{+}(x)  } اور \عددی{ \psi_{n}^{-}(x)  } کہہ سکتے ہیں۔\حوالہء{    سوال 2.45}
کے مسئلہ  کو مد نظر رکھتے ہوئے آپ اس انحطاط کے بارے میں کیا کہیں گے یہ مسئلہ یہاں کارآمد کیوں نہیں ہے؟  
\انتہا{سوال}
