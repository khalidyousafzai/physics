\documentclass[leqno, b5paper]{khalid-urdu-book}
\begin{document}
	\chapter{پیمائش}
	\section{چیزوں کی پیمائش بشمول لمبائی}
	\subsection{طبیعیات کیا ہے؟}
	سائنس اور انجینئری پیمائش اور موازنہ پر مبنی ہے۔ یوں چیزوں کی پیمائش اور موازنہ کے لئیے ہمیں قواعد کی ضرورت پیش آتی ہے، اور ان پیمائش اور موازنوں کے بُعد تعین کرنے کے لئیے ہمیں تجربات کا سہارا لینا پڑھتا ہے۔ طبیعیات اور انجینئری کا ایک مقصد ان تجربات کی بناوٹ اور تجربہ کرنا ہے۔
	\subsection{چیزوں کی پیمائش}
	طبیعیات میں ملوث مقداروں کی پیمائش کی طریقے جان کر ہم طبیعیات دریافت کرتے ہیں۔ ان مقداروں میں لمبائی، وقت، کمیت، درجہ حرارت، دباؤ، اور برقی رو شامل ہیں۔\\ہم ہر طبی مقدار کا موازنہ ایک معیار کے ساتھ کرکے اسکو اپنی اکائیوں میں ناپتے ہیں۔ اس مقدار کی ناپ کو ایک منفرد نام دیا جاتا ہے جسے اکائی کہتے ہیں۔ مثلاً لمبائی کی ناپ کو میٹر میں ناپا جاتا ہے۔ معیار سے مراد مقدار کی ٹھیک ایک اکائی ہے۔ جیسا آپ دیکھیں گے لمبائی کا معیار جو ٹھیک ایک میٹر کے برابر ہے۔ اُس فاصلہ کو کہتے ہیں جو خلاء میں حرکت کرتے ہوئے کسی ایک مخصوص دورانیہ میں ایک شواع طے کرتا ہے۔ ہم ایک اکائی اور اسکے معیار کی تعریف جیسا چاہیں کر سکتے ہیں۔ تاہم، یہ ضروری ہے کہ دنیا کے باقی سائنسدان بھی اس تعریف کو معنی خیز اور قابلِ عمل کہیں۔\\ایک معیار مثلاً لمبائی کے لئیے طے کرنے کے بعد ہمیں وہ طریقہ کار واضح کرنے ہونگے جن سے ہم کسی بھی لمبائی چاہے وہ ہائڈروجن جوہر کا رداس ہو یا دور ستارے تک کا فاصلہ ہو اس معیار کی صورت میں ظاہر کر سکیں۔ ایسئ ایک ترکیب فیتہ کا استعمال ہے جو ہماری لمبائی کے معیار کو تخمینی طور پرظاہر کرتا ہے۔ بہرحال، بہت سارے موازنوں میں بِلا واسطہ طریقے استعمال کئیے جائیں گے۔ مثلاً ایک جوہر کا رداس یا قریبی ستارے تک کا فاصلہ فیتہ استعمال کرکے نہیں ناپا جا سکتا۔
	\subsection{اساسی مقداریں}
	اتنی زیادہ طبی مقداریں پائی جاتی ہیں کہ انہیں منظم کرنا ایک مسئلہ ہے۔ ہماری خوش قسمتی ہے کہ یہ تمام غیر تابع نہیں ہیں، مثلاً رفتار درحقیقت لمبائی اور وقت کا تناسب ہے۔ یوں بین الاقوامی متفقہ معاہدہ کے تحت چند طبی مقدار مثلاً لمبائی اور وقت منتخب کرکے صرف اِنہی کو معیار مختص کئیے جاتے ہیں۔ اس کے بعد باقی تمام طبی مقداروں کو انہی اساسی مقداروں اور اساسی معیاروں کے روپ میں ناپا جاتا ہے۔ مثال کی طور پر لمبائی اور وقت کی اساسی قیمتیں اور اِنکے اساسی معیار کی روپ میں رفتار تعین کیا جاتا ہے۔\\ ضروری ہے کہ اساسی معیار قابلِ رسائی اور غیر متغیر ہوں۔ اگر ہم بازو کی لمبائی کو معیار لمبائی لیں تب یہ قابلِ رسائی ضرور ہوگا۔ البتہ ہر شخص کے لئیے یہ لمبائی مختلف ہوگی۔ سائنس اور انجینئرنگ میں زیادہ سے زیادہ درستگی مطلوب ہونے کے پیش نظر ہم پہلے غیرمتغیریت پر زور ڈالتے ہیں۔ اس کے بعد اساسی معیار کی بہتر سے بہتر نقل بنا کر اُنہیں فراہم کیا جاتا ہے جنہیں اِنکی ضرورت ہو۔
	\subsection{اکائیوں کا بین الاقوامی نظام}
	سن 1971 میں تول اور پیما کے عمومی اجلاس میں سات مقداروں کو بطور اساسی مقدار منتخب کرکے بین الاقوامی نظامِ اکائی کے اساس چنے گئے۔ جدول 1.1 میں تین اساسی مقدار لمبائی، کمیت اور وقت دیکھائے گئے ہیں۔
	\begin{table}[h!]
		\centering
		\begin{tabular}{|c c c|} 
			\hline
			مقدار & اکائی کا نام & اکائی کی علامت\\ \hline\hline
			لمبائی & میٹر & m \\
			\hline
			وقت & سیکنڈ & s \\
			\hline
			کمیت & کلوگرام & kg \\
			\hline
		\end{tabular}
		\caption{تین اساسی مقداروں کی اکئیاں}
		\label{tab:my_label}
	\end{table}
	ان اکائیوں کی تعریف انسانی جسامت کو مدِ نظر رکھتے ہوئے کی گئی ہے۔\\کئی ماخوذ اکائیوں کی تعریف ان اساسی اکائیوں کی صورت میں کی جاتی ہے۔ مثلاً طاقت کی SI اکائی جسے واٹ کہتے ہیں۔ کمیت، لمبائی اور وقت کی اساسی اکائیوں کی صورت میں کی جاتی ہے۔ یوں جیسا باب 7 میں آپ دیکھیں گے درج ذیل ہوگا۔
	\begin{equation}
		1 watt = 1 W =1 kg\cdot m^2/s^3
	\end{equation}
	جہاں آخر میں اکائیوں کو کلوگرام مربع میٹر فی مکعب سیکنڈ پڑھا جائے گا۔\\بہت بڑی یا بہت چھوٹی مقداروں کو جن سے ہمیں طبیعیات میں عموماً واسطہ پڑھتا ہے جن کو سائنسی علامتیت میں لکھا جاتا ہے، جو دس کی طاقت استعمال کرتا ہے۔ یوں درج ذیل ہوں گے۔
	\begin{equation}
		3560000000 m = 3.56\times 10^9 m
	\end{equation}
	\begin{equation}
		0.000 000 492 s = 4.92\times 10^-7 s
	\end{equation}
	کمیوٹرز پر سائنسی علامتیت اس سے بھی مختصر لکھی جاتی ہے۔ مثلاً 9E65.3 اور 4.92E-7 جہاں دس کی طاقت کو E سے ظاہر کیا جاتا ہے۔۔ کئی کیلکولیٹر میں اس سے بھی مختصر انداز میں لکھتے ہوئے E کو خالی جگہ سے ظاہر کیا جاتا ہے۔\\ہم اپنی آسانی کے لئیے بہت بڑی یا بہت چھوٹی پیمائشوں کو جدول 2.1 میں دی گئی سابقہ کی مدد سے لکھتے ہیں۔
	\begin{table}[h!]
		
		\centering
		\begin{tabular}{|c c c|} 
			\hline
			Symbol & Prefix & Factor\\
			\hline\hline
			Y & yotta- & $10^{24}$\\
			\hline
			Z & zetta- & $10^{21}$\\
			\hline
			E & exa- & $10^{18}$\\
			\hline
			P & peta- & $10^{15}$\\
			\hline
			T & tera- & $10^{12}$\\
			\hline
			G & giga- & $10^{9}$\\
			\hline
			M & mega- & $10^{6}$\\
			\hline
			k & kilo- & $10^{3}$\\
			\hline
			h & hecto- & $10^{2}$\\
			\hline
			da & deka- & $10^{1}$\\
			\hline
			d & deci- & $10^{-1}$\\
			\hline
			c & centi- & $10^{-2}$\\
			\hline
			m & milli- & $10^{-3}$\\
			\hline
			$\upmu$ & micro- & $10^{-6}$\\
			\hline
			n & nano- & $10^{-9}$\\
			\hline
			p & pico- & $10^{-12}$\\
			\hline
			f & femto- & $10^{-15}$\\
			\hline
			a & atto- & $10^{-18}$\\
			\hline
			z & zepto- & $10^{-21}$\\
			\hline
			y & yocto- & $10^{-24}$\\
			\hline
		\end{tabular}
		\caption{سابقے SI اکائیوں کے لئیے}
		\label{tab:my_label}
	\end{table}
	جیسا آپ دیکھ سکتے ہیں ہر ایک سابقہ دس کی کسی مخصوص طاقت کو ظاہر کرتا ہے۔ جس کو بطور جذ ضربی استعمال کیا جاتا ہے۔ بین الاقوامی نظام اکائی کے ساتھ ایک سابقہ منسلک کرنے سے مراد اس اکائی کو مدابقتی جذ ضربی سے ضرب دینا ہے۔ یوں ہم کسی ایک مخصوص برقی طاقت کو
	\begin{equation}
		\num{1.27}\times \SI{e9}{\watt} = \SI{1.27}{gigawatts} = \SI{1.27}{\giga\watt}
	\end{equation}
	یا کسی مخصوص وقتی دورانیہ کع درج ذیل لکھ سکتے ہیں۔
	\begin{equation}
		\num{2.35}\times \SI{e-9}{\second} = \SI{2.35}{nanoseconds} = \SI{2.35}{\nano\second}
	\end{equation}
	چند سابقہ جو ملی لیٹر، سنٹی میٹر، کلوگرام یا میگابائٹ میں استعمال ہوتے ہیں اِن سے آپ ضرور واقف ہوں گے۔
	\subsection{اکائیوں کی تبدیلی}
	ہمیں بعض اوقات طبی مقداروں کی اکائی تبدیل کرنے کی ضرورت پیش آتی ہے۔ اس ترکیب میں ہم اصل پیمائش کو ایک تبادلی جذ جو اکائی کے برابر اکائیوں کا نسبت ہوتا ہے سے ضرب دیتے ہیں۔ مثال کے طور پر چونکہ ایک منٹ اور ساٹھ سیکنڈ مماثل دورانیہ کو ظاہر کرتے ہیں لحاظہ درج ذیل ہو گا۔
	\[\frac{\SI{1}{\minute}}{\SI{60}{\second}} = 1\]
	یا
	\[\frac{\SI{60}{\second}}{\SI{1}{\minute}} = 1\]
	یوں \(\frac{\SI{1}{\minute}}{\SI{60}{\second}}\) یا \(\frac{\SI{60}{\second}}{\SI{1}{\minute}}\) کے نسبت کو تبادلی جذ کے طور پر استعمال کیا جا سکتا ہے۔ ہم ہرگز \(\frac{\num{1}}{\num{60}}=1\) یا \(\num{60} = \num{1}\) نہیں لکھ سکتے۔ ہر عدد اور اسکی اکائی کو اکٹھے رکھنا ہو گا۔\\چونکہ اکائی سے ضرب دینے سے مقدار کی قیمت تبدیل نہیں ہوتی لحاظہ ہم جہاں چاہیں تبادلی جذ کا استعمال کر سکتے ہیں۔ ایسا کرتے ہوئے ہم غیر ضروری اکائیوں کو منسوخ کر سکتے ہیں۔ مثال کے طور پر دو منٹوں کو سیکنڈوں میں تبدیل کرتے ہوئے درج ذیل لکھا جائے گا۔
	\begin{equation}
		\SI{2}{\minute} = (\SI{2}{\minute})(\num{1}) = (\SI{2}{\minute})(\frac{\SI{60}{\second}}{\SI{1}{\minute}}) = \SI{120}{\second}
	\end{equation}
	اگر تبادلہ جذ ضرب متعارف  کرنے سے غیر  ضروری  اکائیاں ایک دوسرے کے ساتھ منسوخ نہ ہوتی ہوں تب جذ ضربی کو اُلٹا کر دوبارہ کوشش کریں۔ اکائیوں کی تبادلہ میں اکائیوں پر متغیرات اور اعداد کے الجبرائی قواعد لاگو ہوں گے۔
	\subsection{لمبائی}
	سن 1972 میں فرانس کے نوزائیدہ جمہوریہ نے ناپ اور تول کا ایک نیا نظام قائم کیا۔ اسی کا سنگِ بنیاد میٹر تھا جو قطب شمال سے خط استوا کے فاصلے کا کڑوڑواں حصہ لیا گیا بعد میں عملی وجوہات کے بنا اس زمینی معیار کو ترک کرتے ہوئے پیرس شہر کے قریب ناپ اور تول کے ایک بین الاقوامی محکمہ میں رکھے گئے پلاٹینم، آئریڈئم کے ڈنڈے پر لگائے گئے دو باریک لکیروں کے بیچ فاصلے کو میٹر کہا گیا۔ اس ڈنڈے کے بہترین نقل پوری دنیا کے معیار سازی تجربہ گاہوں کو بیشے گئے۔ ان ثانوی معیاروں سے مزید زیادہ قابلے رسائی معیار تیار کئیے گئے حتہ کے آخر کار ہر پیمائشی آلہ اس معیاری میٹر ڈنڈے پر مبنی تھا۔\\کچھ عرصہ کے بعد ایک دھاتی ڈنڈے پر دو باریک لکیروں کے بیچ فاصلہ سے زیادہ بہتر معیار کی ضرورت در پیش آئی۔ سن 1960 میں شواع کی طولِ موج پر مبنی میٹر کے ایک نئے معیار پر اتقاو کیا گیا۔ یہ معیار کرپٹن 86 جو کرپٹن کا ایک مخصوص ہم جا ہے کے جوہروں سے خارج ایک مخصوص سرخ-نارنگی شواع کی 73.1650763 طولِ موج کے برابر فاصلہ لیا گیا۔ یہ شواع دنیا میں کہیں پر بھی گیس کے اخراگی نلی سے حاصل کی جا سکتی ہے۔ طولِ موج کی یہ تعداد اس لئیے منتخب کی گئی تا کہ نیا معیار پورانے میٹر کے قریب سے قریب تر ہو۔\\زیادہ سے زیادہ مطلوبہ درستگی کو آخر کار کرپٹن 86 کا معیار پورا نہیں کر سکتاتھا لحاظہ سن 1983 میں ایک نڈر فیصلہ کیا گیا۔ ناپ اور تول کے سترویں عمومی اجلاس میں درج ذیل تہ کیا گیا۔\\"خلاء میں ایک سیکنڈ کے \(\frac{\num{1}}{\num{299792458}}\) حصہ میں روشنی کے تہ شدہ فاصلہ کو ایک میٹر قرار دیا گیا"۔\\وقت کا یہ دورانیہ یوں منتخب کیا گیا کہ روشنی کی رفتار c ٹھیک ٹھیک درج ذیل ہو۔
	\[c = \SI{299792458}{\meter\per\second}\]
	روشنی کی رفتار کی انتہائی درست پیمائش کرنا ممکن ہوا تھا لحاظہ روشنی کی رفتار کو استعمال کرتے ہوئے میٹر اخذ کرنا ایک بہتر قدم تھا۔\\جدول 3.1 میں لمبائیوں کی ایک بڑی ذات دیکھائی گئی ہے۔ جو کائنات سے لے کر انتہائی چھوٹی چیزوں کی لمبائیاں دیتا ہے۔
	\begin{table}[h!]
		\centering
		\begin{tabular}{|c c|} 
			\hline
			Meters in Length & Measurement\\
			\hline\hline
			$2\times 10^{26}$ & formed galaxies first the to Distance\\
			\hline
			$2\times 10^{22}$ & galaxy Andromeda the to Distance\\
			\hline
			$4\times 10^{16}$ & Centauri Proxima star nearby the to Distance\\
			\hline
			$6\times 10^{12}$ & Pluto to Distance\\
			\hline
			$6\times 10^{6}$ & Earth of Radius\\
			\hline
			$9\times 10^{3}$ & Mt.Everest of Height\\
			\hline
			$1\times 10^{-4}$ & page this of Thickness\\
			\hline
			$1\times 10^{-8}$ & virus typical a of Length\\
			\hline
			$5\times 10^{-11}$ & atom hydrogen a of Radius\\
			\hline
			$1\times 10^{-15}$ & proton a of Radius\\
			\hline
		\end{tabular}
		\caption{کچھ تخمینی لمبائیاں}
		\label{tab:my_label}
	\end{table}
	\subsection{بامعنی اعداد اور اشاریہ}
	مثال کے طور پر آپ ایک مسلے پر کام کر رہے ہیں جس میں ہر قیمت دو ہندسوں پر مشتمل ہوتی ہے۔ ان ہندسوں کو بامعنی ہندسے کہتے ہیں۔ اپنا جواب پیش کرتے ہوئے آپ اتنے ہی ہندسے استعمال کریں گے اگر آپ کا مواد دو ہندسوں میں دیا گیا ہو تب آپ کا جواب بھی دو ہندسوں پر مشتمل ہونا چاہئیے۔ اگر چہ آپ کے کیلکولیٹر میں زیادہ ہندسے نظر آئیں گے یہ ہندسے بے معنی ہوں گے۔\\اس کتاب میں دئیے گئے مواد میں کم سے کم با معنی ہندسوں کے برابر حساب کے اختتامی نتائج کو پورمپور کر کے پیش کیا جائے گا۔ ہاں بعض اوقات ایک اضافی ہندسہ بھی رکھا جائے گا۔ اگر ضائع کئیے جانے والے ہندسوں میں بایاں ترین ہندسہ پانچ کے برابر یا اس سے بڑا ہو تب آخری رہنے دیا گیا ہندسے کو اوپر جانب پورمپور کیا جاتا ہے۔ دیگر صورت اس کو اپنے حال میں ہی رکھا جاتا ہے۔ مثال کے طور پر 3516.11 کو تین با معنی ہندسوں تک پورمپور کرکے 4.11 جبکہ 3279.11 کو تین با معنی ہندسوں تک پورمپور کرتے ہوئے 3.11 لکھا جائے گا۔ اس کتاب میں نتائج پیش کرتے ہوئے پورمپور استعمال کئیے گئے ہونے کے باوجود $\approx$ کے بجائے = کی علامت استعمال کی جائے گی۔\\ایک عدد مثلاً 15.3 یا  $\times 10^{3}$15.3 میں با معنی ہندسوں کی تعداد صاف ظاہر ہے البتہ عدد 3000 میں با معنی ہندسے کتنے ہوں گے؟ کیا یہ صرف ایک با معنی ہندسے تک $\times10^{3}$3 معلوم ہے یا یہ ہمیں چار با معنی ہندسوں تک  $\times 10^{3}$000.3 تک معلوم ہے؟ اس کتاب میں 3000 کی طرح اعداد میں تمام صفروں کو بامعنی عدد تصور کیا جائے گا۔\\بامعنی ہندسے اور اشاریہ مقامات دو علیحدہ علیحدہ چیزیں ہیں۔ چرد ذیل لمبائیاں 6.35 ملی میٹر، 56.3 میٹر اور 00356.0 میٹر پر غور کریں۔ ان تمام میں تین با معنی ہندسے جبکہ بلترتیب ایک، دو اور پانچ اشاریہ مقامات پائے جاتے ہیں۔
	\ابتدا{مثال}
	دنیا میں دھاگے کے سب سے بڑے گیند کا رداس 2 میٹر ہے اس دھاگے کی کل لمبائی L کتنی ہوگی؟ اگر چہ ہم گیند سے دھاگہ کھول کر لمبائی L ناپ سکتے ہیں تاہم ہم ایسا نہیں کرنا چاہتے ہیں۔ ہم حساب کے ذریعہ اس کی لمبائی کا تخمینہ لگانا چاہتے ہیں۔
	
	حل: فرض کریں یہ گیند کروی ہو جس کا رداس R=2 میٹر ہے۔ دھاگہ لپیٹتے ہوئے قریبی حصوں کے بیچ خالی جگہ پائی جاتی ہے۔ ان خالی جگہوں کو مدِ نظر رکھتے ہوئے ہم دھاگے کا عمودی تراش ذرا زیادہ تصور کرتے ہیں۔ ہم کہتے ہیں کہ دھاگے کا عمودی تراش چکور ہے جس کی زلی لمبائی d=4 ملی میٹر ہے۔ یوں اس کا رقبہ عمودی تراش $d^{2}$ اور لمبائی L ہوگا، دھاگے کا کل حجم درج ذیل ہوگا-
	\[v=(\text{\RL{رقبہ عمودی تراش}})(\text{\RL{لمبائی}})=d^{2}L\] 
	جو گیند کے حجم \( \pi R^{3}\frac{4}{3}\) کے برابر ہوگا اور چونکہ $\pi$ تقریباً 3 کے برابر ہے لحاظہ اس حجم کو $4R^{3}$ لکھا جا سکتا ہے۔ یوں درج ذیل ہوگا۔
	
	\[d^{2}L =4R^{3}\]
	یا
	\begin{align*}
		\text{L} &= \frac{4R^{3}}{d^{2}}\\
		\text{L} &= \frac{4\left(\SI{2}{\meter}\right)^{3}}{\left(\SI{4 e-3}{\meter}\right)^{2}}\\
		\text{L} &= \SI{2e6}{\meter}\approx\SI{e6}{\meter} =\SI{e3}{\kilo\meter}
	\end{align*}
	آپ کو اتنے سادہ حساب کے لئیے کیلکولیٹر کی ضرورت پیش نہیں آنی چاہئیے۔ جسامت کی قریبی رطبہ تک اس گیند میں تقریباً $\SI{1000}{\kilo\meter}$ دھاگہ پایا جاتا ہے۔
	\انتہا{مثال}
	\section{وقت}
	وقت کے دو پہلو ہیں۔ روز مرہ زندگی میں ہم وقت جاننا چاہتے ہیں تاکہ دن کے کام کاج کو کسی ترتیب سے رکھنا ممکن ہو۔ سائنس کی دنیا میں ہم عموماً یہ جاننا چاہتے ہیں کہ ایک واقعہ کتنے دیر کے لیئے وقوع مذیر ہوا۔ یوں وقت کے کسی بھی معیار کو دو سوالات کا جواب دینا ہوگا: کب ہوا؟ اسکا دورانیہ کتنا تھا؟ جدول $\num{1.4}$ میں چند وقتی وقفوں کو پیش کیا گیا ہے۔
	
	\begin{table}[h!]
		\centering
		\begin{tabular}{|c c|c c|}
			\hline
			پیمائش & سیکنڈ میں وقتی وقفہ & پیمائش & سیکنڈ میں وقتی وقفہ\\
			\hline\hline
			پروٹون کی عرصہ زندگی(پیش خیمہ) & $3\times 10^{40}$ & انسانی دل کی دھڑکنوں کے بیچ وقت & $8\times 10^{-1}$\\
			\hline
			کائنات کی عمر & $5\times 10^{17}$ & مُعون کی عرصہ زندگی & $2\times 10^{-6}$\\
			\hline
			چیپ کے اہرام کی عمر & $1\times 10^{11}$ & مختصرترین تجربہ گاہ کی شعاع کی دھڑکن & $1\times 10^{-16}$\\
			\hline
			انسانی زندگی کی متوقع & $2\times 10^{9}$ & سب سے زیادہ غیر مستحکم ذرے کی عرصہ زندگی & $1\times 10^{-23}$\\
			\hline
			دن کی لمبائی & $9\times 10^{4}$ & پلینک وقت & $1\times 10^{-43}$\\
			\hline
		\end{tabular}
	\caption{کچھ قریب وقتی وقفے}
	\label{tab:my_label}
	\end{table}
ایک ایسا مظہرجو اپنے آپ کو دہراتا ہو وقت کا ممکنہ معیار بن سکتا ہے۔ اپنے محور کے گرد زمین کا ایک چکر جو دن کی لمبائی تعین کرتا ہے کو یوں صدیوں تک استعمال کیا گیا۔ ایک کوارٹز گھڑی جس میں ایک کوارٹز چھلّا کو مسلسل ارتعاش پذیر رکھا جاتا ہے کی پیمانہ بندی زمین کے گھومنے کے ساتھ فلکیاتی مشاہدات کے ذریعہ تجربہ گاہ میں وقتی وقفوں کو ناپنے کے لیئے استعمال کیا جا سکتا ہے۔ تاہم جدید سائنس و انجینئرنگ میں درکار درستگی کی حد تک ایسی پیمانہ بندی ممکن نہیں ہے۔

بہتر معیارِ وقت کی ضرورت کے درپیش جوہری گھڑیاں تیار کی گئیں۔ سن 1967 میں ناپ و طول کے تیرویں عمومی اجلاس میں سیزیم گھڑی پر مبنی معیاری سیکنڈ پر اتفاق کیا گیا۔

سیزیم133 جوہر سے خارج ایک مخصوص طولِ موج کی شعاع کے $\num{9192631770}$ ارتعاش کو درکار وقت کو ایک سیکنڈ کہا گیا۔

جوہری گھڑیاں اتنی بلاتضاد ہوتی ہیں کہ دو سیزیم گھڑیوں کو چھ ہزار سال چلنا ہو گا تاکہ اِن میں ایک سیکنڈ کا فرق پیدا ہو۔ اِس وقت تیار کی جانے والی گھڑیوں کی درستگی $10^{18}$ میں ایک حصہ کے برابر ہے یعنی $10^{18}$ سیکنڈ میں صرف ایک سیکنڈ کا فرق ہو سکتا ہے۔
\حصہ{کمیت} 
\جزوحصہ{معیاری کلوگرام} 
فرانس کے شہر پیرس کے قریب ناپ و طول کے بین الاقوامی مہکمہ میں رکھے گئے پلاٹینم اور آئریڈیم کے ایک سلنڈر کو بین الاقوامی معاہدہ کے تحت ایک کلوگرام کمیت منتخب کیا گیا۔ اس کی بہترین نقلیں دنیا کی بیشتر معیار سازی تجربہ گاہوں کو بھیجی گئی ہیں جن کو استعمال کرتے ہوئے ترازو کی مدد سے کسی بھی جسم کی کمیت ناپی جاسکتی ہے۔ جدول 1.5 میں 83 قدری رطبہ تک کلوگرام کی صورت میں  چند کمیتیں پیش کی گئی ہیں۔
\begin{table}[h!]
	\centering
	\begin{tabular}{|c c|}
		\hline
		چیز & کلوگرام میں کمیتیں\\
		\hline\hline
		معروف کائنات & $1\times 10^{53}$\\
		\hline
		ہماری کہکشاں & $2\times 10^{41}$\\
		\hline
		سورج & $2\times 10^{30}$\\
		\hline
		چاند & $7\times 10^{22}$\\
		\hline
		سیارچہ ایروز & $5\times 10^{15}$\\
		\hline
		چھوٹے پہاڑ & $1\times 10^{12}$\\
		\hline
		بحر لائنر & $7\times10^{7}$\\
		\hline
		ہاتھی & $5\times10^{3}$\\
		\hline
		انگور & $3\times10^{-3}$\\
		\hline
		دھول کی سپیک & $7\times10^{-10}$\\
		\hline
		پینسلن سالمہ & $5\times10^{-17}$\\
		\hline
		یورینیم جوہر & $4\times10^{-25}$\\
		\hline
		 پروٹان & $2\times10^{-27}$\\
		 \hline
		 الیکٹران & $9\times10^{-31}$\\
		 \hline
	\end{tabular}
\caption{کچھ تخمینی کمیتیں}
\label{tab:my_label}
\end{table}
\جزوحصہ{دوم معیار کمیت} 
جوہروں کی کمیت کا موازنہ معیاری کلوگرام کی بجائے زیادہ درستگی کے ساتھ دیگر جوہروں کے ساتھ کیا جا سکتا ہے۔ اَسی کی بنا ہم دوم معیار کمیت بھی رکھتے ہیں۔ یہ کاربن 12 جوہر ہے جس کو بین الاقوامی معاہدہ کے تحت 12 جوہری کمیتی اکئیاں کی کمیت مختص کی گئی ہے۔ ان دو اکائیوں کے بیچ رشتہ درج ذیل ہے۔
\begin{align}
	\num{1} u = \SI{1.66053886e-27}{\kilogram}
\end{align}
جس کے آخری دو ہندسوں میں عدم یقینیت $\pm 10$ ہے۔ سائنس دان کافی درستگی کے ساتھ تجربہ کے ذریعہ کسی بھی جوہر کی کمیت کو کاربن 12 کی کمیت کی لحاظ سے تعین کر سکتے ہیں۔ اس وقت کمیت کی عام اکئیاں مثلاً کلوگرام کو استعمال کرتے ہوئے ہم اتنی درستگی حاصل کرنے سے قاصر ہیں۔
\جزوحصہ{کثافت} 
کثافت $\rho$ سے مراد  اکائی حجم میں کمیت ہے۔
\begin{align}
	\rho = \frac{m}{V}
\end{align}
اس پر باب 14 میں مزید تبصرہ کیا جائے گا۔ کثافت کو عام طور پر کلوگرام فی مربع میٹر یا گرام فی مربع سنٹی میٹر میں ناپا جاتا ہے۔ پانی کی کثافت ایک گرام فی مربع سنٹی میٹر یا ایک ہزار کلوگرام فی مربع میٹر کع عموماً موازنہ کے لیئے استعمال کیا جاتا ہے۔ پانی کی کثافت کے لحاظ سے پلاٹینم کی کثافت تقریباً اکیس گناہ جبکہ لکڑی کی کثافت صرف چونسٹھ فیصد ہوتی ہے۔
\end{document}
