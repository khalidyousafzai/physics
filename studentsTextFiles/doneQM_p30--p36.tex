\{}
%======Section 2.2======
\حصہ {لا متناہی چوکور کنواں}
  فرض کریں
\begin{align}
V(x)=
\begin{cases}
0& if\le x\le a\\
\infty otherwise
\end{cases}
\end{align}
دیگر صورت  \حوالہء{شکل2.1} اس خفی توانائ میں ذرہ مکمل آذاد ہوگا مساواۓ گو کے دونوں اطراف یعنی \عددی{x=0}\عددی{x=a}جہاں لا متناہی قوت اسکو فرار ہونے سے روکے گی ۔ اسکا کلاسیکی نمونہ اس کھو میں ربر کا گیند ہو سکتا ہے جو دیواروں سے ٹکرا کر دائیں سے بائیں اور بائیں سے دائیں ہمیشہ کے لیے حرکت کرتا ہے اگر چہ یہ ایک فرضی مخفی گو ہے آپ اسکو اہمیت دیں ۔ اگر چہ یہ بہت سادہ نظر آتا ہے البتہ اسکی سادگی کے بنا ہی یہ بہت ساری معلومات فراہم کرنے کے قابل ہے ۔ ہم اس سے بار بار رجوع کریں گے۔کنویں سے باہر \عددی{\psi (x)=0} ہوگا۔ لہٰذہ کنویں سے باہر ذرے کی موجودگی کا احتمال صفر ہوگا ۔ کنویں کے اندر 
\عددی{V=0} جہاں وقت کا غیر تابع شوڈنگر مساوات  \حوالہء{مساوات 2.5} ایسے پڑھی جاۓ گی ۔ 
\begin{align}
-\frac{\hslash^{2}}{2m}\frac{\dif^{2}\psi}{\dif x^{2}}&=E\psi
\end{align} 
یا 
\begin{align}
\frac{\dif^{2}\psi}{\dif x^{2}}&=-k^{2}\psi \quad k=\frac{\sqrt{2mF}}{\hslash}
\end{align}
اسکو یوں لکھتے ہوۓ میں فرض کر رہا ہوں کہ \عددی{E\ge 0} ۔ ہم  \حوالہء{سوال 2.2} سے جانتے ہیں کہ \عددی{E\textless 0 } ممکن نہیں ہے۔   \حوالہء{مساوات 2.21}
 کلاسیکی سادہ ہارمونی مرتعش کی مساوات ہے جسکا عمومی حل 
\begin{align}
\psi(x)=A\sin kx+B\cos kx
\end{align}
 جہاں \عددی{A }  اور  \عددی{B} اختیاری مستقل ہیں۔ 
\عددی{\psi (x)} کی موضوع سرحدی شرائط کیا ہونگے؟ عموماً یہ دونوں استمراری ہونگے، لیکن جہاں صرف مخفی گو لامتناہی تک پہنچنے کی کوشش کرے وہاں ان میں سے صرف پہلا درست ہوگا۔ میں حصہ  \حوالہء{حصہ2.5}میں ان سرحدی شرائط کو ثابت کروں گا اور \عددی{V=\infty}   کی صورتحال کو بھی دیکھوں گا ۔ میں امید کروں گا ابھی آپ مجھ پر یقین کر سکتے ہیں ۔ 

استمرار کے بغیر (\عددی{\psi (x)}) درج ذیل پر پورا اترے گا 
\begin{align}
\psi(0)=\psi(a)=0
\end{align} 
تاکہ کنویں سے باہر حل کے ساتھ اسکا جوڑ پیدا ہو، اس سے ہمیں  \عددی{A}اور  \عددی{B}کے بارے میں کیا معلومات حاصل ہوں گی ۔
\begin{align}
\psi(x)=A\sin kx
\end{align}
تب \عددی{\psi(x)=A\sin ka}  \عددی{     A=0  }    ایسا ہونے کی صورت میں حل \عددی{   \psi(x)=0}  ملتا ہے جو معمول پر لانے کے قابل نہیں ہے یعنی ایک غیر اہم حل ہے ۔  یا \عددی{   \sin ka =0} 
\begin{align}
ka&=0,\pm\pi,\pm2\pi,\pm3\pi\cdots
\end{align} 
ہوگا۔ لیکن \عددی{ k =0}  بھی \عددی{ \psi(x)=0}  دے گا جس میں ہم دلچسپی نہیں رکھتے اور   \عددی{K} کی منفی قیمتیں نیا حل نہیں دیتی ہیں کیونکہ \عددی{ \sin(-\theta)=-\sin(\theta)}    اور ہم منفی کی علامت کو  \عددی{A} میں ضم کر سکتے ہیں یوں منفرد حل درج ذیل ہوں گے 
\begin{align}
k_{n}=\frac{n\pi}{a}\quad n=1,2,3,\cdots
\end{align}
ہوگی۔ دلچسپ بات یہ ہے کہ \عددی{ x=a} پر سرحدی شرط مستقل   \عددی{A} نہیں دیتی ہے بلکہ یہ    \عددی{K} کی قیمت بیان کرتی ہے ، جس سے   \عددی{E} کی مندرجہ ذیل قیمتیں حاصل ہوتی ہیں
\begin{align}
E_{n}=\frac{\hslashK^{2}_{n}}{2m}=\frac{n^{2}\pi^{2}\hslash^{2}}{2ma^{2}}
\end{align} 
کلاسیکی صورت کے برعکس لامتناہی چوکور کنویں میں کوانٹم ذرہ ہر توانائ رکھنے کے قابل نہیں ہے ۔ یہ مخصوص اجازتی قیمتوں پر ہو سکتا ہے،   \عددی{A} کی قیمت حاصل کرنے کی خاطر میں \عددی{ \psi}  کو معمول پر لانا ہوگا۔ 
\begin{align}
\int_{0}^{a}\abs{A}^{2}\sin^{2}(kx)\dif{x}=\abs{A}^{2}\frac{a}{2}=1, \quad \abs{A}^{2}=\frac{2}{a}
\end{align} 
یہ صرف   \عددی{A} کی متعلق قیمت دیتا ہے مثبت حقیقی جزر کا انتخاب سادہ ترین ہوگا \عددی{A=\sqrt{2/a}}     \عددی{A} کا زاویہ کوئ طبعی معنی نہیں رکھتا کنویں کے اندر شوڈنگر مساوات کے حل درج ذیل ہوں گے۔
\begin{align}
\psi_{n}(x)=\sqrt{\frac{2}{a}\sin(\frac{n\pi}{a}x)}
\end{align}
جیسا ہم ذکر کر چکے وقت کا غیر پابند شوڈنگر مساوات لا متناہی حلوں کا سلسہ دیتا ہے ، یعنی ہر مثبت عدد ایک حل دیتا ہے ان میں سے اولین چند کو  \حوالہء{شکل 2.2}میں ترسیم کیا گیا ہے۔ جو دھاگے پر ساکن امواج کی طرح نطر آتے ہیں ۔ \عددی{\psi_{1}}      کی توانائ کم سے کم ہے اور اسے زمینی حال کہتے ہیں۔باقی حال جنکی توانائ \عددی{n^{2}}    کے براہ راست بڑھتی ہے کو ہیجان حالتیں کہتے ہیں ۔ تمام \psi(x) کے چند اہم اور دلچسپ خواص پاۓ جاتے ہیں
\begin{enumerate}
\item
 کنویں کے وسط کے حساب سے یہ باری باری جفت اور طاق ہوں گے ۔ \عددی{\psi_{1}}      جفت \عددی{\psi_{2}}      طاق \عددی{\psi_{3}}  جفت وغیرہ وغیرہ
\item
جیسے جیسے بڑھتی توانائ کی طرف جائیں   ہر اگے حال میں جوڑ یا گرہ کی تعداد میں ایک کا اضافہ ہوگا۔ گرہ سے مراد وہ نقطہ ہے جہاں \عددی{\psi}  کی قیمت صفر ہوتی ہے۔ \عددی{\psi_{1}} میں کوئ جوڑ نہیں پایا جاتا کیونکہ آخری نقاط کے صفر کو نہیں گنا جاتا \عددی{\psi_{2}}  میں ایک جوڑ ہے \عددی{\psi_{3}} میں دو جوڑ ہیں ۔
\item
 یہ تمام ایک دوسرے کے ساتھ عمودی ہیں 
\begin{align}
\int\psi_{m}(x)\psi_{n}(x)\dif{x}=0
\end{align}
جہاں پر بھی (\عددی{m\neq n}  ) اسکا ثبوت پیش کرتے ہیں 
\begin{align}
&\int\psi_{m}(x)\psi_{n}(x)\dif{x}=\frac{2}{a}\int_{0}^{a}\sin(\frac{m\pi}{a}x)\sin(\frac{n\pi}{a}x)\dif{x}\\
&\frac{1}{a}\int_{0}^{a}\big[\cos\big(\frac{m-n}{a}\pi x\big)-\cos\big(\frac{m+n}{a}\pi x\big)\big]\dif{x}\\
&\big[\frac{1}{(m-n)\pi}\sin\big(\frac{m-n}{a}\pi x\big)-\frac{1}{(m+n)\pi}\sin\big(\frac{m+n}{a}\pi x\big)\big]_{0}^{a}\\
&\frac{1}{\pi}\big[\frac{\sin[(m-n)\pi]}{(m-n)}-\frac{\sin[(m+n)\pi]}{(m+n)}\big]=0
\end{align} 
دھیان رہے کے درج بالا دلیل \عددی{m= n}            کی صورت میں پیش نہیں کی جا سکتی ہے ، یہاں پر رک کر دیکھیں کے  \عددی{m= n}  کی صورت میں اس میں کہاں غلطی ہے ۔ ایسی صورت میں معمول پر لانے کا عمل ہمیں بتاتا ہے کہ تکمل کی قیمت  \عددی{1} ہے، بلکہ ہم عمودیت اور معمول پر لانے کا عمل کو ایک فقرے میں بیان کر سکتے ہیں 
\begin{align}
\int\psi_{m}(x)\psi_{n}(x)\dif{x}=\delta_{mn}
\end{align}
جہاں \عددی{\delta_{mn}}   کو کرونیکر ڈیلٹا کہتے ہیں جسکی تعریف درج ذیل ہے
\begin{align}
\delta_{mn}=
\begin{cases}
0& if m\neq n\\
1 & if m=n
\end{cases}
\end{align} 
ہم کہتے ہیں کہ تمام \عددی{\psi}   ایک دوسرے کے ساتھ معیاری عمودی ہیں 
\item
 یہ مکمل ہیں جس سے مراد کہ کوئ بھی دوسرا تفاعل       \عددی{f(x)} کو ان کا خطی جوڑ لکھا جا سکتا ہے ۔ 
\begin{align}
f(x)=\Sigma_{n=1}^{\infty}c_{n}\psi_{n}(x)=\sqrt{\frac{2}{a}}\Sigma_{n=1}^{\infty}c_{n}\sin\big(\frac{n\pi}{a}x\big)
\end{align}
 میں تفاعل \عددی{\sin(n\pi x/a)}   کی کاملیت کو یہاں ثابت نہیں کروں گا لیکن اگر آپ نے آلہ علم الاحسا پڑھی ہو تو آپ مساوات \حوالہء{مساوات 2.32} کو پہچان پائیں گے جو تفاعل کا فوریر تسلسل ہے۔ اور آپ جانتے ہیں کہ کسی بھی تفاعل کو فوریر تسلسل کی صورت میں پھیلا کر لکھا جا سکتا ہے جس سے بعض اوقات مسئلہ ڈروشلے کہتے ہیں۔ \عددی{c_{n}}  کو \عددی{\psi_{n}}  کی معیاری عمودیت کی مدد سے حاصل کیا جاتا ہے۔ مساوات \حوالہء{مساوات 2.32} کے دونوں اطراف کو \عددی{\psi_{m}(x)} سے ضرب دے کر تکمل لیں
 \begin{align}
\int \psi_{m}(x)f(x)\dif{x}=\Sigma_{n=1}^{\infty}c_{n}\int\psi_{m}(x)\psi_{n}(x)\dif{x}=\Sigma_{n=1}^{\infty}c_{n}\delta_{mn}=c_{m}
\end{align}
آپ دیکھ سکتے ہیں کہ کرونیکل ڈیلٹا مجموعے میں تمام اجزاء کو ختم کر دیتا ہے ما سواۓ اس جز کے جہاں  \عددی{n=m} ہو ۔ یوں تفاعل   \عددی{f(x)} کے پھیلاؤ کے جز کا عددی سر درج ذیل ہوگا۔
\begin{align}
c_{n}\int\psi_{n}(x)f(x)\dif{x}
\end{align}
\end{enumerate}


درج بالا   \عددی{4} خواص انتہائ طاقتور خواص ہیں جو خصوصاً لامتناہی چوکور کنویں کے لیے نہیں ہیں ۔ پہلا خواص ہر اس صورت میں درست ہوگا جب خفی توانائ از خود تشاکلی ہو۔ دوسرا خواص عالمگیر خواص ہے جو خفی گو کی شکل و صورت پر منحصر نہیں ہے۔ عمودیت بھی کافی عمومی خاصیت ہے ، جسکا ثبوت میں باب  \حوالہء{باب 3} میں پیش کروں گا۔  کاملیت تمام خفی گو کے لیے کارآمد ہوگا۔ جن سے آپ کو واسطہ ہو سکتا ہے۔ لیکن اسکا ثبوت کافی پیچیدہ ہے۔ جس کے بنا عموماً ماہر طبیعات اس کا ثبوت دیکھے بغیر مان لیتے ہیں ۔ لا متناہی چوکور کنویں کے ساکن حال مساوات  \حوالہء{مساوات 2.18} درج ذیل ہوں گے۔ یہ مساوات
 \begin{align}
\psi_{n}(x,t)=\sqrt{\frac{2}{a}}\sin\big(\frac{n\pi}{a}x\big)e^{-i(n^{2}\pi^{2}h/2ma^{2})t}
\end{align}
ہے۔ میں نے مساوات  \حوالہء{مساوات 2.17} میں دعوہ کیا کہ وقت کے غیر تابع شوڈنگر مساوات کا عمومی ترین حل ساکن حالات کے خطی جوڑ ہوگا۔
\begin{align}
\psi_{n}(x,t)=\sqrt{\frac{2}{a}}\sin\big(\frac{n\pi}{a}x\big)e^{-i(n^{2}\pi^{2}h/2ma^{2})t}
\end{align}  
 اگر آپ کو اس حل پر شق ہو تو اسکی تصدیق ضرور کیجیے گا۔ مجھے صرف اتنا دکھانا ہوگا کہ کسی بھی ابتدائ تفاعل موج \عددی{\psi(x,0)}   پر اس حل کو   بٹھانے کے لیے موضوع   \عددی{c_{n}} درکار ہوں گے
\begin{align*}
\psi(x,0)=\Sigma_{n=1}^{\infty}c_{n}\psi_{n}(x)
\end{align*}
\عددی{\psi }  کی کملیت اس کی ضمانت دیتی ہے کہ میں ہر صورت \عددی{\psi(x,0)}   کو اسطرح بیان کر سکتا ہوں اور اسکی معیاری عمودیت استعمال کرتے ہوۓ ہم   \عددی{c_{n}} کی قیمت حاصل کرتے ہیں۔ 
\begin{align}
c_{n}=\sqrt{\frac{2}{a}}\int_{0}^{a}\sin\big(\frac{n\pi}{a}x\big)\psi(x,0)\dif{x}
\end{align}
یوں آپ نے دیکھا کہ کسی بھی ابتدائ تفاعل موج کو جانتے ہوۓ ہم سب سے پہلے  \عددی{c_{n}}کو مساوات \حوالہء{مساوات 2.37} سے حاصل کرتے ہیں،  اور پھر اسکے بعد ہم انہیں (مساوات  \حوالہء{مساوات 2.36}) میں پر کر \عددی{\psi(x,t)}             حاصل کرتے ہیں ۔ تفاعل موج جانتے ہوۓ ہم دلچسپی کی کوئ بھی حرکی مقدار کا  کا حساب کر سکتے ہیں ۔ جیسا ہم نے باب  \حوالہء{باب 1}میں کیا۔ یہی ترکیب کسی بھی خفی توانائ کے لیے کارآمد ہے۔ جو صرف  \عددی{\psi}  کی  قیمتیں اور اجازتی توانائیوں کی قیمتیں تبدیل کریں گے
%===Example 2.2==== 
\ابتدا{مثال}

لا متناہی چوکور کنویں میں ایک ذرے کی ابتدای تفاعل موج درج ذیل ہے 
\begin{align*}
\psi(x,0)=Ax(a-x),\quad (0\le x\le  a)
\end{align*}
 یہاں   \عددی{A} کوئ مستقل ہے جس کے لیے شکل  \حوالہء{شکل 2.3} سے رجوع کریں، کنویں سے باہر \عددی{\psi=0}  ہے ۔ \عددی{\psi(x,t)}  تلاش کریں۔ 
ہم سب سے پہلے \عددی{\psi(x,0)}            کو معمول پر لاتے ہیں تا کہ       \عددی{A} کی قیمت حاصل کر سکیں 
\begin{align*}
1=\int_{0}^{a}\abs{\psi(x,0)}^{2}\dif{x}=\abs{A}^{2}\int_{0}^{a}x^{2}(a-x)^{2}\dif{x}=\abs{A}^{2}\frac{a^{2}}{30}
\end{align*}
   جس سے
\begin{align*}
A=\sqrt{\frac{30}{a^{5}}}
\end{align*}
 حاصل ہوتا ہے ۔ 

مساوات  \حوالہء{مساوات 2.37} کے تحت اینواں عددی سر درج ذیل ہوگا
\begin{align*}
c_{n}&=\sqrt{\frac{2}{a}}\int_{0}^{a}\sin\big(\frac{n\pi}{a}x\big)\sqrt{frac{30}{a^{5}}}x(a-x)\dif{x}\\
&=\frac{2\sqrt{15}}{a^{3}}\Big[a\int_{0}^{a}x\sin\big(\frac{n\pi}{a}x\big)\dif{x}-\int_{0}^{n}x^{2}\sin\big(\frac{n\pi}{a}x\big)\Big]_{0}^{a}\\
&=\frac{2\sqrt{15}}{a^{3}}a\Big[\big(\frac{a}{n\pi}\big)^{2}\sin\big(\frac{n\pi}{a}x\big)-\frac{ax}{n\pi}\cos\big(\frac{n\pi}{a}x\big)\Big]_{0}^{a}\\
&-\Big[2\big(\frac{a}{n\pi}\big)^{2}x\sin\big(\frac{n\pi}{a}x\big)-\frac{(n\pi x/a)^{2}-2}{(n\pi)^{3}}\cos\big(\frac{n\pi}{a}x\big)\Big]_{0}^{a}\\
&=\frac{2\sqrt{15}}{a^{3}}\Big[-\frac{a^{3}}{n\pi}\cos(n\pi)+a^{3}\frac{(n\pi)^{2}-2}{(n\pi)^{3}}\cos(n\pi)+a^{3}\frac{2}{(n\pi)^{3}}\cos(0)\\
&=\frac{4\sqrt{15}}{(n\pi)^{3}}[\cos(0)-\cos(n\pi)]\\
&=\begin{cases}
0 & if n is even\\
8\sqrt{15}/(n\pi)^{3}&if n is odd
\end{cases}
\end{align*}
یوں مساوات \حوالہء{مساوات 2.36} درج ذیل صورت اختیار کرتی ہے ۔
 \begin{align*}
\psi(x,t)=\sqrt{\frac{30}{a}}\big(\frac{2}{\pi}\big)\Sigma_{n=1,3,5\cdots}\frac{1}{n}^{3}\sin\big(\frac{n\pi}{a}x\big)e^{-in^{2}\pi^{2}\hslash t/2ma^{2}}
\end{align*}



























 










