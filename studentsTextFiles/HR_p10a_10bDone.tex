%Q17 p10
\ابتدا{سوال}  تجربہ گاہ میں گھڑیوں کی  جانچ پڑتال کی جا رہی ہے۔  ہفتے کے سات دن ٹھیک دوپہر  \عددی{12}  بجے  گھڑیوں کا وقت  ذیل جدول میں  پیش ہے۔  بہترین وقت رکھنے والے گھڑی پہلے  رکھ کر گڑیوں کی درجہ بندی کریں۔ اپنے انتخاب کی وجہ پیش کریں۔
\begin{center}
\begin{tabular}{CCCCCCCC}
\toprule
\text{\RL{گھڑی}} & \text{\RL{اتوار}} & \text{\RL{پیر}} & \text{\RL{منگل}} & \text{\RL{بدھ}} & \text{\RL{جمعرات}} & \text{\RL{جمعہ}} &  \text{\RL{ہفتہ}} \\
\midrule
A&12:36:40 &12:36:56 &12:37:12 &12:37:27 &12:37:44 &12:37:59 &12:38:14\\
B&11:59:59 &12:00:02 &11:59:57 &12:00:07 &12:00:02 &11:59:56 &12:00:03\\
C&15:50:45 &15:51:43 &15:52:41 &15:53:39 &15:54:37 &15:55:35 &15:56:33\\
D&12:03:59 &12:02:52 &12:01:45 &12:00:38 &11:59:31 &11:58:24 &11:57:17\\
E&12:03:59 &12:02:49 &12:01:54 &12:01:52 &12:01:32&12:01:22 &12:01:12\\
\bottomrule
\end{tabular}
\end{center}
 \انتہا{سوال} 
%-------------------------------
%Q18 p10
\ابتدا{سوال} 
زمین کی گردش دن بدن آہستہ ہو رہی ہے، اور دن لمبا ہوتا جا رہا ہے۔ پہلی عیسوی صدی کا آخری دن   صدی کے پہلے دن سے  \عددی{\SI{1}{\milli\second}}لمبا ہے۔  \عددی{20}صدیوں میں ایک دن کا دورانیہ کل  کتنا بڑا؟
 \انتہا{سوال} 
%---------------------------------
%Q19 p10
\ابتدا{سوال}  خط استوا پر  پرسکون سمندر کے کنارے ریت پر لیٹ کر آپ  غروب ہوتے سورج کا نظارہ کر رہے ہیں۔ جیسے ہی سورج کا بالا  سر سمندر کے پیچھے غروب ہوتا ہے،  آپ گھڑی میں وقت دیکھ کر قلم بند کرتے ہیں ۔ اس کے بعد  اتنی بلندی پر کھڑے ہو کر کہ آپ کی آنکھ \عددی{H=\SI{1.70}{\meter}}زیادہ اونچائی پر ہو،  آپ دوبارہ سورج کے بالا سر کو غروب ہوتے دیکھ کر وقت قلم بند کرتے ہیں۔ کل دورانیہ\عددی{t=\SI{11.1}{\second}} ملتا ہے۔ زمین کا رداس\عددی{r}کتنا ہے؟ 
\انتہا{سوال} %------------------
%module 1.3 mass p10
\جزوحصہء{کمیت} 
%Q20 p10
\ابتدا{سوال} 
\سن{1992} میں شیشے کی سب سے بڑی بوتل بنائی گئی جس کا حجم\عددی{193} امریکی  گیلن تھا۔ 
(ا) یہ   بوتل \عددی{\SI{1.0}{\micro\meter\cubed}} سے کتنا کم ہے؟  (ب)  اگر  بوتل \عددی{\SI{1.8}{\gram\per\minute}}  کی شرح سے پانی سے بھری  جائے،  کتنا  وقت درکار ہو  گا؟  پانی کی کثافت\عددی{\SI{1000}{\kilo\gram\per\meter\cubed}} ہے۔ 
\انتہا{سوال} 
%-----------------
%Q21 p10
\ابتدا{سوال} 
زمین کی کمیت\عددی{\SI{5.98e24}{\kilo\gram\per\meter\cubed}} ہے۔ زمین کے  جوہر   (ایٹم) کی اوسط کمیت تقریباً \عددی{\SI{40}{\atomicmassunit}}  ہے۔ زمین میں کل کتنے  جوہر  ہیں؟ 
\انتہا{سوال} 
%------------------
%Q22 p10
\ابتدا{سوال} 
سونے کی کثافت\عددی{\SI{19.32}{\gram\per\centi\meter\cubed}} ہے، اور یہ سب سے زیادہ \اصطلاح{ تار پذیر }\فرہنگ{تار پذیر}\حاشیہب{ductile}\فرہنگ{ductile} دھات ہے ، جس کو دبا کر باریک پتہ  یا کھینچ کر باریک تار بنایا  جا سکتا ہے۔ (ا) اگر\عددی{\SI{27.63}{\gram}} سونے 
سے\عددی{\SI{1.000}{\micro\meter}}موٹی چادر بنائی جائے، اس چادر کا رقبہ کتنا ہوگا؟   (ب) اس کے برعکس ، اگر اس
 سے\عددی{\SI{2.500}{\micro\meter}}رداس کا  تار بنایا  جائے، اس تار کی لمبائی کتنی ہوگی؟ 
\انتہا{سوال} 
%---------------------
%Q23 p10
\ابتدا{سوال} 
(ا)  پانی کی کثافت ٹھیک\عددی{\SI{1}{\gram\per\centi\meter\cubed}} فرض کرتے ہوئے،\عددی{\SI{1}{\cubic\meter}}پانی کی کمیت تلاش کریں۔ 
(ب) اگر ایک برتن سے\عددی{\SI{5700}{\meter\cubed}} پانی کی نکاسی\عددی{10.0} گھنٹوں میں  ہو، نکاسی کمیت کی شرح\عددی{\si{\kilo\gram\per\second}} میں  کتنی  ہوگی؟ 
\انتہا{سوال} 
%------------------
%Q24 p10
\ابتدا{سوال} 
 ساحل سمندر پر ریت زیادہ تر  کروی سلیکان ڈائی اکسائیڈ کے دانوں پر مشتمل ہے ، جن کا اوسط رداس\عددی{\SI{50}{\micro\meter}} اور 
 کثافت\عددی{\SI{2600}{\kilo\gram\per\meter\cubed}}ہے۔ کتنی کمیت کے ریتیلی دانوں کا کل سطحی رقبہ (تمام انفرادی کروں کا مجموعی رقبہ)
 \عددی{\SI{1.00}{\meter}}  ضلع   کے  مکعب کے  سطحی رقبہ کے برابر ہوگا؟ 
\انتہا{سوال} 
%--------------------
%Q25 p10
\ابتدا{سوال} 
تیز بارش کے دوران پہاڑی کا ایک حصہ ، جس کی افقی  لمبائی\عددی{\SI{2.5}{\kilo\meter}} ، ڈھلوان کے ہمراہ لمبائی\عددی{\SI{0.8}{\kilo\meter}} ، اور موٹائی \عددی{\SI{2}{\meter}} ہے،  نیچے گرتا ہے۔  مٹی وادی میں\عددی{\SI{0.4}{\kilo\meter \times \SI{0.4}{\kilo\meter}}}  رقبے پر یکساں تقسیم ہوتی ہے۔ مٹی کی کثافت\عددی{\SI{1900}{\kilo\gram\per\meter\cubed}} لیں۔ وادی کے\عددی{\SI{4}{\meter\squared}} رقبے پر مٹی کی کمیت کیا ہوگی؟
 \انتہا{سوال} 
 %------------------------------------
 %Q26 p10
\ابتدا{سوال} 
\اصطلاح{ تودہ ابر بادل  }\فرہنگ{بادل!تودہ ابر}\حاشیہب{cumulus clouds}\فرہنگ{clouds!cumulus} کے  \عددی{\SI{1}{\centi\meter\cubed}} میں تقریباً \عددی{50} تا\عددی{500} پانی کے قطرے پائے جاتے ہیں، جن کا عمومی   رداس\عددی{\SI{10}{\micro\meter}} ہو گا۔ دیے گئے  سعت  کے لیے، درج ذیل کی کمتر اور بلندتر قیمتیں کیا ہوں گی؟  (ا) نلکی شکل  و صورت کے تودہ ابر بادل ، جس کا  رداس\عددی{\SI{1}{\kilo\meter}}اور قد \عددی{\SI{3}{\kilo\meter}} ہو ، میں کتنا \عددی{\si{\meter\cubed}} پانی  ہو  گا؟  (ب) یہ پانی ایک لیٹر کی  کتنی بوتلیں بھر سکتا ہے؟  (ج) پانی کی کثافت\عددی{\SI{1000}{\kilo\gram\per\meter\cubed}}ہے۔   بادل میں  پانی کی کمیت کیا ہو گی؟ 
\انتہا{سوال} 
%-----------------------------
%Q27 p10
\ابتدا{سوال} 
لوہے کی کثافت\عددی{\SI{7.87}{\gram\per\centi\meter\cubed}} ہے ، جبکہ لوہے کے جوہر  (ایٹم)کی کمیت\عددی{\SI{9.27e-26}{\kilo\gram}} ہے۔ فرض کریں  جوہر کروی ہے اور ان کے بیچ  فاصلہ نہیں پایا جاتا۔  (ا)   لوہے کے جوہر کا حجم اور  (ب) قریبی جوہر کے مراکز کے بیچ فاصلہ کیا  ہو گا؟ 
\انتہا{سوال} 
%-------------------------------
%Q28 p10
\ابتدا{سوال} 
جوہر کے  ایک  \اصطلاح{ مول }\فرہنگ{مول}\حاشیہب{mole}\فرہنگ{mole} سے مراد عدد \عددی{\num{6.02e23}} ہے۔  موٹی گھریلو بلی میں، مقدار کے قریبی رتبہ تک،  کتنے  مول  جوہر  ہوں گے؟ ہائیڈروجن جوہر، آکسیجن جوہر، اور کاربن جوہر کی کمیتیں بالترتیب\عددی{\SI{1}{\atomicmassunit}}،
\عددی{\SI{16}{\atomicmassunit}} ، اور\عددی{\SI{12}{\atomicmassunit}}ہیں۔ 
\انتہا{سوال} 
%--------------------------------
%Q29 p10
\ابتدا{سوال} 
آپ ملائیشیا کے مویشی منڈی میں  بیل خریدتے ہیں ، جس کا  وزن مقامی اکائیوں میں \عددی{28.9} پکول  ہے:  ایک پکول  \عددی{100} جن کے برابر ہے،   ایک جن   \عددی{16}  تاہل، ایک تاہل   \عددی{10}  چی ، اور ایک چی \عددی{10} ہون  کے برابر ہے۔ ایک ہون کی کمیت\عددی{\SI{0.3779}{\gram}} ہے۔ بیل کی کمیت\عددی{\si{\kilo\gram}}میں کتنی ہے؟ 
\انتہا{سوال} 
%----------------------------------
%Q30 p10
\ابتدا{سوال} 
رستا ہوئے ظرف میں پانی انڈیلا جاتا ہے۔ پانی کی کمیت وقت \عددی{t} کا تفاعل \عددی{m=5.00t^{0.8} - 3.00t + 20.00} ہے، جہاں\عددی{t\ge 0} ،\عددی{m} کی اکائی گرام  ، اور\عددی{t}کی اکائی سیکنڈ  ہے۔  (ا) کس لمحے پر  پانی کی کمیت  اعظم ہے، اور   (ب)   اعظم کمیت کتنی ہے؟  کمیت میں تبدیلی کی شرح،\عددی{\si{\kilo\gram\per\minute}}  اکائیوں میں ،  (ج)\عددی{t=\SI{2.00}{\second}} اور  (د) \عددی{t=\SI{5.00}{\second}} پر  کیا ہے؟ 
\انتہا{سوال}
%----------------------------
%Q31 p10
 \ابتدا{سوال} 
 سیدھا کھڑا برتن ، جس کی تہہ  کا رقبہ\عددی                            {\SI{14}{\centi\meter}} با \عددی{\SI{17}{\centi\meter}}ہے ،   مٹھائی  سے بھرا جاتا ہے۔ انفرادی  مٹھائی کی کمیت\عددی{\SI{0.0200}{\gram}} اور حجم\عددی{\SI{50.0}{\milli\meter\cubed}} ہے۔ مٹھائیوں  کے بیچ  خلا نظر انداز کریں۔  برتن میں مٹھائیوں کی بلندی کی  شرح   \عددی{\SI{0.250}{\centi\meter\per\second}} ہے۔برتن میں  مٹھائی کی  کمیت میں اضافہ  کی شرح   ( کلوگرام فی منٹ) کیا ہے؟ 
\انتہا{سوال} 
%--------------------------------

%additional problems p10
\موٹا{اضافی سوالات }\\
%Q32 p10
\ابتدا{سوال} 
حقیقی گھر کے لحاظ سے\عددی{1:12}پیمانہ   سے گڑیا  گھر بنایا جاتا ہے (یعنی گڑیا کے گھر کا ہر ضلع حقیقی  گھر کے مطابقتی ضلع کا\عددی{\tfrac{1}{12}}ہوگا)  ۔ ساتھ ہی حقیقی گھر کے\عددی{1:144} پیمانہ سے  مزید چھوٹا گھر تعمیر کیا جاتا ہے، جو گڑیا  گھر کے اندر رکھا جائے گا۔ فرض کریں، حقیقی گھر (شکل\حوالہء{1.7} )  کی لمبائی (سامنے سے) \عددی{\SI{20}{\meter}} ، گہرائی \عددی{\SI{12}{\meter}} ، اور بلندی \عددی{\SI{6}{\meter}} ہے،  اور اس کا چھت ڈھلوانی ہے،  جس کی اونچائی \عددی{\SI{3.0}{\meter}}  ہے۔  (ا) گڑیا گھر  اور  (ب)  گڑیا گھر کے اندر رکھے جانے والے مزید چھوٹے گھر کا حجم  ، مربع میٹر میں  ، کیا ہو گا؟ 
\انتہا{سوال}
%-----------------------
