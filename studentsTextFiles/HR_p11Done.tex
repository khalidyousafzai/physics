%question 33, 34, 35, 36,38,41   is different from the book. answer must be handled correctly
%Q33 p11
\ابتدا{سوال} 
برصغیر میں لمبائی کی قدیم اکائی کوس ہے جو آئین  اکبری میں پانچ ہزار گز کے برابر رکھا گیا۔ برطانوی سامراج نے  اکبری گز \عددی{33} انچ مقرر کیا۔ یوں  کوس تقریباً\عددی{\SI{419}{\meter}}کے برابر ہے۔ مغلیہ دور کی شاہراہوں (جی ٹی روڈ) پر جگہ جگہ اب بھی کوس مینار کھڑے نظر آتے ہیں۔ دریائے سندھ پر اٹک کے قریب پرانے پُل کے نزدیک ایسا ایک مینار اب بھی کھڑا ہے۔ مغلیہ دور میں پرسنگ (یا فرسخ)   بھی استعمال ہوتا رہا۔ پرسنگ  وہ فاصلہ ہے جو  گھوڑا چل کر ایک گھنٹے میں طے کرتا ہے۔ یوں ایک پرسنگ تقریباً \عددی{3} میل کے برابر ہے۔ موٹروے پر لاہور سے ملتان تک کا فاصلہ\عددی{\SI{350}{\kilo\meter}}اور\عددی{\SI{306}{\meter}}ہے۔ یہ فاصلہ کتنے کوس اور کتنے پرسنگ ہے؟  

جواب: \عددی{83.6}کوس،\عددی{73} پرسنگ  
\انتہا{سوال} 
%------------------------------------
%Q34 p11
\ابتدا{سوال} 
موجودہ دور میں ایک گز\عددی{36}  \اصطلاح{انچ }\فرہنگ{انچ}\حاشیہب{inch}\فرہنگ{inch} یعنی تین فٹ کے برابر مانا جاتا ہے۔ پاکستان میں  اراضی  کی پیمائش ایکڑ، کنال، مرلہ میں کی جاتی ہے۔ ایک ایکڑ میں آٹھ کنال اور  ایک کنال\عددی{20}مرلہ کے برابر ہے۔ ایک کنال ٹھیک\عددی{605}مربع گز یعنی\عددی{\SI{505.857}{\meter\squared}}  کے برابر ہے۔  (ا)\عددی{48.5}کنال کا رقبہ کتنے مرلہ ہوگا،   (ب) یہی رقبہ کتنے مربع گز ہوگا؟  (ج)\عددی{1135}ایکڑ کا رقبہ چوکور  ہے۔ اس کا ایک کنارہ کتنے گز ہوگا؟ 

جواب: (ا)\عددی{970}مرلہ ، (ب)\عددی{29343}مربع گز، (ج)\عددی{2344}گز
\انتہا{سوال} 
%----------------------------------------------------------------
%Q35 p11
\ابتدا{سوال} 
ایک  \اصطلاح{من }\فرہنگ{من}\حاشیہب{maund}\فرہنگ{maund} وزن دس \اصطلاح{ دھڑی }\فرہنگ{دھڑی} کے برابر، ایک دھڑی  چار \اصطلاح{ سیر }\فرہنگ{سیر} کے برابر، ایک سیر چار \اصطلاح{   پاو }\فرہنگ{پاو} کے برابر،  اور ایک پاو  چار \اصطلاح{ چھٹانک }\فرہنگ{چھٹانک} کے برابر ہے۔ ایک من ٹھیک\عددی{\SI{37.3242}{\kilo\gram}}کے برابر ہے۔ ایک شخص تین من دو دھڑی پانچ سیر تین پاو  اور دو چھٹانک  گندم   خریدتا ہے۔ گندم کی کمیت\عددی{\si{\kilo\gram}}میں کتنی  ہے؟ 

جواب: \عددی{\SI{124.9194319}{\kilo\gram}}
\انتہا{سوال} 
%---------------------
%Q36 p11
\ابتدا{سوال} 
سونے کے وزن کی اکائی  \اصطلاح{تولہ  }\فرہنگ{تولہ}\حاشیہب{tola}\فرہنگ{tola}ہے۔ ایک تولہ\عددی{12} \اصطلاح{ماشوں }\فرہنگ{ماشہ} کے برابر ہے۔ ایک تولہ ٹھیک\عددی{\SI{11.6638038}{\gram}}کے برابر ہے۔ آپ سنار سے پانچ تولہ اور سات ماشے کا سونا خریدتے ہیں۔ یہ کتنے گرام کے برابر ہوگا؟ 

جواب:  \عددی{\SI{65.12288}{\gram}}
\انتہا{سوال}
%---------------------------------------------
%Q37 p11
\ابتدا{سوال} 
چینی کے  کعبی  دانے کا ضلع \عددی{\SI{1}{\centi\meter}}ہے۔ اگر   کعبی ڈبے میں ایک \ترچھا{مول } چینی کے کعبی دانے  ہوں،  ڈبے   کا ضلع کیا ہو گا؟ (ایک مول  \عددی{6.02e23} کو کہتے ہیں۔) 
\انتہا{سوال}
%---------------------------------------
%Q38 p11
\ابتدا{سوال} 
ایک یوسف زئی خان کے پاس\عددی{12}ہزار جریب کی اراضی ہے۔ ایک جریب چار کنال کے برابر ہے، اور ایک کنال\عددی{\SI{101.1716}{\meter\squared}}کے برابر ہے۔ یہ اراضی  مربع ہے۔ اس مربع کا ضلع کتنے\عددی{\si{\kilo\meter}}ہوگا؟ 

جواب: \عددی{\SI{1.102}{\kilo\meter}}
\انتہا{سوال} 
%----------------------------------------
%Q39 p11
\ابتدا{سوال} 
برطانوی  گیلن امریکی  گیلن سے مختلف ہے: ایک برطانوی  گیلن \عددی{\num{4.546090}} لیٹر ، جبکہ ایک امریکی   
گیلن \عددی{\num{3.7854118}} لیٹر کے برابر ہے۔ برطانیہ میں خریدی گئی  گاڑی  کے بنانے والے دعویٰ کرتے ہیں کہ  ان کی گاڑی  ایک گیلن
 تیل  میں\عددی{\SI{40}{\kilo\meter}}فاصلہ طے کرتی ہے۔ ایک شخص  گاڑی خرید کر امریکہ لے جاتا ہے۔ امریکہ میں\عددی{750}  میل فاصلہ  (ا) کتنے گیلن تیل میں طے ہونا متوقع ہے اور  (ب) گاڑی حقیقتاً کتنا تیل استعمال کرے گی؟ 
\انتہا{سوال} 
%---------------------------------------
%Q40 p11
\ابتدا{سوال} 
اس باب میں پیش کیے گئے مواد کو استعمال کرتے ہوئے دریافت کریں کہ\عددی{\SI{1.0}{\kilo\gram}}ہائیڈروجن حاصل کرنے کے لیے ہائیڈروجن جوہروں کی تعداد کتنی ہوگی۔  ہائیڈروجن جوہر کی کمیت\عددی{\SI{1.0}{\atomicmassunit}}ہے۔ 
\انتہا{سوال}
%----------------------
%Q41 p11
\ابتدا{سوال} 
ایک ڈرمی جس کی لمبائی دو   فٹ ہے کا حجم\عددی{100} لیٹر ہے۔ اس کا رداس\عددی{\si{\centi\meter}}میں کیا  ہوگا؟ 

جواب: \عددی{\SI{22.85}{\centi\meter}}
\انتہا{سوال} 
%----------------------------------------------------
%Q42 p11
\ابتدا{سوال} 
پانی (\ce{H2O})    کے سالمہ میں دو ہائیڈروجن اور ایک آکسیجن   جوہر پایا جاتا ہے۔ ہائیڈروجن جوہر کی کمیت\عددی{\SI{1.0}{\atomicmassunit}}اور آکسیجن جوہر کی کمیت\عددی{\SI{16}{\atomicmassunit}}ہے۔  (ا) پانی کے سالمہ کی کمیت کتنے\عددی{\si{\kilo\gram}} ہو گی؟  (ب) دنیا کے  تمام بحر میں ( تخمیناً)  کل\عددی{\SI{1.4e21}{\kilo\gram}}پانی پایا جاتا ہے۔ اس پانی میں کتنے سالمے  ہوں گے؟ 
\انتہا{سوال} 
%---------------------------------
%Q43 p11
\ابتدا{سوال} 
ایک شخص خوراک کی مقدار کم کر کے ایک ہفتہ میں\عددی{\SI{2.3}{\kilo\gram}}کمیت گھٹا سکتا ہے۔ کمیت گھٹنے کی شرح \عددی{\si{\milli\gram\per\second}}میں لکھیں۔ 
\انتہا{سوال}
%-----------------------------
%Q44 p11
\ابتدا{سوال}
پانی کی کثافت\عددی{\SI{1.0e3}{\kilo\gram\per\meter\cubed}}ہے۔ سوال  \حوالہ{سوال_پیمائش_ایکڑ_فٹ} میں دیے گئے شہر پر کتنی  کمیت کا پانی برسا؟ 
\انتہا{سوال}
%----------------------------------------
%Q45 p11
\ابتدا{سوال} 
(ا)  بعض اوقات خوردبینی  طبیعیات میں وقت کی اکائی \اصطلاح{ شیک }\فرہنگ{شیک}\حاشیہب{shake}\فرہنگ{shake} استعمال کی جاتی ہے ، جو  تقریباً\عددی{\SI{e-8}{\second}}کے برابر ہے۔ کیا  سال میں سیکنڈوں سے زیادہ سیکنڈ میں شیک پائے جاتے ہیں؟  (ب)  زمین پر بنی آدم \عددی{e6}سال گزار چکا ہے، جبکہ کائنات\عددی{e10}سال پرانی ہے۔ اگر کائنات کی پوری عمر  ایک \قول{ کائناتی دن } تصور کی جائے ، جس میں \قول{ کائناتی سیکنڈ } کی تعداد اتنی ہی ہو جتنی  ایک سادہ دن میں سادہ سیکنڈوں کی تعداد ہوتی ہے، تو بنی آدم  کتنے سیکنڈ سے زمین پر  ہے؟ 
\انتہا{سوال} 
%------------------------------------------
%Q46 p11
\ابتدا{سوال} 
کوئلے کی   کھان سے سالانہ\عددی{\SI{26}{\meter}}گہرائی تک\عددی{200}ایکڑ رقبے کا کوئلہ حاصل کیا جاتا ہے۔ یہ کوئلہ 
کتنے \عددی{\si{\kilo\meter\cubed}}کے برابر ہے؟ 
\انتہا{سوال} 
%-----------------------------------
%Q47 p11
\ابتدا{سوال} 
سورج اور زمین کے درمیان اوسط فاصلے  کو ایک \اصطلاح{ فلکیاتی اکائی }\فرہنگ{فلکیاتی اکائی}\حاشیہب{astronomical unit}\فرہنگ{astronomical unit} کہتے ہیں۔ روشنی کی رفتار تقریباً\عددی{\SI{3.0e8}{\meter\per\second}}ہے۔ روشنی کی رفتار  فلکیاتی اکائی فی منٹ  میں لکھیں۔ 
\انتہا{سوال} 
%------------------------
%Q48 p11
\ابتدا{سوال} 
  مشرقی نیولا کی کمیت تقریباً\عددی{\SI{75}{\gram}}ہوتی ہے، جو تخمیناً \عددی{7.5}مول جوہر کے برابر ہے۔ (جوہروں کا ایک مول \عددی{6.02e23}جوہر  کے برابر ہے۔)  نیولا کے  جسم میں  جوہر کی اوسط کمیت کو جوہری کمیتی اکائی (\عددی{\si{\atomicmassunit}})  میں لکھیں۔ 
\انتہا{سوال} 
%---------------------------------
%Q49 p11
\ابتدا{سوال} 
جاپان میں لمبائی کی روایتی اکائی  \اصطلاح{کین }\فرہنگ{کین}\حاشیہب{ken}\فرہنگ{ken} ہے۔ (ایک کین \عددی{\SI{1.97}{\meter}} کے برابر ہے۔)(ا)  مربع کین اور مربع میٹر  کی نسبت اور  (ب) کعبی کین اور کعبی میٹر کی  نسبت کیا ہے؟ ایک بیلن ، جس کی بلندی\عددی{5.05} کین  اور رداس\عددی{3.00} کین ہے ، کا حجم  (ج)  کعبی کین اور   (د)   کعبی  میٹر میں  کیا  ہوگا؟ 
\انتہا{سوال} 
%--------------------------------
%Q50 p11
\ابتدا{سوال} 
آپ مشرق کی طرف\عددی{24.5} میل کشتی  چلاتے ہیں  جبکہ آپ کو\عددی{24.5}سمندری میل سفر کرنا تھا۔ ایک سمندری  میل \عددی{1.1508}  زمینی میل کے برابر ہے۔ آپ اصل مقام سے کتنا دور ہیں؟
\انتہا{سوال}
%----------------------------------------
