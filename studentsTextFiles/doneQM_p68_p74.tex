%68, 69
\حصہ{ڈراک ڈیلٹا تفاعلی مخفی توانائ}

\جزحصہ{مقید حالات اور بکھرتے حالات}
 ہم وقت کے غیر طابع شوڈنگر مساوات کے دو مختلف حل دیکھ چکے ہیں،  لا متناہی چوکور کون اور ہارمونی مرتعش کے لیے یہ معمول پر لانے کے قابل تھے اور انہیں غیر مسلسل عشاریہ \عددی{ n  }سے نام دیا جاتا ہے ۔  آزاد ذرے کے لیے یہ معمول پر لانے کے قابل نہیں ہیں اور انہیں استمراری متغیر\عددی{   k} سے نام دیا جاتا ہے، ان میں سے پہلے اقسام طبعی طور پر قابل حاصل حل کو ظاہر کرتے ہیں جبکہ دوئم صورت ایسا نہین کرتی ہیں لیکن دونوں صورتوں میں وقت کے طابع شودنگر مساوات کے عمومی حل ساکن حالات کا خطی جوڑ ہوگا۔ پہلی قسم میں یہ جوڑ مختلف \عددی{ n  } دیتے ہوۓ مجموعہ ہوگا جبکہ دوسری صورت میں یہ \عددی{   k} پر تکمل ہوگا۔ اس امتیاز کی طبعی اہمیت کیا ہے ؟ کلاسیکی میقینات میں یہ دوری وقت کے غیر طابع مخفی توانائ دو مکمل طور پر مختلف حرکت پیدا کرتے ہیں اگر دونوں جانب ذرے کی کل توانائ ذرے کی \( V(x)\)سے \( V(x)\) ذیادہ بلندی کی طرف پہنچے (شکل  \حوالہء{شکل2.12 a } تب یہ ذرہ اس مخفی توانائ کے کنویں میں پھنسا رہے گا ۔ یہ واپس پلٹنے والے نقاط کے درمیان آگے پیچھے حرکت کرتا رہتا ہے لیکن اس کنویں سے باہر نہیں نکل سکتا ماسواۓ آپ اسے اضافی توانائ فراہم کریں جسکی ابھی ہم بات نہیں کر رہے ۔ ہم اسے مقید حالت کہتے ہیں۔ اس کے برعکس اگر ایک جانب یا دونوں جانب \( V(x)\)  سے   \عددی{E} کی قیمت تجاوز کر جاۓ تو ذرہ لامتناہی سے آتے ہوۓ مخفی توانائ کے ذیر اثر رفتار ذیادہ یا کم کرتا ہے اور واپس لا متناہی تک حرکت کرتا ہے ،  شکل  \حوالہء{شکل2.12  }یہ مخفی توانائ میں پھنس نہیں سکتا ماسواۓ اس صورت  کے اس کی توانائ گھٹے لیکن ہم یہاں بھی ایسی صورت کی بات نہیں کر رہے ہم اسے بکھرتا حال کہتے ہیں۔ کچھ مخفی توانائیاں صرف مقید حال ہیدا کرتی ہیں مثلاً ہارمونی  مرتعش کچھ صرف ببکھرت حال پیدا کرتے ہیں مثلاً بتدریج بڑھتا مخفی توانائ جو کہیں پر بھی گھٹتا نا ہو اور چند دونوں اقسام کے حال پیدا کرتے ہیں۔ جو ذرہ کی توانائ پر منحصر ہوتے ہیں ۔ شوڈنگر مساوات کے اقسام کی دو قسمیں ٹھیک انہیں مقید اور بکھرتے حال کو ظاہر کرتی ہیں ۔ کوانٹم کے دائرہ کار میں یہ فرق اس سے بھی ذیادہ ظاہر ہے جہاں  سرنگ زنی   جس پر کٓہم کچھ دیر میں بات کریں گے ۔ ایک ذرے کو کو مخفی توانائ کے رکاوٹ کے اندر سے گزرنے دیتا ہے ،لہزٰہ صرف لامتناہی پر مخفی توانائ اہمیت رکھتی ہے ۔  شکل  \حوالہء{شکل c 2.12  } 
\begin{align}
\begin{cases}
E\textless[V(-\infty) and V(+\infty)]\Rightarrow bound state\\
E\textgreater[V(-\infty) or V(+\infty)]\Rightarrow scattering state
\end{cases}
\end{align} 
حقیقت میں عموماً مخفی توانائ لمتناہی تک صفر تک پہنچتی ہے ایسی صورت میں مسلمہ معیار مزید سادہ صورت اختیار کرتی ہے
\begin{align}
\begin{cases}
E\textless0\Rightarrow bound state\\
E\textgreater0\Rightarrow scattering state
\end{cases}
\end{align}
کیونکہ   \عددی{x\shortrightarrow\pm\infty} پر لا متناہی چوکور کنویں اور ہارمونی مرتعش کی مخفی توانائ لمتناہی تک پہنچتی ہے لہٰذہ یہ صرف مقید حالات پیدا کرتے ہیں جبکہ آذاد ذرے کی مخفی توانائ ہر مقام پر صفر ہوتی ہے لہٰذہ یہ صرف بکھرتے حالات پیدا کرتا ہے ۔  اس حصہ میں اور اگلے حصہ میں ایسی مخفی توانائیوں پر غور کریں گے جو ایسی دونوں اقسام کے حالات پیدا کرتے ہیں ۔ 
%70
%=======The Delta function Well========
\حصہ{ڈیلٹا تفاعلی کنواں} 
ڈراک ڈیلٹا تفاعل سے مبدا پر ایک ایسی لا متناہی حد تک کم چوڑائ کی نوکیلی تفاعل ہے جس کی بلندی لا متناہی ہو اور جس کا رقبہ اکائ ہو  شکل \حوالہء{شکل 2.13 } 
\begin{align}
\delta(x)=
\begin{cases}
0&if x\neq 0\\
\infty & if x=0
\end{cases}
with \int_{-\infty}^{+\infty}\delta(x)\dif{x}=1
\end{align} 
کیونکہ یہ  \عددی{x=0} پر یہ متناہی نہیں ہے لہٰذہ تکنیکی طور پر ایک تفاعل کو ظاہر نہیں کرتا ریاضی دان اسے تامیمی تفاعل یا تامیمی تقسیم کہتے ہیں ، تاہم اس کا تصور طبعیات کے نظریات میں اہم کردار ادا کرتا ہے ۔  مچال کے طور پر برقی حرکیات کے میدان میں نقطہ بار کی کثافت بار ڈیلٹا تفاعل ہوگا، آپ دیکھ سکتے ہیں کہ  \عددی{\delta(x-a)} نقطہ \عددی{a}  پر ایک رقبے کا نوکیلی تفاعل ہوگا کیونکہ  \عددی{\delta(x-a)} اور ایک سادہ تفاعل کا حاصل ضرب نقطہ \عددی{a}    کے علاوہ ہر مقام پر صفر ہوگا لہٰذہ یہ 
\begin{align}
f(x)\delta(x-a)-f(a)\delta(x-a)
\end{align}
ہوگا۔ بالخصوص درج ذیل لکھا جا سکتا ہے جو کہ ڈیلٹا تفاعل کی اہم ترین خصوصیت ہے 
\begin{align}
\int_{-\infty}^{+\infty}f(x)\delta(x-a)\dif{x}=f(a)\int_{-infty}^{+\infty}\delta(x-a)\dif{x}=f(a)
\end{align}
 تکمل کی علامت کے اندر یہ نقطہ \عددی{a}   پر تفاعل کی قیمت اٹھاتا ہے ۔ ضروری نہیں کہ تکمل  \عددی{-\infty} سے   \عددی{+\infty} لیا جاۓ،  صرف اتنا ضروری ہے کہ تکمل کے دائرہ کار میں  \عددی{-\epsilon \,to\, a\,+\epsilon} شامل ہو۔ 
\begin{align}
V(x)=-\alpha\delta(x)
\end{align}
آئیں درج ذیل صورت کے مخفی توانائ پر غور کریں جہاں  \عددی{\alpha} ایک مثبت مستقل ہے۔ یہاں یہ جان لینا ضروری ہے کہ یہ ایک مصنوعی مخفی توانائ ہے جیسا کہ لا متناہی چوکور کنواں تھا لیکن اس کے ساتھ کام کرنا حیرت کن حد تک سادہ ثابت ہوتا ہے۔ جو کم سے کم اہدیلی پریشانیوں کے ساتھ بنیادی نظریے پر روشنی ڈالتا ہے۔  ڈیلٹا تفاعلی کنویں کے لیے شوڈنگر مشاوات درج ذیل روپ اختیار کرتی ہے،
\begin{align}
-\frac{\hslash^{2}}{2m}\frac{\dif^{2}\psi}{\dif{x^{2}}}-\alpha\delta(x)\psi=E\psi
\end{align} 
جو مقید حالات  \عددی{(E\textless 0)} اور بکھرت حالات  \عددی{(E\textgreater 0)} دونوں پیدا کرتی ہے۔ ہم پہلے مقید حالات پر غور کریں گے  \عددی{(x\textless 0)} میں خطی  \عددی{V(x)=0} ہوگا۔ لہٰذہ 
\begin{align}
\frac{\dif^{2}\psi}{\dif{x^{2}}}=-\frac{2mE}{\hslash^{2}}\psi=k^{2}\psi
\end{align}
جہاں
\begin{align}
k=\frac{\sqrt{-2mE}}{\hslash}
\end{align}
ہوگا جہاں \عددی{E} کو منفی تصور کرتے ہوۓ \عددی{K}  کی قیمت حقیقی اور مثبت ہوگی۔  مساوات \حوالہء{مساوات 2.116 } کا عمومی حل 
\begin{align}
\psi(x)=Ae^{-kx}+Be^{kx}
\end{align}
ہوگا جہاں  \عددی{x\arrow-\infty} پر پہلا جز لا متناہی کی طرف بڑھتا ہے لہٰذہ ہمیں \عددی{A=0}              منتخب کرنا ہوگا۔ 
\begin{align}
\psi(x)=Be^{kx}\quad (x\textless0)
\end{align}
خطہ  \عددی{x\textgreater0} میں بھی \عددی{V(x)} صفر ہوگا اور عمومی حل \عددی{F e(-kx)+G e(kx)}  روپ کا ہوگا۔ اس بار \عددی{x\arrow+\infty} پر دوسرا جز لا متناہی کی طرف بڑھتا ہے 
\begin{align}
\psi(x)=Fe^{-kx}\quad (x\textgreater0)
\end{align}
 ہمیں   \عددی{x=0} پر موضوع سرحدی شرائط استعمال کرتے ہوۓ ان دونوں تفاعلوں کو ایک دوسرے کے ساتھ جوڑنا ہوگا۔ میں     \عددی{\psi} کی معیاری سرحدی شرائط پہلے بیان کر چکا ہوں
\begin{align}
\begin{cases}
1. \psi & is\, always\, continuous\\
2. \dif{\psi}/\dif{x} & is\,continuous\,except\,at\,points\,where\,the\,potential\,is\,infinite.
\end{cases}
\end{align}
 یہاں پہلے سرحدی شرائط کے تحت \عددی{F=B}  ہوگا 
\begin{align}
\psi(x)=
\begin{cases}
Be^{kx}&(x\le0)\\
Be^{-kx}&(x\ge0)
\end{cases}
\end{align} 
%=======Page 72=======
سائ ایکس کی شکل کو (سکل2.14) میں ترسیم کیا گیا ہے۔ دوسرا سرحدی شرط ہمیں کچھ نہیں بتاتا یہ لا متناہی چوکور کنویں کی طرح خود نمائ صورت ہے جہاں (وی) جوڑ پر لا متناہی ہے۔اور تفاعل کی ترسیل سے ظاہر ہہے کہ (مساوات) پر اس میں بل پایا جاتا ہے۔ مزید اب تک کی کہانی میں ڈیلٹا تفاعل کا کوئ کردار نہیں رہا ہے۔ ظاہر ہے کہ (ایکس برابر صفر میں ) پر سائ کی تفرق میں عدم استمرار اسی ڈیلٹا تفاعل سے قائم ہوگا میں یہ عمل آپ کو کر کے دکھاتا ہوں جہاں آپ یہ بھی دیکھیں گے کہ (مساوات ) عموماً استمراری ہوگا۔ ہم منفی (مساوات) تا مثبت ایبسلون شوڈنگر مساوات کا تکمل لے کر (مساوات 2.123) کرتے ہیں ۔ ان میں سے پہلے تکمل میں دونوں آخرے نقطوں پر (مساوات لیا جاۓ گا) آخری تکمل اس پٹی کا رقبہ ہے جس کا قد متناہی اور اجسکی چوڑائ(مساوات) کرنے سے صفر تک پہنچتی ہے لہٰزہ یہ تکمل صفر ہوگا۔ یوں درج ذیل ہوگا (2.14) عمومی طور پر دائیں ہاتھ کا حد صفر کے برابر ہوگا ۔ لہٰذہ (مساوات) عمومی طور پر استرماری ہوگا، لیکن جب سرحد پر (مساوات) لا متناہی ہو تب یہ دلیل قابل قبول نہیں ہوگی۔ بالخصوص (مساوات) کی صوردت میں مساوا ت 2.113
 درج ذیل صورت اختیار کرتی ہے (حوالہ مساوات2.15)) موجودہ صورت مساوات 2.122 درج ذیل ہوگا (مساوات) لہٰذہ (مساوات) ہوگا۔ یوں (مساوات) ہوگا۔اور مساوات (2.125) کہتا ہے کہ(حوالہ مساوات2.126)) جب کہ مساوات 2.117 درج ذیل اجازتی توانای دے گی (حوالہ مساوات2.127) آخر میں ہم سائ کو معمول پر لاتے ہیں (مساوات ) یوں اپنی آسانی کے لیے مثبت حقیقی جزر کا انتخاب کرتے ہوۓ درج ذیل حاصل ہوگا۔ (مساوات 2.18) ظاہری طور پر ڈیلٹا تفاعل کے () کے قطع نظر یہ ٹھیک ایک مقید حال دیتا ہے۔ ((حوالہ مساوات)2.129)
ہم (مساوات ) کی صورت میں بکھرتے حال کے بارے میں کیا کہ سکتے ہیں شوڈنگر مساوات (مساوات) کی صورت میں درج ذیل روپ اختیار کرتی ہے (مساوات ) جہاں (حوالہ مساوات2.130) حقیقی اور مثبت ہے ۔ اسکا عمومی حل  
(حوالہ مساوات 2.131)  ہے۔  جہاں کوئ بھی جز لا متناہی تک نہیں بڑھتا لہٰذۃ ہم انہیں رد نہیں کر سکتے ہیں (مساوات 2.132) کے لیے درض ذیل ہوگا ۔ نقطہ (مساوات) پر ایک کی استرماریت کے بنا ((حوالہ مساوات ) 2.133) اس کے تفرقات (مساوات)  ، ہوں گے لۃٰذہ (مساوات) ہوگا۔ ساتھ ہی (مساوات) ہوگا لۃٰذہ دوسری سرحدی شرط مساوات (مساوات 2.125) کہتا ہے (حوالہ مساوات )2.134  یا مختصرا	 (حوالہ مساوات 2.135)  دونوں سرحدی شرائط مسلط کرنے کے بعد ہمارے پاس دو مساوات (مساوات 1.33 اور 1.35) جب کہ چار نو معلوم مستقل (اے- بی ----) بلکہ (کے) شامل کرتے ہوۓ پانچ نا معلوم مستقل ہوں گے۔ یہ معمول پر لانے کے قابل حال نہیں ہیں لہٰذہ معمول پا لانا مدد گار ثابت نہیں ہوگا۔ بہتر یہ ہوگا کہ ہم رک کر ان مختف مستلوؐ کی طبعی اہمیت پر غور کریں اگر آپ کو یاد ہو (مساوات) کے ساتھ وقت کے طابع جز (مساوات) جوڑے سے دائیں رخ حرکت کرتا تفاعل موج پیدا ہوتا ہے ۔ اسی طرح (مساوات) بائیں رخ حرت کرتا ہوا موج دیتا ہے۔ یوں مساوات (2.131) میں مستقل (اے) بائیں سے آتے ہوۓ معج کا ہیتا دے گا، جبکہ (بی) بائیں رخ واپس لاٹتے ہوۓ موج کا ہیتا دے گا۔ دائیں رخ چلتا ہوا موج کا ہیتا (مساوات 2.132) دے گا۔ جبکہ دائیں سے آتے ہوے موج کا ہیتا (جی) دے گا ۔ شکل 2.15 دیکھیں۔ بکھراؤ کے تجربہ میں عموماً ایک رخ سے ذرات پھینکے جاتے ہیں مثلاً بائیں سے ایسی صورت میں دائیں جانب سے آتی ہوے موج کا ہیتا صفر ہوگا (مساوات 2.136) آمدی موج کا ہیتا (اے) منوکس موج کا ہیتا (بی) جبکہ تسیلے موج کا ہیتا (ایف) ہوگا   مساوات 2.133 اور مساوات ) 2.135
 کو (بی اور ایف کے لیے حل کر کے درج ذیل حاصل ہوگا۔ (حوالہ مساوات 2.137) ہوگا اگر آپ دائیں ست بکھراؤ جاننا چاہیں تب (مساوات) ہوگا۔ آمدی موج کا ہیتا ()
جی ہوگا منعکس موج کا ہیتا ایف اور ترسیلی موج کا ہیتا بی ہوگا۔

%=====================#######===========






























 


















