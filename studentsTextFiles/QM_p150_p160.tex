\documentclass{book}
\usepackage{fontspec}
\usepackage{makeidx}
\usepackage{amsmath}                                                                         %\tfrac for fractions in text
\usepackage{amssymb}    
\usepackage{gensymb}  
\usepackage{amsthm}      						%theorem environment. started using in the maths book
\usepackage{mathtools}
\usepackage{multicol}
\usepackage{commath}									%differentiation symbols
\usepackage{polyglossia}    
\setmainlanguage[numerals=maghrib]{arabic}     %for english numbers use numerals=maghrib, for arabic numerals=arabicdigits
\setotherlanguages{english}

\newfontfamily\arabicfont[Scale=1.0,Script=Arabic]{Jameel Noori Nastaleeq} 
\setmonofont{DejaVu Sans Mono}                                                                  %had to add this and the next line to get going after ubuntu upgrade
\let\arabicfontt\ttfamily                                                                                  %had to add this and the above line to get going after ubuntu upgrade
\newfontfamily\urduTechTermsfont[Scale=1.0,Script=Arabic]{AA Sameer Sagar Nastaleeq Bold}
\newfontfamily\urdufont[WordSpace=1.0,Script=Arabic]{Jameel Noori Nastaleeq}
\newfontfamily\urdufontBig[Scale=1.25,WordSpace=1.0,Script=Arabic]{Jameel Noori Nastaleeq}
\newfontfamily\urdufontItalic[Scale=1.25,WordSpace=1.0,Script=Arabic]{Jameel Noori Nastaleeq Italic}
\setlength{\parskip}{5mm plus 4mm minus 3mm}
\begin{document}
%150-153
ہائیڈروجن جوہر لے فضائی تفاعل  امواج کو 3 کوانٹم اعداد M اور N٫L سے نام دیا جاتا ہے 
 \[\psi_{n,l,m}(r,\theta,\phi)=R_{nl}(r)Y_{l}^{m}(\theta,\phi)\]
 جہاں مساوات 4.36 اور 4.60 کو دیکھتے ہوئے
 \[R_{n,l}(r)=\frac{1}{r}\rho^{l+1}e^{-\rho}v(\rho)\] 
 ہوگا. جبکہ وی رو متغير رو میں جے بلندتر
\(j_{max}=n-l-1\)
درجہ کا كثير رکنی ہوگا جس کے عددی سر جنہے معمول پر لانا باقی ہوگا  درجہ ذیل کلیہ توالی دے گا
 \[c_{j+1}=\frac{2(j+l+1-n)}{(j+1)(j+2l+2)}c_{j}\]
كم سے كم توانائی کا حال جسے زمینی حال کہتے ہیں٬ کے لیے
 \(n=1\) 
ہوگا. تبی مستقلوں کی یہ قیمتیں پر کرتے ہوئے درجہ ذیل حاصل ہوگا
 \[\boxed{E_{1}=-\big[\frac{m}{2\hbar^{2}}\big(\frac{e^{2}}{4\pi\epsilon}\big)^{2}\big]=13.6\text{eV}}\]
. اس سے ظاہر ہے کہ  ہائیڈروجن کی توانائی بھندن 13.6 eV ہے. یہ وہ توانائی ہے جو زمینی حال میں الیکٹرون کو محیا کرنے سے ایٹم بدارا بن جائے گا. مساوات 4.67 کے تحت l=0  لہذا m=0  ہوگا. مساوات 4.29 دیکھے یوں درجہ ذیل ہوگا
 \[\psi_{100}(r,\theta,\phi)=R_{10}(r)Y_{0}^{0}(\theta,\phi)\]
 کلیہ توالی پہلے جز پرشی رکھ جاتا ہے. j=0 کے لئیے مساوات 4.76 سے
 \(c_{1}=0\) 
 حاصل ہوگا. یوں وی رو ایک مستقل 
  \(c_{0}\)
  ہوگا. لہذا درجہ ذیل ہوگا
   \[R_{10}(r)=\frac{c_{0}}{a}e^{-r/a}\]
  . اس کو مساوات 4.31 کے تحت معمول پر لانے
سے
..\[\int_{0}^{\infty}\abs{R_{10}}^{2}r^{2}\dif{r}=\frac{c_{0}^{2}}{a^{2}}\int_{0}^{\infty}e^{-2r/a}r^{2}\dif{r}=\abs{c_{0}}^{2}\frac{a}{4}=1\]
حاصل ہوگا. ساتھ ہی
\(c_{0}=2/\sqrt{a}\)
لہذا بائیڈروجن کا زمینی حال درجہ ذیل ہوگا
\[\psi_{100}(r,\theta,\phi)=\frac{1}{\sqrt{\pi a^{3}}}e^{-r/a}\]
اسی طرح n=2  کے لئے توانائی درجہ ذیل ہوگی
\[E_{2}=\frac{-13.6\text{eV}}{4}=-3.4\text{eV}\]
چونکہ
 \(m=2\) 
کے لئے یا 
\(l=0\) 
ہوگا جو
\( m=0\) 
دیتا ہے يا
 \(l=1\) 
ہوگا جو 1 اور M= -1,0 دیتا ہے لہذا چار مختلف حالات کی توانائی E2 ہوگی. کلیہ توالی مساوات 4.76  L=0 کے لیئے درجہ ذیل دے گا:
\[c_{1}=-c_{0} (j=0) \quad c_{2}=0 (j=1)\]
لہذا
\(v(\rho)=c_{0}(1-\rho)\)
اور درجہ ذیل ہونگے
 \[R_{20}(r)=\frac{c_{0}}{2a}\big(1-\frac{r}{2a}\big)e^{-r/2a}\]
دیہان رہے کہ مختلف کوانٹم اعداد l اور N کے لئے پھیلاؤ کے عددی سر
 \(c_{j}\)
 مکمل تور پر مختلف ہونگے. کلیہ توالی
  \(l= 1\)
  کی صورت میں پہلے جز پر اختتام پذیر ہوگا. وی رو ایک مستقل ہوگا لہذا درجہ ذیل حاصل ہوگا.
   \[R_{21}(r)=\frac{c_{0}}{4a^{2}}re^{-r/2a}\]
ہر ایک صورت میں معمول پر لانے سے
 \(c_{0}\) 
تعین ہوگا.\\
سوال 4.11 دیکھے. کسی بھی n کے لئے مساوات 4.67 کی بلا تضاد L کی ممکنہ قیمتیں درجہ ذیل ہوں گی 
\[0,1,2\dotsc n-1\]
جبکہ ہر
 \(l \)
کے لئیے 
\(m\)
 کی ممکنہ قیمتوں کی تعداد
 \( 2l+1\) 
 ہوگی. (مساوات 4.29) . لہذا
  \(E_{n}\)
  سطح کی توانائی کے لئے کل انحطاطيت درجہ ذیل ہوگی.
\[d(n)=\sum_{l=0}^{n-1}(2l+1)=n^{2}\]
كثير رکنی وی رو جو مساوات 4.76 سے حاصل ہو گی ایک ایسا تفاعل ہے جس سے عملی ریاضی دان بخوبی واقف ہیں. ما سوائے معمول زنی کے اسے تجذيل لکھا جاسکتا ہے.
 \[v(\rho)=L_{n-l-1}^{2l+1}(2\rho)\]
 جہاں
  \[L_{q-p}^{p}(x)\equiv(-1)^{p}\big(\frac{\dif}{\dif{x}}\big)^{p}L_{q}(x)\] 
 ایک شریک لاگیغ  كثير رکنی ہے جبکہ 
\[ L_{q}(x)\equiv e^{x}\big(\frac{\dif}{\dif{x}}\big)^{q}(e^{-x}x^{q})\]
 لاگیغ كثير رکنی ہے. جدول 4.5 میں چند ابتدائی لاگيغ كثور رکنیا پیش کی گئی ہیں.  جبکہ جدول 4.6 میں چند شریک لاگيغ كثير رکنیا پیش کئے گئی ہیں. جدول 4.7 میں چند ابتدائی رداسی تفاعل امواج پیش کئے گئے ہیں جنہیں شکل 4.4 میں ترسیم کیا گیا ہے. ہائیڈروجن کے معمول شده تفاعل امواج درجہ ذیل ہیں.
  \[\boxed{\psi_{nlm}=\sqrt{\big(\frac{2}{na^{3}}\big)\frac{(n-l-1)!}{2n[(n+1)!]^{3}}}e^{-r/na}\big(\frac{2r}{na}\big)^{l}[L_{n-l-1}^{2l+1}(2r/na)]Y_{l}^{m}(\theta,\phi)}\]
یہ تفاعل کچ خوفناک ہیں٬ ليكن شكوه نہ کیجیے گا. یہ اُن چند حقیقی نظاموں میں سے ایک ہے جن کا مکمل حل حاصل کرنا ممکن ہے.  دیہان رہے کہ اگرچہ تفاعل امواج تینوں کوانٹم اعداد پر منحصر ہے جبکہ توانائیاں مساوات 4.70 کو صرف n تعین کرتا ہے. یہ کوولومب( coulomb) توانائی کی ایک خاصیت ہے. آپ کو یاد ہوگا کہ کروی كنواں کی صورت میں توانائیاں L پر منحصر تھی (مساوات 4.50).
تفاعل موج باہمی امودی ہوں گے 
.\[\int\psi_{nlm}^{*}\psi_{nlm}r^{2}\sin{\theta}\dif{\theta}\dif{\phi}=\delta_{nn'}\delta_{ll'}\delta_{mm'}\]
 یہ کروی ہارمونیات کی امودیت مساوات 4.33 کی بنا اور
\(n\neq n'\)
کی صورت میں تفاعلات موج کا H کی الگ تلگ امتیازی اعقدار کے امتیازی تفاعل ہونے کی بنا ہے. \\
پائیڈروجن تفاعلات موج کی تصویر کشی آسان کام نہیں ہے. ماحر کیمیا کسافتی اشکال بناتے ہیں جہاں چمک
\(\abs{\psi}^{2}\)
کا راست متناسب ہوتا ہے (شکل 4.5). ان سے زیادہ معلومات مستقل كسافت کے احتمال کی سطحوں کے اشکال دیتی ہے جنہیں پڑھنا نسبتاً مشکل ہوگا. (شکل 4.6).\\
سوال 
4.10\\
کلیه توالی مساوات 4.76 استعمال کرتے ہو ئے  تفال موج
\(R_{30},R_{31}\)
اور
.\(R_{32}\)
حاصل کریں . انہیں معمول پر لانے کی ضرورت نہیں \\
سوال 
4.11\\
جز ( الف )\\
مساوات 4.82 میں دیے گئے
\(R_{20}\)
کو معمول پر لا کر
\(\psi_{200}\)
تیار کریں .\\
جز ( ب )\\
مساوات 4.83 میں دیے گئے
\(R_{21}\)
تو معمول پر لا کر
\(\psi_{211}\)
.
\(\psi_{210}\)
اور
\(\psi{21-1}\)
تیار کریں .\\
سوال 
4.12\\
مساوات 4.88 استعمال کرتے ہوئے ابتدائی چار لا گیغ کثیر رکنیاں حاصل کریں .\\
جز ( ب )\\
مساوات 4.86 ، 4.87 اور 4.88 استعمال کرتے ہوئے
\(n=5\)
اور 
\(l=2\)
کے لیے 
\(v(\rho)\)
تلاش کریں .\\
جز ( ج )\\
کلیہ توالی مساوات 4.76 استعمال کرتے ہوئے
\(n=5\)
اور
\(l=2\)
کے لیے
\(v(\rho)\)
تلاش کریں .\\
سوال 
4.13\\
جز ( الف )\\
ہائیڈروجن جو ہر کے زمینی حال میں الیکٹرون   کے لیے
\(\langle r \rangle\)
اور
\(\langle r^{2} \rangle\)
تلاش کریں . اپنے جواب کو رداس بوہر کی صورت میں لکھیں \\
جز ( ب )\\
ہائیڈروجن جویر کے زمینی حال میں الیکٹرون  کے لیے
\(\langle x \rangle\)
اور
\(\langle x^{2} \rangle\)
تلاش کریں۔ اشاره : آپکو کوئی نیا تکمل حاصل کرنے کی ضرورت نہیں . دھیاں رہے کہ
\(r^{2}=x^{2}+y^{2}+z^{2}\)
ہوگا . لہذا زمینی حال میں تشاکلی کو بروکار لائیں \\
جز ( ج )\\
حال
\(n=2\)
،
\(l=1\)
،
\(m=1\)
کے لیے 
\(\langle x^{2} \rangle\)
تلاش کریں . یہ حال
\(x\)
،
\(y\)
اور
\(z\)
کے لہذ سے تشاکلی نہیں ہے . یہاں
\(x=r\sin{\theta}\cos{\phi}\)
کو
استعمال کرنا ہوگا .\\
سوال 
4.14\\
یا بیڈروجن کے زمینی حال میں
\(r\)
کی کس قیمت کا احتمال زیادہ سے زیادہ ہوگا . اسکا جواب صفر نہیں ہے .\\
اشاره : آپکو پہلے یہ معلوم کرنا ہو گا کہ
\(r\)
اور
\(r+\dif{r}\)
کے درمیان الیکٹرون کے پائے جانے کا احتمال کیا ہوگا۔\\
سوال 
4.15\\
پائیڈروجن جوبر کی حال
\((n=2,l=1,m=1)\)
اور
\((n=2,l=1,m=-1)\)
کی ساکن حال کی خطی جوڑ سے ابتداء کرتا ہے .\\
\[\Psi(r,0)=\frac{1}{\sqrt{2}}(\psi_{211}+\psi_{21-1})\]
جز ( الف )\\
آپ
\(\Psi(r,t)\)
تبار کریں . اس کی ساده ترین صورت حاصل کریں .\\
جز ( ب )\\
مخفی توانائی 
\(\langle V \rangle\)
تو توقعاتی قیمت تلاش کریں . کیا یہ
t
کا تابع ہوگا . اصل کلیہ اور عدد دونوں کو الیکٹرون وولٹ تو صورت میں پیش کریں .\\
جز حصہ 
4.2.2\\
يا ئیڈروجن کا طيف .\\
اصولی طور پر ایک ہائیڈروجن جویر جو ساکن حال
\(\psi_{nlm}\)
میں پایا جاتا ہو بمیشہ کے لیے اس حال میں رہے گا البتہ اس کو دوسرے جوبر کے ساتھ ٹکرانے یا اس پرروشنی ڈال کر چھیڑ نےالیکٹرون کسی دوسری ساکن حال میں عبور کر سکتا ہے . یہ توانائی جذب کرکے زیادہ توانائی والے حال میں منتقل ہو سکنا ہے یا برمکنتیسی فوٹون کو خارج کر کہ کم توانائی والے حال میں منتقل ہو سکتا ہے . عملاً ایسی چھیڑ خانیاں پروفت پائی جائیں گی لہذا کوانٹم چھلانگیں مستقل طور پر ہورہی ہونگی۔ جس کو بنا پایڈروجن سے بروقت روشنی فوٹون کی صورت میں خارخ ہوگی۔ ان فوٹون کی تو انائی ابتدائی اور اختتامی حالات تو توانائی کے فرق کے برابر ہوگا .\\
\[E_{\gamma}=E_{i}-E_{f}=-13.6\text{eV}\big(\frac{1}{n^{2}_{i}}-\frac{1}{n^{2}_{f}}\big)\]
اب قلیہ planck کے تحت photon کی توانائی اسکے تعدد کے راست تنصبی ہوگی 
\[E_{y}=hv\quad[4.92]\]
ساتھ ہی طول موج
\(\lambda=c/v\)
ہوگا
\[\frac{1}{\lambda}=R\big(\frac{1}{n^{2}_{f}}-\frac{1}{n^{2}_{i}}\big),\quad[4.93]\]
جہاں 
\[R=\frac{m}{4\pi{c}\hslash^{3}}\big(\frac{e^{2}}{4\pi\epsilon_{o}}\big)^{2}=1.097\times10^{7}m^{-1}\quad[4.94]\]
Rydberg مستقل کہلاتا ہے ۔
مساوات 4.93 hydrogen کی تعف کا Rydberg قلیہ ہے ۔ یہ قلیہ اُنیسوی صدی میں تجرباتی طور پر اخذ کیا گیا ۔نظریہ bohr کی سب سے بڑی فتح اس قلیے کا حصول ہے جو قدرت کی بنیادی مستقلوں کی صورت میں r کی قیمت دیتی ہے۔ زمینی حال
\((n_{f}=1)\)
میں منتقلی کی بینہ بلاۓبصری شعاع خارج ہوگی جنھیں ماہرِتعف بینی Lyman سلسلہ کہتے ہیں پہلی ہیچین حال
\((n_{f}=2)\)
میں منتقلی سے دیکھائی دینے والی روشنی پیدا ہوتی ہے جسے Balmer سلسلہ کہتے ہیں اسی طرح 
\(n_{f}=3\)
میں منتقلی Paschen سلسلہ دیتی ہے جو زری بصری شعاعیں ہیں وغیرہ وغیرہ۔ شکل 4.7 دیکھیں رہائشی حرارت پر عموماً hydrogen جوہر زمینی حال میں ہونگے ۔اخراجی تعف حاصل کرنے کی خاطر آپکو مختلف ہےچین حالتیں  electron سے آباد کرنی ہونگیں ایسا عموماً گیس میں برقی چنگاری پیدا کرکے کیا جاتا ہے ۔\\
سوال 4.16:\\
Hydrogen جوہر z proton کے مرکز کے گرد طواف کرتے ہوے ایک electron پر مشتمل ہوتا ہے۔ hydrogen میں 
\(Z=1\)
ہوگا جبکے بردارا helium میں 
\(Z=2\)
اور دہری بردارا  lethium میں
\(Z=3\)
ہوگا وغیرہ وغیرہ ۔ hydrogen جوہر کے لئے bohr توانائیاں
\(E_{n}(Z)\)
بندشی توانائياں
\(E_{t}(Z)\)
راداس bohr 
\(a(Z)\)
اور Rydberg مستقل 
\(R(Z)\)
تعین کریں ۔ (اپنے جوابات کو hydrogen کی متعلقہ قیمتوں کے لحاظ سے پیش کریں.)
\(Z=2\)
اور 
\(Z=3\)
کے لئے Lyman سلسلہ مقناطیسی تعف کے کس خطے میں پایا جائے گا ۔ (اشارہ : کسی نیے حساب کی ضرورت نہیں ہے ۔مساوات 4.52 کی مخفی توانائی میں 
\(e^{2}\)
کی جگہ 
\(Ze^{2}\)
استعمال کرنا ہوگا لہٰذا نتائج میں بھی ایسا کیجیے گا ) ۔\\
سوال 4.17:\\
زمین اور سورج کو hydrogen جوہر کا متبادل تاجازبی نظام تصور کریں.\\
جزو الف ) مساوات 4.52 کی جگہ مخفی توانائی تفعال کیا ہوگا ۔ یہاں m کو زمین کی قیمت اور M کو سورج کی قیمت لیں ۔\\
جزو ب )اس نظام کے لئے راداسے bohr 
\(a_{g}\)
کیا ہوگا ۔اس کی عددی قیمت تلاش کریں ۔\\
جزو ج )تاجازبی قلیہ bohr لکھیں راداس 
\(E_{n}\)
کے دائری مدار میں سیارہ کی كلاسیكی توانائی کے برابر
\(r_{o}\)
رکھتے ہوے دیکھائیں کے 
\(n=\sqrt{\frac{r_{0}}{a_{g}}}\)
ہوگا ۔ اس سے زمین کے quantum عدد n کی اندازاً قیمت تلاش کریں ۔\\
جزو د)فرض کریں کے زمین کی اگلی نچلی ستہ 
\((n-1)\)
میں منتقل ہوتی ہے تو اسکی کتنی توانائی کا اخراج ہوگا جواب joules  میں دیں ۔خارج کردہ photon یا graviton کا طولِ موج کیا ہوگا اپنے جواب کو نوری سالوں میں پیش کریں کیا یہ حیرت انگیز نتیجہ محض ایک اتفاق ہے ۔\\
حصہ 
4.3\\
زاویائی معیار حرکت\\
ہم دیکھ چکے ہیں کہ ہائیڈروجن جو ہر کے ساکن حالات کو تین کو انٹائی عداد
\(n\)
،
\(l\)
اور
\(m\)
کے لحاظ سے نام دیا جاتا ہے . صدر تو انتم عدد
\((n)\)
حال
کی توانائی تعین کرتا ہے . مساوات 4.70.
ہم دیکھیں گے کہ
\(l\)
اور
\(m\)
مداری زاویائی معیار حرکت سے تعلق رکھتے ہیں . کلاسکی نظریہ میں وسطی قوتیں توانائی اور معیار حرکت بنیادی مقداریں ہیں . اور یہ حیرت تو بات نہیں ہے کہ زاویائی معیار کت اس سے بھی زیادہ اہمیت توانٹم نظریہ میں رکھتا ہے . کلاسکی طور پر ایک زرے کی زاویائی معیار حرکت درج ذیل کلیہ دیتی ہے .\\
\[\text{\textbf{L}}=\text{\textbf{r}}\times\text{\textbf{p}}\]
جس کے تحت درج ذیل ہوگا\\
\[L_{x}=yp_{z}-zp_{y} \hspace{1cm} L_{y}=zp_{x}-xp_{z} \hspace{1cm},L_{z}=xp_{z}-yp_{x}\]
ان کے مطالقہ کوانٹم عاملین معیاری نسخہ
\(p_{x}\rightarrow i\hbar\partial/\partial{x},p_{y}\rightarrow i\hbar\partial/\partial{y},p_{z}\rightarrow i\hbar\partial/\partial{z}\)
سے حاصل ہوگا۔ باب 2 میں ہم نے ہارمونی مرتعش کے جز تی توانایوں کو خالص الجرائی ترقیب سے حاصل کیا اس حصہ میں اسی الجبرائی ترقیب تو استعمال کرتے ہوئے زاویائی معیار حرکت کے عاملین کے امتیازی اقدار حاصل کیے جائیں گے۔ یہ ترقیب عاملین کے تبادلی تعلقات پر مبنی ہے . اس کے بعد ہم امتیازی تفالات حاصل کریں گے۔\\
حصہ 
4.3.1\\
امتیازی اقدار\\
عاملین
\(L_{x}\)
اور
\(L{y}\)
آپس میں ناقابل تبادل ہیں. درحقیقت درج ذیل ہوگا۔\\
\begin{align*}
[L_{x},L_{y}]&=[yp_{z}-zp_{y},zp_{x}-xp_{z}]\\
&=[yp_{z},zp_{x}]-[yp_{z},xp_{z}]-[zp_{y},zp_{x}]+[zp_{y},xp_{z}]\\
\end{align*}

\end{document}
